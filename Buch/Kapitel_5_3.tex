%%%%%%%%%%%%%%%%%%%%% chapter.tex %%%%%%%%%%%%%%%%%%%%%%%%%%%%%%%%%
%
% sample chapter
%
% Use this file as a template for your own input.
%
%%%%%%%%%%%%%%%%%%%%%%%% Springer-Verlag %%%%%%%%%%%%%%%%%%%%%%%%%%

\chapter{"'Echte"' Keynes-Anhänger und der Post-Keynesianismus}
\label{Post-Keynes}


\section{Kalecki: Post-Keynesianer vor Keynes}



\section{Axel Leijonhufvud: Vater der Ungleichgewichtstheorie}

HIER WEITER: \parencite{Patinkin1990}
Was von Keynes fehlt im IS-LM-Modell? (Verweis auf Post-Keynesianer) Unsicherheiten bzgl. zukünftiger Entwicklungen, die sogenannten Animal Spirits als Ad-hoc Annahme im Fall der Investitionsfalle. Außerdem wird akzeptiert, dass die primäre Quelle der Unterbeschäftigung rigide Löhne sind. Dies ist grundsätzlich nichts typisch Keynesianisches. Lohn ist ja nichts anderes als der Preis für Arbeit. Wenn dieser zu hoch ist, kommt es auch in der (Neo-)Klassik zu keiner vollständigen Markträumung.


Leijonhufvud (1968): Einflussreiche Arbeit. Auch Robert Barro startete in diesem Bereich!
Verlor mit der Theorie der Rationalen Erwartungen rasch an Bedeutung
Ansiedelung als Post-Keynesianer umstritten.



\section{Harrod \& Domar: Das erste Wachstumsmodell}
Man kann auf jeden Fall darüber streiten, ob das Harrod-Domar-Modell hier richtig angesiedelt ist oder nicht doch besser im "`Mainstream-Teil"' dieses Buches als keynesianisches Wachstumsmodell. Faktum ist, dass es nach dem Zweiten Weltkrieg das erste anerkannte, modell-theoretische Wachstumsmodell war. Begründet wurde es unabhängig voneinander durch \textcite{Harrod1939} und \textcite{Domar1946, Domar1947}. Entsprechend dem damaligen Zeitgeist ist es ein typisch keynesianisches Modell, also eigentlich dem damaligen Mainstream entsprechend. Faktum ist aber auch, dass das Harrod-Domar-Modell mit dem Aufkommen der exogenen Wachstumsmodelle (vgl. Kapitel \ref{sec: Solow-Modell}) Mitte der 1950er Jahre, rasch an Bedeutung verlor. In der Neoklassischen Synthese - also der Verschmelzung von Teilbereichen von Keynes' General Theorie mit der Neoklassik - war kein Platz für das Harrod-Domar-Modell und die neuen Mainstream-Ökonomen erklärten es rasch als widerlegt, beziehungsweise als sehr unwahrscheinlichen Spezialfall eines Wachstumsverlaufs \parencite{Solow1987}. 

Als typisch keynesianisches Modell ist der Ausgangspunkt beim Harrod-Domar-Modell die Darstellung des Bruttoinlandsprodukts (BIP) als Summe aus Konsum und Sparen. Wobei Sparen und Investitionen als identisch angenommen werden. Die Annahmen - die letztendlich auch als zu restriktiv identifiziert wurden - lauten, dass der Kapitelstock in einer Ökonomie einen konstant bleibenden Anteil des gesamten Outputs, also des BIPs darstellt. Da auch das Harrod-Domar-Modell nur zwei Produktionsfaktoren kennt - Arbeit und Kapital - impliziert diese Annahme, dass auch der Faktor Arbeit einen konstant bleibenden Anteil des BIPs ausmacht, also dass die Lohnquote konstant ist. Zur Entstehungszeit des Modells war dies durchaus eine anerkannte Annahme. Bowley's Law (vgl. Kapitel \ref{Ungleichheit}) behauptete genau dies und war lange Zeit recht unumstritten. Aus diesen Annahmen lässt sich recht einfach ableiten, dass BIP-Wachstum von der Sparquote und dem Kapitalstock-Anteil am BIP abhängt. Je höher die Sparquote und je niedriger der Kapitalstock-Anteil desto höher ist das Wirtschaftswachstum \parencite[S. 601]{Snowdon2005}. Das ist natürlich vor allem für Entwicklungsländer - schließlich haben diese meist eine geringe Kapitalquote - ein interessantes Ergebnis, weil es impliziert, dass durch Erhöhung der Sparquote (Sparen = Investitionen) Wirtschaftswachstum ausgelöst wird. Tatsächlich wurde dieses Modell vor allem in den 1960er und 1970er Jahren primär im Bereich der Entwicklungshilfe angewendet. Laut \textcite[S. 601]{Snowdon2005} zum Beispiel in der Weltbank. In der akademischen Mainstream-Literatur wurden bereits in den 1950er-Jahren kritische Stimmen unüberhörbar: ein hohe Investitionsquote alleine garantiert noch kein Wachstum, stattdessen entscheidet die Produktivität des eingesetzten Kapitals darüber ob es auch Wachstum schafft. Die Annahme, dass die Produktionsfaktoren Arbeit und Kapital sich nicht gegenseitig ersetzen können und deren jeweiliger Anteil langfristig stabil bleibt, war nicht aufrechtzuerhalten. Langfristiges Wirtschaftswachstum entlang eines Gleichgewichtspfades wäre demnach dann gegeben, wenn das Angebot an Arbeitskräften und Kapital stets im gleichen Ausmaß wachsen würde. Dies ist der oben bereits von \textcite{Solow1987} angesprochene unwahrscheinliche Spezialfall eines Wachstumsverlaufs, der in der Literatur oft als "`Ritt auf der Rasierklinge"' \parencite[S. 65]{Solow1956} bezeichnet wurde \parencite[S. 602]{Snowdon2005}.

Tatsächlich spielt das "`Harrod-Domar-Modell"' heute in den Wirtschaftswissenschaften keine Rolle mehr. Wirtschaftshistoriker wie \textcite{Berg2013} und \textcite{Boianovsky2018} argumentieren allerdings, dass das Modell von dessen Kritikern falsch interpretiert wurde. So muss es im Kontext mit der keynesianischen Theorie der Nachfrageseite interpretiert werden, die bezüglich langfristigen Wachstums ganz andere Annahmen trifft, als die Neoklassik \parencite[S. 479]{Boianovsky2018}. Auch die Konstanz von Kapital- und Lohnquote als unwahrscheinlichen Fall interpretiert die Neoklassik anders als die Keynesianer. Die Neoklassiker vertrauen darauf, dass sich laut Produktivitätstheorie die Kapital- und Lohnquoten automatisch zum optimale Gleichgewicht bewegen. Dann wäre es tatsächlich ein Zufall, wenn diese stets konstant blieben, da Technologiefortschritte und Bevölkerungsveränderungen stets im Einklang wachsen müssten. Die Keynesianer hingegen lehnten eine aktive Wirtschaftspolitik eben gerade nicht ab. Dahingehend kann man interpretieren, dass eine ebensolche dafür sorgen würde, dass die Kapitalquote eben nicht übermäßig fällt oder steigt.



\section{Kaldor: Der Begründer des Post-Keynesianismus}
Durch die Ablehnung des IS-LM-Modells.
Einer der ersten, der die Geldmengensteuerung durch die Zentralbank als nicht realistisch darstellte. Diese ist aber eine wichtige Voraussetzung für die LM-Kurve im IS-LM-Modell. Diese Annahme gilt heute als State of the Art der Wissenschaften und wird von Zentralbankern immer wieder betont.

HIER nicht richtig eingeordnet
Bedeutung des Mark-Ups. Ist Teil der Mainstream.Ökonomie heute (vgl.: Mankiw-Interview im Buch Snowdon!)


\section{Myrdal \& Sraffa}

\section{Linkskeynesianismus: Lernen \& Robinson}

Vergleiche Kapitel \ref{Neoklassik}:

Robinson und Lerner erkannten als erste die Notwendigkeit die Nutzentheorie von Pareto nochmal aufzugreifen
"`War of the two Cambridges"' \parencite{Tobin1985}: Kapitalkontroverse oder die Kontroverse der zwei Cambridges

Robinson, Economics of Imperfect Competition, p. 256; Lerner, "Elasticity of Substitution" (Review of Economic Studies, Oct. 1933, pp. 68-70

\section{Minsky: Die Absurdität des Finanzmarktgleichgewichts}





