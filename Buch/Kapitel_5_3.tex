%%%%%%%%%%%%%%%%%%%%% chapter.tex %%%%%%%%%%%%%%%%%%%%%%%%%%%%%%%%%
%
% sample chapter
%
% Use this file as a template for your own input.
%
%%%%%%%%%%%%%%%%%%%%%%%% Springer-Verlag %%%%%%%%%%%%%%%%%%%%%%%%%%

\chapter{Post-Keynesianismus}
\label{Post-Keynes}


\section{Kalecki: Post-Keynesianer vor Keynes}

\section{Kaldor: Der Begründer des Post-Keynesianismus}
Durch die Ablehnung des IS-LM-Modells.

\section{Myrdal \& Sraffa}

\section{Linkskeynesianismus: Lernen \& Robinson}

Vergleiche Kapitel \ref{Neoklassik}:

Robinson und Lerner erkannten als erste die Notwendigkeit die Nutzentheorie von Pareto nochmal aufzugreifen
"`War of the two Cambridges"' \parencite{Tobin1985}

Robinson, Economics of Imperfect Competition, p. 256; Lerner, "Elasticity of Substitution" (Review of Economic Studies, Oct. 1933, pp. 68-70

\section{Minsky: Die Absurdität des Finanzmarktgleichgewichts}

