%%%%%%%%%%%%%%%%%%%%% chapter.tex %%%%%%%%%%%%%%%%%%%%%%%%%%%%%%%%%
%
% sample chapter
%
% Use this file as a template for your own input.
%
%%%%%%%%%%%%%%%%%%%%%%%% Springer-Verlag %%%%%%%%%%%%%%%%%%%%%%%%%%

\chapter{Post-Keynesianismus}
\label{Post-Keynes}


\section{Kalecki: Post-Keynesianer vor Keynes}

\section{Harrod \& Domar: Das erste Wachstumsmodell}
Man kann auf jeden Fall darüber streiten, ob das Harrod-Domar-Modell hier richtig angesiedelt ist oder nicht doch besser im "`Mainstream-Teil"' dieses Buches als keynesianisches Wachstumsmodell anzusiedeln wäre. Faktum ist, dass es nach dem Zweiten Weltkrieg das erste anerkannte, modell-theoretische Wachstumsmodell war. Entsprechend dem damaligen Zeitgeist ein typisch keynesianisches Modell, also eigentlich dem damaligen Mainstream entsprechend. Faktum ist aber auch, dass das Harrod-Domar-Modell mit dem Aufkommen der exogenen Wachstumsmodelle (vgl. Kapitel \ref{sec: Solow-Modell}) Mitte der 1950er Jahre, rasch an Bedeutung verlor. In der Neoklassischen Synthese - also der Verschmelzung von Teilbereichen von Keynes' General Theorie mit der Neoklassik - war kein Platz für das Harrod-Domar-Modell und die neuen Mainstream-Ökonomen erklärten es rasch als widerlegt, beziehungsweise als sehr unwahrscheinlichen Spezialfall \parencite{Solow1987}.

HIER WEITER: Snowdon/Vane S. 598

Harrod-Domar Modell (später Kontroverse Solow - Robinson!)



\section{Kaldor: Der Begründer des Post-Keynesianismus}
Durch die Ablehnung des IS-LM-Modells.

\section{Myrdal \& Sraffa}

\section{Linkskeynesianismus: Lernen \& Robinson}

Vergleiche Kapitel \ref{Neoklassik}:

Robinson und Lerner erkannten als erste die Notwendigkeit die Nutzentheorie von Pareto nochmal aufzugreifen
"`War of the two Cambridges"' \parencite{Tobin1985}

Kapitalkontroverse oder die Kontroverse der zwei Cambridges

Robinson, Economics of Imperfect Competition, p. 256; Lerner, "Elasticity of Substitution" (Review of Economic Studies, Oct. 1933, pp. 68-70

\section{Minsky: Die Absurdität des Finanzmarktgleichgewichts}

