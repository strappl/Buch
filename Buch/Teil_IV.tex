%%%%%%%%%%%%%%%%%%%%%%%% part.tex %%%%%%%%%%%%%%%%%%%%%%%%%%%%%%%%%%
%
% sample part title
%
% Use this file as a template for your own input.
%
%%%%%%%%%%%%%%%%%%%%%%%% Springer-Verlag %%%%%%%%%%%%%%%%%%%%%%%%%%


\part{Seit 1975\\Neue Klassische Makroökonomie -- Neu-Keynesianismus -- Neue Neoklassische Synthese}

In den 1970er Jahren steckte die ökonomische Forschung in der Sackgasse. Der Keynesianismus (in der Form der neoklassischen Synthese) galt, zumindest im angelsächsischen Raum, spätestens seit der Ölkrise von 1973 als überholt. Die Zeit spielte schon lange gegen die österreichische Schule: Ihre liberalen Theorien waren zu radikal um sie wirtschaftspolitisch umzusetzen. Und ihre konkreten Befürchtungen, etwa in Hinsicht auf internationale Geldsysteme erwiesen sich als zu pessimistisch. Zum Beispiel waren die Währungssysteme auch ohne Goldstandard relativ stabil. Zudem galt ihre Methodik als widerlegt: Mathematik und empirische Forschung setzten sich in der Wissenschaft immer mehr durch. Aber auch Milton Friedman's Monetarismus galt in seiner Grundidee als zu kurz gegriffen: Der Monetarismus wurde durchaus wirtschaftspolitisch praktiziert. Aber die Konzentration alleine auf Geldpolitik hatte nicht die erwartete stabilisierende Wirkung auf die Preise.

Wer aber dachte das Pendel würde nach dem wirtschaftsliberalen Monetarismus wieder nach links, in Richtung Keynesianismus ausschlagen, täuschte sich. Die Lucas-Kritik kam wie ein Paukenschlag über die Ökonomie und brachte nichts anderes als die Rückbesinnung auf die "`(Neo-)Klassik"'.

Mit dem Aufkommen der sogenannten "`Neuen Klassischen Makroökonomie"' ist vor allem der Name Robert Lucas und seine Lucas-Kritik \parencite{Lucas1976} verbunden. Tatsächlich kam es Mitte der 1970er Jahre zu einer Revolution und gleichzeitig zu einer Spaltung der Wirtschaftswissenschaften.

Beginnen wir den neuen Teil mit einem kurzen Abriss des letzten Teils: Die wirtschaftswissenschaftliche Community war während und nach dem Zweiten Weltkrieg praktisch zur Gänze von Europa in die USA umgezogen. Dort sammelten sich die unterschiedlichen Schulen rasch an verschiedenen Orten. Angeblich war es Robert Hall im Jahre 1976, der die Ökonomen - durchaus humorvoll - in "`Salzwasser-"' und "`Süßwasserökonomen"' einteilte: "`Needless to say, individual contributors vary across a spectrum of salinity. [...] A few examples: Sargent corresponds to distilled water, Lucas to Lake Michigan, Feldstein to Charles River above the dam, Modigliani to the Charles below the dam, and Okun to the Salton See"' \parencite[S. 1]{Hall1976}. Tatsächlich war der Keynesianismus nach 1945 als Mainstream-Ökonomie sowohl an der Westküste als auch an der Ostküste die dominierende Schule. In Chicago, also an einem der großen Süßwasser-Seen, formte sich hingegen um Frank Knight und später um Milton Friedman, sowie schließlich eben um Robert Lucas eine Clique von Ökonomen, die den wirtschaftlichen Liberalismus wieder zurück in den Mainstream brachte.

Direkt nach dem Zweiten Weltkrieg wandten sich die meisten Wirtschaftswissenschaftler dem Keynesianismus zu. Die Österreichische Schule lehnte dessen Ideen hingegen von Anfang an strikt ab und war nach 1945 auch noch ein wissenschaftlich bedeutender Gegenpol.\footnote{Mit dem Tod von Schumpeter und Mises und dem Aufkommen der Chicago School in der Makroökonomie, vor allem mit Milton Friedman's Monetarismus, verlor die Österreichische Schule in der akademischen Welt allerdings bald an Bedeutung.} Der Keynesianismus (in der Form der neoklassischen Synthese) war hingegen unumstritten bis Mitte der 1960er Jahre die dominierende Schule. Spätestens mit dem Schock der Ölkrise 1973 verlor er aber den Status als "`State-of-the-Art"' in der Ökonomie. 

Schon Mitte der 1960er Jahre begann der wissenschaftliche Aufstieg des Monetarismus. Milton Friedman's und Anna Jacobson Schwartz' Werk \textit{A Monetary History of the United States, 1867–1960} im Jahr 1963 kann als wissenschaftlicher Grundstein des Monetarismus gesehen werden, der Höhepunkt erfolgte gegen Ende der 1960er Jahre. 

Mitte der 1970er Jahre schließlich - hier ist die Veröffentlichung der Lucas-Kritik im Jahr 1976 ein typischer zeitlicher Startpunkt - wurden die beiden vorher genannten Schulen abgelöst, beziehungsweise eigentlich im Endeffekt erweitert, von der Neuen Klassischen Makroökonomie. Interessant und wichtig ist der Unterschied bezüglich des vorherrschenden ökonomischen Denkens zwischen der wirtschafts-wissenschaftlichen Community und den Entscheidungsträgern in der Wirtschaftspolitik. Gerade in den 1970er und 1980er Jahren gab es hier einen beträchtlichen Time-Lag. In Bezug auf realpolitische Umsetzung kam die Zeit der wirtschaftsliberalen Schulen, Monetarismus und Neue Klassische Makroökonomie, nämlich erst später. Bis in die späten 1960er Jahre war der Keynesianismus in den USA sowohl unter demokratischen als auch republikanischen Präsidenten als wissenschaftliche Grundlage ihres wirtschaftspolitischen Handelns anerkannt \parencite[S. 12]{Woodford1999}. Das änderte sich erst in den 1970er Jahren: Die Wirtschaft der USA war damals von einer Berg-und-Talfahrt geprägt. Häufig spricht man im Hinblick auf diese Phase von "`Stagflation"', also niedriger BIP-Wachstumsraten, bei gleichzeitig permanent hoher Inflation. Genau diese wurde von den Keynesianern (zu) lange Zeit als relativ unproblematisch hingenommen, schließlich ging mit hoher Inflation zumeist eine niedrige Arbeitslosigkeit einher. Tatsächlich wurde die allgemeine Teuerung allerdings zunehmend als Problem wahrgenommen. Ein Problem für das die liberalen Ökonomen die besseren Erklärungsmodelle und Bekämpfungsmethoden bereitstellten. 

Interessanterweise läutete ein demokratischer Präsident, nämlich Jimmy Carter, die Hinwendung zu liberaler Wirtschaftspolitik ein. Er bestellte Alfred Edward Kahn zu einem wichtigen wirtschaftspolitischen Berater, der in den Folgejahren vor allem die Deregulierung und Privatisierung weiter Teile der staatlichen Industrie (Luftfahrt) vorantrieb. 1979 wurde Paul Volcker zum Vorsitzenden der US-Notenbank Federal Reserve (Fed) bestellt. In weiterer Folge wurde Inflationsbekämpfung als primäres Ziel ausgegeben. Schon mit diesen beiden Schritten hatten sich die USA wirtschaftspolitisch deutlich dem Monetarismus zugewendet. Eindeutiger wurde dies in den 1980er Jahren. Die \textit{Reaganomics} von Präsident Ronald Reagan in den USA, sowie der \textit{Thatcherism} in Großbritannien der "`Eisernen-Lady"', Premierministerin Margaret Thatcher,  bezogen sich offen auf den liberalen Monetarismus, aber auch auf die Österreichischen Schule. Friedrich Hayek und Milton Friedman fungierten dabei sogar direkt als wirtschaftspolitische Berater. Der Zeit-Lag ist deutlich sichtbar: Der politische Einfluss von Hayek und Friedman erfolgte Jahre nachdem ihre ökonomischen Schulen in der Wissenschaft ihren Zenit erreicht hatten.

Schon ab den frühen 1970er Jahren waren die wissenschaftlichen Erkenntnisse der "`Neuen Klassischen Makroökonomie"' entwickelt worden. Zwar setzten sich deren Erkenntnisse nach und nach durch, der enorme und vor allem \textit{direkte} wirtschaftspolitische Einfluss, den Hayek und Friedman erlangten, blieb Lucas, Sargent und Co aber verwehrt.

