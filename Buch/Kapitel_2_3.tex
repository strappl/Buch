%%%%%%%%%%%%%%%%%%%%% chapter.tex %%%%%%%%%%%%%%%%%%%%%%%%%%%%%%%%%
%
% sample chapter
%
% Use this file as a template for your own input.
%
%%%%%%%%%%%%%%%%%%%%%%%% Springer-Verlag %%%%%%%%%%%%%%%%%%%%%%%%%%

\chapter{Das Ende der Ökonomie?}
\label{Neoklassik}

Die - wie wir mittlerweile wissen nur sogenannte - Marginalistische Revolution brauchte einige Zeit um sich zu etablieren. Gegen Ende des 19. Jahrhunderts setzten sich aber schließlich einige Gegebenheiten durch, die bis heute von Einfluss sind. Erstens, verlagerte sich die ökonomische Forschung fast ausschließlich in den universitären Bereich. Ökonomen wie Thünen oder Gossen, die gänzlich außerhalb des wissenschaftlichen Apparates arbeiteten, aber auch solche wie Jevons und Walras, die erst nach Tätigkeiten in der Privatwirtschaft im universitären Bereich Fuß fassten, sind seither eher die Ausnahme. Damit verlagerte sich, zweitens, die Publikationstätigkeit von gesamtheitlichen Wälzern auf hochspezialisierte Journalbeiträge. Ein Prozess, der zwar recht langsam voranschritt, die Gründungsjahre der heute noch führenden Journale deuten dies aber an: American Economic Review in 1911, das Economic Journal seit 1891, das Quarterly Journal of Economics 1886 und das Journal of Political Economy 1892 \parencite[S. 340]{Rosner2012}.
Drittens etablierte sich in der Ökonomie eine weitgehend einheitliche Sprache mit einheitlichen, fachspezifischen Ausdrücken. Einen wesentlichen Beitrag dazu leistete der britische Ökonom und Nachfolger Jevons in Cambridge Alfred Marshall.

\section{Marshall: Der Vollender}

\textcite{Marshall1890}: \textit{Principles of Economics - an introductory volume} gilt bis heute als eines der prägendsten Lehrbücher aller Zeiten. Es fasst nicht nur den State of the Art der Ökonomie zusammen, sondern erweiterte denselben auch. Der durchschlagende Erfolg dieses Buches hängt sicherlich auch damit zusammen, dass sich die mikroökonomische Theorie seit damals kaum mehr verändert hat. Natürlich wurde sie entscheidend und an vielen Stellen erweitert. Aber die damals schon bestehenden Theorien zur Mikroökonomie gelten bis heute unverändert und sind tatsächlich in einführenden Lehrbüchern praktisch identisch abgedruckt.

Mit Marshall änderte sich der Blick auf die Ökonomie grundlegend. Er gilt daher nicht umsonst als der "`Vollender der Neoklassik"'.  Neben inhaltlicher Punkte, revolutionierte Marshall auch allgemein das Bild der Ökonomie als Wissenschaft:

Erstens, zunächst besticht sein Hauptwerk durch die Kombination von Sprachlicher Verständlichkeit und mathematische Präzision. Vergleicht man sein Werk mit jenen von, zum Beispiel, Menger, Walras oder Böhm-Bawerk, so merkt man sofort eine deutlich bessere Verständlichkeit. Dies ist zwar nicht unbedingt ausschließlich der Arbeit Marshalls eigen, sondern eher dem Zeitgeist zuzuschreiben und so auch zum Beispiel bei Irving Fisher zu finden, aber es besticht dennoch in \textcite{Marshall1890}. 

Zweitens, Marshall grenzte als erster die Volkswirtschaftslehre ("`Economics"') als eher formale Wissenschaft von den übrigen Sozialwissenschaften ab. Dazu "`kürzte"' er die politischen Entscheidungs- und Machtstrukturen aus seinem wirtschaftswissenschaftlichem Werk. Die bis dahin übliche Bezeichnung der "`Politischen Ökonomie"', wurde zur "`Volkswirtschaftslehre"'. Dies war für sein Werk notwendig. Die von ihm so elegant durchgezogene formale Herangehensweise ist nur dann möglich, wenn man Nebenbedingungen definiert und diese als gegeben annimmt. Dies ist eine der Stärken der Neoklassik. Es brachte ihr aber auch harsche Kritik ein, weil sich die Volkswirtschaftslehre damit von der erlebten Realität ein gutes Stück entfernt. Diese Kritik ist im Falle Marshalls ungerechtfertigt. Er war sich durchaus bewusst, dass seine "`Principles"' eine zu starke Vereinfachung der wirtschaftlichen Realität darstellten. Dementsprechend plante er, ja rang förmlich, um die Entstehung einer Fortsetzung seines Werkes \parencite[S. 146]{Rieter1989}. Tatsächlich ließ er erst mit der sechsten Auflage der "`Principles of Economics"' den Zusatz "`Volume I"' streichen. Mit 80 Jahren - 1920: "`Industry and Trade"', sowie 1923: "`Money Credit and Commerce"' - publizierte er schließlich seine Vorarbeiten zu weiteren Bänden noch, diese können aber nur mehr als fragmentierte Beiträge zu einzelnen Themenkomplexen gesehen werden.

Drittens, Er war als einer der ersten Ökonomen extrem gut informiert über den "`State of the Art"' der Ökonomie. Heute ist es unumgänglich die wissenschaftlichen Arbeiten des Forschungszweiges, in welchem man selbst publizieren möchte, umfänglich zu kennen. Im 19. Jahrhundert allerdings sorgten Probleme der Sprache, Verfügbarkeit und bloßen Kenntnis dafür, dass oft Theorien entwickelt wurden, ohne dass die Urheber den Stand der Wissenschaft kannten. Wir erinnern uns zum Beispiel an Gossen. Aber auch Walras verfasste zunächst sein Hauptwerk und trat erst gegen Ende seiner Karriere in intensiven Austausch mit zeitgenössischen Ökonomen. Ganz anders war dies bei Marshall: Wie \textcite{Groenewegen1995} beschreibt, beschäftigte sich dieser bereits in den späten 1860er Jahren mit den deutschen Ökonomen wie Thünen, Roscher und Rau und kannte die Arbeiten der Franzosen Cournot und Dupuit. Selbstverständlich waren ihm die englischen Klassiker bekannt, aber auch die Historische Schule der Deutschen war ihm nicht fremd. Aufbauend auf all dem Wissen publizierte er sein Hauptwerk. Dies verhältnismäßig spät mit 48 Jahren 1890. Die Erkenntnisse waren zu dieser Zeit bereits allesamt weitgehend bekannt und zirkulierten als Mitschriften aus seinen Vorlesungen. Er zitierte in den "`Principles"' auch nur wenig seinen Zeitgenossen Jevons, sondern eben vor allem die "`Vorläufer"'. Es gilt auch bis heute als umstritten, ob Marshall tatsächlich primär der \textit{Entwickler} bedeutender ökonomischer Theorien ist, oder doch eher der \textit{Vereiniger} ökonomischer Elemente, die bereits jemand anderer entwickelt hatte \parencite[S. 207ff]{Ekelund2002}. 

HIER WEITER: 
Viertens, Ceteris Paribus

Erst jetzt kommen wir zu seinen inhaltlichen Beiträgen:
 
Erste Darstellung Angebot und NAchfrage (Kreuz). In wirklichkeit: bereits in Rau 1826! \textcite[S. 159]{Blaug2001}
Zusammenführung Klassik und Neoklassik. Nachfragefunktion, abgeleitet aus abnehmenden Grenznutzen und 
Elastizität
Produzenten- und Konsumentenrente
Element der Zeit als wichtiges Element(?)


Beitrag Marshalls: \textcite{Ekelund2002}


Tatsächlich war die ursprüngliche "`neoklassische Theorie"' mit Marshall abgeschlossen, wenn man dies in dem Sinne versteht, dass seine Erkenntnisse wie in seinem Werk "`Principles of Economics"' auch in modernen Mikroökonomie Büchern praktisch identisch dargestellt werden. Diese "`Vollendung der Neoklassik"' kann deshalb vielleicht als Ausgangspunkt der \textit{modernen} Ökonomie gelten. 


Wir wissen aber natürlich, dass sich die Wirtschaftswissenschaften seither vielfältig weiterentwickelt haben.  Sein Nachfolger in Cambridge machte sich als einer der ersten Gedanken über "`Marktversagen"', also 



Sein direkter Nachfolger in Cambridge: Pigou. Seine Arbeiten könnten auch in diesem Kapitel dargestellt werden, da er direkt an Marshall's Wert anschloss. Allerdings war Pigou's späteres Werk geprägt von den Angriffen auf seine Theorie durch Keynes. 







\section{Edgeworth und Pareto}
The rediscovery of Pareto optimality in the 1930s after
26 years of neglect \textcite[S. 148]{Blaug2001}





\section{Fisher and Clark: Economics goes USA}
Der erste amerikanische Top-Ökonom?
Zinstheorie (Intertemporale Konsumentscheidung)

Fisher-Separation
beides schon im Kapitel \ref{Finance}
Debt-Deflation-Theory
Quantitätsgleichung als Vorläufer zum Monetarismus
Erster Benutzer von Indexnummern. (Vereinzelt wird dies aber auch schon Stanley Jevons zugeschrieben \parencite[S. 232]{Jevons1934})

Einer der ersten "`modernen"' Ökonomen. Seine Ideen sind teilweise bis heute unverändert in Anwendung

Seine Darstellung war wesentlich besser - im Sinne eine eindeutigeren und formaleren Darstellung - als jene der Neoklassiker, wie zum Beispiel Böhm-Bawerk.


\textcite{Tobin2005}