%%%%%%%%%%%%%%%%%%%%% chapter.tex %%%%%%%%%%%%%%%%%%%%%%%%%%%%%%%%%
%
% sample chapter
%
% Use this file as a template for your own input.
%
%%%%%%%%%%%%%%%%%%%%%%%% Springer-Verlag %%%%%%%%%%%%%%%%%%%%%%%%%%

\chapter{Das Ende der Ökonomie?}
\label{Neoklassik}

Die - wie wir mittlerweile wissen nur sogenannte - Marginalistische Revolution brauchte einige Zeit um sich zu etablieren. Gegen Ende des 19. Jahrhunderts setzten sich aber schließlich einige Gegebenheiten durch, die bis heute von Einfluss sind. Erstens, verlagerte sich die ökonomische Forschung fast ausschließlich in den universitären Bereich. Ökonomen wie Thünen oder Gossen, die gänzlich außerhalb des wissenschaftlichen Apparates arbeiteten, aber auch solche wie Jevons und Walras, die erst nach Tätigkeiten in der Privatwirtschaft im universitären Bereich Fuß fassten, sind seither eher die Ausnahme. Damit verlagerte sich, zweitens, die Publikationstätigkeit von gesamtheitlichen Wälzern auf hochspezialisierte Journalbeiträge. Ein Prozess, der zwar recht langsam voranschritt, die Gründungsjahre der heute noch führenden Journale deuten dies aber an: American Economic Review in 1911, das Economic Journal seit 1891, das Quarterly Journal of Economics 1886 und das Journal of Political Economy 1892 \parencite[S. 340]{Rosner2012}.
Drittens etablierte sich in der Ökonomie eine weitgehend einheitliche Sprache mit einheitlichen, fachspezifischen Ausdrücken. Einen wesentlichen Beitrag dazu leistete der britische Ökonom und Nachfolger Jevons in Cambridge Alfred Marshall.

\section{Marshall: Der Vollender}

\textcite{Marshall1890}: \textit{Principles of Economics - an introductory volume} gilt bis heute als eines der prägendsten Lehrbücher aller Zeiten. Es fasst nicht nur den State of the Art der Ökonomie zusammen, sondern erweiterte denselben auch. Der durchschlagende Erfolg dieses Buches hängt sicherlich auch damit zusammen, dass sich die mikroökonomische Theorie seit damals nicht mehr verändert hat. Natürlich wurde sie entscheidend und an vielen Stellen erweitert. Aber die damals schon bestehenden Theorien zur Mikroökonomie gelten bis heute unverändert und sind tatsächlich in einführenden Lehrbüchern praktisch unverändert abgedruckt.

Erste Darstellung Angebot und NAchfrage (Kreuz). In wirklichkeit: bereits in Rau 1826! \textcite[S. 159]{Blaug2001}


Beitrag Marshalls: \textcite{Ekelund2002}


Tatsächlich war die ursprüngliche "`neoklassische Theorie"' mit Marshall abgeschlossen, wenn man dies in dem Sinne versteht, dass seine Erkenntnisse wie in seinem Werk "`Principles of Economics"' auch in modernen Mikroökonomie Büchern praktisch identisch dargestellt werden. Diese "`Vollendung der Neoklassik"' kann deshalb vielleicht als Ausgangspunkt der \textit{modernen} Ökonomie gelten. 


Wir wissen aber natürlich, dass sich die Wirtschaftswissenschaften seither vielfältig weiterentwickelt haben.  Sein Nachfolger in Cambridge machte sich als einer der ersten Gedanken über "`Marktversagen"', also 



In den USA: John Bates Clark?


Sein direkter Nachfolger in Cambridge: Pigou. Seine Arbeiten könnten auch in diesem Kapitel dargestellt werden, da er direkt an Marshall's Wert anschloss. Allerdings war Pigou's späteres Werk geprägt von den Angriffen auf seine Theorie durch Keynes. 







\section{Edgeworth und Pareto}
The rediscovery of Pareto optimality in the 1930s after
26 years of neglect \textcite[S. 148]{Blaug2001}





\section{Fisher and Clark: Economics goes USA}
Der erste amerikanische Top-Ökonom?
Zinstheorie (Intertemporale Konsumentscheidung)

Fisher-Separation
beides schon im Kapitel \ref{Finance}
Debt-Deflation-Theory
Quantitätsgleichung als Vorläufer zum Monetarismus
Erster Benutzer von Indexnummern. (Vereinzelt wird dies aber auch schon Stanley Jevons zugeschrieben \parencite[S. 232]{Jevons1934})

Einer der ersten "`modernen"' Ökonomen. Seine Ideen sind teilweise bis heute unverändert in Anwendung

Seine Darstellung war wesentlich besser - im Sinne eine eindeutigeren und formaleren Darstellung - als jene der Neoklassiker, wie zum Beispiel Böhm-Bawerk.


\textcite{Tobin2005}