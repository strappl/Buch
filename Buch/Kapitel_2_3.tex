%%%%%%%%%%%%%%%%%%%%% chapter.tex %%%%%%%%%%%%%%%%%%%%%%%%%%%%%%%%%
%
% sample chapter
%
% Use this file as a template for your own input.
%
%%%%%%%%%%%%%%%%%%%%%%%% Springer-Verlag %%%%%%%%%%%%%%%%%%%%%%%%%%

\chapter{Das Ende der Ökonomie?}
\label{Neoklassik}

Die - wie wir mittlerweile wissen nur sogenannte - Marginalistische Revolution brauchte einige Zeit um sich zu etablieren. Gegen Ende des 19. Jahrhunderts setzten sich aber schließlich einige Gegebenheiten durch, die bis heute von Einfluss sind. Erstens, verlagerte sich die ökonomische Forschung fast ausschließlich in den universitären Bereich. Ökonomen wie Thünen oder Gossen, die gänzlich außerhalb des wissenschaftlichen Apparates arbeiteten, aber auch solche wie Jevons und Walras, die erst nach Tätigkeiten in der Privatwirtschaft im universitären Bereich Fuß fassten, sind seither eher die Ausnahme. Damit verlagerte sich, zweitens, die Publikationstätigkeit von gesamtheitlichen Wälzern auf hochspezialisierte Journalbeiträge. Ein Prozess, der zwar recht langsam voranschritt, die Gründungsjahre der heute noch führenden Journale deuten dies aber an: American Economic Review in 1911, das Economic Journal seit 1891, das Quarterly Journal of Economics 1886 und das Journal of Political Economy 1892 \parencite[S. 340]{Rosner2012}.
Drittens etablierte sich in der Ökonomie eine weitgehend einheitliche Sprache mit einheitlichen, fachspezifischen Ausdrücken. Einen wesentlichen Beitrag dazu leistete der britische Ökonom und Nachfolger Jevons in Cambridge Alfred Marshall.

\section{Die Vollendung der neoklassischen Mikroökonomie}

\subsection{Marshall: Der Beginn der modernen Ökonomie}

\textcite{Marshall1890}: \textit{Principles of Economics - An Introductory Volume} gilt bis heute als eines der prägendsten Lehrbücher aller Zeiten. Es fasst nicht nur den State of the Art der Ökonomie zusammen, sondern erweiterte denselben auch. Der durchschlagende Erfolg dieses Buches hängt sicherlich auch damit zusammen, dass sich die mikroökonomische Theorie seit damals kaum mehr verändert hat. Natürlich wurde sie entscheidend und an vielen Stellen erweitert. Aber die damals schon bestehenden Theorien zur Mikroökonomie gelten bis heute unverändert und sind tatsächlich in einführenden Lehrbüchern praktisch identisch abgedruckt.

Was sein Privatleben anging, stammte Marshall aus einer hoch-religiösen Familie und war in seiner Kindheit vom "`tyrannischen"' Vater \parencite[S. 313]{Keynes1924} geprägt. Gegen den Willen seines Vaters und mit der Hilfe eines Darlehens von seinem Onkel studierte er Mathematik in Cambridge. Nach dem Abschluss seines Studiums nahm er 1868 bereits seine Lehrtätigkeit dort auf. Er publizierte in seinen frühen Jahren wenig, obwohl er viel verfasst und sich auf Reisen auch viel Wissen in den USA und Kontinental-Europa aneignete. 1877 musste er Cambridge verlassen, weil er eine ehemalige Studentin heiratete. 1885 aber konnte er zurückkehren. Ab dann stieg er rasch zum führenden englischen Ökonomen auf. Bereits 1879 hatte sein Mentor Henry Sidgwick einige seiner Werke veröffentlicht, die viel Aufsehen erregten. 1890 folgte sein Hauptwerk, die "`Principles of Economics"'. In weiterer Folge war er auch als Berater öffentlicher Stellen in England tätig. Ständig erweiterte er nebenbei seine "`Principles"'. Als Person, die empfindlich auf Kritik reagierte, wurde er zum "`Workaholic"', was schließlich seine Gesundheit angriff. 1908 wurde er emeritiert. In seinen letzten eineinhalb Lebensjahrzehnten publizierte er weiter, getrieben von seinem Perfektions-Drang \parencite[S. 145f]{Rieter1989}. Die inhaltliche Bedeutung seiner späten Arbeiten blieb aber beschränkt.

Mit Marshall änderte sich der Blick auf die Ökonomie grundlegend. Er gilt daher nicht umsonst als der "`Vollender der Neoklassik"'. Neben inhaltlicher Punkte, revolutionierte Marshall vor allem auch die ökonomische Methodologie und allgemein das Bild der Ökonomie als Wissenschaft:

Erstens, zunächst besticht sein Hauptwerk durch die Kombination von sprachlicher Verständlichkeit und mathematischer Präzision. Vergleicht man sein Werk mit jenen von, zum Beispiel, Menger, Walras oder Böhm-Bawerk, so merkt man sofort deutliche Unterschiede in Aufbau, sowie ein bessere Verständlichkeit. Dies ist zwar nicht unbedingt ausschließlich der Arbeit Marshalls eigen, sondern eher dem Zeitgeist zuzuschreiben und so auch zum Beispiel bei Irving Fisher zu finden, aber es besticht dennoch in \textcite{Marshall1890}. 

Zweitens, Marshall grenzte als erster die Volkswirtschaftslehre ("`Economics"') als Wissenschaft von den übrigen Sozialwissenschaften ab. Dazu "`kürzte"' er die politischen Entscheidungs- und Machtstrukturen aus seinem wirtschaftswissenschaftlichem Werk. Die bis dahin übliche Bezeichnung der "`Politischen Ökonomie"', wurde zur "`Volkswirtschaftslehre"'. Dies war für sein Werk notwendig. Die von ihm so elegant durchgezogene formale Herangehensweise ist nur dann möglich, wenn man Nebenbedingungen definiert und diese als gegeben annimmt. Dies ist eine der Stärken der Neoklassik. Es brachte ihr aber auch harsche Kritik ein, weil sich die Volkswirtschaftslehre damit von der erlebten Realität ein gutes Stück entfernt. Diese Kritik ist im Falle Marshalls ungerechtfertigt. Er war sich durchaus bewusst, dass seine "`Principles"' eine zu starke Vereinfachung der wirtschaftlichen Realität darstellen. Dementsprechend plante er, ja rang förmlich, um die Entstehung einer Fortsetzung seines Werkes \parencite[S. 146]{Rieter1989}. Tatsächlich ließ er erst mit der sechsten Auflage der "`Principles of Economics"' den Zusatz "`Volume I"' streichen. Mit 80 Jahren - 1920: "`Industry and Trade"', sowie 1923: "`Money Credit and Commerce"' - publizierte er schließlich seine Vorarbeiten zu weiteren Bänden, diese können aber nur mehr als fragmentierte Beiträge zu einzelnen Themenkomplexen gesehen werden. Die Abgrenzung der Volkswirtschaftslehre betrieb Marshall nicht nur inhaltlich, sondern auch organisatorisch. An der Universität von Cambridge setzte er durch, dass sie aus der Fakultät für "`Moral Science"' herausgelöst und stattdessen eine eigene Fakultät, mit eigenem Studiengang wurde \parencite[S. 141]{Rieter1989}. Nicht nur \textcite[S. 365]{Keynes1924} bezeichnete Marshall daher als "`Begründer der Cambridge School of Economics"'.

Drittens, Methodisch gilt die Arbeit Marshall's heute vielen als bahnbrechender als seine inhaltlichen Beiträge. Er führte die Ceteris-Paribus-Betrachtung in die Ökonomie ein. Also die Auswirkungen der Änderung eines einzelnen Einflussfaktors auf das Gesamtergebnis. Diese Betrachtung von statischen Gleichgewichten blieb in der Ökonomie lange der Standard. Marshall selbst sah darin einen Ausgangspunkt, war sich aber bewusst, dass eine dynamische Betrachtung besser wäre, aber in Modellen auch schwerer zu erfassen \parencite[S. 153]{Rieter1989}.

Viertens, Er war als einer der ersten Ökonomen extrem gut informiert über den "`State of the Art"' der Ökonomie. Heute ist es unumgänglich die wissenschaftlichen Arbeiten des Forschungszweiges, in welchem man selbst publizieren möchte, umfänglich zu kennen. Im 19. Jahrhundert allerdings sorgten Probleme der Sprache, Verfügbarkeit und bloßen Kenntnis dafür, dass oft Theorien entwickelt wurden, ohne dass die Urheber den Stand der Wissenschaft kannten. Wir erinnern uns zum Beispiel an Gossen. Aber auch Walras verfasste zunächst sein Hauptwerk und trat erst gegen Ende seiner Karriere in intensiven Austausch mit zeitgenössischen Ökonomen. Ganz anders war dies bei Marshall: Wie \textcite{Groenewegen1995} beschreibt, beschäftigte sich dieser bereits in den späten 1860er Jahren mit den deutschen Ökonomen wie Thünen, Roscher und Rau und kannte die Arbeiten der Franzosen Cournot und Dupuit. Selbstverständlich waren ihm die englischen Klassiker bekannt, aber auch die Historische Schule der Deutschen war ihm nicht fremd \parencite[S. 140]{Rieter1989}. Aufbauend auf all dem Wissen publizierte er sein Hauptwerk. Dies verhältnismäßig spät mit 48 Jahren im Jahr 1890. Die Erkenntnisse waren zu dieser Zeit bereits allesamt weitgehend bekannt und zirkulierten als Mitschriften aus seinen Vorlesungen. Er zitierte in den "`Principles"' auch nur wenig seinen Zeitgenossen Jevons, sondern eben vor allem die "`Vorläufer"'. Es gilt auch bis heute als umstritten, ob Marshall tatsächlich primär der \textit{Entwickler} bedeutender ökonomischer Theorien ist, oder doch eher der \textit{Vereiniger} ökonomischer Elemente, die bereits jeweils jemand anderer entwickelt hatte \parencite[S. 207ff]{Ekelund2002}. Marshall selbst behauptete in hohem Alter, dass er den Rahmen für seine "`Principles"' schon vor 1871, also dem Erscheinungsjahr der Arbeiten von Jevons und Menger, fertig hatte und er aus diesem Grund mehrheitlich die "`Vorläufer"' und kaum das Trio Jevons, Menger und Walras, zitiert hatte \parencite[S. 140]{Rieter1989}. 

Erst jetzt kommen wir zu seinen inhaltlichen Beiträgen: Allseits bekannt ist sicherlich das "`Marshall'sche Kreuz"', also seine Darstellung von Angebot und Nachfrage und dem daraus entstehenden Gleichgewicht. Darüber ob Marshall damit eine bahnbrechende Leistung erbracht habe, diskutierten Generationen von Ökonomen. Bekannt ist, dass die erste Darstellung von Angebot und Nachfrage als sich schneidende Kurven um das Jahr 1840 entstand und auf Antoine-Augustin Cournot zurückgeht \parencite[S. 3]{Humphrey1992}, oder eventuell von Karl Rau sogar schon noch etwas früher so dargestellt wurde \parencite[S. 159]{Blaug2001}. Definitiv auf Marshall geht damit aber die Verbindung zwischen marginalistischer Nachfragefunktion und eher klassischer Angebotsfunktion zurück. Die negativ verlaufende Nachfragefunktion leitet Marshall aus dem sinkenden Grenznutzen beim Konsum von Gütern ab. Die steigende Angebotsfunktion von den mit steigender Menge steigenden Produktionskosten. Dies findet sich bereits bei Ricardo. Wenn dieser auch davon ausging, dass die Produktionskosten deshalb stiegen, weil knappe Ressourcen in abnehmender Qualität zur Verfügung stehen würden.

Quasi als Nebenprodukte seiner Herleitung von Nachfrage- und Angebotsfunktion, schuf er Instrumente, die heute noch gängige Praxis sind. So zum Beispiel die Elastizität, die er konkret als relative Änderungen der nachgefragten Menge bei Änderung des Preises definiert. Schon Marshall führt ein Beispiel von Robert Giffen an, der darlegte, dass die Brot-Nachfrage in Irland im 19. Jahrhundert trotz steigender Preise anstieg. Bis heute lernt man das seltene Phänomen, dass trotz steigender Preise die Nachfrage steigt, als Giffen-Paradoxon (bzw. Giffen-Gut) kennen. Ebenfalls direkt auf \textcite{Marshall1890} geht die Analyse der Produzenten- und Konsumentenrente in heute üblicher Form zurück. Also die Differenz zwischen Preis, zu dem ein Produzent anbieten würde und dem Marktpreis, bzw. dem Preis, den ein Nachfragender bereit wäre zu bezahlen und dem Marktpreis. Marshall behandelte auch schon die Auswirkungen von Steuern auf diese die Konsumentenrente \parencite[S. 351]{Rosner2012}. 

Das Gesamtwerk Marshalls bildet bis heute die Grundlage der Mikroökonomie. Im idealisierten Bild des vollkommenen Marktes, auf dem vollständige Konkurrenz herrscht, ohne Zugangsschranken, mit streng nutzen-maximierenden Teilnehmern und ohne jegliches Marktversagen, gilt noch heute: "`It's all in Marshall!"'\footnote{Ein Zitat, das Arthur C. Pigou zugeschrieben wird, aber wohl nur sinngemäß tatsächlich geäußert wurde. \parencite{Audretsch2007, Pigou1925}}. Tatsächlich war die ursprüngliche "`neoklassische Theorie"' mit Marshall gewissermaßen abgeschlossen, wenn man dies in dem Sinne versteht, dass seine Erkenntnisse wie in seinem Werk "`Principles of Economics"' auch in modernen Mikroökonomie Büchern praktisch identisch dargestellt werden\footnote{Wichtige Ergänzungen etwa im Hinblick auf die Nutzendarstellung kamen etwas später noch von Vilfredo Pareto und Francis Edgeworth.}. Diese "`Vollendung der Neoklassik"' kann deshalb vielleicht als Ausgangspunkt der \textit{modernen} Ökonomie gelten.

Wie soeben dargestellt: Welcher Umfang von bedeutenden Beiträgen erst durch Marshall bekannt wurde, ist unumstritten, worin seine bahnbrechende Leistung nun tatsächlich lag, kann hingegen nicht eindeutig beantwortet werden. Nicht durch die \textit{eine} bedeutende Leistung gilt Marshall als einer der bedeutendsten Ökonomen, sondern durch sein Gesamtwerk. Wie \textcite{Rieter1989} es ausdrückt: "`Man empfindet ihn als Ganzes [...]. Ein komfortabler Neubau, errichtet auf alten Fundamenten."' 

Marshall war von verschiedensten Richtungen Kritik ausgesetzt. Wenig überraschend lehnten ihn die Sozialisten praktisch einfach grundsätzlich ab. Aber auch seine Zeitgenossen aus der "`Historischen Schule"'(vgl. Kapitel \ref{Historisch}) kritisierten ihn heftig, vor allem für die unrealistischer Annahmen, die für seine Modell notwendig sind und die gesetzmäßigen Zusammenhänge, die diese Modelle liefern. Die Amerikaner um Veblen (vgl. Kapitel \ref{Institut}) kritisierten, dass Marshall das Marktwirtschaftliche Wirtschaftssystem implizit als effizientes und gerechtes System akzeptierte. Sogar die österreichische Schule hielt wenig von Marshall's Lehre \parencite[S. 151]{Rieter1989}. Die Kritik traf Marshall hart und ungerechtfertigt. Er selbst stellte "`wirkliche"' ökonomische Probleme in den Vordergrund, hielt wenig vom starren Rationalprinzip und nannte die Bekämpfung der Armut als zentrales Ziel der Volkswirtschaftslehre \parencite{Rieter1989}. Heute wissen wir, dass sich Marshall's Lehren, entgegen aller Kritik, als Mainstream durchgesetzt haben. Interessant ist in diesem Zusammenhang, dass die Kritik an der Neoklassik bis heute eine ähnliche geblieben  ist. Auf wissenschaftlich stärker fundierter Ebene wurde und wird nach wie vor primär die Realität verschiedener Modellannahmen in Zweifel gezogen. Auch heute gibt es kaum einen Ökonomen, der die Modellannahmen für 100\% richtig hält und die gesetzmäßige Gültigkeit der Modellergebnisse als gegeben annimmt. Aber auf der anderen Seite hat sich bis heute in der Mikroökonomie keine umfassende Alternative zur Neoklassik durchgesetzt.  

Die moderne ökonomische Forschung \parencite{Ekelund2002, Blaug2001, Humphrey1992} sieht Marshall eher als "`Synthesizer"', denn als Entwickler, \parencite[S. 212]{Ekelund2002} der ökonomischen Theorien zum neoklassischen Gesamtwerk. Aber er gilt auch als Entwickler der modernen wirtschaftswissenschaftlichen Methodik. So schuf er den Rahmen für die lange vorherrschende statisch-komparative Analyse und verband induktive Theoriebildung mit deduktiver empirischen Überprüfung \parencite[S. 212]{Ekelund2002}. Insgesamt war er auf jeden Fall \textit{die} prägende Figur der frühen Neoklassik im England des späten 19. Jahrhunderts. Seine Arbeiten sind wohl die frühesten, die noch heute fast unverändert Teil der Mainstream-Ökonomie sind. Konkret wenn es um die mikroökonomische Analyse auf vollständigen Konkurrenzmärkten geht. 

Wir wissen aber natürlich, dass sich die Wirtschaftswissenschaften seither vielfältig weiterentwickelt haben. Sein Nachfolger in Cambridge, Arthur C. Pigou, machte sich als einer der ersten Gedanken über "`Marktversagen"', also Situationen, in dem eine rein rationale-mikroökonomische Analyse unerwünschte Marktergebnisse zum Vorschein bringt (vgl. Kapitel \ref{sec: Pigou}. Die Verbindung seiner Arbeiten mit der Gleichgewichtstheorie von Walras (vgl. Kapitel \ref{Arrow-Debreu}) und nicht zuletzt die neoklassische Wachstumstheorie (vgl. Kapitel \ref{sec: Solow-Modell}), sind Forschungsgebiete, die die Marshall'sche Neoklassik nach 1945 wesentlich weiterentwickelten.


\section{Edgeworth und Pareto: Die Lösung des Nutzenproblems, die bis heute hinkt}

Fassen wir zusammen: Eines, wenn nicht \textit{das}, die Neoklassik auszeichnende Element, ist das Nutzenkonzept, bzw. die Theorie des abnehmenden Grenznutzens. Dieses Konzept ist an und für sich intuitiv verständlich und leicht nachzuvollziehen: Nach einer langen Wanderung liefert mir das erste Bier einen enormen Nutzen, das fünfte Bier liefert ebenfalls einen Nutzen, doch ist dieser zweifelsohne deutlich geringer. Alleine dieses Konzept des abnehmende Grenznutzens lässt eine grafische Transformation von Geldeinheiten in Nutzeneinheiten schon zu: Der erste Euro liefert den höchsten Nutzen, der zweite einen etwas geringeren, der dritte Euro eine wiederum etwas geringeren, usw. Wenn man den Nutzen auf der y-Achse und die Geldeinheiten auf der x-Achse abträgt, erhält man eine Funktion die vom Ursprung ausgehend durchgehend einen positiven Anstieg aufweist. Der Anstieg verläuft dabei aber immer flacher. Vorausgesetzt eine Nutzenfunktion erfüllt diese Anforderung, dann können Geldeinheiten einfach in Nutzeneinheiten "`umgerechnet"' werden. Der entsprechende Nutzen wird dann in Zahlen ausgedrückt. Man spricht vom "`kardinalen Nutzenprinzip"'. Genau hier liegt aber ein Problem, das die frühen Neoklassiker\footnote{Walras und Menger behandelten das Problem tatsächlich nicht, Jevons meinte zwar, Nutzen sei nicht direkt messbar, er akzeptierte aber den Umweg über Geldeinheiten. Der Nutzen verschiedener Güter kann demnach im äquivalenten Geldwert ausgedrückt werden.} schlicht ignoriert haben \parencite[S. 328]{Blaug1962}: Nutzen ist in Wirklichkeit nicht direkt messbar. Es gibt keine sinnvolle Einheit in der man Nutzen quantifizieren könnte. Dementsprechend sind Rechenoperationen mit Nutzeneinheiten sinnlos.

Der erste, der dies ausführlich thematisierte war Francis Ysidro Edgeworth. Er war sowohl mit Stanley Jevons als auch mit Alfred Marshall befreundet und auch ein früher Verfechter der Mathematik in der Ökonomie. In seinen "`Mathematical Psychics"' \parencite{Edgeworth1881} kritisierte er, dass die Neoklassiker Nutzen unzulässigerweise als quantitative Variable behandelten. \textcite{Edgeworth1881} schlug vor nach Wegen zu suchen, den Nutzen tatsächlich direkt zu messen. Er verfolgte also auch ein kardinales Nutzenkonzept. Dazu wollte er einen "`Hedonimeter"' entwickeln, also ein Messgerät, dass den Nutzen direkt messen kann.  Übrigens verfolgte wenig später auch der junge Irving Fisher - der uns noch mehrmals unterkommen wird - in seiner Dissertation das Ziel einer kardinalen Nutzenmessung, allerdings schlug er vor diese indirekt vorzunehmen, also von getätigten Handlungen auf deren Nutzen zu schließen\parencite{Colander2007}. Das Konzept blieb schließlich in der Ökonomie ohne wesentliche Resonanz\footnote{Moderne Ansätze der Neuro-Ökonomie gehen allerdings wieder in Richtung kardinaler Nutzenmessung. Dabei wird in Magnetresonanz-Tomographen versucht Gehirnströme hinsichtlich Glücksgefühle zu messen}. Allerdings lieferte \textcite{Edgeworth1881} dennoch wichtige Bausteine für die Nutzentheorie. So entwickelte er darin das Konzept der - heute in der Ökonomie-Lehre nach wie vor omnipräsenten und ebenso beliebten - Indifferenzkurven. Abgeleitet können diese aus einer, wie oben beschriebenen, Nutzenfunktion. Interessant sind die Indifferenzkurven im Zwei-Güter-Fall. Angenommen ich bilde in einem Koordinatensystem die Menge von Gut A auf der x-Achse und die Menge von Gut B auf der y-Achse ab. Wenn ich mein ganzes Geld für Gut A ausgebe erreiche ich einen bestimmten Punkt direkt auf der x-Achse (und vice versa). Aus dem Konzept des abnehmenden Grenznutzens wissen wir, dass eine Güterkombination aus A und B gegenüber nur A (oder nur B) vorteilhaft ist. Oder mit anderen Worten: Ich bekomme für eine geringere Geldmenge, die ich für eine Güterkombination ausgebe den gleichen Nutzen, wie für ein höhere Geldmenge, die ich ausschließlich nur für A (oder nur für B) ausgebe. Verbindet man alle Güterkombinationen aus A und B, die den identischen Nutzen liefern, miteinander spricht man von einer Indifferenzkurve. Diese beginnt jeweils an einem Punkt auf der x- und y-Achse und ist zum Ursprung geneigt \parencite[S. 21ff]{Edgeworth1881}. Bildet also eine Linkskurve ab, bzw. verläuft konvex. \textcite{Edgeworth1881} beschreibt in weiterer Folge, wie zwei Personen miteinander über das Austauschverhältnis dieser zwei Güter verhandeln. Man stelle sich nun das soeben beschriebene Koordinatensystem mit zwei Personen vor. Zusätzlich zur Person A, deren Ausgangspunkt der Ursprung, also "`links unten"' ist, eine zweite Person B, deren Ausgangspunkt "`rechts oben"' ist. Seine Indifferenzkurven verlaufen spiegelverkehrt zu jenen von A, das heißt diese sind zum Punkt "`rechts oben"' geneigt. B hält in diesem Fall alle Güter 1 (aber kein 2), A hält die gesamte Menge 2 (aber keine 1). Beide wollen nun in einen Tauschprozess kommen. Mögliche "`Tauschpunkte"' sind überall dort wo sich die Indifferenzkurven der beiden Personen schneiden. Ein Tauschgleichgewicht und gleichzeitig eine maximale aggregierte Wohlfahrt (welfare) wird aber bei Edgeworth nur in einem Punkt erreicht, nämlich wo sich die Indifferenzkurve von A und B genau tangieren. Dies ist aber eine falsche Annahme seitens Edgeworth - es gibt mehrere Tangentialpunkte und vor allem keine Möglichkeit eine "`allemeines Nutzenmaximum"' zu identifizieren \parencite[S. 49]{Humphrey1996}. Das soeben beschriebene Tool hat dennoch extreme Bedeutung erlangt ist heute als die "`Edgeworth-Box"' bekannt\footnote{Edgeworth selbst stellte die Box leicht abweichend dar, nämlich mit den Personen A und B "`rechts unten"', bzw. "`links oben"'. Details zur reichhaltigen Geschichte der Edgeworth-Box sind in \textcite{Humphrey1996} dargestellt}. Sie ist ein wichtiges Element in der allgemeinen Gleichgewichtstheorie zu der wir gleich wieder kommen werden. In \textcite{Edgeworth1881} sind beide Konzepte, also Indifferenzkurven und Edgeworth-Box, mathematisch und verbal beschrieben. Bekannt gemacht und angewendet hat beides schließlich Vilfredo Pareto, wobei er der erste war, der beide Konzepte grafisch wie heute üblich darstellte.

Was Marshall nicht beachtete waren die Fragen nach dem Allgemeinen Gleichgewicht im Sinne Walras'. Laut \textcite[S. 360]{Rosner2012} war Marshall das Werk von \textcite{Walras1874} zwar bekannt, aber er hatte ihm offenbar nicht den Stellenwert beigemessen, des es später erhalten sollte. Damit die Verbindung zu einem weiter wichtigen Wegbereiter der "`älteren"' Neoklassik vollends hergestellt: Vilfredo Pareto. 


Nachfolger von Walras in Lausanne. Deutsche Name. Herkunft Italien und Frankreich, War zunächst als Ingeneur tätig und wurde 1893 nach Lausanne als Nachfolger des kränklichen Walras berufen, mit dem er sich bald überwarf.

Wesentlicher Beitrag: "`Theorie der Wahlakte"': Ordinale Nutzentheorie anstatt kardinaler als wesentlicher Fortschritt. Benutzte dazu Edgworth Framework. (auch S. 394 Rosner) Nutzenaggregation nicht möglich!

Einbau des ordinalen Nutzens in Gleichgewichtstheorie und dabei Entwicklug des Konzepts der "`Pareto-Effizienz"' (Pareto-Optimum): Niemand kann besser gestellt werden, ohne dass ein anderer schlechter gestellt wird. 

Damit Anstoß zur "`Welfare Economics"' gegeben, die später Pigou weiterentwickelte. 

Nicht zu vergessen: Empirische Arbeiten zur personellen Einkommensverteilung: "`Pareto-Verteilung"'. Pareto-Prinzip: 80-20-Regel weitverbreitet bekannt auch außerhalb VWL.







The rediscovery of Pareto optimality in the 1930s after
26 years of neglect \textcite[S. 148]{Blaug2001}


Pareto: homo oeconomicus


Pareto: Nicht-kardinaler Nutzen, sondern ordinale Nutzenfunktion.




Später: Von Neumann-Morgenstern-Risikonutzenfunktion
Allais-Paradoxon.


Seite 397 in Rosner:

Grenzrate der Substitution ersetzt Grenznutzen
HIER WEITER:

Hicks und Samuelson in den 1930 Jahren

Sir John R. Hicks (1904-1989) lehrte an der London School of Economics (LSE), an der Universität 
von Manchester und in Oxford. 1972 erhielt er (gemeinsam mit Arrow) den Nobelpreis. Er vollendete 
die Marshallsche Konsumtheorie, indem er konsequent auf die Quantifizierung des Nutzens 
verzichtete und die Indifferenzkurven für die Herleitung der individuellen Nachfragekurve nutzte 
("Eine erneute Betrachtung der Werttheorie", 1934). Die seither übliche Darstellung eines 
nutzenmaximierenden Haushaltes als Tangentiallösung mit einer Indifferenzkurve und der 
Budgetgeraden benötigt keine kardinale (eindeutig bis auf eine lineare Transformation) 
Nutzenmessung mehr, sondern begnügt sich mit der bloßen ordinalen (eindeutig bis auf eine 
monotone Transformation) Messung. "The quantitative concept of utility is not necessary in order to 
explain market phenomena". Damit fällt auch das Konzept des Grenznutzens und wird durch das der 
Grenzrate der Substitution ersetzt ("Value and Capital", 1939): "The consumer is only in full 
equilibrium if the marginal rate of substitution between any two goods equals their price-ratio." 
Die graphische Darstellung als Tangentiallösung von Budgetgerade und Indifferenzkurve findet sich 
in jedem mikroökonomischen Lehrbuch. Die Einkommen-Konsum-Kurve ermöglicht die Herleitung 
von Engelkurven und die Preis-Konsum-Kurve die von Preis-Nachfragekurven. Durch die Zerlegung 
des von einer Preisänderung ausgelösten Nachfrageeffektes in einen Substitutions- und einen 
Einkommenseffekt - beide im Normalfall negativ, d.h. der Preisänderung entgegengesetzt wirkend


%https://de.wikipedia.org/wiki/Nutzen_(Wirtschaft)

\section{Fisher and Clark: Economics goes USA}
Der erste amerikanische Top-Ökonom?
Zinstheorie (Intertemporale Konsumentscheidung)

Fisher-Separation
beides schon im Kapitel \ref{Finance}
Debt-Deflation-Theory
Quantitätsgleichung als Vorläufer zum Monetarismus
Erster Benutzer von Indexnummern. (Vereinzelt wird dies aber auch schon Stanley Jevons zugeschrieben \parencite[S. 232]{Jevons1934})

Einer der ersten "`modernen"' Ökonomen. Seine Ideen sind teilweise bis heute unverändert in Anwendung

Seine Darstellung war wesentlich besser - im Sinne eine eindeutigeren und formaleren Darstellung - als jene der Neoklassiker, wie zum Beispiel Böhm-Bawerk.


\textcite{Tobin2005}