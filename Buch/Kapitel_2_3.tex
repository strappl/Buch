%%%%%%%%%%%%%%%%%%%%% chapter.tex %%%%%%%%%%%%%%%%%%%%%%%%%%%%%%%%%
%
% sample chapter
%
% Use this file as a template for your own input.
%
%%%%%%%%%%%%%%%%%%%%%%%% Springer-Verlag %%%%%%%%%%%%%%%%%%%%%%%%%%

\chapter{Das Ende der Ökonomie?}
\label{Neoklassik}

\section{Marshall: Der Vollender}

Erste Darstellung Angebot und NAchfrage (Kreuz). In wirklichkeit: bereits in Rau 1826! \textcite[S. 159]{Blaug2001}


In den USA: John Bates Clark?


\section{Edgeworth und Pareto}
The rediscovery of Pareto optimality in the 1930s after
26 years of neglect \textcite[S. 148]{Blaug2001}




\section{Clark}




\section{Fisher: Economics goes USA}
Der erste amerikanische Top-Ökonom?
Zinstheorie (Intertemporale Konsumentscheidung)

Fisher-Separation
beides schon im Kapitel \ref{Finance}
Debt-Deflation-Theory
Quantitätsgleichung als Vorläufer zum Monetarismus
Erster Benutzer von Indexnummern.

Einer der ersten "`modernen"' Ökonomen. Seine Ideen sind teilweise bis heute unverändert in Anwendung

Seine Darstellung war wesentlich besser - im Sinne eine eindeutigeren und formaleren Darstellung - als jene der Neoklassiker, wie zum Beispiel Böhm-Bawerk.


\textcite{Tobin2005}