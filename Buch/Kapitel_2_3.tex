%%%%%%%%%%%%%%%%%%%%% chapter.tex %%%%%%%%%%%%%%%%%%%%%%%%%%%%%%%%%
%
% sample chapter
%
% Use this file as a template for your own input.
%
%%%%%%%%%%%%%%%%%%%%%%%% Springer-Verlag %%%%%%%%%%%%%%%%%%%%%%%%%%

\chapter{Das Ende der Ökonomie?}
\label{Neoklassik}

\section{Marshall: Der Vollender}

Erste Darstellung Angebot und NAchfrage (Kreuz). In wirklichkeit: bereits in Rau 1826! \textcite[S. 159]{Blaug2001}


Tatsächlich war die ursprüngliche "`neoklassische Theorie"' mit Marshall abgeschlossen, wenn man dies in dem Sinne versteht, dass seine Erkenntnisse wie in seinem Werk "`Principles of Economics"' auch in modernen Mikroökonomie Büchern praktisch identisch dargestellt werden. Diese "`Vollendung der Neoklassik"' kann deshalb vielleicht als Ausgangspunkt der \textit{modernen} Ökonomie gelten. 


Wir wissen aber natürlich, dass sich die Wirtschaftswissenschaften seither vielfältig weiterentwickelt haben.  Sein Nachfolger in Cambridge machte sich als einer der ersten Gedanken über "`Marktversagen"', also 



In den USA: John Bates Clark?


Sein direkter Nachfolger in Cambridge: Pigou. Seine Arbeiten könnten auch in diesem Kapitel dargestellt werden, da er direkt an Marshall's Wert anschloss. Allerdings war Pigou's späteres Werk geprägt von den Angriffen auf seine Theorie durch Keynes. 







\section{Edgeworth und Pareto}
The rediscovery of Pareto optimality in the 1930s after
26 years of neglect \textcite[S. 148]{Blaug2001}




\section{Clark}




\section{Fisher: Economics goes USA}
Der erste amerikanische Top-Ökonom?
Zinstheorie (Intertemporale Konsumentscheidung)

Fisher-Separation
beides schon im Kapitel \ref{Finance}
Debt-Deflation-Theory
Quantitätsgleichung als Vorläufer zum Monetarismus
Erster Benutzer von Indexnummern. (Vereinzelt wird dies aber auch schon Stanley Jevons zugeschrieben \parencite[S. 232]{Jevons1934})

Einer der ersten "`modernen"' Ökonomen. Seine Ideen sind teilweise bis heute unverändert in Anwendung

Seine Darstellung war wesentlich besser - im Sinne eine eindeutigeren und formaleren Darstellung - als jene der Neoklassiker, wie zum Beispiel Böhm-Bawerk.


\textcite{Tobin2005}