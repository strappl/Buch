%%%%%%%%%%%%%%%%%%%%% chapter.tex %%%%%%%%%%%%%%%%%%%%%%%%%%%%%%%%%
%
% sample chapter
%
% Use this file as a template for your own input.
%
%%%%%%%%%%%%%%%%%%%%%%%% Springer-Verlag %%%%%%%%%%%%%%%%%%%%%%%%%%

\chapter{Das Ende der Ökonomie?}
\label{Neoklassik}

\section{Edgeworth und Pareto}

\section{Clark}

\section{Marshall}

\section{Pigou}

\section{Fisher}

\section{Nach dem Zweiten Weltkrieg}

Arrow-Debreu:
Das Arrow-Debreu Gleichgewichtsmodell (auch: Arrow-Debreu-McKenzie-Modell) ist ein mikroökonomisches Modell der gesamten Volkswirtschaft. Es ist nach Gérard Debreu und Kenneth Arrow sowie Lionel W. McKenzie benannt, stellt eine Weiterentwicklung des von Léon Walras entwickelten walrasianischen Gleichgewichtsmodells dar und untersucht einen gesamtwirtschaftlichen Gleichgewichtszustand. 
Das Modell erweitert das allgemeine Gleichgewichtsmodell um unsichere Erwartungen und zustandsabhängige Größen und ist damit für die Finanzierungstheorie von großer Bedeutung. Es zeigt, dass es in einer Marktwirtschaft unter idealisierenden Bedingungen nicht möglich ist, jemanden besserzustellen, ohne jemand anderen schlechterzustellen. Kurz gesagt ist ein Marktgleichgewicht ein Pareto-Optimum. 