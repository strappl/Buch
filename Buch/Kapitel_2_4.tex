%%%%%%%%%%%%%%%%%%%%% chapter.tex %%%%%%%%%%%%%%%%%%%%%%%%%%%%%%%%%
%
% sample chapter
%
% Use this file as a template for your own input.
%
%%%%%%%%%%%%%%%%%%%%%%%% Springer-Verlag %%%%%%%%%%%%%%%%%%%%%%%%%%

\chapter{Die Neoklassik neben Keynes}
\label{Neoklassik_nach1945}

Teil des Kapitels könnte auch Finanzierungstheorie, Spieltheorie sein (sind aber eigene Kapitel), aber auch die ganze Input-Output-Analyse.


\section{Pigou: Die Neoklassik erfindet sich neu}
\label{sec: Pigou}
Arthur Cecil Pigou ist der letzte große Ökonom in der Reihe der englischen Vertreter der Klassik und Neoklassik in Cambridge. Nach Pigou entwickelte sich die Ökonomie dort in Richtung Keynesianismus weiter, wie wir im letzten Kapitel (vgl. Kapitel \ref{Keynes}) bereits gelesen haben. Pigou ist wohl einer der meist unterschätzten Ökonomen der ersten Hälfte des 20. Jahrhunderts. Schließlich war er der erste Mainstream-Ökonom, der postulierte und auch anerkannte, dass rein marktwirtschaftliche Ergebnisse nicht immer effizient sein müssen. Er erkannte also bereits vor Keynes, und auf ganz andere Weise als dieser, die Sinnhaftigkeit von Staatseingriffe in bestimmten Situationen. 

Seine Thesen zu den verschiedenen Formen des Marktversagen sind heute - Stichwort Klimawandel - aktueller denn je, wie gleich konkretisiert werden wird. Seine Karriere begann dann auch wie die eines ganz großen: Er war der Lieblingsschüler Marshall's der damals unangefochtenen Lichtgestalt der englischen Ökonomie. Mit erst 30 Jahre wurde Pigou schließlich dessen Nachfolger an der Universität in Cambridge. Und bereits 1912 legte er das bedeutende Werk "Wealth and Welfare" auf, das in der Neuauflage von 1920 als "The Economics of Welfare" ein bis heute bedeutendes Werk darstellt. Darin werden eben unter anderem die Abhandlungen zu Marktversagen dargestellt, die heute wieder so aktuell diskutiert werden. Mit 43 Jahren war er allerdings bereits auf den Zenit seiner Karriere angekommen. Denn sein eigener Assistent, ein uns bereits aus dem letzten Kapitel bekannter, gewisser John Maynard Keynes, revolutionierte die Ökonomie. Pigou war einer jener Ökonomen, der an der fast magischen Strahlkraft des Keynesianismus und dessen Hauptvertreter, fast zerbrach. Ähnlich ging es übrigens Joseph Schumpeter, Michal Kalecki und in gewissen Maße auch Friedrich Hayek.   

Ab den 1930er Jahren wendete er sich von der Weiterentwicklung der "Wohlfahrtsökonomie" ab und wendete sich - getrieben durch die Arbeiten von Keynes - 




Beginnend mit Arthur Cecil Pigou entwickelte sich die Neoklassik in eine neue Richtung. Wie in Kapitel \ref{Neoklassik} dargestellt, kann die Neoklassische Theorie mit Marshall in gewisser Hinsicht als abgeschlossen gelten. Dies vor allem in dem Hinblick, dass sich die mikroökonomischen Ausgangstheorien seit damals tatsächlich kaum noch geändert haben. Selbst in modernen Mikroökonomie Büchern sind die ersten Kapitel im wesentlich identisch in den Arbeiten Marshall's zu finden. Die Mikroökonomie wurde seither nicht mehr grundlegend verändert, in dem Sinne, dass zuvor geltende Konzepte über den Haufen geworfen wurden, sondern sie wurde seither immer wieder erweitert. Marshall's Principles sind nach wie vor richtig. Aber man kann sie eben nicht mehr als Erklärung aller wirtschaftlichen Tätigkeiten heranziehen, sondern gelten mittlerweile nur mehr für einen sehr speziellen Fall. Heute wissen wir, dass die damals angenommenen Voraussetzungen, wie vollkommene Märkte und perfekte Konkurrenz nicht die Regel, sondern die Ausnahme sind.

Den ersten Schritt dieser Erweiterung der Neoklassik ging Arthur Cecil Pigou mit seinen 1920 erschienen "`Economics of Welfare"'. Er erweiterte die rein marktwirtschaftliche Analyse um Aspekte des Marktversagens und der gesamtwirtschaftlichen Wohlfahrt. Er brachte damit den Staat als wichtigen Player ins Spiel, ohne aber von den grundsätzlichen Ideen der Mikroökonomie abzugehen. Pigou 1920 stellt den Beginn der Mikroökonomischen Wirtschaftspolitik dar.

Wohlfahrtsökonomie und Marktversagen


\section{Die Welt im Arrow-Debreu-Gleichgewicht}
\label{Arrow-Debreu}
Ursprünglichste Form: Walras.
Vorarbeiten von Neumann (1937, siehe oben) und Leontieff. Danach: \textit{Ramsey-Cass-Koopmans!}

Arrow-Debreu:
Das Arrow-Debreu Gleichgewichtsmodell (auch: Arrow-Debreu-McKenzie-Modell) ist ein mikroökonomisches Modell der gesamten Volkswirtschaft. Es ist nach Gérard Debreu und Kenneth Arrow sowie Lionel W. McKenzie benannt, stellt eine Weiterentwicklung des von Léon Walras entwickelten walrasianischen Gleichgewichtsmodells dar und untersucht einen gesamtwirtschaftlichen Gleichgewichtszustand. 
Das Modell erweitert das allgemeine Gleichgewichtsmodell um unsichere Erwartungen und zustandsabhängige Größen und ist damit für die Finanzierungstheorie von großer Bedeutung. Es zeigt, dass es in einer Marktwirtschaft unter idealisierenden Bedingungen nicht möglich ist, jemanden besserzustellen, ohne jemand anderen schlechterzustellen. Kurz gesagt ist ein Marktgleichgewicht ein Pareto-Optimum. 



\section{Die Cobb-Douglas-Produktionsfunktion} \label{sec: Cobb-Douglas-Produktionsfunktion}
Die Cobb-Douglas-Produktionsfunktion wurde bereits in den 1920er Jahren entwickelt. Ihre Bedeutung erlangte sie aber erst später unter anderem als Ausgangspunkt der neoklassischen Wachstumstheorie. Daher hier anzusiedeln.



\section{Solow: Technischer Fortschritt als Wachstumsquelle} \label{sec: Solow-Modell}


Harrod-Domar Modell (später Kontroverse Solow - Robinson)
Vorläufer: Von-Neumann Growth-Theory: 
"Über ein ökonomisches Gleichungssystem und eine Verallgemeinerung des Brouwerschen Fixpunktsatzes",  1937, in K. Menger, editor, Ergebnisse eines mathematischen Kolloquiums, 1935-36. [English 1945 trans. as "A Model of General Economic Equilibrium", RES].

Ausgangspunkt ist eine Produktionsfunktion, wie jene, die wir gerade in Kapitel \ref{sec: Cobb-Douglas-Produktionsfunktion} kennen gelernt haben.  Der Output – gesamtwirtschaftlich das BIP – wird durch verschiedene Inputs – standardmäßig in der Neoklassik Arbeit und Kapital – hervorgebracht. Der Output ergibt sich also aus eine Kombination der beiden Inputfaktoren, Ökonomen würden sagen: „Der Output ist eine Funktion der Inputfaktoren“. 
Folgende Überlegung macht recht schnell klar, warum man mit diesem einfachen Modell stetige Wachstumsraten nicht erklären kann: Angenommen der Output ist einfach eine Addition der beiden Inputfaktoren Arbeit und Kapital. Möchte ich den Output verdoppeln, so müsste ich \textit{beide} Inputfaktoren Arbeit und Kapital jeweils verdoppeln. Ökonomen sprechen hier von konstanten Skalenerträgen. 
Was passiert aber wenn nur einer der beiden Input-Faktoren steigen kann? Gesamtwirtschaftlich könnte man argumentieren, dass der Produktionsfaktor Arbeit durch die Bevölkerungszahl begrenzt ist. Wenn man dies in unserer Überlegung berücksichtigt, würden wir folglich nicht beide Inputfaktoren gleichzeitig erhöhen, sondern nur einen, nämlich Kapital. Erhöhen wir diesen Inputfaktor nun um eine Einheit und der andere Inputfaktor bleibt gleich, so steigt der Gesamtoutput zwar selbstverständlich an, allerdings pro zusätzlicher Einheit um einen immer geringeren Prozentsatz. Intuitiv ist das leicht verständlich: Erhöhe ich bei gleichbleibender Mitarbeiterzahl ständig das Kapital – zum Beispiel die Anzahl der Computer – dann wird der erste eingesetzte Computer einen hohen Zuwachs an Produktivität bringen. Mit jedem weiteren Computer wird die Produktivität zwar weiter steigen, allerdings mit immer geringerer Zuwachsrate. Wenn jeder Mitarbeiter mehr als einen Computer besitzt, wird der Produktivitätszuwachs verschwindend gering werden. Diesen Zusammenhang bezeichnen Ökonomen als „Abnehmenden Grenzertrag“.
Das würde aber bedeuten, dass bei ungefähr gleich bleibender Arbeitsbevölkerung – eine Annahme, die man zumindest für die mittlere Frist in Industriestaaten, bedenkenlos machen kann – die Wirtschaftsleistung stagnieren sollte. Geht man davon aus, dass der Kapitaleinsatz ständig steigt, wäre zwar stetiges Wachstum möglich, aber nur mit immer geringer werdenden Wachstumsraten \footnote{Ständig steigender Kapitaleinsatz wäre nur mit steigenden Sparquoten erklärbar. In einer Ökonomie ohne Außenhandel gilt ja, dass das Sparen der Haushalten den Investitionen der Firmen entspricht. Investitionen wiederum bedeuten einen Aufbau von Kapital. Spezielle Beobachtungen von Fällen von Wirtschaftswachstum werden tatsächlich darauf zurückgeführt, dass die Sparquoten gestiegen sind. So ist zum Beispiel die Ökonomie in der stalinistischen Sowjetunion tatsächlich beträchtlich gewachsen. Da man aber keine wesentlichen technologischen Vorsprünge des Landes in dieser Zeit ausmachen kann, vermutet man, dieses Wachstum sei eben alleine auf den Anstieg der Sparquote zurückzuführen.}. Langfristig würden die Zuwachsraten aber gegen Null tendieren, womit auch in diesem Fall die Wirtschaftsleistung stagniert.

Bisher haben wir aber eine Möglichkeit außer Acht gelassen: Nämlich, dass die eingesetzten Maschinen (hier als Synonym für Kapital verwendet) immer besser werden. Tatsächlich wird eine Arbeitskraft mit zwei Computern nicht wesentlich produktiver sein, als mit einem Computer. Sie könnte aber mit einem \textit{besseren} Computer wesentlich produktiver sein. Es könnte sich also nicht nur die \textit{Menge} des Kapitals verändern, sondern auch dessen \textit{Qualität}. Dies nennen wir „technischen Fortschritt“ \footnote{Technischer Fortschritt umfasst nicht nur die Weiterentwicklung bestehender Produkte zu „besseren“ Produkten, sondern auch die Einführung neuer Produkte}.
Berücksichtigt man diesen Umstand, kommt man zu dem Ergebnis, dass stetiges Wirtschaftswachstum nur dann möglich ist, wenn sich die eingesetzten Kapitalgüter – also zum Beispiel Maschinen, Computer, Transportmittel, Kommunikationsmittel – immer weiter verbessern.
Der Inhalt des „technischen Fortschritts“, also was sich wie verbessert – ist allerdings ist nicht Teil der Ökonomie. Die ersten Wachstumstheorien haben sich also damit abgefunden festzustellen, dass technischer Fortschritt für Wachstum notwendig ist, dieser selbst allerdings nicht durch ökonomisches Handeln beeinflusst werden kann. Der technische Fortschritt wurde also als „exogen“ betrachtet. Daher der Name „exogene Wachstumstheorie“.
















