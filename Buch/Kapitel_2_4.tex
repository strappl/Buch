%%%%%%%%%%%%%%%%%%%%% chapter.tex %%%%%%%%%%%%%%%%%%%%%%%%%%%%%%%%%
%
% sample chapter
%
% Use this file as a template for your own input.
%
%%%%%%%%%%%%%%%%%%%%%%%% Springer-Verlag %%%%%%%%%%%%%%%%%%%%%%%%%%

\chapter{Die Neoklassik neben Keynes}
\label{Neoklassik_nach1945}

Teil des Kapitels könnte auch Finanzierungstheorie, Spieltheorie sein (sind aber eigene Kapitel), aber auch die ganze Input-Output-Analyse.


\section{Pigou: Die Neoklassik erfindet sich neu}
\label{sec: Pigou}
Arthur Cecil Pigou ist der letzte große Ökonom in der Reihe der englischen Vertreter der Klassik und Neoklassik in Cambridge. Nach Pigou entwickelte sich die Ökonomie dort in Richtung Keynesianismus weiter, wie wir im letzten Kapitel (vgl. Kapitel \ref{Keynes}) bereits gelesen haben. Pigou ist wohl einer der meist unterschätzten Ökonomen des 20. Jahrhunderts. Schließlich war er der erste Mainstream-Ökonom, der postulierte und auch anerkannte, dass rein marktwirtschaftliche Ergebnisse nicht immer effizient sein müssen. Er erkannte also bereits vor Keynes die Sinnhaftigkeit von Staatseingriffe in bestimmten Situationen, wenn auch deren Wirkung und Berechtigung von ganz anderer Art und Weise sind. Während Keynes die makroökonomische Wirtschaftspolitik begründete, tat Pigou dies für die mikroökonomische Wirtschaftspolitik. Seine daraus unter anderem resultierenden Thesen zu den verschiedenen Formen des Marktversagen sind heute - Stichwort Klimawandel - aktueller denn je, wie gleich konkretisiert werden wird. Pigou's Karriere begann dann auch wie die eines ganz großen: Er war der Lieblingsschüler Marshall's der damals unangefochtenen Lichtgestalt der englischen Ökonomie. Mit erst 30 Jahre wurde Pigou schließlich dessen Nachfolger an der Universität in Cambridge. Und bereits 1912 legte er das bedeutende Werk "Wealth and Welfare" (\parencite{Pigou1912}) auf, das in der Neuauflage von 1920 als "The Economics of Welfare" (\parencite{Pigou1920}) ein bis heute bedeutendes Werk darstellt. Diese Jahreszahlen legen nahe, dass Pigou eigentlich auch in Kapitel \ref{Neoklassik} gut aufgehoben wäre. Inhaltlich von Bedeutung wurde der Forschungsbereich, den Pigou damit eröffnete, aber erst nach 1945. 

Mit zwei wesentlichen Punkten hat Pigou in seinem Hauptwerk die Neoklassik revolutioniert. Erstens, hat er mit dem Credo aufgeräumt, das seit Adam Smith die Neoklassik prägte, nämlich, dass individuell-nutzenmaximierendes Verhalten grundsätzlich auch für die gesamte Gesellschaft das optimale Ergebnis hervorbringt \parencite[S. 111]{Pigou1920}. Das hat enorme Auswirkungen. Damit verbunden ist nämlich die Notwendigkeit staatlicher Eingriffe, immer dann wenn der Markt darin versagt gesamtwirtschaftlich optimale Ergebnisse hervorzubringen. Das riesige Gebiet der mikroökonomischen Wirtschaftspolitik war damit geboren. Zweitens, wollte er einen formalen Ansatz zur Analyse des \textit{gesamt}wirtschaftlichen Nutzens etablieren - was die Begründung der Wohlfahrtsökonomie bedeutete. Diese wurde aber recht rasch als gescheitert betrachtet. Dazu aber später mehr, werfen wir zunächst einen Blick auf seine Ansätze zum ersten Punkt. Ausgangspunkt sind die Gleichgewichtstheorien in der Tradition von Walras (vgl. Kapitel \ref{Walras}). Darin wird davon ausgegangen, dass alle Märkte im Gleichgewicht sind. Damit wurde stets auch impliziert, dass die Märkte jeweils das \textit{optimale} Ergebnis liefern. \textcite{Pigou1920} zeigte nun systematisch Beispiele, in denen das nicht der Fall ist. Der Ausgangspunkt sind hier stets Externalitäten \parencite[S. 115]. Um diese systematisch zu analysieren führt er den Vergleich zwischen privaten und sozialen Grenzprodukten ein \parencite[S. 114]{Pigou1920}. Nur wenn die sozialen und privaten Grenzerträge übereinstimmen, ist die Marktlösung auch eine gesamtwirtschaftlich optimale Lösung. Weichen die sozialen Grenzerträge von den privaten ab, liegt ein externer Effekt vor. Der könnte zum Beispiel darin bestehen, dass der Produzent A bei der Herstellung seiner Güter Kosten auf die Allgemeinheit überträgt, für die er nicht aufkommen muss. Zum Beispiel könnte seine Fabrik das Umland verschmutzen ohne, dass er eine entsprechende Reinigungsgebühr entrichten muss. Diese sozialen Kosten trägt stattdessen die Bevölkerung. Als Resultat sind die privaten Kosten des Fabrikanten zu niedrig, was wiederum in einer zu hohen Produktion und zu niedrigen Preisen führt. Nur wenn der Staat eingreift und durch Steuern die sozialen Kosten wieder auf den Produzenten überträgt, ist das gesamtwirtschaftliche Ergebnis tatsächlich optimal. Damit begründete er die heute wieder so oft zitierte Pigou-Steuer. Überhaupt erlangte Pigou mit dem Aufkommen der Umweltproblematik eine enorme Bekanntheit in den letzten Jahren. Seine Arbeiten zu Marktversagen begründeten die mikroökonomische Wirtschaftspolitik. Allerdings muss man einschränkend sagen, dass in \textcite{Pigou1920} zwar bereits verschiedene Marktversagensformen wie natürliche Monopole \parencite[S. 240]{Pigou1920}, Informationsmängel \parencite[S. 131]{Pigou1920} und eben externe Effekte, allerdings werden die Probleme eher anhand von Beispielen und nicht systematisch-analytisch behandelt. Auch finden sich noch theoretische Fehler in seinen Abhandlungen, wie zum Beispiel der fehlende Unterschied zwischen technologischen und pekuniären negativen Effekten \parencite[S. 242]{Cansier1989}. Letztgenannte sind definiert als Kosten, die einem Unternehmen erwachsen, wenn ein Konkurrent ein identisches Gut zu besseren Preisen anbieten kann. In diesem Fall ist keine Korrektur durch staatliche Einriffe sinnvoll, weil die "Kosten" darin bestehen, dass die Gewinne eines Unternehmens reduziert werden, und eben keine gesamtwirtschaftlichen Kosten entstehen. Pigou war relativ stur diesen Fehler einzugestehen \parencite[S. 153]{Johnson1960}. Er betrachtete externe Effekte außerdem stets eindimensional. Schließlich gibt es grundsätzlich zwei verschiedene Möglichkeiten, wie externe Effekte internalisiert werden können. Entweder zahlt der Verursacher der Gesellschaft einen Ausgleich für den Schaden, oder aber die Gesellschaft zahlt umgekehrt dem Verursacher die Kosten zur Vermeidung des Schadens. Ronald Coase (vgl. Kapitel \ref{Neue Institut}) kritisierte Pigou später vehement für diese einseitige Darstellung \parencite[S. 243]{Cansier1989}.

Der zweite neuartige Punkt in \textcite{Pigou1920} erscheint zwar reizvoll, wurde aber fast umgehend formal widerlegt: Der Versuch Pigou's eine gesamtwirtschaftliche Nutzenbetrachtung durchzuführen. Pigou beschäftigte sich dabei zunächst ausführlich damit den Begriff des Volkseinkommens, bzw. des Sozialprodukts zu definieren. In weiterer Folge wollte eine Funktion für die gesamtwirtschaftliche Wohlfahrt ableiten. Diese sei abhängig von der Höhe, der Verteilung und der Stabilität des Sozialprodukts \parencite[S. 42]{Pigou1920}. Der Ansatz klingt natürlich zunächst vielversprechend: Wir wissen ja, dass das Vermögen alleine nur eingeschränkte Aussagen über die tatsächliche Wohlfahrt, also den erlebten Nutzen, zulässt. Es war \textit{die} zentrale Errungenschaft der Neoklassik das Wertparadoxon der Klassiker zu überwinden und den Nutzen in den Vordergrund von Maximierungsentscheidungen zu stellen. Verlockend ist es daher eine "gesamtwirtschaftliche" Wohlfahrtsfunktion anzustreben. Das Bruttoinlandsprodukt (Volkseinkommen) als aggregierte Wertschöpfung der Bevölkerung in einem Staat, sagt wenig über die Wohlfahrt aus. Man könnte stattdessen die einzelnen Vermögen der Individuen als Nutzenwerte ausdrücken und das Aggregat dieser Werte als "nationalen Gesamtnutzen" - oder eben Wohlfahrt - interpretieren. \textcite[S. 48]{Pigou1920} legte also die individuelle Grenznutzen-Theorie auf die Gesamtwirtschaft um: "Das Gesetz des abnehmenden Grenznutzens lehrt uns, dass ein steigendes Sozialprodukt nur zu einem unterproportionalen Wohlfahrtsgewinn führt." Aus der Annahme des abnehmenden Grenznutzens lässt sich laut \textcite[S. 53]{Pigou1920} außerdem ableiten, dass eine zusätzliche Geldeinheit einer armen Person mehr Nutzen stiftet, als einer reichen Person. Oder mit anderen Worten: Neben einem möglichst hohem Gesamt-Volkseinkommen erhöht auch eine möglichst gleiche Verteilung der Einkommen zu einer höheren gesamtwirtschaftlichen Wohlfahrt. Daher sollte der Staat auch dahingehend eingreifen und zum Beispiel durch progressive Besteuerung die Einkommenshöhe angleichen. Pigou missachtete dabei allerdings die wesentlichen Erkenntnisse seiner Vorgänger: Nutzen ist nicht kardinal, also in Geldeinheiten ausgedrückt, messbar. Stattdessen behilft man sich mit ordinalen Nutzenkonzepten wie in Kapitel \ref{Pareto} dargestellt. Es ist ein Irrtum zu glauben, Pigou wäre sich dessen nicht bewusst gewesen. Er wusste, dass interpersonelle Nutzenvergleiche formal-mathematisch nicht möglich sind. Aber er war auch pragmatisch genug um anzuerkennen, dass zum Beispiel bei unverändertem Sozialprodukt eine gleichmäßigere Einkommensverteilung gesamtwirtschaftlich einen positiven Wohlfahrtseffekt hat. Auch eine bewusst vereinfachend einheitliche Einkommens-Nutzenfunktion dachte Pigou zumindest an \parencite[S. 237]{Pigou1920}. Das Konzept von \textcite{Pareto1906} plädiert hingegen für eine die Unzulässigkeit von interpersonellen Nutzenvergleichen. Man kann eben gerade \textit{nicht} behaupten, dass eine arme Person durch eine zusätzliche Geldeinheit mehr Nutzen generiert als ein Millionär. Der Effekt von Umverteilungsmaßnahmen auf die Gesamtwohlfahrt ist damit in der Neoklassik in keinster Weise abbildbar. Stark kritisiert \parencite[S. 123]{Robbins1932} wurde Pigou für seine Idee der gesamtwirtschaftlichen Wohlfahrtsfunktion eben, weil die formale Unzulänglichkeit seines Konzepts schon seit dem Werk von \textcite{Pareto1906} bekannt war. Interessant ist der Ansatz von Pigou aus heutiger Sicht aber möglicherweise dennoch. Natürlich, formal-mathematisch ist es anerkannt, dass Nutzen nicht kardinal gemessen werden kann. Das stattdessen angewendete  Konzept von \textcite{Pareto1906}, bzw. der modernere Ansatz mittels Grenzrate der Substitution nach \textcite{Hicks1934b} ist eine elegante und in formaler Hinsicht höchst erfolgreiche Lösung. Diese führt aber eben auch dazu, dass Fragen der Einkommensverteilung in der Neoklassik einfach keinen Platz haben. Gerade solche Fragen sind in den letzten Jahren aber wieder vermehrt in den Vordergrund getreten.

Mit der Veröffentlichung von "Wealth and Welfare" und gerade einmal 43 Jahren war Pigou allerdings bereits auf den Zenit seiner Karriere angekommen. Denn sein eigener Assistent, ein uns bereits aus dem letzten Kapitel bekannter, gewisser John Maynard Keynes, revolutionierte kurze Zeit später die Ökonomie. Pigou nahm während dieser Revolution eine eher unglückliche Position ein. Er war einer jener Ökonomen, der an der fast magischen Strahlkraft von Keynes und dessen Werk fast zerbrach. Ähnlich ging es übrigens Joseph Schumpeter, Michal Kalecki und in gewissen Maße auch Friedrich Hayek. Ab den 1930er Jahren wendete Pigou sich nämlich von der Weiterentwicklung der "Wohlfahrtsökonomie" ab und wendete sich - getrieben durch die Arbeiten von Keynes - anderen Inhalten zu.

Pigou könnte als das genau Gegenteil von Keynes beschrieben werden. Nach erschütternden Ereignissen als Sanitäter im Ersten Weltkrieg, wurde er zum exzentrischen Einzelgänger \parencite[S. 153]{Johnson1960}. Der lockere Bloomsbury-Group-Lebemann Keynes trat schon diesbezüglich ganz anders auf. Auch wird Pigou als Berater als Vertreter veralteter Ideen beschrieben, der unter anderem die Wiedereinführung des Goldstandards empfahl \parencite[S. 232]{Cansier1989}. Eine Idee, die Keynes als "barbarisches Relikt" ansah und nach 1918 auch nicht mehr wirklich erfolgreich implementiert werden konnte. HIER WEITER \textcite[S. 232]{Cansier1989}

Unemployment wurde von Keynes kritisiert.
Überhaupt war Pigou "das Hauptopfer" von Keynes. Dieser reagierte mit dem Artikel \textcite{Pigou1936}


HIER WEITER




Beginnend mit Arthur Cecil Pigou entwickelte sich die Neoklassik in eine neue Richtung. Wie in Kapitel \ref{Neoklassik} dargestellt, kann die Neoklassische Theorie mit Marshall in gewisser Hinsicht als abgeschlossen gelten. Dies vor allem in dem Hinblick, dass sich die mikroökonomischen Ausgangstheorien seit damals tatsächlich kaum noch geändert haben. Selbst in modernen Mikroökonomie Büchern sind die ersten Kapitel im wesentlich identisch in den Arbeiten Marshall's zu finden. Die Mikroökonomie wurde seither nicht mehr grundlegend verändert, in dem Sinne, dass zuvor geltende Konzepte über den Haufen geworfen wurden, sondern sie wurde seither immer wieder erweitert. Marshall's Principles sind nach wie vor richtig. Aber man kann sie eben nicht mehr als Erklärung aller wirtschaftlichen Tätigkeiten heranziehen, sondern gelten mittlerweile nur mehr für einen sehr speziellen Fall. Heute wissen wir, dass die damals angenommenen Voraussetzungen, wie vollkommene Märkte und perfekte Konkurrenz nicht die Regel, sondern die Ausnahme sind.

Den ersten Schritt dieser Erweiterung der Neoklassik ging Arthur Cecil Pigou mit seinen 1920 erschienen "`Economics of Welfare"'. Er erweiterte die rein marktwirtschaftliche Analyse um Aspekte des Marktversagens und der gesamtwirtschaftlichen Wohlfahrt. Er brachte damit den Staat als wichtigen Player ins Spiel, ohne aber von den grundsätzlichen Ideen der Mikroökonomie abzugehen. Pigou 1920 stellt den Beginn der Mikroökonomischen Wirtschaftspolitik dar.


\section{Die Welt im Arrow-Debreu-Gleichgewicht}
\label{Arrow-Debreu}
Ursprünglichste Form: Walras.
Vorarbeiten von Neumann (1937, siehe oben) und Leontieff. Danach: \textit{Ramsey-Cass-Koopmans!}

Arrow-Debreu:
Das Arrow-Debreu Gleichgewichtsmodell (auch: Arrow-Debreu-McKenzie-Modell) ist ein mikroökonomisches Modell der gesamten Volkswirtschaft. Es ist nach Gérard Debreu und Kenneth Arrow sowie Lionel W. McKenzie benannt, stellt eine Weiterentwicklung des von Léon Walras entwickelten walrasianischen Gleichgewichtsmodells dar und untersucht einen gesamtwirtschaftlichen Gleichgewichtszustand. 
Das Modell erweitert das allgemeine Gleichgewichtsmodell um unsichere Erwartungen und zustandsabhängige Größen und ist damit für die Finanzierungstheorie von großer Bedeutung. Es zeigt, dass es in einer Marktwirtschaft unter idealisierenden Bedingungen nicht möglich ist, jemanden besserzustellen, ohne jemand anderen schlechterzustellen. Kurz gesagt ist ein Marktgleichgewicht ein Pareto-Optimum. 



\section{Die Cobb-Douglas-Produktionsfunktion} \label{sec: Cobb-Douglas-Produktionsfunktion}
Die Cobb-Douglas-Produktionsfunktion wurde bereits in den 1920er Jahren entwickelt. Ihre Bedeutung erlangte sie aber erst später unter anderem als Ausgangspunkt der neoklassischen Wachstumstheorie. Daher hier anzusiedeln.



\section{Solow: Technischer Fortschritt als Wachstumsquelle} \label{sec: Solow-Modell}


Harrod-Domar Modell (später Kontroverse Solow - Robinson)
Vorläufer: Von-Neumann Growth-Theory: 
"Über ein ökonomisches Gleichungssystem und eine Verallgemeinerung des Brouwerschen Fixpunktsatzes",  1937, in K. Menger, editor, Ergebnisse eines mathematischen Kolloquiums, 1935-36. [English 1945 trans. as "A Model of General Economic Equilibrium", RES].

Ausgangspunkt ist eine Produktionsfunktion, wie jene, die wir gerade in Kapitel \ref{sec: Cobb-Douglas-Produktionsfunktion} kennen gelernt haben.  Der Output – gesamtwirtschaftlich das BIP – wird durch verschiedene Inputs – standardmäßig in der Neoklassik Arbeit und Kapital – hervorgebracht. Der Output ergibt sich also aus eine Kombination der beiden Inputfaktoren, Ökonomen würden sagen: „Der Output ist eine Funktion der Inputfaktoren“. 
Folgende Überlegung macht recht schnell klar, warum man mit diesem einfachen Modell stetige Wachstumsraten nicht erklären kann: Angenommen der Output ist einfach eine Addition der beiden Inputfaktoren Arbeit und Kapital. Möchte ich den Output verdoppeln, so müsste ich \textit{beide} Inputfaktoren Arbeit und Kapital jeweils verdoppeln. Ökonomen sprechen hier von konstanten Skalenerträgen. 
Was passiert aber wenn nur einer der beiden Input-Faktoren steigen kann? Gesamtwirtschaftlich könnte man argumentieren, dass der Produktionsfaktor Arbeit durch die Bevölkerungszahl begrenzt ist. Wenn man dies in unserer Überlegung berücksichtigt, würden wir folglich nicht beide Inputfaktoren gleichzeitig erhöhen, sondern nur einen, nämlich Kapital. Erhöhen wir diesen Inputfaktor nun um eine Einheit und der andere Inputfaktor bleibt gleich, so steigt der Gesamtoutput zwar selbstverständlich an, allerdings pro zusätzlicher Einheit um einen immer geringeren Prozentsatz. Intuitiv ist das leicht verständlich: Erhöhe ich bei gleichbleibender Mitarbeiterzahl ständig das Kapital – zum Beispiel die Anzahl der Computer – dann wird der erste eingesetzte Computer einen hohen Zuwachs an Produktivität bringen. Mit jedem weiteren Computer wird die Produktivität zwar weiter steigen, allerdings mit immer geringerer Zuwachsrate. Wenn jeder Mitarbeiter mehr als einen Computer besitzt, wird der Produktivitätszuwachs verschwindend gering werden. Diesen Zusammenhang bezeichnen Ökonomen als „Abnehmenden Grenzertrag“.
Das würde aber bedeuten, dass bei ungefähr gleich bleibender Arbeitsbevölkerung – eine Annahme, die man zumindest für die mittlere Frist in Industriestaaten, bedenkenlos machen kann – die Wirtschaftsleistung stagnieren sollte. Geht man davon aus, dass der Kapitaleinsatz ständig steigt, wäre zwar stetiges Wachstum möglich, aber nur mit immer geringer werdenden Wachstumsraten \footnote{Ständig steigender Kapitaleinsatz wäre nur mit steigenden Sparquoten erklärbar. In einer Ökonomie ohne Außenhandel gilt ja, dass das Sparen der Haushalten den Investitionen der Firmen entspricht. Investitionen wiederum bedeuten einen Aufbau von Kapital. Spezielle Beobachtungen von Fällen von Wirtschaftswachstum werden tatsächlich darauf zurückgeführt, dass die Sparquoten gestiegen sind. So ist zum Beispiel die Ökonomie in der stalinistischen Sowjetunion tatsächlich beträchtlich gewachsen. Da man aber keine wesentlichen technologischen Vorsprünge des Landes in dieser Zeit ausmachen kann, vermutet man, dieses Wachstum sei eben alleine auf den Anstieg der Sparquote zurückzuführen.}. Langfristig würden die Zuwachsraten aber gegen Null tendieren, womit auch in diesem Fall die Wirtschaftsleistung stagniert.

Bisher haben wir aber eine Möglichkeit außer Acht gelassen: Nämlich, dass die eingesetzten Maschinen (hier als Synonym für Kapital verwendet) immer besser werden. Tatsächlich wird eine Arbeitskraft mit zwei Computern nicht wesentlich produktiver sein, als mit einem Computer. Sie könnte aber mit einem \textit{besseren} Computer wesentlich produktiver sein. Es könnte sich also nicht nur die \textit{Menge} des Kapitals verändern, sondern auch dessen \textit{Qualität}. Dies nennen wir „technischen Fortschritt“ \footnote{Technischer Fortschritt umfasst nicht nur die Weiterentwicklung bestehender Produkte zu „besseren“ Produkten, sondern auch die Einführung neuer Produkte}.
Berücksichtigt man diesen Umstand, kommt man zu dem Ergebnis, dass stetiges Wirtschaftswachstum nur dann möglich ist, wenn sich die eingesetzten Kapitalgüter – also zum Beispiel Maschinen, Computer, Transportmittel, Kommunikationsmittel – immer weiter verbessern.
Der Inhalt des „technischen Fortschritts“, also was sich wie verbessert – ist allerdings ist nicht Teil der Ökonomie. Die ersten Wachstumstheorien haben sich also damit abgefunden festzustellen, dass technischer Fortschritt für Wachstum notwendig ist, dieser selbst allerdings nicht durch ökonomisches Handeln beeinflusst werden kann. Der technische Fortschritt wurde also als „exogen“ betrachtet. Daher der Name „exogene Wachstumstheorie“.
















