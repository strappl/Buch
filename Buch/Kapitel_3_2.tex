%%%%%%%%%%%%%%%%%%%%% chapter.tex %%%%%%%%%%%%%%%%%%%%%%%%%%%%%%%%%
%
% sample chapter
%
% Use this file as a template for your own input.
%
%%%%%%%%%%%%%%%%%%%%%%%% Springer-Verlag %%%%%%%%%%%%%%%%%%%%%%%%%%

\chapter{Die Verschmelzung: Ein bisschen Neoklassik, ein bisschen Keynes}
\label{Synthese}

Die sogenannte Neoklassische Synthese war vielleicht die prägendste ökonomische Schule des 20. Jahrhunderts. Wie bereits im letzten Kapitel erwähnt wird diese Schule umgangssprachlich oft einfach als Keynesianismus bezeichnet. Die Vertreter der Schule bezeichneten sich ebenso meist schlicht als Keynesianer. Dies ist aber unscharf und verwirrend. Der Begriff Neoklassische Synthese ist zwar etwas umständlich aber sehr gut passend, wie wir auch anhand von Beispielen sehen werden. Sie galt bis in die 1970er Jahre hinein relativ unumstritten als \textit{die} makroökonomische Mainstream-Theorie schlechthin. Diese Zeit zwischen 1950 und 1970 kann in den westlichen Industriestaaten als goldenes wirtschaftliches Zeitalter gesehen werden. Die westeuropäischen Staaten und Japan schafften aus den Trümmern des Zweiten Weltkrieges heraus einen erstaunlichen Wiederaufbau und schlossen wirtschaftlich rasch zur unumstrittenen kapitalistischen Hegemonialmacht, den USA, auf. Die beachtlichen technischen Fortschritte in Kombination mit der erfolgreichen wirtschaftlichen Entwicklung von Europa, Japan und den USA, prägte die Welt nachhaltig bis zum Ende des 20. Jahrhunderts.

Nach der Veröffentlichung der "`General Theory"' durch Keynes im Jahr 1936 begann unmittelbar deren Siegeszug. Während des Zweiten Weltkrieges spielte ökonomische Theorie verständlicher Weise eine geringe Rolle. Danach kam es zu einer doch überraschenden Entwicklung in der Volkswirtschaftslehre. Die Österreichische Schule, die vor dem Zweiten Weltkrieg, noch als eine dem Keynesianismus ebenbürtige makroökonomische Theorie galt, verlor enorm an Bedeutung und galt rasch nur mehr als Außenseiter-Theorie. Dafür näherte sich die junge Generation mikroökonomischer Neoklassiker den makroökonomischen Keynesianern an. Samuelson schrieb in seinem berühmten Lehrbuch sinngemäß: Nach 1936 teilten sich die Ökonomen auf in eine Gruppe von Keynesianern und ein Gruppe von Ökonomen, die dem Keynesianismus kritisch gegenüberstanden. Nach dem Zweiten Weltkrieg haben sich die jungen Neoklassiker mit den Keynesianern arrangiert und ihre Theorien miteinander verbunden. Das Resultat nennen wir heute die Neoklassische Synthese \parencite[S. 46]{DeVroey2016}. Die überwiegende Anzahl der bedeutenden Ökonomen sah sich Ende der 1940er Jahre als Teil dieser Bewegung. Zunächst nur die älteren Generationen konnten damit wenig anfangen, diese verloren aber zunehmend an Bedeutung und rutschten in die Heterodoxie ab. Neben der Österreichischen Schule gilt dies insbesondere auch für die Ökonomen im englischen Cambridge, also der Wirkungsstätte von Keynes selbst. Das Zentrum der Mainstream-Ökonomie war nun vollends über den Atlantik in die USA gewandert. Konkret in den Raum Boston/Massachusetts. Die junge Generation am Massachusetts Institute of Technology (MIT) und im benachbarten Harvard prägten die Ökonomie nach 1945 weltweit. Der Name Neoklassische Synthese ist hierfür extrem passend. Denn die führenden Ökonomen dieser Zeit waren zwar inhaltlich überzeugte Anhänger des Keynesianismus, arbeiteten aber selbst teilweise in der Tradition der neoklassischen Schule. Paradebeispiele hierfür ist zum Beispiel Robert Solow, der sich bis heute als Keynesianer bezeichnet, aber für die \textit{neoklassische} Wachstumstheorie (vgl. Kapitel \ref{sec: Wachstum}) bekannt wurde, die die keynesianische Wachstumstheorie verdrängte. Auch Franco Modigliani und James Tobin, die keynesianische Modelle entwickelten, aber später vor allem als wichtige Vertreter der Neoklassischen Finanzierungstheorie bekannt wurden, waren wichtige Vertreter der Neoklassischen Synthese.

Als inhaltlicher Begründer der Neoklassischen Synthese kann der Brite John Hicks gesehen werden, der bereits 1937 Keynes' General Theory in einem Zeitschriftenartikel formalisierte und als Gleichgewichtstheorie darstellte, dabei aber um wesentliche Elemente beraubte. Der am MIT tätige Paul Samuelson, sowie der in Harvard lehrende Alvin Hansen griffen die Arbeit von Hicks nach dem Zweiten Weltkrieg auf ,entwickelten sie gemeinsam mit den oben genannten Ökonomen weiter und hatten vor allem entscheidenden Anteil an der Popularisierung und Verbreitung der Neoklassischen Synthese in den USA und in weiterer Folge in der ganzen Welt.

\section{Die Neoklassische Synthese}

Bereits im April 1937, also nur wenige Monate nach der Veröffentlichung der General Theory, erschien der  Artikel "`Mr. Keynes and the "Classic"; A Suggested Interpretation"' von John R. Hicks \parencite{Hicks1937}. Der nur zwölfseitige Artikel entwickelte aus der doch recht umständlich verfassten General Theory heraus ein Modell, das wesentliche Elemente von Keynes in wenige Gleichungen zusammenfasste und in einem Diagramm recht einfach und anschaulich darstellte. Das Resultat war das berühmte IS-LM-Modell, das noch heute in Volkswirtschafts-Einführungs-Lehrveranstaltungen gelehrt wird. Die wirtschaftstheoretisch  bedeutsamere Arbeit von Hicks in Bezug auf die Synthese von mikroökonomischer Neoklassik und makroökonomischem Keynesianismus, war vielleicht "`Value and Capital"' \parencite{Hicks1939}. Darin greift er die damals in der Mikroökonomie stark beforschte Allgemeine Gleichgewichtstheorie auf (vgl. Kapitel \ref{Arrow-Debreu}) und führte diese in eine makroökonomische, keynesianische Theorie über. Sein Fokus war natürlich ganz ein anderer als jener der Mikroökonomen. Dort hatte der Mathematiker John von Neumann ein hoch-mathematisches System aus Ungleichungen entwickelt um die theoretische Existenz eines Gleichgewichts beweisen zu können. Hicks' Ansatz kann als makroökonomische Abspaltung der "`Allgemeinen Gleichgewichtstheorie"' interpretiert werden, die später in keynesianischer Tradition vor allem von Paul \textcite{Samuelson1947} fortgeführt wurde. \textcite{Hicks1939} wendete aber einen anderen Gleichgewichtsansatz an, der die Problematik des Beweises der Existenz eines Allgemeinen Gleichgewichts zur Seite schob \parencite[S. 21]{Weintraub1983}. Das Verhältnis zwischen neoklassischen Allgemeinen Gleichgewichtsmodellen, wie jenem von \textcite{Neumann1937} und später von \textcite{Arrow1954} und \textcite{McKenzie1954} auf der einen Seite und den Modellen der Neoklassische Synthese eben von \textcite{Hicks1939} auf der anderen Seite, wurde zu Beginn heftig diskutiert. \textcite{Morgenstern1941} kritisierte die mathematische Unzulänglichkeit des Letztgenannten vehement. Wirtschaftspoltisch setzten sich die keynesianischen Gleichgewichtsmodelle sicherlich als die praktisch bedeutenderen durch. 

Kommen wir aber hier zum IS-LM-Modell selbst und damit zu \textcite{Hicks1937} und zu dessen Inhalt zurück. Hicks beschreibt die unkonventionelle Herangehensweise in der General Theory und verfolgt danach einen praktischen Ansatz: Zunächst stellt er fest, dass es sich um eine Theorie der kurzen Frist handeln muss, weil Kapitalstock und Arbeitsangebot als fix angesetzt werden. Danach bricht er die "`alte"' (neo-)klassische Theorie in dieselbe Form herunter, in der Keynes seine Theorie darstellte \parencite[S. 148]{Hicks1937}. Daraus formuliert er drei Gleichungen des (neo-)klassischen Modells:
Erstens: $M = kY$. Hier ist die Quantitätsgleichung des Geldes abgebildet. Mit steigendem Gesamteinkommen $Y (=BIP)$ steigt die Nachfrage nach Geld $M$. Der Faktor $k$ zeigt in welchem Verhältnis dies passiert. Dieser wird als konstant oder exogen angenommen.
Zweitens: $I = C(i)$. Investitionen $I$ hängen vom Zinssatz $i$ ab. Je höher der Wille zu investieren $C(i)$, desto höher darf der Zinssatz liegen bei dem Investitionen getätigt werden.
Drittens: $I = S(r,Y)$. Bekanntermaßen gilt bei den (Neo-)Klassikern investieren $I$ entspricht sparen $S$. Dabei gilt je höher das Einkommen und je höher die Zinsen, desto mehr wird gespart.
Bei den (Neo-)Klassikern gibt es keinen Multiplikatoreffekt. Dennoch gilt, dass sich das BIP als Summe aus Konsum und Investitionen ergibt $(Y = C + I)$. Dabei macht es keinen Unterschied, ob der Konsumgüterkreislauf oder die Produktion für die Beschäftigung sorgen. Allerdings sorgen steigende Reallöhne für Arbeitslosigkeit. Wenn diese zu hoch sind, kommt es auch in der (neo-)klassischen Theorie zu keiner vollständigen Markträumung und damit zu Arbeitslosigkeit. Das grundsätzliche Problem rigider Löhne sieht Hicks also auch im neo(-klassischem) System durchaus abgebildet. Allerdings verschwindet Unterbeschäftigung dort durch natürlichen Inflationsdruck oder sinkende Preise wieder. \textcite[S. 150ff]{Hicks1937} hält dann fest, dass in diesem (neo-)klassischen System Konjunkturschwankungen über Schwankungen der Geldmenge, oder über Schwankungen der exogenen Variable $k$ erklärt werden können. Beide Erklärungstheorien seien aber wenig überzeugend. 

Er betrachtet daher in weiterer Folge die keynesianischen Äquivalente der drei eben gezeigten Gleichungen:
Erstens: $M = L(i)$. Die Geldnachfrage hängt über die Liquiditätspräferenz mit dem Zinssatz zusammen. Die Geldmenge bestimmt bei Keynes den Zinssatz und nicht wie bei den (Neo-)Klassikern das BIP.
Zweitens: $I = C(i)$ Die Investitionen hängen bei Keynes von der Grenzleistungsfähigkeit des Kapital, also der erwartete Rendite der Investitionen, ab. Je höher der Zinssatz, desto höher muss die erwartete Rendite einer Investition sein, damit sie lukrativ ist.
Drittens: $I = S(Y)$ Auch in der General Theory gilt die Identität aus sparen und investieren. Allerdings wird die Sparquote vom Nationaleinkommen (BIP) bestimmt.

Aus diesen drei Gleichungen leitet \textcite{Hicks1937} schließlich zwei Funktionen ab, die er in ein Diagramm, das auf der y-Achse den Zinssatz $i$ abbildet und auf der x-Achse das Nationaleinkommen $Y$, einbettet. Im Schnittpunkt der beiden Kurven befindet sich die Ökonomie im Gleichgewicht. Erwähnt werden muss in diesem Zusammenhang allerdings, dass das IS-LM-Modell davon ausgeht, dass in diesem Gleichgewicht  ein Zustand von Unterbeschäftigung herrscht\footnote{Außer im Spezialfall ein durchgehend vertikalen LM-Kurve. Dann wäre nämlich das neo(-klassische) Modell perfekt anwendbar, wodurch das IS-LM-Modell aber hinfällig werden würde.}. Optisch gleicht das Ergebnis jetzt dem berühmten Marshall'schen Kreuz (vgl. Kapitel \ref{Neoklassik}), wie \textcite[S. 153]{Hicks1937} feststellt.

Aus den beiden letztgenannten Gleichungen leitet \textcite{Hicks1937} die Investition-Sparen-Kurve (IS-Kurve) ab. Das sogenannte Gütermarkt-Gleichgewicht. Das BIP wird im Keynesianismus vollkommen von der Nachfrage bestimmt, die sich aus Konsum, Investitionen und exogener, öffentlicher Nachfrage zusammensetzt. Das IS-LM-Modell nach Hicks betrachtete ausschließlich geschlossene Ökonomien, Außenhandel bleibt also unberücksichtigt. Der Gütermarkt ist im Gleichgewicht, wenn Investitionen und Sparen gleich hoch sind. Die IS-Kurve ist dann jenes Verhältnis aus BIP und Zinssatz bei dem diese Identität gilt. Der Multiplikator hebelt die Konsumausgaben. Je höher die Konsumneigung, desto höher der Multiplikator, desto höher das Einkommen. Die Konsumneigung $c$ bestimmt aber auch indirekt die Sparquote $1-c$. Je höher der Zinssatz, desto höher ist die Sparquote $1-c$. Nehmen wir also einen hohen (niedrigen) Zinssatz an, dann haben wir eine hohe (niedrige) Sparquote und dafür eine niedrige (hohe) Konsumneigung - was einen niedrigen (hohen) Multiplikator ergibt - und daher ein niedriges (hohes) BIP. Daraus ergibt sich die negative Steigung der IS-Kurve. Dies ist konsistent mit den Überlegungen zu Investitionen: Bei hohen Zinsen sind Investitionen unattraktiv. Sinkende Zinsen machen Investitionen attraktiver. Je steiler die IS-Kurve, desto höher ist der Multiplikator und desto stärker müssen die Zinsen sinken, damit Investitionen getätigt werden. Die Überlegungen wurden sehr ähnlich bereits von Keynes durchgeführt (vgl. Kapitel \ref{Keynes}), aber eben nicht in Form von Gleichungen dargestellt. Die Gütermarkt-Funktion bildet praktisch den Realmarkt im IS-LM-Model ab. 

Die zweite Funktion bildet das Geldmarkt-Gleichgewicht ab. \textcite{Hicks1937} erstellt dafür die LL-Kurve. Bereits seit dem Lehrbuch \textcite{Samuelson1998} spricht man allerdings allgemein vom Gleichgewicht aus Liquiditätspräferenz und Geldangebot (Money Supply), also von der LM-Kurve. Das Geldangebot wird im IS-LM-Modell als exogen von der Zentralbank vorgegeben angenommen\footnote{Das wurde von den Post-Keynesianern (vgl. Kapitel \ref{Post-Keynes}) in Frage gestellt und gilt heute als widerlegt und wird immer wieder von Zentralbankern selbst vehement bestritten.}. Bei höherem BIP steigt die Geldnachfrage. Bei konstantem Geldangebot und steigender Geldnachfrage, müssen die Zinsen daher steigen damit das Gleichgewicht aufrecht bleibt. Daraus ergibt sich die positive Steigung der LM-Kurve. Diese Herleitung ist aber etwas zu einfach dargestellt. Tatsächlich bestimmt bei \textcite{Keynes1936} die Liquiditätspräferenz die Nachfrage nach Geld und diese setzt sich aus verschiedenen Komponenten zusammen (vgl. Kapitel \ref{Keynes}), unter anderem aus dem Spekulationsmotiv und aus dem Vorsichtsmotiv. Bei höherem Einkommen wollen Individuen mehr Bargeld halten und verkaufen Wertpapiere, dadurch sinkt deren Preis und deren Zinsen steigen\footnote{Deren Zinsen steigen, weil Wertpapiere hier immer im Sinne von Anleihen am Ende immer mit 100 getilgt werden und bei niedrigerem Preis die Differenz zu 100 und damit die Rendite, also der Zins, steigt.}. Ein steigender Zinssatz senkt grundsätzlich also die Liquiditätspräferenz und damit die Nachfrage nach Geld. Für ein Gleichgewicht am Geldmarkt muss bei konstant angenommenem Geldangebot muss der Zinssatz genau die Höhe haben bei der auf den Finanzmärkten kein Anreiz zu Verkäufen besteht und auf dem Geldmarkt kein Anreiz mehr Geld zu halten. Insgesamt muss bei gegebenem Geldangebot der Zinssatz steigen, damit die Geldnachfrage mit dem Geldangebot im Gleichgewicht bleibt.
Interessant ist, dass \textcite[S. 154]{Hicks1937} die LM-Kurve praktisch in drei Teile zerlegte. Insgesamt bildet die LM-Kurve über das gesamte Einkommen-Zinssatz-Diagramm eine konvexe Funktion ab. Ganz links verläuft sie sehr flach, fast horizontal. Ganz rechts hingegen vertikal. Der Grund für den horizontalen Teil liegt in der natürlichen Zinsuntergrenze von 0\%. Bereits \textcite{Keynes1936} hatte diesen Sonderfall (vgl. Kapitel \ref{Keynes}) betrachtet. Heute wird dieser Bereich als Liquiditätsfalle bezeichnet, ein Zustand, der für die Zeit nach der "`Great Recession"', als niedrige Inflation trotz Geldschwemme vorherrschten für viele westliche Industriestaaten immer wieder diskutiert wurde. \textcite[S. 154]{Hicks1937} nannte diesen Bereich den "`Spezialfall der keynesianischen Theorie"'. In diesem Bereich ist Geldpolitik vollkommen wirkungslos, Fiskalpolitik hingegen hochwirksam, weil durch staatliche Ausgaben der Zinssatz nicht/kaum erhöht wird und es so zu keiner Verdrängung privater Investitionen durch staatlichen Investitionen kommt. Schneidet die Gütermarktkurve (IS-Kurve) die LM Kurve hingegen ganz rechts in deren vertikalen Bereich, gilt die (neo-)klassische Theorie perfekt. Zusätzliche (staatliche) Investitionen führen ausschließlich zu einem höheren Zinssatz, nicht aber zu höherem Einkommen. Dies würde zum Beispiel im Falle von Vollbeschäftigung eintreten.

Für \textcite{Hicks1937} war die General Theory aus Sicht des IS-LM-Modells damit eigentlich eine "`Special Theory"' einer ansonsten (neo-)klassischen Welt. Die General Theory of Employment ist demnach eine Theorie für Krisenzeiten \parencite[S. 155]{Hicks1937}, konkret für den Fall, dass Geldpolitik jegliche Wirksamkeit verloren hat. Im Falle einer Unterauslastung der Wirtschaft, bei gleichzeitig aber ansteigend verlaufender LM-Kurve ist das klassische Rezept einer Geldmengenausweitung im Falle von Krisen ausreichend. Dadurch steigt das Preisniveau, die Reallöhne sinken wieder und die Beschäftigung steigt \parencite[S. 30]{DeVroey2016}.

HIER WEITER Snowden Buch Seite 106. Die Wirtschaftspolitischen Implikationen und Geldpolitik und Fiskalpolitik
Auch die Rolle von Modigliani noch beschreiben
Der Erfolg von Hicks IS-LM-Modell war überwältigend. Staatliche Ausgaben sind als exogen angenommen. Eine Erhöhung der Staatsausgaben führt in dem Modell zu einer Rechtsverschiebung der IS-Kurve und damit zu einer Erhöhung des BIPs. Die daraus eventuell entstehenden Schulden sind im IS-LM-Modell nicht abgebildet, schließlich handelt es sich um eine reine Analyse der kurzen Frist.


Natürlich auch dadurch bedingt, dass 
Interessant ist in diesem Zusammenhang aber auch der offensichtliche "`Missbrauch"' der Arbeit von Hicks. So spricht er explizit \parencite[S. 154]{Hicks1937} davon, dass keynesianische Expansionspolitik dann sinnvoll ist, wenn Anreize durch Geldpolitik nicht mehr wirken. Die Wirtschaftspolitik versuchte stattdessen eine "`Feinsteuerung"' der Ökonomie mit diesem Instrument. Außerdem beschreibt er sinngemäß, dass sein Modell nur einen kleinen Teilbereich der makroökonomischen Diskussion abbilden kann \parencite[S. 158]{Hicks1937}. In der Realität war das IS-LM-Modell der Ausgangspunkt für die Einschränkung von Keynes' Werk auf einige Teilbereiche seiner Arbeit.


HIER WEITER: \parencite{Patinkin1990}
Keynes selbst kritisierte die fehlende Bedeutung der "`Effektiven Nachfrage"' im IS-LM-Modell.
Was von Keynes fehlt im IS-LM-Modell? (Verweis auf Post-Keynesianer) Unsicherheiten bzgl. zukünftiger Entwicklungen, die sogenannten Animal Spirits als Ad-hoc Annahme im Fall der Investitionsfalle. Außerdem wird akzeptiert, dass die primäre Quelle der Unterbeschäftigung rigide Löhne sind. Dies ist grundsätzlich nichts typisch Keynesianisches. Lohn ist ja nichts anderes als der Preis für Arbeit. Wenn dieser zu hoch ist, kommt es auch in der (Neo-)Klassik zu keiner vollständigen Markträumung.


Hicks selbst war später verwundert über den Erfolg des Artikels und sogar etwas verärgert über dessen Verwendung als "`classroom gadget"'. 


Hicks  übernahm diese Formalisierung und verband einen Teil von Keynes' Theorie mit neoklassischen Elementen zum \textit{IS-LM-Modell}. Dieses Modell stellt noch heute den finalen Punkt in vielen Einführungslehrveranstaltungen zu Makroökonomie dar.

Diese Formalisierung durch Hicks enthält aber zwei extrem wichtige Punkte:
\begin{itemize}
	\item "`Er übernahm einen \textit{Teil} von Keynes' Theorie"': Bei dieser Formalisierung gingen im Gegenzug viele Teile der General Theory verloren. Eine Tatsache, die bis heute in der Mainstream-Ökonomie hingenommen wird. Diese "`verlorenen Teile"' sollten später von den \textit{Post-Keynesianern} aufgegriffen werden.
	\item "`Er verband diesen Teil mit neoklassischen Elementen"': Wenn wir lesen, dass die 1950er und 1960er Jahre die Hochzeit des "`Keynesianismus"' waren, dann meinen Ökonomen eigentlich, dass das "`alte"' Neoklassische Wissen herangezogen wurde und um "`Keynesianische"' Elemente erweitert wurde. Es entstand also eine "`Synthese"' aus zwei Wissensgebieten, folglich wird das Ganze unter Ökonomen die \textsc{Neoklassische Synthese} genannt.
\end{itemize}






Das IS-LM-Modell kann auch als "`Allgemeines Gleichgewichtstheorem"' betrachtet werden:
Hicks, J. R., Value and Capital. Oxford: Clarendon Press. 1939.
Samuelson, P. A., Foundations of Economic Analysis. Cambridge, Mass. Harvard University Press. 1947.

\ref{Arrow-Debreu}




Mundell-Fleming-Modell


Modigliani

Franco Modigliani belegt, die diesen wie folgt um-schreibt1): „Non-monetarists accept what 1 regard to be the fundamental practical message of the General Theory: that a private enterprise economy using an intangible money needs to be stabilized, can be stabilized and therefore should be stabilized by appropriate monetary and fiscal policies. Mone-tarists by contrast take the view that there is no serious need to stabilize the economy; that even if there were a need, it could not be done, for stabilizing policies would be more likely
Mone-tarists by contrast take the view that there is no serious need to stabilize the economy; that even if there were a need, it could not be done, for stabilizing policies would be more likely to increase than to decrease instability; and, at least some monetarists would, 1 believe, go so far as to hold that, even in the unlikely event that stabilization policies could on balance prove beneficial, the government should not be trusted with the necessary power."

1) American Economic Review, März 1977.

Patinkin, Alvin Hansen, Solow, Tobin, Okun

Tobin: war skeptisch gegenüber Neu-Keynesianern. \textcite[S. 398]{Snowdon2005}

Werke angeführt: \parencite[S. 57]{Snowdon2005}


\section{Die Phillipskurve: Tausche Inflation gegen die Arbeitslosigkeit} \label{sec: Phillips}

Einen entscheidenden Beitrag zur "`Neoklassischen Synthese"' lieferte die Phillips-Kurve. Diese hat bis heute eine entscheidende Bedeutung in der Makroökonomie, wenn auch in der modernen Form in ganz anderer Art und Weise. Der Phillips-Kurven-Zusammenhang, in welcher Form er auch immer dargestellt wird, ein zentraler Diskussionspunkt der Makroökonomie seit den späten 1960er Jahren. Dementsprechend wird uns der Begriff in den Folgekapitel häufig unterkommen. Die ursprüngliche Form geht auf den an der London School of Economics (LSE) tätigen neuseeländischen Ökonomen Alban William Phillips zurück. Dieser veröffentlichte 1958 einen Artikel, in dem er einen (nicht-linearen) negativen Zusammenhang zwischen der Änderungsrate der Löhne (Lohninflation) und Arbeitslosigkeit postulierte. Dieser grundsätzliche Zusammenhang klingt zunächst banal: Geringe Arbeitslosigkeit bedeutet, geringes Arbeitskräfte-Angebot. Bei konstanter Arbeitskräfte-Nachfrage wird man daher davon ausgehen, dass geringer Arbeitslosigkeit der Preis für Arbeit hoch ist und bei hoher Arbeitslosigkeit niedrig. Eine historische Besonderheit in dieser Arbeit \parencite{Phillips1958} ist, dass die Aufstellung einer Hypothese im selben Werk mit empirischen Daten untersucht wurde. In weiterer Folge war die dahinterliegende Theorie weniger interessant, als die Tatsache, dass der empirisch festgestellte, negative Zusammenhang zwischen Arbeitslosigkeit und Lohnänderungsraten relativ unumstritten als stabil angesehen wurde. \textcite[S. 186]{Samuelson1960} bemerkten schon: "`Seine [Phillip's] Ergebnisse sind bemerkenswert, auch dann, wenn man seine Interpretationen dazu nicht teilt"'. Die damals führenden Mainstream-Makroökonomen, Paul Samuelson und Robert Solow griffen diesen Zusammenhang auf und erweiterten ihn zur berühmten Phillips-Kurve. Auf dieser wird der negative Zusammenhang zwischen Arbeitslosigkeit und Preisänderungsraten (also Inflation im allgemeinen) abgebildet. Nicht Phillips selbst, sondern die Keynesianer in den USA etablierten diese Theorie also in den 1960er Jahren als fixen Bestandteil der damaligen Mainstream-Ökonomie. Phillips selbst war skeptisch gegenüber der Möglichkeit durch das Tolerieren von Inflation die Arbeitslosigkeit zu senken \parencite[S. 142]{Snowdon2005}. Aufgefallen ist vielleicht die späte Jahreszahl. Das IS-LM-Modell, als zentraler Baustein der Neoklassischen Synthese wurde bereits 1937 verfasst und erlangte unmittelbare Berühmtheit. Der Phillips-Kurven-Zusammenhang hingegen, aus heutiger Sicht ebenso ein wichtiger Baustein der Neoklassischen Synthese, wurde erst nach 1960 in der Theorie etabliert.
\textcite[S. 192]{Samuelson1960} leiteten aus dem Phillips-Kurven-Zusammenhang ab, dass man für Preisstabilität den Preis von 5-6\% Arbeitslosenquote bezahlen müsse, oder aber 4-5\% Inflation akzeptiere und dafür nur 3\% Arbeitslose habe. Der \textcite{Samuelson1960}-Journalbeitrag interpretierten die ursprüngliche Arbeit von \textcite{Phillips1958} in zweifacher Hinsicht um: Erstens, Lohninflation wurde mit Preisinflation gleichgesetzt. Heute mein man üblicherweise den negativen Zusammenhang zwischen \textit{Preis}-Inflation und Arbeitslosigkeit. Zweitens stellten die \textcite{Samuelson1960} den Zusammenhang als langfristig fix dar, sodass sich Entscheidungsträger praktisch niedrige Arbeitslosigkeit für den Preis hoher Inflation kaufen könnten.

Die Keynesianer nahmen den Inflations-Arbeitslosigkeits-Zusammenhang jedenfalls recht dankbar auf, erlaubte er doch eine erweiterte Interpretation des IS-LM-Modells. In diesem fehlt nämlich die Betrachtung der Inflation. Im ursprünglichen IS-LM-Modell herrscht eine Unterbeschäftigung vor und Preise werden als stabil angenommen. Fiskalpolitik erhöht den realen Output, bis der vertikale Bereich der LM-Kurve erreicht wird. Dann herrscht Vollbeschäftigung und jedwede zusätzliche fiskalpolitische Maßnahme führt dann zu Inflation. Die Phillips-Kurve ermöglichte nun einen Zusammenhang zwischen der IS-LM-Interpretation betreffend Output und Beschäftigung auf der einen Seite und Inflation auf der anderen Seite herzustellen. Der Output hängt dabei positiv von der Beschäftigung ab und Preisinflation ergibt sich direkt aus der Lohninflation (abzüglich des Produktivitätszuwachses), also so wie von \textcite{Samuelson1960} angenommen. Dies führt dazu, dass im IS-LM-Modell eine vertikale LM-Kurve nicht mehr Voraussetzung ist für Inflationssteigerung durch Fiskalpolitische Maßnahmen. Bei Vollbeschäftigung und normaler (also ansteigender aber nicht vertikaler) LM-Kurve führt Fiskalpolitik zu einer Verschiebung der IS-Kurve nach Links auf ein höheres Output-Niveau. Der Output verschiebt sich auf ein Niveau über der Vollbeschäftigung, was zu steigenden Preisen führt. Diese führen nun aber dazu, dass der reale Wert des Geldangebotes fällt, was die LM-Kurve nach links verschiebt. Damit ist wieder das alte Vollbeschäftigungs-Gleichgewicht hergestellt, allerdings bei höherem Zinssatz und die Inflation ist im Modell damit abgebildet \parencite[S. 143f]{Snowdon2005}.

Die Erweiterung des IS-LM-Modells um den Phillips-Kurven-Zusammenhang löst also das Problem fehlender Inflationsberücksichtigung. Bereits gegen Ende der 1960er-Jahre war dieser Erklärungsansatz aber obsolet. Die ersten Anzeichen von Stagflation, also das gleichzeitige Auftreten höher Inflationswerte und höher Arbeitslosenquoten, könnten von den Vertretern der Neoklassischen Synthese nicht befriedigend erklärt werden. Die bahnbrechenden Arbeiten von \textcite{Phelps1968} (vgl. Kapitel \ref{cha: Neu Keynes}) und \textcite{Friedman1968} (vgl. Kapitel \ref{Monetarismus}) wider den hier vorgestellten "`originalen"' Phillips-Kurven-Zusammenhang und führten in weiterer Folge zum Niedergang des großen ökonomischen Zeitalters der Neoklassischen Synthese.

Diese hohe Zeit der Neoklassischen Synthese in der Makroökonomie, im langen Zeitraum zwischen Ende des Zweiten Weltkrieges bis in die später 1960er Jahre hinein, ist aus heutiger Sicht bemerkenswert. Die Makroökonomie war in dieser Zeit in einer Schule vereinigt. Es gab kaum bedeutenden Gegenansätze zur Neoklassischen Synthese. Den wenigen Gegenspielern, zum Beispiel der Österreichische Schule, oder der Freiburger Schule, starben wortwörtlich die Hauptvertreter weg. Ab den 1970er Jahren war die Makroökonomie schließlich geprägt von teils heftigen Auseinandersetzungen verschiedener Schulen. Die jüngeren Vertreter der Neoklassischen Synthese, allen voran James Tobin und Robert Solow, blieben übrigens der Schule bis an deren Lebensende treu. Sie konnten sich natürlich nicht mit den Ideen der Monetaristen und Neuen Klassiker anfreunden, waren aber auch skeptisch gegenüber den Neu-Keynesianern \parencite[S. 148]{Snowdon2005}.