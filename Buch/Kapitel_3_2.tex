%%%%%%%%%%%%%%%%%%%%% chapter.tex %%%%%%%%%%%%%%%%%%%%%%%%%%%%%%%%%
%
% sample chapter
%
% Use this file as a template for your own input.
%
%%%%%%%%%%%%%%%%%%%%%%%% Springer-Verlag %%%%%%%%%%%%%%%%%%%%%%%%%%

\chapter{Synthese: Ein bisschen Neoklassik, ein bisschen Keynes}
\label{Synthese}

\section{Hicks \& Samuelson}

IS-LM, 




Mundell-Fleming-Modell


Modigliani
Franco Modigliani belegt, die diesen wie folgt um-schreibt1): „Non-monetarists accept what 1 regard to be the fundamental practical message of the General Theory: that a private enterprise economy using an intangible money needs to be stabilized, can be stabilized and therefore should be stabilized by appropriate monetary and fiscal policies. Mone-tarists by contrast take the view that there is no serious need to stabilize the economy; that even if there were a need, it could not be done, for stabilizing policies would be more likely
Mone-tarists by contrast take the view that there is no serious need to stabilize the economy; that even if there were a need, it could not be done, for stabilizing policies would be more likely to increase than to decrease instability; and, at least some monetarists would, 1 believe, go so far as to hold that, even in the unlikely event that stabilization policies could on balance prove beneficial, the government should not be trusted with the necessary power."

1) American Economic Review, März 1977.






\section{Solow-Wachstumsmodell} \label{sec: Solow-Modell}

Ausgangspunkt ist eine Produktionsfunktion, wie jene, die wir gerade in Kapitel \ref{sec: Cobb-Douglas-Produktionsfunktion} kennen gelernt haben.  Der Output – gesamtwirtschaftlich das BIP – wird durch verschiedene Inputs – standardmäßig in der Neoklassik Arbeit und Kapital – hervorgebracht. Der Output ergibt sich also aus eine Kombination der beiden Inputfaktoren, Ökonomen würden sagen: „Der Output ist eine Funktion der Inputfaktoren“. 
Folgende Überlegung macht recht schnell klar, warum man mit diesem einfachen Modell stetige Wachstumsraten nicht erklären kann: Angenommen der Output ist einfach eine Addition der beiden Inputfaktoren Arbeit und Kapital. Möchte ich den Output verdoppeln, so müsste ich \textit{beide} Inputfaktoren Arbeit und Kapital jeweils verdoppeln. Ökonomen sprechen hier von konstanten Skalenerträgen. 
Was passiert aber wenn nur einer der beiden Input-Faktoren steigen kann? Gesamtwirtschaftlich könnte man argumentieren, dass der Produktionsfaktor Arbeit durch die Bevölkerungszahl begrenzt ist. Wenn man dies in unserer Überlegung berücksichtigt, würden wir folglich nicht beide Inputfaktoren gleichzeitig erhöhen, sondern nur einen, nämlich Kapital. Erhöhen wir diesen Inputfaktor nun um eine Einheit und der andere Inputfaktor bleibt gleich, so steigt der Gesamtoutput zwar selbstverständlich an, allerdings pro zusätzlicher Einheit um einen immer geringeren Prozentsatz. Intuitiv ist das leicht verständlich: Erhöhe ich bei gleichbleibender Mitarbeiterzahl ständig das Kapital – zum Beispiel die Anzahl der Computer – dann wird der erste eingesetzte Computer einen hohen Zuwachs an Produktivität bringen. Mit jedem weiteren Computer wird die Produktivität zwar weiter steigen, allerdings mit immer geringerer Zuwachsrate. Wenn jeder Mitarbeiter mehr als einen Computer besitzt, wird der Produktivitätszuwachs verschwindend gering werden. Diesen Zusammenhang bezeichnen Ökonomen als „Abnehmenden Grenzertrag“.
Das würde aber bedeuten, dass bei ungefähr gleich bleibender Arbeitsbevölkerung – eine Annahme, die man zumindest für die mittlere Frist in Industriestaaten, bedenkenlos machen kann – die Wirtschaftsleistung stagnieren sollte. Geht man davon aus, dass der Kapitaleinsatz ständig steigt, wäre zwar stetiges Wachstum möglich, aber nur mit immer geringer werdenden Wachstumsraten \footnote{Ständig steigender Kapitaleinsatz wäre nur mit steigenden Sparquoten erklärbar. In einer Ökonomie ohne Außenhandel gilt ja, dass das Sparen der Haushalten den Investitionen der Firmen entspricht. Investitionen wiederum bedeuten einen Aufbau von Kapital. Spezielle Beobachtungen von Fällen von Wirtschaftswachstum werden tatsächlich darauf zurückgeführt, dass die Sparquoten gestiegen sind. So ist zum Beispiel die Ökonomie in der stalinistischen Sowjetunion tatsächlich beträchtlich gewachsen. Da man aber keine wesentlichen technologischen Vorsprünge des Landes in dieser Zeit ausmachen kann, vermutet man, dieses Wachstum sei eben alleine auf den Anstieg der Sparquote zurückzuführen.}. Langfristig würden die Zuwachsraten aber gegen Null tendieren, womit auch in diesem Fall die Wirtschaftsleistung stagniert.

Bisher haben wir aber eine Möglichkeit außer Acht gelassen: Nämlich, dass die eingesetzten Maschinen (hier als Synonym für Kapital verwendet) immer besser werden. Tatsächlich wird eine Arbeitskraft mit zwei Computern nicht wesentlich produktiver sein, als mit einem Computer. Sie könnte aber mit einem \textit{besseren} Computer wesentlich produktiver sein. Es könnte sich also nicht nur die \textit{Menge} des Kapitals verändern, sondern auch dessen \textit{Qualität}. Dies nennen wir „technischen Fortschritt“ \footnote{Technischer Fortschritt umfasst nicht nur die Weiterentwicklung bestehender Produkte zu „besseren“ Produkten, sondern auch die Einführung neuer Produkte}.
Berücksichtigt man diesen Umstand, kommt man zu dem Ergebnis, dass stetiges Wirtschaftswachstum nur dann möglich ist, wenn sich die eingesetzten Kapitalgüter – also zum Beispiel Maschinen, Computer, Transportmittel, Kommunikationsmittel – immer weiter verbessern.
Der Inhalt des „technischen Fortschritts“, also was sich wie verbessert – ist allerdings ist nicht Teil der Ökonomie. Die ersten Wachstumstheorien haben sich also damit abgefunden festzustellen, dass technischer Fortschritt für Wachstum notwendig ist, dieser selbst allerdings nicht durch ökonomisches Handeln beeinflusst werden kann. Der technische Fortschritt wurde also als „exogen“ betrachtet. Daher der Name „exogene Wachstumstheorie“.

\section{Tobin}

Tobin war skeptisch gegenüber Neu-Keynesianern. \textcite[S. 398]{Snowdon2005}

\section{Philipskurve} \label{sec: Phillips}

Zusammenhang von Philips
Samuelson und Solow für die USA: Name: Philipskurve


Später: Konzept NAIRU


Dagegen Friedman und Phelps:
Friedman (1968): The Role of Monetary Policy
Phelps (1968): Money-Wage Dynamics and Labor-Marlet Equilibrium (viel formaler als Friedman)








\section{Arrow-Debreu-Gleichgewicht}
\label{Arrow-Debreu}
Vorarbeiten von Neumann und Leontieff. Danach: \textit{Ramsey-Cass-Koopmans!}

Arrow-Debreu:
Das Arrow-Debreu Gleichgewichtsmodell (auch: Arrow-Debreu-McKenzie-Modell) ist ein mikroökonomisches Modell der gesamten Volkswirtschaft. Es ist nach Gérard Debreu und Kenneth Arrow sowie Lionel W. McKenzie benannt, stellt eine Weiterentwicklung des von Léon Walras entwickelten walrasianischen Gleichgewichtsmodells dar und untersucht einen gesamtwirtschaftlichen Gleichgewichtszustand. 
Das Modell erweitert das allgemeine Gleichgewichtsmodell um unsichere Erwartungen und zustandsabhängige Größen und ist damit für die Finanzierungstheorie von großer Bedeutung. Es zeigt, dass es in einer Marktwirtschaft unter idealisierenden Bedingungen nicht möglich ist, jemanden besserzustellen, ohne jemand anderen schlechterzustellen. Kurz gesagt ist ein Marktgleichgewicht ein Pareto-Optimum. 


