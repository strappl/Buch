%%%%%%%%%%%%%%%%%%%%% chapter.tex %%%%%%%%%%%%%%%%%%%%%%%%%%%%%%%%%
%
% sample chapter
%
% Use this file as a template for your own input.
%
%%%%%%%%%%%%%%%%%%%%%%%% Springer-Verlag %%%%%%%%%%%%%%%%%%%%%%%%%%

\chapter{Synthese: Ein bisschen Neoklassik, ein bisschen Keynes}
\label{Synthese}

\section{Hicks \& Samuelson}

IS-LM, 




Mundell-Fleming-Modell


Modigliani
Franco Modigliani belegt, die diesen wie folgt um-schreibt1): „Non-monetarists accept what 1 regard to be the fundamental practical message of the General Theory: that a private enterprise economy using an intangible money needs to be stabilized, can be stabilized and therefore should be stabilized by appropriate monetary and fiscal policies. Mone-tarists by contrast take the view that there is no serious need to stabilize the economy; that even if there were a need, it could not be done, for stabilizing policies would be more likely
Mone-tarists by contrast take the view that there is no serious need to stabilize the economy; that even if there were a need, it could not be done, for stabilizing policies would be more likely to increase than to decrease instability; and, at least some monetarists would, 1 believe, go so far as to hold that, even in the unlikely event that stabilization policies could on balance prove beneficial, the government should not be trusted with the necessary power."

1) American Economic Review, März 1977.



\section{Tobin}

Tobin war skeptisch gegenüber Neu-Keynesianern. \textcite[S. 398]{Snowdon2005}

\section{Philipskurve} \label{sec: Phillips}

Zusammenhang von Philips
Samuelson und Solow für die USA: Name: Philipskurve


Später: Konzept NAIRU


Dagegen Friedman und Phelps:
Friedman (1968): The Role of Monetary Policy
Phelps (1968): Money-Wage Dynamics and Labor-Marlet Equilibrium (viel formaler als Friedman)



