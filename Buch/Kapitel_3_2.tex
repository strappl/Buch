%%%%%%%%%%%%%%%%%%%%% chapter.tex %%%%%%%%%%%%%%%%%%%%%%%%%%%%%%%%%
%
% sample chapter
%
% Use this file as a template for your own input.
%
%%%%%%%%%%%%%%%%%%%%%%%% Springer-Verlag %%%%%%%%%%%%%%%%%%%%%%%%%%

\chapter{Die Verschmelzung: Ein bisschen Neoklassik, ein bisschen Keynes}
\label{Synthese}

Werke angeführt: \parencite[S. 57]{Snowdon2005}

\section{Hicks und Samuelson: Die neoklassische Synthese}


Bereits 1937 veröffentlichte John R. Hicks den Artikel Mr. Keynes and the "Classic"; A Suggested Interpretation. Das Werk von Keynes hat nämlich die Besonderheit, dass es schwer zu lesen ist, aber vor allem auf formale Darstellungen verzichtet. Hicks übernahm diese Formalisierung und verband einen Teil von Keynes' Theorie mit neoklassischen Elementen zum \textit{IS-LM-Modell}. Dieses Modell stellt noch heute den finalen Punkt in vielen Einführungslehrveranstaltungen zu Makroökonomie dar.

Diese Formalisierung durch Hicks enthält aber zwei extrem wichtige Punkte:
\begin{itemize}
	\item "`Er übernahm einen \textit{Teil} von Keynes' Theorie"': Bei dieser Formalisierung gingen im Gegenzug viele Teile der General Theory verloren. Eine Tatsache, die bis heute in der Mainstream-Ökonomie hingenommen wird. Diese "`verlorenen Teile"' sollten später von den \textit{Post-Keynesianern} aufgegriffen werden.
	\item "`Er verband diesen Teil mit neoklassischen Elementen"': Wenn wir lesen, dass die 1950er und 1960er Jahre die Hochzeit des "`Keynesianismus"' waren, dann meinen Ökonomen eigentlich, dass das "`alte"' Neoklassische Wissen herangezogen wurde und um "`Keynesianische"' Elemente erweitert wurde. Es entstand also eine "`Synthese"' aus zwei Wissensgebieten, folglich wird das Ganze unter Ökonomen die \textsc{Neoklassische Synthese} genannt.
\end{itemize}



Das IS-LM-Modell kann auch als "`Allgemeines Gleichgewichtstheorem"' betrachtet werden:
Hicks, J. R., Value and Capital. Oxford: Clarendon Press. 1939.
Samuelson, P. A., Foundations of Economic Analysis. Cambridge, Mass. Harvard University Press. 1947.

\ref{Arrow-Debreu}

Das Verhältnis zwischen neoklassischen Allgemeinen Gleichgewichtsmodellen, wie jenem von \textcite{Arrow1954} und \textcite{McKenzie1954} und den Modellen der Keynesianer (eigentlich Neoklassische Synthese) \textcite{Hicks1939} wurde häufig diskutiert. Während \textcite{Morgenstern1941} die mathematische Unzulänglichkeit der Letztgenannten hart kritisierte, setzten sich in der wirtschaftspolitischen Anwendung die keynesianischen Gleichgewichtsmodelle durch.



Hicks' Value and Capital 1939 \parencite[S. 20]{Weintraub1983} kann als keynesianische und makroökonomische Abspaltung der "`Allgemeinen Gleichgewichtstheorie"' gesehen werden, die dann von Hicks und Samuelson in keynesianischer Tradition fortgeführt wurde. 
Die beiden wendeten einen anderen Gleichgewichtsansatz an, der die Problematik des Beweises der Existenz eines Allgemeinen Gleichgewichts zur Seite schob.\parencite[S. 21]{Weintraub1983}

\textcite{Morgenstern1941} kritisierte die vereinfachte Annahme eines "`Allgemeinen Gleichgewichts"' durch Hicks und Samuelson vehement. 

IS-LM, 


Mundell-Fleming-Modell


Modigliani

Franco Modigliani belegt, die diesen wie folgt um-schreibt1): „Non-monetarists accept what 1 regard to be the fundamental practical message of the General Theory: that a private enterprise economy using an intangible money needs to be stabilized, can be stabilized and therefore should be stabilized by appropriate monetary and fiscal policies. Mone-tarists by contrast take the view that there is no serious need to stabilize the economy; that even if there were a need, it could not be done, for stabilizing policies would be more likely
Mone-tarists by contrast take the view that there is no serious need to stabilize the economy; that even if there were a need, it could not be done, for stabilizing policies would be more likely to increase than to decrease instability; and, at least some monetarists would, 1 believe, go so far as to hold that, even in the unlikely event that stabilization policies could on balance prove beneficial, the government should not be trusted with the necessary power."

1) American Economic Review, März 1977.

Patinkin, Alvin Hansen

Tobin: war skeptisch gegenüber Neu-Keynesianern. \textcite[S. 398]{Snowdon2005}

\section{Die Phillipskurve: Inflation gegen die Arbeitslosigkeit} \label{sec: Phillips}

Zusammenhang von Philips
Samuelson und Solow für die USA: Name: Philipskurve


Später: Konzept NAIRU


Dagegen Friedman und Phelps:
Friedman (1968): The Role of Monetary Policy
Phelps (1968): Money-Wage Dynamics and Labor-Marlet Equilibrium (viel formaler als Friedman)



