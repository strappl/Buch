%%%%%%%%%%%%%%%%%%%%% chapter.tex %%%%%%%%%%%%%%%%%%%%%%%%%%%%%%%%%
%
% sample chapter
%
% Use this file as a template for your own input.
%
%%%%%%%%%%%%%%%%%%%%%%%% Springer-Verlag %%%%%%%%%%%%%%%%%%%%%%%%%%

\chapter{Exkurs: Behavioral Economics}
\label{Behavioral}


Die Kritik am rational denkenden Menschen, den Homo Oeconomicus ist im wahrsten Sinne so alt wie das Konzept selbst. Der moderne Homo Oeconomicus ist einer, der seinen Nutzen im Sinne einer "`Von Neumann-Morgenstern-Nutzenfunktion"' \parencite{VonNeumann1944} maximiert. Diese hat den Vorteil den Nutzen mit quasi-kardinalen Werten angeben zu können. Nutzen kann seither quantifiziert werden, also mit konkreten Werten, unterlegt werden. Erst dadurch kann man den Nutzen mit Wahrscheinlichkeitswerten hinterlegen und zu Nutzenerwartungswerten weiterentwickeln. Laut \textcite[S. 3]{Selten2001} waren sich Oskar Morgenstern und John von Neumann dieser Schwäche ihres Konzepts nicht nur bewusst, sie nahmen dieses auch durchaus ernst.




\section{Allais: Es begann mit einem Paradoxon}

\section{Simon: Bounded Rationality}
Verweis auf Kapitel Institutionalismus: Simon, H.A., (1957), Models of Man: Social and Rational, New York- Wiley.
Simon (1955): Annahme vollständiger Rationalität wenig Zweckmäßig --> Bounded Rationality

\section{Kahneman und Tversky}
Prospect Theory

\section{Thaler}