%%%%%%%%%%%%%%%%%%%%% chapter.tex %%%%%%%%%%%%%%%%%%%%%%%%%%%%%%%%%
%
% sample chapter
%
% Use this file as a template for your own input.
%
%%%%%%%%%%%%%%%%%%%%%%%% Springer-Verlag %%%%%%%%%%%%%%%%%%%%%%%%%%

\chapter{Neue Neoklassische Synthese}
\label{Neue Neoklassische Synthese}

Mit diesem Kapitel sind wir in der Gegenwart der Ökonomie angekommen. Man kann zwar durchaus argumentieren, dass die Makroökonomie nach der weltweiten Wirtschaftskrise ab 2008 und der immer noch praktizierten globalen Nullzinspolitik eine erneute "`Revolution"' nötig hätte, aber Stand 2022 ist die State-of-the-Art Mainstream-Ökonomie die \textit{Neue Neoklassische Synthese}. Der Begriff "`Neue Neoklassische Synthese"' ist (noch) nicht wirklich etabliert als Bezeichnung für den aktuellen wirtschaftswissenschaftlichen "`Mainstream"'. Meist spricht man stattdessen von "`Neu-Keynesianismus"' oder auch von "`Neoklassik"'\footnote{In Lehrbüchern wird häufig ohne Unterschied vom "`Neu-Keynesianismus"' gesprochen. Auch "`Neu-Keynesianismus der 2. Generation"', "`Neue Synthese"', "`Neue Keynesianische Synthese"'  kommen vor. Selten werden die Modelle auch als "`Neo-Wicksellianisch"' bezeichnet \parencite[S. 28]{Gali2007}, dies wegen der Ähnlichkeit zur Theorie von Wicksell, der Abweichungen vom natürlichen Gleichgewicht beschreibt}. Beide Begriffe sind aber nicht eindeutig. Um Unklarheiten zu vermeiden wird hier der etwas holprige, aber eindeutige und inhaltlich meiner Meinung nach passende Begriff "`Neue Neoklassische Synthese"' (oder schlicht "`Neue Synthese"') verwendet.

Ungefähr um 1990 versuchten Ökonomen, die nicht vom Streit zwischen Neu-Klassikern und Neu-Keynesianern vorbelastet waren, unvoreingenommen das beste aus beiden Welten zu übernehmen und zu einer "`Neuen Synthese"' zusammen zuführen. Die Abgrenzung zwischen "`Neu-Keynesianismus"' und "`Neuer Synthese"' ist hierbei sowohl inhaltlich als auch zeitlich verlaufend. Vor allem, weil viele Ökonomen, die uns im letzten Kapitel untergekommen sind, auch in diesem Kapitel die "`Hauptdarsteller"' sein werden. Es gibt aber auch durchaus Abspaltungen bei den Vertretern des "`Neu-Keynesianismus"': Paul Krugman und Joseph Stiglitz, zum Beispiel, lehnen viele Ansätze der "`Neuen Synthese"' weitgehend ab. Mankiw, Blanchard und David Romer sind schwieriger einer der beiden Schulen zuzuordnen, sie stehen für den Übergang von "`Neu-Keynesianismus"' zur "`Neuer Synthese"'. Ein zentraler Vertreter der "`Neuen Sythese"' (ohne Vergangenheit im "`Neu-Keynesianismus) ist Jordi Gali. Ein spezieller Vertreter ist John Taylor. Er begründete - gemeinsam mit \textcite{Phelps1968} und \textcite{Fischer1977} - den "`Neu-Keynesianismus"' mit \parencite{Taylor1977}, und auch für die "`Neue Synthese"' lieferte er einen der grundlegenden Beiträge \parencite{Taylor1993}. Es gibt heute in der Ökonomie nicht mehr jene klar abgrenzbaren, konkurrierenden Schulen, die die Wirtschaftsgeschichte des 20. Jahrhunderts geprägt haben: Neoklassiker vs. Keynesianer, Keynesianer vs. Monetaristen, Neu-Klassiker vs. Neu-Keynesianer. Vielmehr ist die gesamte Mainstream-Ökonomie unter einem sehr breiten Dach zusammengefasst. Was aber nicht bedeutet, dass unter diesem Dach alle einer Meinung sind, ganz im Gegenteil.\footnote{Neben diesem Mainstream, gibt es immer noch mehrere heterodoxe Schulen, die im Teil \ref{Heterodox} beschrieben werden.}

Passend dazu verschwammen auch die ideologischen Unterschiede zwischen den verschiedenen ökonomischen Gruppen. Konnte man bis in die 1990er Jahre hinein die ökonomischen Richtungen meist auch einer politischen Richtung zuweisen, ist dies heute nicht mehr möglich. Sozialdemokraten (Kontinental-Europa), Labour-Party (UK) und Demokraten (USA) waren fast ausschließlich dem (Neu-) Keynesianismus zugeneigt. Christ-Demokraten (Kontinental-Europa), Tories (UK) und Republikaner (USA) meist den Neoklassikern,  Monetaristen und Neuen Klassikern. Das Spektrum der Vertreter der Neuen Synthese reicht vom Erzliberalen John Taylor über den bekennenden Republikaner Mankiw bis zu Janet Yellen, die Finanzministerin im Kabinett des demokratischen US-Präsidenten Joe Biden ist. 

Bevor wir uns die rein ökonomischen Aspekte der "`Neuen neoklassischen Synthese"' im Detail ansehen, blicken wir auf das Umfeld. Welchen Herausforderungen waren die Volkswirtschaften Anfang der 1990er Jahre ausgesetzt? Politisch gesehen war natürlich der Zusammenbruch der Sowjetunion und damit des real existierenden Sozialismus dominierend. Die Marktwirtschaft, also die Grundlage fast aller in diesem Buch beschriebenen Ideen, hatte sich durchgesetzt. Der Kommunismus, der ohnehin nie so wie von Marx beschrieben praktiziert wurde, galt endgültig als gescheitert. In den westlichen Marktwirtschaften trat das Problem der Inflation in den Hintergrund. Dafür traten Probleme der Arbeitslosigkeit in den Vordergrund, in Europa stärker ausgeprägt als in den USA. Das Problem der steigenden Staatsschulden wurde zunehmend thematisiert, mit ein Grund warum Fiskalpolitik aus dem Fokus geriet. Wechselkurssysteme waren Anfang der 1990er zwar noch einmal ein Thema, als sich nacheinander mehrere europäische Zentralbanken dem Treiben von Spekulanten ausgesetzt sahen, die versuchten die fixierten Wechselkurse zu manipulieren. Doch mit der Schaffung der Europäischen Wirtschafts- und Währungsunion, die in der Einführung des Euros gipfelte, verlor diese Thematik an Bedeutung. Technologisch Begann mit den frühen 1990er Jahren das Zeitalter der Computer. Rechenmaschinen wurden für Haushalte und Kleinunternehmen leistbar und veränderten damit auch das zentrale Verwaltungssystem. Damit verbunden waren wesentliche Verbesserungen in der Datenverfügbarkeit und der Datenauswertung. Die in diesem Buch nicht gesondert behandelte Ökonometrie, sowie die empirische Wirtschaftswissenschaft erfuhr in dieser Zeit einen enormen Aufschwung. Das ist nicht unwesentlich bei der nun folgenden Betrachtung der Weiterentwicklung der Makroökonomie. Die modernen DSGE-Modelle, die im folgenden erläutert werden, sind nur numerisch und damit mit hohem Rechenaufwand zu lösen. 

Zurück zur Entwicklung der Ökonomie: Es gibt vor allem zwei Punkte, die sich Anfang der 1990er-Jahre entwickelt haben und die Makroökonomie und deren Wirtschaftspolitik seither eindeutig prägen und sehr wohl eine eindeutige Abgrenzung vom Neu-Keynesianismus und der Neuen Klassik ermöglichen:
\begin{enumerate}
	\item Erstens, in der makroökonomischen Theorie: die Zusammenführung der formal-mathematischen Real-Business-Cycle-Gleichgewichtsmodelle mit Elementen der Neu-Keynesianer.
	\item Zweitens, in der Wirtschaftspolitik: Die Dominanz der Bedeutung der Geldpolitik und der Aufstieg der Zentralbanken zum wichtigsten wirtschaftspolitischen Player.
\end{enumerate}





\section{Taylor-Rule oder die Verwissenschaftlichung der Zentralbanken}
\label{Taylor-Rule}

\subsection{Exkurs: Die Evolution der Zentralbanken}
Vom Verwalter des Goldstandards, zum Hüter der Wechselkurse, zum Spieler gegen Spekulanten, zur zentralen Player der Wirtschaftspolitik

\subsection{Taylor-Rule: Ein pragmatischer Zugang zur Geldpolitik}
Alle  in diesem Kapitel behandelten Themen umfassten Aspekte die begründen warum Märkte in der Regel nicht vollkommen reibungslos funktionieren. Wäre dies der Fall wäre aktive Wirtschaftspolitik wirkungslos und Konjunkturschwankungen wären rein zufällig, wie im Real-Business-Cycle-Framework dargestellt.
\textcite[S. 823]{Akerlof1985}

Taylor ist eigentlich eher dem erzkonservativem Spektrum zuzuordnen. Unter anderem ist er Vorsitzender der Mont-Pelerin-Gesellschaft. Aber die von ihm etablierte Idee der Zentralbankensteuerung war ein zentraler Schritt zur erneuten "`Wiedervereinigung"' der Ökonomie, also zur "`Neuen neoklassischen Synthese"'.

Die "`Neue Klassische Makroökonomie"' sorgte für viel Wirbel innerhalb der Ökonomie. Der Ton der vorgebrachten Kritik war ungewöhnlich scharf. Die wirtschafts-wissenschaftliche Community war deutlich zerstritten. Aber die "`Neue Klassik"' brachte eben auch viele neue Erkenntnisse und Wege aus den empirisch beobachteten Problemen. Eine gewisse Zeit lang sah es so aus, als würde diese Schule der neue "`Mainstream"' werden. Aber wieder erwiesen sich die Ideen als zu radikal. Empirisch hielt vor allem die Annahme, es gebe keine Preis- und Lohnrigiditäten nicht stand.
Und so kam es dazu, dass sich in der langen Frist die etablierte Mainstream-Ökonomie, also die neoklassische Synthese, durchsetzte. Nicht aber ohne jene Ideen aus der Neuen Klassischen Makroökonomie zu übernehmen, die sich als richtig erwiesen hatten. Dies waren vor allem:

Die Taylor-Rule veränderte die Geldpolitik. Von den Geldmengenzielen (Friedman) zu den Zinssatz-Regeln \parencite[S. 36]{Gali2007} Inflations-targeting


HIER WEITER



\section{DSGE: Die Zweckehe zwischen "`Neuen Klassikern"' und "`Neu-Keynesianern"'}

Dieses Unterkapitel rechtfertigt den Begriff "`Neue Neoklassische \textit{Synthese}"' wie kein anderes. \textit{Die} wesentliche Erweiterung in der Makroökonomie in den 1990er Jahren, war die Entstehung der ersten \textit{Dynamic, Stochastic, General Equilibrium}-Modelle (DSGE-Modelle). Wie wir im Kapitel \ref{Neue Makro} gelesen haben, wurde durch die Arbeiten von \textcite{Kydland1982, Plosser1983} die Methodik der Makroökonomie geradezu revolutioniert. Die dabei entwickelten Real-Business-Cycle-Modelle waren aus mathematisch-modelltheoretischer Sicht den in den 1970er Jahren vorherrschenden Keynesianischen und später auch Monetaristischen Totalmodellen, weit überlegen. Es waren walrasianische Gleichgewichtsmodelle, die damit die damals neu entstandene Mikrofundierung der Makroökonomie umsetzten. Ihre mathematische Eleganz passte perfekt in die Zeit der Formalisierung der Ökonomie, wenngleich die Lösung des Modells, mangels Möglichkeit das Modell analytisch zu lösen, ein paar "`Tricks"' erforderte. Die RBC-Modelle hatten nur einen Haken: Sie scheiterten grandios darin die Empirie gut abzubilden. Eigentlich ein Ausschlusskriterium für ökonomische Modelle. Die Methodik galt aber dennoch als zukunftsträchtig. 

Als Antwort auf diese "`Neu-klassischen"'-Modelle, entwickelten die "`Neu-Keynesianer"' punktuell Ansätze um empirisch beobachtete ökonomische Auffälligkeiten zu erklären. Diese Elemente wurden im letzten Kapitel (\ref{cha: Neu Keynes}) beschrieben. Die "`Verehelichung"' zwischen der RBC-Methodik und Neu-Keynesianischen Lösungsansätzen führte schließlich zu den ersten DSGE-Modellen \parencite[S. 5]{Gali2015}. Konkret wurde das mathematische RBC-Framework um folgende Neu-Keynesianische Elemente erweitert:
\begin{itemize}
	\item "`Nominale Rigiditäten"'
	\item "`Monopolistische Konkurrenz"'
	\item "`Nicht-Neutralität der Geldpolitik in der kurzen Frist"'   
\end{itemize}

Die Kombination der "`neu-keynesianischen"' Ideen in die "`neu-klassischen"' Modelle begründet damit schließlich den Namen "`Neue Neoklassische Synthese"'\footnote{Dies ist aber insofern etwas verwirrend, als die meisten Entwickler von DSGE-Modellen, diese als "`Neu-Keynesianisch"' bezeichnen. Z.B.: \textcite{Gali2015}, \textcite{Romer2019}. \textcite[S. 28]{Gali2007} meint richtigerweise, dass sich der Begriff "`Neu-Keynesianisch"' weitgehend durchgesetzt hat, räumt aber ein, dass dieser Begriff eigentlich unzureichend ist!}.

Das Zusammenführen der beiden Ansätze verlief weniger geradlinig und einfach, als im letzten Absatz dargestellt. So gibt es verschiedene Ansätze \parencite[S. 310]{Romer2019} wie die dynamische Preisanpassung modelliert wird, was wiederum sehr unterschiedliche Auswirkungen der mikroökonomischen Rigiditäten auf die makroökonomischen Ergebnisse verursacht. Die DSGE-Modelle wurden zudem laufend erweitert. Insbesondere die Anfang der 1990er Jahre ebenfalls neuen Erkenntnisse, wie Geldpolitik praktiziert wird (siehe Kapitel \ref{Taylor-Rule}), sowie die Berücksichtigung der "`Neuen Phillipskurve"' (siehe Kapitel \ref{NeuePhillips}) zur Modellierung von Arbeitslosigkeit, stellen wesentliche Weiterentwicklungen dar.   

Konkret werden die drei angeführten Neu-Keynesianische Elemente durch zwei Annahmen eingeführt. Erstens, jedes Unternehmen produziert ein spezielles Produkt, für das es den Preis setzt. Damit ist die Annahme eines monopolistischen Konkurrenzmarktes erfüllt. Zweitens, nominale Preisrigiditäten werden modelliert, indem man annimmt, dass jedes Unternehmen nur zu einem bestimmten Zeitpunkt seine Preise an das Gleichgewicht anpassen kann \parencite[S. 52]{Gali2015}. Im Kapitel \ref{cha: Neu Keynes} wurden drei verschiedene Ansätze erwähnt, wie man dies simulieren kann. Jenen von \textcite{Fischer1977} und \textcite{Taylor1977}, jenen von \textcite{Taylor1979} und schließlich jenen von \textcite{Calvo1983}. Man kann heute sagen, dass sich der Ansatz von \textcite{Calvo1983} durchgesetzt hat. Dieses kann aus mathematischen Gründen am einfachsten in das DSGE-Framework integriert werden. Zur Erinnerung: Welches Unternehmen seinen Preis anpassen darf, wird hier durch eine Poisson-Verteilung, also durch einen Zufallsprozess, bestimmt. Die Annahme der "`Nicht-Neutralität der Geldpolitik in der kurzen Frist"' ist durch die zwei eben genannten Elemente mitumfasst, denn die Preise sind im Ergebnis eben keine walrasianischen Gleichgewichtspreise mehr. Das heißt aber auch, dass sich auch der Geldwert nicht ständig an sein wahres Gleichgewicht anpasst. Diese Abweichung vom Gleichgewicht kann eben für Geldpolitik genutzt werden.

Darauf aufbauend kommen wir nun aber zum DSGE-Grundmodell. Dieses besteht aus drei Gleichungen. Jene für:
\begin{itemize}
	\item Haushalte (Aggregierte Nachfrage)
	\item Unternehmen (Aggregiertes Angebot)
	\item Zentralbank
\end{itemize}

Wie man sieht werden internationaler Handel, aber auch Staatsausgaben und damit Fiskalpolitik, nicht explizit betrachtet\footnote{Mit anderen Worten: Die Ricardianische Äquivalenz ist in DSGE-Modellen eine Standardannahme. Der Nachfrage-Multiplikator ist nicht größer als eins. Fiskalpolitik hat in der Folge keinen positiven Effekt auf das BIP (solange der nominale Zins gesteuert werden kann).}. Der repräsentative Haushalt hat eine unendliche Lebenserwartung und optimiert sein Verhalten unter der Prämisse, dass Konsum einen positiven Nutzen generiert, zu tätigende Arbeit generiert hingegen einen negativen Nutzen.  Außerdem wissen die Haushalte, dass es in der Zukunft zu zufälligen Schocks kommt. Haushalte optimieren also ihren Nutzen, gegeben ihren Erwartungen und dem Wissen um zufällige zukünftige Schocks. Die Optimierungsaufgabe liegt darin das Verhältnis aus Konsum und Arbeit für alle Zeitperioden festzulegen. Die Zukunft (zukünftige Werte), wird dabei - wie immer in der Ökonomie - abgezinst. Der Realzinssatz ist daher \textit{die} entscheidender Stellschraube bei der Optimierung.
Unternehmen haben eine Produktionsfunktion, die von einheitlicher Technologie ausgeht (das heißt nur, dass kein Unternehmen einen Vorteil aus einem technischen Fortschritt generieren kann). Der Produktionsfaktor Arbeit muss zum Nominallohn zugekauft werden. Die Unternehmen produzieren - wie schon erwähnt, Stichwort monopolistische Konkurrenz - jeweils ein einzigartiges Gut. Unternehmen können nur abhängig von einem Zufallsprozess - ebenfalls schon erwähnt bildet dies die nominalen Rigiditäten ab - ihre Preise an Marktgegebenheiten anpassen. Die Optimierungsaufgabe der Unternehmen besteht natürlich in der Gewinnmaximierung. Im Detail verlangt dies eine optimale Preisfestsetzung unter den Nebenbedingungen, dass Preisanpassungen erst wieder in einer zufällig zu bestimmenden Periode in der Zukunft möglich sein werden und in der Zwischenzeit sowohl Änderungen beim Nominalzinssatz, also auch bei Inflationsraten auftreten werden. Aufgrund der Annahme, dass monopolistische Konkurrenzmärkte vorherrschen, führt eine Abweichung des individuellen Preises eines Unternehmens vom Gleichgewichtspreis ja nicht zu einem totalen Rückgang des Gewinns. Unternehmen, die sich monopolistischer Konkurrenz ausgesetzt sehen, schlagen ja ohnehin immer ein "`Mark-up"' auf den Gleichgewichtspreis auf, um ihren Gewinn zu optimieren. Das heißt, Unternehmen setzen ihr Mark-up so, dass sie die Erwartungen ihrer zukünftigen Kostenänderung (durch Inflation) einpreisen und zusätzlich ihre Erwartungen der Dauer bis zur nächsten Preissetzung berücksichtigen. Ich denke es erscheint klar, erstens, dass dies mathematisch herausfordernd zu modellieren ist und Annahmen getroffen werden müssen, zweites, dass es unterschiedliche Ansätze gibt dies zu modellieren, drittens, dass der Prozess als Ganzes Kritik ausgesetzt werden kann, doch dazu später.
Als dritte Gleichung kommt jene der Geldpolitik ins Spiel. Wie wir bereits gesehen haben sind für Haushalte und Unternehmen vor allem die Variablen Nominalzins und Inflation - im Resultat also der Realzins -  von Bedeutung bei ihrer Optimierungsaufgabe. Und an dieser Stelle kommt wieder die Taylor-Regel\parencite{Taylor1993} ins Spiel, die wir bereits im letzten Kapitel \ref{Taylor-Rule} kennengelernt haben. Diese ist zwar nicht mikrofundiert, bestimmt aber zu jedem Zeitpunkt die resultierenden Handlungen der Zentralbanken. Es ist auch möglich, die Geldpolitik anders zu modellieren, aber im Ausgangsmodell hat sich das Konzept der Zinsregel etabliert. 


HIER WEITER:

mit ersten Modellen und Überleitung zu "`Neuen Phillips-Kurve"'

Erstes Neu-keynesianisches DSGE-Modell: Rotemberg und Woodford
Rotemberg, Julio; Woodford, Michael (1993), "Dynamic General Equilibrium Models with Imperfectly Competitive Product Markets", NBER Working Paper No. 4502

Rotemberg, Julio; Woodford, Michael (1995). "Dynamic general equilibrium models with imperfectly competitive product markets". In Cooley, Thomas (ed.). Frontiers of Business Cycle Research. Princeton University Press. ISBN 0-691-04323-X.

Rotemberg, Julio J.; Woodford, Michael (1997). "An optimization-based econometric framework for the evaluation of monetary policy" (PDF). NBER Macroeconomics Annual. 12: 297–346. doi:10.1086/654340. JSTOR 3585236. S2CID 154438345


Als Standardmodell etablierte sich schließlich jenes Drei-Gleichungen-Modell, das \textcite{Gali2000} veröffentlichten\parencite[S. 311]{Romer2019}.




Yun (1996)


WElche Aussagen liefern die Modell überhaupt? 





\subsection{Erweiterungen}
Sticky Wages und Sticky Prices: Kapitel 6 im Gali-Buch!
Unemployment (Neue Phillips Kurve?): Kapitel 7 im Gali-Buch! + Romer-Buch S. 327ff + S. 338. Ursprünglich: Roberts 1995!
"`HANK"'!: Kapitel 9 im Gali-Buch!


\subsection{Erweiterungen}


\section{Neue Phillips Kurve}
\label{NeuePhillips}

In Gali and Gertler (1999, 2007, S. 32ff) and Gali, Gertler and Lopez-Salido (2005),
Rotemberg 1982 (laut Gali-Buch, S. 80)


Jordi Gali usw nach 2008




Richard Clarida and Jordi Gali and Mark Gertler, 2000. "Monetary Policy Rules and Macroeconomic Stability: Evidence and Some Theory," The Quarterly Journal of Economics, Oxford University Press, vol. 115(1), pages 147-180.

Mark Gertler and Jordi Gali and Richard Clarida, 1999. "The Science of Monetary Policy: A New Keynesian Perspective," Journal of Economic Literature, American Economic Association, vol. 37(4), pages 1661-1707, December.



Inhaltlich spielt in der Wirtschaftspolitik fast ausschließlich nur mehr die Geldpolitik eine Rolle. Zur Fiskalpolitik haben die Vertreter der "`neuen Neoklassischen Synthese"' praktisch ausschließlich eine ablehnende Haltung.






















RBC + \textcite{RomerDavid1990}

Formulierung der Taylor-Rule (1993?) als  Übergangszeitpunkt 1. Generation --> 2. Generation
Zweiter Übergangspunkt: Ab 1990 viel stärker empirisch (bis dahin sehr theoretische Arbeiten der Neu-Keynesianer)

Problem der Arbeitslosigkeit trat in den Vordergrund.


(Taylor Rule)

Neue Phillips Kurve

Cost of Inflation (\textcite{Snowdon2005} ab S. 401) und Inflations-targeting


Wirtschaftspolitisch Dominanz der Geldpolitik
Fiskalpolitik selbst in Krisen umstritten (Paper zur empirischen Bestimmung der Wirksamkeit der Fiskalpolitik (Blanchard))



Interessant ist, das

Elemente aus verschiedenen Schulen:
\begin{itemize}
	\item Keynesianismus: Rigiditäten. Teilweise Fiskalpolitik im Krisenfall
	\item Monetarismus: Zentrale Bedeutung der Geldpolitik und der Zentralbanken, Natürliche Arbeitslosigkeit
	\item Österreichische Schule: Konzept des Gleichgewichtszinssatzes (Wicksell)
	\item Neu-Keynesianismus: "`Nominale Rigiditäten"', "`Monopolistische Konkurrenz"' und "`Nicht-Neutralität der Geldpolitik in der kurzen Frist"'. Älter: NAIRU, Marktversagen
	\item Neue Klassische Makroökonomie: Annahme "`Rationale Erwartungen"', Real-Business-Cycle-Modelle.
	\item Post-Keynes: Mark-up (monopolistische Konkurrenz) kommt eigentlich daher.
\end{itemize}





\section{Auf den Schultern von Giganten}
\label{Giganten}

\subsection{Krugman}

\subsection{Blanchard}

\subsection{David Romer \& Mankiv}





Solow (2010) äußerte sich sehr kritisch gegenüber DSGE-Modellen. Er vertrat die Meinung, dass 
DSGE-Modelle keinen Bezug zur Realität hätten und daher ungeeignet für die Politikberatung 
seien.



Blanchard, Oliver J., 2016. Do DSGE Models Have a Future?, in: Peterson Institute of International Economics, Policy Brief, Nr. 16-11, Washington, D.C.

Romer, Paul, 2016, The trouble with macroeconomics, in: The American Economist, forthcomin










