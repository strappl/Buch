%%%%%%%%%%%%%%%%%%%%% chapter.tex %%%%%%%%%%%%%%%%%%%%%%%%%%%%%%%%%
%
% sample chapter
%
% Use this file as a template for your own input.
%
%%%%%%%%%%%%%%%%%%%%%%%% Springer-Verlag %%%%%%%%%%%%%%%%%%%%%%%%%%

\chapter{Neue Neoklassische Synthese}
\label{Neue Neoklassische Synthese}

RBC +
Nominale Rigiditäten
Monopolistische Konkurrenz
Geldpolitik in der kurzen Frist (Taylor Rule)


Wirtschaftspolitisch Dominanz der Geldpolitik
Fiskalpolitik selbst in Krisen umstritten (Paper zur empirischen Bestimmung der Wirksamkeit der Fiskalpolitik)

Formulierung der Taylor-Rule (1993?) als zweiter Übergangszeitpunkt 1. Generation --> 2. Generation

Im vorherigen Kapitel haben wir die direkten Antworten auf unrealistische Neu-Klassische Konzepte durch Ökonomen beschrieben, die die Defizite der Keynesianer erkannten aber auch die Modelle der Neuen Klassiker ablehnten. Diese Ökonomen nannten wir "`Neu-Keynesianer"' (Neu-Keynesianer der 1. Generation). Ungefähr um 1990 versuchten junge Ökonomen, die nicht vom Streit zwischen Neu-Klassikern und Neu-Keynesianern vorbelastet waren, unvoreingenommen das beste aus beiden Welten zu übernehmen und zu einer "`Neuen Synthese"' zusammen zuführen. Die Abgrenzung ist natürlich sowohl inhaltlich als auch zeitlich schwierig. Der hier verfolgte inhaltlich Ansatz zwischen der 1. Generation von der 2. Generation der Neu-Keynesianer zu unterscheiden ist, dass die erste Generation bereits die typisch neu-keynesianischen Elemente "`Nominale Rigiditäten"', "`Monopolistische Konkurrenz"' und "`Nicht-Neutralität der Geldpolitik in der kurzen Frist"' allesamt ausgearbeitet hat, aber erst die zweite Generation bettete diese Elemente in die ursprünglich "`neu-klassischen"' DSGE-Modelle vollständig ein \parencite[S. 6]{Gali2015}. Die Kombination der "`neu-keynesianischen"' Ideen in die "`neu-klassischen"' Modelle begründet schließlich den Namen "`Neue Neoklassische Synthese"'\footnote{Häufig eben einfach "`Neu-Keynesianismus"', "`Neu-Keynesianismus, 2. Generation"', "`Neue Synthese"', "`Neue Keynesianische Synthese"' genannt. Selten werden die Modelle auf "`Neo-Wicksellianische"' Modelle genannt \parencite[S. 28]{Gali2007}, wegen der Ähnlichkeit der Abweichungen vom natürlichen Gleichgewicht zur Theorie von Wicksell} Daher wäre unrichtig die "`Neuen Keynesianer 1. Generation"' und die Vertreter der "`Neuen Synthese"' nicht zu trennen. Die "`Neu-Keynesianer 1. Generation"' Krugman, Stiglitz, Fischer haben doch ganz andere Ansätze als die Vertreter der "`Neuen Synthese"' wie zum Beispiel Jordi Gali. Andere Vertreter wie Mankiw, Blanchard und David Romer sind schwieriger zuzuordnen und stehen für den Übergang von 1. Generation zu 2. Generation. 
Inhaltlich spielt in der Wirtschaftspolitik fast ausschließlich nur mehr die Geldpolitik eine Rolle. Zur Fiskalpolitik haben die Vertreter der "`neuen Neoklassischen Synthese"' praktisch ausschließlich eine ablehnende Haltung.

Der Begriff "`Neue Neoklassische Synthese"' ist (noch) nicht wirklich etabliert als Bezeichnung für den aktuellen wirtschaftswissenschaftlichen "`Mainstream"'. Meist spricht man stattdessen von "`Neukeynesianismus"' oder auch von "`Neoklassik"'. Beide Begriffe sind aber nicht eindeutig. Man behilft sich dann mit Konstrukten wie "`Neukeynesianer der ersten und zweiten Generation"'. Um dies zu vermeiden wird hier der etwas holprige aber eindeutige und inhaltlich meiner Meinung nach passende Begriff "`Neue Neoklassische Synthese"' verwendet.

Passend dazu verschwammen damit auch die ideologischen Unterschiede zwischen den Gruppen. Konnte man bis in die 1980er Jahre hinein die ökonomischen Richtungen einer politischen Richtung zuweisen, ist dies nun nicht mehr möglich.
Die Ideen zur Geldpolitik des Erzliberalen John Taylor, die Arbeiten des bekennenden Republikaners Mankiw aber auch 


heutige Mainstream-Modell basieren auf dem "`Neu-Keynesianischen Ausgangsmodell"', das bis heute das Arbeitstier in der Analyse von wirtschaftspolitischen Maßnahmen, Gleichgewichtsabweichungen und Wohlstand ist \parencite[S. 52]{Gali2015}






\section{Taylor-Rule: Ein pragmatischer Zugang zur Geldpolitik}
Taylor ist eigentlich eher dem erzkonservativem Spektrum zuzuordnen. Unter anderem ist er Vorsitzender der Mont-Pelerin-Gesellschaft. Aber die von ihm etablierte Idee der Zentralbankensteuerung war eher ein Schritt zur erneuten "`Wiedervereinigung"' der Ökonomie, also zur "`Neuen neoklassischen Synthese"'.

Die "`Neue Klassische Makroökonomie"' sorgte für viel Wirbel innerhalb der Ökonomie. Der Ton der vorgebrachten Kritik war ungewöhnlich scharf. Die wirtschafts-wissenschaftliche Community war deutlich zerstritten. Aber die "`Neue Klassik"' brachte eben auch viele neue Erkenntnisse und Wege aus den empirisch beobachteten Problemen. Eine gewisse Zeit lang sah es so aus, als würde diese Schule der neue "`Mainstream"' werden. Aber wieder erwiesen sich die Ideen als zu radikal. Empirisch hielt vor allem die Annahme, es gebe keine Preis- und Lohnrigiditäten nicht stand.
Und so kam es dazu, dass sich in der langen Frist die etablierte Mainstream-Ökonomie, also die neoklassische Synthese, durchsetzte. Nicht aber ohne jene Ideen aus der Neuen Klassischen Makroökonomie zu übernehmen, die sich als richtig erwiesen hatten. Dies waren vor allem:

Die Taylor-Rule veränderte die Geldpolitik. Von den Geldmengenzielen (Friedman) zu den Zinssatz-Regeln \parencite[S. 36]{Gali2007}





\begin{itemize}
	\item die Annahme rationaler Erwartungen
	\item die Mikrofundierung der Makroökonomie
	\item (vergleiche Kapitel Neue Klassische Makro)
\end{itemize}






\section{Auf den Schultern von Giganten}
\label{Giganten}

\subsection{Krugman}

\subsection{Blanchard}

\subsection{David Romer \& Mankiv}



\section{HANK}
"`HANK"'!
Kapitel 9 im Gali-Buch!

\section{Neue Philips Kurve}

In Gali and Gertler (1999) and Gali, Gertler and Lopez-Salido (2005),


Jordi Gali usw nach 2008







