%%%%%%%%%%%%%%%%%%%%% chapter.tex %%%%%%%%%%%%%%%%%%%%%%%%%%%%%%%%%
%
% sample chapter
%
% Use this file as a template for your own input.
%
%%%%%%%%%%%%%%%%%%%%%%%% Springer-Verlag %%%%%%%%%%%%%%%%%%%%%%%%%%

\chapter{Neue Neoklassische Synthese}
\label{Neue Neoklassische Synthese}

Der Begriff "`Neue Neoklassische Synthese"' ist (noch) nicht wirklich etabliert als Bezeichnung für den aktuellen wirtschaftswissenschaftlichen "`Mainstream"'. Meist spricht man stattdessen von "`Neukeynesianismus"' oder auch von "`Neoklassik"'. Beide Begriffe sind aber nicht eindeutig. Man behilft sich dann mit Konstrukten wie "`Neukeynesianer der ersten und zweiten Generation"'. Um dies zu vermeiden wird hier der etwas holprige aber eindeutige und inhaltlich meiner Meinung nach passende Begriff "`Neue Neoklassische Synthese"' verwendet.



\section{Taylor-Rule: Ein pragmatischer Zugang zur Geldpolitik}
Taylor ist eigentlich eher dem erzkonservativem Spektrum zuzuordnen. Unter anderem ist er Vorsitzender der Mont-Pelerin-Gesellschaft. Aber die von ihm etablierte Idee der Zentralbankensteuerung war eher ein Schritt zur erneuten "`Wiedervereinigung"' der Ökonomie, also zur "`Neuen neoklassischen Synthese"'.

Die "`Neue Klassische Makroökonomie"' sorgte für viel Wirbel innerhalb der Ökonomie. Der Ton der vorgebrachten Kritik war ungewöhnlich scharf. Die wirtschafts-wissenschaftliche Community war deutlich zerstritten. Aber die "`Neue Klassik"' brachte eben auch viele neue Erkenntnisse und Wege aus den empirisch beobachteten Problemen. Eine gewisse Zeit lang sah es so aus, als würde diese Schule der neue "`Mainstream"' werden. Aber wieder erwiesen sich die Ideen als zu radikal. Empirisch hielt vor allem die Annahme, es gebe keine Preis- und Lohnrigiditäten nicht stand.
Und so kam es dazu, dass sich in der langen Frist die etablierte Mainstream-Ökonomie, also die neoklassische Synthese, durchsetzte. Nicht aber ohne jene Ideen aus der Neuen Klassischen Makroökonomie zu übernehmen, die sich als richtig erwiesen hatten. Dies waren vor allem:

\begin{itemize}
	\item die Annahme rationaler Erwartungen
	\item die Mikrofundierung der Makroökonomie
	\item (vergleiche Kapitel Neue Klassische Makro)
\end{itemize}






\section{Auf den Schultern von Giganten}
\label{Giganten}

\subsection{Krugman}

\subsection{Blanchard}

\subsection{David Romer \& Mankiv}









