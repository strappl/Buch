%%%%%%%%%%%%%%%%%%%%%%%% part.tex %%%%%%%%%%%%%%%%%%%%%%%%%%%%%%%%%%
%
% sample part title
%
% Use this file as a template for your own input.
%
%%%%%%%%%%%%%%%%%%%%%%%% Springer-Verlag %%%%%%%%%%%%%%%%%%%%%%%%%%


\part{1870 -- 1930\\Neoklassik: Revolutionen auf Ökonomisch}

Innerhalb der Wirtschaftswissenschaften kam es immer wieder zu schlagartigen Umbrüchen. Der "`Marxistischen Revolution"' (vgl. Kapitel \ref{Marx}) war - zumindest aus heutiger Sicht - kein nachhaltiger Erfolg beschieden. Ganz anders der sogenannten "`Marginalistischen Revolution"' in deren Folge sich die Ökonomie als Wissenschaft grundlegend änderte. Die in der Folge entstehende "`Neoklassik"' enthält viele Elemente, die bis heute den "`State of the Art"' der Wirtschaftswissenschaften bilden. Tatsächlich wird die heutige "`Mainstream"'-Ökonomie heute noch häufig einfach als Neoklassik bezeichnet. Aus wirtschaftshistorischer Sicht ist dies allerdings ungünstig, da damit zu viele moderne Elemente mitgemeint wären. Betreffend Mikroökonomie wurden die Grundlagen für das heute verwendete Instrumentarium aber tatsächlich durch die (späten) Neoklassiker entwickelt (vgl. Kapitel \ref{Neoklassik}). Die Neoklassik hat viele verschiedene Bezeichnung, die mehr oder weniger gleichwertig benutzt werden. So spricht man häufig vom "`Marginalismus"' oder der "`Grenznutzenschule"'.

Die Entstehung der Neoklassik ist damit verbunden, dass die Klassiker kein befriedigendes Konzept hatten um den Unterschied zwischen Preis und Wert eines Gutes zu bestimmen. Damit konnten sie nicht erklären warum bestimmte Güter für verschiedene Personen unterschiedlich wertvoll sind. Die Neoklassik führte dazu das Konzept des Nutzens ein und damit den Homo Oeconomicus - den nutzenmaximierenden Menschen. Die Nutzentheorie stellt bis heute einen wesentlichen Teil der mikroökonomischen Haushaltstheorie dar und ist bis heute ein hoch umstrittenes und viel beforschtes Thema. Mit Hilfe des Nutzenkonzepts lässt sich erklären, warum der Wert einer Ware unterschiedlich hoch bewertet wird - je nach vorhandener Menge, bzw. auch von verschiedenen Personen. Der Wert einer Ware ist also nicht mehr objektiv bestimmbar, sondern subjektiv. Man spricht daher auch von subjektiver Wertlehre. Diese "`Subjektivität"' ist einer der wesentlichen Punkte in der Neoklassik. Allerdings gab es recht unterschiedliche Ansätze damit umzugehen. In Wien\footnote{Zur Bedeutung der genannten Orte mehr in den folgenden Kapiteln} war man der Meinung, dass man so etwas subjektives wie den Nutzen schlicht nicht quantitativ bewerten kann und soll. Ganz anders in Cambridge, wo die höhere Mathematik Einzug in die Ökonomie fand. Wissenschaftlich gesehen ist die Entstehung der Neoklassik mit dem Einzug der höheren Mathematik - konkret der hundert Jahre zuvor entwickelten Infinitesimalrechnung -  in die Ökonomie gleichzusetzen. Der Wert eines Gutes wird in der Neoklassik als subjektiv angesehen. Eine zusätzliche Menge eines Gutes, bringt keinen konstanten Nutzenzuwachs, sondern einen etwas geringeren. Das Konzept des abnehmenden Grenznutzens ist damit geboren. Im Deutschen spricht man vom \textit{Grenz}nutzen, wenn der zusätzliche Nutzen einer zusätzlichen Einheit des Gutes gemeint ist. Im Englischen spricht man von "`Marginal Utility"', daher der Begriff des "`Marginalismus"' als äquivalent zur Grenznutzenschule. Der Nutzen einer zusätzlichen Einheit eines bestimmten Gutes ist stets kleiner als der Nutzen der Einheit die man davor erhalten hat. Oder einfach ausgedrückt: Nach einer langen Wanderung freut man sich extrem auf das erste Bier. Auch das zweite mag noch Freude bereiten, aber nicht mehr soviel wie das erste. Man spricht vom abnehmenden Grenznutzen, der ein wesentliches Konzept der Neoklassik darstellt. Man beachte, dass der Preis eines Gutes konstant bleibt. Natürlich verlangt ein Wirt für das erste Bier gleich viel wie für das zweite Bier. Abhängig von der individuellen Nutzenfunktion lässt sich damit eine optimale Menge eines bestimmten Gutes berechnen, bei der der generierte Nutzen maximiert wird. Bei den ersten Lesern werden sich schon die Nackenhaare aufstellen: Ja genau, wenn man die optimale Menge als stetige Menge betrachtet, handelt es sich hier um ein Maximierungsproblem, wie wir es in der Mathematik-Oberstufe kennen gelernt haben. Die Ökonomie war damit als Spielwiese für die Mathematik entdeckt worden. Bis heute ist umstritten inwieweit sich wirtschaftliches Verhalten durch mathematische Modelle abbilden lässt. Es ist interessant, dass die Österreichische Schule der Nationalökonomie, deren Ursprung auf die marginalistische Revolution in Wien zurückgeht, bis heute der höheren Mathematik in der Ökonomie abgeneigt ist.


Das politische und wirtschaftliche Umfeld der 1870er-Jahre war geprägt von den späten Jahren der Monarchien und der langsamen Entstehung des Nationalismus in Zentraleuropa. Damit verbunden die späte Zeit des Kolonialismus und damit die Zeit des Hochimperialismus. 

HIER WEITER


