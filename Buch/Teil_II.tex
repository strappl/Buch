%%%%%%%%%%%%%%%%%%%%%%%% part.tex %%%%%%%%%%%%%%%%%%%%%%%%%%%%%%%%%%
%
% sample part title
%
% Use this file as a template for your own input.
%
%%%%%%%%%%%%%%%%%%%%%%%% Springer-Verlag %%%%%%%%%%%%%%%%%%%%%%%%%%


\part{1870 -- 1930\\Neoklassik: Revolutionen auf Ökonomisch}

Innerhalb der Wirtschaftswissenschaften kam es immer wieder zu schlagartigen Umbrüchen. Die "`Marxistischen Revolution"' (vgl. Kapitel \ref{Marx}) war - zumindest aus heutiger Sicht - kein nachhaltiger Erfolg beschieden. Ganz anders der sogenannten "`Marginalistischen Revolution"' in deren Folge sich die Ökonomie als Wissenschaft grundlegend änderte. Die in der Folge entstehende "`Neoklassik"' enthält viele Elemente, die bis heute den "`State of the Art"' der Wirtschaftswissenschaften bilden. Tatsächlich wird die heutige "`Mainstream"'-Ökonomie heute noch häufig einfach als Neoklassik bezeichnet. Aus wirtschaftshistorischer Sicht ist dies allerdings ungünstig, da damit zu viele moderne Elemente mitgemeint wären. Betreffend Mikroökonomie wurden die Grundlagen für das heute verwendete Instrumentarium aber tatsächlich durch die (späten) Neoklassiker entwickelt (vgl. Kapitel \ref{Neoklassik}). Die Neoklassik hat viele verschiedene Bezeichnung, die mehr oder weniger gleichwertig benutzt werden. So spricht man häufig vom "`Marginalismus"' oder der "`Grenznutzenschule"'.

Die Entstehung der Neoklassik ist damit verbunden, dass die Klassiker kein befriedigendes Konzept hatten um den Unterschied zwischen Preis und Wert eines Gutes zu bestimmen. Damit konnten sie nicht erklären warum bestimmte Güter für verschiedene Personen unterschiedlich wertvoll sind. Die Neoklassik fügte dazu das Konzept des Nutzens ein und damit den Homo Oeconomicus - den nutzenmaximierenden Menschen. Die Nutzentheorie stellt bis heute einen wesentlichen Teil der mikroökonomischen Haushaltstheorie dar und ist bis heute ein hoch umstrittenes und viel beforschtes Thema. Mit Hilfe des Nutzenkonzepts lässt sich erklären, warum der Wert einer Ware unterschiedlich hoch bewertet wird - je nach vorhandener Menge, bzw. auch von verschiedenen Personen. Der Wert einer Ware ist also nicht mehr objektiv bestimmbar, sondern subjektiv. Man spricht daher auch von subjektiver Wertlehre. Diese "`Subjektivität"' ist einer der wesentlichen Punkte in der Neoklassik. Allerdings gab es recht unterschiedliche Ansätze damit umzugehen. In Wien\footnote{Zur Bedeutung der genannten Orte folgt gleich mehr.} war man der Meinung, dass man so etwas subjektives wie den Nutzen schlicht nicht quantitativ bewerten kann und soll. Ganz anders in Cambridge, wo die höhere Mathematik Einzug in die Ökonomie fand. Wissenschaftlich gesehen ist die Entstehung der Neoklassik mit dem Einzug der höheren Mathematik - konkret der hundert Jahre zuvor entwickelten Infinitesimalrechnung -  in die Ökonomie gleichzusetzen. Der Wert eines Gutes wird in der Neoklassik als subjektiv angesehen. Eine zusätzliche Menge eines Gutes, bringt keinen konstanten Nutzenzuwachs. Die Ökonomie war damit als Spielwiese für die Mathematik entdeckt worden. Bis heute ist umstritten inwieweit sich wirtschaftliches Verhalten durch mathematische Modelle abbilden lässt. Es ist interessant, dass die Österreichische Schule der Nationalökonomie, deren Ursprung auf die marginalistische Revolution in Wien zurückgeht, bis heute der höheren Mathematik in der Ökonomie abgeneigt ist.

Die Geburtsstunde der Neoklassik wird um das Jahr 1870 angesiedelt, obwohl bedeutende "`Vorläufer"' schon deutlich früher sehr ähnliche Konzepte vorweggenommen haben (vgl. Kapitel \ref{Vorläufer}). Interessant ist der "`Geburtsort"': Tatsächlich wurden sehr ähnliche Grundkonzepte in drei verschiedenen Städten entwickelt. An allen drei Orten entwickelte sich in der Folge eine rege wirtschaftswissenschaftliche Forschungstätigkeit und daraus entstanden drei verschiedene Schulen, die bis heute existieren, wenn sie sich auch in ganz unterschiedliche Richtungen entwickelt haben. In Wien entwickelte Carl Menger sein nicht quantitatives Konzept des Grenznutzens und bildete damit die Grundlage der bis heute existierenden "`Österreichischen Schule der Nationalökonomie"' (vgl. Kapitel \ref{Austria}). In Cambridge begründete Stanley Jevons die Theorie der subjektiven Wertlehre und begründete damit die "`Cambridge School"', die mit ihren späten Vertretern Alfred Marshall, Irving Fisher und Arthur Pigou die "`Vollendung der Neoklassik"' darstellt (vgl. Kapitel \ref{Neoklassik}. Die Lausanner Schule wurde von Leon Walras begründet. Dieser ist heute noch im Begriff des "`Walras-Gleichgewicht"' allgegenwärtig. Tatsächlich stellte er sich als erster die Frage ob alle einzelnen Märkte gemeinsam im Gleichgewicht sein können oder müssen. Eine Frage, die im 20. Jahrhundert stark beforscht wurde (vgl. Kapitel \ref{Arrow-Debreu}).

Das politische und wirtschaftliche Umfeld der 1870er-Jahre war geprägt von den späten Jahren der Monarchien und der langsamen Entstehung des Nationalismus in Zentraleuropa. Damit verbunden die späte Zeit des Kolonialismus und damit die Zeit des Hochimperialismus. 

HIER WEITER


