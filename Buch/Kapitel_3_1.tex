%%%%%%%%%%%%%%%%%%%%% chapter.tex %%%%%%%%%%%%%%%%%%%%%%%%%%%%%%%%%
%
% sample chapter
%
% Use this file as a template for your own input.
%
%%%%%%%%%%%%%%%%%%%%%%%% Springer-Verlag %%%%%%%%%%%%%%%%%%%%%%%%%%

\chapter{Keynesianismus: Nachfrage statt Angebot?}
\label{Keynes}

\textsc{John Maynard Keynes} war sicherlich der prägendste Ökonom des 20. Jahrhunderts. Sein Name ist über die Wirtschaftswissenschaften hinaus bekannt und wird im Begriff "`Keynesianismus"' häufig für jene ökonomische Schule verwendet, die nach 1945 für mehrere Jahrzehnte Volkswirtschaftstheorie und auch Wirtschaftspolitik prägte. Zumindest direkt nach dem Erscheinen seines Hauptwerkes -  "`The General Theory	of Employment, Interest and Money"' - wurde er fast uneingeschränkt gefeiert. Heute gelten seine Theorien als umstritten oder zumindest überholt. In der wirtschaftspolitischen Praxis sind jene Ansätze - die er als einer der ersten theoretisch beschrieben hat, und die bis heute nach ihm benannt werden - trotz aller formaler Bedenken allerdings weit verbreitet - vor allem im Fall von Wirtschaftskrisen. Das Auftreten der "`Great Recession"' nach 2008 machte dies deutlich. Bis heute wird die Mainstream-Ökonomie, sowie deren Modelle, als neu-\textit{keynesianisch} bezeichnet. Und noch heute müssen sich Top-Ökonomen häufig dazu bekennen, ob sie pro oder contra Keynes sind. Tatsächlich definierten sich alle führenden makroökonomischen Wirtschaftsschulen seit 1946 auch an deren Einstellung zu den keynesianischen Ideen.
Was macht Keynes zu \textit{der} allgegenwärtigen Figur in der Volkswirtschaftslehre? Nun, er gilt als Begründer der modernen Makroökonomie. Wobei dies in dem Sinne zu verstehen ist, dass er als erster eine völlig neue Sichtweise auf die Wirtschaft als Ganzes lieferte und auch eine wirtschaftspolitische Steuerung vorschlug. Er betrachtete Wirtschaft nicht mehr nur als Summe der Leistungen einzelner Individuen, sondern als Gesamtsystem in dem sich Angebots- und Nachfrageseite gegenseitig bedingen. Der Wirtschaftspolitik wurde damit eine völlig neue Bedeutung gegeben. Dies alleine kann seine einzigartige Stellung aber nicht rechtfertigen. Vielmehr deutete der Zeitgeist nach der "`Great Depression"' bereits an, dass Ökonomie neu zu denken sei. Keynes war in diesem Sinne eben nicht der einzige, der dies tat. Die Wirtschaftspolitik kannte die stimulierende Wirkung von Staatsausgaben schon vor 1936 \parencite{Fishback2010}. Es gab aber auch schon ähnliche \textit{theoretische} Ansätze, welche die Erkenntnisse Keynes' vorwegnahmen. Vor allem der polnische Ökonom Michal Kalecki arbeitete schon längere Zeit an einer solchen Theorie und publizierte diese \parencite{Kalecki1935} bereits vor Keynes auch in englischer Sprache. Diese wurde allerdings nur in Fachkreisen anerkennend aufgenommen. \textcite{Kahn1931} hatte schon deutlich zuvor den Multiplikator und dessen Wirkung vorgestellt und damit wichtige Vorarbeiten geliefert. Die beiden genießen aber bis heute nicht einmal einen Bruchteil des Ruhmes, den Keynes nach 1936 rasch erlangte. Noch dazu ist Keynes' Werk schwer verständlich und umständlich geschrieben: Ökonomen und Wirtschaftshistoriker sind sich recht einig über das Buch "`The General Theory"' selbst. \textcite [S. 31]{Mankiw2006} findet es gleichzeitig "`erheiternd und frustrierend"' und rät Studierenden davon ab das Werk zu lesen. Robert Lucas antwortete auf die Frage, ob man das Werk heute noch lesen soll kurz und knapp mit "`Nein"' \parencite{Lucas2013}. \textcite[S. 429]{Rosner2012} bezeichnet den Charakter des Buches als "`eigenartig"' und selbst einer der eifrigsten Verfechter der keynesianischen Ideen, Paul Samuelson, findet das Werk selbst "`schlecht geschrieben, ärmlich organisiert"' und zudem "`höchst verwirrend"' \parencite[S. 190]{Samuelson1946}. Und doch zweifelt keiner der genannten an der hohen Bedeutung des Werks. Keynes' einzigartige Stellung in der Ökonomie erscheint angesichts dieser Tatsachen aus heutiger Sicht dennoch überraschend. Vor allem, wenn man selbst einen Blick in das Buch wirft. Seine bahnbrechende Leistung soll und darf aber auch nicht geschmälert werden. Die "`General Theory"' besticht als Buch nicht als Gesamtwerk. Die unumstrittene Genialität des Werks liegt vielmehr in den Einzelideen die dort an verschiedenen Stellen vorgebracht werden und in ihrer Summe zur damaligen Zeit revolutionär waren. Außerdem ist es aus heutiger Sicht schwierig die Tragweite der Veröffentlichung der "`General Theory"' abzuschätzen, wie \textcite[S. 187]{Samuelson1946} früh erkannte: "`Es ist für heutige Studierende fast unmöglich den Einfluss der 'Keynesianischen Revolution' zu realisieren"'. Deren Inhalt ist nämlich spätestens nach dem Zweiten Weltkrieg in kürzester Zeit zum neuen Standard der Makroökonomie geworden. 

Das Bild des dominierenden Ökonomen, das von Keynes entstand, ist Resultat einer Kombination aus privilegierter Geburt und notwendigem Glück, aber vor allem Genialität gepaart mit einem enormen Tätigkeitsumfang. Zudem verfügte Keynes über ein gesundes Selbstvertrauen. Schon vor der Veröffentlichung seines Hauptwerkes schrieb er an den befreundeten Literaten George Bernard Shaw: "`I believe myself to be writing a book of economic theory which will largely revolutionize [...] the way the world thinks about economic problems"' \parencite[S.13]{Warsh}. 

Die Kombination aus seinem beruflichen Werdegang, seinem Lebensstil, sowie seinem relativ frühen Tod machten ihn später zu einer Art Mythos: Keynes wurde 1883 geboren. Seine Familienverhältnisse scheinen wie gemacht für seine spätere Karriere. Sein Vater, John Neville Keynes, war ebenfalls Ökonom und später einer der Gründungsväter des "`Economic Journals"', das bis heute eine der führenden Fachzeitschriften in den Wirtschaftswissenschaften ist. Seine Mutter war eine der ersten Student\textit{innen} in Cambridge und später Bürgermeisterin der Stadt. Sein Bruder Geoffrey wurde 1955 als berühmter Chirurg geadelt und seine Schwester Margret heiratete einen späteren Medizin-Nobelpreisträger \parencite[S. 275]{Scherf1989}. John Maynard Keynes studierte zunächst Mathematik und schien danach eine Art Beamtenlaufbahn einzuschlagen. Bei der Vorbereitung auf das "`Civil Service Exam"' genoss er den Unterricht der damals führenden Ökonomen Marshall und Pigou. Im Anschluss arbeitete er tatsächlich kurz, nämlich zwischen 1906 und 1908, bei einer Verwaltungsbehörde, dem India Office, in London, bevor er 1908 an die University of Cambridge zurückkehrte. Dort wurde er Lektor und erhielt die "`Fellowship"'-Auszeichnung, die dort bis heute für besondere wissenschaftliche Leistungen, vergeben wird, und zwar für seine frühen Arbeiten zur Wahrscheinlichkeitstheorie \parencite[S. 276]{Scherf1989}. Wirtschaftswissenschaftliche Artikel hatte er bis zu diesem Zeitpunkt noch keine publiziert, doch seine Leistungen, sein Engagement und seine Ausstrahlung müssen überdurchschnittlich gewesen sein, schließlich wurde er bereits 1911 Editor des Economic Journals. Während des Ersten Weltkrieges wechselte er ins englische Finanzministerium, wo er bereits 1917 zum Abteilungsleiter wurde. Ein - aus wissenschaftlicher Sicht - unscheinbarer, aber wohl entscheidender Karriere-Schritt. Denn als solcher nimmt er 1919 an den Friedensverhandlungen von Versailles teil. Dort stellt er sich bekanntermaßen gegen die - seiner Meinung nach nicht erfüllbaren - Reparationsforderungen gegenüber Deutschland. Er reist ab und verfasst innerhalb kurzer Zeit sein Werk "`The Economic Consequences of the Peace"' \parencite{Keynes1919}, das ihm bereits damals zu Weltruhm verhalf. 

In diese Zeit fällt auch sein so häufig diskutiertes Privatleben als Teil der Bloomsbury Group, einer Gruppe junger Intellektueller, Philosophen und Künstler. Unter anderem war auch Virginia Woolf Mitglied. Über seine Offenheit gegenüber sexuellen Beziehungen - Keynes hatte in dieser Zeit häufige, offen homosexuelle Beziehungen - wird bis heute geschrieben. Ab 1921 war er schließlich mit der russischen Ballerina Lydia Lopokova liiert, die er 1925 heiratete. Eine Zeitung schrieb damals "`Was there ever be such a union of beauty and brains?"' \parencite[S. 13]{Warsh}. Diese Tatsachen, dass die Hochzeit des damals schon bekannten Ökonomen ein gesellschaftliches Ereignis war, und dass bis heute über die sexuelle Orientierung eines Ökonomen geschrieben wird, zeigt seine Berühmtheit über die Wirtschaftswissenschaften hinaus und erinnert eher an einen Musik- oder Filmstar. Vor allem trug es aber wohl bei zum Bild des genialen, aber auch lebensfrohen Winner-Typen, das von John Maynard Keynes bis heute häufig gezeichnet wird.

Es ist interessant, dass der führende Ökonomie-Theoretiker des 20. Jahrhunderts bereits über 15 Jahre vor der Veröffentlichung seines bahnbrechenden Hauptwerkes den Bekanntheitsgrad eines Stars erreichte. Anstatt sich jetzt in seinem Ruhm zu sonnen und sich zurückzulehnen, beginnt nun eine Phase unglaublicher Produktivität. In Cambridge ist er nach wie vor als Editor des "`Economic Journals"' tätig, zusätzlich verwaltet er die Finanzen der Universität. Seine dabei erlangten Erfolge an den Kapitalmärkten trugen später auch zur Entstehung des Mythos um seine Person bei. Als Wissenschaftler veröffentlichte er zwischen 1920 und 1933 praktisch jedes Jahr ein ökonomisches Buch. Daneben ist er zu dieser Zeit gefragter Berater von Versicherungen, Investmentfonds und Politikern.

Gegen Ende der 1920er Jahre strebte Keynes immer mehr danach sein wissenschaftliches Gesamtwerk abzuschließen \parencite[S. 198]{Samuelson1946}, das darin bestehen sollte eine umfassende Geldtheorie vorzulegen. Dazu veröffentlichte er 1930 seine "`Treatise on Money"' \parencite{Keynes1930}. Aber schon bei der Veröffentlichung seines Werkes war er unzufrieden damit \parencite[S. 282]{Scherf1989},   \parencite[S. 198]{Samuelson1946}. Aus heutiger Sicht gesehen ist "`Treatise on Money"' nicht besonders bahnbrechend. Versetzt man sich zurück in das Erscheinungsjahr 1930 bricht es zwar mit dem damals dominierenden makroökonomischem State of the Art, dessen wichtigster Baustein die Quantitätstheorie des Geldes war, liefert aber keine umfassend neuen Antworten. Die "`Treatise on Money"' war ein erster Schritt aus dem herrschenden Dogma heraus \parencite[S. 282]{Scherf1989}, wenn auch kein hinreichend großer um bereits von einer ökonomischen Zeitenwende zu sprechen. Keynes verwendete darin schon sein Instrument von Kreislaufzusammenhängen. Konjunkturzyklen entstehen demnach aus der Differenz zwischen Produktionskosten und Marktpreis, die zu "`Übergewinnen"', oder "`Verlusten"' für die Unternehmen führt \parencite[S. 422]{Rosner2012}. Ist die Differenz Null so befindet sich die Ökonomie im Gleichgewicht, Sparen und Investitionen sind ident, der entsprechende Zinssatz ist jener, bei dem das Preisniveau konstant bleibt, wobei sich Keynes bereits auf den natürlichen Zinssatz von \textcite{Wicksel1898} (vgl. Kapitel \ref{Wicksell}) bezieht. Übersteigen die Produktionskosten die Marktpreise kommt es zu Einschränkungen der Produktion und damit bei \textcite{Keynes1930} zu Arbeitslosigkeit. Ein Senken des Zinssatzes fördert Investitionen und die Wirtschaft nähert sich wieder dem Gleichgewicht. Übersteigen die Preise die Produktionskosten, kommt es zu einem Anstieg der Investitionen \parencite[S. 282]{Scherf1989}. Ein Erhöhen des Zinssatzes fördert Sparen und die Wirtschaft nähert sich auch hier dem Gleichgewicht. Warum es zu Abweichungen vom Gleichgewicht kommt, also warum die Märkte nicht sofort durch Preisanpassungen jederzeit vollständig geräumt werden, behandelt \textcite{Keynes1930} nicht. Der zweite Teil des Werks analysierte das Finanzsystem und dessen Wirkung auf die Realwirtschaft. Damit betrat \textcite{Keynes1930} zwar akademisches Neuland, aber revolutionärer Durchbruch waren auch diese Ansätze nicht. \textcite{Scherf1989} bezeichnete die "`Treatise"' als "`Standardwerk der Geldlehre"', \textcite[S. 198]{Samuelson1946} als "`bedeutend, allerdings wenig aufregend"' und seine Gleichungen als "`Umweg und Sackgasse"'\footnote{Der später häufig als Duell auf Augenhöhe hochstilisierte wissenschaftliche Disput zwischen John Maynard Keynes und Friedrich August Hayek bezieht sich am ehesten auf die Zeit um 1930. \textcite{Keynes1930}' "`Treatise on Money"' stellte \textcite{Hayek1931} "`Preise und Produktion"' gegenüber. Deren Diskussionen zur Geld- und Konjunkturtheorie fanden tatsächlich auf Augenhöhe statt. Die "`General Theory"' \parencite{Keynes1936} wurde zwar von Hayek von Beginn an abgelehnt. Ein wissenschaftliches Gegenmodell, das eine ähnliche Bedeutung wie die "`General Theory"' erlangte, konnte er aber nicht liefern. Sein Werk \textcite{Hayek1944} "`Der Weg zur Knechtschaft"' hat aus wirtschaftstheoretischer Sicht nicht den gleichen Stellenwert wie die "`General Theory"'. Die Theorie-Mängel des Keynesianismus deckten erst später \textcite{Friedman1963} und vor allem \textcite{Lucas1976} auf}.

Hätte Keynes zu diesem Zeitpunkt seine wissenschaftliche Tätigkeit eingestellt, würde er aus heutiger Sicht unzweifelhaft als wenig bedeutender Ökonom eingestuft. Dies, obwohl er damals, im Jahr 1930, schon lange international bekannt war. Erst in weiterer Folge, im Jahr 1936, erschien schließlich die "`General Theory"'. Dieses Werk ist unzweifelhaft das einflussreichste ökonomische Buch des 20. Jahrhunderts! Es war der alleinige Auslöser der "`Keynesianischen Revolution"'. Die Paradoxien und Eigenheiten, die darin zu finden sind, wurden weiter oben schon beschrieben. Unabhängig davon wird es nach wie vor häufig diskutiert. Der Versuch sämtliche unterschiedliche Untersuchungen oder Interpretationen des Werks auch nur anzuführen, müsste scheitern. \textcite[S. 38ff]{Weintraub1979} benannte ein Kapitel seines Buches scherzeshalber "`The $4,827^{th}$ re-examination of Keynes's system"'.

Was ist nun der Inhalt von "`The General Theory of Employment, Interest and Money"'? Es wurde bereits dargelegt, dass dieses Werk auf so unterschiedliche Art und Weise ausgelegt werden kann. Das liegt auch daran, dass die "`General Theory"' selbst einen Recht niedrigen Grad der Formalisierung aufweist. Insgesamt finden sich im Buch verhältnismäßig wenige Formeln und diese eher punktuell. Dies ist einigermaßen überraschend. Ging doch der Trend gerade zu dieser Zeit in Richtung höherer Formalisierung (vgl. Kapitel \ref{Neoklassik_nach1945}), außerdem war Keynes ja studierter Mathematiker. Dennoch sprach er sich explizit gegen eine zu hohe Formalisierung der Wirtschaftswissenschaften aus. Ein formalisiertes Gesamtmodell wird also in der "`General Theory"' auf jeden Fall nicht geliefert. Keynes revolutioniert die Denkweise der Ökonomie, ohne dabei aber den Fehler zu machen den bisherigen State of the Art komplett über den Haufen werfen zu wollen. So kann seine Theorie auch als Ungleichgewichtstheorie verstanden werden, dennoch bleibt er auf den etablierten Wegen von Marshall, seines großen Lehrers in Cambridge: der komparativ-statischen Analyse. Er ändert allerdings den Ausgangspunkt bei der Betrachtung ökonomischer Zusammenhänge. Dieser ist nämlich bei Keynes das Nationaleinkommen, welches stets mit der Gesamtbeschäftigung eng zusammenhängt. Entscheidend und neu ist der Begriff der "`Effektiven (Gesamt)-Nachfrage"', die in Kapitel 3 von \textcite[S. 25]{Keynes1936} entwickelt wurde. Dies ist jener Punkt, an dem die aggregierte Nachfragefunktion die aggregierte Angebotsfunktion schneidet. Oder mit anderen Worten: Das ist die Gesamtnachfrage, die am Markt realisiert wird. Mit der Einführung der Effektiven Nachfrage greift Keynes auch früh das "`Say'sche Gesetz"' an: Wenn das Angebot die Nachfrage bestimmt, gibt es keinen Grund für die Unternehmen die Produktion nicht bis an jenen Punkt auszuweiten, an dem schlicht keine Arbeitskräfte mehr verfügbar sind. Demnach steht einer Vollbeschäftigung nichts im Wege \parencite[S. 26]{Keynes1936}. Erst wenn die aggregierte Nachfrage entscheidend ist für die Produktion, also das aggregierte Angebot, sind Situationen denkbar, in denen die Gesamtwirtschaft unter stabiler Unterbeschäftigung leidet. Keynes dreht das "`Say'sche Gesetz"' um, die Gültigkeit seiner Theorie verlangt, dass die Nachfrage die Produktion bestimmt. Das ist die Grundlage für die später daraus abgeleiteten Empfehlungen für "`Nachfrageorientierte Wirtschaftspolitik"'. \textit{Der} Bruch mit der neoklassischen Mainstream-Ökonomie war die Annahme, dass die Preise - und damit auch die Löhne - gegeben sind\parencite[S. 58]{Snowdon2005}. Anpassungen erfolgen damit über die gehandelten Mengen. Dies macht die Theorie zu einer der kurzen Frist. Die Höhe der Nachfrage nach Arbeit - also die Beschäftigung - hängt damit direkt von der Effektiven Nachfrage ab.

Die Beschäftigungstheorie (Theorie zur Arbeitslosigkeit) von Keynes ist die erste bahnbrechende Erkenntnis, die wir aus der "`General Theory"' näher betrachten (In \textcite{Keynes1936} selbst ist die Beschäftigungstheorie am Ende des Buches angesiedelt). In der (Neo-)Klassik gibt es das Phänomen der Arbeitslosigkeit nicht wirklich. Funktionierende Märkte sorgen demnach immer für eine Bereinigung der Märkte - auch des Arbeitsmarktes. Damit kann unfreiwillige Arbeitslosigkeit nicht existieren. Dies war natürlich auch schon vor der "`Great Depression"' empirisch nicht zu halten und wurde während dieser zu blankem Hohn. Tatsächlich lieferte \textcite{Pigou1933} mitten in der "`Great Depression"' die erste umfassende Beschäftigungstheorie. Demnach widerspräche die beobachtbare Arbeitslosigkeit nicht grundsätzlich der klassischen Theorie. Er ging stattdessen davon aus, dass eine veränderte Nachfrage nach Arbeitskräften, sowie zu hohe Löhne dazu führten, dass es Arbeitslosigkeit gäbe, die aber nur von kurzfristiger Dauer sein könne (vgl. Kapitel \ref{sec: Pigou}). Das Problem zu hoher Reallöhne kannten die (Neo-)Klassiker bereits. Als einziges Mittel dagegen galten Lohnsenkungen. \textcite{Keynes1936} stellte dem seine fundamentale Beschäftigungstheorie entgegen. Zunächst akzeptiert \textcite[S. 258f]{Keynes1936} die Möglichkeit, dass durch Nominallohn-Kürzungen die Produktion angeregt wird und dadurch positive gesamtwirtschaftliche Effekte entstehen. Er schwenkt aber schnell dazu über, dass es ein Trugschluss sein muss, wenn man davon ausgeht, dass eine allgemeine Kürzung der Nominallöhne mit konstanter aggregierter Effektiver Nachfrage einhergehen würden. Aus heutiger Sicht klingt das vollkommen einleuchtend: Wenn man alle Löhne und Gehälter kürzt, muss man davon ausgehen, dass die Bezieher dieser ihre Konsumausgaben auch zurückfahren müssen \parencite[S. 269]{Keynes1936}. Aber den (Neo-)Klassikern blieb vor Keynes gar nichts anderes übrig als davon auszugehen, dass die Nachfrage von der Lohnhöhe unabhängig sei. Schließlich waren diese makroökonomisch in der Quantitätstheorie "`gefangen"' und darin ergibt sich die Gesamtnachfrage aus dem Produkt der Geldmenge und deren Umlaufgeschwindigkeit (vgl. Kapitel \ref{FisherandClark}). Eventuelle Lohnrückgänge müssten demnach durch entsprechende Gewinnsteigerungen ausgeglichen werden, wenn der Weg zurück zum Gleichgewicht dies erforderte. \textcite{Keynes1936} brach hier ganz ausdrücklich und durchaus verbal aggressiv aus den alten makroökonomischen Vorstellungen aus. Was schon vor Keynes bekannt war, ist die Tatsache, dass durch rigide Nominallöhne - also solche, die nach unten nicht flexibel sind - steigende Reallöhne entstehen können. Nämlich dann wenn das Preisniveau fällt, also Deflation herrscht. Dadurch sind die Reallöhne zu hoch und unfreiwillige Arbeitslosigkeit ist die Folge\footnote{\textcite{Keynes1936} behandelt den möglichen positiven Effekt steigender Reallöhne auf die Gesamtnachfrage nicht. Dies blieb später \textcite{Pigou1943} vorbehalten (vgl. Kapitel \ref{sec: Pigou}).}. \textcite{Keynes1936} akzeptiert diesen Effekt. Er postuliert aber auch, dass sinkende Nominallöhne die Gesamtnachfrage so weit fallen lassen können, dass dadurch wiederum das Preisniveau soweit fällt, dass insgesamt sogar steigende Reallöhne realisiert werden. Er plädiert daher dafür nicht die Nominallöhne zu senken, sondern das Preisniveau zu erhöhen. Dazu empfahl er allerdings nicht primär nur geldpolitische Maßnahmen, sondern eben fiskalpolitische. Durch eine Ausweitung der Aggregierten Nachfrage, durch staatliche Ausgaben, kommt es zu steigenden Preisen und damit sinkenden Reallöhnen und in weiterer Folge zu steigender Beschäftigung.

\textcite{Keynes1936} diskutierte auch die Möglichkeiten mit sinkenden Nominallöhnen wieder ein Vollbeschäftigungs-Gleichgewicht am Arbeitsmarkt herzustellen, lehnte diese aber durchwegs alle ab. Erstens, aus rein praktischen Gründen: Sinkende Nominallöhne sind in liberalen Demokratien schwer durchsetzbar - das klassische Problem rigider Löhne. Sinkende Reallöhne durch Inflation treffen hingegen auf weit weniger Widerstand wie schon \textcite[S. 14]{Keynes1936} postulierte. Er selbst hatte also - noch lange vor der Diskussion über die Gültigkeit der Phillips-Kurve (vgl. Kapitel \ref{sec: Phillips}) - wenig Berührungsängste mit mäßiger Inflation. Zweitens führte Keynes theoretische Gründe gegen Nominallohn-Senkungen an. Theoretisch, so Keynes, könnten diese zwar positive Effekte haben, indem der folgende Rückgang der aggregierten Nachfrage bei konstanter Geldmenge zu sinkenden Real-Zinsen führt. Dadurch werden Investitionen angeregt, die wiederum einen stimulierenden Effekt auf die Nachfrage haben und so schlussendlich zum Gleichgewicht zurück führen. Dieser Mechanismus wird heute Keynes-Effekt genannt, obwohl er selbst an dessen Wirkung zweifelte. Und zwar aus zwei Gründen. Erstens, wenn der nominale Zinssatz bereits bei Null ist, kann der Real-Zins nicht mehr fallen und Investitionen werden nicht angeregt. Eine Situation, die bis vor Kurzem als "`Liquiditätsfalle"' in aller Munde war. \textcite[S. 207]{Keynes1936} beschreibt diese Situation als "`für die Zukunft praktisch wichtig"' und, dass es bis heute (also 1936) kein Beispiel für so eine Situation gegeben habe. Zweitens, könnten Unternehmer, zum Beispiel aufgrund negativer Zukunftsaussichten, selbst dann nicht investieren wollen, wenn der Realzinssatz fällt. Dies wird heute als Investitionsfalle bezeichnet.

Keynes sah also die Gefahr einer stabilen Unterauslastung der Ökonomie und forderte aktive Wirtschaftspolitik im Sinne von Fiskalpolitik \textit{und} Geldpolitik um dieser Situation zu entkommen, wenn sie eintritt. Das war zu seiner Zeit revolutionär. (Neo-)klassische Ökonomen tendierten dazu Markteingriffe abzulehnen, zumindest dann wenn keine Marktunvollkommenheiten wie Monopole auftraten. Keynes akzeptierte interessanterweise die Annahme perfekter Märkte, argumentierte nun aber dennoch für aktive Wirtschaftspolitik, indem er Situationen aufzeigte, in denen ein Gleichgewicht trotz Unterbeschäftigung vorliegt. Dieser Zugang war natürlich im Angesicht der 1936 gerade überwundenen "`Great Depression"' unter sehr speziellen ökonomischen Umständen entstanden, oder wie \textcite[S. 199]{Samuelson1946} es ausdrückte: "`While Keynes did much for the Great Depression, it is no less true that the Great Depression did much for him"'. Trotzdem ist dieses Plädoyer für aktive Wirtschaftspolitik unerwartet. Wurde als ideologisch-ökonomisch größte Gefahr zu jener Zeit doch die Planwirtschaft in Form des real existierenden Sozialismus gesehen, der in den 1930er Jahren durchaus Erfolge verbuchen konnte. In Wirklichkeit aber wollte Keynes den Kapitalismus nicht an die Planwirtschaft annähern, sondern eben verhindern, dass diese für westeuropäische Bürger attraktiv wird.

Die zweite bahnbrechende Erkenntnis der "`General Theory"' ist die Funktionsweise \textit{wie} das Prinzip der Effektiven Nachfrage auf die Gesamtwirtschaft wirkt. Im Gleichgewicht der Effektiven Nachfrage entspricht die aggregierte Nachfrage $D$ dem aggregierten Angebot (Produktion $Z$) und damit dem Gesamteinkommen $Y$. Die aggregierte Nachfrage setzt sich zusammen aus dem Konsum der Haushalte $C$ und aus den Investitionen der Unternehmen $I$. Außerdem zeigt er die Identität von Ersparnis $S$ und Investition aus den Definitionen der beiden Begriffe heraus \parencite[S. 63]{Keynes1936}. Diese Gleichheit von Ersparnis und Investition war auch zu Keynes' Zeiten nicht unumstritten. Die Klassiker (vgl. Kapitel \ref{Klassik}) gingen bereits von der Richtigkeit der Annahme aus, \textcite{Keynes1930} selbst und andere Autoren hatten die ständige Gültigkeit allerdings angezweifelt, was er in der General Theory \parencite[S. 74ff]{Keynes1936} dadurch rechtfertigte, dass er ursprünglich von einer anderen Definition der Begriffe ausgegangen ist.

Der Konsum ist für Keynes eine Funktion des Einkommens. In den Kapiteln acht und neun verweist Keynes zunächst auf die überragende Bedeutung der aggregierten Nachfrage auf die Beschäftigung. Diese ergibt sich aus der Summe aus Konsum $C$ und Investitionen $I$. Das Nationaleinkommen (=Aggregierte Nachfrage im Gleichgewicht) und der aggregierte Konsum hängen damit wechselseitig voneinander ab, was später bei der Anwendung des Multiplikators wichtig ist. Konsum und Investitionen kommt daher die entscheidende Bedeutung in der Keynesianischen Theorie zu. Für Keynes ist unumstritten, dass der \textit{aktuelle} (also der derzeitige) Konsum vom \textit{aktuellen} Einkommen abhängt. Diese heute als "`Absolute Einkommenshypothese"' bezeichnete Annahme, galt bald nach dem Zweiten Weltkrieg als umstritten und hat später zu verschiedenen "`Einkommenshypothesen"' und unterschiedlichen Interpretationen geführt, zum Beispiel durch Franco Modigliani (vgl. Kapitel \ref{Synthese}) und Milton Friedman (vgl. Kapitel \ref{Monetarismus}). Viel wichtiger ist für \textcite[S. 89, S. 107]{Keynes1936} die Konsum\textit{neigung}. Das ist der Anteil am Einkommen, den die Individuen konsumieren. Der Rest wird gespart, was durch die oben beschriebene Identität wiederum den Investitionsanteil bestimmt. Das Verhältnis aus Konsum und Investitionen leitet direkt über zum wohl bekanntesten Teil der "`General Theory"' für die spätere wirtschaftspolitische Anwendung: Den "`Multiplikator-Effekt"'. Das zehnte Kapitel der "`General Theory"' widmet sich diesem Thema. Wie \textcite[S. 114]{Keynes1936} selbst beschreibt, war es Richard \textcite{Kahn1931}, der den "`Multiplikator"' erstmals beschrieb, und zwar im Sinne des Zusammenhangs zwischen (staatlichen) Investitionen und Arbeitslosigkeit. Keynes zeigte, dass die aktuelle Beschäftigung vom Ausmaß der aktuellen Effektiven Nachfrage abhängt. Eine Steigerung der Beschäftigung kann nur durch zusätzliche Investitionen erreicht werden, oder wenn die Individuen einen größeren Anteil ihres Einkommens für Konsum ausgeben.  Der Anteil, den Personen ausgeben, wenn sie ein zusätzliches Einkommen generieren, nennt Keynes die "`Marginale Konsumneigung"'. Wenn eine Person also zusätzlich 100GE verdient und davon 800GE ausgibt, hat sie eine Marginale Konsumneigung von 80\%. Der zweite Begriff sind die Investitionen. Betrachten wir zunächst nur die \textit{Auswirkung} zusätzlicher solcher im Zusammenhang mit der marginalen Konsumneigung. Angenommen es werden Investitionen in Höhe von 1.000.000GE getätigt, dann wird auf verschiedenen Märkten eine Nachfrage in eben dieser Höhe getätigt. Diese zusätzliche aggregierte Nachfrage erhöht die Beschäftigung, sowie die Einkommen, die wiederum den Konsum bestimmen. Bei einer angenommenen marginalen Konsumneigung von 80\%, werden in weiterer Folge 800.000GE als Konsum getätigt, die wiederum auf verschiedenen Märkten für zusätzliche Nachfrage sorgen. Dies erhöht die Beschäftigung, sowie die Einkommen, die wiederum den Konsum bestimmen. Bei einer angenommenen marginalen Konsumneigung von 80\%, werden in weiterer Folge 640.000GE... Man sieht, man kann diesen Satz beliebig wiederholen. Abgekürzt: Zusätzliche Investitionen wirken über Kreislaufeffekte überproportional auf das Gesamteinkommen und damit die Beschäftigung. Diesen Effekt nennt man den Multiplikatoreffekt \parencite[S. 115]{Keynes1936}. Mathematisch ist dies eine sogenannte geometrische Reihe, wobei das Ausmaß des Multiplikators eben von der Marginalen Konsumquote abhängt. Je höher diese ist, desto höher der Multiplikator\footnote{Umgekehrt gilt daher: Je höher die Marginale Sparquote, desto geringer der Multiplikator.}. Bei einer angenommenen Marginalen Konsumquote von 0,8 ergibt sich: $$\frac{1}{1-0,8} = 5$$ \textcite[S. 128]{Keynes1936} selbst schätzte Multiplikator-Werte von ungefähr 2,5 als realistisch ein. Der Multiplikator-Effekt wurde so berühmt, dass er praktisch als gleichbedeutend mit Keynesianischer Wirtschaftspolitik angesehen wurde. Politiker in den 1960er und 1970er-Jahren hatten in ihm unzweifelhaft ihr wirtschaftspolitisches Lieblingsspielzeug entdeckt. \textcite[S. 118ff]{Keynes1936} selbst war sich übrigens schon bewusst, dass der Fiskalpolitik - also der Umsetzung staatlicher Investitionen mit dem Zweck über den Multiplikator-Effekt Einkommen und Beschäftigung anzukurbeln - Grenzen gesetzt sind. Sollte der Zustand der Vollbeschäftigung erreicht sein, wäre "`wahre Inflation"' die Folge von Fiskalpolitik, aber auch Crowding-out Effekte - also das Verdrängen privater Investitionen durch die staatliche Investitionen - thematisierte er bereits.

Nun betrachten wir die Investitionen selbst im Detail. Wir haben gerade gesehen, dass das Nationaleinkommen und damit die Beschäftigung stark von diesen abhängen. Investitionen der öffentlichen Hand sind exogen gegeben und werden nicht näher hinterfragt. Aber der "`Normalfall"' sind ja Investitionen der Unternehmen zur Erhaltung oder Erweiterung ihres Kapitalstockes. Bei Keynes hängen die Investitionen vom Zinssatz, aber auch von den "`erwarteten Renditen"' ab. Diese nennt \textcite[S. 135]{Keynes1936} die "`Grenzleistungsfähigkeit des Kapitals"' ("`Marginal Efficiency of Capital"'). Die Grundüberlegung ist einfach: Ein Unternehmen muss ständig in neue Maschinen (etc.) investieren um seiner Geschäftstätigkeit langfristig nachgehen zu können. Eine Investition in neue Maschine rentiert sich aber nur, wenn mit dieser mehr Gewinn gemacht werden kann, als das dafür notwendige Kapital auf einem Sparkonto an Zinsen abwerfen würde. Ob sich eine bestimmte neue Maschine rentiert hängt für ein einzelnes Unternehmen damit von vielen gesamtwirtschaftlichen und einzelwirtschaftlichen Faktoren ab. Die damit produzierten Güter werden erst in den folgenden Jahren verkauft, die Nachfrage nach diesen kann in ihrer Gesamtheit in keiner Weise vorhergesagt werden. Keynes hatte bekanntermaßen einen intensiven aber aus heutiger Sicht unkonventionellen Zugang zu Risiko und Unsicherheit, schließlich beschäftigte er sich in seiner Dissertation mit dem Thema der Wahrscheinlichkeit und veröffentlichte später daraus sein Werk \textcite{Keynes1921}: "`A Treatise on Probability"'. Wie in Kapitel \ref{Finance} zu sehen sein wird, entstand die Finanzierungstheorie und damit die moderne Modellierung von Risiko im Wesentlichen erst ab den 1950er-Jahren. \textcite{Keynes1936} vertrat zu seiner Zeit den Ansatz, dass die zukünftige Rentabilität von Investitionen nicht zuverlässig prognostiziert oder modelliert werden könne. Solchen Prognosen ist ein hohes Maß an Unsicherheit in Form von stark schwankenden Erwartungen immanent. Die heute übliche Berechnung von Erwartungswerten als Produkt aus Eintrittswahrscheinlichkeiten und möglicher Ertrags-Ergebnisse war für ihn ein falscher Ansatz. Stattdessen spielten Psychologie und Herdenverhalten eine große Rolle. Zusammengefasst wird diese Einstellung mit dem heute noch geläufigen Begriff der "`Animal Spirits"' \parencite[S. 161f]{Keynes1936}. Insgesamt waren für Keynes die Erwartungen zu zukünftigen Entwicklungen \textbf{der} entscheidende Einflussfaktor auf die Investitionen. Die hohe Unsicherheit mit der Zukunfts-Erwartungen verbunden sind, führen zu Schwankungen bei den privaten Investitionen und in weiterer Folge zu Konjunkturschwankungen. 

Natürlich behandelte Keynes dennoch die Möglichkeit Investitionen insgesamt attraktiver zu machen, indem der Zinssatz reduziert wird. Allerdings war dies für ihn im Vergleich zu den hohen Schwankungsbreiten der Renditeerwartungen ein in der Wirkung recht beschränktes Mittel. Nachdem die Investitionen für Nationaleinkommen und Beschäftigung so entscheidend sind, spielen Zukunftserwartungen eine große Rolle bei der Analyse von Konjunkturschwankungen \parencite[S. 313]{Keynes1936}. Das heißt aber auch, dass die primären Quellen von Konjunkturschwankungen in den Realmärkten zu finden sind, nicht auf den Finanzmärkten. Insgesamt spielt der Zinssatz bei Keynes eine ganz andere Rolle als bei den (Neo-)Klassikern. Bei denen wird dieser als Grenzprodukt des Kapitals (vgl. Kapitel \ref{FisherandClark}) auf den Real-Märkten bestimmt. Oder, wie bei \textcite{Fisher1930}, alleine als Belohnung für sofortigen Konsumverzicht (vgl. Kapitel \ref{Finance}). \textcite[S. 165]{Keynes1936} unterscheidet, wie bereits erwähnt, zwischen Grenzleistungsfähigkeit des Kapitals, als erwartete Rendite, auf der einen Seite, und dem Zinssatz auf der anderen Seite. Letzterer hängt zunächst - ähnlich wie bei \textcite{Fisher1930} - grundsätzlich von der Konsumneigung \parencite[S. 166]{Keynes1936} ab, aber in weiterer Folge letztlich entscheidend davon, wie das gesparte Kapital gehalten wird. Wird Geld nämlich in bar gehalten und nicht zum Zinssatz angelegt, werden gar keine Zinsen lukriert. Der Zins als Belohnung für Konsumverzicht ist in diesem Fall keine hinreichende Erklärung, weil bei Bargeldhaltung sowohl auf Konsum als auch auf Zinszahlungen verzichtet wird. \textcite[S. 194]{Keynes1936} bringt auch hier wieder psychologische Gründe ins Spiel und entwickelt den Terminus "`Liquiditätspräferenz"': Aus Spekulationsgründen oder als Vorsichtsmaßnahme können Individuen Bargeldhaltung der Veranlagung vorziehen. Diese Liquiditätspräferenz bestimmt letztendlich die Nachfrage nach Geld und damit den Zinssatz. Der Zins wird somit alleine auf den Finanzmärkten bestimmt \parencite[S. 62]{Snowdon2005}. Und für Keynes ist die Liquiditätspräferenz von weit größerem Einfluss auf den Zinssatz als die Konsumneigung, also die Entscheidung welcher Anteil von zusätzlichem Einkommen gespart werden soll \parencite[S. 63]{Snowdon2005}. Vor allem die "`psychologischen Anreize"' zur Haltung von Liquidität, denen \textcite[S. 194ff]{Keynes1936} ein halbes Kapitel widmet, machen den Zinssatz bei Keynes wenig greifbar. \textcite[S. 202]{Keynes1936} selbst meint, dass "`es [...] klar ist, dass der Zinssatz eine in hohem Grad psychologische Erscheinung ist"'. Wieder spielen die Erwartungen eine große Rolle. Und deren Eigenschaft unsicher zu sein, führt wieder dazu, dass die Liquiditätspräferenz und damit die Geldnachfrage stark schwankt. Auch auf diesem Weg greift Keynes damit die Quantitätsgleichung des Geldes an. Eine schwankende Geldnachfrage ist gleichbedeutend mit einer schwankenden Umlaufgeschwindigkeit des Geldes \parencite[S. 201]{Keynes1936}. Damit beschreibt Keynes die Bedeutung der Geldpolitik, welche in der Rezeption seines Werkes im Vergleich zur Fiskalpolitik zu Unrecht wenig Beachtung findet. Für Keynes sorgen unsichere Zukunftserwartungen auch für eine höhere Liquiditätspräferenz, damit für eine geringere Geldnachfrage und somit weniger Investitionen. Aktive Geldpolitik, also eine Steuerung des Geldangebots, kann dies ausgleichen. Damit ist die Geld bei Keynes nicht neutral, wie bei den (Neo-)Klassikern. Stattdessen kann die Geldmenge reale Effekte haben. Wird der Zinssatz verringert, steigt die Nachfrage nach Geld und in weiterer Folge steigen die Investitionen, die über den bereits beschriebenen Multiplikatoreffekt das Nationaleinkommen und die Beschäftigung erhöhen.

Keynes hält also sowohl Fiskal- als auch Geldpolitik für wichtige Stellschrauben der Wirtschaftspolitik. Bezüglich Geldpolitik schränkt er allerdings ein, dass erstens, die Unsicherheit im Krisenfall so hoch werden kann, dass kein noch so niedriger Zinssatz die Unternehmer zu neuen Investitionen motivieren kann \parencite[S. 208]{Keynes1936} - heute spricht man von Investitionsfalle. Zweitens beschreibt \textcite[S. 207]{Keynes1936} die Möglichkeit, dass der Zinssatz auf so niedriges Niveau gefallen ist, dass Bargeldhaltung kein Nachteil gegenüber Geldanlage ist, weil es beiden Fällen keine Zinszahlungen gibt - heute spricht man von Liquiditätsfalle. In beiden Fällen ist Geldpolitik dann weitgehend wirkungslos, Fiskalpolitik hingegen hoch-wirksam. Vermutlich wurde daraus die oftmals zu findende Story der allgemeinen Präferenz der Keynesianer von Fiskalpolitik gegenüber Geldpolitik abgeleitet.

Was ist das Fazit aus der Theorie von Keynes? Wir haben gesehen, dass der Multiplikator Investitionsausgaben hebelt und somit aggregierten Konsum und Nationaleinkommen beeinflusst. Keynes argumentierte, dass eine Ökonomie ständig droht nicht voll ausgelastet zu sein, also keine Vollbeschäftigung aufzuweisen. Den Grund hierfür sieht Keynes in der Tendenz, dass die Individuen aufgrund von Unsicherheiten ständig eine zu hohe Sparneigung aufweisen. Die selben Unsicherheiten führen zu zu niedrigen Investitionsanreizen \parencite[S. 63]{Snowdon2005}. Das Ergebnis ist eine instabile Ökonomie, die vom Vollbeschäftigungs-Gleichgewicht ständig wegdriftet. Hier muss man konkretisieren: Hohe Investitionen wirken also positiv auf das Nationaleinkommen. Sparen hingegen verringert den Konsum und wirkt daher negativ auf das Nationaleinkommen. Gleichzeitig gilt aber ständig die Identität aus Sparen gleich Investitionen. Auf den ersten Blick ist dies ein Widerspruch. Diesen aufzulösen gelingt nur mit einem Zahlenbeispiel.

Bei Keynes gilt: Das Nationaleinkommen $Y$ setzt sich zusammen aus dem aggregiertem Konsum $C$ und den Investitionen $I$. Außerdem gilt stets die Identität: Sparen $S$ ist gleich Investieren $I$. Der Multiplikator $\kappa$ ergibt sich aus der Marginalen Konsumneigung $c$. Der Anteil am Einkommen, der nicht konsumiert wird ist Sparen. Die marginale Sparquote ist daher $1-c$.

Nehmen wir folgendes an:\\
$Y = 800GE$\\
$C = 600GE$\\
$I = S = 200GE$\\
$c = 0,75 \rightarrow \kappa = 4$\\

Angenommen die Individuen in der Ökonomie wollen nun mehr sparen, wie dies von Keynes befürchtet wurde. Zum Beispiel: zusätzlich $50GE$. Mehr Sparen bedeutet weniger Konsum im selben Ausmaß. Über den Multiplikator werden Änderungen beim Konsum "`gehebelt"'. Die Auswirkung des um $50GE$ niedrigeren Konsum's ist daher ein um $50GE * 4 = 200GE$ niedrigeres Nationaleinkommen $Y$. Das neue Gleichgewicht ist also: $Y = 600GE$ Die marginale Konsumquote hat sich nicht geändert. Das heißt $c = 0,75$ oder 75\% des Nationaleinkommens liefert der Konsum: $C = 450GE$, 25\% die Investitionen: $I = 150GE$. Das höhere Sparen führt paradoxerweise über den (negativen) Multiplikator-Effekt zu niedrigerem Nationaleinkommen. Bei konstantem Verhältnis aus Konsumanteil und Sparanteil, führt dies dazu, dass der Sparbetrag in Absolutzahlen in der Gesamtwirtschaft sogar sinkt und zwar genau um den Betrag des ursprünglich geplanten \textit{höheren} Sparbetrags. Dieses Phänomen ist in der Literatur als Sparparadoxon bekannt. Im Ergebnis bleibt aber dadurch auch die Identität aus Sparen ist gleich Investition gewahrt. Bei Keynes passen sich die Ersparnisse an die Investitionen also über Änderungen des Nationaleinkommens an. Die "`zu hohe Sparneigung"' bei gleichzeitig "`zu niedrigem Investitionsanreiz"' ist daher kein Widerspruch. 

Wie bereits erwähnt hatte die "`General Theory"' nach ihrer Veröffentlichung im Jahre 1936 fast augenblicklich eine enorme Wirkung auf die Volkswirtschaftslehre. Vor allem junge, amerikanische Ökonomen verhalfen der Theorie zu einer raschen Verbreitung. Seine gleichaltrigen Zeitgenossen hingegen lehnten die "`General Theory"' weitgehend ab. Arthur Pigou zum Beispiel, war einer der direkt angesprochenen Adressaten, der in der "`General Theory"' schlecht wegkam. Erst spät erkannte er die Bedeutung von Keynes' Werk an. August Friedrich Hayek war in den frühen 1930er Jahren der intellektuelle Gegenspieler und einer der führenden Ökonomen an der London School of Economics. Obwohl persönlich wertschätzend, lehnte er Keynes' Ideen zeitlebens strikt ab. Auf die "`General Theory"' von Keynes liefert er keine wirklich wirtschaftstheoretische Antwort. Er blieb zwar Zeit seines langen Lebens Ikone der Anhänger extrem liberaler Theorien, er ist aber für die Zeit nach 1936 außerhalb der Mainstream-Ökonomie anzusiedeln. Joseph Schumpeter schließlich gehörte in Harvard zu jener Gruppe von Ökonomen, die nicht viel von den neuen keynesianischen Ideen anfangen konnten. Heute gilt er als einer der wichtigsten Ökonomen des 20. Jahrhunderts. Zumindest gegen Ende seiner wissenschaftlichen Karriere blieb sein Werk allerdings weitgehend unbeachtet, vor allem sein monumentales Werk "`Business Cycles"' aus dem Jahr 1939. Durch die keynesianischen Ideen schien die Konjunktur steuerbar geworden zu sein. Langfristige Konjunkturzyklen schienen damit aus der Zeit gefallen. Man kann durchaus sagen, dass die drei genannten Ökonomen unter dem Erfolg von Keynes geradezu litten.

Neben Keynes' inhaltlich revolutionären Ideen sorgten auch zwei eher banale Gründe für die zunehmende Mystifizierung der "`General Theory"'. Erstens, der niedrige Formalisierungsgrad der Arbeit war entgegengesetzt zum damaligen Zeitgeist und ließ entsprechend viel Platz für unterschiedliche Auslegungen der Theorie. Es ist wenig überraschend, dass die folgenden Interpretation des Werks, zumindest großteils, in Form von mathematischen Modellen erfolgte. Damit verbunden ist aber immer ein gewisser Grad an notwendiger Abstraktion. Ein Teil der Theorie von Keynes wurde daher übernommen, während andere Teile unbeachtet blieben. Das sorgte für unterschiedlichste Forschungsarbeiten, dies sich allerdings alle auf die "`General Theorie"' beriefen. Zweitens, äußerte sich Keynes selbst zu den verschiedenen Interpretationen seiner Arbeit kaum. Mit dem Artikel \textcite{Keynes1937} antwortete er zwar noch einmal auf davor erfolgte Rezeptionen der "`General Theory"', aber insbesondere die enorm einflussreiche Interpretation, die sein Landsmann \textcite{Hicks1937} (vgl. Kapitel \ref{Synthese}) bereits im Folgejahr des Erscheinens der "`General Theory"' vornahm, blieb von Keynes völlig unkommentiert. Sein Biograf Skidelsky \parencite{Snowdon2005} beantwortete die Frage warum dem so war damit, dass Keynes von Hicks im allgemeinen keine allzu hohe Meinung hatte, aber auch die spätere enorme Bedeutung von Hicks' IS-LM-Modell absolut nicht abschätzen konnte \parencite[S. 96]{Snowdon2005}. Keynes erlitt außerdem bereits 1937 einen schweren gesundheitlichen Rückschlag. Während des Zweiten Weltkrieges übernahm er wieder wichtige Tätigkeiten im britischen Staatsdienst und gegen Ende des Krieges spielte er noch einmal eine wichtige Rolle in internationalen Verhandlungen \parencite[S. 277]{Scherf1989}. Bei der Etablierung des "`Bretton-Woods"'-Systems vertrat er Großbritannien und setzte sich für den Einsatz einer internationalen Referenzwährung ein. Sein Vorschlag diesbezüglich wurde nicht angenommen. Es setzten sich die Amerikaner um Harry Dexter White weitgehend durch, die den US-Doller als Referenzwährung durchsetzten. Im Jahr 1946 schließlich starb John Maynard Keynes mit 62 Jahren an einem Herzinfarkt.

Keynes war sowohl was seine persönliche Geschichte angeht, als auch seine wissenschaftliche Karriere betreffend ein Phänomen. Er hatte schon früh unkonventionelle und durchaus bedeutende Arbeiten wie \textcite{Keynes1919} "`The Economic Consequences of the Peace"' oder später  \textcite{Keynes1930} "`A Treatise on Money"' veröffentlicht. Der einschneidende Punkt war - wie für so viele Ökonomen, jedoch im Falle von Keynes noch ausgeprägter - die "`Great Depression"'. Die größte Wirtschaftskrise aller Zeiten, die ab 1929 zunächst für Kursstürze an den Börsen und in weiterer Folge fast weltweit für BIP-Rückgänge und hohe Arbeitslosenraten sorgte, brachte Keynes dazu eine völlig neue Wirtschaftstheorie zu verfassen. Mit seinem Tod endete sein Einfluss aber nicht, sondern im Gegenteil, nach 1945 folgte erst die große Zeit des Keynesianismus. Die "`General Theory"' war in weiterer Folge der Ausgangspunkt dreier ganz verschiedener Forschungsrichtungen, die sich allerdings alle direkt auf \textcite{Keynes1936} beriefen. Deshalb ist der Begriff Keynesianismus auch unscharf. Da waren erstens, die Vertreter der "`Neoklassischen Synthese"'. Diese griffen einzelne Ideen von Keynes auf und verschmolzen sie mit den bereits bestehenden (Neo)-Klassischen Arbeiten zur Mainstream-Ökonomie der nächsten Jahrzehnte. Dieser Forschungszweig wird von Historikern oft alternativ als "`Orthodoxer Keynesianimus" und umgangssprachlich eben meist schlicht als "`Keynesianismus"'. Das ist auch jene Interpretation, die bis heute in Lehrbüchern dargestellt wird. Hier wird sie im nächsten Kapitel \ref{Synthese} präsentiert. Daneben gab und gibt es, zweitens, eine Gruppe von Ökonomen, die sich mehr oder weniger \textit{direkt} auf die "`General Theory"' bezieht. Sie werden heute meist als Post-Keynesianer (vgl. Kapitel \ref{Post-Keynes}) bezeichnet und waren von Anfang außerhalb der "`Mainstream-Ökonomie"' angesiedelt. Diese Schule stellt unter anderem die fundamentale Unsicherheit bei Entscheidungen in den Mittelpunkt der Analyse. Ein Teil der General Theory, der bei den Vertretern der Neoklassischen Synthese verloren ging.  (Neo-)klassische Ansätze und damit auch die Synthese von Keynesianismus und (Neo-)Klassik werden von den Post-Keynesianern gänzlich abgelehnt, Die frühe Vertreter waren in Cambridge in England angesiedelt, also an der gleichen Universität, an der auch Keynes tätig war. 

Eine weitere Gruppe von Ökonomen lehnte die "`General Theory"' hingegen von Anfang an ab. Was aber nicht heißt, dass sie diese ignorierten. Im Gegenteil, deren Forschung widmete sich ebenso Keynes' Arbeit, allerdings mit dem Ziel zu zeigen, deren Schwachstellen aufzuzeigen. Dazu gehörten zunächst vor allem die europäischen Liberalen (vgl. Kapitel \ref{Neoliberalismus}) wie August Friedrich Hayek aber auch die Freiburger Schule, sowie auch schon die frühen Vertreter der Chicago School, Jacob Viner und der junge Milton Friedman. Dieser wurde später zum wirtschaftspolitischen Totengräber des Keynesianismus (vgl. Kapitel \ref{Monetarismus}). An seiner Wirkungsstätte - in Chicago - entwickelte sich schließlich der "`totale Widerpart"' zum Keynesianismus, die "`Neue Klassische Makroökonomie"' (vgl. Kapitel \ref{Neue Makro}).

Daneben gab es auch Bereich in der Ökonomie, die Keynes nicht betrachtete. So war für Keynes die "`Vertrustung"' der Ökonomie, die sein Zeitgenosse Schumpeter (vgl. Kapitel \ref{Schumpeter}) thematisierte, kein Problem. Er war in dieser Hinsicht stark geprägt von Marshall und ging von funktionierendem Wettbewerb - im Sinne vollkommener Konkurrenzmärkte - auf den Einzelmärkten aus. Das ist auch dahingehend interessant, da gerade in Cambridge\footnote{Nämlich konkret durch die bereits kurz genannten, späteren Post-Keynesianer.} bereits in den 1920er und 1930er Jahren Skepsis gegenüber der Idee perfekter Wettbewerbsmärkte aufkam. Interessanterweise schenkte Keynes den Ideen seiner Kollegen aus dieser Richtung auch kaum Bedeutung. Obwohl Keynes Politikern gegenüber allgemein skeptisch gegenüberstand \parencite[S. 291]{Scherf1989}, kam er nicht auf die Idee, dass aktive Wirtschaftspolitik auch falsch laufen könnte und Politiker ausschließlich ihren eigenen Nutzen maximierten, nicht jenen des Volkes. Diese Erkenntnis blieb viel später der "`Neuen Politischen Ökonomie"' (vgl. Kapitel \ref{Neue_Politik}) vorbehalten. 



