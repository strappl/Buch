%%%%%%%%%%%%%%%%%%%%% chapter.tex %%%%%%%%%%%%%%%%%%%%%%%%%%%%%%%%%
%
% sample chapter
%
% Use this file as a template for your own input.
%
%%%%%%%%%%%%%%%%%%%%%%%% Springer-Verlag %%%%%%%%%%%%%%%%%%%%%%%%%%

\chapter{Keynesianismus: Nachfrage statt Angebot?}
\label{Keynes}

\textsc{John Maynard Keynes} war sicherlich der prägendste Ökonom des 20. Jahrhunderts. Sein Name ist über die Wirtschaftswissenschaften hinaus bekannt und wird im Begriff "`Keynesianismus"' häufig für jene ökonomische Schule verwendet, die nach 1945 für mehrere Jahrzehnte Volkswirtschaftstheorie und auch Wirtschaftspolitik prägte. Zumindest direkt nach dem Erscheinen seines Hauptwerkes - der "`General Theory"' - wurde er fast uneingeschränkt gefeiert. Heute gelten seine Theorien als umstritten. In der wirtschaftspolitischen Praxis sind jene Ansätze - die er als einer der ersten theoretisch beschrieben hat, und die bis heute nach ihm benannt werden - trotz aller formaler Bedenken weit verbreitet. Vor allem im Fall von Wirtschaftskrisen. Das Auftreten der "`Great Recession"' nach 2008 machte dies deutlich. Bis heute wird die Mainstream-Ökonomie, sowie deren Modelle, als neu-\textit{keynesianisch} bezeichnet. Und noch heute müssen sich Top-Ökonomen häufig dazu bekennen, ob sie pro oder contra Keynes sind. Tatsächlich definierten sich alle führenden Wirtschaftsschulen seit 1946 auch an deren Einstellung zu den keynesianischen Ideen.
Was macht Keynes zu \textit{der} allgegenwärtigen Figur in der Volkswirtschaftslehre? Nun, er gilt als Begründer der modernen Makroökonomie. Wobei dies in dem Sinne zu verstehen ist, dass er als erster eine völlig neue Sichtweise auf die Wirtschaft als Ganzes lieferte und auch eine wirtschaftspolitische Steuerung vorschlug. Er betrachtete Wirtschaft nicht mehr als Summe der Leistungen einzelner Individuen, sondern als Gesamtsystem in dem sich Angebots- und Nachfrageseite gegenseitig bedingen. Der Wirtschaftspolitik wurde damit eine völlig neue Bedeutung gegeben. Dies alleine kann seine einzigartige Stellung aber nicht rechtfertigen. Vielmehr deutete der Zeitgeist nach der "`Great Depression"' bereits an Ökonomie neu zu denken. Keynes war in diesem Sinne eben nicht der einzige, der dies tat. Die Wirtschaftspolitik kannte die stimulierende Wirkung von Staatsausgaben schon vor 1936 \parencite{Fishback2010}. Es gab aber auch schon ähnliche \textit{theoretische} Ansätze, welche die Erkenntnisse Keynes' vorwegnahmen. Vor allem der polnische Ökonom Michal Kalecki arbeitete schon längere Zeit an einer solchen Theorie und publizierte diese \parencite{Kalecki1935} schon vor Keynes auch in englischer Sprache. Diese wurde allerdings nur in Fachkreisen anerkennend aufgenommen. \textcite{Kahn1931} hatte schon deutlich zuvor den Multiplikator und dessen Wirkung vorgestellt und damit wichtige Vorabreiten geliefert. Die beiden genießen aber bis heute nicht einmal einen Bruchteil des Ruhmes, den Keynes nach 1936 rasch erlangte. Noch dazu ist Keynes' General Theory schwer verständlich und umständlich geschrieben. Ökonomen und Wirtschaftshistoriker sind sich recht einig über das Buch "`The General Theory"' selbst. \textcite [S. 31]{Mankiw2006} findet es "`erheiternd und frustrierend"' und rät Studierenden ab das Werk zu lesen. Robert Lucas antwortete auf die Frage, ob man das Werk heute noch lesen soll kurz und knapp mit "`Nein"' \parencite{Lucas2013}. \textcite[S. 429]{Rosner2012} bezeichnet den Charakter des Buches als eigenartig und selbst einer der eifrigsten Verfechter der keynesianischen Ideen Paul Samuelson findet das Werk selbst "`schlecht geschrieben und ärmlich organisiert"' und zudem "`höchst verwirrend"' \parencite[S. 190]{Samuelson1946}. Und doch zweifelt keiner der genannten an der hohen Bedeutung des Werks. Keynes' einzigartige Stellung in der Ökonomie erscheint angesichts dieser Tatsachen dennoch fast überraschend. Vor allem wenn man dazu noch selbst einen Blick in das Buch wirft. Seine bahnbrechende Leistung soll und darf aber auch nicht geschmälert werden. Die General Theory besticht nicht als Gesamtwerk. Die unumstrittene Genialität des Werks liegt vielmehr in den Einzelideen die dort an verschiedenen Stellen gesammelt vorgebracht werden und in ihrer Gesamtheit zur damaligen Zeit revolutionär waren. Das Bild des dominierenden Ökonomen, das von Keynes entstand, ist Resultat einer Kombination aus privilegierter Geburt und notwendigem Glück, aber vor allem Genialität gepaart mit einem enormen Tätigkeitsumfang. Zudem verfügte Keynes über ein gesundes Selbstvertrauen. Schon vor der Veröffentlichung seines Hauptwerkes schrieb er an den befreundeten Literaten George Bernard Shaw: "`I believe myself to be writing a book of economic theory which will largely revolutionize [...] the way the world thinks about economic problems"' \parencite[S.13]{Warsh}. 

Sein beruflicher Werdegang, sein Lebensstil, sowie sein relativ früher Tod machten ihn in weiterer Folge zu einer Art Mythos.


Keynes wurde in priviligerten Verhältnissen geboren und schaffte raschen Aufstieg.
 Indien, danach Consequences of the PEace über Ökonomie hinaus bekannt schon vor seinem großen Werk. Bloomsbury Group, Bisexuell. Hochzeit mit Ballerina. Superstar der Ökonomie und Lebemann und so wird er auch dargestellt. Zeitgenossen litten darunter Pigou, Kalecki, Hayek, Schumpeter. 




Keynes war sowohl was seine persönliche Geschichte angeht, als auch seine wissenschaftliche Karriere betreffend ein Phänomen. Er hatte schon früh unkonventionelle und durchaus bedeutende Arbeiten wie \textcite{Keynes1919} "`The Economic Consequences of the Peace"' oder \textcite{Keynes1930} "`A Treatise on Money"' veröffentlicht. Der einschneidende Punkt war - wie für so viele Ökonomen, jedoch im Falle von Keynes noch ausgeprägter - die "`Great Depression"'. Die größte Wirtschaftskrise aller Zeiten,  die ab 1929 zunächst für Kursstürze an den Börsen und in weiterer Folge weltweit für BIP-Rückgänge und hohe Arbeitslosenraten sorgte, brachte Keynes dazu eine völlig neue Wirtschaftstheorie zu verfassen. Diese veröffentlichte er im bahnbrechenden Werk \textcite{Keynes1936} "`The General Theory of Employment, Interest and Money"', kurz meist einfach als "`General Theory"' bezeichnet. 


Dieses Werk ist wohl das einflussreichste ökonomische Buch des 20. Jahrhunderts. Es wird nach wie vor häufig diskutiert. Der Versuch sämtliche unterschiedliche Untersuchungen oder Interpretationen des Werks auch nur anzuführen, müsste scheitern. \textcite[S. 38ff]{Weintraub1979} benannte ein Kapitel seines Buches scherzeshalber "`The $4,827^{th}$ re-examination of Keynes's system"'. Die General Theory ist der direkte Ausgangspunkt dreier ganz verschiedener grober Forschungsrichtungen, die in weiterer Folge entstanden. Erstens, die Vertreter der "`Neoklassischen Synthese"' (Vergleich Kapitel \ref{Synthese}) griffen einzelne Ideen von Keynes auf und verschmolzen diese mit den bereits bestehenden (Neo)-Klassischen Arbeiten zur Mainstream-Ökonomie der nächsten Jahrzehnte. Zweitens, eine Gruppe von Ökonomen bezog und bezieht sich bis heute mehr oder weniger \textit{direkt} auf die "`General Theory"'. Diese Gruppe - häufig unter dem Begriff "`Post-Keynesianer"' zusammengefasst - sieht die Keynesianische Theorie durch die "`Neoklassischen Synthese"' verstümmelt und auf nur einzelne Punkte beschränkt. Eine dritte Gruppe von Ökonomen lehnte die "`General Theory"' von Anfang an ab. Dazu gehörten zunächst vor allem die kontinental-europäischen Liberalen (vgl. Kapitel \ref{Neoliberalismus}) zum Beispiel die Freiburger Schule und auch August Friedrich Hayek, aber auch schon der junge Milton Friedman. Dieser wurde später zum wirtschaftspolitischen Totengräber des Keynesianismus (vgl. Kapitel \ref{Monetarismus}). An seiner Wirkungsstätte - in Chicago - entwickelte sich schließlich der "`totale Widerpart"' zum Keynesianismus, die "`Neue Klassische Makroökonomie"' (vgl. Kapitel \ref{Neue Makro}). 





Sie mögen sich vielleicht fragen warum dieses Kapitel ein eher kurzes ist?! Schließlich ist doch der Keynesianismus die zentrale ökonomische Errungenschaft des 20. Jahrhunderts gewesen und deren Schöpfer \textsc{John Maynard Keynes} der größte Ökonom des 20. Jahrhunderts. Wenn man anstatt von "`der Größte"' die Notation "`einer der Größten"' verwendet, so kann man dies sicherlich bestätigen.

Warum ist aber das Kapitel nun so kurz ausgefallen? Nun, Keynes veröffentlichte sein bahnbrechendes Werk \textit{The General Theory of Employment, Interest and Money} im Jahr 1936. Bald darauf, 1939, brach der Zweite Weltkrieg aus und die Welt beschäftigte sich bis 1945 mit anderen Sachen, als rein ökonomischen Fragestellungen. 1946 starb Keynes. Wenig später startete "`sein"' Keynesianismus den ökonomischen Siegeszug um die Welt. Und hier muss man entscheidend einhaken! Was meist als "`Keynesianismus"' bezeichnet wird, ist in Wirklichkeit bereits eine Weiterentwicklung des eigentlichen, ursprünglichen Keynesianismus.

HIER WEITER
Inhalt Keynes:

Einer der wichtigsten Punkte für die spätere wirtschaftspolitische Anwendung war der "`Multiplikator-Effekt"'. Das zehnte Kapitel der "`General Theory"' widmet sich diesem Thema. Wie \textcite[S. 114]{Keynes1936} selbst beschreibt, war es Richard \textcite{Kahn1931}, der den "`Multiplikator-Effekt"' als erster beschrieb, und zwar im Sinne des Zusammenhangs zwischen (staatlichen) Investitionen und Arbeitslosigkeit.






Bereits 1937 veröffentlichte \textsc{John R. Hicks} den Artikel \textit{Mr. Keynes and the Classics: A Suggested Interpretation}. Das Werk von Keynes hat nämlich die Besonderheit, dass es schwer zu lesen ist, aber vor allem auf formale Darstellungen verzichtet. Hicks übernahm diese Formalisierung und verband einen Teil von Keynes' Theorie mit neoklassischen Elementen zum \textit{IS-LM-Modell}. Dieses Modell stellt noch heute den finalen Punkt in vielen Einführungslehrveranstaltungen zu Makroökonomie dar.

Diese Formalisierung durch Hicks enthält aber zwei extrem wichtige Punkte:
\begin{itemize}
	\item "`Er übernahm einen \textit{Teil} von Keynes' Theorie"': Bei dieser Formalisierung gingen im Gegenzug viele Teile der General Theory verloren. Eine Tatsache, die bis heute in der Mainstream-Ökonomie hingenommen wird. Diese "`verlorenen Teile"' sollten später von den \textit{Post-Keynesianern} aufgegriffen werden.
	\item "`Er verband diesen Teil mit neoklassischen Elementen"': Wenn wir lesen, dass die 1950er und 1960er Jahre die Hochzeit des "`Keynesianismus"' waren, dann meinen Ökonomen eigentlich, dass das "`alte"' Neoklassische Wissen herangezogen wurde und um "`Keynesianische"' Elemente erweitert wurde. Es entstand also eine "`Synthese"' aus zwei Wissensgebieten, folglich wird das Ganze unter Ökonomen die \textsc{Neoklassische Synthese} genannt.
\end{itemize}

Sie können natürlich jetzt argumentieren das sei Haarspalterei und zu behaupten die allgemeine Bezeichnung Keynesianismus müsste eigentlich Neoklassische Synthese heißen, sei Besserwisserisch und verwirrt mehr als sie Nutzen bringt. Das stimmt im Großen und Ganzen. Aber ich denke auch es ist wichtig zu erwähnen, dass die Theorien von Keynes praktisch von ihrem erscheinen weg unterschiedlich verwendet und interpretiert wurden. Die Interpretation, die sich in den Wirtschaftswissenschaften als am erfolgreichsten erwiesen hat, ist eben jene, die ich als \textsc{Neoklassische Synthese} im nächsten Kapitel vorstelle. 



