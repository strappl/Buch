%%%%%%%%%%%%%%%%%%%%% chapter.tex %%%%%%%%%%%%%%%%%%%%%%%%%%%%%%%%%
%
% sample chapter
%
% Use this file as a template for your own input.
%
%%%%%%%%%%%%%%%%%%%%%%%% Springer-Verlag %%%%%%%%%%%%%%%%%%%%%%%%%%

\chapter{Keynesianismus: Nachfrage statt Angebot?}
\label{Keynes}

\textsc{John Maynard Keynes} war sicherlich der prägendste Ökonom des 20. Jahrhunderts. Sein Name ist über die Wirtschaftswissenschaften hinaus bekannt und wird im Begriff "`Keynesianismus"' häufig für jene ökonomische Schule verwendet, die nach 1945 für mehrere Jahrzehnte Volkswirtschaftstheorie und auch Wirtschaftspolitik prägte. 


Er gilt als Begründer der modernen Makroökonomie. Wobei dies in dem Sinne zu verstehen ist, dass er als erster eine völlig neue Sichtweise auf die Wirtschaft als Ganzes lieferte und auch eine wirtschaftspolitische Gesamtsteuerung vorschlug. Der Wirtschaftspolitik wurde damit eine völlig neue Bedeutung gegeben.
Keynes war sowohl was seine persönliche Geschichte angeht, als auch seine wissenschaftliche Karriere betreffend ein Phänomen. Er hatte schon früh unkonventionelle aber durchaus bedeutende Arbeiten wie \textcite{Keynes1919} "`The Economic Consequences of the Peace"' oder \textcite{Keynes1930} "`A Treatise on Money"'. Der einschneidende Punkt war - wie für so viele Ökonomen, jedoch im Falle von Keynes noch ausgeprägter - die "`Great Depression"'. Die größte Wirtschaftskrise aller Zeiten,  die ab 1929 zunächst für Kursstürze an den Börsen und in weiterer Folge weltweit für BIP-Rückgänge und hohe Arbeitslosenraten sorgte, brachte Keynes dazu eine völlig neue Wirtschaftstheorie zu verfassen. Diese veröffentlichte er im bahnbrechenden Werk \textcite{Keynes1936} "`The General Theory of Employment, Interest and Money"', kurz meist einfach als "`General Theory"' bezeichnet.
Dieses Werk ist wohl das einflussreichste ökonomische Buch des 20. Jahrhunderts. Es wird nach wie vor häufig diskutiert. Der Versuch sämtliche unterschiedliche Untersuchungen oder Interpretationen des Werks auch nur anzuführen, müsste scheitern. \textcite[S. 38ff]{Weintraub1979} benannte ein Kapitel seines Buches scherzeshalber "`The $4,827^{th}$ re-examination of Keynes's system"'. Es ist der direkte Ausgangspunkt dreier ganz verschiedener grober Forschungsrichtungen, die in weiterer Folge entstanden. Erstens, die Vertreter der "`Neoklassischen Synthese"' (Vergleich Kapitel \ref{Synthese}) griffen einzelne Ideen von Keynes auf und verschmolzen diese mit den bereits bestehenden (Neo)-Klassischen Arbeiten zur Mainstream-Ökonomie der nächsten Jahrzehnte. Zweitens, eine Gruppe von Ökonomien bezog und bezieht sich bis heute mehr oder weniger \textit{direkt} auf die "`General Theory"'. Diese Gruppe - häufig unter dem Begriff "`Post-Keynesianer"' zusammengefasst - sieht die Keynesianische Theorie durch die "`Neoklassischen Synthese"' verstümmelt und auf nur einzelne Punkte beschränkt. Eine dritte Gruppe von Ökonomen lehnte die "`General Theory"' von Anfang an ab. Dazu gehörten zunächst vor allem die kontinental-europäischen Liberalen (vgl. Kapitel \ref{Neoliberalismus}) zum Beispiel die Freiburger Schule und auch August Friedrich Hayek, aber auch schon der junge Milton Friedman. Dieser wurde später zum wirtschaftspolitischen Totengräber des Keynesianismus (vgl. Kapitel \ref{Monetarismus}). An seiner Wirkungsstätte - in Chicago - entwickelte sich schließlich der "`totale Widerpart"' zum Keynesianismus, die "`Neue Klassische Makroökonomie"' (vgl. Kapitel \ref{Neue Makro}). Zumindest direkt nach dem Erscheinen der "`General Theory"' wurde das Werk allerdings fast uneingeschränkt gefeiert. 

HIER WEITER
Notizen auf Seite 39 Snowdon +
Schlechtes Buch: \textcite{Samuelson1946} und \textcite[S. 56]{Snowdon2005}. 

Sie mögen sich vielleicht fragen warum dieses Kapitel ein eher kurzes ist?! Schließlich ist doch der Keynesianismus die zentrale ökonomische Errungenschaft des 20. Jahrhunderts gewesen und deren Schöpfer \textsc{John Maynard Keynes} der größte Ökonom des 20. Jahrhunderts. Wenn man anstatt von "`der Größte"' die Notation "`einer der Größten"' verwendet, so kann man dies sicherlich bestätigen.

Warum ist aber das Kapitel nun so kurz ausgefallen? Nun, Keynes veröffentlichte sein bahnbrechendes Werk \textit{The General Theory of Employment, Interest and Money} im Jahr 1936. Bald darauf, 1939, brach der Zweite Weltkrieg aus und die Welt beschäftigte sich bis 1945 mit anderen Sachen, als rein ökonomischen Fragestellungen. 1946 starb Keynes. Wenig später startete "`sein"' Keynesianismus den ökonomischen Siegeszug um die Welt. Und hier muss man entscheidend einhaken! Was meist als "`Keynesianismus"' bezeichnet wird, ist in Wirklichkeit bereits eine Weiterentwicklung des Keynesianismus.

Bereits 1937 veröffentlichte \textsc{John R. Hicks} den Artikel \textit{Mr. Keynes and the Classics: A Suggested Interpretation}. Das Werk von Keynes hat nämlich die Besonderheit, dass es schwer zu lesen ist, aber vor allem auf formale Darstellungen verzichtet. Hicks übernahm diese Formalisierung und verband einen Teil von Keynes' Theorie mit neoklassischen Elementen zum \textit{IS-LM-Modell}. Dieses Modell stellt noch heute den finalen Punkt in vielen Einführungslehrveranstaltungen zu Makroökonomie dar.

Diese Formalisierung durch Hicks enthält aber zwei extrem wichtige Punkte:
\begin{itemize}
	\item "`Er übernahm einen \textit{Teil} von Keynes' Theorie"': Bei dieser Formalisierung gingen im Gegenzug viele Teile der General Theory verloren. Eine Tatsache, die bis heute in der Mainstream-Ökonomie hingenommen wird. Diese "`verlorenen Teile"' sollten später von den \textit{Post-Keynesianern} aufgegriffen werden.
	\item "`Er verband diesen Teil mit neoklassischen Elementen"': Wenn wir lesen, dass die 1950er und 1960er Jahre die Hochzeit des "`Keynesianismus"' waren, dann meinen Ökonomen eigentlich, dass das "`alte"' Neoklassische Wissen herangezogen wurde und um "`Keynesianische"' Elemente erweitert wurde. Es entstand also eine "`Synthese"' aus zwei Wissensgebieten, folglich wird das Ganze unter Ökonomen die \textsc{Neoklassische Synthese} genannt.
\end{itemize}

Sie können natürlich jetzt argumentieren das sei Haarspalterei und zu behaupten die allgemeine Bezeichnung Keynesianismus müsste eigentlich Neoklassische Synthese heißen, sei Besserwisserisch und verwirrt mehr als sie Nutzen bringt. Das stimmt im Großen und Ganzen. Aber ich denke auch es ist wichtig zu erwähnen, dass die Theorien von Keynes praktisch von ihrem erscheinen weg unterschiedlich verwendet und interpretiert wurden. Die Interpretation, die sich in den Wirtschaftswissenschaften als am erfolgreichsten erwiesen hat, ist eben jene, die ich als \textsc{Neoklassische Synthese} im nächsten Kapitel vorstelle. 



