%%%%%%%%%%%%%%%%%%%%% chapter.tex %%%%%%%%%%%%%%%%%%%%%%%%%%%%%%%%%
%
% sample chapter
%
% Use this file as a template for your own input.
%
%%%%%%%%%%%%%%%%%%%%%%%% Springer-Verlag %%%%%%%%%%%%%%%%%%%%%%%%%%

\chapter{Schumpeter: Der ganz eigene Weg}
\label{Schumpeter}

\section{Vorläufer Zyklentheorie: Kondratieff}

\section{Die schöpferische Zerstörung}
Eigentlich hätte sich \textit{Joseph Alois Schumpeter} ein eigenes Kapitel verdient. Die Zurechnung zur Österreichischen Schule hinkt ebenfalls gewaltig und ist am ehesten durch seine Herkunft gerechtfertigt, wobei auch das nur bedingt richtig ist. Aber der Reihe nach.

Schumpeter wurde 1883 in Trest geboren, das heute im südlichen Tschechien liegt, geboren. Trest (deutsch Triesch) gehörte damals zum österreichischen Teil der Doppelmonarchie Österreich-Ungarn. Er machte in Österreich früh eine beeindruckende Karriere: Jüngster Professor der Monarchie, später Finanzminister und Bankvorstand

Und trotzdem wirkt seine Biographie eher traurig als erfolgreich: Mit vier Jahren verstarb sein Vater. Die Mutter wurde zur wichtigen Bezugsperson. Seine Frau -- vielleicht die einzige, die er wirklich liebte, denn er galt sonst als Frauenheld und lebte dies für die damalige Zeit in skandalöser Weise aus -- starb bei der Geburt seines Kindes. Das Kind wenige Stunden später. 
Als Finanzminister trat er nach wenigen Monaten zurück. Die Bank, die er als Vorstand leitete, ging Pleite und führte ihn selbst an den Rand des Bankrotts. Wissenschaftlich war er geprägt von der Konkurrenz zu Keynes. Und wenn man diese Konkurrenz als Wettbewerb sehen will, zog er ständig den Kürzeren: Ein begonnenes geldpolitisches Werk vollendete Schumpeter nicht mehr, nachdem Keynes seine "`Treatise on Money"' veröffentlichte. Sein monumentales Wert zur Zyklentheorie "`Business Cycles"' stellte er 1939 fertig. Eine Unzeit wenn man bedenkt, dass in diesem Jahr der Zweite Weltkrieg begann, Konjunkturzyklen also wenig Interesse hervorriefen. Nach 1945 wiederum interessierte sich ebenso fast niemand für das Werk. Durch die "`General Theory"' von Keynes schien es so als gehörten Konjunkturzyklen für alle Zeit der Vergangenheit an: Keynes's Deficit Spendings sollten doch gerade diese Konjunkturzyklen glätten.
Und doch gilt Schumpeter heute als einer der bedeutensten Ökonomen des 20. Jahrhundert. Ich habe zu Anfang des Kapitels erwähnt, dass man Schumpeter nur schwierig einer ökonomischen Schule zuordnen kann. Vielleicht ein Grund warum er sich in seiner Gegenwart nur schwer durchsetzte. Vielleicht aber auch ein Grund dafür warum er heute unverändert aktuell ist: Seine Theorie ist praktisch zeitlos. Insbesondere die auf ihn zurückzuführende Idee der "`schöpferischen Zerstörung"' wird immer wieder zitiert. Und, wesentliche bedeutender: Kaum jemand widerspricht dieser Idee der schöpferischen Zerstörung! Keynes hatte seine Blütezeit, aber er wurde sehr bald auch kritisiert und später von \textit{Robert Lucas} wissenschaftlich für tot erklärt. Auf Schumpeter berufen sich noch heute -- fast 100 Jahre nach dem Erscheinen seines Werkes "`The Theory of Development"' -- Ökonomen unterschiedlichster Richtungen (Acemoglu und Paul Romer, früher Baumol und Minsky).

Interessant, dass sein großer Zeitgenosse John Maynard Keynes, das Thema des Marktversagens gänzlich unbehandelt ließ. Im Gegenteil, Keynes folgte diesbezüglich seinem Lehrer Marshall und übernahm das Konzept des perfekten Wettbewerbs auf den Einzelmärkten. In Bezug auf die "`Vertrustung"' der Ökonomie hatte Keynes auch kaum Bedenken, er glaubte an die "`Drei-Generationen-Theorie"', wonach Unternehmen nur mittelfristig erfolgreich sind \parencite[S. 93]{Snowdon2005}. Schumpeter vertrat diesbezüglich einen ganz anderen Ansatz und prophezeite, dass der Kapitalismus langfristig daran scheitern wird, weil die Monopolisierung der Märkte immer weiter voranschreitet, bis der Wettbewerb schließlich gänzlich ausgeschaltet sei.

Schumpeter hatte auch Einfluss auf Douglass North \textcite[S. 11]{Menard2014}