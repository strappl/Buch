%%%%%%%%%%%%%%%%%%%%% chapter.tex %%%%%%%%%%%%%%%%%%%%%%%%%%%%%%%%%
%
% sample chapter
%
% Use this file as a template for your own input.
%
%%%%%%%%%%%%%%%%%%%%%%%% Springer-Verlag %%%%%%%%%%%%%%%%%%%%%%%%%%

\chapter{Auf dem Historikerweg}
\label{Historisch}

Deutsche Wissenschaftler waren um die Jahrhundertwende in so vielen Disziplinen in Wissenschaft und Technik führend. Deutsche Ökonomen aus dieser Zeit sind hingegen heute kaum noch bekannt. Dies ist vor allem auch in Hinblick auf die benachbarte damalige Donaumonarchie Österreich-Ungarn interessant, die im Gegensatz so viele heute noch bedeutende Ökonomen, sowie die Österreichische Schule der Nationalökonomie hervorbrachte. Warum blieb die Volkswirtschaftslehre im deutschsprachigen Raum damals weitgehend auf Österreich beschränkt? Ein bedeutender Grund liegt im Methodenstreit, der zwischen dem uns schon bekannten Carl Menger und den Vertretern der jüngeren Historischen Schule um Gustav Schmoller geführt wurde. Wir erinnern uns aus Kapitel \ref{Marginalismus}, dass wichtige Entwicklungen der Neoklassik ursprünglich von deutschen Ökonomen, nämlich Gossen, Thünen, aber auch Rau, vorweggenommen wurden. Ihre Arbeiten blieben aber unentdeckt. Dies vor allem deshalb, weil die universitäre Ökonomie in Deutschland eine ganz andere Hauptrichtung vertrat: Die Historische Schule der Nationalökonomie. Nicht nur die deutschen Vordenker der Neoklassik blieben dadurch zunächst unentdeckt, auch die Österreichische Schule blieb deshalb räumlich vor allem auf Wien beschränkt \parencite[S. 339]{Rosner2012}.


Historische Schule

\section{List}
Vor allem natürlich Methodenstreit mit Menger! (Menger 1883, "`Untersuchungen über die Methoden der Socialwiss. Vera Linß-Buch, Seite 66)

\section{Schmoller \& Hildebrand}

Die historische Schule wird häufig als Vorläufer des Institutionalismus gesehen.
Pierenkamp Toni, Buch, Seite 182.