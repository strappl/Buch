%%%%%%%%%%%%%%%%%%%%% chapter.tex %%%%%%%%%%%%%%%%%%%%%%%%%%%%%%%%%
%
% sample chapter
%
% Use this file as a template for your own input.
%
%%%%%%%%%%%%%%%%%%%%%%%% Springer-Verlag %%%%%%%%%%%%%%%%%%%%%%%%%%

\chapter{Auf dem Historikerweg}
\label{Historisch}

Deutsche Wissenschaftler waren um die Jahrhundertwende in sehr vielen Disziplinen in Wissenschaft und Technik führend. Deutsche Ökonomen aus dieser Zeit sind hingegen heute kaum noch bekannt.

Als Grund gilt heute, dass die damalige Mainstream-Ökonomie in Deutschland, die "`Historische Schule der Nationalökonomie"', sich aus heutiger Sicht als "`wissenschaftliche Sackgasse"' \parencite[S. 229]{Rosner2012} erwies. Die Mainstream-Ökonomie in Deutschland befand sich also über ein halbes Jahrhundert lang quasi auf dem Holzweg.

Wir erinnern uns an Kapitel \ref{Marginalismus}, dass wichtige Entwicklungen der Neoklassik ursprünglich von deutschen Ökonomen, nämlich Hermann Heinrich Gossen, Johann Thünen, aber auch Karl Heinrich Rau, vorweggenommen wurden. Ihre Arbeiten blieben aber unentdeckt. Dies eben vor allem deshalb, weil die universitäre Ökonomie in Deutschland mit der Historischen Schule eine ganz andere Hauptrichtung vertrat. Dies ist vor allem auch in Hinblick auf die benachbarte damalige Donaumonarchie Österreich-Ungarn interessant, die im Gegensatz so viele heute noch bedeutende Ökonomen, sowie die Österreichische Schule der Nationalökonomie hervorbrachte. Die nachhaltig wirkende Volkswirtschaftslehre im deutschsprachigen Raum blieb also lange auf den Raum Wien beschränkt \parencite[S. 339]{Rosner2012}, obwohl die deutschen Vordenker den Neoklassiker Carl Menger entscheidend beeinflussten \parencite{Streissler1990}, aber aufgrund der Dominanz vor allem der Historischen Schule in der universitären Lehre nicht Fuß fassen konnten.

Was waren die Gründe dafür, dass sich in Deutschland Klassik und Neoklassik erst im 20. Jahrhundert durchsetzen konnten? Die Ideen von Smith wurden schon um 1800 nach Deutschland gebracht und dort diskutiert. Allerdings war das Deutschland des frühen 19. Jahrhunderts gänzlich anders organisiert, als Großbritannien zur gleichen Zeit. Erstens, gab es absolutistische Fürstentümer, die liberale Ideen, sowie wirtschaftlichen Freihandel schon an sich utopisch erscheinen ließen. Zweitens, war die Landwirtschaft noch der wichtigste Produktionszweig. Drittens war das Handwerk den Mitgliedern der Zünfte vorbehalten. Es gab also umfangreiche Marktzugangsbeschränkungen, wie man heute sagen würde \parencite[S. 213f]{Rosner2012}. Zudem war in Deutschland - das gilt zumindest bis zum Revolutionsjahr 1848 - eine freie Diskussion über Wirtschaftspolitik nicht möglich. So wurde Friedrich List (vgl. Unterkapitel \ref{List}) aus politischen Gründen zu einer Haftstrafe verurteilt. Adam Müller - wie eben beschrieben - aus Dresden ausgewiesen und Bruno Hildebrand (vgl. Unterkapitel \ref{Hist_Schulen}) wegen der Einfuhr kritischer Arbeiten ebenfalls inhaftiert \parencite[S. 211]{Rosner2012}. Es war in Deutschland lange also gar nicht möglich die Wirtschaftstheorie von Smith im Hinblick auf wirtschaftspolitische \textit{Anwendungen} zu diskutieren. Stattdessen war der Hauptvertreter der "`Romantischen Schule"', Adam Müller einer der einflussreichsten deutschen Ökonomen des frühen 19. Jahrhunderts. Seine Arbeiten beeinflussten sowohl die Vertreter der Historischen Schule, als auch Karl Marx. Er wurde aus politischen Gründen aus Dresden ausgewiesen und war in der Folge in Wien tätig. Sein Werk ist heute völlig unbedeutend, er hob die Bedeutung der Gesellschaft als Ganzes im Produktionsprozess hervor und stellte sich damit gegen den Individualismus bei Smith. Seine Kritik an der Ökonomie des Adam Smith blieb in Deutschland verwurzelt. Müller ging aber noch weiter, er befürwortete eine Zunft-artige Ordnung der Wirtschaft und lehnte eine freie Marktwirtschaft als solche gänzlich ab \parencite[S. 24ff]{Rosner2012}. 

\section{Vorläufer der Historischen Schule}
\label{List}

Friedrich List ist einer der bekanntesten deutschen Ökonomen des 18. Jahrhunderts. Seine spannende Biographie ist von Höhen und Tiefen geprägt. Er war sein Leben lang vor allem eines: ein tatkräftiger Mann. Bis 1822 machte er eine beeindruckende Karriere als Beamter, Professor an der Universität Tübingen und schließlich auch als Abgeordneter in Württemberg. Die Stelle als Professor erhielt List, der nie ein ordentliches Studium abgeschlossen hatte, durch seine Arbeit für einen befreundeten Minister \parencite[S. 227]{Hauser1989}. Aus heutiger Sicht also eher eine zweifelhafte Ehre. Nichtsdestotrotz legte er schon zu dieser Zeit enorme - wenn auch letztendlich nutzlose - Bemühungen an den Tag, die damals üblichen Binnenzölle zwischen den vielen deutschen Staaten abzuschaffen und stattdessen einen Außenzoll einzuführen. Zu dieser Zeit wohlgemerkt noch ohne bedeutende theoretische Schriften diesbezüglich zu verfassen. 1822 folgte allerdings die erste Wende in seinem Leben. Er forderte per Flugblatt, kurz gesagt, mehr Bürgerrechte für die Menschen ein. Sein Parlamentssitz bewahrte ihn - während dieser realpolitisch doch recht absolutistischen Zeit - nicht vor Auslieferung und Verurteilung. Er wurde zu zehn Monaten Haft verurteilt, wobei das Urteil auch die Enthebung von all seinen Ämtern umfasste und es de facto unmöglich machte, dass List in einem deutschen Staat je wieder ein öffentliches Amt bekleiden könne \parencite[S. 229]{Hauser1989}. List, der in vielleicht in der kurzen Darstellung wie heldenhafter Revolutionär wirkte, war in Wirklichkeit eher ein patriotischer Beamter, eigentlich treu der Monarchie und dem König. Die Hoffnung auf Gnade erwies sich als unberechtigt. Nach zweijähriger Flucht, kehrte er nach Württemberg zurück um dort seine Haftstrafe abzusitzen um danach erst recht des Landes verwiesen zu werden. Er wanderte 1925 in die USA aus, wo er bald als Herausgeber einer Zeitung und als Eisenbahn-Pionier, Erfolge erzielte, zu Wohlstand und Ansehen kam und zudem 1830 die amerikanische Staatsbürgerschaft erlangte. In den USA verfasste er auch erstmals ökonomisch bedeutende Schriften, in denen er die Idee der Schutzzölle erstmals entwickelte. Sein Schicksal war wohl seine Sehnsucht nach Deutschland, wohin er immer zurückkehren wollte. Dies gelang ihm schließlich auch mit Hilfe des US-amerikanischen Präsidenten, der ihn als Konsul einsetzte. Nach erneuten Problemen dabei trat List im Jahre 1832 tatsächlich seine Stelle als Konsulat für Baden an \parencite[S. 232]{Hauser1989}. In weiterer Folge wollte er in Deutschland die Eisenbahn etablieren. Er war dabei auch führend eingebunden, sein Arbeit wurde jedoch wiederholt nicht belohnt. Weder finanziell - er wurde kein Anteilseigner der Bahnprojekte - noch im Hinblick auf eine einflussreiche Position, die ihm wiederholt verwehrt wurde. Erst nach dem Scheitern im Eisenbahngewerbe, Ende der 1830er Jahre, begann er sein Hauptwerk \parencite{List1841}, das schließlich als "`Das nationale System der politische Oekonomie"' veröffentlicht wurde. Dem Werk war unmittelbarer Erfolg beschieden, er kam zu erneutem Ansehen, hatte eine  Audienz beim König und schließlich erhielt er auch seine "`bürgerlichen Ehrenrechte"' zurück \parencite[S. 235]{Hauser1989}. Eine feste berufliche Position erlangte er aber nicht mehr und musste weiterhin von seinen unsicheren journalistischen Tätigkeiten leben, dabei war er als Berater im In- und Ausland durchaus gefragt. Der rastlos bleibende List litt zunehmend an körperlichen Gebrechen und verfiel in Depressionen. Ende 1846 erschoss er sich auf der Durchreise in Kufstein in Tirol, wo er auch begraben wurde und bis heute ein Denkmal an ihn erinnert. "`Die Tragik seines Lebens bestand darin, dass es nach seiner Wirkung überaus erfolgreich und folgenreich und dennoch, für List persönlich, voller Misserfolge gewesen ist"', fasst \textcite[S. 237]{Hauser1989} zusammen.
Das ökonomische Gesamtwerk von List umfasst vor allem drei wesentliche Punkte. Erstens, die Schutz- und Erziehungszölle, zweitens, die Bedeutung einer Organisation, zum Beispiel in Form einer Nation, für die langfristige ökonomische Entwicklung eines Staates und drittens, die Evolution von Nationen in Entwicklungsstufen. Vor allem die ersten beiden Punkte wurden als Antwort und Kritik an die Werke Smith' und Say's (vgl. Kapitel \ref{Klassik}). Am bekanntesten ist sicher seine Arbeit zu den Schutz- und Erziehungszöllen, die allerdings nicht originär auf \textcite{List1840, List1841} zurückgehen, sondern auch in den USA von Alexander Hamilton schon beschrieben wurden \parencite[S. 243]{Hauser1989}. Demnach können sich geringer entwickelte Staaten nicht aus eigener Kraft auf das Entwicklungslevel des führenden Staates bringen, weil dessen Industrie das "`Entwicklungsland"' ständig mit billigeren Industriegütern beliefert, sodass der Aufbau einer eigenen Industrie nicht gelingen kann. Seine empirische Grundlage für diese Theorie bestand aus seinen Erfahrungen die er in Deutschland, Frankreich und den USA sammelte. Die damalige Hegemonialmacht Großbritannien verhinderte demnach eben die Entwicklung der Industrie in diesen Staaten. Schutz- und Erziehungszölle sollten dem entgegenwirken. \textcite{List1841} stellt sich damit gegen den Freihandel, den eben der Brite Smith forderte. Die viel später entstandene "`Neoklassische Wachstumstheorie"' (vgl. Kapitel \ref{sec: Wachstum}) widerspricht der Theorie List's aus theoretischer Sicht und geht stattdessen davon aus, das Kapital gerade dort eingesetzt wird, wo Arbeit im Verhältnis zu Kapital in hohem Ausmaß verfügbar und damit billiger ist, was in Entwicklungsländern eher der Fall ist. Interessanterweise greift \textcite{List1841} mit dem Ansatz, hoch-entwickelte Staaten würden das Wachstum in weniger-entwickelten Nationen bremsen, einem zentralen Punkt in der Entwicklungsökonomie, Mitte des 20. Jahrhunderts vor. Die Vertreter der Dependenztheorie (vgl. Kapitel \ref{sec: Wachstum}) sehen dies nämlich genauso. Dennoch beziehen sich die Vertreter der Dependenztheorie kaum auf \textcite{List1841}, was an dessen Einschätzung liegen könnte, die Länder der "`heißen Zone"' haben kein eigenständiges Entwicklungspotenzial \parencite[S. 96]{Bachinger2005}. Auch der, vor allem in \textcite{List1840}, verarbeitete Ansatz, die Ökonomie entwickle sich in verschiedenen Stufen, wurde im 20. Jahrhundert vom Modernisierungstheoretiker \textcite{Rostow1960} ebenfalls im Rahmen der Entwicklungsökonomie vorgebracht.
Die Bedeutung von Organisationen ist der zweite wesentliche Beitrag von \textcite{List1841}. Insgesamt wurde damit natürlich der von Smith so stark hervorgehobene Individualismus kritisiert. Bei \textcite{List1841} ist die Gesamtleistung einer Gemeinschaft - gemeint wurde dabei eine Nation - mehr als die Summe der individuellen Gewinne der Einzelnen.  \textcite{List1841} stellt sich damit gegen die Ansicht, dass maximaler Nutzen dann erreicht wird, wenn jedes Individuum egoistisch handelt. Er befürchtet vor allem eine Vernachlässigung der langfristigen Ziele einer Volkswirtschaft, wie das Aufrechterhalten von Gesundheitseinrichtungen und Bildungseinrichtungen. Die Thematik wurde in der Klassik, vielmehr aber vor allem später in der Neoklassik in der Wirtschaftstheorie tatsächlich lange Zeit kaum behandelt, obwohl deren realpolitische Bedeutung selbstverständlich stets von Interesse war. Erst die Institutionenökonomik (vgl. Kapitel \ref{Institut}) und aktuell vor allem der Neue Institutionalismus (vgl. Kapitel \ref{Neue Institut}). beschäftigen sich eingehend mit der Bedeutung von Organisationen. List aber als deren Vorläufer darzustellen würde zu weit gehen, zu rudimentär waren seine Ausführungen und seine methodischen Ansätze um dies zu behaupten.

List wurde nach seinem Tod in Deutschland, im Gegensatz zu manch unbekannt gebliebenen Ökonomen (vgl. Kapitel \ref{Neoklassik}), durchaus gewürdigt. Allerdings vor allem für sein "`Gesamtwerk"', als omnipräsenter Eisenbahn-Entrepreneur, Journalist und Herausgeber, Politiker und eben auch Ökonom. Aus heutiger Sicht ist es durchaus interessant zu sehen welch unterschiedliche, wirtschaftswissenschaftliche Fragestellungen von List bereits aufgegriffen wurden, sein Hauptwerk \textcite{List1841} wurde auch durchaus häufig zitiert, seine Themen und sein methodischer Zugang galten in der Mainstream-Ökonomie aber bald als überholt.

\section{Ältere Historische Schule}
\label{Hist_Schulen}

Die Hauptvertreter der "`Älteren Historischen Schule"' waren Wilhelm Roscher, Bruno Hildebrand und Karl Knies. Sie alle wurden grob um 1820 geboren, die Gründung der "`Historischen Schule"' fiel damit um die Zeit des Revolutionsjahr 1848, ihre Hauptschaffenszeit in den folgenden Jahrzehnten. Diese Schule einigte vor allem die Ablehnung der englischen Klassik und hierbei die Überzeugung, dass die Wirtschaftswissenschaften nicht wie eine Naturwissenschaft mittels Gesetzmäßigkeiten analysiert werden kann. Vielmehr sei die Wirtschaftswissenschaft eine Sozialwissenschaft, wobei man hier vorsichtig sein muss, da es diese Kategorien damals noch nicht in dieser Form existiert haben. Schließlich gilt die Wirtschaftswissenschaft heute als Teil der Sozialwissenschaften, was sie aber nicht davon abhält so quantitativ wie nie zuvor geprägt zu sein und Gesetzmäßigkeiten in den Wirtschaftswissenschaften definitiv State of the Art sind. Aus heutiger Sicht wird als entscheidender Unterschied vor allem die methodische Zugangsweise gesehen. Dieser Konflikt wurde später als Methodenstreit bekannt und wird weiter unten dargestellt. Es war aber auch bei der "`Älteren Historischen Schule"' schon so, dass der Zugang zur Analyse einer Ökonomie aus heutiger Sicht sehr unkonventionell. Anstatt der heute üblichen Herangehensweise einer ceteris paribus Analyse - also den Zustand mit und ohne einer bestimmten Änderung zu betrachten und so auf die Auswirkung dieser Änderung zu schließen - ist für die deutschen Ökonomen diese Vorgehensweise nicht zielführend \parencite[S. 236]{Rosner2012}. Stattdessen sei jede beobachtete wirtschaftliche Ausgangssituation für sich einzigartig und daher nur im Rahmen der jeweiligen Gegebenheiten zu analysieren. Ein Beispiel: Eine Rezession in England kann niemals mit einer in Deutschland verglichen werden, weil die Gesamtheit der begleitenden Umstände zu unterschiedlich wäre. Dementsprechend macht es auch keinen Sinn wirtschaftspolitische Gegenmaßnahmen, die halfen eine konkrete Rezession zu überwinden, in einem anderen Land oder in einer anderen Zeit usw., anzuwenden. Damit in Verbindung zu sehen ist die "`ganzheitliche"' Betrachtung der Ökonomie. So ließen die Vertreter der Historischen Schule stets auch ethische Aspekte des Wirtschaftsleben und in dessen Analyse einfließen. HIER WEITER Rosner Seite 236,

Das ist ein wichtiger Punkt. Denn betrachtet man alleine die \textit{Themen}, welche die Historische Schule behandelt hat, findet man darin aus heutiger Sicht keine ganz und gar unvernünftige Herangehensweise. Der grundsätzlich liberale Ansatz, allerdings mit Berücksichtigung der hohen Bedeutung eines Staates und dessen Institutionen, aber auch das infrage stellen des rein nutzen-optimierenden Individuums, findet sich ja auch in der modernen wirtschaftswissenschaftlichen Literatur wieder. 


\section{Jüngere Historische Schule}


Vor allem natürlich Methodenstreit mit Menger! (Menger 1883, "`Untersuchungen über die Methoden der Socialwiss. Vera Linß-Buch, Seite 66)

Bekannt geblieben ist die Historische Schule der Nationalökonomie hauptsächlich durch ihre Rolle im Methodenstreit, der zwischen dem uns schon bekannten Carl Menger und den Vertretern der jüngeren Historischen Schule um Gustav Schmoller geführt wurde.



Die historische Schule wird häufig als Vorläufer des Institutionalismus gesehen.
Pierenkamp Toni, Buch, Seite 182.


https://muse.jhu.edu/article/13303


https://muse.jhu.edu/pub/4/article/13173/summary




Fragen der Historischen Schule durchaus von Bedeutung, aber wissenschaftlicher Zugang nicht Zielführend. \textcite[S. 210]{Rosner2012} führt in diesem Zusammenhang an, dass die Fragen heute zum Beispiel von den Vertretern des Neuen Institutionalismus, oder der Politischen Ökonomie diskutiert werden. Trotzdem keine Vorläufer, weil die Erklärungsansätze ganz andere sind.


