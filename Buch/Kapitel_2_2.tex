%%%%%%%%%%%%%%%%%%%%% chapter.tex %%%%%%%%%%%%%%%%%%%%%%%%%%%%%%%%%
%
% sample chapter
%
% Use this file as a template for your own input.
%
%%%%%%%%%%%%%%%%%%%%%%%% Springer-Verlag %%%%%%%%%%%%%%%%%%%%%%%%%%

\chapter{Auf dem Historikerweg}
\label{Historisch}

Deutsche Wissenschaftler waren um die Jahrhundertwende in sehr vielen Disziplinen in Wissenschaft und Technik führend. Deutsche Ökonomen aus dieser Zeit sind hingegen heute kaum noch bekannt. Als Grund gilt heute, dass die damalige Mainstream-Ökonomie in Deutschland - die "`Historische Schule der Nationalökonomie"' - sich aus heutiger Sicht als "`wissenschaftliche Sackgasse"' \parencite[S. 229]{Rosner2012} erwies. Die Mainstream-Ökonomie in Deutschland befand sich also über ein halbes Jahrhundert lang quasi auf dem Holzweg. Schon der Name "`Historische Schule der Nationalökonomie"' ist irreführend, wie schon \textcite[S. 775]{Schumpeter1954} anmerkte. Zwar arbeiteten viele Vertreter der "`Historischen Schule"' tatsächlich mit historischen Beispielen, ihr Ansatz und vor allem ihre Intention geht aber durchaus über eine reine Beschreibung vergangener Ereignisse hinaus. 

Wir erinnern uns an Kapitel \ref{Marginalismus}, dass wichtige Entwicklungen der Neoklassik ursprünglich von deutschen Ökonomen, nämlich Hermann Heinrich Gossen, Johann Thünen, aber auch Karl Heinrich Rau, vorweggenommen wurden. Ihre Arbeiten blieben aber unentdeckt. Dies eben vor allem deshalb, weil die universitäre Ökonomie in Deutschland mit der "`Historischen Schule"' eine ganz andere Hauptrichtung vertrat. Dies ist vor allem auch in Hinblick auf die benachbarte damalige Donaumonarchie Österreich-Ungarn interessant, die im Gegensatz dazu so viele heute noch bedeutende Ökonomen, sowie die Österreichische Schule der Nationalökonomie hervorbrachte. Die nachhaltig wirkende Volkswirtschaftslehre im deutschsprachigen Raum blieb also lange Zeit auf den Raum Wien beschränkt \parencite[S. 339]{Rosner2012}, obwohl die deutschen Vordenker den Neoklassiker Carl Menger entscheidend beeinflussten \parencite{Streissler1990}. Aufgrund der Dominanz der "`Historischen Schule"' in der universitären Lehre konnten Neoklassiker und deren Vorläufer in Deutschland lange Zeit weder selbst Fuß fassen, noch wurden sie später im 19. Jahrhundert entsprechend gewürdigt.

Was waren die Gründe dafür, dass sich in Deutschland Klassik und Neoklassik erst im 20. Jahrhundert durchsetzen konnten? Die Ideen von Smith wurden schon um 1800 nach Deutschland gebracht und dort diskutiert. Allerdings war das Deutschland des frühen 19. Jahrhunderts gänzlich anders organisiert, als Großbritannien zur gleichen Zeit. Erstens, gab es absolutistische Fürstentümer, die liberale Ideen, sowie wirtschaftlichen Freihandel schon an sich utopisch erscheinen ließen. Zweitens, war die Landwirtschaft noch der wichtigste Produktionszweig. Drittens war das Handwerk den Mitgliedern der Zünfte vorbehalten. Es gab also umfangreiche Marktzugangsbeschränkungen, wie man heute sagen würde \parencite[S. 213f]{Rosner2012}. Zudem war in Deutschland - das gilt zumindest bis zum Revolutionsjahr 1848 - eine freie Diskussion über Wirtschaftspolitik nicht möglich. So wurde Friedrich List (vgl. Unterkapitel \ref{List}) aus politischen Gründen zu einer Haftstrafe verurteilt. Adam Müller - wie eben beschrieben - aus Dresden ausgewiesen und Bruno Hildebrand (vgl. Unterkapitel \ref{Hist_Schulen}) wegen der Einfuhr kritischer Arbeiten ebenfalls inhaftiert \parencite[S. 211]{Rosner2012}. Es war in Deutschland lange also gar nicht möglich die Wirtschaftstheorie von Smith im Hinblick auf wirtschaftspolitische \textit{Anwendungen} zu diskutieren. Stattdessen war der Hauptvertreter der "`Romantischen Schule"', Adam Müller einer der einflussreichsten deutschen Ökonomen des frühen 19. Jahrhunderts. Seine Arbeiten beeinflussten sowohl die Vertreter der Historischen Schule, als auch Karl Marx. Er wurde aus politischen Gründen aus Dresden ausgewiesen und war in der Folge in Wien tätig. Sein Werk ist heute völlig unbedeutend, er hob die Bedeutung der Gesellschaft als Ganzes im Produktionsprozess hervor und stellte sich damit gegen den Individualismus bei Smith. Seine Kritik an der Ökonomie des Adam Smith blieb in Deutschland verwurzelt. Müller ging aber noch weiter, er befürwortete eine Zunft-artige Ordnung der Wirtschaft und lehnte eine freie Marktwirtschaft als solche gänzlich ab \parencite[S. 24ff]{Rosner2012}. 

\section{Vorläufer der Historischen Schule}
\label{List}

Friedrich List ist einer der bekanntesten deutschen Ökonomen des 18. Jahrhunderts. Seine spannende Biographie ist von Höhen und Tiefen geprägt. Er war sein Leben lang vor allem eines: ein tatkräftiger Mann. Bis 1822 machte er eine beeindruckende Karriere als Beamter, Professor an der Universität Tübingen und schließlich auch als Abgeordneter in Württemberg. Die Stelle als Professor erhielt List, der nie ein ordentliches Studium abgeschlossen hatte, durch seine Arbeit für einen befreundeten Minister \parencite[S. 227]{Hauser1989}. Aus heutiger Sicht also eher eine zweifelhafte Ehre. Nichtsdestotrotz legte er schon zu dieser Zeit enorme - wenn auch letztendlich nutzlose - Bemühungen an den Tag, die damals üblichen Binnenzölle zwischen den vielen deutschen Staaten abzuschaffen und stattdessen einen Außenzoll einzuführen. Zu dieser Zeit wohlgemerkt noch ohne bedeutende theoretische Schriften diesbezüglich zu verfassen. 1822 folgte allerdings die erste Wende in seinem Leben. Er forderte per Flugblatt, kurz gesagt, mehr Bürgerrechte für die Menschen ein. Sein Parlamentssitz bewahrte ihn - während dieser realpolitisch doch recht absolutistischen Zeit - nicht vor Auslieferung und Verurteilung. Er wurde zu zehn Monaten Haft verurteilt, wobei das Urteil auch die Enthebung von all seinen Ämtern umfasste und es de facto unmöglich machte, dass List in einem deutschen Staat je wieder ein öffentliches Amt bekleiden könne \parencite[S. 229]{Hauser1989}. List, der in vielleicht in der kurzen Darstellung wie heldenhafter Revolutionär wirkte, war in Wirklichkeit eher ein patriotischer Beamter, eigentlich treu der Monarchie und dem König. Die Hoffnung auf Gnade erwies sich als unberechtigt. Nach zweijähriger Flucht, kehrte er nach Württemberg zurück um dort seine Haftstrafe abzusitzen um danach erst recht des Landes verwiesen zu werden. Er wanderte 1925 in die USA aus, wo er bald als Herausgeber einer Zeitung und als Eisenbahn-Pionier, Erfolge erzielte, zu Wohlstand und Ansehen kam und zudem 1830 die amerikanische Staatsbürgerschaft erlangte. In den USA verfasste er auch erstmals ökonomisch bedeutende Schriften, in denen er die Idee der Schutzzölle erstmals entwickelte. Sein Schicksal war wohl seine Sehnsucht nach Deutschland, wohin er immer zurückkehren wollte. Dies gelang ihm schließlich auch mit Hilfe des US-amerikanischen Präsidenten, der ihn als Konsul einsetzte. Nach erneuten Problemen dabei trat List im Jahre 1832 tatsächlich seine Stelle als Konsulat für Baden an \parencite[S. 232]{Hauser1989}. In weiterer Folge wollte er in Deutschland die Eisenbahn etablieren. Er war dabei auch führend eingebunden, sein Arbeit wurde jedoch wiederholt nicht belohnt. Weder finanziell - er wurde kein Anteilseigner der Bahnprojekte - noch im Hinblick auf eine einflussreiche Position, die ihm wiederholt verwehrt wurde. Erst nach dem Scheitern im Eisenbahngewerbe, Ende der 1830er Jahre, begann er sein Hauptwerk \parencite{List1841}, das schließlich als "`Das nationale System der politische Oekonomie"' veröffentlicht wurde. Dem Werk war unmittelbarer Erfolg beschieden, er kam zu erneutem Ansehen, hatte eine  Audienz beim König und schließlich erhielt er auch seine "`bürgerlichen Ehrenrechte"' zurück \parencite[S. 235]{Hauser1989}. Eine feste berufliche Position erlangte er aber nicht mehr und musste weiterhin von seinen unsicheren journalistischen Tätigkeiten leben, dabei war er als Berater im In- und Ausland durchaus gefragt. Der rastlos bleibende List litt zunehmend an körperlichen Gebrechen und verfiel in Depressionen. Ende 1846 erschoss er sich auf der Durchreise in Kufstein in Tirol, wo er auch begraben wurde und bis heute ein Denkmal an ihn erinnert. "`Die Tragik seines Lebens bestand darin, dass es nach seiner Wirkung überaus erfolgreich und folgenreich und dennoch, für List persönlich, voller Misserfolge gewesen ist"', fasst \textcite[S. 237]{Hauser1989} zusammen.
Das ökonomische Gesamtwerk von List umfasst vor allem drei wesentliche Punkte. Erstens, die Schutz- und Erziehungszölle, zweitens, die Bedeutung einer Organisation, zum Beispiel in Form einer Nation, für die langfristige ökonomische Entwicklung eines Staates und drittens, die Evolution von Nationen in Entwicklungsstufen. Vor allem die ersten beiden Punkte wurden als Antwort und Kritik an die Werke Smith' und Say's (vgl. Kapitel \ref{Klassik}). Am bekanntesten ist sicher seine Arbeit zu den Schutz- und Erziehungszöllen, die allerdings nicht originär auf \textcite{List1840, List1841} zurückgehen, sondern auch in den USA von Alexander Hamilton schon beschrieben wurden \parencite[S. 243]{Hauser1989}. Demnach können sich geringer entwickelte Staaten nicht aus eigener Kraft auf das Entwicklungslevel des führenden Staates bringen, weil dessen Industrie das "`Entwicklungsland"' ständig mit billigeren Industriegütern beliefert, sodass der Aufbau einer eigenen Industrie nicht gelingen kann. Seine empirische Grundlage für diese Theorie bestand aus seinen Erfahrungen die er in Deutschland, Frankreich und den USA sammelte. Die damalige Hegemonialmacht Großbritannien verhinderte demnach eben die Entwicklung der Industrie in diesen Staaten. Schutz- und Erziehungszölle sollten dem entgegenwirken. \textcite{List1841} stellt sich damit gegen den Freihandel, den eben der Brite Smith forderte. Die viel später entstandene "`Neoklassische Wachstumstheorie"' (vgl. Kapitel \ref{sec: Wachstum}) widerspricht der Theorie List's aus theoretischer Sicht und geht stattdessen davon aus, das Kapital gerade dort eingesetzt wird, wo Arbeit im Verhältnis zu Kapital in hohem Ausmaß verfügbar und damit billiger ist, was in Entwicklungsländern eher der Fall ist. Interessanterweise greift \textcite{List1841} mit dem Ansatz, hoch-entwickelte Staaten würden das Wachstum in weniger-entwickelten Nationen bremsen, einem zentralen Punkt in der Entwicklungsökonomie, Mitte des 20. Jahrhunderts vor. Die Vertreter der Dependenztheorie (vgl. Kapitel \ref{sec: Wachstum}) sehen dies nämlich genauso. Dennoch beziehen sich die Vertreter der Dependenztheorie kaum auf \textcite{List1841}, was an dessen Einschätzung liegen könnte, die Länder der "`heißen Zone"' haben kein eigenständiges Entwicklungspotenzial \parencite[S. 96]{Bachinger2005}. Auch der, vor allem in \textcite{List1840}, verarbeitete Ansatz, die Ökonomie entwickle sich in verschiedenen Stufen, wurde im 20. Jahrhundert vom Modernisierungstheoretiker \textcite{Rostow1960} ebenfalls im Rahmen der Entwicklungsökonomie vorgebracht.
Die Bedeutung von Organisationen ist der zweite wesentliche Beitrag von \textcite{List1841}. Insgesamt wurde damit natürlich der von Smith so stark hervorgehobene Individualismus kritisiert. Bei \textcite{List1841} ist die Gesamtleistung einer Gemeinschaft - gemeint wurde dabei eine Nation - mehr als die Summe der individuellen Gewinne der Einzelnen.  \textcite{List1841} stellt sich damit gegen die Ansicht, dass maximaler Nutzen dann erreicht wird, wenn jedes Individuum egoistisch handelt. Er befürchtet vor allem eine Vernachlässigung der langfristigen Ziele einer Volkswirtschaft, wie das Aufrechterhalten von Gesundheitseinrichtungen und Bildungseinrichtungen. Die Thematik wurde in der Klassik, vielmehr aber vor allem später in der Neoklassik in der Wirtschaftstheorie tatsächlich lange Zeit kaum behandelt, obwohl deren realpolitische Bedeutung selbstverständlich stets von Interesse war. Zunächst die Institutionenökonomik und aktuell vor allem der Neue Institutionalismus (vgl. Kapitel \ref{Neue Institut}) beschäftigen sich eingehend mit der Bedeutung von Organisationen. Einer der führenden Vertreter des Neuen Institutionalismus, Douglass \textcite[S. 99]{North1991}, verweist auf die Historische Schule. List aber als deren Vorläufer darzustellen würde zu weit gehen, zu rudimentär waren seine Ausführungen und seine methodischen Ansätze um dies zu behaupten.

List wurde nach seinem Tod in Deutschland, im Gegensatz zu manch unbekannt gebliebenen Ökonomen (vgl. Kapitel \ref{Neoklassik}), durchaus gewürdigt. Allerdings vor allem für sein "`Gesamtwerk"', als omnipräsenter Eisenbahn-Entrepreneur, Journalist und Herausgeber, Politiker und eben auch Ökonom. Aus heutiger Sicht ist es durchaus interessant zu sehen welch unterschiedliche, wirtschaftswissenschaftliche Fragestellungen von List bereits aufgegriffen wurden, sein Hauptwerk \textcite{List1841} wurde auch durchaus häufig zitiert, seine Themen und sein methodischer Zugang galten in der Mainstream-Ökonomie aber bald als überholt.

\section{Ältere Historische Schule}
\label{Hist_Schulen}

Die Hauptvertreter der "`Älteren Historischen Schule"' waren Wilhelm Roscher, Bruno Hildebrand und Karl Knies. Sie alle wurden grob um 1820 geboren, die Gründung der "`Historischen Schule"' fiel damit in die Zeit um das Revolutionsjahr 1848, ihre Hauptschaffenszeit in den folgenden Jahrzehnten. Wobei es irreführend ist bei den genannten von einen einheitlichen Schule zu sprechen, verfolgten doch alle drei recht voneinander unabhängige Forschungstätigkeiten \parencite[S. 121]{Pierenkemper2012}. Diese Generation einigte vor allem die weitgehende Ablehnung der englischen Klassik und hierbei die Überzeugung, dass die Wirtschaftswissenschaften nicht wie eine Naturwissenschaft mittels Gesetzmäßigkeiten analysiert werden könne. Vielmehr sei die Wirtschaftswissenschaft eine Sozialwissenschaft. Bei dieser Aussage muss man allerdings vorsichtig sein, da es diese Kategorien damals noch nicht in dieser Form existiert haben. Schließlich gilt die Wirtschaftswissenschaft heute als Teil der Sozialwissenschaften, was sie aber nicht davon abhält so quantitativ wie nie zuvor geprägt zu sein und Gesetzmäßigkeiten in den Wirtschaftswissenschaften definitiv State of the Art sind. Aus heutiger Sicht wird als entscheidender Unterschied vor allem die methodische Zugangsweise gesehen. Dieser Konflikt wurde später als Methodenstreit bekannt und wird weiter unten dargestellt. Es war aber auch bei der "`Älteren Historischen Schule"' schon so, dass der Zugang zur Analyse einer Ökonomie aus heutiger Sicht sehr unkonventionell war. Anstatt der heute üblichen Herangehensweise einer ceteris paribus Analyse - also den Zustand mit und ohne einer bestimmten Änderung zu betrachten und so auf die Auswirkung dieser Änderung zu schließen - ist für die deutschen Ökonomen diese Vorgehensweise nicht zielführend \parencite[S. 236]{Rosner2012}. Stattdessen sei jede beobachtete wirtschaftliche Ausgangssituation für sich einzigartig und daher nur im Rahmen der jeweiligen Gegebenheiten zu analysieren. Ein Beispiel: Eine Rezession in England kann niemals mit einer in Deutschland verglichen werden, weil die Gesamtheit der begleitenden Umstände zu unterschiedlich wäre \parencite[S. 32]{Balabkins1993}. Dementsprechend macht es auch keinen Sinn wirtschaftspolitische Gegenmaßnahmen, die halfen eine konkrete Rezession zu überwinden, gesetzmäßig in einem anderen Land oder in einer anderen Zeit usw., anzuwenden. Damit in Verbindung zu sehen ist die "`ganzheitliche"' Betrachtung der Ökonomie. So ließen die Vertreter der Historischen Schule stets auch ethische Aspekte des Wirtschaftsleben in deren Analyse einfließen. Roscher zum Beispiel analysierte die sozialen und ökonomischen Probleme stets als zusammenhängende Probleme. Er war es auch, der primär historische Beispiele und Ansätze zur Lösung der aktuellen Probleme heranzog \parencite[S. 33]{Balabkins1993}. In \textcite{Roscher1843} führt er in vier Punkten an was unter "`historischer Methode"' gemeint ist \parencite[S. 123]{Pierenkemper2012}: Erstens, das Verhältnis der Völker und ihrer Wirtschaft zu analysieren, zweitens, auch deren Vergangenheit miteinzubeziehen, drittens, aus der "`großen Masse von Erscheinungen das Wesentliche, Gesetzmäßige"' ziehen und viertens, die Geschichte verschiedener Völker zu vergleichen. \textcite{Roscher1854, Roscher1843} war in seinen wirtschaftspolitischen Empfehlungen zurückhaltend und vage, \textcite{Hildebrand1948} wiederum sah die Wirtschaftswissenschaften als keine ausschließlich akademische Disziplin \parencite{Balabkins1993}. Sein Hauptwerk umfasste die typischen Lösungsansätze der "`Älteren Schule der Historischen Schule"': Der englische "`Laissez-faire"'-Ansatz wird abgelehnt, stattdessen wird für eine Versöhnung zwischen Arbeit und Kapital plädiert. Sein Ansatz ist sehr praxisnah, verurteilte Spekulation und trat für Sozial-Gesetzgebung ein \parencite[S. 33]{Balabkins1993}. Abgelehnt wurde hingegen der radikal-sozialistische Ansatz, zu dem zu dieser Zeit - das "`Manifest der Kommunistischen Partei"' wurde ebenfalls 1848 veröffentlicht - erste theoretische Arbeiten publiziert wurden. So zweifelte er an der sozialistischen Prognose der Verarmung breiter Bevölkerungsschichten als Folge der Industrialisierung. Stattdessen erkennt er mit Hilfe seiner historischen Analyse im dadurch ausgelösten Wirtschaftswachstum eine mögliche Entspannung des Armutsproblems \parencite[S. 122]{Pierenkemper2012}. Etwas später, nämlich im Jahr 1862 gründet Bruno Hildebrand schließlich das Journal "`Jahrbücher für Nationalökonomie und Statistik"', das bis heute veröffentlicht wird. Darin veröffentlichte er auch seine Form der Stufentheorie, nach der sich Ökonomien zuerst als Naturalwirtschaft, danach als Geldwirtschaft und schließlich als Kreditwirtschaft darstellen lassen. 

Betrachtet man also alleine die \textit{Themen}, welche die Historische Schule behandelt hat, findet man darin aus heutiger Sicht keine ganz und gar unvernünftige Herangehensweise. Der grundsätzlich Ansatz, sowohl jener von Roscher als auch von der Hildebrand ist durchaus wirtschaftsliberal, allerdings mit Berücksichtigung der hohen Bedeutung eines Staates und dessen Institutionen. Diese Themen finden sich ja auch später wieder, wie zum Beispiel die Marktversagensformen in der Wohlfahrtstheorie (vgl. Kapitel \ref{Wohlfahrt}) Auch in der heute durchaus noch aktuellen wirtschaftswissenschaftlichen Literatur wie dem Neuen Institutionalismus werden ähnliche Themen behandelt (vgl. Kapitel \ref{sec: Neue Inst}). Die methodische Herangehensweise ist aus heutiger Sichtweise allerdings eher naiv. Die gemeinsame Betrachtung von ökonomischen und ethischen Problemstellungen führt dazu, dass stets Werturteile in die Analyse mit einfließen, was natürlich die objektive Aussagekraft derselben fragwürdig macht. Insgesamt vertritt die "`Ältere historische Schule der Nationalökonomie"' eine interessante Position zwischen den aufstrebendem frühen Marxismus und der Klassik. Der philosophisch-ethisch-ökonomische Zugang war aber schon im 19. Jahrhundert unter den anderen ökonomische Schulen wenig akzeptiert.

\section{Jüngere Historische Schule}

\textit{Der} Vertreter der "`Jüngeren Historischen Schule"' war sicherlich Gustav von Schmoller. Er wurde 1838 geboren und seine Haupt-Schaffenszeit war um die 1870er Jahre. Diese fällt damit in eine Zeit des extremen Wandels in Deutschland. Mit 1871 entstand das geeinte Deutsche Reich als Nationalstaat mit dem Kanzler Otto von Bismarck. Gleichzeitig setzte die Zeit der Hochindustrialisierung ein und Deutschland schloss allmählich wirtschaftlich zu Großbritannien auf. Damit entstanden Industrieunternehmen auf der einen Seite und eine große Lohn-Arbeiterschaft auf der anderen Seite, womit aber auch soziale Fragen zunehmend in den Vordergrund rückten. In diesem Umfeld hatte Gustav von Schmoller einen enormen und nachhaltigen Einfluss auf die Entwicklung der Wirtschaftswissenschaften in Deutschland, der im Nachhinein häufig sehr kritisch gesehen wird. 

Schmoller studierte "`Kameralwissenschaften"' in Tübingen, besuchte aber auch philosophische und historische Vorlesungen. Bei ihm findet man die Verbindung zwischen Ökonomie und Geschichte also sehr früh und über sein ganzes Werk betrachtet, sehr deutlich ausgeprägt vor. Ab den frühen 1860er Jahren publiziert er wissenschaftliche Schriften und 1864 erhält er eine erste Professur in Halle \parencite[S. 100]{Winkel1989}. Sein dort entstandenen Werke sind symptomatisch, was man schon an deren Titel erkennt. \textcite{Schmoller1870}: "`Zur Geschichte der deutschen Kleingewerbe im 19. Jahrhundert"' und \textcite{Schmoller1879}: "`Die Straßburger Tucher- und Weberzunft"'. In seinen Werken betrachtet er einzelne historische Fallbeispiele, wie eben die Tucher- und Weberzunft, und leitet daraus Kritikpunkte ab. Wobei er sowohl den englischen Liberalismus als auch den Sozialismus ablehnte. Ähnlich wie bereits für die "`Ältere Historische Schule"' beschrieben leitete auch er daraus einen durchaus interessanten Mittelweg ab, der die Fähigkeiten des Marktes schätzte, aber auch die Wichtigkeit eines wirtschafts- und sozialpolitische eingreifenden Staates beinhaltete. Seine wissenschaftliche Herangehensweise dabei wirkt aus heutiger Sicht aber geradezu jenseitig. Zieht man als durchaus repräsentatives Beispiel \textcite{Schmoller1879} heran, so findet man darin vor allem ausschweifende Erzählungen über die Lebensverhältnisse der Weberzunft, garniert mit erstaunlich viel Zahlenmaterial. Dieses ist aber nicht systematisch in Kapiteln unterteilt, oder Tabellen dargestellt, sondern wird im Text wiedergegeben \parencite[z.B.: S. 199]{Schmoller1879}. Dazwischen finden sich Kritik oder Befürwortung einzelner Institutionen oder wirtschaftspolitischer Maßnahmen. Erst beim auszugsweisen Lesen einiger Passagen von Werken von Vertretern der historischen Schule, kann man die vehemente Kritik aus verschiedenen anderen ökonomischen Richtungen verstehen. Zwar erscheint das Gesamtkonzept mit einem Sozialstaat bei gleichzeitiger marktwirtschaftlicher Organisation verlockend, doch hat man wenig Idee, wie ein Gesamtkonzept bei den "`Historikern"' aussehen könnte. Die oft als "`ethisch-historische"' Forschung beschriebene Methodik erscheint aus heutiger Zeit nicht mehr wie eine Forschungsmethode, sondern eher wie das bloße Zusammentragen empirisch-historischer Einzelbeobachtungen. Wobei der Umfang der Werke und der "`Einzelgeschichten"' durchaus beeindruckende ist. Oft wird behauptet, die "`historische Schule"' vertrat eine induktive Forschungsmethoden. Zwar werden tatsächlich Einzelbeobachtungen herangezogen, allerdings fehlt der systematische Ansatz der Verarbeitung dieser Beobachtungen um von induktiver Forschung im modernen Sinne sprechen zu können. Dass die "`Historische Schule"' das "`höchste Ziel"', nämlich eine "`geschlossene Erklärung des Wirtschaftsprozesses heute noch nicht erreichen kann"', erkannte selbst \textcite{Schmoller1900, Schmoller1904} in seinem späten Hauptwerk \parencite[S. 110]{Winkel1989}. Dementsprechend hatte die "`Historische Schule"' schon unter Zeitgenossen einen schweren Stand. Durch den Journalisten Heinrich Oppenheim wurde von liberaler Seite die spöttische Bezeichnung "`Kathedersozialisten"' geprägt \parencite[S. 102]{Winkel1989}. Karl \textcite[S. 361]{Marx1962} verspottete deren Ansätze als "`kindischen Kohl"' und bekannt geblieben ist die Historische Schule der Nationalökonomie hauptsächlich durch den Methodenstreit, der zwischen dem uns schon bekannten Carl Menger (vgl. Kapitel \ref{Wiener Schule}) und Gustav Schmoller geführt wurde. Auslöser war das Werk von \textcite{Menger1883}, in welchem er die eigene Theorie-geleitete Herangehensweise über jene der "`Historiker"' stellte, die vor allem ökonomische Einzelfälle analysierten. Heute spricht man diesbezüglich oft von der damaligen Konkurrenz zwischen abstrakt-deduktiven und empirisch-historischen Untersuchungen\footnote{Die moderne ökonomische Forschung ist zweifellos dominiert von der deduktiven Methode. Dabei wird von einer allgemein-gültigen Theorie ausgegangen, mit der man einzelne Beobachtungen möglichst gut erklären kann. Heute wird das natürlich ergänzt mit umfangreichen empirischen Untersuchungen, die auf statistische, bzw. ökonometrische Verfahren zurückgreifen. Eine absolute Außenseiterstellung in der Ökonomie nimmt heute die induktive Forschung ein, bei der Einzelbeobachtungen detailliert untersucht werden, um aus diesen allgemeingültige Aussagen zu generieren, oder Theorien zu erstellen.}. Das Buch wurde von Schmoller negativ rezensiert, was wiederum dazu führte, dass \textcite{Schmoller1904} - gelinde gesagt - noch eindeutiger "`die Irrtümer des Historismus"' beschrieb. Weitere Vertreter auf beiden Seiten, also der frühen "`Österreichischen Schule"' aber auch der "`Historiker"' stiegen in die Debatte ein \parencite[S. 782]{Schumpeter1954}. Einen eindeutigen Sieger, in dem Sinn, dass der Verlierer seine Niederlage eingesteht, gibt es typischerweise bei solchen Debatten nicht. Die Angriffspunkte der "`Historiker"' in Bezug auf die faktische Unmöglichkeit eines rein rational denkenden Menschen, oder die Undurchführbarkeit einer ceteris paribus Untersuchung, erscheinen auf den ersten Blick nachvollziehbar und gewissermaßen auch sympathisch. Diese Kritikpunkte werden der Mainstream-Ökonomie ja bis heute gerne vorgeworfen. Aber die Alternative, nämlich auf eine möglichst allgemeine Theorie gänzlich zu verzichten, verbessert die resultierenden Forschungsaussagen nicht. Im Nachhinein kann man feststellen, dass der Methodenstreit der historischen Schule geschadet hat. Anstatt die eigenen Forschungsmethodiken als Alternative zu jenen der Neoklassiker zu etablieren, kritisierte man deren Ansätze vehement. 

Die allgemeine Lehrmeinung ist, dass mit dem Tod von Schmoller die Zeit der "`Historischen Schule"' und auch der Methodenstreit vorbei war. Manche Wirtschaftshistoriker sehen im bedeutenden Soziologen Max Weber eine "`Jüngste Historische Schule"' heranwachsen und im sogenannten Werturteilsstreit eine Fortsetzung des Methodenstreites bis in die Mitte des 20. Jahrhunderts. Eher eine Mindermeinung ist schließlich die Identifikation der Lehren Walter Eucken's und der frühen Freiburger Schule als Fortsetzung der "`Historischen Schule"'. Als Berührungspunkt kann aber nur hohe Bedeutung des Sozialstaates in beiden Schulen gesehen werden (vgl. Kapitel \ref{Neoliberalismus}). 

Bleibt die Frage, warum vor allem die "`Jüngere Historische Schule"' und hier vor allem Gustav von Schmoller ein so großer Einfluss auf die Volkswirtschaftslehre und -politik attestiert wird? Dazu muss man zunächst den wissenschaftliche Zugang der "`Historiker"' betrachten. Umstritten ist bis heute die immer wieder erwähnte "`Theorie-Feindlichkeit"', vor allem der "`Jüngeren Historischen Schule"'. In manchen Passagen seiner Arbeit - und sogar in Bezug auf den Methodenstreit - akzeptiert Schmoller die deduktive Methode, also grob gesagt das Ausbreiten einer allgemeinen Theorie auf Einzelfälle \parencite[S. 108]{Winkel1989}. An anderer Stelle wieder äußert er sich eindeutig ablehnend gegenüber jeglicher allgemeingültiger Theorie. Faktum ist, dass die "`Jüngere Historische Schule"' keine geschlossene Theorie vorlegen konnte. Manche Wirtschaftshistoriker vertreten die Meinung, dass die "`Historische Schule"' sehr wohl eine allgemeine Theorie als Ziel ihrer Arbeit im Auge hatte. Ihre Arbeiten wären demnach "`nur"' das Zusammentragen von vorbereitendem Material für den großen Wurf gewesen. Die hohe Komplexität aus wirtschaftlich-ethisch-historischen Ansätzen verhinderte aber schlussendlich das entstehen dieses "`großen Wurfes"'. Die hohe Wirkung und das hohe Prestige, das Schmoller zu seinen Lebzeiten genoss wären damit zu erklären. Im Sinne von "`Wir wissen, dass die Theorien der Sozialisten und der Klassiker falsch sind. Unsere Werke sind Vorarbeiten für eine kommende, große Theorie, die wir am Tag X präsentieren werden"'. Der Tag X war aber eben stets nur ein vager Punkt in der Zukunft \parencite[S. 115]{Winkel1989}. Ein Ansatz, den übrigens auch \textcite[S. 775]{Schumpeter1954} vertrat, der ein recht positives Gesamtbild von den "`Historikern"' zeichnete. Im jungen Deutschen Reich mit der zunehmenden Industrialisierung schienen weder die Klassiker noch die "`Ältere Historische Schule"', aber auch nicht die Sozialisten richtige Antworten auf die neu entstandenen Fragen geben zu können. In dieses "`Gelehrtenvakuum"' stießen die "`Historiker"' auf zwei Wege. Erstens, über den "`Verein für Socialpolitik"'. Dieser wurde 1872 gegründet, wobei Bruno Hildebrand, Adolph Wagner und eben auch Gustav Schmoller federführend waren. Zweitens, durch ihre Besetzungspolitik in Bezug auf Professorenstellen. \textcite[S. 112]{Winkel1989} beschreibt, dass Schmoller's "`Lehrstuhlpolitik"' die wissenschaftlichen Karrieren von vielen Verfechtern konkurrierender ökonomischen Schulen verhindert hat. Was wohl dazu führte, dass die deutschen Wirtschaftswissenschaften in der Zeit um 1900 kaum einen heute bedeutenden Volkswirten hervorbrachte. Unter anderem dieser Punkt rückte ihn bald nach seinem Tod in ein Zwielicht. Andere deutsche Wirtschaftswissenschaftler beklagten später "`die schwere Schädigung, die die ökonomische Wissenschaft in Deutschland"' durch Schmoller erfuhr \parencite[S. 116]{Winkel1989}. Schmoller hatte zu Lebzeiten einen ausgezeichneten Ruf als Wissenschaftler und allgemein als Gelehrter, bis heute wird "`Schmollers Jahrbuch"' als wirtschafts- und sozialwissenschaftliche Fachzeitschrift veröffentlicht. Die Geschehnisse nach seinem Tod im Jahr 1917 - konkret die Depression und Hyperinflation in Deutschland in den 1920er Jahren - auf die die "`Historiker"' keinerlei Lösungsansätze parat hatten, ließen seinen Ruhm als führender deutscher Ökonom aber rasch verblassen.    

Umstritten ist bis heute der Einfluss der "`Historischen Schule"' bei der Entwicklung des Sozialstaates Bismarck'scher Prägung. Faktum ist, dass der Sozialstaat, wie er heute noch in Kontinental-Europa gelebt wird, seine Ursprünge im Deutschland jener Zeit hat, in der die Vertreter der "`Historischen Schule"' praktisch sämtliche Ökonomie-Lehrstühle innehatten. Laut \textcite[S. 103]{Winkel1989} bezeichnete sich Bismarck gegenüber Schmoller selbst als "`Kathedersozialist"'. Derselbe schreibt aber auch, dass der wirtschaftspolitische Einfluss des Vereins für Socialpolitik und Schmollers im Speziellen als "`größer erachtet wird als er tatsächlich war"' \parencite[S. 113]{Winkel1989} und weiter, "`dass Bismarck und die Kathedersozialisten ihre sozialpolitischen Vorstellungen nicht  mit- sondern nebeneinander entwickelte"' \parencite[S. 114]{Winkel1989}. Schmoller's Leistung bestand demnach eher im "`wissenschaftlicher Flankenschutz"' für die Bismarck und weniger darin direkt wirtschaftspolitische Empfehlungen geben zu können \parencite[S. 117]{Winkel1989}.

Insgesamt ist die Einordnung der wissenschaftlichen Bedeutung der "`Historischen Schule"' extrem schwierig. Man merkt dies auch beim Durcharbeiten der wirtschaftsgeschichtlichen Literatur zu diesem Thema. So ergibt sich die paradoxe Situation, dass die "`Historiker"' keine einheitliche Theorie vorgelegt haben, die man einordnen könnte, während ihr großer Einfluss auf die deutsche Nationalökonomie zumindest zwischen 1870 und 1918 unumstritten enorm hoch war. Dementsprechend unterschiedlich fallen auch die Bewertungen durch Wirtschaftshistoriker aus. Die zeitgenössischen Ökonomen anderer Schulen blickten fast spöttisch auf die "`Historiker"' hinab. \textcite{Schumpeter1954} beurteilt sie hingegen tendenziell eher wohlwollend. Spätere deutsche Ökonomen kritisierten die "`Historiker"' wiederum teils auf das Schärfste und sehen diese verantwortlich dafür, dass deutsche Ökonomen noch nach 1945 kaum an den international führenden Ökonomie-Hochschulen Fuß fassen konnten \parencite[S. 115f]{Winkel1989}. \textcite{Pearson1999} zweifelte die Einordnung der "`Historischen Schule"' als "`Historiker"' an, was wiederum \textcite{Caldwell2001} versuchte zu widerlegen. In jüngerer Zeit wurden außerdem Themen, welche die "`Historiker"' behandelten wieder aufgegriffen. \textcite[S. 210]{Rosner2012} führt in diesem Zusammenhang den "`Neuen Institutionalismus"' (vgl. Kapitel \ref{Neue Institut}), und die "`Neue Politischen Ökonomie"' an (vgl. Kapitel \ref{Pol_Econ}). So heben zum Beispiel \textcite{Wischermann1993} oder auch \textcite{Plumpe1999} hervor, dass die Ansätze des Neuen Institutionalismus, bzw. der Neuen Wirtschaftsgeschichte - konkret die Ansätze von Douglas North \parencite[S. 243]{Wischermann1993}, \parencite[S. 257]{Plumpe1999} - schon von den Vertretern der historischen Schule vorgebracht wurden. Faktum ist, dass die Historische Schule die Bedeutung des Staates und seiner Institutionen stets hervorhob und außerdem die ethischen Aspekte des Wirtschaftslebens auf gleicher Stufe wie die Nutzenmaximierung stellte. \textcite{Richter1996} vergleicht die Ansätze recht ausführlich dahingehend. Augenscheinlich ist, erstens, dass die Fragen, die bereits die Vertreter der "`Historischen Schule"' behandelten noch heute - oder heute wieder - aktuell sind. Und zweitens, dass zumindest bis zur Jahrtausendwende doch einiges an neuer Literatur bezüglich der Einordnung und Wirkung der "`Historischen Schule"' entstanden ist. Insgesamt wird ihre Bedeutung - auch als Vorläufer des "`Neuen Institutionalismus"' - heute als gering eingeschätzt. Vor allem weil die methodischen Wege und Erklärungsansätze ganz andere waren als heute üblich. Dies wird vor allem beim direkten Blick in die wichtigsten Arbeiten der "`Historiker"' offensichtlich. 
