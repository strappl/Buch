%%%%%%%%%%%%%%%%%%%%% chapter.tex %%%%%%%%%%%%%%%%%%%%%%%%%%%%%%%%%
%
% sample chapter
%
% Use this file as a template for your own input.
%
%%%%%%%%%%%%%%%%%%%%%%%% Springer-Verlag %%%%%%%%%%%%%%%%%%%%%%%%%%

\chapter{Drei Orte, die gleiche Idee}
\label{Marginalismus}

Die Geburtsstunde der Neoklassik wird um das Jahr 1870 angesiedelt, obwohl bedeutende "`Vorläufer"' schon deutlich früher sehr ähnliche Konzepte vorweggenommen haben (vgl. Kapitel \ref{Vorläufer}). Interessant ist der "`Geburtsort"': Tatsächlich wurden sehr ähnliche Grundkonzepte in drei verschiedenen Städten entwickelt. An allen drei Orten entwickelte sich in der Folge eine rege wirtschaftswissenschaftliche Forschungstätigkeit und daraus entstanden drei verschiedene Schulen, die bis heute existieren, wenn sie sich auch in ganz unterschiedliche Richtungen entwickelt haben. In Wien entwickelte Carl Menger sein nicht quantitatives Konzept des Grenznutzens und bildete damit die Grundlage der bis heute existierenden "`Österreichischen Schule der Nationalökonomie"' (vgl. Kapitel \ref{Austria}). In Cambridge begründete Stanley Jevons die Theorie der subjektiven Wertlehre und begründete damit die "`Cambridge School"', die mit ihren späten Vertretern Alfred Marshall, Irving Fisher und Arthur Pigou die "`Vollendung der Neoklassik"' darstellt (vgl. Kapitel \ref{Neoklassik}. Die "`Lausanner Schule"' wurde von Leon Walras begründet. Dieser ist heute noch im Begriff des "`Walras-Gleichgewicht"' allgegenwärtig. Tatsächlich begründete er die Theorie des allgemeinen Gleichgewichts, die sich mit der Frage beschäftigt, ob alle einzelnen Märkte gemeinsam im Gleichgewicht sein können. Eine Frage, die im 20. Jahrhundert stark beforscht wurde (vgl. Kapitel \ref{Arrow-Debreu}) und bis heute umstritten ist.

Der Übergang von "`Klassik"' zu "`Neoklassik"' wird bis heute an verschiedenen Elementen festgemacht \parencite[S. 198]{Ekelund2002}.
\begin{enumerate}
	\item Ein entscheidendes Element ist sicherlich die Etablierung höherer Mathematik in der Ökonomie. Insbesondere die Implementierung der Differenzialrechnung ist für "`Grenzbetrachtungen"' und daraus entstehenden Maximierungsaufgaben notwendig.
	\item Der Marginalismus selbst, die Betrachtung von Grenzwerten, also den Werten des "`letzten"' Gutes, ist eine wesentliche Erweiterung gegenüber der Klassik, die diesbezüglich statisch nur von einheitlichen Gütern sprach. Damit verbunden ist die Optimierungsbetrachtung der "`Statischen Effizienz"': Der maximale Gewinn wird dann erzielt wenn das nächste produzierte Gut gleich hohe Kosten verursacht als es Ertrag einbringt. 
	\item Der Subjektivismus, also die Zuweisung eines individuellen Nutzens, kann ebenfalls als \textit{das} wesentliche Abgrenzungskriterium zur "`Klassik"' gesehen werden. In der Klassik sind die Güter objektiv bewertbar, meist aus dem Herstellungskosten heraus. Die beiden Bewertungszugänge unterscheiden sich fundamental.
\end{enumerate}

Natürlich lassen sich diese drei Element nicht streng voneinander trennen - im Gegenteil in gewissem Ausmaß bedingen sie einander eher. Gemeinsam bilden sie allerdings eine recht saubere Abgrenzung zwischen klassischer Ökonomie und der Neoklassik.

\section{Die Vorläufer der Neoklassik}
\label{Vorläufer}

Die Entstehung der Neoklassik an den drei eben genannten Orten um das Jahr 1870 wird häufig auch "`Marginalistische Revolution"' genannt. Mittlerweile weiß man längst, dass es sich nicht um eine Revolution im Sinne einer sprunghaften Veränderung handelte, sondern dass einzelne Elemente der Neoklassik bereits seit Anfang des 19. Jahrhunderts entwickelt und vielfach auch wieder vergessen wurden. \textcite{Ekelund2002} führen dazu für Großbritannien, Frankreich, Deutschland, Italien und auch die USA zahlreiche Beispiele an. Die meisten dieser Beiträge brachten nur eingeschränkt und isoliert Elemente der späteren Neoklassik hervor. Vier Autoren stechen allerdings heraus. Ihre Beiträge, die um 1850 entstanden, nehmen die wesentlichen Bausteine der Neoklassik in umfangreichem Ausmaß vorweg \parencite[S. 205]{Ekelund2002}. Eine besonders bemerkenswerte Geschichte ist jene des deutschen Ökonomen Hermann Heinrich Gossen.

\subsection{Unbelohnter, unbekannter Hochmütiger}

Hermann Heinrich Gossen ist eine der tragischsten Figuren in der Geschichte der Ökonomie. Heute wissen wir, dass seine Erkenntnisse geradezu bahnbrechend waren. Zu seinen Lebzeiten blieb seine Arbeit hingegen vollkommen unbekannt und er erhielt keinerlei Wertschätzung. Laut \textcite{Kurz2009} gibt es von Gossen weder Foto noch sonstiges Bildnis\footnote{Man findet unter Google einige Bilder, die Gossen darstellen sollen. Inwieweit dies falsche Informationen sind, oder doch unbekannte Quellen mit seinem wahren Bildnis vorhanden sind, kann nicht verifiziert werden.}. Interessant ist auch, dass es - wenn auch posthum - dem Ökonomen \textcite{Walras1885} zu weitgehend verdanken ist, dass Gossen doch noch zu Ruhm gekommen ist. Dies ist insofern bemerkenswert, als Walras selbst ja als einer der Väter der Neoklassik gilt. Nachdem er von der Existenz von Gossens' Werk gehört hat, bezeichnete er dieses als "`allgemeiner und ausführlicher"' als sein eigenes \parencite[S. 1]{Kurz2009}.
Sein einziges Werk \textcite{Gossen1854}: "`Entwicklung der Gesetze des menschlichen Verkehrs [...]"' wurde zu seinen Lebzeiten praktisch nicht gelesen \parencite[S. 282]{Rosner2012}. Er hatte keine akademisch-ökonomische Ausbildung, war stattdessen diesbezüglich Autodidakt \parencite[S. 3]{Kurz2009} und dafür studierter Jurist. Für ihn gilt - ebenso wie in Kürze für Johann Heinrich von Thünen und die anderen Vorläufer dargestellt - dass er einer der ersten war, der die Differentialrechnung in seinen Modellen anwendete. Die höhere Mathematik war in ökonomischen Arbeiten zur damaligen Zeit noch nicht angekommen. Dies und die Tatsache, dass seine Arbeit \parencite{Gossen1854} recht unstrukturiert aufgebaut war, sowie der Umstand, dass Gossen bereits vier Jahr nach Erscheinen des Buchs mit 47 Jahre verstarb, führte dazu, dass er zu Lebzeiten - wie oben beschrieben - vollkommen unbekannt blieb. Nach eigenen Angaben im Buch, arbeitete Gossen 20 Jahre lang \parencite[S. 3]{Kurz2009}, also fast sein ganzes Erwachsenenleben an seinem Werk. Als er es 1953 fertigstellt, ist kein Verlag an dem Manuskript interessiert. Gossen lässt es auf eigene Kosten drucken, aber fast niemand kauft das Werk. Die akademische Welt nimmt praktisch keine Notiz davon. Gossen erkrankt kurz darauf schwer und stirbt schon 1858 \parencite[S.4]{Kurz2009}. Er war zu diesem verbittert über seinen Misserfolg, aber dennoch stets überzeugt, dass er eine bahnbrechende Arbeit erstellt habe, wie er in seinem Buch immer wieder hochmütig beschreibt \parencite{Kurz2009, Gossen1854}. Welch grausames Schicksal, dass erst Jahrzehnte später seine Fortschrittlichkeit erkannt wurde.

HIER WEITER:
Pionier des Nutzenkonzepts \textcite[S. 283]{Rosner2012}
Entwickler der Konsumtheorie \textcite[S. 202]{Ekelund2002}
\parencite{Kurz2009}
\parencite{Ikeda2000}


In der akademischen Lehre wird die Arbeit von Gossen meist auf die zwei Gossenschen Gesetze reduziert: Erstens, das Gesetz des abnehmenden Grenznutzens: Jedes zusätzliche (identische) Gut liefert einen geringeren Nutzen als das Gut davor. Für ökonomische Modelle hat dies wenig Bedeutung, da es ja solange zum Austausch verschiedener Güter kommt, bis der Gesamtnutzen maximiert ist. Aber die Idee des abnehmenden Grenznutzen an und für sich ist bis heute von Bedeutung. Bei Entscheidungen unter Risiko wird eine konkave Nutzenfunktion herangezogen. Diese Risikofunktion wird "`Von Neumann-Morgenstern-Nutzenfunktion"' genannt  Der abnehmende Grenznutzen nach Gossen ist hierbei identisch mit dem später entwickelten Konzept der Risikoaversion (vgl. \ref{Erwartungsnutzen}).
Zweitens: Jedes Individuum wird seine Güter solange gegen andere Güter tauschen, bis alle Güter den gleichen Nutzen liefern. Später wurde dieses Prinzip in ähnlicher Form in den "`Wohlfahrtstheoremen"' formalisiert (vgl. Kapitel \ref{Neoklassik_nach1945}). Gossen liefert dazu auch den Versuch eines mathematischen Beweis: Solange ein Individuum durch einen einzigen Gütertausch seinen Nutzen erhöhen kann, solange kann der individuelle Nutzen nicht maximal sein.





Dem zweiten großen deutschen Vorläufer der Neoklassik wurde zu Lebzeiten schon Ehre zuteil, wenn auch nicht für seine eigentlichen Leistungen. Die Rede ist von Johann Heinrich von Thünen. In seinem Werk "`Der isolirte Staat [...]"' \textcite{Thunen1826} erstellte er ein ökonomisches Modell am Beispiel der landwirtschaftlichen Produktion. In Abhängigkeit vom geographischen Abstand zur zentral gelegenen Stadt, sollten verschiedene Güter in verschiedenen Zonen produziert werden. Gemüse und Milch zum Beispiel näher als Getreide. Das Konzept wird heute noch als die "`Thünschen Kreise"' gelehrt. Er gilt vielen aufgrund dieser Aspekte seiner Arbeit als Begründer der Wirtschaftsgeographie \parencite[S. 283]{Kurz2009}. Seine größere Leistung liegt aber in der von ihm verwendeten Methodik. Er war wohl der erste, der so formal präzise die verschiedenen Produktionsfaktoren zusammenführte und als Grenzprodukte analysierte. Er nahm damit die moderne Produktionstheorie vorweg: Er ging, erstens, von sinkenden Grenzerträge für jeden Produktionsfaktor aus und, zweitens, analysierte er als erster Grenzprodukte. Also was passiert mit dem Gesamtoutput wenn ein Produktionsfaktor verändert wird und alle anderen gleich bleiben \parencite[S. 282]{Rosner2012}, nichts anderes wird heute noch als komparativ statische Analyse in der Mikroökonomie gelehrt. Bezeichnend für seinen Fortschritt ist auch, dass er zur Berechnung seiner Modelle bereits die Infinitesimalrechnung, konkret die Differentialrechnung, heranzog. Diese wurde bis dahin praktisch ausschließlich in den Naturwissenschaften angewendet \parencite[S. 202]{Ekelund2002}. Seine konkreten Modelle, die mehrere verschiedene Produkte umfassten, konnte er damit aber dennoch nicht lösen, die dazu nötigen Methoden Input-Output-Analyse bzw., der dynamische Optimierung wurden erst im 20. Jahrhundert entwickelt. Seine Arbeit war damit seiner Zeit weit voraus, keiner der zeitgenössischen Ökonomen baute zu seinen Lebzeiten - er starb im Jahr 1850 - auf seinen Arbeit auf. Dennoch wurde sein Werk zumindest gewürdigt, die Arbeit wurde laut \textcite[S. 283]{Rosner2012} gelesen und gelobt und ihm wurde auch der Ruhm einer Ehrendoktorwürde und einer Ehrenbürgerschaft zuteil.

\subsection{Cournot, Dupuit}

 Cournot, Dupuit

\section{Jevons}
\label{Jevons}

\section{Walras}
\label{Walras}

\section{Menger}
\label{Wiener Schule}

