%%%%%%%%%%%%%%%%%%%%% chapter.tex %%%%%%%%%%%%%%%%%%%%%%%%%%%%%%%%%
%
% sample chapter
%
% Use this file as a template for your own input.
%
%%%%%%%%%%%%%%%%%%%%%%%% Springer-Verlag %%%%%%%%%%%%%%%%%%%%%%%%%%

\chapter{Drei Orte, die gleiche Idee! - oder doch nicht so einfach?}
\label{Marginalismus}

Die Geburtsstunde der Neoklassik wird um das Jahr 1870 angesiedelt, obwohl bedeutende "`Vorläufer"' schon deutlich früher sehr ähnliche Konzepte vorweggenommen haben (vgl. Kapitel \ref{Vorläufer}). Interessant ist der "`Geburtsort"': Tatsächlich wurden sehr ähnliche Grundkonzepte in drei verschiedenen Städten entwickelt. An allen drei Orten entwickelte sich in der Folge eine rege wirtschaftswissenschaftliche Forschungstätigkeit und daraus entstanden drei verschiedene Schulen, die bis heute existieren, wenn sie sich auch in ganz unterschiedliche Richtungen entwickelt haben. In Wien entwickelte Carl Menger sein nicht-quantitatives Konzept des Grenznutzens und bildete damit die Grundlage der bis heute existierenden "`Österreichischen Schule der Nationalökonomie"' (vgl. Kapitel \ref{Austria}). In London begründete Stanley Jevons die Theorie der subjektiven Wertlehre, die als Ausgangspunkt der "`Cambridge School of Economics"' gelten kann. Wenn diese auch erst später von Alfred Marshall begründet wurde. Die "`Lausanner Schule"' wurde von Leon Walras begründet. Dieser ist heute noch im Begriff des "`Walras-Gleichgewicht"' allgegenwärtig. Tatsächlich begründete er die Theorie des allgemeinen Gleichgewichts, die sich mit der Frage beschäftigt, ob alle einzelnen Märkte gemeinsam im Gleichgewicht sein können. Eine Frage, die im 20. Jahrhundert stark beforscht wurde (vgl. Kapitel \ref{Arrow-Debreu}) und bis heute umstritten ist.

Der Übergang von "`Klassik"' zu "`Neoklassik"' wird bis heute an verschiedenen Elementen festgemacht \parencite[S. 198]{Ekelund2002}.
\begin{enumerate}
	\item Ein entscheidendes Element ist sicherlich die Etablierung höherer Mathematik in der Ökonomie. Insbesondere die Implementierung der Differenzialrechnung ist für "`Grenzbetrachtungen"' und daraus entstehenden Maximierungsaufgaben notwendig.
	\item Der Marginalismus selbst, die Betrachtung von Grenzwerten, also den Werten des "`letzten"' Gutes, ist eine wesentliche Erweiterung gegenüber der Klassik, die diesbezüglich statisch nur von einheitlichen Gütern sprach. Damit verbunden ist die Optimierungsbetrachtung der "`Statischen Effizienz"': Der maximale Gewinn wird dann erzielt wenn das nächste produzierte Gut gleich hohe Kosten verursacht als es Ertrag einbringt. 
	\item Der Subjektivismus, also die Zuweisung eines individuellen Nutzens, kann ebenfalls als \textit{das} wesentliche Abgrenzungskriterium zur "`Klassik"' gesehen werden. In der Klassik sind die Güter objektiv bewertbar, meist aus dem Herstellungskosten heraus. Die beiden Bewertungszugänge unterscheiden sich fundamental.
\end{enumerate}

Natürlich lassen sich diese drei Element nicht streng voneinander trennen - im Gegenteil in gewissem Ausmaß bedingen sie einander eher. Gemeinsam bilden sie allerdings eine recht saubere Abgrenzung zwischen klassischer Ökonomie und der Neoklassik.


\section{Die Vorläufer der Neoklassik}
\label{Vorläufer}

Wie bereits erwähnt wurde die Neoklassik an drei verschiedenen Orten unabhängig voneinander und fast gleichzeitig entwickelt. Man nennt deren Entstehung daher häufig "`Marginalistische \textit{Revolution}"'. In vielen Wissenschaften gibt es das Phänomen, dass neue Erkenntnisse von verschiedenen Forschern gleichzeitig entwickelt wurden - berühmt ist in diesem Zusammenhang vor allem die Entwicklung der Infinitesimalrechnung, die von Leibnitz und Newton unabhängig entwickelt wurde. Häufig finden Historiker einen gemeinsamen "`Auslöser"' für solch parallele Fortschritte. Für die "`Marginalistische Revolution"' hingegen, finden sich keine solchen Auslöser. Dafür waren die gesellschaftlichen und wissenschaftlichen Rahmenbedingungen in Wien, Lausanne und London - die drei Orte wo die Neoklassik entwickelt wurde - zu unterschiedlich \parencite[S. 269]{Blaug1973}. Vielmehr kann man davon ausgehen, dass die oben beschriebenen, zentralen Ideen der "`Neoklassik"' immer wieder punktuell "`erfunden"', aber auch wieder vergessen wurden. \textcite[S. 274]{Blaug1973} zählt gleich neun Ökonomen auf, die ähnliche Ideen beschrieben \footnote{Nur die bekanntesten davon werden in weiterer Folge als "`Vorläufer"' beschrieben.}. Außerdem weiß man heute, dass die "`Marginalistische Revolution"' als solche von den Zeitgenossen nicht wahrgenommen wurde, sondern erst später - gegen Ende des 19. Jahrhunderts - als solche bezeichnet wurde.
Mittlerweile weiß man also, dass es sich nicht um eine Revolution im Sinne einer sprunghaften Veränderung handelte, sondern dass einzelne Elemente der Neoklassik bereits seit Anfang des 19. Jahrhunderts entwickelt wurden. \textcite{Ekelund2002} führen dazu für Großbritannien, Frankreich, Deutschland, Italien und auch die USA zahlreiche Beispiele im Detail an. Die meisten dieser Beiträge brachten nur eingeschränkt und isoliert Elemente der späteren Neoklassik hervor. Drei Autoren stechen allerdings als "`Vorreiter der Neoklassik"' heraus. Ihre Beiträge, die um 1850 entstanden, nehmen die wesentlichen Bausteine der Neoklassik in umfangreichem Ausmaß vorweg \parencite[S. 205]{Ekelund2002}. Eine besonders bemerkenswerte Geschichte ist jene des deutschen Ökonomen Hermann Heinrich Gossen.

\subsection{Unbelohnter, unbekannter Hochmütiger}

Hermann Heinrich Gossen ist eine der tragischsten Figuren in der Geschichte der Ökonomie. Heute wissen wir, dass seine Erkenntnisse geradezu bahnbrechend waren. Zu seinen Lebzeiten blieb seine Arbeit hingegen vollkommen unbekannt und er erhielt keinerlei Wertschätzung. Laut \textcite{Kurz2009} gibt es von Gossen weder Foto noch sonstiges Bildnis\footnote{Man findet unter Google einige Bilder, die Gossen darstellen sollen. Inwieweit dies falsche Informationen sind, oder doch unbekannte Quellen mit seinem wahren Bildnis vorhanden sind, kann nicht verifiziert werden.}. Interessant ist auch, dass es - wenn auch posthum - dem Ökonomen \textcite{Walras1885} zu weitgehend verdanken ist, dass Gossen doch noch zu Ruhm gekommen ist. Dies ist insofern bemerkenswert, als Walras selbst ja als einer der Väter der Neoklassik gilt. Nachdem er von der Existenz von Gossens' Werk gehört hat, bezeichnete er dieses als "`allgemeiner und ausführlicher"' als sein eigenes \parencite[S. 1]{Kurz2009}. Die Geschichte, wie Gossen's Werk wiederentdeckt wurde, ist in \textcite{Ikeda2000} ausführlich beschrieben. Sie zeigt auch, wie hoch die späteren Neoklassiker die Leistung Gossen's einschätzten.
Zu seinen Lebzeiten wurde sein einziges Werk, \textcite{Gossen1854}: "`Entwicklung der Gesetze des menschlichen Verkehrs [...]"', praktisch nicht gelesen \parencite[S. 282]{Rosner2012}. Er hatte keine akademisch-ökonomische Ausbildung, war stattdessen diesbezüglich Autodidakt \parencite[S. 3]{Kurz2009} und dafür studierter Jurist. Für ihn gilt - ebenso wie in Kürze für Johann Heinrich von Thünen und die anderen Vorläufer dargestellt - dass er einer der ersten war, der die Differentialrechnung in seinen Modellen anwendete. Die höhere Mathematik war in ökonomischen Arbeiten zur damaligen Zeit noch nicht angekommen. Dies und die Tatsache, dass seine Arbeit \parencite{Gossen1854} recht unstrukturiert aufgebaut war \parencite[S. 20]{Kurz2009}, sowie der Umstand, dass Gossen bereits vier Jahr nach Erscheinen des Buchs mit 47 Jahren verstarb, führte dazu, dass er zu Lebzeiten - wie oben beschrieben - vollkommen unbekannt blieb. Nach eigenen Angaben im Buch, arbeitete Gossen 20 Jahre lang \parencite[S. 3]{Kurz2009}, also fast sein ganzes Erwachsenenleben an seinem Werk. Als er es 1953 fertigstellt, ist kein Verlag an dem Manuskript interessiert. Gossen lässt es auf eigene Kosten drucken, aber fast niemand kauft das Werk. Die akademische Welt nimmt praktisch keine Notiz davon. Gossen erkrankt kurz darauf schwer und stirbt schon 1858 \parencite[S.4]{Kurz2009}. Er war zu diesem Zeitpunkt schwer verbittert über seinen Misserfolg, aber dennoch stets überzeugt, dass er eine bahnbrechende Arbeit erstellt habe. So vergleicht er seine eigenen Leistungen mit den umwälzenden Arbeiten von Kopernikus \parencite{Kurz2009, Gossen1854}. Welch grausames Schicksal, dass erst Jahrzehnte später seine tatsächliche Fortschrittlichkeit erkannt wurde.

Heute gilt Gossen als der eigentliche Begründer des quantitativen Nutzenkonzepts in der Ökonomie. Er ging davon aus, dass Menschen nach dem "`maximalen Lebensgenuss"' streben \parencite[S. 284]{Rosner2012}. Diese Idee war auch damals nicht bahnbrechend, sondern wirkt selbstverständlich. Aber diese Idee in eine wirtschaftswissenschaftliche Analyse überzuführen und dies auch noch zu Quantifizieren ist ein entscheidender Schritt. Ob und inwieweit dies überhaupt möglich ist, ist bis heute umstritten, wie in einigen der folgenden Kapiteln diskutiert werden wird. Der Utilitarismus, also das in den Vordergrund stellen des "`Nutzens"', war ursprünglich bereits um 1790 als philosophische Schule in England von \textcite{Bentham1789} begründet worden. Es ist aber unwahrscheinlich, dass Gossen von diesen Lehren etwas gehört hat, geschweige denn von diesen geprägt wurde. 

In der akademischen Lehre stößt man auf die Arbeit von Gossen meist in Form der Gossenschen Gesetze: Erstens, das Gesetz des abnehmenden Grenznutzens: Jedes zusätzliche Gut liefert einen geringeren Nutzen als das (identische) Gut davor. Für ökonomische Modelle hat dies heute wenig Bedeutung, da es ja solange zum Austausch verschiedener Güter kommt, bis der Gesamtnutzen maximiert ist. Aber die Idee des abnehmenden Grenznutzen an sich ist bis heute von Bedeutung. Bei Entscheidungen unter Risiko wird eine konkave Nutzenfunktion herangezogen. Diese Risikofunktion wird heute "`Von Neumann-Morgenstern-Nutzenfunktion"' genannt. Der abnehmende Grenznutzen bei Gossen ist hierbei identisch mit dem später entwickelten Konzept der Risikoaversion (vgl. \ref{Erwartungsnutzen}).
Zweitens: Jedes Individuum wird seine Güter solange gegen andere Güter tauschen, bis alle Güter den gleichen Nutzen liefern. Später wurde dieses Prinzip in ähnlicher Form in den "`Wohlfahrtstheoremen"' formalisiert (vgl. Kapitel \ref{Neoklassik_nach1945}). Gossen liefert dazu auch den Versuch eines mathematischen Beweis: Solange ein Individuum durch einen einzigen Gütertausch seinen Nutzen erhöhen kann, solange kann der individuelle Nutzen nicht maximal sein.
Drittens: Einen ökonomischen Wert haben nur Güter, die nicht uneingeschränkt verfügbar sind. Dieses dritte Gossensche Gesetz, wird übrigens häufig nur im englischsprachigen Raum so genannt (vgl. z.B.: \parencite{Blaug1973}). Im Deutschen wird hingegen oft von zwei Gossenschen Gesetzen gesprochen.
Der Fokus auf die seine "`Gesetze"' schränkt die wahre Leistung Gossen's ein. Vor allem in Kombination mit seinen mathematischen Formulierungen kann das Werk als Vorläufer der modernen Konsumtheorie gesehen werden.

Dem zweiten großen deutschen Vorläufer der Neoklassik wurde zu Lebzeiten durchaus Ehre zuteil, wenn auch nicht für seine eigentlichen Leistungen. Die Rede ist von Johann Heinrich von Thünen. In seinem Werk "`Der isolirte Staat [...]"' \textcite{Thunen1826} erstellte er ein ökonomisches Modell am Beispiel der landwirtschaftlichen Produktion. In Abhängigkeit vom geografischen Abstand zur zentral gelegenen Stadt, sollten verschiedene Güter in verschiedenen Zonen produziert werden. Gemüse und Milch zum Beispiel näher als Getreide. Das Konzept wird heute noch als die "`Thünschen Kreise"' gelehrt. Er gilt vielen aufgrund dieser Aspekte seiner Arbeit als Begründer der Wirtschaftsgeographie \parencite[S. 283]{Kurz2009}. Seine größere Leistung liegt aber in der von ihm verwendeten Methodik. Er war wohl der erste, der so formal präzise die verschiedenen Produktionsfaktoren zusammenführte und als Grenzprodukte analysierte. Er nahm damit die moderne Produktionstheorie vorweg: Er ging, erstens, von sinkenden Grenzerträgen für jeden Produktionsfaktor aus. Zweitens, analysierte er als erster Grenzprodukte: Also was passiert mit dem Gesamtoutput wenn ein Produktionsfaktor verändert wird und alle anderen gleich bleiben \parencite[S. 282]{Rosner2012}. Nichts anderes wird heute noch als komparativ statische Analyse in der Mikroökonomie gelehrt. Bezeichnend für seinen Fortschritt ist auch, dass er zur Berechnung seiner Modelle bereits die Infinitesimalrechnung, konkret die Differentialrechnung, heranzog. Diese wurde bis dahin praktisch ausschließlich in den Naturwissenschaften angewendet \parencite[S. 202]{Ekelund2002}. Seine konkreten Modelle, die mehrere verschiedene Produkte umfassten, konnte er damit aber dennoch nicht lösen. Die dazu nötigen Methoden - Input-Output-Analyse bzw., der dynamische Optimierung - wurden erst im 20. Jahrhundert entwickelt. Seine Arbeit war damit seiner Zeit weit voraus, keiner der zeitgenössischen Ökonomen baute zu seinen Lebzeiten - er starb im Jahr 1850 - auf seinen Arbeiten auf. Dennoch wurde sein Werk zumindest gewürdigt, die Arbeit wurde laut \textcite[S. 283]{Rosner2012} gelesen und gelobt und ihm wurde auch der Ruhm einer Ehrendoktorwürde und einer Ehrenbürgerschaft zuteil.

\subsection{Der Vater der Profitmaximierung}

Auch in Frankreich gab es einen interessanten Vorläufer zur Neoklassik: Augustin Antoine Cournot. Er war aber Vorläufer auf eine ganz andere Art als die eben genannten deutschen Thünen und Gossen. Erstens, war er anerkannter Professor an der Universität Lyon und zweitens, beschäftigte er sich nicht mit dem Nutzenbegriff, sondern mit Ertrags- und Kostenfunktionen \parencite[S. 287f]{Rosner2012}. Das für die Neoklassik so zentrale Element des Nutzens fehlte also. Was machte Cournot dennoch zu einem Vorläufer der Neoklassik und nicht einfach zu einem Klassiker? In seiner Arbeit \textcite{Cournot1838} nahm er die Analyse der Profitmaximierung vorweg. Mit Hilfe der Differentialrechnung analysierte er zunächst die Nachfragefunktion und bestimmte eine Funktion zur Erlösmaximierung. Danach wendete er dieses wissen bereits für die verschiedenen Marktformen Monopol, Duopol und den "`uneingeschränkten Wettbewerb"', also die heutige vollkommenen Konkurrenz an und ermittelte bereits die Bedingungen der Profitmaximierung \parencite[S. 289]{Rosner2012}. Diese gelten bis heute praktisch unverändert.  Nicht nur in der Volkswirtschaft, sondern auch in der betriebswirtschaftlichen Kostenrechnung spricht man vom "`Cournotschen Punkt"' als jene Preis-Mengen-Kombination bei der der Monopolist seinen Gewinn maximiert. Bereits Cournot definierte diesen Punkt als "`Grenzkosten gleich Grenzerlös"'. Es ist interessant, dass Cournot der erste war, der vollkommene Konkurrenz in der heute gültigen Definition als Marktform definierte. Nämlich als Markt auf dem es so viele Anbieter gibt, dass Produzenten den Preis nicht beeinflussen können \footnote{"'Vollkommene Konkurrenzmärkte"' werden oft mit "`Freien Märkten"' in der Definition von Adam Smith verwechselt. Smith meinte allerdings, dass es keine künstlichen Zugangsbeschränkungen und andere Marktbarrieren geben sollte. Die Anzahl der Marktteilnehmer nannte Smith bei seiner Bestimmung von "`Monopolen"' oder "`Wettbewerbsmärkten"' nicht \parencite{Blaug2001}.}. Besonders bemerkenswert ist auch seine Profitmaximierungslösung im Duopol-Fall, also wenn es am Markt nur zwei Anbieter gibt. Tatsächlich kommt Cournot bereits auf die Lösung, dass es ein stabiles Gleichgewicht gibt, das allerdings nicht Pareto-optimal sein muss. Das heißt, im Duopol-Fall können beide Anbieter jeweils kurzfristig einen höheren Gewinn erzielen, wenn sie ihre Produktion ausweiten. Da diese Möglichkeit wechselseitig besteht, würde durch diese Produktionsausweitung der Preis und in weiterer Folge der Gewinn für beide Duopol-Anbieter sinken. Cournot nahm mit seiner Lösung die bahnbrechenden Arbeiten von \textcite{Nash1950} zur Spieltheorie in diesem Punkt vorweg \parencite{Leonard1994}. Tatsächlich entspricht die Lösung, die Cournot gefunden hat, bereits einem Nash-Gleichgewicht (vgl. Kapitel \ref{Spieltheorie})!

\section{Menger}
\label{Wiener Schule}

Karl Menger, wird häufig als "`Vater des Subjektivismus' bezeichnet. Dieser Subjektivismus bezieht sich bei ihm auf die Güterpreise. Er revolutionierte also die Wert- und Preistheorie. Zu seiner Zeit war die Preistheorie der Klassiker vorherrschend. Diese erklärten Güterpreise aus den Produktionskosten\footnote{Ein Ansatz der übrigens bis heute in der Kostenrechnung teilweise suggeriert wird, indem man in vielen Lehrbüchern zunächst häufig die Herstellungskosten berechnet und danach einen Gewinnaufschlag hinzurechnet.}, was aber häufig alltäglichen, empirischen Beobachtungen widerspricht. 

Es wurde - konkret für die Arbeit von Menger - von \textcite{Streissler1990} gezeigt und auch von anderen Autoren beschrieben (\textcite{Blaug1973}, \textcite{Ekelund2002}), dass die allermeisten Ideen, die \textcite{Menger1871} umfasst, in Kontinentaleuropa schon deutlich vor 1870 bekannt waren. Vor allem in Deutschland gab es dahingehend einige, bis heute weitgehend unbekannt gebliebene Autoren, die Menger's Ideen ab dem frühen 19. Jahrhundert vorwegnahmen. \textcite[S. 159]{Blaug2001} führt diesbezüglich neben Thünen vor allem Karl Heinrich Rau an, der  Produktions-seitig schon Nutzen und Grenzbetrachtungen behandelte. Menger's große Leistung bestand laut \textcite[S. 295]{Rosner2012} aber vor allem darin, alle Bausteine zu einer einheitlichen Theorie zusammenzuführen. Denn nicht nur die Preistheorie der Klassiker hatte offensichtliche, empirische Probleme. Auch die eben erwähnten Ansätze der deutschen Autoren um Rau, konnten einige Fragen bezüglich Preise und Werte von Gütern nicht lösen. Die Preistheorie der Klassiker scheitert vor allem schon an den ständig schwankenden Preisen. Die Produktionskosten bleiben üblicherweise - zumindest bei Gütern, die keine Vorleistungen benötigen - kurzfristig konstant. Es wäre also eine Korrelation zwischen Güterpreisen und Produktionskosten notwendig um die klassische Theorie aufrechtzuerhalten, diese ist aber offensichtlich nicht durchgehend zu finden. Die Klassiker hatten Probleme bei der Unterscheidung zwischen Preis und Wert eines Gutes. Der Preis eines Gutes ergab sich demnach aus einem \textit{einheitlichen, objektiven} Wert dieses Gutes. Offensichtlich unterschiedliche Bewertung verschiedener Güter durch verschiedene Menschen lässt sich damit nicht in Einklang bringen. Damit verbunden ist das fehlende Verständnis von Nutzen, den ein Gut stiftet: Nahrungsmittel stillen Hunger, Schmuck stillt das Bedürfnis nach Geltung. Dennoch sind Nahrungsmittel günstiger zu erwerben als Schmuck - ein fundamentales Problem - genannt das "`Wertparadoxon"' - welches die Preistheorie der Klassiker nicht lösen kann, aber in der Lebensrealität eine Rolle spielt.

\textcite{Menger1871} - wenig überraschend, schließlich nennt man ihn nicht umsonst den "`Vater des Subjektivismus"' - stellt die subjektiven Werteinschätzungen der Menschen in den Vordergrund seiner Arbeit. Er beginnt damit festzustellen, dass - sinngemäß -  Güter als Dinge definiert werden, die einen Nutzen\footnote{Wörtlich eine \textit{Nützlichkeit}} stiften um menschliche Bedürfnisse befriedigen können \parencite[S. 1f]{Menger1871}. In Kapitel drei schließlich, beschreibt er, dass Güter einen Wert haben, der davon abhängt, wie stark das Bedürfnis nach einem Gut ist und die Verfügbarkeit des Gutes \parencite[S. 78]{Menger1871}. Güter werden also, individuell subjektiv (nach gestiftetem Nutzen) bewertet, womit Menger die Frage, wodurch Güter einen \textit{Wert} haben, behandelt, noch nicht aber die Frage nach dem \textit{Preis} dieser Güter. Außerdem beschreibt er, dass das Ausmaß der Bedürfnisbefriedigung von der verfügbaren Menge des Gutes abhängt. Der Nutzen eines Gutes ist also nicht konstant, mit mit zunehmender Verfügbarkeit des Gutes, nimmt die Nutzenstiftung einer zusätzlichen Einheit ab. Heute spricht man diesbezüglich vom abnehmenden Grenznutzen eines Gutes. Schon alleine damit lässt sich das oben beschriebene "`Wertparadoxon"' lösen. Außerdem wird fährt Menger fort, dass bei ausreichender Verfügbarkeit eines Gutes, das Bedürfnis danach gestillt werden kann - heute sprechen wir von "`Sättigung"'. 
Wesentlich für seine Preistheorie ist seine "`Lehre vom Tausche"'. Aufbauend auf der eben dargestellten Werttheorie ist diese elegant herzuleiten. Menschen wollen Bedürfnisse befriedigen und haben eine gewisse Menge an Gütern (heute würde man Budgetrestriktion sagen), es liegt also nahe, die eigenen Güter nach deren individuell-subjektiver Nutzenstiftung (=Wert) zu bewerten und anschließend Güter anderer Personen ebenso zu bewerten. Kommt zwischen zwei Individuen ein Tausch zustande, hatten beide Vertragspartner das Gut des jeweils anderen offensichtlich höher bewertet, als das eigene Gut \parencite[S. 156]{Menger1871}. Das klingt heute banal, war aber damals in der ökonomischen Theorie bahnbrechend. Damit wird nämlich unterstellt, dass der Wert eines Gutes alleine durch den Tauschvorgang steigt - für die Klassiker (inklusive Marx) war dies undenkbar (vgl. Kapitel \ref{Klassik}). Möglich ist dies nur durch die Anerkennung des "`Subjektivismus"': Der Wert eines Gutes ist individuell von Mensch zu Mensch unterschiedlich.
\textcite[S. 172]{Menger1871} selbst schreibt sinngemäß, dass mit dem Wissen über den Tausch, die Theorie zur Bildung von Marktpreisen eigentlich bereits mit-umfasst ist: "`Die Preise [...] sind doch nichts weniger als das Wesentliche der ökonomischen Erscheinung des Tausches"'. Marktpreise ergeben sich also aus dem Tauschverhältnis verschiedener Güter zueinander. Er schneidet in weiterer Folge auch noch die Auswirkung der verschiedenen Marktformen auf den Marktpreis an, bleibt aber hier - wahrscheinlich auch, weil er auf mathematische Beispiele fast gänzlich verzichtet - hinter den Analysen von \textcite{Cournot1838} inhaltlich zurück \parencite[S. 304]{Rosner2012}.

Liest man \textcite{Menger1871} findet man ohne Probleme an vielen Stellen die engen Verbindungen zur heutigen Mainstream-Mikroökonomie. Aber Menger beschrieb dies eben alles ohne Rückgriff auf mathematische Formeln und durchaus auch mit bildlichen Vergleichen. So vergleicht er Preisschwankungen mit Wellen, die entstehen, "`wenn man die Schleussen zwischen zwei ruhig stehenden Gewässern [...] wegräumt \parencite[S. 172]{Menger1871}. Dadurch sind die Aussagen nicht immer ganz leicht nachzuvollziehen. Vielleicht ist dies ein Grund dafür, warum die unmittelbare Wirkung des Werkes eingeschränkt blieb und seine Fortschrittlichkeit erst im Nachhinein festgestellt wurde \parencite[S. 304]{Rosner2012}. 

Insgesamt wird die Leistung von Carl Menger - vor allem hinsichtlich ihrer Neuartigkeit - heute sehr unterschiedlich eingeschätzt. Auf der einen Seite gilt \textcite{Menger1871} eben als eines von drei Werken, dass  zur "`Marginalistischen Revolution"' geführt hat. Auf der anderen Seite weiß man heute, dass praktisch alle dort festgehaltenen Ideen schon zuvor Bestand hatten und Menger vor allem die - selbstverständlich keinesfalls zu unterschätzende - Leistung erbracht hat, diese Ideen geeint darzustellen. Auch gilt er als Begründer der "`Österreichischen Schule"' der Nationalökonomie (vgl. Kapitel \ref{Austria}), deren wesentliche Vertreter allerdings seine Schüler, bzw. Schüler seiner Schüler waren. Interessant ist auch seine Rolle im berühmt gewordenen Methodenstreit mit den Vertretern der "`Historischen Schule"' (vgl. Kapitel \ref{Historisch}). 

\section{Jevons}
\label{Jevons}

Im selben Jahr wie Menger, nämlich 1871, veröffentlichte Auch Stanley Jevons sein Hauptwerk: \textit{Theory of Political Economy}. Der Vergleich zwischen Menger und Jevons ist interessant, da es viele Gemeinsamkeiten, aber gleichzeitig deutliche Unterschiede in ihren Werken und deren Entstehungsgeschichte gibt. Im deutschsprachigen Raum fehlte um 1870 eine weitgehend akzeptierte Werttheorie. Während Menger die vereinzelten Arbeiten der Vorläufer im deutschsprachigen Raum (vgl. Kapitel \ref{Vorläufer}) zusammentrug, musste Jevons in England die gängige Werttheorie erst überwinden. Schließlich gab es in dort eine ausgeprägte Tradition klassischer Ökonomen. Die Werttheorie von Ricardo war anerkannter State of the Art \parencite[S. 320]{Rosner2012}. \textcite{Jevons1871} baute auf die, oben bereits erwähnte, philosophische Schule des "`Utilitarismus"' auf und verwendete auch dessen Sprache. Er versuchte wörtlich "`Ökonomie als Berechnung von Vergnügen und Schmerz (pleasure and pain)"' zu sehen \parencite[S. 321]{Rosner2012}. Darin versteckt sich auch schon der zweite wesentliche Unterschied zu Menger: Jevons war ein großer Verfechter der Anwendung höherer Mathematik in der Ökonomie. Er war wohl der erste, der die Differentialrechnung nachhaltig in der Ökonomie verankerte. Sein Argument war, dass wirtschaftliche Aktivität ja in Zahleneinheiten (z.B.: Mengen und Preise) ausgedrückt wird und daher Mathematik die logischste Darstellungsform sei \parencite[S. 71]{Jevons1871}. Ganz ähnlich wie Menger verwendet er das Konzept des Nutzens. Bei \textcite[S. 106]{Jevons1871} ist der Nutzen der subjektive Vorteil, den ein Gut bei einem Menschen auslöst. Er leitet daraus den abnehmenden Grenznutzen ab und stellt den Nutzen in der heute noch üblichen Form dar: $u(x)$. Also der Nutzen$u$ als eine Funktion des Gutes$x$ \parencite{Rosner2012}. Das Konzept des Grenznutzens - als wesentliches Konzept seiner Arbeit und der ganzen "`Marginalistischen Revolution"' - nannte er "`final degree of utility"' und ist bei ihm eben die erste Ableitung des Gesamtnutzens. Er wendet seine Erkenntnis zu Grenznutzen an, um Marktpreise bestimmen zu können und kommt - ebenso wie Menger, aber mit mathematischer Präzision - zur Bestimmung des Haushaltsoptimums. Das Austauschverhältnis zweier Güter ist reziprok zum Verhältnis der Grenznutzen zweier Güter. Dies entspricht dem oben bereits erwähnten "`Zweiten Gossenschen Gesetz"'. Das Konzept der Grenzbetrachtung wendete er in weiterer Folge auch auf die Theorie der Rente, des Kapitals und der Arbeit an. Das Verrichten von Arbeit hat hierbei einen negativen Nutzen, ähnlich wie dies auch noch in den modernen Gleichgewichtsmodellen berücksichtigt wird.

"`Ein bekannter Mann"' \parencite[S. 227]{Jevons1934} wurde der Ökonom übrigens bereits mit der Veröffentlichung seines Artikels "`The Coal Question"' \parencite{Jevons1865}. Darin beschrieb er ein Phänomen - mittlerweile genannt das Jevons-Paradoxon -, das im Hinblick auf den Klimawandel heute aktueller denn je ist: Höhere Effizienz in der z.B. Kohle-Förderung führt zu höherem Wohlstand bei gleichem Kohleverbrauch und nicht zu geringerem Kohleverbrauch bei konstant-bleibenden Wohlstand.

Jevons war ein extrem vielseitiger Wissenschaftler. Er war begeistert von den Naturwissenschaften \parencite[S. 225]{Jevons1934} und publizierte auch Beiträge in anderen Disziplinen, unter anderem zu Logik. Ein umstrittener Beitrag war jener, in dem er einen kausalen Zusammenhang zwischen Sonnenflecken und Wirtschaftszyklen unterstellte. Eine hohe Anzahl an Sonnenflecken sollte demnach zu Abkühlung, schlechteren Ernteerträgen und in weiterer Folge Rezessionen führen. Die Theorie erwies sich rasch als falsch. Sein Sohn argumentierte in \textcite[S. 229, S. 232]{Jevons1934}, dass Jevons selbst nicht wirklich an einen derartigen Zusammenhang glaubte \footnote{In \textcite[S. 232]{Jevons1934} schreibt sein Sohn, dass Jevons als einer der ersten Ökonomen Konjunkturzyklen analysierte. Dabei nahm er an, dass ungefähr alle zehn Jahre eine Wirtschaftskrise stattfindet. Demnach fand er eine Korrelation zwischen Sonnenflecken und Rezessionen, er glaubte aber demzufolge nie an einen kausalen Zusammenhang, der ihm häufig unterstellt wird}. Wie auch immer, der Begriff "`Sunspot Equilibrium"' schaffte es in das Vokabular der modernen Ökonomie. \textcite{Cass1983} verwendeten den Begriff erstmals und beschrieben damit - in Anlehnung an Jevons - die Möglichkeit eines Marktgleichgewichtes, das Zustande kommt, weil die Marktteilnehmer an die Bedeutung einer, in Wirklichkeit völlig bedeutungslosen, Variable glauben. Der allgemeine Glaube daran wird schließlich zur selbst-erfüllenden Prophezeiung. Im Neu-Keynesianismus wird diese Theorie vereinzelt zur Erklärung von Marktunvollkommenheiten herangezogen (vgl. \textcite{Woodford1990b}).

Wie die Geschichte manch anderer berühmter Wissenschaftler, hat auch jene von Stanley Jevons eine tragische Seite. Die eben erwähnte Vielseitigkeit könnte man heute auch so auslegen, dass er wohl ein "`Workaholic"' war, was ihn allerdings - laut seinem Sohn \parencite[S. 230]{Jevons1934} - körperlich überlastete. 1866, mit 29 Jahren erhilet er ein Professur in Manchester, aber schon mit 36 Jahren musste er krankheitsbedingt eine berufliche Auszeit nehmen und wenige Jahre später, 1875, die Professur aufgeben. Zwar nahm er wenig später am University College in London erneut eine Professur an, aber auch diese musste er 1880 krankheitsbedingt aufgeben \parencite[S. 230]{Jevons1934}. Schon 1882 schließlich, starb er bei einem Badeunfall in Südengland mit nur 46 Jahren.



HIER WEITER \textcite{Keynes1936a}











\section{Walras}
\label{Walras}


