%%%%%%%%%%%%%%%%%%%%% chapter.tex %%%%%%%%%%%%%%%%%%%%%%%%%%%%%%%%%
%
% sample chapter
%
% Use this file as a template for your own input.
%
%%%%%%%%%%%%%%%%%%%%%%%% Springer-Verlag %%%%%%%%%%%%%%%%%%%%%%%%%%

\chapter{Spieltheorie} \label{cha: Spieltheorie}
\label{Spieltheorie}

Sie wird häufig als Spezialfall und Weiterentwicklung der Entscheidungstheorie, oder als "`Interaktive Entscheidungstheorie"' bezeichnet und in der Ökonomie noch immer eher als Randthema behandelt, dabei ist sie wohl eine \textit{der} wesentlichen Weiterentwicklungen der Wirtschaftswissenschaften im 20. Jahrhundert: Die Spieltheorie. Ihre Bedeutung kann kaum überschätzt werden. Sowohl in der Mikroökonomie, als auch in der Makroökonomie ist die Spieltheorie Teil unzähliger Modelle. Auch diesbezüglich wandelte sich die Ökonomie: Ausgehend von den Neoklassikern, aber eben auch die Keynesianer und die Monetaristen, suchten nach "`Pareto-optimalen"' Lösungen. Die Neuen Klassiker und Neu-Keynesianer hingegen sprechen stattdessen meist von "`Nash-Gleichgewichten"' - also einem spieltheoretischen Gleichgewichtszustand. Wie bereits beschrieben, fristet die Spieltheorie in gewisser Art und Weise eine Außenseiterrolle innerhalb der Ökonomie. Über die Gründe kann man nur spekulieren. Wahrscheinlich spielt es aber eine Rolle, dass die Spieltheorie keine volkswirtschaftliche Theorie im eigentlichen Sinne ist. Ganz im Gegenteil, ihre Aussagen sind auch in der Politik, Biologie,  Betriebswirtschaft und im Sport von Interesse. Dazu passt auch, dass die Spieltheorie nicht von Ökonomen, sondern von Mathematikern entwickelt wurde. Tatsächlich findet man Vorlesungen zur Spieltheorie aber vor allem in wirtschaftswissenschaftlichen Curricula wieder. Zuletzt ist es wohl kaum zu bestreiten, dass die Spieltheorie - stärker als jede andere Disziplin - von bemerkenswerten und oft kontroversen Persönlichkeiten geprägt wurde, von denen uns in diesem Kapitel mehrere unterkommen werden

HIER WEITER: Was ist Spieltheorie grundsätzlich?

Gefangenendilemma







\section{Von Neumann Morgenstern}

Da ist zunächst ihr Entwickler, John von Neumann. Ein Mathematik-Genie. Tätig in unzähligen Gebieten, neben Mathematiker und Ökonom gilt er als einer der Entwickler des modernen Computers und entwickelte eine binäre Programmiersprache. Er arbeitete an der Entwicklung der Quantenmechanik und der Wasserstoffbombe mit. \textcite[S. 232]{Bernstein1996} zitiert, dass er "`während seiner Zeit beim Militär Admiräle gegenüber Generälen bevorzugte, da diese trinkfester waren"' und weitere Geschichten, die ihn als lebenslustiges Genie darstellen. 


Spieltheorie 1920
Erwartungsnutzen und Spieltheorie


\section{Nash}
Berührend ist die Geschichte von John Forbes Nash, die nicht zuletzt durch den Film "`A Beautiful Mind"' weit über wissenschaftliche Kreise hinaus bekannt wurde. Nash verfasste 1950 eine geniale und nur ca. 30 Seiten lange Dissertation, die später in verkürzter Form publiziert wurde \parencite{Nash1951} und mit der er die Spieltheorie entscheidend weiterentwickelte. Ende der 1950er Jahre erkrankte er aber schwer an Schizophrenie und war die folgenden 25 Jahre stark eingeschränkt. Erst in seinen Fünfzigern erholte er sich. 1994 wurde ihm gemeinsam mit Reinhard Selten und John Harsanyi der Nobelpreis für Ökonomie zugesprochen. In einem interessanten Interview \parencite{Nash2004}\footnote{https://www.nobelprize.org/prizes/economic-sciences/1994/nash/interview/} im Rahmen des Ersten Nobelpreisträgertreffens 2004 erzählt er, dass die Verleihung des Nobelpreise einen enormen Einfluss auf sein Leben hatte, war er doch zuvor schon lange Zeit arbeitslos, trotzdem er schon längere Zeit bei guter Gesundheit war. Die Tragik in seinem Leben setzte sich übrigens bis zu seinem Tod fort: Im Jahr 2015 erhielt er den Abel-Preis, eine Auszeichnung für Mathematiker. Bei der Rückkehr aus Norwegen, wo der Preis verliehen wird, war das Taxi, das ihn vom Flughafen nach Hause bringen sollte in einen Autounfall verwickelt, der Nash und seiner Frau das Leben kostete. 

Studierende der Ökonomie kennen seinen Namen selbst dann, wenn ihr Studium keine Vorlesung zur Spieltheorie enthält. Schließlich haben "`Nash-Gleichgewichte"' in der modernen Ökonomie einen Fixplatz in vielen Modellen eingenommen. 




\section{Hurwicz, Aumann, Harshani, Selten}

\section{Marktdesign und Auktionstheorie}

Marktdesign: Alvin Roth und Lloyd Shapley 
Auktionstheorie: \textit{Paul Milgrom} und \textit{Robert Wilson}



