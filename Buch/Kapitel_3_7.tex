%%%%%%%%%%%%%%%%%%%%% chapter.tex %%%%%%%%%%%%%%%%%%%%%%%%%%%%%%%%%
%
% sample chapter
%
% Use this file as a template for your own input.
%
%%%%%%%%%%%%%%%%%%%%%%%% Springer-Verlag %%%%%%%%%%%%%%%%%%%%%%%%%%

\chapter{Spieltheorie} \label{cha: Spieltheorie}
\label{Spieltheorie}

Sie wird häufig als Spezialfall und Weiterentwicklung der Entscheidungstheorie, oder als "`Interaktive Entscheidungstheorie"' bezeichnet und in der Ökonomie noch immer eher als Randthema behandelt, dabei ist sie wohl eine \textit{der} wesentlichen Weiterentwicklungen der Wirtschaftswissenschaften im 20. Jahrhundert: Die Spieltheorie. Ihre Bedeutung kann kaum überschätzt werden. Sowohl in der Mikroökonomie, als auch in der Makroökonomie ist die Spieltheorie Teil unzähliger Modelle. Auch diesbezüglich wandelte sich die Ökonomie: Ausgehend von den Neoklassikern, aber eben auch die Keynesianer und die Monetaristen, suchten nach "`Pareto-optimalen"' Lösungen. Deren Modelle gehen von vollkommenen Konkurrenzmärkten aus, alle Teilnehmer sind Preisnehmer. Sie optimieren also ihr individuelles Verhalten im Anbetracht eines Marktgleichgewichts. Vertreter der Neuen Neoklassischen Synthese hingegen sprechen stattdessen meist von "`Nash-Gleichgewichten"' - also einem spieltheoretischen Gleichgewichtszustand.  Diese neueren Modelle (vgl. Kapitel \ref{cha: Neu Keynes} und \ref{Neue Neoklassische Synthese}, aber auch der Bereich Politische Ökonomie) berücksichtigen, dass die Annahme vollkommener Konkurrenzmärkte häufig unrealistisch ist. Die entsprechenden Optimierungsaufgaben hängen also wechselseitig vom Verhalten der Marktgegenseite ab. 

Wie bereits beschrieben, fristet die Spieltheorie dennoch in gewisser Art und Weise eine Außenseiterrolle innerhalb der Ökonomie. Über die Gründe kann man hier nur spekulieren. Wahrscheinlich spielt es aber eine Rolle, dass die Spieltheorie keine volkswirtschaftliche Theorie im eigentlichen Sinne ist. Ganz im Gegenteil, ihre Aussagen sind auch in der Politik, Biologie, Betriebswirtschaft, Spiel und Sport von Interesse. Dazu passt auch, dass die Spieltheorie nicht von Ökonomen, sondern von Mathematikern entwickelt wurde. Tatsächlich findet man Vorlesungen zur Spieltheorie aber vor allem in wirtschaftswissenschaftlichen Curricula wieder. Zuletzt ist es wohl kaum zu bestreiten, dass die Spieltheorie - stärker als jede andere Disziplin - von bemerkenswerten und oft kontroversen Persönlichkeiten geprägt wurde, von denen uns in diesem Kapitel mehrere unterkommen werden.

Was ist Spieltheorie grundsätzlich? \textcite[S. 136]{Harsanyi1994} bringt es in einem Satz auf den Punkt: "`Spieltheorie ist eine Theorie der strategischen Interaktion. Das heißt, sie ist eine Theorie des rationalen Verhaltens in Situationen, in denen jeder Spieler die wahrscheinlichen Gegenzüge seines Gegenspielers bedenkt und darauf basierend seine eigenen Züge setzt."' Das erinnert zunächst an wirkliche Spiele wie Mühle, Schach oder Poker. Tatsächlich wird zum Beispiel beim Poker mittels spieltheoretischer Ansätze die Gewinnwahrscheinlichkeit einer "`Hand"' berechnet. Der Name \textit{Spiel}theorie ist dennoch etwas irreführend, weil sie in der Realität auf verschiedene Situationen angewendet werden kann, wie individuelle soziale Interaktionen, politische Konflikte, oder eben in verschiedenen wirtschaftlichen Situationen. Die frühe Spieltheorie bei Von Neumann und Morgenstern behandelte sogenannte Nullsummen-Situationen. Das sind Situationen in denen die Gewinne des einen Spielers betragsmäßig den Verlusten seines Gegenspielers stets entsprechen. Eine Situation, die eben tatsächlich vor allem bei wirklichen Spielen auftritt: Wenn Weiß beim Schach einen Läufer verliert kann man auch vom Gewinn eines Läufers durch Schwarz sprechen. Eine wesentliche Erweiterung erfuhr die Spieltheorie Anfang der 1950er Jahre durch John Forbes Nash, dessen Arbeit die Spieltheorie auch auf Nicht-Nullsummen-Situationen ausweitete \parencite[S. 163]{Nash1994}. Wirtschaftliche Kooperation zum Beispiel kann zum Beispiel dafür sorgen, dass beide "`Spielteilnehmer"' ihre Position zu verbessern. Berühmt geworden sind aber vor allem jene Beispiele, bei denen individuelle Nutzenmaximierung zu gesamtwirtschaftlich schlechten Ergebnissen führen - dazu aber später mehr, Stichwort: Gefangenendilemma.

Vorab machen wir das abstrakte Feld der Spieltheorie ein bisschen greifbarer. Die nachstehende Grafik zeigt eine typische Darstellung eines spieltheoretischen Problems. Konkret handelt es sich um ein \textit{nicht kooperatives} (das sieht man nicht aus der Darstellung), \textit{zwei Personen, Nicht-Nullsummen}-Spiel. \textit{Spieler 1} kann aus seinen beiden Strategien wählen \textit{Leugnen, $S_{1,1}$} oder \textit{Gestehen, $S_{1,2}$}. Das Gleiche gilt hier für \textit{Spieler 2}. Es entsteht eine $2x2$-Matrix mit jeweils einem Auszahlungspaar, wobei die erste Zahl jeweils als der Nutzen für Spieler 1 gelesen werden kann. In diesem Fall ist der Nutzen negativ angegeben, was nur ausdrückt, dass man seinen Nutzen maximiert, indem man den geringsten negativen Betrag anstrebt. Vorweggenommen: Rein intuitiv ist klar, welche Lösung gesamtwirtschaftlich (Spieler 1 und Spieler 2 stellen die Gesamtwirtschaft dar) angestrebt wird: Beide sollten die Strategie \textit{Leugnen} wählen. Gesamtwirtschaftlich tritt dann der größte Nutzen (=kleinster Schaden) ein und die Situation keines Spielers könnte verbessert werden \textit{ohne} die Situation des anderen zu verschlechtern. Definitionsgemäß ist ein Pareto-Optimum erreicht. Aber zu welcher Lösung kommt man mit spieltheoretischen Ansätzen? - auch dazu im Laufe des Kapitels mehr.


\begin{tikzpicture}[element/.style={minimum width=3.0cm, minimum height=0.8cm}]
\matrix (m) [matrix of nodes,nodes={element},column sep=-\pgflinewidth, row sep=-\pgflinewidth,]{
	& Leugnen $S_{2,1}$  & Gestehen $S_{2,2}$  \\
	 Leugnen $S_{1,1}$ & |[draw]|-2 / -2 & |[draw]|-10 / -1 \\
	 Gestehen $S_{1,2}$ & |[draw]|-1 / -10 & |[draw]|-8 / -8 \\    };

\node[above=0.15cm] at ($(m-1-2)!0.5!(m-1-3)$){\textbf{Spieler 2}};
\node[rotate=90] at ($(m-2-1)!0.5!(m-3-1)+(-1.25,0)$){\textbf{Spieler 1}};
\end{tikzpicture}

Diese Form der Darstellung wird übrigens "`Normalform"' genannt. Die zweite übliche Darstellungsform nennt man "`extensive Form"'. Diese umfasst für den gesamten Spielverlauf alle notwendigen Informationen zu Entscheidungen und Auszahlungen und gleicht von der Darstellung her einem Entscheidungsbaum. Beide Formen wurden laut \textcite{Selten2001} übrigens schon in der Geburtsstunde der Spieltheorie so verwendet.


\section{Von Neumann Morgenstern}

Vereinzelte Ansätze, die Ideen der Spieltheorien vorwegnahmen, gab es bereits im 19. Jahrhundert. Das bekannteste Beispiel ist wohl die Duopol-Theorie von \textcite{Cournot1836}. Weithin gilt aber die Veröffentlichung des fundamentalen Werks "`Theory of Games and Economic Behavior"' im Jahr 1944 durch Oskar Morgenstern und John von Neumann als Ursprung der Spieltheorie. \textcite{VonNeumann1928} behandelte bereits einen speziellen Ansatz der Theorie, im Buch von 1944 wurde dieser verbreitert und verallgemeinert. In der zweiten Auflage im Jahr 1947 wurde der Beweis für die Axiome der Erwartungsnutzentheorie erbracht. Diese haben wir bereits im Kapitel \ref{Erwartungsnutzen} kennen gelernt und sind eigentlich mehr Voraussetzung für die Spieltheorie als Teil derselben \parencite[S. 3]{Selten2001}. 

Als "`Vater der Spieltheorie"' gilt also John von Neumann. Ein Mathematik-Genie. Tätig in unzähligen Gebieten, neben Mathematiker und Ökonom gilt er als einer der Entwickler des modernen Computers und er entwickelte eine binäre Programmiersprache. Er arbeitete an der Entwicklung der Quantenmechanik und der Wasserstoffbombe mit. \textcite[S. 232]{Bernstein1996} zitiert, dass er "`während seiner Zeit beim Militär Admiräle gegenüber Generälen bevorzugte, da diese trinkfester waren"' und weitere Geschichten, die ihn als lebenslustiges Genie darstellen. Der Beitrag seines Koautors Oskar Morgenstern - der in Österreich das Institut für Höhere Studien mitbegründete - wird in der modernen Literatur häufig in Frage gestellt. In \textcite[S. 494]{Leonard1994} wird dargestellt, dass Morgenstern, erstens unter Ökonomen-Kollegen in den USA recht unbeliebt war und zweitens, einen\textit{inhaltlichen}-bescheidenen Beitrag zur hoch-mathematischen Spieltheorie von Neumann's beigetragen hat. Insbesondere im Artikel \textcite{Morgenstern1976} stellt der Autor seinen eigenen Beitrag zur Entstehung der "`Theory of Games and Economic Behavior"' gänzlich anders dar. Laut \textcite{Nash1994} ist es aber unzweifelhaft, dass dieses fundamentale Werk ohne die Zusammenarbeit von Oskar Morgenstern und John von Neumann in dieser Form nicht entstanden wäre.

HIER WEITER (aus Selten-Text) Vor allem Nullsummenspiele
Der Beitrag war speziell: Ein (nicht-kooperatives) zwei-Personen Nullsummenspiel und die Minimax-Lösung dafür.
Was war nun aber der Inhalt dieser frühen Spieltheorie?




\section{Nash: Das tragische Genie}
Berührend ist die Geschichte von John Forbes Nash, die nicht zuletzt durch den Film "`A Beautiful Mind"' weit über wissenschaftliche Kreise hinaus bekannt wurde. Nash verfasste 1950 eine geniale und nur großzügige\parencite[S. 164]{Nash1994} 27 Seiten lange Dissertation \parencite{Nash1950}, die später als Journalbeitrag publiziert wurde \parencite{Nash1951}, und mit der er die Spieltheorie entscheidend weiterentwickelte. Ende der 1950er Jahre erkrankte er aber schwer an Schizophrenie und war die folgenden 25 Jahre stark eingeschränkt. Erst in seinen Fünfzigern erholte er sich. 1994 wurde ihm gemeinsam mit Reinhard Selten und John Harsanyi der Nobelpreis für Ökonomie zugesprochen. In einem interessanten Interview \parencite{Nash2004}\footnote{https://www.nobelprize.org/prizes/economic-sciences/1994/nash/interview/} im Rahmen des Ersten Nobelpreisträgertreffens im Jahr 2004 erzählt er, dass die Verleihung des Nobelpreise einen enormen Einfluss auf sein Leben hatte, war er doch zuvor schon lange Zeit arbeitslos, obwohl er schon längere Zeit bei guter Gesundheit war und er daher mehr oder weniger kaum noch am öffentlichen Leben teilnahm. Die Tragik in seinem Leben setzte sich übrigens bis zu seinem Tod fort: Im Jahr 2015 erhielt er den Abel-Preis, eine Auszeichnung für Mathematiker. Bei der Rückkehr aus Norwegen, wo der Preis verliehen wird, war das Taxi, das ihn vom Flughafen nach Hause bringen sollte, in einen Autounfall verwickelt, der Nash und seiner Frau das Leben kostete. 

Nash erweiterte die Erweiterung in verschiedener Hinsicht. Am grundlegendsten ist wohl seine \textit{Unterscheidung} zwischen "`Kooperativer Spieltheorie"' und "`Nicht-kooperative Spieltheorie"'. In Letztgenannter geben Spieler kein Commitment zu einer bestimmten Strategie ab, während bei der "`Kooperativen Spieltheorie"' durchsetzbare Verträge vorhanden sein können.
Den größten nachhaltigen Beitrag stellt sicherlich seine Gleichgewichtslösung für nicht-kooperative Spiele dar \parencite{Nash1951, Nash1950b}. Studierende der Ökonomie kennen vor allem deshalb seinen Namen, selbst dann, wenn ihr Studium keine Vorlesung zur Spieltheorie enthält. Schließlich haben diese "`Nash-Gleichgewichte"' in der modernen Ökonomie einen Fixplatz in vielen Modellen eingenommen. Für "`Nicht-kooperative Spiele"' bewies Nash, dass es in solchen Situation immer zumindest ein Nash-Gleichgewicht gibt. Also ein Gleichgewicht, in dem kein Spieler seine Auszahlungen erhöhen kann, indem er einseitig seine Strategie verändert. Das Konzept des Nash-Gleichgewichtes ist grundsätzlich damit erklärt, in der Realität aber recht schwierig vollständig zu erfassen. Um ein Nash-Gleichgewicht zu finden, muss man in einem zwei-Personen Spiel zunächst die "`besten Antworten"' auf alle Strategien des Gegenspieler suchen. Angewendet auf die obenstehende Grafik würde das bedeuten Spieler 1 überlegt sich nacheinander, welche Strategie - Leugnen oder Gestehen - er spielen würde, gegeben Spieler 2 spielt seinerseits eine \textit{bestimmte} Strategie. Spieler 1 hat nun seine "`besten Antworten"' gefunden. Dies allein bringt aber noch gar nichts, denn entscheidend ist der zweite Schritt. Spieler 1 muss nämlich davon ausgehen, dass auch Spieler 2 in gleicher Weise vorgeht. Das heißt, Spieler 1 berücksichtigt, dass Spieler 2 seinerseits die besten Antworten auf die verschiedenen Strategien von Spieler 1 identifiziert. Findet sich nun ein Strategiepaar, von dem keiner der beiden Spieler ein Grund hat einseitig abzuweichen, liegen wechselseitig beste Antworten vor und somit ein Nash-Gleichgewicht.

Betrachten wir diese Vorgehensweise anhand der dargestellten Matrix: Spieler 1 betrachtet also nacheinander möglichen Strategien von Spieler 2. Sollte dieser "`Leugnen $S_{2,1}$"' (Beachten Sie nur Spalte 1), so müsste Spieler 1 zweifellos "`Gestehen $S_{1,2}$"' als "`beste Antwort"' wählen. Schließlich würde er so seinen Nutzen von -2 auf -1 erhöhen. Sollte Spieler 2 "`Gestehen $S_{2,2}$"' (Beachten Sie nur Spalte 2), so müsste Spieler 1 ebenfalls mit "`Gestehen $S_{1,2}$"' antworten, da der Nutzen von -10 auf -8 steigt. Die Betrachtung des Spiels aus Sicht von Spieler 2 sieht in diesem Fall genau identisch aus. Das Ergebnis ist also, dass beide Spiel jeweils "`Gestehen"' als "`wechselseitig beste Antwort"' identifizieren. 

Das dargelegte Beispiel ist das "`Gefangenendilemma"', welches der Doktorvater und Förderer von Nash, Albert Tucker, im Frühjahr 1950 für eine Psychologie(!)-Vorlesung in Stanford entwickelte \parencite[S. 161]{Nash1994}. Darin wird ein Verbrecherduo unabhängig voneinander verhört. Leugnen beide, kann man ihnen wenig nachweisen und beide kommen nach zwei Jahren aus dem Gefängnis. Leugnet nur einer, profitiert er von der Kronzeugenregelung und kommt nach einem Jahr frei, während sein Partner die Höchststrafe absitzen muss. Gestehen beide, erhalten zwar beide eine kleine Milderung, kommen aber erst nach acht Jahren frei. Bei Gesamt-Betrachtung wäre es natürlich am besten für beide zu Leugnen. Wir haben aber gerade gesehen, dass es individuell Nutzen-maximierend ist zu gestehen. Das gefundene Nash-Gleichgewicht (beide Gestehen) führt zu der paradoxen Lösung ist, dass das insgesamt schlechtest-mögliche Ergebnis - nämlich beide "`sitzen"' für jeweils acht Jahre - realisiert wird. Haben Sie schon mal überlegt, was eine Kronzeugenregelung bringen soll? - Ganz genau, formalisiert betrachtet ist die Kronzeugenregelung ein spieltheoretischer Ansatz Gesetzesbrecher dazu zu bringen aus individuell-rationalen Gründern zu gestehen. Bekannt wurde in diesem Zusammenhang auch das Beispiel von \textcite{Blinder1982}, in dem er eine entsprechende Dilemma-Situation in der Wirtschaftspolitik darlegte. So wäre es im Fall einer Wirtschaftskrise für den Staat rational expansive Fiskalpolitik zu betreiben. Eine unabhängige Zentralbank, die nur Preisstabilität als Ziel hat, müsste als (wechselseitig beste und strikt dominante) Antwort darauf - weil sie steigende Inflation aufgrund der expansiven Fiskalpolitik befürchtet - eine restriktive Geldpolitik durchführen. Das entsprechende Nash-Gleichgewicht wäre genau das Gegenteil vom wünschenswerten Vorgehen laut \textcite{Blinder1982}: Nämlich eine expansive Geldpolitik bei restriktiver Fiskalpolitik. Aber es gibt durchaus auch aktuelle und lebensnahe Beispiele: Haben Sie sich schon mal geärgert warum die Leute so "`dumm"' sind und ihren eigenen Planeten ausbeuten und die Umwelt zerstören? Menschen und auch Entscheidungsträger - also Politiker - sind keineswegs dumm. Aber wenn eine einzelne Person (oder Staat), für sich entscheidet "`Null-Emissionen"' zu verursachen, bewirkt dies für die gesamte Umwelt verschwindend wenig, vermindert aber die Lebensqualität des einzelnen ganz erheblich. Nachdem wir das alle wissen, starten wir erst gar nicht damit "`Null-Emissionen"' anzustreben. Eben eine Dilemmata-Situation. Aber keine Dummheit, sondern individuell Nutzen-maximierendes Verhalten\footnote{Moderne spieltheoretische Überlegungen analysieren aber genau dieses Problem: Welcher Rahmen müsste geschaffen werden, dass es individuell-rational ist, die Umwelt zu schonen.}.

Das "`Gefangenendilemma"' wurde weltberühmt, wird in jedem Buch, welches Spieltheorie auch nur streift, erwähnt und gilt als \textit{das} Einführungsbeispiel schlechthin. Es ist aber in Wahrheit kein sehr gutes Beispiel, weil das Ergebnis eine "`strikt dominante Strategie"' ist. Für beide Spieler gibt es nur eine optimale Strategie, unabhängig davon was der andere Spieler macht. Diese Lösung ist also eine sehr einfache und derartige Spiele sind in der Realität sehr selten. Das Beispiel neigt daher dazu die Komplexität der Spieltheorie zu unterschätzen. Interessante spieltheoretische Situationen entstehen erst, wenn die eigenen Erwartungen über das Verhalten der Gegenspieler miteinbezogen werden muss. Der Findungsprozess ist dann eben jener wie der oben beschriebene zweistufige Prozess, allerdings führt der zweite Schritt nicht dazu, dass Spieler 1 eine Strategie findet, die er immer anwenden kann, weil sie strikt dominant ist. Stattdessen findet Spieler 1 heraus, dass er eine bestimmte Strategie spielen muss, damit er davon ausgehen kann, dass Spieler 2 ebenfalls keinen rationalen Grund findet von einer bestimmte Strategie abzuweichen. Wenn sich dadurch wechselseitig beste Antworten finden handelt es sich um ein "`striktes Nash Gleichgewicht in reinen Strategien"'. Es sind auch Lösungen möglich bei denen ein Spieler sich bei Abweichung weder verschlechtert noch verbessert, dann handelt es sich um ein "`schwaches Nash Gleichgewicht"'. Oftmals existieren schlicht keine "`wechselseitig besten Antworten"'. Dann existiert eben auch kein Nash-Gleichgewicht in \textit{reinen} Strategien. Dann müssen die Spieler sich "`zufällig"' für eine Strategie entscheiden. Formal werden dann die einzelnen Strategien mit Eintrittswahrscheinlichkeiten hinterlegt. Man spricht dann von einem "`Spiel mit gemischten Strategien"'. \textcite{Nash1951} bewies, dass es in allen (endlichen) Spielen zumindest ein "`Nash-Gleichgewicht in gemischten Strategien"' gibt.

Das Nash-Gleichgewicht hat sich laut \textcite[S. 61]{Holler2005} aus zwei Gründen als die wesentliche Gleichgewichts-Strategie etabliert. Erstens, weil sich jeder Spieler darin rational verhält und zweitens, weil es der natürliche Endpunkt eines dynamischen Anpassungsprozesses ist, in dessen Verlauf die Spieler aus Enttäuschungen lernen. Gerade der zweite Punkt ist wichtig, denn häufig stößt man auf Unverständnis, wenn man das Nicht-Pareto-optimale Nash-Gleichgewichtsergebnis als Endzustand präsentiert, im Sinne von: "`Das gibt es doch nicht, dass die Spieler dieses Ergebnis akzeptieren, wenn doch alle wissen, es gäbe eine für alle bessere Lösung."'\footnote{Vor allem Pseudo-Wirtschaftswissenschaftliche Schulen, wie zum Beispiel die "`Gemeinwohl-Ökonomie"' verlocken mit solchen Argumenten.} Aber gerade der dynamische Anpassungsprozess ist das entscheidende, warum am Ende eben doch die suboptimale Nash-Lösung übrig bleibt. Angenommen sie wären einer von zwei Spielern im Gefangenendilemma und würden großmütig beschließen zu leugnen in der Überzeugung ihr Partner würde ebenso handeln. Erstens, kann sich ihr Partner nicht darauf verlassen, dass sie tatsächlich selbstlos handeln und müsste dementsprechend gestehen. Und zweitens - und das ist das entscheidende - wäre die Verlockung für ihren Partner nicht viel zu groß ihnen eins auszuwischen und selbst dann zu gestehen, wenn sie ihm versprechen zu leugnen? Beachten sie, dass sich im klassischen Gefangenendilemma-Beispiel die beiden Spieler offensichtlich kennen. In der Realität treffen aber Personen aufeinander, die sich erstens, nicht kennen und zweitens, womöglich niemals wieder aufeinander treffen. In solchen Situationen wird der großmütige Partner solange ausgenutzt, bis er alle Ressource verloren hat, oder seine edle Strategie aufgibt.

Laut dem Artikel \textcite{Cassidy2015} war John von Neumann übrigens wenig beeindruckt von der Erweiterung "`seiner"' Spieltheorie durch Nash, vielmehr sah er darin eine "`triviale Folge"' aus dem Fixpunkttheorem von \textcite{Brouwer1912}. Wichtige Wegbegleiter für Nash waren dessen Kommilitone Harold Kuhn und sein Doktorvater Albert Tucker, die in der Folge ebenfalls wichtige Beiträge zur Spieltheorie lieferten. Studierenden der Ökonomie sind die beiden aber vor allem für ihren Beitrag zur "`Nicht-lineare Optimierung"` und der dort verwendeten "`Kuhn-Tucker-Bedingung"' \parencite{Kuhn1951} bekannt. Vor allem Harold Kuhn soll laut \textcite{Rubinstein2003} das Nobelpreis-Komitee überzeugt haben, dass Nash's mentale Verfassung der Nobelpreis-Verleihung nicht im Wege stehen sollte. Diesen bekam er schließlich im Jahr 1994 tatsächlich verliehen, zusammen mit zwei Kollegen, die auf seinen Theorien aufbauend die Spieltheorie weiterentwickelten und im nächsten Unterkapitel behandelt werden.


\section{Hurwicz, Aumann, Harsanyi, Selten}

Die Entwicklung der Spieltheorie fand mit den Arbeiten von Nash Anfang der 1950er Jahre zweifellos einen Höhepunkt. Dies ist unter anderem daran festzumachen, dass "`Nash-Gleichgewichte"' wie oben dargestellt, noch immer ein wesentliches Entscheidungsinstrument in der modernen Ökonomie ist. Die Spieltheorie wurde in der Folge in verschiedenen Bereichen verfeinert und weiterentwickelt. Ohne näher darauf einzugehen, wurde bislang angenommen, dass die zu spielenden Optionen (Strategiemengen der Spieler) und die daraus folgenden Ergebnisse (Auszahlungsfunktionen) jeweils allen Spielern bekannt sind. Wir sind also - ohne es so zu nennen, also implizit - von einem "`Spiel mit vollständiger Information"' ausgegangen. Sämtliche Informationen sind dann gemeinsames Wissen ("`Common Knowledge"'), das \textcite{Aumann1976} formal beschrieben hat.

Schon \textcite{Morgenstern1944} unterschieden zwischen "`Spielen mit vollständiger Information"' und solchen mit "`unvollständiger Information"', aber erst in den 1960er Jahren wurden letztgenannte tatsächlich spieltheoretisch analysiert \parencite[S. 137]{Harsanyi1994}. Damit wurde die Tatsache berücksichtigt, dass das Vorliegen vollständiger Information in der Realität häufig eher Ausnahme als Regel ist. "`Unvollständige Information"' bedeutet, dass die Spieler recht allgemein kein vollständiges Wissen über die Handlungen der Mitspieler, möglichen Strategien, Ressourcen oder Auszahlungsfunktionen der Gegner haben \parencite[S. 137]{Harsanyi1994}. Diese Unsicherheit besteht bereits vor Spielbeginn und ist das spieltheoretische Äquivalent zur "`Adversen Selection"', die in Kapitel \ref{Info} beschrieben wird. Neben der (Un)vollständigkeit der Information unterscheidet man weiters zwischen "`Perfekter"' und "`Imperfekter"' Information. Hier beschränkt sich der Informationsmangel auf konkrete, vergangene Handlungen der Gegenspieler.  Die Informationsasymmetrie tritt hier während des Spiels auf. Der Versuch diese Informationsvorteile einseitig zu nutzen kann als die spieltheoretische Analyse des "`Moral Hazards"' interpretiert werden, die in Kapitel \ref{Info} beschrieben wird. Man könnte diese Unterschiede auch so zusammenfassen: Bei unvollständiger Information wissen die Spieler nicht einmal die Regeln des Spiels. Bei vollständiger, aber imperfekter Information wissen alle Spieler die Regeln und auch die möglichen Strategien des Spiels, können aber während des Spiels die Züge des Gegenspielers nicht beobachten. Bei perfekter Information schließlich sind Regeln und alle Züge des Gegners bekannt. Schach, zum Beispiel, ist ein Spiel mit vollständiger, perfekter Information. Theoretische könnte man jeweils eine "`optimale Strategie"' berechnen. Die Faszination dieses Spiels liegt darin, dass es eine in der Praxis unendlich erscheinende Zahl an Kombination verschiedener Züge gibt. Bei Kartenspielen gibt es in der Regel wesentlich weniger Züge. Die Faszination dieser Spiele liegt dafür darin, dass man seine eigenen Karten und möglichen Züge vor dem Gegenspieler verbergen kann. Hier liegt ein Spiel mit imperfekter Information vor. In diesem Fall ist eine konkrete, optimale Strategie zu berechnen nicht möglich. Sehr wohl kann man aber aus den eigenen Karten berechnen, wie hoch die statistische Gewinn-Wahrscheinlichkeit mit dem Blatt ist. Geübte Pokerspieler benötigen zwar ein Poker-Face um sich nicht zu verraten, ihre wahre Stärke liegt aber darin, genau diese Gewinn-Wahrscheinlichkeiten rasch berechnen zu können.

\textcite{Harsanyi1967} zeigte nun, in drei aufgeteilten Artikeln, die aber alle im gleichen Journal erschienen sind, dass "`Spiele mit unvollständiger Information"' unter der Anwendung bedingter Wahrscheinlichkeiten wie Spiele mit "`vollständiger, aber imperfekter Information"' analysiert werden können. Aufgrund der Anwendung bedingter Wahrscheinlichkeiten werden diese Spiele "`Bayes'sche Spiele genannt"'.  HIER WEITER

Selten:  Perfekte vs. Imperfekte Information





davon abzugrenzen:

Harsanyi: Nach 1960: Einführung der Spiele mit inkompletter Information . Harsanyi 1994.
Bayes'sche Spiele Harsanyi 1967/68. S. 78 im Holler-Illing-Buch.


Aumann 1987: Gleichgewicht in korrelierten Spielen (S. 88 Holler Illing) Aumann 1976: Gemeinsames Wissen (S. 43 Holler Illing)


Irgendwo einbauen: Die Evolution der Kooperation (Axelrod)
\section{Marktdesign und Auktionstheorie}

Marktdesign: Alvin Roth und Lloyd Shapley 
Auktionstheorie: \textit{Paul Milgrom} und \textit{Robert Wilson} (Seite 243 im Bernstein-Buch)



