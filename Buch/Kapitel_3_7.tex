%%%%%%%%%%%%%%%%%%%%% chapter.tex %%%%%%%%%%%%%%%%%%%%%%%%%%%%%%%%%
%
% sample chapter
%
% Use this file as a template for your own input.
%
%%%%%%%%%%%%%%%%%%%%%%%% Springer-Verlag %%%%%%%%%%%%%%%%%%%%%%%%%%

\chapter{Spieltheorie} \label{cha: Spieltheorie}
\label{Spieltheorie}

Sie wird häufig als Spezialfall und Weiterentwicklung der Entscheidungstheorie, oder als "`Interaktive Entscheidungstheorie"' bezeichnet und in der Ökonomie noch immer eher als Randthema behandelt, dabei ist sie wohl eine \textit{der} wesentlichen Weiterentwicklungen der Wirtschaftswissenschaften nach dem Zweiten Weltkrieg: Die Spieltheorie. Ihre Bedeutung kann kaum überschätzt werden. Sowohl in der Mikroökonomie, als auch in der Makroökonomie ist die Spieltheorie teil vieler Modelle. Auch diesbezüglich wandelte sich die Ökonomie. Ausgehend von den Neoklassikern, aber eben auch die Keynesianer und die Monetaristen, suchten nach "`Pareto-optimalen"' Zuständen. Die Neuen Klassiker, Neu-Keynesianer hingegen sprechen stattdessen meist von "`Nash-Gleichgewichten"' - also einem spieltheoretischen Gleichgewichtszustand. Wie bereits beschrieben, fristet die Spieltheorie in gewisser Art und Weise eine Außenseiterrolle innerhalb der Ökonomie. Über die Gründe kann man nur spekulieren. Wahrscheinlich spielt es aber eine Rolle, dass Spieltheorie nicht notwendigerweise eine volkswirtschaftliche Theorie ist. Ganz im Gegenteil, ihre Aussagen sind auch in der Politik, Betriebswirtschaft und im Sport von Interesse. Dazu passt auch, dass die Spieltheorie nicht von Ökonomen, sondern von Mathematikern entwickelt wurde. Tatsächlich findet man Vorlesungen zur Spieltheorie aber vor allem in wirtschaftswissenschaftlichen Curricula wieder. Zuletzt ist es wohl kaum zu bestreiten, dass die Spieltheorie - stärker als jede andere Disziplin - von bemerkenswerten und oft kontroversen Persönlichkeiten geprägt wurde. Da ist zunächst ihr Entwickler, John von Neumann. Ein Mathematik-Genie. Tätig in unzähligen Gebieten, neben Mathematiker und Ökonom gilt er als einer der Entwickler des modernen Computers und entwickelte eine binäre Programmiersprache. Er arbeitete an der Entwicklung der Quantenmechanik und der Wasserstoffbombe mit. \textcite[S. 232]{Bernstein1996} zitiert, dass er "`während seiner Zeit beim Militär Admiräle gegenüber Generälen bevorzugte, da diese trinkfester waren."' Insgesamt wird er häufig als lebenslustiges Genie dargestellt. Berührend ist die Geschichte von John Forbes Nash, der die Spieltheorie mit seiner 8-Seiten Dissertation entscheidend weiterentwickelte, danach aber psychisch erkrankte und erst im fortgeschrittenen Alter wieder einer normalen Tätigkeit nachgehen konnte. Seine Geschichte wurde mit "`A beautiful Mind"' verfilmt. Tragisch ist sein Tod: Taxi Autounfall. Berührendes Interview (Nobelpreis-Seite)


\section{Von Neumann Morgenstern}
Spieltheorie 1920
Erwartungsnutzen und Spieltheorie


\section{Nash und Hurwicz}

\section{Aumann, Harshani, Selten}

\section{Marktdesign und Auktionstheorie}

Marktdesign: Alvin Roth und Lloyd Shapley 
Auktionstheorie: \textit{Paul Milgrom} und \textit{Robert Wilson}



