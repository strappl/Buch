%%%%%%%%%%%%%%%%%%%%% chapter.tex %%%%%%%%%%%%%%%%%%%%%%%%%%%%%%%%%
%
% sample chapter
%
% Use this file as a template for your own input.
%
%%%%%%%%%%%%%%%%%%%%%%%% Springer-Verlag %%%%%%%%%%%%%%%%%%%%%%%%%%

\chapter{Spieltheorie} \label{cha: Spieltheorie}
\label{Spieltheorie}

Sie wird häufig als Spezialfall und Weiterentwicklung der Entscheidungstheorie, oder als "`Interaktive Entscheidungstheorie"' bezeichnet und in der Ökonomie noch immer eher als Randthema behandelt, dabei ist sie wohl eine \textit{der} wesentlichen Weiterentwicklungen der Wirtschaftswissenschaften im 20. Jahrhundert: Die Spieltheorie. Ihre Bedeutung kann kaum überschätzt werden. Sowohl in der Mikroökonomie, als auch in der Makroökonomie ist die Spieltheorie Teil unzähliger Modelle. Auch diesbezüglich wandelte sich die Ökonomie: Ausgehend von den Neoklassikern, aber eben auch die Keynesianer und die Monetaristen, suchten nach "`Pareto-optimalen"' Lösungen. Deren Modelle gehen von vollkommenen Konkurrenzmärkten aus, alle Teilnehmer sind Preisnehmer. Sie optimieren also ihr individuelles Verhalten im Anbetracht eines Marktgleichgewichts. Vertreter der Neuen Neoklassischen Synthese hingegen sprechen stattdessen meist von "`Nash-Gleichgewichten"' - also einem spieltheoretischen Gleichgewichtszustand.  Diese neueren Modelle (vgl. Kapitel \ref{cha: Neu Keynes} und \ref{Neue Neoklassische Synthese}, aber auch der Bereich Politische Ökonomie) berücksichtigen, dass die Annahme vollkommener Konkurrenzmärkte häufig unrealistisch ist. Die entsprechenden Optimierungsaufgaben hängen also wechselseitig vom Verhalten der Marktgegenseite ab. 

Wie bereits beschrieben, fristet die Spieltheorie dennoch in gewisser Art und Weise eine Außenseiterrolle innerhalb der Ökonomie. Über die Gründe kann man hier nur spekulieren. Wahrscheinlich spielt es aber eine Rolle, dass die Spieltheorie keine volkswirtschaftliche Theorie im eigentlichen Sinne ist. Ganz im Gegenteil, ihre Aussagen sind auch in der Politik, Biologie, Betriebswirtschaft, Spiel und Sport von Interesse. Dazu passt auch, dass die Spieltheorie nicht von Ökonomen, sondern von Mathematikern entwickelt wurde. Tatsächlich findet man Vorlesungen zur Spieltheorie aber vor allem in wirtschaftswissenschaftlichen Curricula wieder. Zuletzt ist es wohl kaum zu bestreiten, dass die Spieltheorie - stärker als jede andere Disziplin - von bemerkenswerten und oft kontroversen Persönlichkeiten geprägt wurde, von denen uns in diesem Kapitel mehrere unterkommen werden.

Was ist Spieltheorie grundsätzlich? \textcite[S. 136]{Harsanyi1994} bringt es in einem Satz auf den Punkt: "`Spieltheorie ist eine Theorie der strategischen Interaktion. Das heißt, sie ist eine Theorie des rationalen Verhaltens in Situationen, in denen jeder Spieler die wahrscheinlichen Gegenzüge seines Gegenspielers bedenkt und darauf basierend seine eigenen Züge setzt."' Das erinnert zunächst an wirkliche Spiele wie Mühle, Schach oder Poker. Tatsächlich wird zum Beispiel beim Poker mittels spieltheoretischer Ansätze die Gewinnwahrscheinlichkeit einer "`Hand"' berechnet. Der Name \textit{Spiel}theorie ist dennoch etwas irreführend, weil sie in der Realität auf verschiedene Situationen angewendet werden kann, wie individuelle soziale Interaktionen, politische Konflikte, oder eben in verschiedenen wirtschaftlichen Situationen. Die frühe Spieltheorie bei Von Neumann und Morgenstern behandelte sogenannte Nullsummen-Situationen. Das sind Situationen in denen die Gewinne des einen Spielers betragsmäßig den Verlusten seines Gegenspielers stets entsprechen. Eine Situation, die eben tatsächlich vor allem bei wirklichen Spielen auftritt: Wenn Weiß beim Schach einen Läufer verliert kann man auch vom Gewinn eines Läufers durch Schwarz sprechen. Eine wesentliche Erweiterung erfuhr die Spieltheorie Anfang der 1950er Jahre durch John Forbes Nash, dessen Arbeit die Spieltheorie auch auf Nicht-Nullsummen-Situationen ausweitete \parencite[S. 163]{Nash1994}. Wirtschaftliche Kooperation zum Beispiel kann zum Beispiel dafür sorgen, dass beide "`Spielteilnehmer"' ihre Position zu verbessern. Berühmt geworden sind aber vor allem jene Beispiele, bei denen individuelle Nutzenmaximierung zu gesamtwirtschaftlich schlechten Ergebnissen führen - dazu aber später mehr, Stichwort: Gefangenendilemma.

Vorab machen wir das abstrakte Feld der Spieltheorie ein bisschen greifbarer. Die nachstehende Grafik zeigt eine typische Darstellung eines spieltheoretischen Problems. Konkret handelt es sich um ein \textit{nicht kooperatives} (das sieht man nicht aus der Darstellung), \textit{zwei Personen, Nicht-Nullsummen}-Spiel. \textit{Spieler 1} kann aus seinen beiden Strategien wählen \textit{Leugnen, $S_{1,1}$} oder \textit{Gestehen, $S_{1,2}$}. Das Gleiche gilt hier für \textit{Spieler 2}. Es entsteht eine $2x2$-Matrix mit jeweils einem Auszahlungspaar, wobei die erste Zahl jeweils als der Nutzen für Spieler 1 gelesen werden kann. In diesem Fall ist der Nutzen negativ angegeben, was nur ausdrückt, dass man seinen Nutzen maximiert, indem man den geringsten negativen Betrag anstrebt. Vorweggenommen: Rein intuitiv ist klar, welche Lösung gesamtwirtschaftlich (Spieler 1 und Spieler 2 stellen die Gesamtwirtschaft dar) angestrebt wird: Beide sollten die Strategie \textit{Leugnen} wählen. Gesamtwirtschaftlich tritt dann der größte Nutzen (=kleinster Schaden) ein und die Situation keines Spielers könnte verbessert werden \textit{ohne} die Situation des anderen zu verschlechtern. Definitionsgemäß ist ein Pareto-Optimum erreicht. Aber zu welcher Lösung kommt man mit spieltheoretischen Ansätzen?


\begin{tikzpicture}[element/.style={minimum width=2.85cm, minimum height=1.50cm}]
\matrix (m) [matrix of nodes,nodes={element},column sep=-\pgflinewidth, row sep=-\pgflinewidth,]{
	& Leugnen $S_{2,1}$  & Gestehen $S_{2,2}$  \\
	Leugnen $S_{1,1}$ & |[draw]|-2 / -2 & |[draw]|-10 / -1 \\
	Gestehen $S_{1,2}$ & |[draw]|-1 / -10 & |[draw]|-8 / -8 \\    };



\node[above=0.15cm] at ($(m-1-2)!0.5!(m-1-3)$){\textbf{Spieler 2}};
\node[rotate=90] at ($(m-2-1)!0.5!(m-3-1)+(-1.25,0)$){\textbf{Spieler 1}};
\end{tikzpicture}

Diese Form der Darstellung wird übrigens "`Normalform"' genannt. Die zweite übliche Darstellungsform nennt man "`extensive Form"'. Diese umfasst für den gesamten Spielverlauf alle notwendigen Informationen zu Entscheidungen und Auszahlungen und gleicht von der Darstellung her einem Entscheidungsbaum. Beide Formen wurden übrigens schon in der Geburtsstunde der Spieltheorie so verwendet \parencite{Selten2001}.


\section{Von Neumann Morgenstern}

Vereinzelte Ansätze, die Ideen der Spieltheorien vorwegnahmen gab es bereits im 19. Jahrhundert. Das bekannteste Beispiel ist wohl die Duopol-Theorie von \textcite{Cournot1836}. Weithin gilt aber die Veröffentlichung des fundamentalen Werks "`Theory of Games and Economic Behavior"' im Jahr 1944 durch Oskar Morgenstern und John von Neumann als Ursprung der Spieltheorie. \textcite{VonNeumann1928} behandelte bereits einen speziellen Ansatz der Theorie, im Buch von 1944 wurde dieser verbreitert und verallgemeinert. In der zweiten Auflage im Jahr 1947 wurde der Beweis für die Axiome der Erwartungsnutzentheorie erbracht. Diese haben wir bereits im Kapitel \ref{Erwartungsnutzen} kennen gelernt und sind eigentlich mehr Voraussetzung für die Spieltheorie als Teil derselben \parencite[S. 3]{Selten2001}. 

Als "`Vater der Spieltheorie"' gilt also John von Neumann. Ein Mathematik-Genie. Tätig in unzähligen Gebieten, neben Mathematiker und Ökonom gilt er als einer der Entwickler des modernen Computers und er entwickelte eine binäre Programmiersprache. Er arbeitete an der Entwicklung der Quantenmechanik und der Wasserstoffbombe mit. \textcite[S. 232]{Bernstein1996} zitiert, dass er "`während seiner Zeit beim Militär Admiräle gegenüber Generälen bevorzugte, da diese trinkfester waren"' und weitere Geschichten, die ihn als lebenslustiges Genie darstellen. Der Beitrag seines Koautors Oskar Morgenstern - der in Österreich das Institut für Höhere Studien mitbegründete - wird in der modernen Literatur häufig in Frage gestellt. In \textcite[S. 494]{Leonard1994} wird dargestellt, dass Morgenstern, erstens unter Ökonomen-Kollegen in den USA recht unbeliebt war und zweitens, einen\textit{inhaltlichen}-bescheidenen Beitrag zur hoch-mathematischen Spieltheorie von Neumann's beigetragen hat. Insbesondere im Artikel \textcite{Morgenstern1976} stellt der Autor seinen eigenen Beitrag zur Entstehung der "`Theory of Games and Economic Behavior"' gänzlich anders dar. Laut \textcite{Nash1994} ist es aber unzweifelhaft, dass dieses fundamentale Werk ohne die Zusammenarbeit von Oskar Morgenstern und John von Neumann in dieser Form nicht entstanden wäre.

HIER WEITER (aus Selten-Text) Vor allem Nullsummenspiele
Der Beitrag war speziell: Ein (nicht-kooperatives) zwei-Personen Nullsummenspiel und die Minimax-Lösung dafür.
Was war nun aber der Inhalt dieser frühen Spieltheorie?




\section{Nash}
Berührend ist die Geschichte von John Forbes Nash, die nicht zuletzt durch den Film "`A Beautiful Mind"' weit über wissenschaftliche Kreise hinaus bekannt wurde. Nash verfasste 1950 eine geniale und nur großzügige\parencite[S. 164]{Nash1994} 27 Seiten lange Dissertation \parencite{Nash1950}, die später als Journalbeitrag publiziert wurde \parencite{Nash1951}, und mit der er die Spieltheorie entscheidend weiterentwickelte. Ende der 1950er Jahre erkrankte er aber schwer an Schizophrenie und war die folgenden 25 Jahre stark eingeschränkt. Erst in seinen Fünfzigern erholte er sich. 1994 wurde ihm gemeinsam mit Reinhard Selten und John Harsanyi der Nobelpreis für Ökonomie zugesprochen. In einem interessanten Interview \parencite{Nash2004}\footnote{https://www.nobelprize.org/prizes/economic-sciences/1994/nash/interview/} im Rahmen des Ersten Nobelpreisträgertreffens im Jahr 2004 erzählt er, dass die Verleihung des Nobelpreise einen enormen Einfluss auf sein Leben hatte, war er doch zuvor schon lange Zeit arbeitslos, obwohl er schon längere Zeit bei guter Gesundheit war und er daher mehr oder weniger kaum noch am öffentlichen Leben teilnahm. Die Tragik in seinem Leben setzte sich übrigens bis zu seinem Tod fort: Im Jahr 2015 erhielt er den Abel-Preis, eine Auszeichnung für Mathematiker. Bei der Rückkehr aus Norwegen, wo der Preis verliehen wird, war das Taxi, das ihn vom Flughafen nach Hause bringen sollte in einen Autounfall verwickelt, der Nash und seiner Frau das Leben kostete. 

Nash erweiterte die Erweiterung in verschiedener Hinsicht. Am grundlegendsten ist wohl seine \textit{Unterscheidung} zwischen "`Kooperativer Spieltheorie"' und "`Nicht-kooperative Spieltheorie"'. In Letztgenannter geben Spieler kein Commitment zu einer bestimmten Strategie ab, während bei der "`Kooperativen Spieltheorie"' durchsetzbare Verträge vorhanden sein können.
Den größten nachhaltigen Beitrag stellt sicherlich seine Gleichgewichtslösung für nicht-kooperative Spiele dar \parencite{Nash1951, Nash1950b}. Studierende der Ökonomie kennen vor allem deshalb seinen Namen, selbst dann, wenn ihr Studium keine Vorlesung zur Spieltheorie enthält. Schließlich haben diese "`Nash-Gleichgewichte"' in der modernen Ökonomie einen Fixplatz in vielen Modellen eingenommen. Für "`Nicht-kooperative Spiele"' bewies Nash, dass es in solchen Situation immer zumindest ein Nash-Gleichgewicht gibt. Also ein Gleichgewicht, in dem kein Spieler seine Auszahlungen erhöhen kann, indem er einseitig seine Strategie verändert. HIER WEITER: Was ist genau ein Nash-Gleichgewicht!




Von Neumann war übrigens wenig beeindruckt von der Erweiterung "`seiner"' Spieltheorie durch Nash. Vielmehr sah er darin eine "`triviale Folge"' aus dem Fixpunkttheorem, das Brouwer einige Jahre zuvor bewiesen hatte. (Zitieren aus \textcite{Cassidy2015} )

Kollegen Kuhn und Tucker, die vor allem durch die nicht-lineare Programmierung "`Kuhn-Tucker-Bedingung"'.


GEfangenendilemma - Ursprung Tucker: \parencite[S. 161]{Nash1994} - auch das Beispiel von Alan Blinder (1982) mit den Zentralbanken vs. Staatsausgaben











\section{Hurwicz, Aumann, Harsanyi, Selten}

Harsanyi: Nach 1960: Einführung der Spiele mit inkompletter Information (davon abzugrenzen: Perfekte vs. Imperfekte Information). Harsanyi 1994. 

\section{Marktdesign und Auktionstheorie}

Marktdesign: Alvin Roth und Lloyd Shapley 
Auktionstheorie: \textit{Paul Milgrom} und \textit{Robert Wilson} (Seite 243 im Bernstein-Buch)



