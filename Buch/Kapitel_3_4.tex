%%%%%%%%%%%%%%%%%%%%% chapter.tex %%%%%%%%%%%%%%%%%%%%%%%%%%%%%%%%%
%
% sample chapter
%
% Use this file as a template for your own input.
%
%%%%%%%%%%%%%%%%%%%%%%%% Springer-Verlag %%%%%%%%%%%%%%%%%%%%%%%%%%

\chapter{Der Monetarismus: Money Matters!}
\label{Monetarismus}

Der Monetarismus ist eng verbunden mit dem Namen \textsc{Milton Friedman} und der "`Chicago School"'. Genau genommen ist Friedman der Hauptvertreter der sogenannten \textit{zweiten Generation} der Chicago School\footnote{Die Chicago School wurde ursprünglich schon von \textit{Frank Knight}, den wir im Kapitel \ref{FisherundKnight} kennenlernen, begründet. Sie umfasst weit mehr als den Monetarismus Friedman's und kommt in diesem Buch noch öfter vor. Häufig verwendet man den Begriff der Chicago School aber auch einfach als Synonym für Friedman's Lehren. So zum Beispiel in "`Chicago Liberalismus"' und "`Chicago Boys"'. Man kann den Begriff aber auch als Überbegriff von allen in den USA entstandenen wirtschaftsliberalen Schulen verwenden. Dann umfasst er neben den Arbeiten von Frank Knight und Jacob Viner auch die "`Neue Klassische Makroökonmie"' (Kapitel \ref{Neue Makro}) und die "`Neoklassische Finance"' (Kapitel \ref{Finance})}, seine Lehren werden aber häufig synonym etwas unscharf einfach als "`Chicago School"' bezeichnet. Neben Milton Friedman werden meist nur zwei weitere Ökonomen als Vertreter des Monetarismus\footnote{Monetarismus im engeren Sinn. Denn auch der Begriff des Monetarismus wird häufig unscharf und allgemein für verschiedene liberale Denkrichtungen verwendet} genannt. Vergessen wird häufig aber auf die Co-Autorin Friedman's: \textsc{Anna Jacobson Schwartz}. Sie ist neben Joan Robinson wohl die erste bedeutende Ökonom\textit{in}. Einer der beiden häufig genannten ist \textsc{Karl Brunner}, der vielleicht einzige bedeutende Schweizer Ökonom\footnote{Neben dem Ricardo-Kritiker Jean-Charles-Léonard Simonde de Sismondi}. Ihm wird die breite Etablierung des Begriffs Monetarismus zugeschrieben \parencite{Pierenkemper2012, Brunner1968}. Der zweite ist \textsc{Allan Meltzer}. Die beiden publizierten gemeinsam zum Monetarismus \parencite{Meltzer1971}, standen aber insgesamt stets im Schatten von Milton Friedman. Der Monetarismus im engeren Sinn war im Großen und Ganzen weitgehend seine "`One-Man-Show"'. 

Inhaltlich ist eine Gesamteinordnung des Monetarismus gar nicht so einfach. Man muss hier zwei Komponenten auseinander halten. Erstens, die wirtschaftswissenschaftliche Komponente. Diese findet sich primär in der "`Wiederentdeckung"' der Bedeutung der Geldpolitik, abgeleitet aus empirischen Untersuchungen. Damit verbunden ist auch die strikte - aber hier wissenschaftlich getriebene und auf konkrete Punkte abzielende - Ablehnung des Keynesianismus. Tatsächlich fehlt dem Monetarismus ein eigenes formal-theoretisches Modell lange Zeit. \textcite{Friedman1970}: "`A Theoretical Framework for Monetary Analysis"', soll das beheben, das Werk ist aber eher bedeutungslos geblieben. Friedman akzeptierte stattdessen das damals vorherrschende IS-LM-Modell als theoretischen Rahmen. Er argumentierte allerdings der Verlauf der Funktionen, insbesondere jener der LM-Kurve, sei nicht so wie in der neoklassischen Synthese dargestellt.
Die zweite Komponente, neben der wissenschaftlichen, ist die politische Komponente. Friedman's Monetarismus (wie übrigens die gesamte Chicago School) ist verbunden mit einer strikten Ablehnung von staatlichem Einfluss. Friedman war ein vehementer Kämpfer für "`persönliche Freiheit"'. Dies geht häufig über wirtschaftspolitische Themen hinaus. So setzte sich Friedman sein Leben lang für die Legalisierung sämtlicher Drogen ein. Viele seiner extrem liberalen Einstellungen sind auch gar nicht wissenschaftlich begründet, sondern wohl eher seiner persönlichen Überzeugung geschuldet. 

Friedman wurde 1912 in New York geboren und musste nach dem frühen Tod seines Vaters schon früh Geld verdienen. Dennoch konnte er bereits 1932 ein Bachelorstudium in Mathematik und Ökonomie abschließen. Danach war er in verschiedenen öffentlichen Institutionen tätig, bevor er 1946 sein Doktorratsstudium abschloss und im selben Jahr eine Professur in Chicago erhielt \parencite[S. 204]{Linss2017}. An der University of Chicago sollte er nicht nur die nächsten 30 Jahre bleiben, sondern diese auch nachhaltig prägen.

Friedman wird häufig als der größte Ökonom des 20. Jahrhunderts neben John Maynard Keynes bezeichnet. Aus wirtschaftswissenschaftlicher Sicht waren die 1960er Jahre geprägt von der Debatte zwischen Vertretern der Neoklassischen Synthese und Monetaristen. Sicherlich waren Friedman's Leistungen großartig. Er etablierte die "`Chicago School"' über Generationen als eine der führenden wirtschaftswissenschaftlichen Denkschulen weltweit. Sein Ruf als hervorragender Lehrer ist legendär. Viele spätere große Ökonomen berufen sich darauf, dass der Besuch in Milton Friedman's Vorlesung sie erst vom Fach Ökonomie überzeugt habe \parencite[S. 165]{Snowdon2005}. Ebenso enorm, und auch enorm umstritten, war sein politischer Einfluss. Er selbst hatte wohl entscheidenden Anteil an der liberalen US-Politik unter Ronald Reagan, dessen informeller politischer Berater er schon früh war. Die Chicago Boys, Chicago-School-Absolventen, die sich primär auf seine Lehren bezogen, nahmen - allerdings höchst umstritten - direkt Einfluss auf die diktatorische Regierung in Chile.

Aus ökonomischer Sicht war seine Arbeit, wie bereits angedeutet, vielfältig und in vielerlei Hinsicht bahnbrechend. Aber vom rein wissenschaftlichen Standpunkt, waren seine Arbeiten bei weitem nicht so revolutionär wie jene von Keynes. Zwar spricht man in der Literatur häufig von der "`Monetaristischen Gegenrevolution", es war aber schlussendlich nicht Friedman's Monetarismus, der den Keynesianismus als "`State of the Art"'-Lehre überwand. Das waren vielmehr seine Nachfolger in Chicago, die "`Neuen Klassiker"' (Vergleiche \ref{Neue Makro}). Allerdings war Friedman, neben den Vertretern der Österreichischen Schule, der erste bedeutende Ökonom, der die Theorien des Keynesianismus in Frage stellte und dessen Unzulänglichkeiten erkannte.


\section{Das Ende der harmonischen Synthese}

Die wissenschaftliche Ablehnung des Keynesianismus durch Friedman beginnt mit der von ihm entwickelten "`Permanenten Einkommenshypothese"'. Mit dieser zeigte \textcite{Friedman1957}, dass nicht das aktuelle Einkommen, wie bei Keynes, entscheidend für Konsum ist, sondern das langfristige Einkommen, bzw. die Erwartungen bezüglich des Lebenseinkommens. Im Resultat sehr ähnlich wie bei \textcite{Modigliani1954}, deren "`Lebenszyklushypothese"' wir gerade im Kapitel \ref{Synthese} kennen gelernt haben, hängt der aktuelle Konsum nicht vom aktuellen Einkommen ab, sondern wird langfristig geplant. Bei \textcite{Friedman1957} bilden die Individuen abhängig von Herkunft, Ausbildung und anderen Faktoren Erwartungen hinsichtlich ihres Lebenseinkommens und konsumieren dementsprechend langfristig recht konstant. Dennoch auftretende, kurzfristige Abweichungen des Einkommens von den Erwartungen werden durch (zusätzliche) Kreditaufnahme, bzw. (zusätzliches) Sparen ausgeglichen. Friedman leitet aus dieser Annahme ab, dass der Konsum langfristig recht stabil ist. Das impliziert aber auch, dass  die Nachfrage nach Geld dementsprechend stabil ist. Für Friedman ist das ein Widerspruch zur Annahme der Anhänger von Keynes, die in den Schwankungen der aggregierten Nachfrage die Hauptgründe für Konjunkturschwankungen sehen. Darauf basiert schließlich auch die Annahme der Vertreter der neoklassischen Synthese, dass Fiskalpolitik die Nachfragelücken ausgleichen muss die Konjunktur zu glätten. Für Friedman ist Fiskalpolitik daher weitgehend unwirksam. Die "`Permanente Einkommenshypothese"' etablierte sich rasch als "`State of the Art"', wurde noch Jahrzehnte später empirisch überprüft und gilt bis heute als immer noch aktuell \parencite{Bernanke1984, Mankiw1985}.  

Friedmans's wissenschaftliches Steckenpferd war aber zweifelsohne die Wiederentdeckung der Bedeutung der Geldpolitik und hier vor allem die erneute Etablierung der Quantitätstheorie des Geldes. Der \textit{Wert} des Geldes ist hierbei rein von der \textit{Menge} des Geldes abhängig. Es gilt $ M * v = BIP * P$. $M$ bezeichnet hierbei die Geldmenge. $v$ ist die Umlaufgeschwindigkeit des Geldes. Auf der anderen Seite ist das $BIP$ die Wirtschaftsleistung des Landes und $P$ das Preisniveau. Diese Quantitätsgleichung des Geldes war schon David Hume, dem Freund und Zeitgenossen von Adam Smith, bekannt, und ihre moderne Form wurde von Irving Fisher \parencite{Fisher1911} Anfang des 20. Jahrhunderts beschrieben. Mit Keynes aber verlor sie allerdings an Bedeutung. Die Quantitätstheorie des Geldes wurde von \textcite{Keynes1936} zwar nicht grundsätzlich abgelehnt, aber als weitgehend unbedeutend angesehen. Damit stand Friedman auch in diesem Zusammenhang von Anfang an im Widerspruch zu den damals etablierten keynesianischen Ansichten. In Krisensituationen, kommt es bei Keynes zur Investitionsfalle, oder zur Liquiditätsfalle. Wie wir in Kapitel \ref{Synthese} gesehen haben, sind gerade dann geldpolitische Maßnahmen weitgehend wirkungslos, weil eine Änderung der Umlaufgeschwindigkeit des Geldes etwaige geldpolitische Maßnahmen in Form der Geldmengenausweitung, zunichte machen. Genau dies lehnte Friedman aber ab. Für ihn war, als Folge der "`Permanenten Einkommenshypothese"' die Nachfrage nach Geld, und damit die Umlaufgeschwindigkeit des Geldes aus der Quantitätsgleichung, als weitgehend stabil anzusehen. Wäre dies der Fall ist klar, dass die Quantitätsgleichung und damit vor allem die Geldmenge eine zentrale Rolle spielen. Aus der Formel sieht man, dass - wenn man die Umlaufgeschwindigkeit $v$ vernachlässigt - die Geldmenge in direktem Zusammenhang mit dem BIP, oder dem Preisniveau, oder beidem steht. Daraus können verschiedene Implikationen gezogen werden. 
Erstens, für Friedman war klar, dass nicht zu geringe Nachfrage, sondern eine zu geringe Geldmenge das Hauptproblem in Wirtschaftskrisen ist. Eine zu geringe Geldmenge führt nämlich zu Deflation und all ihren bekannten (vgl. Kapitel \ref{Keynes}) negativen Auswirkungen.
Zweitens, während Wirtschaftskrisen ist Geldpolitik wichtig und wirksamer als Fiskalpolitik. Dies leitet sich direkt daraus ab, dass eine steigende Geldmenge in Wirtschaftskrisen Deflation und somit die Krise selbst bekämpft.
Die wissenschaftlichen Arbeiten dazu, und zwar wiederum primär in Form empirischer Arbeiten, lieferte er just in der Hoch-Zeit des Keynesianismus. Als sein Hauptwerk kann diesbezüglich das fast schon monumentale Wert \textit{A Monetary History of the United States, 1867–1960} gesehen werden. \textcite{Friedman1963} ist ein wirtschaftsgeschichtliches Werk, in dem er gemeinsam mit Anna Jacobson Schwartz beachtliches Datenmaterial zusammentrug und schließlich zeigte, dass die "`Great Depression"' der 1930er Jahre erst durch einen Mangel an Geld zur größten Wirtschaftskrise aller Zeiten wurde. Er machte dafür zum Teil den Goldstandard, vor allem aber die Unfähigkeit der Federal Reserve verantwortlich. Seiner Meinung nach hätte die Fed die Schwere der Krise wesentlich abschwächen können, wenn sie die Geldmenge rechtzeitig ausgeweitet hätte. Ein Punkt übrigens, in dem die Monetaristen eben gar nicht so erz-liberal waren wie sie sich oft gern selbst sahen. Friedman meinte zeitlebens, dass sobald die Krise einmal eingetreten sei, der Staat in Form von Geldpolitik sehr wohl eingreifen sollte. Die Österreichische Schule in Form von Friedrich Hayek hingegen kritisierte jeglichen Eingriff in den freien Markt auch in Krisenzeiten. Friedman kritisierte Hayek noch 1999 vehement für diese Ansicht: "`I think the Austrian business-cycle theory has done the world a great deal of harm [in the 1930s]"' \parencite{Epstein1999}.

In weiterer Folge, also ab der Mitte der 1960er Jahre, kam es zu heftigen akademischen Auseinandersetzungen zwischen Friedman und den Vertretern der Neoklassischen Synthese. Friedman hatte bereits davon den Keynesianismus noch an einer weiteren Front frontal angegriffen. Er argumentierte \parencite{Friedman1961}, dass aktive Wirtschaftspolitik \textit{allgemein} keine Konjunktur-stabilisierende Wirkung habe, weil der Effekt geldpolitische Maßnahmen zwar recht kurzfristig durchgeführt werden könnten (kurzer inside-lag), aber erst mit einer erheblichen Verzögerung wirken würden (langer outside-lag). Neben der Fiskalpolitik, sei also auch Geldpolitik abzulehnen. 
Wie schon angedeutet, hatten die Monetaristen allerdings keine neuartige Theorie, bzw., zumindest kein eigenes monetaristisches Modell. Stattdessen wurde praktisch das IS-LM-Modell der Neoklassischen Synthese herangezogen und argumentiert, die LM-Kurve darin verliefe stets vertikal. Das heißt die Geldnachfrage ist stabil und unabhängig vom Zinssatz. Durch Ausweitung und Reduktion der Geldmenge könnte das BIP theoretisch beeinflusst werden. Fiskalpolitik, also eine Verschiebung der IS-Kurve, bleibt hingegen wirkungslos. Die Monetaristen stützen sich für ihre Argumentation lange alleine auf die empirisch zunächst recht gut abgesicherte, stabile Geldnachfrage.  \textcite{Tobin1970} stellte den Zusammenhang zwischen Einkommen und Geldnachfrage, wie von Friedman stets als Argument für die Unwirksamkeit von Fiskalpolitik herangezogen, in Frage und verwies darauf, dass empirische Beobachtungen nur eine Korrelation zeigen können, jedoch keine Kausalität. Der Monetarismus wurde zunehmend für seine fehlende theoretische Unterlegung kritisiert. \textcite{Friedman1970} und \textcite{Friedman1971a} lieferte schließlich eine "`Monetaristische Theorie"', die allerdings wiederum von \textcite{Tobin1972} recht forsch kritisiert wurden. Diese Diskussion wurde schließlich von der wahren ökonomischen Revolution der 1970er Jahre, nämlich jener, die durch die "`Lucas-Kritik"' begründet wurde, beendet. Dazu aber später in Kapitel \ref{Neue Makro}.

An einer weiteren Front eröffnete Milton Friedman einen weiteren Angriff auf die damalige Mainstream-Ökonomie. Die Vertreter der Neoklassischen Synthese hatten den Phillipskurven-Zusammenhang aus Arbeitslosigkeit und Inflation (vgl. Kapitel \ref{Synthese}) in ihre Theorie aufgenommen. In einem extrem einflussreichen Artikel zeigte \textcite{Friedman1968} die Unzulänglichkeit der Phillips-Kurve auf. "`The Role of Monetary Policy"' wurde später von mehreren Autoren, wie James Tobin und Paul Krugman,  als der "`wahrscheinlich einflussreichste Zeitschriftenartikel aller Zeiten"' \parencite[S. 160]{Snowdon2005} bezeichnet. Aus heutiger Sicht ist dies einigermaßen überraschend, enthält doch der Artikel keinerlei formal-mathematischer Herleitungen. Edmund \textcite{Phelps1967} hatte fast zeitgleich inhaltlich die gleichen Schlüsse wie Friedman gezogen (vgl. Kapitel \ref{micmac}) und zwar formal-wissenschaftlich begründet. Worum ging es inhaltlich in dem Artikel? Richtigerweise kritisierte \textcite{Friedman1968}, dass es bei Lohnverhandlungen zwischen Arbeitnehmern und Arbeitgebern stets darum gehe, wie hoch die realen Löhne in Zukunft sein werden. Das heißt, beide Parteien ziehen bei ihren Verhandlungen nicht die realisierten Inflationsraten heran, sondern die für die Zukunft erwarteten Inflationsraten. Diese wiederum leiten sich direkt von den Inflationsraten der Vergangenheit ab. Das heißt, wenn die Inflation im letzten Jahr 4\% betrug, so gehen die Verhandlungspartner davon aus, dass die Inflation auch im nächsten Jahr 4\% beträgt. Möchte die Regierung in weiterer Folge die Arbeitslosigkeit weiter senken, muss sie dafür sorgen, dass die Inflation höher als 4\% ist. Alleine aus diesen Erwartungen ergibt sich ein scheinbarer Zusammenhang zwischen Inflation und Arbeitslosigkeit. Friedman geht also davon aus, dass der Phillipskurven-Zusammenhang kein tatsächlicher Zusammenhang ist, sondern alleine durch die Inflationserwartungen zustande kommt. Das Konzept wurde als die "`Expectation Augmented Phillips Curve"' bekannt. Der ständige Versuch die Arbeitslosigkeit zu senken, würde demnach nur zu immer weiter steigender Inflation führen. Wirklich steuern könne man die Arbeitslosigkeit über Inflation aber nur kurzfristig. Langfristig hingegen müsse es ein Arbeitslosenquote geben, bei der die Inflation stabil bleibt. Wenn die Löhne im gleichen Ausmaß wie die Preise steigen, bleiben die Real-Löhne stabil und es kommt zu einem stabilen Gleichgewicht zwischen Arbeitsangebot und Arbeitsnachfrage. Die Arbeitslosenquote bei der dies der Fall ist, nennt Friedman die "`Natürliche Arbeitslosenquote"'. Ein Begriff der seither fix in der Volkswirtschaftslehre etabliert ist, wenn auch in leicht abgewandelter Formen (vgl. Kapitel \ref{cha: Neu Keynes}). Friedman etablierte damit mehrere wichtige Konzepte. Erstens, er entlarvte das Ziel der Vertreter der Neoklassischen Synthese Vollbeschäftigung zu erreichen als unrealistisch. Zweitens, er zeigte, dass die Phillips-Kurve keinen stabilen Zusammenhang abbilden kann, stattdessen verläuft die Phillips-Kurve langfristig vertikal: Wird die zukünftige Inflationsrate ständig richtig an antizipiert, gibt es keine Überschussnachfrage, bzw. kein Überschussangebot am Arbeitsmarkt und die Arbeitslosenquote pendelt sich bei ihrem natürlich Wert ein. Drittens, leitet Friedman daraus direkt den Sinn von Geldpolitik ab: Langfristig gilt die Quantitätsgleichung des Geldes und eine Geldmengenänderung hat keine realen Effekte. Eine Ausweitung führt nur zu Inflation. Kurzfristig könnte die Arbeitslosigkeit durchaus durch Geldmengenausweitung verringert werden. Als Konsequenz leidet Friedman ab, dass eine langfristig stabile Geldmengenausweitung um einen bestimmten, vorher festgelegten und veröffentlichten Prozentsatz sinnvoll. Friedman ging ja von konstanter Geld-Umlaufgeschwindigkeit aus. Wenn die Geldmenge konstant erhöht wird, können die Arbeitsmärkte die Inflation gut antizipieren, und die Konjunktur wird sich um ihre langfristige Wachstumsrate stabilisieren. 


\section{Die kurze Zeit des Monetarismus als führende Schule}

Die zeitlichen Geschehnisse spielten in weiterer Folge Friedman natürlich in die Hände. Seine Krisenerklärung durch die Quantitätstheorie stellt Geldpolitik, und die damit eng verbundene Inflation, in den Mittelpunkt der Forschung. Und genau die Inflation wurde in den 1970er Jahren in den USA zum zentralen wirtschaftspolitischen Problem. Als in den 1970er Jahren Stagflation auftrat, konnte Friedman quasi triumphierend seine Erklärung für Inflation mittels Quantitätstheorie aus der Tasche ziehen, während die führenden Vertreter der Neoklassischen Synthese noch erfolglos nach Erklärungen suchten. Seine Aussage: "`Inflation is always and everywhere a monetary phenomenon"', wurde berühmt. Ebenso sein Helikopter-Geld-Beispiel \parencite{Friedman1969} (sinngemäß): Angenommen die Geldmenge würde sich wundersam von einem Moment auf den anderen verdoppelt - Was würde mit den Preisen geschehen? Jeder Erstsemester-Student antwortet intuitiv: "`Die würden sich auch verdoppeln!"', wird gerade jetzt wieder häufig zitiert\footnote{In vielen Fällen aber grundlegend falsch. So wird diese rein \textit{geld}politische Maßnahme häufig mit \textit{fiskal}politischen Maßnahmen vermischt. Außerdem deutete Friedman damit die Neutralität des Geldes an und eben nicht die Wirksamkeit geldpolitischer Maßnahmen}. Seine Erklärung der Inflation und die wirtschaftspolitischen Empfehlungen waren einfach: Das Geldmengenwachstum müsse eingeschränkt werden!   
Die erste Zentralbank, die sich dann an die praktische Umsetzung seiner Ideen machte, war übrigens die Deutsche Bundesbank, die ab 1975 eine reine Geldmengenstrategie verfolgte \parencite[S. 36]{BBK2016}. 
In den USA war das Inflationsproblem allerdings wesentlich ausgeprägter. Ende der 1970er Jahre wurden die Inflationsraten zweistellig. Die praktische Umsetzung des Monetarismus in den USA erfolgte durch \textsc{Paul Volcker}, der 1979 von Präsident Jimmy Carter zum Fed-Vorsitzenden gemacht wurde. Er wendete den Monetarismus im Sinne einer Geldmengensteuerung an. Das heißt die Zentralbank erhöhte nicht nur den Leitzinssatz um die Inflation einzudämmen, sondern verhinderte aktiv ein zu starkes Ansteigen der Geldmenge. Das ganze wurde als "`Monetaristisches Experiment"' bezeichnet und hatte als einziges Ziel eine "`Disinflation"', also ein Sinken der Inflation, herbeizuführen. Das klingt eigentlich recht einfach. Man darf aber nicht vergessen, dass die Inflationserwartungen aufgrund der zuletzt ständig hohen Inflationsraten sehr hoch waren und beim Abschluss von Verträgen eine große Rolle spielten. Schwieriger als die Inflation kurzfristig nach unten zu drücken war es die Inflationserwartungen langfristig zu senken. Die Monetaristen argumentierten, wenn erst einmal die Inflationserwartungen auf einem vernünftigen Niveau seien, würde sich rasch ein Gleichgewichtswachstum einstellen. Man kann bis heute darüber streiten ob dieses monetaristische Experiment erfolgreich war. Faktum ist, dass Milton Friedman die Maßnahmen nicht als Monetarismus in seinem Sinne akzeptierte "`Seine Ideen werden hier nicht korrekt umgesetzt"', rechtfertigte er sich, weil sie nicht vollständig seinen Vorstellungen entsprachen\parencite{Kremer2018}. Faktum ist auch, dass die Inflationsraten tatsächlich schnell und dauerhaft sanken, diesbezüglich war das Experiment auf jeden Fall erfolgreich. Bereits 1982 war die Inflation in den USA auf ca. 4\% gesunken. Die Nebeneffekte, nämlich zweistellige Arbeitslosenraten und eine mehrjährige Rezession sorgen bis heute für Diskussionen, ob die Maßnahmen als positiv oder überwiegend negativ zu bewerten waren. Eine "`schmerzlose"' Anit-Inflationsstrategie war der tatsächlich angewendete Monetarismus jedenfalls nicht, der Beweis, dass Geldpolitik wirksam sein kann, war aber erbracht \parencite[S. 703]{Samuelson1998}.

Insgesamt zusammengefasst bleibt die Geldtheorie der Monetaristen damit aber auch etwas verwirrend: Für Friedman stand die Geldpolitik im Mittelpunkt, als einzig wirksames wirtschaftspolitisches Instrument  - also gilt: MONEY MATTERS! Aber er plädierte dafür dieses Instrument nicht aktiv einzusetzen, weil es, aktiv eingesetzt in der Praxis doch nur zu Inflation führe. Als Beispiel nannte er das eben angeführte Helikopter-Geld, das eben \textit{nicht} zu BIP-Wachstum, sondern ausschließlich zu Inflation führen sollte - also eigentlich gilt MONEY DOESN'T MATTER, oder? Lösung: Es kommt bei Friedman immer auf den Zusammenhang an: Während Krisen (und bei Missmanagement eben auch als Auslöser von Krisen) ist Geldpolitik sinnvoll und wichtig. Außerhalb von Krisen sollten Politiker aber die Finger von Geldpolitik lassen, weil sie im Hinblick auf ihre kurzfristigen Ziele verlockende aber falsche Maßnahmen setzen und nur Inflation auslösen würden. 

Auf die Praxis angewendet war dem Monetarismus nur kurzer Erfolg vergönnt. Aktive Geldmengensteuerung kann nämlich nur dann erfolgreich sein, wenn die Umlaufgeschwindigkeit des Geldes als "`weitgehend konstant"' vernachlässigt werden kann. Aber schon Anfang der 1980er Jahre begann diese Umlaufgeschwindigkeit - im Gegensatz zu den Jahrzehnten davor - zu schwanken \parencite[S. 703]{Samuelson1998}. In wissenschaftlichen wie auch wirtschaftspolitischen Belangen wurde der Monetarismus bereits Anfang der 1980er Jahre von den "`Neuen Klassikern"' überholt. Viele Ökonomen, wie zum Beispiel Robert Solow, sind der Ansicht, die "`Neue Klassik"' sei nur eine Erweiterung des Monetarismus\parencite[S.342]{Warsh}. Aber die neuen Methoden\footnote{Mikrofundierung, dynamische Modelle, rationale Erwartungen (vergleiche Kapitel \ref{Neue Makro})} der "`Neuen Klassik"' unterscheiden sich doch stark vom "`Monetarismus"' und waren die \textit{wirkliche} Revolution, die die Keynesianer schließlich dazu "`zwangen"' ihre Ideen anzupassen und zu "`Neu Keynesianern"' zu werden. In der Retrospektive meint etwa \textcite[S. 697]{Samuelson1998} - immerhin das erfolgreichste VWL-Lehrbuch -, dass der Monetarismus dem Keynesianismus "`geistig nahe stehe"' und die beiden Schulen "`weitgehend einträchtig"' \parencite[S. 702]{Samuelson1998} seien. Ein "`radikal neuer Ansatz"'\parencite[S. 704]{Samuelson1998} sei hingegen die "`Neue Klassische Makroökonomie"'. Die Vergangenheits-orientierten "`adaptiven Erwartungen"' im Hinblick auf die Inflationserwartungen wurden durch das Konzept der "`rationalen Erwartungen"' abgelöst. Geldpolitik und die optimale Organisation von Zentralbanken wurde von den "`Neuen Klassikern"' ebenso völlig neu gedacht. 

Milton Friedman war in den 1970er Jahren auf dem Höhepunkt seiner wissenschaftlichen Karriere angelangt. 1976 erhielt er den Nobelpreis für Wirtschaftswissenschaften. Schon zuvor und auch in weiterer Folge sollte Milton Friedman als Polit-Berater des späteren US-Präsidenten Ronald Reagan, aber auch mit einer eigenen Fernseh-Show, die später ein Buch-Bestseller wurde \parencite{Friedman1980}, eine enorme öffentliche Wahrnehmung erfahren. Diesbezüglich war sein Einfluss für einen Ökonomen wirklich außergewöhnlich. Möglicherweise wird dadurch seine wissenschaftliche Leistung aber auch etwas überschätzt.
Seine wissenschaftliche Arbeit war entgegen dem späteren Trend, wenig mathematisch-formal - obwohl Friedman in frühen Jahren auch zu statistischen Themen publizierte.  Die Arbeiten der "`Neuen Klassiker"', hoben die ökonomische Methodik Mitte der 1970er Jahre auf eine höhere Stufe. Aber auch seine keynesianisch geprägten Zeitgenossen wie zum Beispiel Edmund Phelps oder James Tobin, brachten aus formal-analytischer Sicht höherwertige Wissenschaft hervor. Seine empirischen Analysen - hauptsächlich seine "`Monetary History of the US"' waren hingegen bahnbrechend.
Friedman's Stärke war, dass er die richtigen Themen zur richtigen Zeit wählte und auf seinen Aussagen beharrte. Aussagen, die zu seiner Zeit dem keynesianischen Zeitgeist radikal widersprachen. Friedman war in jeder Hinsicht radikal liberal: So hatte er einen nicht zu unterschätzenden Anteil an der Abschaffung der Wehrpflicht in den USA \textcite{Appelbaum2019}, er war durchgehend ein Befürworter frei schwankender Wechselkurse. Heute sind diese längst Realität, aber nach dem Zweiten Weltkrieg waren diese unvorstellbar. Und als der Keynesianismus Ende der 1940er Jahre seinen Siegeszug in den USA startete, etablierte er sich früh als Gegenspieler und verband sich stattdessen - wenn auch nur vorübergehend mit den europäischen Liberalen\footnote{Friedman war ein der Gründungsmitglieder der ultra-liberalen Mont-Pelerin-Gesellschaft und von 1970 bis 1972 deren Präsident. Aus wirtschaftswissenschaftlicher Sicht emanzipierte er sich aber bald weg von den Ideen der "`Freiburger Schule"' und auch der "`Österreichischen Schule"' (vgl. Kapitel \ref{Neoliberalismus})}. Noch im hohen Alter protestierte er auf den Straßen für die Liberalisierung von Drogen, aber auch für die Entkriminalisierung des Schwangerschaftsabbruchs oder die Rechte von Homosexuellen. Seine liberalen Ansichten legte er auch bereits 1962 in "`Capitalism and Freedom"' \parencite{Friedman1962} und vor allem später in "`Free to Choose"' \parencite{Friedman1980} der breiten Bevölkerung außerhalb der wirtschaftswissenschaftlichen Community dar. Sein politischer Einfluss ist umstritten. Der Einfluss auf Ronald Reagan, der die USA wirtschaftspolitisch radikal revolutionierte, ist wohl belegt. Hier spielten aber auch andere liberale Ökonomen eine Rolle. \textcite[S. 77]{Warsh} schreibt, dass Friedman und seine Anhänger "`Chile, den Rest von Lateinamerika und Ost-Europa Richtung "`freie Märkte"' führte. Ob sein Einfluss wirklich so groß war wird man wohl nie realistisch abschätzen können.


Im Gegensatz zu den Keynesianern war er der Überzeugung, dass sich die Wirtschaft ohne jegliche Eingriffe auf einem stabilen Wachstumskurs einfinden würde und dass Preise und Löhne grundsätzlich eher flexibel sind und nicht so rigide, wie im Keynesianismus angenommen \footnote{Diese Auffassung blieb übrigens für immer Teil der Ideologie der Chicago School}. Feinabstimmungen mittels Fiskal- und auch Geldpolitik wären daher eher nutzlos bis kontraproduktiv. Theoretisch wird dies damit unterlegt, dass der Monetarismus auch von einer sehr elastischen Anpassung der Zinssätze ausgeht. Das heißt die keynesianische Fiskalpolitik führt zu höheren Zinssätzen und damit auch zu Rückgängen in der Investitionstätigkeit am privaten Sektor. Die staatlichen Investitionen verdrängen also die privaten Investitionen ("`Crowding-Out"'). Somit kommt es nicht zu den erwünschten Wachstumseffekten. Eine wirklich tiefgreifende formal-theoretische oder empirische Fundierung dieser Empfehlungen blieb Friedman selbst aber weitgehend schuldig.
Dennoch sollte konsequenterweise eine langfristig erfolgreiche Wirtschaftspolitik seiner Meinung nach daher einfach darin bestehen, die Geldmenge mit einer konstanten Rate wachsen zu lassen \parencite{Friedman1960}. Zum Beispiel könnte man die Geldmenge jedes Jahr um 4\% wachsen lassen. Wenn das langfristige Wachstum 2\% beträgt, wird sich das BIP-Wachstum bei diesem einpendeln und gleichzeitig - entsprechend der Quantitätsgleichung - die Inflation konstant 2\% betragen. Für stabiles Geldmengenwachstum brauche es laut Friedman keine Zentralbank. Die Geldmenge konstant wachsen zu lassen schaffe schließlich auch ein Computer, der die Zentralbanken ersetzen könne, meinte er. Bei seiner Nobel-Preis-Bankett-Rede \parencite{Friedman1976b} scherzte er darüber und meinte sinngemäß "`Zum Glück haben sich die Entscheidungsträger nicht an den Vorschlag gehalten, ansonsten wäre ich um diesen Preis gekommen"'\footnote{Der sogenannte Wirtschaftsnobelpreis wird von der Schwedischen Zentralbank vergeben}


Er selbst war ein sehr gnadenloser Diskussionspartner und herausragender Lehrender, der, wie sonst fast niemand, Generationen von Studierenden in seinen Bann zog. 
Auch wenn sein Monetarismus als solcher im Endeffekt als gescheitert gilt, so kann man die heute unumstrittenen Bedeutung der Geldpolitik im wirtschaftspolitischen Instrumentarium doch noch immer zu einen guten Teil auf Milton Friedman zurückführen. Als Milton Friedman 2006 starb verpasste er nur um wenige Jahre die erneute, gnadenlose Anwendung seiner Ideen: Das in den führenden Industriestaaten während der "`Great Recession"' nach 2008 durchgeführte "`Quantitative Easing"'  kann auf Friedman's Lehren aus seinen Arbeiten zur "`Great Depression"' zurückgeführt werden. Zu den Feierlichkeiten im Rahmen von Milton Friedman's 90 Geburtstag im Jahr 2002 sagte der spätere Fed-Vorsitzende Ben Bernanke zu Milton Friedman and Anna J. Schwartz: "`Regarding the Great Depression, you’re right. We [the Fed] did it [fail to provide a stable monetary background]. We’re very sorry. But, thanks to you, we won't do it again."' Zu diesem Zeitpunkt wusste Bernanke noch nicht, dass er nur wenige Jahre später tatsächlich, als Fed-Vorsitzender, die USA mithilfe der Lehren von Milton Friedman und Anna J. Schwartz relativ glimpflich durch die größte Wirtschaftskrise seit der "`Great Depression"' führen sollte.



