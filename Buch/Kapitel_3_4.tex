%%%%%%%%%%%%%%%%%%%%% chapter.tex %%%%%%%%%%%%%%%%%%%%%%%%%%%%%%%%%
%
% sample chapter
%
% Use this file as a template for your own input.
%
%%%%%%%%%%%%%%%%%%%%%%%% Springer-Verlag %%%%%%%%%%%%%%%%%%%%%%%%%%

\chapter{Der Monetarismus: Money Matters!}
\label{Monetarismus}

Der Monetarismus ist eng verbunden mit dem Namen \textit{Milton Friedman} und der "`Chicago School"'. Eigentlich ist Friedman der Hauptvertreter der sogenannten zweiten Generation der Chicago School\footnote{Die Chicago School wurde ursprünglich schon von \textit{Frank Knight}, den wir im Kapitel \ref{FisherundKnight} kennenlernen, begründet. Sie umfasst weit mehr als den Monetarismus Friedman's und kommt in diesem Buch noch öfter vor. Häufig verwendet man den Begriff der Chicago School aber als Synonym für Friedman's Lehren. So zum Beispiel in "`Chicago Liberalismus"' und "`Chicago Boys"'. Man kann den Begriff aber auch als Überbegriff aber alle in den USA entstandenen wirtschaftsliberalen Schulen verwenden. Dann umfasst er neben den Arbeiten von Frank Knight und Jacob Viner auch die "`Neue Klassische Makroökonmie"' (Kapitel \ref{Neue Makro}) und die "`Neoklassische Finance"' (Kapitel \ref{Finance})}, seine Lehren werden aber häufig Synonym etwas unscharf einfach als "`Chicago School"' bezeichnet. 

Inhaltlich ist eine Gesamteinordnung von Friedman gar nicht so einfach. Meiner Meinung nach muss man hier zwei Komponenten auseinander halten. Erstens die wirtschaftswissenschaftliche Komponente. Diese findet sich primär natürlich in der "`Wiederentdeckung"' der Geldpolitik. Damit verbunden ist auch die strikte - aber hier wissenschaftlich getriebene und auf konkrete Punkte abzielende - Ablehnung des Keynesianismus. So zeigte Friedman zum Beispiel, dass die keynesianische Konsumtheorie zu kurz greift. Auch die Unzulänglichkeit der Phillipskurve wurde zunächst von Friedman (und Edmund Phelps, siehe Kapitel \ref{micmac}) in Frage gestellt.
Die zweite Komponente nenne ich die politische Komponente. Friedman's Monetarismus (wie übrigens die gesamte Chicago School) ist verbunden mit einer strikten Ablehnung jeglichen staatlichen Einflusses. Dies kann nur zum Teil aus seiner wissenschaftlichen Arbeit abgeleitet werden. So versucht Friedman zu zeigen, dass Fiskalpolitik weitgehend unwirksam sei, daraus kann man ablehnen, dass aktive Wirtschaftspolitik und somit staatliche Einmischung abzulehnen ist. Aber viele seiner extrem liberalen Einstellungen sind nicht wissenschaftlich begründet, sondern wohl eher seine persönliche Einstellung.
Der kleine Zwiespalt zwischen wissenschaftlicher und politischer Überzeugung zeigt sich in Nuancen in seiner Arbeit. So lehnt die rigoros-liberale Österreichischen Schule jeglichen Eingriff in die Ökonmie ab. Sogar während der "`Great Depression"' hätte man besten nicht intervenieren sollen, lautete lange die



Friedman wird häufig als der größte Ökonom des 20. Jahrhunderts neben Keynes bezeichnet. Sicherlich waren seine Leistungen großartig. Er etablierte die "`Chicago School"' als eine der führenden Wirtschaftshochschulen weltweit. Sein Ruf als hervorragender Lehrer ist unumstritten. Viele große Ökonomen berufen sich darauf, dass der Besuch in Milton Friedman's Vorlesung sie erst vom Fach Ökonomie überzeugt habe. Ebenso enorm, und auch enorm umstritten, war sein politischer Einfluss. Er selbst hatte wohl entscheidenden Anteil an der liberalen US-Politik unter Ronald Reagan dessen informeller politischer Berater er schon früh war. Die Chicago Boys, die sich primär auf seine Lehren bezogen nahmen aber auch direkt Einfluss auf die diktatorische Regierung in Chile.

Aus ökonomischer Sicht war seine Arbeit auch extrem vielfältig und in vielerlei Hinsicht bahnbrechend. Aber vom rein wissenschaftlichen Standpunkt her waren seine Arbeiten bei weitem nicht so revolutionär wie jene von Keynes. Sein Monetarismus war es auch nicht der den Keynesianismus als "`State of the Art"'-Lehre überwand. Das waren meines Erachtens vielmehr seine Nachfolger in Chicago, die "`Neuen Klassiker"' (Vergleiche \ref{Neue Makro}). Auch auf die Praxis angewendet war dem Monetarismus nicht wirklich von Erfolg geprägt(ZITAT). In beiden Belangen, rein wissenschaftlich als auch in der Praxis angewendet, wurde der Monetarismus bereits Anfang der 1980er Jahre von den "`Neuen Klassikern"' überholt. Viele Ökonomen sind der Ansicht, die "`Neue Klassik"' sei nur eine Erweiterung des Monetarismus(Solow in Warsh?ZITAT). Aber die neuen Methoden\footnote{Mikrofundierung, dynamische Modelle, rationale Erwartungen (vergleiche Kapitel \ref{Neue Makro})} waren die wirkliche Revolution, die die Keynesianer schließlich dazu "`zwangen"' ihre Ideen anzupassen und zu "`Neu Keynesianern"' zu werden.

Friedman Biographie

Friedman liberal, aber zunächst Wissenschaftler. Durchbruch mit Arbeit darüber, dass Great Depression durch zuwenig Geld verursacht wurde. Höhepunkt als Wissenschaftler hatte kurze Zeit vor Lucas-Kritik und während Stagflation. Danach vor allem als Politberater.




Betrachten wir seine wissenschaftlichen Ansätze im Detail.

Wiederentdeckung der Quantitätstheorie des Geldes in einer Zeit als diese keine Rolle spielte. (Bei Keynes nur gültig bei Vollauslastung der Wirtschaft). Durchbruch wissenschaftlich mit ERklärung der Great Depression (mit Anna J. Schwartz). und Kritik an der Konsumfunktion (gleichzeitig mit Modigliani?). Später Kritik an der Phillipskurve (mit Phelps).
Damit hatte er gutes Gespür. Später durch die Stagflation war sein Thema "`Geld"' plötzlich das entscheidende Thema, als Keynesianer nicht mehr weiter wussten.

Extrem früh der Bedeutung der Geldpolitik in einer Zeit in der diese keine Rolle spielte. Politisch Kennedy und Johnson sehr beliebt. Wirtschaft klar Keynes bis inklusive Nixon (den nannte er Sozialist in einem Interview!)



Gedankenexperiment: Helikoptergeld (Quantitätsgleichung: Nur Inflation). Aber Unterschied kurze Frist und lange Frist.
Danach: adaptive Erwartungen: Leute erwarten Fortschreibung der derzeitigen Inflation. Daher wirksam nur Änderungen der Inflationsrate. Erweitert durch die rationalen Erwartungen, die das noch ausweiten
Zunächst: Steuerung der Wirtschaft über die Geldmenge. Später: Konstantes Wachstum, keine aktive Politik

Deutsche Bundesbank die erste, die die Theorie anwendete.

Geldmengensteuerung wenig erfolgreich, weil plötzlich die Umlaufgeschwindigkeit schwankte.

Deflation und Inflation daher für ihn rein monetäre Erscheinungen.


Wissenschaft recht formal (Publizierte auch zu Statistik), aber weit weniger als später die "`Neuen Klassiker"', oder auch z.B.: Phelps (und Tobin?). Seine Stärke war eher das auf den Punkt bringen.

Legendäre Auseinandersetzung mit Keynesianern (Monetaristische Gegenrevolution), obwohl Friedman und Keynes nicht einer Generation angehörten. Lehnte vor allem jede Form der Fiskalpolitik ab.










Zunächst in Verbindung mit Österreichischer Schule, später wendete er sich ab (über die Mont-Pelerin Gesellschaft, deren zweiter Vorsitzender er nach Hayek war)


Als Politberater: Abschaffung der Wehrpflicht (Appelbaum-Buch). Abschaffung des Goldstandards? (Im Gegensatz zur Österreichischen Schule.)




Später extrem liberal. Für Liberalisierung von Drogen und praktisch allem im hohen Altern noch auf der Straße.

Friedman extrem bekannt. Wirtschaftswissenschaftlicher Einfluss bestand aber vor allem darin, dass er als Gegenspieler der Keynesianer sich etablierte. Ökonomische Theorie selbst wurde rasch von den Neuen Klassikern überholt. Er selbst war ein sehr gnadenloser Diskussionspartner und herausragender Lehrer, der wie sonst fast keiner Generationen von Studierenden in seinen Bann zog. 




