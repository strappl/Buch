%%%%%%%%%%%%%%%%%%%%% chapter.tex %%%%%%%%%%%%%%%%%%%%%%%%%%%%%%%%%
%
% sample chapter
%
% Use this file as a template for your own input.
%
%%%%%%%%%%%%%%%%%%%%%%%% Springer-Verlag %%%%%%%%%%%%%%%%%%%%%%%%%%

\chapter{Der Monetarismus: Money Matters!}
\label{Monetarismus}

Der Monetarismus ist eng verbunden mit dem Namen \textsc{Milton Friedman} und der "`Chicago School"'. Genau genommen ist Friedman der Hauptvertreter der sogenannten \textit{zweiten Generation} der Chicago School\footnote{Die Chicago School wurde ursprünglich schon von \textit{Frank Knight}, den wir im Kapitel \ref{FisherundKnight} kennenlernen, begründet. Sie umfasst weit mehr als den Monetarismus Friedman's und kommt in diesem Buch noch öfter vor. Häufig verwendet man den Begriff der Chicago School aber auch einfach als Synonym für Friedman's Lehren. So zum Beispiel in "`Chicago Liberalismus"' und "`Chicago Boys"'. Man kann den Begriff aber auch als Überbegriff Über alle in den USA entstandenen wirtschaftsliberalen Schulen verwenden. Dann umfasst er neben den Arbeiten von Frank Knight und Jacob Viner auch die "`Neue Klassische Makroökonmie"' (Kapitel \ref{Neue Makro}) und die "`Neoklassische Finance"' (Kapitel \ref{Finance})}, seine Lehren werden aber häufig synonym etwas unscharf einfach als "`Chicago School"' bezeichnet. Neben Milton Friedman werden meist nur zwei weitere Ökonomen als Vertreter des Monetarismus\footnote{Monetarismus im engeren Sinn. Denn auch der Begriff des Monetarismus wird häufig unscharf und allgemein für verschiedene liberale Denkrichtungen verwendet} genannt. Vergessen wird häufig aber auf die Co-Autorin Friedman's: \textsc{Anna Jacobson Schwartz}. Sie ist neben Joan Robinson wohl die erste bedeutende Ökonom\textit{in}. Einer der beiden häufig genannten ist \textsc{Karl Brunner}, der vielleicht einzige bedeutende Schweizer Ökonom\footnote{Neben dem Ricardo-Kritiker Jean-Charles-Léonard Simonde de Sismondi}. Ihm wird die breite Etablierung des Begriffs Monetarismus zugeschrieben \parencite{Pierenkemper2012, Brunner1968}. Der zweite ist \textsc{Allan Meltzer}. Die beiden publizierten gemeinsam zum Monetarismus \parencite{Meltzer1971}, standen aber insgesamt stets im Schatten von Milton Friedman. Der Monetarismus war im Großen und Ganzen weitgehend seine "`One-Man-Show"'. 

Inhaltlich ist eine Gesamteinordnung des Monetarismus gar nicht so einfach. Meiner Meinung nach muss man hier zwei Komponenten auseinander halten. Erstens, die wirtschaftswissenschaftliche Komponente. Diese findet sich primär in der "`Wiederentdeckung"' der Bedeutung der Geldpolitik, abgeleitet aus empirischen Untersuchungen. Damit verbunden ist auch die strikte - aber hier wissenschaftlich getriebene und auf konkrete Punkte abzielende - Ablehnung des Keynesianismus. 
Die zweite Komponente, neben der wissenschaftlichen, nenne ich die politische Komponente. Friedman's Monetarismus (wie übrigens die gesamte Chicago School) ist verbunden mit einer strikten Ablehnung jeglichen staatlichen Einflusses. Dies kann nur zum Teil aus seiner wissenschaftlichen Arbeit abgeleitet werden. Aber viele seiner extrem liberalen Einstellungen sind nicht wissenschaftlich begründet, sondern wohl eher seiner persönlichen Überzeugung geschuldet. 

Die wissenschaftliche Ablehnung des Keynesanismus durch Friedman beruht auf der von ihm entwickelten permanente Einkommenshypothese. Mit dieser zeigte Friedman, dass nicht das aktuelle Einkommen, wie bei Keynes, entscheidend für Konsum ist, sondern das langfristige Einkommen. Daraus versucht Friedman abzuleiten, dass Fiskalpolitik weitgehend unwirksam sei, woraus man wiederum ableiten kann, dass aktive Wirtschaftspolitik abzulehnen ist. Ein weiterer wichtiger wissenschaftlicher Beitrag von Friedman war es die Unzulänglichkeit der Phillips-Kurve aufzuzeigen. Allerdings lieferte hierbei Edmund Phelps die formal-wissenschaftliche Begründung, weshalb dieser Punkt in Kapitel \ref{micmac} aufgezeigt wird.

Friedman wurde 1912 in New York geboren und musste nach dem frühen Tod seines Vaters schon früh Geld verdienen. Dennoch konnte er bereits 1932 ein Bachelorstudium in Mathematik und Ökonomie abschließen. Danach war er in verschiedenen öffentlichen Institutionen tätig, bevor er 1946 sein Doktorratsstudium abschloss und im selben Jahr eine Professur in Chicago erhielt \parencite[S. 204]{Linss2017}. An der University of Chicago sollte er nicht nur die nächsten 30 Jahre bleiben, sondern diese auch nachhaltig prägen.

Friedman wird häufig als der größte Ökonom des 20. Jahrhunderts neben John Maynard Keynes bezeichnet. Aus wirtschaftswissenschaftlicher Sicht waren die 1960er Jahre geprägt von der Debatte zwischen Keynesianern und Monetaristen. Sicherlich waren Friedman's Leistungen großartig. Er etablierte die "`Chicago School"' über Generationen als eine der führenden wirtschaftswissenschaftlichen Denkschulen weltweit. Sein Ruf als hervorragender Lehrer ist legendär. Viele spätere große Ökonomen berufen sich darauf, dass der Besuch in Milton Friedman's Vorlesung sie erst vom Fach Ökonomie überzeugt habe. Ebenso enorm, und auch enorm umstritten, war sein politischer Einfluss. Er selbst hatte wohl entscheidenden Anteil an der liberalen US-Politik unter Ronald Reagan, dessen informeller politischer Berater er schon früh war. Die Chicago Boys, Chicago-School-Absolventen, die sich primär auf seine Lehren bezogen, nahmen - allerdings höchst umstritten - direkt Einfluss auf die diktatorische Regierung in Chile.

Aus ökonomischer Sicht war seine Arbeit, wie bereits angedeutet, vielfältig und in vielerlei Hinsicht bahnbrechend. Aber vom rein wissenschaftlichen Standpunkt, waren seine Arbeiten bei weitem nicht so revolutionär wie jene von Keynes. Zwar spricht man in der Literatur häufig von der "`Monetaristischen Gegenrevolution", es war aber schlussendlich nicht Friedman's Monetarismus, der den Keynesianismus als "`State of the Art"'-Lehre überwand. Das waren, meines Erachtens, vielmehr seine Nachfolger in Chicago, die "`Neuen Klassiker"' (Vergleiche \ref{Neue Makro}). Allerdings war Friedman neben den Vertretern der Österreichischen Schule, der erste Ökonom, der die Theorien des Keynesianismus in Frage stellte.
Auch auf die Praxis angewendet war der Monetarismus nicht wirklich von Erfolg geprägt \parencite[S. 709]{Samuelson1998}. In wissenschaftlichen wie auch wirtschaftspolitischen Belangen wurde der Monetarismus bereits Anfang der 1980er Jahre von den "`Neuen Klassikern"' überholt. Viele Ökonomen, wie zum Beispiel Robert Solow, sind der Ansicht, die "`Neue Klassik"' sei nur eine Erweiterung des Monetarismus\parencite[S.342]{Warsh}. Aber die neuen Methoden\footnote{Mikrofundierung, dynamische Modelle, rationale Erwartungen (vergleiche Kapitel \ref{Neue Makro})} der "`Neuen Klassik"' unterscheiden sich doch stark vom "`Monetarismus"' und waren die \textit{wirkliche} Revolution, die die Keynesianer schließlich dazu "`zwangen"' ihre Ideen anzupassen und zu "`Neu Keynesianern"' zu werden. In der Retrospektive meint etwa \textcite[S. 697]{Samuelson1998} - immerhin das erfolgreichste VWL-Lehrbuch -, dass der Monetarismus dem Keynesianismus "`geistig nahe stehe"' und die beiden Schulen "`weitgehend einträchtig"' \parencite[S. 702]{Samuelson1998} seien. Ein "`radikal neuer Ansatz"'\parencite[S. 704]{Samuelson1998} sei hingegen die "`Neue Klassische Makroökomie"'.

Den ersten wissenschaftlich bedeutenden Beitrag lieferte Friedman im Jahr 1957 \parencite{Friedman1957} als er seine, bereits kurz genannte, "`Permanente Einkommenshypothese"' veröffentlichte. Gemeinsam mit der "`Lebenszyklushypothese"' \parencite{Modigliani1954} von Franco Modigliani lieferte er eine theoretische und empirisch fundierte Kritik an der "`Relativen Einkommenshypothese"' der Keynesianischen Theorie. Diese Theorie geht davon aus, dass das langfristige Einkommen - also auch das zukünftig zu erwartende Einkommen - die gegenwärtigen Ausgaben für Konsum bestimmen. Keynes hingegen ging davon aus, dass der aktuelle Konsum ausschließlich vom aktuellen Einkommen abhing. In Friedman's "`Permanente Einkommenshypothese"' wurde später hineininterpretiert, dass sie zeige, dass Steuersenkungen als keynesianische Wirtschaftspolitik häufig weitgehend unwirksam seien, weil die resultierendem Einkommenserhöhungen nicht unmittelbar wirksam werden. Die "`Permanente Einkommenshypothese wurde und wird noch Jahrzehnte später empirisch überprüft und als immer noch aktuell angesehen \parencite{Bernanke1984, Mankiw1985}.  

Friedmans's wissenschaftliches Hauptwert war aber zweifelsohne die Wiederentdeckung der Bedeutung der Geldpolitik und hier vor allem die erneute Etablierung der Quantitätstheorie des Geldes. Die wissenschaftlichen Arbeiten dazu lieferte er in einer Zeit als beides kaum eine Rolle spielte: In den 1960er Jahren, der größten Zeit des Keynesianismus. Sein Hauptwerk das fast schon monumentale Wert \textit{A Monetary History of the United States, 1867–1960} ist ein wirtschaftsgeschichtliches Werk, in dem er gemeinsam mit Anna Jacobson Schwartz beachtliches Datenmaterial zusammentrug und schließlich zeigte, dass die "`Great Depression"' der 1930er Jahre erst durch einen Mangel an Geld zur größten Wirtschaftskrise aller Zeiten wurde. Er machte dafür zum Teil den Goldstandard, vor allem aber die Unfähigkeit der Federal Reserve verantwortlich. Seiner Meinung nach hätte die Fed die Schwere der Krise wesentlich abschwächen können, wenn sie die Geldmenge ausgeweitet hätte. Ein Punkt übrigens, in dem die Monetaristen eben gar nicht so erz-liberal waren wie sie sich oft gern selbst sahen. Friedman meinte zeitlebens, dass sobald die Krise einmal eingetreten sei, der Staat in Form von Geldpolitik sehr wohl eingreifen sollte. Die Österreichische Schule in Form von Friedrich Hayek hingegen kritisierte jeglichen Eingriff in den freien Markt auch in Krisenzeiten. Friedman kritisierte Hayek noch 1999 vehement für diese Ansicht: "`I think the Austrian business-cycle theory has done the world a great deal of harm [in the 1930s]"' \parencite{Epstein1999}.

Die Geldpolitik blieb in weiterer Folge - wie der Name Monetarismus schon sagt - \textit{das} zentrale Thema Friedman's \parencite{Friedman1968}. Sein Instrument hierbei war die Quantitätsgleichung des Geldes. Der \textit{Wert} des Geldes ist hierbei rein von der \textit{Menge} des Geldes abhängig. Es gilt $ M * v = BIP * P$. $M$ bezeichnet hierbei die Geldmenge. $v$ ist die Umlaufgeschwindigkeit des Geldes und wurde stets als relativ konstant vernachlässigt. Auf der anderen Seite ist das $BIP$ natürlich die Wirtschaftsleistung des Landes und $P$ das Preisniveau. Diese Quantitätsgleichung des Geldes war  schon David Hume, dem Freund und Zeitgenossen von Adam Smith, bekannt, und ihre moderne Form wurde von Irving Fisher \parencite{Fisher1911} Anfang des 20. Jahrhunderts beschrieben. Mit Keynes aber verlor sie ihre Bedeutung, weil dieser der Meinung war, die Quantitätsgleichung wäre nur bei Vollauslastung der Wirtschaft gültig, also etwa bei einer Arbeitslosigkeit von 0\%.
Für Friedman hingegen war klar, dass die Quantitätsgleichung und damit vor allem die Geldmenge eine zentrale Rolle spielten. Aus der Formel sieht man, dass - wenn man die Umlaufgeschwindigkeit $v$ vernachlässigt - die Geldmenge in direktem Zusammenhang mit dem BIP, oder dem Preisniveau, oder beidem steht. Daraus können verschiedene Implikationen gezogen werden. 
Erstens, für Friedman war klar, dass nicht zu geringe Nachfrage, sondern eine zu geringe Geldmenge das Hauptproblem in Wirtschaftskrisen ist. Eine zu geringe Geldmenge führt nämlich zu Deflation und all ihren bekannten (vgl. Kapitel \ref{Keynes}) negativen Auswirkungen.
Zweitens, während Wirtschaftskrisen ist Geldpolitik wichtig und wirksamer als Fiskalpolitik. Dies leitet sich direkt daraus ab, dass eine steigende Geldmenge in Wirtschaftskrisen Deflation und somit die Krise selbst bekämpft. Erinnern Sie sich was Keynes zur Wirksamkeit der Geldpolitik in Wirtschaftskrisen sagte - Stichwort Liquiditätsfalle und Investitionsfalle - damit werden potentielle Streitpunkte zwischen Keynesianern und Monetaristen ersichtlich. Dieser Streit "`Keynesianer gegen Monetaristen"' oder "`Fiskalpolitik vs. Geldpolitik"' prägte die wirtschaftswissenschaftliche Debatte in den 1960er-Jahren. 

Daneben entwickelte sich eine weniger wissenschaftliche, dafür umso mehr politische Debatte. Weiter oben die "`politische Komponente"' genannt. Die Monetaristen könnten aus ihrer, wissenschaftlich hergeleiteten, Präferenz für Geldpolitik nämlich auch eine geldpolitische Feinabstimmung empfehlen. Stattdessen befürworten sie eine Politik der freien Märkte. Außerhalb von Wirtschaftskrisen hielt Friedman allgemein nichts von aktiver Wirtschaftspolitik - also auch nichts von aktiver Geldmengensteuerung. Die Vermischung der wirtschaftswissenschaftlichen und politischen Bedeutung Milton Friedman's zeigt nichts deutlicher, als sein Artikel \textcite{Friedman1968}. Diese "`Presidential Address"' hatte enormen Einfluss, was sich auch daran zeigt, dass noch 50 Jahre später darüber geschrieben wird \parencite{Mankiw2018}. Friedman legt darin seine Überzeugungen dar: Erstens, dass die Wirtschaft grundsätzlich auf einem natürlichen, stabilen Wachstumskurs sei. Zweitens, dass geringfügige Instabilitäten grundsätzlich wirtschaftspolitisch nicht beseitigt werden können. Und drittens, wenn Instabilitäten von Regierungen beseitigt werden könnten, dann hätten diese Regierungen oftmals das Interesse im Sinne kurzfristiger Stimmengewinne zu handeln und nicht im Sinne langfristiger Stabilität. 

Im Gegensatz zu den Keynesianern war er der Überzeugung, dass sich die Wirtschaft ohne jegliche Eingriffe auf einem stabilen Wachstumskurs einfinden würde und dass Preise und Löhne grundsätzlich eher flexibel sind und nicht so rigide, wie im Keynesianismus angenommen \footnote{Diese Auffassung blieb übrigens für immer Teil der Ideologie der Chicago School}. Feinabstimmungen mittels Fiskal- und auch Geldpolitik wären daher eher nutzlos bis kontraproduktiv. Theoretisch wird dies damit unterlegt, dass der Monetarismus auch von einer sehr elastischen Anpassung der Zinssätze ausgeht. Das heißt die keynesianische Fiskalpolitik führt zu höheren Zinssätzen und damit auch zu Rückgängen in der Investitionstätigkeit am privaten Sektor. Die staatlichen Investitionen verdrängen also die privaten Investitionen ("`Crowding-Out"'). Somit kommt es nicht zu den erwünschten Wachstumseffekten. Eine wirklich tiefgreifende formal-theoretische oder empirische Fundierung dieser Empfehlungen blieb Friedman selbst aber weitgehend schuldig.
Dennoch sollte konsequenterweise eine langfristig erfolgreiche Wirtschaftspolitik seiner Meinung nach daher einfach darin bestehen, die Geldmenge mit einer konstanten Rate wachsen zu lassen. Zum Beispiel könnte man die Geldmenge jedes Jahr um 4\% wachsen lassen. Wenn das langfristige Wachstum 2\% beträgt, wird sich das BIP-Wachstum bei diesem einpendeln und gleichzeitig - entsprechend der Quantitätsgleichung - die Inflation konstant 2\% betragen. Für stabiles Geldmengenwachstum brauche es laut Friedman keine Zentralbank. Die Geldmenge konstant wachsen zu lassen schaffe schließlich auch ein Computer, der die Zentralbanken ersetzen könne, meinte er. Bei seiner Nobel-Preis-Bankett-Rede \parencite{Friedman1976b} scherzte er darüber und meinte sinngemäß "`Zum Glück haben sich die Entscheidungsträger nicht an den Vorschlag gehalten, ansonsten wäre ich um diesen Preis gekommen"'\footnote{Der sogenannte Wirtschaftsnobelpreis wird von der Schwedischen Zentralbank vergeben}

Die zeitlichen Geschehnisse spielten Friedman natürlich in die Hände. Seine Krisenerklärung durch die Quantitätstheorie stellt Geldpolitik, und die damit eng verbundene Inflation, in den Mittelpunkt der Forschung. Und genau die Inflation wurde in den 1970er Jahren in den USA zum zentralen wirtschaftspolitischen Problem. Als in den 1970er Jahren Stagflation auftrat, konnte Friedman quasi triumphierend seine Erklärung für Inflation mittels Quantitätstheorie aus der Tasche ziehen, während die führenden Keynesianer noch erfolglos nach Erklärungen suchten. Seine Aussage: "`Inflation is always and everywhere a monetary phenomenon"', wurde berühmt. Ebenso sein Helikopter-Geld-Beispiel \parencite{Friedman1969} (sinngemäß): Angenommen die Geldmenge würde sich wundersam von einem Moment auf den anderen verdoppelt - Was würde mit den Preisen geschehen? Jeder Erstsemester-Student antwortet intuitiv: "`Die würden sich auch verdoppeln!"' wird gerade jetzt wieder häufig zitiert\footnote{In vielen Fällen aber grundlegend falsch. So wird diese rein \textit{geld}politische Maßnahme häufig mit \textit{fiskal}politischen Maßnahmen vermischt. Außerdem deutete Friedman damit die Neutralität des Geldes an und eben nicht die Wirksamkeit geldpolitischer Maßnahmen}. Seine Erklärung der Inflation und die wirtschaftspolitischen Empfehlungen waren einfach: Das Geldmengenwachstum müsse eingeschränkt werden!   
Die erste Zentralbank, die sich dann an die praktische Umsetzung seiner Ideen machte, war übrigens die Deutsche Bundesbank, die ab 1975 eine reine Geldmengenstrategie verfolgte \parencite[S. 36]{BBK2016}. 
In den USA war das Inflationsproblem allerdings wesentlich ausgeprägter. Ende der 1970er Jahre wurden die Inflationsraten zweistellig. Die praktische Umsetzung des Monetarismus in den USA erfolgte durch \textsc{Paul Volcker}, der 1979 von Präsident Jimmy Carter zum Fed-Vorsitzenden gemacht wurde. Er wendete den Monetarismus im Sinne einer Geldmengensteuerung an. Das heißt die Zentralbank erhöhte nicht nur den Leitzinssatz um die Inflation einzudämmen, sondern verhinderte aktiv eine zu starkes Ansteigen der Geldmenge. Das ganze wurde als "`Monetaristisches Experiment"' bezeichnet und hatte als einziges Ziel eine "`Disinflation"', also ein Sinken der Inflation, herbeizuführen. Das klingt eigentlich recht einfach. Man darf aber nicht vergessen, dass die Inflationserwartungen aufgrund der zuletzt ständig hohen Inflationsraten sehr hoch waren und beim Abschluss von Verträgen eine große Rolle spielten. Schwieriger als die Inflation kurzfristig nach unten zu drücken war es die Inflationserwartungen langfristig zu senken. Die Monetaristen argumentierten, wenn erst einmal die Inflationserwartungen auf einem vernünftigen Niveau seien, würde sich rasch ein Gleichgewichtswachstum einstellen. Man kann bis heute darüber streiten ob dieses monetaristische Experiment erfolgreich war. Faktum ist, dass Milton Friedman die Maßnahmen nicht als Monetarismus in seinem Sinne akzeptierte "`Seine Ideen werden hier nicht korrekt umgesetzt"', rechtfertigte er sich, weil sie nicht vollständig seinen Vorstellungen entsprachen\parencite{Kremer2018}. Faktum ist auch, dass die Inflationsraten tatsächlich schnell und dauerhaft sanken, diesbezüglich war das Experiment auf jeden Fall erfolgreich. Bereits 1982 war die Inflation in den USA auf ca. 4\% gesunken. Die Nebeneffekte, nämlich zweistellige Arbeitslosenraten und eine mehrjährige Rezession sorgen bis heute für Diskussionen, ob die Maßnahmen als positiv oder überwiegend negativ zu bewerten waren. Eine "`schmerzlose"' Anit-Inflationsstrategie war der tatsächlich angewendete Monetarismus jedenfalls nicht, der Beweis, dass Geldpolitik wirksam sein kann, war aber erbracht \parencite[S. 703]{Samuelson1998}.
Insgesamt zusammengefasst bleibt die Geldtheorie der Monetaristen damit aber auch etwas verwirrend: Für Friedman stand die Geldpolitik im Mittelpunkt, als einzig wirksames wirtschaftspolitisches Instrument  - also gilt: MONEY MATTERS! Aber er war dafür dieses Instrument nicht aktiv einzusetzen, weil es, aktiv eingesetzt in der Praxis doch nur zu Inflation führe. Als Beispiel nannte er das eben angeführte Helikopter-Geld, das eben \textit{nicht} zu BIP-Wachstum, sondern ausschließlich zu Inflation führen sollte - also eigentlich gilt MONEY DOESN'T MATTER, oder? Lösung: Es kommt bei Friedman immer auf den Zusammenhang an: Während Krisen (und bei Missmanagement eben auch als Auslöser von Krisen) ist Geldpolitik sinnvoll und wichtig. Außerhalb von Krisen sollten Politiker aber die Finger von Geldpolitik lassen, weil sie im Hinblick auf ihre kurzfristigen Ziele verlockende aber falsche Maßnahmen setzen und nur Inflation auslösen würden. 
Interessanterweise war dem Monetarismus in der Praxis nur kurzer Erfolg vergönnt. Aktive Geldmengensteuerung kann nämlich nur dann erfolgreich sein, wenn die Umlaufgeschwindigkeit des Geldes als "`weitgehend konstant"' vernachlässigt werden kann. Aber schon Anfang der 1980er Jahre begann diese Umlaufgeschwindigkeit - im Gegensatz zu den Jahrzehnten davor - zu schwanken \parencite[S. 703]{Samuelson1998}. Allgemein wurde der Monetarismus zu dieser Zeit von den tatsächlich revolutionären Ideen der "`Neuen Klassischen Makroökonomen"' aus wissenschaftlicher Sicht in vielerlei Hinsicht "`überholt"'. Seine "`adaptiven Erwartungen"' im Hinblick auf die Inflationserwartungen wurden durch das Konzept der "`rationalen Erwartungen"' abgelöst. Geldpolitik und die optimale Organisation von Zentralbanken wurde von den "`Neuen Klassikern"' ebenso völlig neu gedacht. 

Milton Friedman war in den 1970er Jahren auf dem Höhepunkt seiner wissenschaftlichen Karriere angelangt. 1976 erhielt er den Nobelpreis für Wirtschaftswissenschaften. Schon zuvor und auch in weiterer Folge sollte Milton Friedman als Polit-Berater des späteren US-Präsidenten Ronald Reagan, aber auch mit einer eigenen Fernseh-Show, die später ein Buch-Bestseller wurde \parencite{Friedman1980}, eine enorme öffentliche Wahrnehmung erfahren. Diesbezüglich war sein Einfluss für einen Ökonomen wirklich außergewöhnlich. Möglicherweise wird dadurch seine wissenschaftliche Leistung aber auch etwas überschätzt.
Seine wissenschaftliche Arbeit war recht formal - so publizierte er in frühen Jahren auch zu statistischen Themen - aber weit weniger formal als später die Arbeiten der "`Neuen Klassiker"', die die ökonomische Methodik auf eine höhere Stufe hoben. Aber auch seine keynesianischen "`Zeitgenossen"' wie zum Beispiel Edmund Phelps oder  James Tobin, brachten aus formal-analytischer Sicht höherwertigere Wissenschaft hervor. Seine empirischen Analysen - hauptsächlich seine "`Monetary History of the US"' waren hingegen bahnbrechend.
Seine Stärke war, dass er die richtigen Themen zur richtigen Zeit wählte und auf seinen Aussagen beharrte. Aussagen, die zu seiner Zeit dem keynesianischen Zeitgeist radikal widersprachen. Friedman war in jeder Hinsicht radikal liberal: So hatte er einen nicht zu unterschätzenden Anteil an der Abschaffung der Wehrpflicht in den USA \textcite{Appelbaum2019}, er war durchgehend ein Befürworter frei schwankender Wechselkurse. Heute sind diese längst Realität, aber nach dem Zweiten Weltkrieg waren diese unvorstellbar. Und als der Keynesianismus Ende der 1940er Jahre seinen Siegeszug in den USA startete, etablierte er sich früh als Gegenspieler und verband sich stattdessen - wenn auch nur vorübergehend mit den europäischen Liberalen\footnote{Friedman war ein der Gründungsmitglieder der ultra-liberalen Mont-Pelerin-Gesellschaft und von 1970 bis 1972 deren Präsident. Aus wirtschaftswissenschaftlicher Sicht emanzipierte er sich aber bald weg von den Ideen der "`Freiburger Schule"' und auch der "`Österreichischen Schule"' (vgl. Kapitel \ref{Neoliberalismus})}. Noch im hohen Alter protestierte er auf den Straßen für die Liberalisierung von Drogen, aber auch für die Entkriminalisierung des Schwangerschaftsabbruchs oder die Rechte von Homosexuellen. Seine liberalen Ansichten legte er auch bereits 1962 in "`Capitalism and Freedom"' \parencite{Friedman1962} und vor allem später in "`Free to Choose"' \parencite{Friedman1980} der breiten Bevölkerung außerhalb der wirtschaftswissenschaftlichen Community dar. Sein politischer Einfluss ist umstritten. Der Einfluss auf Ronald Reagan, der die USA wirtschaftspolitisch radikal revolutionierte, ist wohl belegt. Hier spielten aber auch andere liberale Ökonomen eine Rolle. \textcite[S. 77]{Warsh} schreibt, dass Friedman und seine Anhänger "`Chile, den Rest von Lateinamerika und Ost-Europa Richtung "`freie Märkte"' führte. Ob sein Einfluss wirklich so groß war wird man wohl nie realistisch abschätzen können.

Er selbst war ein sehr gnadenloser Diskussionspartner und herausragender Lehrender, der, wie sonst fast niemand, Generationen von Studierenden in seinen Bann zog. 
Auch wenn sein Monetarismus als solcher im Endeffekt als gescheitert gilt, so kann man die heute unumstrittenen Bedeutung der Geldpolitik im wirtschaftspolitischen Instrumentarium doch noch immer zu einen guten Teil auf Milton Friedman zurückführen. Als Milton Friedman 2006 starb verpasste er nur um wenige Jahre die erneute, gnadenlose Anwendung seiner Ideen: Das in den führenden Industriestaaten während der "`Great Recession"' nach 2008 durchgeführte "`Quantitative Easing"'  kann auf Friedman's Lehren aus seinen Arbeiten zur "`Great Depression"' zurückgeführt werden. Zu den Feierlichkeiten im Rahmen von Milton Friedman's 90 Geburtstag im Jahr 2002 sagte der spätere Fed-Vorsitzende Ben Bernanke zu Milton Friedman and Anna J. Schwartz: "`Regarding the Great Depression, you’re right. We [the Fed] did it [fail to provide a stable monetary background]. We’re very sorry. But, thanks to you, we won't do it again."' Zu diesem Zeitpunkt wusste Bernanke noch nicht, dass er nur wenige Jahre später tatsächlich, als Fed-Vorsitzender, die USA mithilfe der Lehren von Milton Friedman und Anna J. Schwartz relativ glimpflich durch die größte Wirtschaftskrise seit der "`Great Depression"' führen sollte.
