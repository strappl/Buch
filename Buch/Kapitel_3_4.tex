%%%%%%%%%%%%%%%%%%%%% chapter.tex %%%%%%%%%%%%%%%%%%%%%%%%%%%%%%%%%
%
% sample chapter
%
% Use this file as a template for your own input.
%
%%%%%%%%%%%%%%%%%%%%%%%% Springer-Verlag %%%%%%%%%%%%%%%%%%%%%%%%%%

\chapter{Der Monetarismus: Money Matters!}
\label{Monetarismus}

Der Monetarismus ist eng verbunden mit dem Namen \textit{Milton Friedman} und der "`Chicago School"'. Eigentlich ist Friedman der Hauptvertreter der sogenannten zweiten Generation der Chicago School\footnote{Die Chicago School wurde ursprünglich schon von \textit{Frank Knight}, den wir im Kapitel \ref{FisherundKnight} kennenlernen, begründet. Sie umfasst weit mehr als den Monetarismus Friedman's und kommt in diesem Buch noch öfter vor. Häufig verwendet man den Begriff der Chicago School aber als Synonym für Friedman's Lehren. So zum Beispiel in "`Chicago Liberalismus"' und "`Chicago Boys"'. Man kann den Begriff aber auch als Überbegriff aber alle in den USA entstandenen wirtschaftsliberalen Schulen verwenden. Dann umfasst er neben den Arbeiten von Frank Knight und Jacob Viner auch die "`Neue Klassische Makroökonmie"' (Kapitel \ref{Neue Makro}) und die "`Neoklassische Finance"' (Kapitel \ref{Finance})}, seine Lehren werden aber häufig Synonym etwas unscharf einfach als "`Chicago School"' bezeichnet. 

Inhaltlich ist eine Gesamteinordnung von Friedman gar nicht so einfach. Meiner Meinung nach muss man hier zwei Komponenten auseinander halten. Erstens, die wirtschaftswissenschaftliche Komponente. Diese findet sich primär natürlich in der "`Wiederentdeckung"' der Geldpolitik. Damit verbunden ist auch die strikte - aber hier wissenschaftlich getriebene und auf konkrete Punkte abzielende - Ablehnung des Keynesianismus. So zeigte Friedman zum Beispiel, dass die keynesianische Konsumtheorie zu kurz greift. Auch die Unzulänglichkeit der Phillips-Kurve wurde zunächst von Friedman (und Edmund Phelps, siehe Kapitel \ref{micmac}) in Frage gestellt.
Die zweite Komponente nenne ich die politische Komponente. Friedman's Monetarismus (wie übrigens die gesamte Chicago School) ist verbunden mit einer strikten Ablehnung jeglichen staatlichen Einflusses. Dies kann nur zum Teil aus seiner wissenschaftlichen Arbeit abgeleitet werden. So versucht Friedman zu zeigen, dass Fiskalpolitik weitgehend unwirksam sei, daraus kann man ableiten, dass aktive Wirtschaftspolitik und somit staatliche Einmischung abzulehnen ist. Aber viele seiner extrem liberalen Einstellungen sind nicht wissenschaftlich begründet, sondern wohl eher seine persönliche ideologische Einstellung.
Der kleine Zwiespalt zwischen wissenschaftlicher und politischer Überzeugung zeigt sich in Nuancen in seiner Arbeit. So lehnt die rigoros-liberale Österreichische Schule jeglichen Eingriff in die Ökonomie ab. Und diese Ablehnung des Staates ist auch Teil der wissenschaftlichen Arbeiten vor allem in von Mises und Hayek. Friedman veröffentlichte zwar auch früh eine "`Bibel"' des Liberalismus mit \textit{Capitalism and Freedom} \textcite{Friedman1962}. Seine wichtigsten \textit{wissenschaftlichen} Beiträge sind aber andere und diese beinhalten kaum Belege für seine "`Anti-Staat-Einstellung"'. 

Friedman wird häufig als der größte Ökonom des 20. Jahrhunderts neben John Maynard Keynes bezeichnet. Sicherlich waren seine Leistungen großartig. Er etablierte die "`Chicago School"' über Generationen als eine  der führenden wirtschaftswissenschaftlichen Denkschulen weltweit. Sein Ruf als hervorragender Lehrer ist legendär. Viele spätere große Ökonomen berufen sich darauf, dass der Besuch in Milton Friedman's Vorlesung sie erst vom Fach Ökonomie überzeugt habe. Ebenso enorm, und auch enorm umstritten, war sein politischer Einfluss. Er selbst hatte wohl entscheidenden Anteil an der liberalen US-Politik unter Ronald Reagan, dessen informeller politischer Berater er schon früh war. Die Chicago Boys, die sich primär auf seine Lehren bezogen nahmen aber auch direkt Einfluss auf die diktatorische Regierung in Chile.

Aus ökonomischer Sicht war seine Arbeit auch extrem vielfältig und in vielerlei Hinsicht bahnbrechend. Aber vom rein wissenschaftlichen Standpunkt her waren seine Arbeiten bei weitem nicht so revolutionär wie jene von Keynes. Es war schlussendlich auch nicht sein Monetarismus, der den Keynesianismus als "`State of the Art"'-Lehre überwand. Das waren meines Erachtens vielmehr seine Nachfolger in Chicago, die "`Neuen Klassiker"' (Vergleiche \ref{Neue Makro}). Auch auf die Praxis angewendet war dem Monetarismus nicht wirklich von Erfolg geprägt(ZITAT). In beiden Belangen wurde der Monetarismus bereits Anfang der 1980er Jahre von den "`Neuen Klassikern"' überholt. Viele Ökonomen, wie zum Beispiel Robert Solow, sind der Ansicht, die "`Neue Klassik"' sei nur eine Erweiterung des Monetarismus\parencite[S.342]{Warsh}. Aber die neuen Methoden\footnote{Mikrofundierung, dynamische Modelle, rationale Erwartungen (vergleiche Kapitel \ref{Neue Makro})} der "`Neuen Klassik"' waren die wirkliche Revolution, die die Keynesianer schließlich dazu "`zwangen"' ihre Ideen anzupassen und zu "`Neu Keynesianern"' zu werden.

Friedman Biographie


Den ersten wissenschaftlich bedeutenden Beitrag lieferte er im Jahr 1957 \parencite{Friedman1957} als er seine "`Permanente Einkommenshypothese"' veröffentlichte. Gemeinsam mit der "`Lebenszyklushypothese"' \parencite{Modigliani1954} von Franco Modigliani lieferte er eine theoretische und empirisch fundierte Kritik an der relativen Einkommenshypothese der Keynesianischen Theorie. Die neuen Theorien gehen davon aus, dass das langfristige Einkommen auch in Form der zukünftig zu erwarteten Einkommen die Ausgaben für Konsum bestimmen. Keynes ging davon aus das der aktuelle Konsum ausschließlich vom aktuellen Einkommen abhing. Aus Friedman's "`Permanenter Einkommenshypothese"' wurde - noch nicht von Friedman selbst in seinem Artikel aus 1957, sondern später - hineininterpretiert, dass sie zeige, dass Steuersenkungen als keynesianische Wirtschaftspolitik häufig weitgehend unwirksam seinen, weil die resultierendem Einkommenserhöhungen nicht unmittelbar wirksam werden. Die "`Permanente Einkommenshypothese wurde und wird noch Jahrzehnte später empirisch überprüft uns als immer noch aktuell angesehen \parencite{Bernanke1984, Mankiw1985}.  

Friedmans's wissenschaftlicher Hauptbeitrag war aber zweifelsohne die Wiederentdeckung der Bedeutung der Geldpolitik und hier vor allem die erneute Etablierung der Quantitätstheorie des Geldes. Die wissenschaftlichen Arbeiten dazu lieferte er in einer Zeit als beides kaum eine Rolle spielte: In den 1960er Jahren, der größten Zeit des Keynesianismus. Sein Hauptwerk das fast schon monumentale Wert \textit{A Monetary History of the United States, 1867–1960} ist ein wirtschaftsgeschichtliches Werk, in dem er gemeinsam mit Anna Jacobson Schwartz beachtliches Datenmaterial zusammentrug und schließlich zeigte, dass die "`Great Depression"' der 1930er Jahre erst durch einen Mangel an Geld zur größten Wirtschaftskrise aller Zeiten wurde. Er machte dafür zum Teil den Goldstandard, vor allem aber die Unfähigkeit der Federal Reserve verantwortlich. Seiner Meinung nach hätte die Fed die Schwere der Krise wesentlich abschwächen können, wenn sie die Geldmenge ausgeweitet hätte. Ein Punkt übrigens in dem die Monetaristen eben gar nicht so erz-liberal waren wie sie sich oft gern selbst sahen. Friedman meinte zeitlebens, dass sobald die Krise einmal eingetreten sei, der Staat in Form von Geldpolitik sehr wohl eingreifen sollte. Die Österreichische Schule in Form von Friedrich Hayek hingegen kritisierte jeglichen Eingriff in den freien Markt auch in Krisenzeiten. Friedman kritisierte Hayek noch 1999 vehement für diese Ansicht: "`I think the Austrian business-cycle theory has done the world a great deal of harm [in the 1930s]"' \parencite{Epstein1999}.

Die Geldpolitik blieb - wie der Name Monetarismus schon sagt - \textit{das} zentrale Thema Friedmans. Sein Instrument hierbei war die Quantitätsgleichung des Geldes. Der \textit{Wert} des Geldes ist hierbei rein von der \textit{Menge} des Geldes abhängig. Es gilt $ M * v = BIP * P$. $M$ bezeichnet hierbei die Geldmenge. $v$ ist die Umlaufgeschwindigkeit des Geldes und wurde stets als relativ konstant vernachlässigt. Auf der anderen Seite ist das $BIP$ natürlich die Wirtschaftsleistung des Landes und $P$ das Preisniveau. Diese Quantitätsgleichung des Geldes war schon angeblich schon David Hume, dem Freund und Zeitgenossen von Adam Smith bekannt, und ihre moderne Form wurde von Irving Fisher (ZITAT) Anfang des 20. Jahrhunderts beschrieben. Mit Keynes aber verlor sie jegliche Bedeutung, weil dieser der Meinung war, die Quantitätsgleichung wäre nur bei Vollauslastung der Wirtschaft gültig, also etwa bei einer Arbeitslosigkeit von 0\%.

Für Friedman hingegen war klar, dass die Quantitätsgleichung und damit vor allem die Geldmenge eine zentrale Rolle spielten. Aus der Formel sieht man, dass - wenn man die Umlaufgeschwindigkeit $v$ vernachlässigt - die Geldmenge in direktem Zusammenhang mit dem BIP, oder dem Preisniveau, oder beidem steht. Daraus können verschiedene Implikationen gezogen werden. 

Erstens, für Friedman war klar, dass nicht zu geringe Nachfrage, sondern eine zu geringe Geldmenge das Hauptproblem in Wirtschaftskrisen - konkret in der Great Depression - sind. Eine zu geringe Geldmenge führt nämlich zu Deflation und all ihren bekannten (vgl. Kapitel \ref{Keynes}) negativen Auswirkungen.

Zweitens, während Wirtschaftskrisen ist Geldpolitik wichtig und wirksamer als Fiskalpolitik - zur seiner vehementen Keynesianismus-Kritik kommen wir später. Dies leitet sich direkt daraus ab, dass eine steigende Geldmenge in Wirtschaftskrisen Deflation und somit die Krise selbst bekämpft. Erinnern Sie sich was Keynes zur Wirksamkeit der Geldpolitik in Wirtschaftskrisen sagte - Stichwort Liquiditätsfalle und Investitionsfalle - damit werden potentielle Streitpunkte zwischen Keynesianern und Monetaristen schön langsam klar!
In normalen Zeiten hingegen - also außerhalb von Wirtschaftskrisen - hielt Friedman nichts von aktiver Geldmengensteuerung. Er war der Meinung, dass Politiker versucht sein könnten mit der Geldmenge das BIP zu stimulieren, was aber in den meisten Fällen ausschließlich zu Inflation führen würde (ZITAT, Ansprache vor Kongress???). Im Gegensatz zu den Keynesianern war er der Überzeugung, dass sich die Wirtschaft ohne jegliche Eingriffe auf einem stabilen Wachstumskurs einfinden würde. Feinabstimmungen mittels Fiskal- und auch Geldpolitik wären eher kontraproduktiv. Diese Auffassung blieb übrigens für immer Teil der Ideologie der Chicago School. Eine langfristig erfolgreiche Wirtschaftspolitik sollte seiner Meinung nach daher einfach darin bestehen die Geldmenge mit einer konstanten Rate wachsen zu lassen. Zum Beispiel könnte man die Geldmenge jedes Jahr um 4\% wachsen lassen. Wenn das langfristige Wachstum 2\% beträgt, wird sich das BIP-Wachstum bei diesem einpendeln und gleichzeitig - entsprechend der Quantitätsgleichung - die Inflation konstant 2\% betragen. Für stabiles Geldmengewachstum brauche es laut Friedman keine Zentralbank. Die Geldmenge konstant wachsen zu lassen schaffe auf ein Computer, der die Zentralbanken ersetzen könne, meinte er. Bei seiner Nobel-Preis-Bankett-Rede \parencite{Friedman1976b} scherzte er darüber und meinte sinngemäß "`Zum Glück haben sich Entscheidungsträger nicht an den Vorschlag gehalten, ansonsten wäre ich um diesen Preis gekommen"'\footnote{Der sogenannte Wirtschaftsnobelpreis wird von der Schwedischen Zentralbank vergeben}

Aus der Politikempfehlung, keine Geldpolitik zu betreiben, folgt drittens, Inflation ist immer ausschließlich ein monetäres Phänomen. Das heißt die Geldmenge wächst schneller als das BIP, was zwangsläufig zu einem höheren Preisniveau, also Inflation, führen muss.

Die zeitlichen Geschehnisse spielten Friedman natürlich in die Hände. Denn als in den 1970er Jahren Stagflation auftrat, konnte Friedman quasi triumphierend seine Erklärung für Inflation mittels Quantitätstheorie aus der Tasche ziehen, während die führenden Keynesianer noch erfolglos nach Erklärungen suchten. Die erste Zentralbank, die sich dann an praktische Umsetzung seiner Ideen machte, war übrigens die Deutsche Bundesbank, die ab 1974 eine reine Geldmengenstrategie verfolgte (ZITAT). In den USA verbindet man die Politik von Volcker aber 1980 mit dem Monetarismus, was Friedman aber strikt ablehnte. "`Seine Ideen werden hier nicht korrekt umgesetzt"', rechtfertigte er sich. Tatsächlich war die praktische Umsetzung in den USA nicht wirklich von Erfolg gekrönt. Interessanterweise begann die als "`weitgehend konstant"' vernachlässigte Umlaufgeschwindigkeit des Geldes zu schwanken, sobald man die Geldmenge auf einen stabilen Wachstumskurs brachte.



Insgesamt zusammengefasst bleibt seine Geldtheorie aber etwas verwirrend: Für Friedman stand die Geldpolitik im Mittelpunkt - also: MONEY MATTERS! - aber er war dafür diese nicht aktiv einzusetzen, weil sie nur zu Inflation führe. Als Beispiel nannte er das eben angeführte Helikopter-Geld, das eben \textit{nicht} zu BIP-Wachstum, sondern ausschließlich zu Inflation führen sollte - also eigentlich MONEY DOESN'T MATTER, oder? Es kommt bei Friedman auf den Zusammenhang an: Während Krisen ist Geldpolitik sinnvoll. Außerhalb von Krisen sollten Politiker aber die Finger von Geldpolitik lassen, weil sie im Hinblick auf ihre kurzfristigen Ziele nur Inflation auslösen würden.



Kritik an Keynes: Legendäre Kontroverse zwischen Keynesianern und Monetaristen (Monetaristische Gegenrevolution). Auf Keynes-Seite: James Tobin. Warum Fiskalpolitk unwirksam. Crowding out: Zins steigt. ABER: Policy Mix. Lehnte vor allem jede Form der Fiskalpolitik ab.



Später Kritik an der Phillipskurve (mit Phelps).


Friedman  Höhepunkt als Wissenschaftler hatte kurze Zeit vor Lucas-Kritik und während Stagflation. Danach vor allem als Politberater.



Gedankenexperiment: Helikoptergeld (Quantitätsgleichung: Nur Inflation). Aber Unterschied kurze Frist und lange Frist.

Danach: adaptive Erwartungen: Leute erwarten Fortschreibung der derzeitigen Inflation. Daher wirksam nur Änderungen der Inflationsrate. Ersetzt später durch die Neuen Klassiker: rationalen Erwartungen, die das noch ausweiten


Wissenschaft recht formal (Publizierte auch zu Statistik), aber weit weniger als später die "`Neuen Klassiker"', oder auch z.B.: Phelps (und Tobin?). Seine Stärke war eher das auf den Punkt bringen.











Zunächst in Verbindung mit Österreichischer Schule, später wendete er sich ab (über die Mont-Pelerin Gesellschaft, deren zweiter Vorsitzender er nach Hayek war)


Als Politberater: Abschaffung der Wehrpflicht (Appelbaum-Buch). Abschaffung des Goldstandards? (Im Gegensatz zur Österreichischen Schule.)




Später extrem liberal. Für Liberalisierung von Drogen und praktisch allem im hohen Altern noch auf der Straße.

Friedman extrem bekannt. Wirtschaftswissenschaftlicher Einfluss bestand aber vor allem darin, dass er als Gegenspieler der Keynesianer sich etablierte. Ökonomische Theorie selbst wurde rasch von den Neuen Klassikern überholt. Er selbst war ein sehr gnadenloser Diskussionspartner und herausragender Lehrer, der wie sonst fast keiner Generationen von Studierenden in seinen Bann zog. 




