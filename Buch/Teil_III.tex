%%%%%%%%%%%%%%%%%%%%%%%% part.tex %%%%%%%%%%%%%%%%%%%%%%%%%%%%%%%%%%
%
% sample part title
%
% Use this file as a template for your own input.
%
%%%%%%%%%%%%%%%%%%%%%%%% Springer-Verlag %%%%%%%%%%%%%%%%%%%%%%%%%%


\part{1936 -- 1975\\Die Geburt der Makroökonomie: Keynesianismus - Synthese - Monetarismus. Daneben wächst die Mikroökonomie aber weiter}

Die großen Ökonomen in der Mitte des 20. Jahrhunderts waren allesamt geprägt von der "`Great Depression"'. Der Weltwirtschaftskrise, die 1929 mit einem Börsenkrach an der New Yorker Börse, der \textsc{New York Stock Exchange} ihren Ausgang nahm. Es war wohl die erste Krise, die nicht auf kriegerische Auseinandersetzungen oder Umweltkatastrophen, wie Dürren oder Erdbeben, zurückzuführen war, aber dennoch für extremes Leid sorgte. Die Zahlen können nüchtern dargestellt werden: Die Wirtschaftsleistung in den (bis dorthin einzigen) Industrienationen ging um ein Viertel bis ein Drittel zurück. Die Arbeitslosenzahlen stiegen auf teilweise mehr als 25\%. Die persönlichen Schicksale dahinter kann man sich aber kaum vorstellen. Obwohl kein Krieg die Völker aufeinander losgehen ließ und obwohl keine Dürre die Felder verdorren ließ, gab es dennoch Menschen, die ihr Hab und Gut verloren und sogar Hunger litten. Es muss eine schreckliche, seltsam ruhige Zeit gewesen sein. Eine Krise, aber keinen Feind oder offensichtlich Auslöser oder Schuldigen. Dafür viele Menschen ohne Arbeit und ohne Einkommen.
Die Ökonomen hatten dafür ganz verschiedene Erklärungen:
\begin{itemize}
	\item Hayek: Überinvestition in den 1920er Jahren
	\item Keynes: Unterkonsumption in der Krise
	\item Schumpeter: Abwärtsbewegung von drei Wirtschaftszyklen gleichzeitig
	\item Friedman: Sinkende Geldmenge durch Goldstandard
	\item Fisher: Schuldenanstieg durch Deflation (Debt-Deflation)
\end{itemize}

HIER WEITER


Mit \textcite{Keynes1936} wurde nicht nur das Gebiet der Makroökonomie (neu) begründet, sondern das gesamte ökonomische Denken revolutioniert. Die Kontinentaleuropäischen Schulen - allen voran die Österreichische Schule, aber ebenso die Freiburger Schule - lehnten die Ideen von Keynes von Anfang an ab, rutschten aber in der Folge auch immer stärker an den Rand der ökonomischen Wahrnehmung und gelten heute als "`Heterodoxe Schulen"' (vgl. Teil \ref{sec: Heterodox}). Die modernere Kritik am Keynesianismus - in Form des Monetarismus (vgl. Kapitel \ref{Monetarismus}) und später in Form der Neuen klassischen Makroökonomie (vgl. Kapitel \ref{Neue Makro}) - entwickelte sich erst in den späten 1960er Jahren. 

Dennoch wäre es ein Irrtum zu glauben, dass es zwischen 1936 und 1960 nur den Keynesianismus als ökonomische Schule gab! 
Die davor entwickelten Erkenntnisse der Mikroökonomie - insbesondere der Neoklassik - wurden ja nicht per Se als falsch angesehen. Sie reichten nur nicht aus um \textit{gesamt}wirtschaftliche Phänomene zu erklären. Zwar stürzten sich viele junge Ökonomen in weiterer Folge auf die Ideen von Keynes und die Makroökonomie, aber die Mikroökonomie bestand ja als eigene Disziplin weiterhin. Nun zwar nicht mehr als die "`einzige"' ökonomische Theorie, sondern als gleichwertiger Partner neben der Makroökonomie. Dies machte sie als Forschungsgebiet aber nicht weniger wertvoll. Natürlich war die Koexistenz zwischen Makroökonomie und Mikroökonomie zu Beginn Konflikt-behaftet. Insbesondere weil in \textcite{Keynes1936} ja immer wieder auf die Fehler der Neoklassiker, hier insbesondere auf die Arbeiten von Pigou, als den damals führenden Neoklassiker, hinwies. Die Revolution der Ökonomie durch \textcite{Keynes1936} ist unumstritten. Aber die Arbeiten der späteren Neoklassiker - ausgehende von Pigou, hin zu Solow, sowie Arrow und Debreu - bildeten in weiterer Folge unumstritten das zweite Standbein der Ökonomie (vgl. Kapitel \ref{Neoklassik_nach1945}).