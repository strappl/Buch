%%%%%%%%%%%%%%%%%%%%%%%% part.tex %%%%%%%%%%%%%%%%%%%%%%%%%%%%%%%%%%
%
% sample part title
%
% Use this file as a template for your own input.
%
%%%%%%%%%%%%%%%%%%%%%%%% Springer-Verlag %%%%%%%%%%%%%%%%%%%%%%%%%%


\part{1936 -- 1975\\Keynesianismus - Synthese - Monetarismus}

Die großen Ökonomen in der Mitte des 20. Jahrhunderts waren allesamt geprägt von der "`Great Depression"'. Der Weltwirtschaftskrise, die 1929 mit einem Börsenkrach an der New Yorker Börse, der \textsc{New York Stock Exchange} ihren Ausgang nahm. Es war wohl die erste Krise, die nicht auf kriegerische Auseinandersetzungen oder Umweltkatastrophen, wie Dürren oder Erdbeben, zurückzuführen war, aber dennoch für extremes Leid sorgte. Die Zahlen können nüchtern dargestellt werden: Die Wirtschaftsleistung in den (bis dorthin einzigen) Industrienationen ging um ein Viertel bis ein Drittel zurück. Die Arbeitslosenzahlen stiegen auf teilweise mehr als 25\%. Die persönlichen Schicksale dahinter kann man sich aber kaum vorstellen. Obwohl kein Krieg die Völker aufeinander losgehen ließ und obwohl keine Dürre die Felder verdorren ließ, gab es dennoch Menschen, die ihr Hab und Gut verloren und sogar Hunger litten. Es muss eine schreckliche, seltsam ruhige Zeit gewesen sein. Eine Krise, aber keinen Feind oder offensichtlich Auslöser oder Schuldigen. Dafür viele Menschen ohne Arbeit und ohne Einkommen.
Die Ökonomen hatten dafür ganz verschiedene Erklärungen:
\begin{itemize}
	\item Hayek: Überinvestition in den 1920er Jahren
	\item Keynes: Unterkonsumation in der Krise
	\item Schumpeter: Abwärtsbewegung von drei Wirtschafzyklen gleichzeitig
	\item Friedman: Sinkenden Geldmenge durch Goldstandard
	\item Fisher: Schuldenanstieg durch Deflation (Debt-Deflation)
\end{itemize}
