%%%%%%%%%%%%%%%%%%%%% chapter.tex %%%%%%%%%%%%%%%%%%%%%%%%%%%%%%%%%
%
% sample chapter
%
% Use this file as a template for your own input.
%
%%%%%%%%%%%%%%%%%%%%%%%% Springer-Verlag %%%%%%%%%%%%%%%%%%%%%%%%%%

\chapter{Neoliberalismus}
\label{Neoliberalismus}

\section{Den Neoliberalismus gibt es nicht!}


Eine der wohl bekanntesten ökonomischen Schulen ist der Neoliberalismus. Aber was ist eigentlich Neoliberalismus? Die Frage ist durchaus berechtigt und eines vorweggenommen: Sie ist nicht befriedigend zu beantworten!
Olivier Blanchard, einer der berühmtesten Ökonomen der Gegenwart, antwortete auf diese Frage, gestellt von einer österreichischen Zeitung: "`Ich habe keine Ahnung. Aber Sie können mich gern zitieren."' \parencite{Widmann2020} 


Es ist tatsächlich ganz und gar nicht unumstritten was mit Neoliberalismus gemeint ist. Sicher ist aber: Thinktanks und Medien, die eher dem linken Spektrum hinzuzurechnen sind, verwenden den Begriff gerne eher abwertend und behaupten dabei meist der Neoliberalismus sei die neue, führende ökonomische Schule, die wesentlichen Einfluss auf die wirtschaftspolitischen Entscheidungsträger hat. Nun, in den letzten Jahren haben sich in den westlichen Demokratien tatsächlich konservative Kräfte gegenüber den sozialdemokratischen Kräften weitgehend durchgesetzt. Und die konservativen Kräfte stehen wirtschaftspolitisch tatsächlich eher dem wirtschaftsliberalen Flügel nahe. Betrachtet man allerdings empirische Daten, dann muss man objektiv zu dem Ergebnis kommen, dass der "`Neoliberalismus"' nicht die führende ökonomische Schule sein kann: Wie \textcite{Aiginger2016} richtig darstellt sind die Staatsausgaben in \% des BIP zwar seit den 1980er Jahren in der EU nicht mehr wesentlich weiter gestiegen, aber auch nicht gesunken. Und am wichtigsten: Die Staatsausgaben in \% des BIPs belaufen sich in den meisten europäischen Staaten immerhin bei über 40\%. Es ist daher schwer zu argumentieren, der Neoliberalismus sei die führende Denkrichtung der Ökonomie.

Wie \textcite[S. 179]{Venugopal2015} recherchierte findet sich der Begriff "`Neoliberalismus"' in keinem der führenden Lehrbücher. Weder in \textit{Blanchard} noch in \textit{Mankiw} noch bei \textit{Obstfeld}, auch nicht bei \textit{Stiglitz} und \textit{Krugman} findet man ihn. Und auch in den führenden wirtschaftswissenschaftlichen Journals kommt der Begriff praktisch nicht vor \parencite[S. 179]{Venugopal2015}. Jedem Studierenden sei damit abgeraten seine Abschlussarbeit über "`Neoliberalismus"' zu schreiben. Klar ist damit auch, dass man sich auf unsicheres Terrain begibt, wenn man in einer ökonomischen Diskussion auf den Neoliberalismus stürzt. Es findet diesen Begriff schlicht und einfach in den Wirtschaftswissenschaften scheinbar nicht. Aber wie kann das sein, wenn doch außerhalb des rein wissenschaftlichen Diskurses der Begriff ständig verwendet wird?

Da der Begriff -- nennen wir es umgangssprachlich -- so häufig vorkommt, versuchen wir uns dennoch der Annäherung. Und hier findet man schnell einen -- zumindest meiner Ansicht nach -- überzeugenden Ansatz: Der Begriff "`Neoliberalismus"' wird etwas unscharf als Überbegriff für verschiedene ökonomische Schulen, die nach 1945 entstanden, verwendet. Diese sind, erstens, die vierte Generation der \textit{Österreichischen Schule} um \textsc{Friedrich August Hayek}, sowie deren amerikanische Nachfolger um \textsc{Murray Rothbard} und \textsc{Israel Kirzner}. Die zweite wichtige Gruppe, die mitumfasst ist, ist die \textit{Chicago School}, gegründet von \textsc{Frank Knight} und \textsc{Jacob Viner}. Wobei hier zwei Untergruppen genannt werden müssen. Erstens, der "`Monetarismus"' von \textsc{Milton Friedman} und seine Schüler, die "`Chicago Boys"', sowie die "`moderne Finance"', deren geistige Väter zu einem Großteil an der Universität Chicago tätig waren, aber ideologisch nicht unbedingt zum Neoliberalismus zu zählen sind. Drittens ist die "`Lucas-Kritik"' von \textsc{Robert Lucas} und die daraus abgeleitete "`Neue klassische Makroökonommie"' mit \textsc{Robert Joseph Barro} und \textsc{Thomas Sargent}  mitumfasst, wenn man von "`Neoliberalismus"' spricht. Viertens -- wobei hier die Abgrenzung nicht sehr scharf ist -- die Vertreter der "`Neuen Institutionenökonomik"' um \textsc{Oliver Williamson} und \textsc{Gary Stanley Becker}. Und, fünftens, die Interpretation von \textsc{James McGill Buchanan} der  "`Neuen Politische Ökonomie"'.

Die allermeisten der genannten Schulen wurden in diesem Buch bereits erwähnt, aber an sehr verschiedenen Stellen! Das gemeinsame aller dieser Schulen ist gar nicht so leicht zu finden. Klarerweise würde man alle als "`wirtschaftsliberal"' und  "`marktgläubig"' einschätzen. Allerdings ist dies eine eher oberflächliche Einteilung.


Dies drückt auch das Problem des "`Neoliberalismus"' aus: Er ist in dem Sinn keine geschlossene wissenschaftliche Schule. Er ist eher ein ideologischer Begriff, und zwar ein ideologischer Begriff für seine Gegner.
Ein Beispiel zeigt wie uneinheitlich der Begriff ist: \textsc{Milton Friedman} ist eine Ikone der Liberalen. Die einzige Aufgabe eines Unternehmens sei es Profite zu machen, keine Spur von "`Corporate Social Responsibility"'. Der Staat soll sich aus dem Leben der Menschen - auch wirtschaftlich - weitgehend zurückziehen. Aber: In seinem wirtschaftswissenschaftlichen Hauptwerk \textit{A Monetary History of the United States} \parencite{Friedman1962} kritisiert er, dass die US-Notenbank Fed die Geldmenge während der "`Great Depression"' in den 1930er Jahren nicht ausgeweitet hat und somit die Wirtschaftskrise in ihrer vollen Härte erst ermöglicht hat.
\textsc{Friedrich Hayek} hingegen kritisierte jeglichen Eingriff in den freien Markt und war der Ansicht, dass die Zentralbanken privatisiert werden sollten, die Geldversorgung also von privaten Unternehmen durchgeführt werden sollte. Friedman kritisierte Hayek noch 1999 vehement für diese Ansicht: "`I think the Austrian business-cycle theory has done the world a great deal of harm [in the 1930s]."' \parencite{Epstein1999} Manchmal wurde der Monetarismus als dem Keynesianismus näher gesehen als der Österreichischen Schule. Das Verhältnis zwischen den "`Österreichern"' und "`Chicago"'\parencite{Skousen2005} ist nicht so eindeutig positiv, als dass man beide unter den gemeinsamen Hut "`Neoliberalismus"' vereinigen könnte. Man sieht: Innerhalb des Begriffs des "`Neoliberalismus"' lässt sich viel vereinen was bei näherer Betrachtung recht unterschiedlich ist. Für Gegenspieler ist dies natürlich interessant: Mit einem einzigen Begriff umfasst man alle ökonomischen Schulen, deren Inhalte man ablehnt.


Aus heutiger Sicht ist schon alleine der Begriff "`Neoliberalismus"' missglückt. Wie sogar die Mont-Pelerin-Gesellschaft -- auf diese kommen wir später zurück -- in ihrer Gründungserklärung hinweisen muss, kommt es beim Wort "`liberal"' zu Verständnis-Schwierigkeiten. Gemeint ist die europäische Bedeutung von liberal: Maximale persönliche Freiheit und möglichst wenig Staatseingriffe. In der amerikanischen Bedeutung wird das Wort paradoxerweise für tendenziell genau das Gegenteil verwendet (siehe: https://www.montpelerin.org/statement-of-aims/). Das Wortpaar "`links"' und "`rechts"' im europäischen Sinn würde man in den USA am ehesten mit "`liberal"' und "`conservative"' übersetzen. Ich kann mich noch erinnern wie verwirrt ich als junger Student in Wien war als ich 2007 vom Buch "`Conscience of a Liberal"' von Paul Krugman hörte. Ich hatte Krugman immer als eher linken Ökonomen eingeschätzt. Nachdem ich im Buch schmökerte hatte ich den Titel noch eine Zeit lang als ironisch gemeint interpretiert - aus heutiger Sicht peinlich. Erst etwas später wurde ich über das europäisch - amerikanische Missverhältnis des Begriffs "`liberal"' aufgeklärt. "`Liberal"' steht in den USA also für eher linke politische und wirtschaftliche Ansichten, als genau das, was die sogenannten Neoliberalen ablehnen.

Die Entstehungs-Geschichte des Neoliberalismus ist interessant. Der US-amerikanische Journalist Walter Lippmann beschrieb in seinem 1937 erschienenen Buch  "`The Good Society"' \parencite{Lippmann1937} die Zukunft und die Erneuerung des Liberalismus. Und zwar des Liberalismus im politischen Sinn: An vorderster Stelle steht die individuelle Freiheit, diese kann es nur in Demokratien geben.  Das historische Umfeld ist bekannt: Die Weltwirtschaftskrise "`Great Depression"' war zwar weitgehend überwunden, allerdings waren die wirtschaftlichen Spätfolgen noch spürbar und politisch haben in vielen kontinentaleuropäischen Staaten Alleinherrscher die Macht übernommen. Der Liberalismus war zu dieser Zeit also weitgehend gescheitert: Von "`rechts"' fanden nationalistisch-faschistische Regime regen Zulauf, von "`links"' marxistisch-sozialistische Regime. In diesem Umfeld trafen sich Ende August des Jahres 1938 in Paris 26 -- heute würde man sagen Intellektuelle --  zum \textsc{Colloque Lippmann}. Die Teilnehmer waren allesamt überzeugte Liberale. 
Dazu zählten neben \textsc{Friedrich Hayek} und \textsc{Ludwig Mises} auch \textsc{Michael Polanyi, Wilhelm Röpke} sowie unter anderen \textsc{Alexander Rüstow}. Im Umfeld sich ausbreitender rechtsextremer und linksextremer Systeme wurde bei diesem Kolloquium primär hinterfragt was der Liberalismus -- in Anbetracht der düsteren wirtschaftlichen und politischen Situation in Europa -- falsch gemacht habe und wie der Liberalismus wieder federführend werden könnte. Einer der Wortführer war \textsc{Alexander Rüstow}. Unter anderem setzte sich sein Namensvorschlag -- eben "`Neoliberalismus"' -- gegenüber anderen Vorschlägen durch. Wahrscheinlich deshalb wird des \textsc{Colloque Lippmann} als Geburtsstunde des "`Neoliberalismus"' angesehen \parencite{Horn2018}.
Das die individuelle Freiheit zentrales Merkmal auch im Neoliberalismus bleiben musste war klar. Wie die Ökonomie dieser Schule ausgestaltet werden sollte, war aber schon bei seiner Geburtsstunde höchst umstritten:  Zwar vertraten \textsc{Friedrich Hayek} und \textsc{Ludwig Mises} natürlich schon beim Colloque Lippmann ihre wirtschaftsliberalen Thesen, aber es gab auch deutlichen Widerspruch. \textsc{Alexander Rüstow} und auch \textsc{Wilhelm Röpke} forderten etwa -- aus heutiger Sicht widersprüchlich wenn man von "`Neoliberalismus"' spricht -- einen "`starken Staat"'. Wirtschaftsliberal sollte in ihrem Sinn heißen, dass zwar tatsächlich die erbrachte Leistung in einem Umfeld individueller Freiheit über den persönlichen Erfolg entscheiden sollte, die \textit{Gestaltung des Umfelds} sollte aber durch einen starken Staat gesichert werden. So sollte es demnach ausgeprägte staatliche Institutionen geben. Insbesondere Marktmacht durch Monopolbildung \parencite[S. 69ff]{Hegner2000}, aber auch Vermögensbildung durch Vererbung \parencite[S. 58ff]{Hegner2000} großer Besitztümer sollten laut Rüstow verhindert werden. 

Die Unstimmigkeiten bezüglich der wirtschaftlichen Ausrichtung des Neoliberalismus gab es also schon von Beginn an. Offensichtlicher wurde dies beim zweiten wichtigen Event in der Geschichte des Neoliberalismus: Der Gründung der Mont Pèlerin Gesellschaft im Jahr 1947. Meines Erachtens nach wurde der "`Neoliberalismus"' -- zumindest in der Form wie er heute oft verstanden wird, nämlich eben als ökonomische Schule -- bei diesem Treffen in den schweizerischen Alpen begründet. Betrachtet man die Liste der Teilnehmer, sieht man, dass jetzt eben schon der \textsc{wirtschaftliche} Liberalismus im Vordergrund stand: Von den 39 Teilnehmern waren diesmal die meisten Ökonomen, unter anderem Friedrich Hayek, Milton Friedman, Ludwig von Mises, Walter Eucken, Wilhelm Röpke, George Stigler und Maurice Allais \parencite[S. 12--19]{Mirowski2009}. Letztgenannter war übrigens der einzige, der die bereits erwähnte Gründungserklärung "`(Statement of Aims"'), die im Rahmen des Mont-Pelerin-Treffens verfasst wurde, nicht unterzeichnete\parencite[S. 57]{Mirowski2009}. Auch abgesehen davon kam es auch während des Treffens rasch zu Unstimmigkeiten bezüglich der Ausrichtung. 

Bis hierher kann man durchaus vom Neoliberalismus als eine einheitliche, ökonomische Schule sprechen. Die erste Spaltung vollzog sich aber schon in den 1960er Jahren. Innerhalb der Mont-Pelerin-Gesellschaft tat sich schon bald ein Spalt auf zwischen den deutschen Vertretern um Wilhelm Röpke und Alexander Rüstow auf der einen Seite und den Vertretern der österreichischen Schule -- vor allem Friedrich Hayek und Ludwig von Mises -- sowie der aufstrebenden Chicago School -- damals vor allem noch um Frank Knight auf der anderen Seite. Die deutschen \textsc{Ordoliberalen} hatten zwar beim Colloque Lippmann und auch bei der Gründung der Mont-Pelerin Gesellschaft einen wesentlichen Einfluss, dieser schwand aber zunehmend. Die Gegensätze zur Österreichischen Schule waren die gleichen wie schon beim Colloque Lippmann. Der Ordoliberalismus wollte einen Staat, in dem die Spielregeln für den Markt klar definiert und durch Institutionen kontrolliert wurden. Die deutschen Vertreter hatten innerhalb der Mont-Pelerin Gesellschaft den Vorteil tatsächlich politischen Einfluss zu haben. Walter Eucken, Gründungsmitglied der Mont Pelerin Gesellschaft, galt als Vater der "`Sozialen Marktwirtschaft"', der erfolgreichen Wirtschaftsordnung des Nachkriegsdeutschland, er stand zeitlebens Friedrich Hayek sehr nah, wenn auch die inhaltlichen Übereinstimmungen zwischen den beiden nicht besonders hoch waren. Eucken starb bereits 1950. Innerhalb des Ordoliberalismus übernahm Alfred Müller-Armack die zentrale Stelle. Auch Röpke und Rüstow waren dieser Strömung zuzurechnen. Der starke Staat im Ordoliberalismus, aber auch aufkommende nationalistische Ideen (siehe unten) im Denken Wilhelm Röpkes, waren mit den Überzeugungen der Österreicher und Amerikaner innerhalb der Mont-Pelerin Gesellschaft nicht vereinbar. Wilhelm Röpke wurde zwar 1961 noch der erste Präsident der Mont Pelerin Gesellschaft nach Gründungspräsidenten Friedrich Hayek, aber kurz darauf wurden die Differenzen zwischen ihm und Hayek unüberwindbar und im Rahmen der sogenannten Hunold-Affäre trat Röpke aus der Mont Pelerin Gesellschaft aus. Die Ordoliberalen Ideen hatten in weiterer Folge keine Zukunft mehr innerhalb der Mont Pelerin Gesellschaft. Die erste Abspaltung im Neoliberalismus war vollzogen. Bis heute wird der Ordoliberalismus -- praktisch äquivalent mit der \textsc{Freiburger Schule} -- häufig auch als Neoliberalismus bezeichnet.

In der Mont-Pelerin Gesellschaft gab es ab Mitte der 1960er Jahre noch zwei zentrale Strömungen: Die Österreichische Schule um Hayek und die aufstrebende Chicago School. Was waren die zentralen Themen zu dieser Zeit? Erstens, politisch die "`Zweiteilung"' der Welt in den kapitalistischen Westen und den sozialistischen Osten. Der Kommunismus mit seinem Kollektivismus und zentraler Planwirtschaft war der gemeinsame Gegner der Mont-Pelerin-Gesellschaft. Zweitens, wirtschaftlich der Aufstieg des Keynesianismus. Die neoklassische Synthese als Kombination der Ideen der Neoklassik mit jenen von Keynes -- häufig als Keynesianismus bezeichnet -- übernahm rasch die Rolle der Mainstream-Ökonomie. Auch das war ein gemeinsamer Gegner. Die Mont-Pelerin-Gesellschaft sah sich geradezu als Gegenpol zu den Keynesianern. Drittens, aus ökonomischer Sicht hatte die Schaffung eines zentralen Währungssystems nach dem Zweiten Weltkrieg zentrale Bedeutung. Bis zur "`Great Depression"' galt der klassische Goldstandard als der Goldstandard der Währungssysteme ;-). Während der Krise konnte das System aber nicht mehr aufrecht gehalten werden und wurde nach und nach von den verschiedenen Staaten aufgekündigt. Aus heutiger Sicht ist das überraschend, aber eine Welt ohne fixe Wechselkurse zwischen den Währungen galt damals als unvorstellbar. Und so wurde noch während des Zweiten Weltkriegs im Jahr 1944 in der US-amerikanischen Stadt Bretton Woods das gleichnamige Währungssystem geschaffen. Ebenfalls ein auf Gold basierendes Wechselkurssystem, allerdings mit dem US-Dollar als Referenzwährung. In Bezug auf das internationale Währungssystem waren sich die Neoliberalen von Anfang an uneinig. Zwar lehnten alle das \textsc{Bretton-Woods-System} aufgrund der darin notwendigen, aber illiberalen Kapitalverkehrskontrollen ab, allerdings war man sich über Alternativen uneinig.
Die Vertreter der österreichischen Schule, insbesondere Mises, waren Verfechter des klassischen Goldstandards. Auch Hayek plädierte zunächst für den Goldstandard, als "`zweitbeste Lösung"'. Er wendet sich später der "`Privatisierung der Geldversorgung"' zu. Beide -- Mises und Hayek -- aber bleiben Zeit ihres Lebens Verfechter fixer Wechselkurse \parencite[S. 256]{Kolev2017} Die Chicago School hingegen hat kein Problem mit den flexiblen Wechselkursen. Im Gegenteil, Friedman bestärkte die US-Regierung in der Aufkündigung des "`Bretto-Woods-Vertrags"' und prophezeite eine stabilisierende Wirkung flexibler Wechselkurse. Den "`Goldstandard"' lehnte die Chicago-School strikt ab. Die Monetaristen um Milton Friedman waren überzeugt von der Überlegenheit von \textsc{Fiat-Geld} und unabhängigen Zentralbanken, die idealerweise für eine stabil wachsende Geldversorgung sorgen sollten.
Mit dem Ende des "`Bretton-Woods-Systems"' Anfang der 1970er Jahre nahm die Bedeutung der Geldpolitik zu. Die Unterschiede zwischen "`Chicago-School"' und "`Austrians"' wurden damit unübersehbar. Beide zusammenzufassen als "`Neoliberalismus"' ist unscharf. Die österreichische Schule war in Belang auf die Geldpolitik zu idealistisch. Jede Geldmengenerhöhung, so die "`Austrians"' bedeutet Inflation. Immer wenn der tatsächliche Zinssatz unter den natürlichen Zinssatz (im Sinne Wicksells) liegt, kommt es zu einem Boom der schlussendlich in einer Krise enden muss. Das klingt in der Theorie überzeugend, ist aber wirtschaftspolitisch nicht umsetzbar. Denn wie hoch ist der natürliche Zinssatz heute? Wie argumentiert man gegenüber dem unter Arbeitslosigkeit leidenden Wahlvolk, wenn man in einer Wirtschaftskrise nicht lenkend eingreift, sondern stattdessen auf "`Gesundschrumpfen"' setzt? Auch in Bezug auf die Methodik spielte die Zeit in den 1970er Jahren gegen die österreichische Schule: Quantitative Methoden, die mittels historischer Daten empirisch überprüft werden, werden von den "`Austrians"' abgelehnt. Genau diese Ansätze aber kamen in den 1970er Jahren in Mode, nicht zuletzt wegen der steigenden Verfügbarkeit empirischer Daten. 
Zwar schrieb \textcite[S. 102]{Kirzner1967}, dass man die Unterschiede zwischen der "`Österreichischen Schule"' und der "`Chicago-School"' nicht überschätzen soll, schließlich stimmen beide in den wesentlichen Fragen wirtschaftlichen Fragen überein, aber mit fortlaufender Zeit wurden die Unterschiede deutlicher. Wenn man so will kann man Anfang der 1970er Jahre von einer zweiten Spaltung im Neoliberalismus sprechen. Wobei die Österreichische Schule in Wirklichkeit in den 1970er Jahren in der Bedeutungslosigkeit versank. Mises starb 1973. \textsc{Hayek, Gottfried Haberler, Fritz Machlup} und \textit{Oskar Morgenstern} wurden alle um die Jahrhundertwende geboren und hatten ihren schöpferischen Zenit überschritten. Nur Hayek erlebte just in dieser Zeit seinen zweiten Frühling. Überraschend wurde er 1974 mit dem Nobelpreis für Wirtschaftswissenschaften ausgezeichnet. Aus wissenschaftlicher Sicht aber blieb im Neoliberalismus ab Anfang der 1970er Jahre ausschließlich die Chicago-School übrig. In dieser kurzen Phase -- zwischen Ölpreisschock 1973 und Lucas-Kritik 1976 -- war die Chicago-School die einzige wesentliche neoliberale Schule und gleichzeitig aus wissenschaftlicher Sicht auf ihrem Zenit.
Aus wirtschaftspolitischer Sicht folgte die Hochzeit des Neoliberalismus ab Anfang der 1980er Jahre. Hayek -- wie soeben erwähnt -- gewann, als seine Karriere als immerhin 75-jähriger schon am abklingen war, noch einmal wirtschaftspolitischen Einfluss. \textit{Margaret Thatcher} zog ihn als wirtschaftspolitischen Berater in Großbritannien heran und auch -- natürlich wesentlich umstrittener -- \textit{Augusto Pinochet} in Chile. Milton Friedman hatte schon bei der Abschaffung fixer Wechselkurse entscheidenden Einfluss. Er spielte auch als \textit{Ronald Reagan's} Berater in den USA eine entscheidende beratende Rolle. Die Zeit von "`Thatcherism"' und "`Reaganomics"'  gelten heute aus politischer Sicht als Hochzeit des Neoliberlismus. Nachhaltiger -- und wissenschaftlich fundierter -- war aber Friedman's Einfluss auf die Zentralbanken:  Die Deutsche Bundesbank etwa bezog sich direkt auf seinen Monetarismus und richtete ihre Politik ab 1975 an der Geldmengensteuerung aus (von 1975 bis 1987 an der Steuerung der Zentralbankengeldmenge) \parencite[S. 36]{BBK2016}. 




Der Keynesianismus galt durch die Stagflation in Folge des Ölpreisschocks aus der Mode. Die Lucas-Kritik wurde erst 1976 veröffentlicht.



Man sieht also: Ursprünglich hat der "`Neoliberalismus"' durchaus eine gemeinsame Wurzel, die auch eindeutig zurückgeführt werden kann (nämlich auf das Mont Pelerin Treffen, oder das Colloque Lippmann). Warum er heute dennoch nicht mehr im wissenschaftlichen Diskurs als Begriff verwendet wird sieht man aber auch am eben ausgeführten: Von Anfang an herrschte innerhalb der Teilnehmer Uneinigkeit über die wirtschaftliche Ausprägung des Neoliberalismus. Als erstes spaltete sich die deutsche Gruppe als Ordoliberalismus ab. Später kam es zu deutlichen Unterschieden zwischen der österreichischen Schule und der Chicago School. Bis in die Gegenwart sind die Unterschiede in den Ideen der Vertreter marktliberaler Ideen zu unterschiedlich als dass man tatsächlich von einer geschlossenen Schule des Neoliberalismus sprechen kann.



Der Neoliberalismus und die Mont-Pelerin-Gesellschaft im Speziellen wird in vielen Publikationen meines Erachtens als zu einflussreich beschrieben. Die hohe Anzahl der Nobelpreisträger, die auch Mitglied der Mont-Pelerin-Gesellschaft sind, sei darauf zurückzuführen, dass diese Gruppe indirekt auf die Vergabe Einfluss nehme, liest man hier. Oder, dass die führende Rolle des Keynesianismus durch die Mont-Pelerin-Gesellschaft untergraben wurde und die politischen Akteure auf die Seite des Neoliberalismus gezogen wurden, der noch dazu immer marktradikaler wurde. Vieles, dass man hier liest und im TV sieht mutet eher übertrieben -- ja fast wie Verschwörungstheorien -- an. Eines stimmt aber zweifellos: Es ist den Personen um Friedrich Hayek -- schlussendlich die führende Person in der Mont Pelerin Gesellschaft, wenn nicht im gesamten Neoliberalismus -- gelungen, ihn zu einem führenden Ökonomen hochzustilisieren. Fast schon legendär sind die "`Rap-Battles"' zwischen Keynes und Hayek. In Wirklichkeit ist der wirtschafts-wissenschaftliche Beitrag Hayeks nicht zu vergleichen mit jenem von beispielsweise John Maynard Keynes. Während seiner wissenschaftlichen Karriere war er kein führender Kopf an einer der führenden Universitäten. Zwar lehrte er an der damals extrem einflussreichen Universität Chicago, dort waren aber andere, allen voran Milton Friedman, die Taktgeber. Auch innerhalb der österreichischen Schule gilt nicht Hayek, sondern vielmehr Mises als der Hauptvertreter. Dessen wissenschaftliches Werk hatte auch nachhaltigeren Eindruck in der ökonomischen Gesellschaft. Und nicht zuletzt musste Hayek selbst eingestehen, dass seine Krisenbekämpfungstheorie zur "`Great Depression"', nämlich am besten nicht zu intervenieren, einfach falsch war. Die Ökonomie des 20. Jahrhunderts als einen Zweikampf zwischen Keynes und Hayek darzustellen, stellt dessen Einfluss sehr übertrieben dar.







\section{Ordoliberalismus}

Eucken und Röpke
Der Begriff "`Neoliberalismus"' bezog sich ursprünglich auf die deutsche ...
Die ursprüngliche Bedeutung -- und wie wir gesehen haben -- auch die einzige 
Freiburger Schule (dort oft auch Hayek dazugezählt)


Sehr pragmatischer Zugang: GmbH kritisch gesehen, weil das Risiko einseitig beschränkt. Für Erbschaftssteuern, weil eben \textit{wirklich} an Chancengleichheit interessiert. Im Zentrum stand auch eine Anti-Kartell-Ordnung.
Also: Marktorientiert und gegen den Keynesianismus. Schlanker Staat, aber auch starke Regulierungen!
Nicht formalisiert. 

Heutiger Vertreter zum Teil: Hans-Werner Sinn wenig formalisiert, sehr pragmatischer realistischer Zugang

