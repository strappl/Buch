%%%%%%%%%%%%%%%%%%%%% chapter.tex %%%%%%%%%%%%%%%%%%%%%%%%%%%%%%%%%
%
% sample chapter
%
% Use this file as a template for your own input.
%
%%%%%%%%%%%%%%%%%%%%%%%% Springer-Verlag %%%%%%%%%%%%%%%%%%%%%%%%%%

\chapter{Neu-Keynesianismus} \label{cha: Neu Keynes}

Der Neu-Keynesianismus ist leider (noch) wesentlich schwieriger von anderen Schulen abzugrenzen als etwa der Keynesianismus oder der Monetarismus. Dies gilt sowohl in inhaltlicher Sicht, also auch in zeitlicher Sicht. Inhaltlich lässt sich der Neu-Keynesianismus am ehesten negativ abgrenzen. Einige Ökonomen erkannten, dass die Theorien der Keynesianer nicht mehr zureichend waren. Sie akzeptierten aber auch nicht die starren Annahmen der "`Neuen Klassiker"', diese Ökonomen könnte man "`Neu-Keynesianer"' nennen. Die "`Neu-Keynesianer"' sind dementsprechend keine geschlossene Gruppierung von Ökonomen, sondern behandelten eher zerstreut einzelne brennende Fragen der Ökonomie. Dazu gehörten zum Beispiel Fragen der Inflation (Phillips-Kurve), der Arbeitslosigkeit (Suchproblem, Natürliche Arbeitslosigkeit) und des Marktversagens (Informationsasymmetrien, Natürliche Monopole). 

Auch was die Personen betrifft ist die Abgrenzung schwieriger. So muss der Erzkonservative \textsc{John Taylor}, der einer der Präsidenten der Mont Pelerin Society war, inhaltlich zweifelsohne als ein früher Vertreter des Neu-Keynesianismus gesehen werden. 

Die schwierigste Abgrenzung erfolgt aber in zeitlicher Hinsicht. Schließlich wird die heutige Mainstream-Ökonomie häufig als "`Neu-Keynesianismus"' bezeichnet. In dieser Logik müsste man den "`Neu-Keynesianismus"' zumindest in zwei Generationen teilen. Zweifelsohne beginnt der "`Neu-Keynesianismus"' nämlich als Antwort auf die "`Neue Klassische Makroökonomie"' - in Wahrheit  sogar schon etwas früher, nämlich mit der Kritik an der Phillips-Kurve ab Mitte der 1960er Jahre - zu existieren. Dies ist der Neu-Keynesianismus, der an dieser Stelle beschrieben wird und bis etwa Ende der 1980er Jahren existierte. Die Hauptproponenten sind hier \textit{Edmund Phelps, Peter Diamond, Joseph Stiglitz, George Akerlof und William Baumol}. Diese Schule war der krasse Gegenpol zur aufstrebenden "`Neuen Klassischen Makroökonomie"'.
Ab Anfang der 1990er Jahre kam es zunehmend zu einer Verschmelzung der beiden Schulen. Diese wird hier als "`Neue Neoklassische Synthese"' beschrieben. Sie wird aber häufig auch einfach "`Neu-Keynesianismus"' genannt\footnote{Man könnte sie auch Zweite Generation des Neu-Keynesianismus nennen}. die Hauptvertreter sind hier \textsc{John Taylor}\footnote{der aber eigentlich auch zur ersten Generation der Neu-Keynesianiser gezählt werden muss} \textsc{David Romer, Greg Mankiw, Blanchard und Paul Krugman}.

Der Name "`Neu-Keynesianismus"' täuscht auf jeden Fall: Er ist nicht etwa eine Weiterentwicklung des Keynesianismus. Schon die hier beschriebene "`Erste Generation der Neu-Keynesianer"' orientierte sich inhaltlich und methodologisch eher an den "`Neuen Klassikern"', bestand aber auf den Bedeutungen der Geld- und Fiskalpolitik, sowie der Existenz von Marktversagen. James Tobin zum Beispiel bestand darauf ein "`Alt-Keynesianer"', nicht  "`Neu-Keynesianer"' zu sein \parencite[S. 45ff]{Tobin1993}. 
Edmund Phelps drückte dies so aus: "`I [had] warm personal relations with Jim [James] Tobin and Bob [Robert] Solow as well as with Bob [Robert] Lucas and Tom [Thomas] Sargent – relations that have survived our differences. But I belonged to neither school."'

Die spätere Form des Neu-Keynesianismus ("`Neue Neoklassische Synthese"') behielt vom Keynesianismus nur die Rigidität von Preisen und Löhnen, meinte etwa Greg Mankiw.
 

\section{Phelps: Mikrofoundation der Makroökonomie}
\label{micmac}

Man findet wohl kaum einen Namen, der den Übergang von "`Keynesianismus"' zu "`Neu-Keynesianismus"' besser repräsentiert als \textsc{Edmund Phelps}. Ökonomisch "`erwachsen"' geworden ist er im eindeutig keynesianischen Umfeld: Er verfasste bei James Tobin seine Dissertation und arbeitete Mitte der 1960er Jahre mit Robert Solow, Paul Samuelson und Franco Modigliani zusammen. Also alles eindeutig keynesianische Ökonomen, die wir aus Kapitel \ref{Synthese} kennen. Laut seines autobiografischen Artikels \textcite[S. 93]{Heertje1995} war diese Zeit, inklusive Gastprofessur am Massachusetts Institute of Technologie (MIT), die prägendste und er selbst innerhalb weniger Jahre ein international anerkannter Ökonom. Schon 1961 veröffentlichte er sein erstes bedeutendes Werk: \textit{The Golden Rule of Accumulation} \parencite{Phelps1961}. Ein bemerkenswerter Artikel, den der gerade mal 28-jährige Phelps im American Economic Review veröffentlicht. Gerade einmal sieben Seiten lang beginnt dieser - so wie im Englischen normalerweise Märchen  - mit "`Once upon a time"' und in weiterer Folge wechseln sich mathematische Formeln mit Dialogen zwischen "`König und dem Volk der Solovians"' ab unter anderem: \textit{[Oiko Nomos (!) said:] "`And now, if these concepts are clear and my assumptions granted, I wish to introduce the following lemma." "A lemma, a lemma," the crowd shouted} \parencite[S. 640]{Phelps1961}. So witzig und amüsant die Geschichte des Artikels, so bahnbrechend ist auch deren Inhalt. Diese Arbeit kann als direkter Anschluss an die Wachstumstheorie Solow's gesehen werden und im Zentrum steht folgende hypothetische Überlegung: Wenn die gesamte aktuelle Wirtschaftsleistung für die Investition (Investition = Sparen!) in neue Produktionsgüter verwendet wird, dann wird nichts für den aktuellen Konsum ausgegeben, was einer Wirtschaftsleistung von Null entspricht. Wird hingegen die gesamte aktuelle Wirtschaftsleistung für Konsum verwendet, werden keine neuen Investitionen getätigt. Sobald alle Produktionsgüter veraltet sind resultiert auch dies in einer Wirtschaftsleistung von Null. Das heißt aber auch, dass dazwischen irgendein optimales Verhältnis zwischen Sparen/Investieren auf der einen Seite und Konsumieren auf der anderen Seite bestehen muss. Dieses erreicht man eben durch \textit{The Golden Rule of Accumulation}. Diese wird erreicht - solange man einige vereinfachenden Annahmen zulässt - wenn die Wachstumsrate des BIPs dem Zinssatz entspricht. Bereits Phelps nannte diese natürliche Wachstumsrate "`nachhaltig"' \parencite[S. 638]{Phelps1961}. Weiters zeigt Phelps formal, dass diese Wachstumsrate erzielt wird, wenn die Summe der Investitionen der Summe der Profite entspricht, also alle Profite investiert werden. Umgekehrt werden im Optimum alle Löhne konsumiert. Zusammengefasst: Wenn alle Löhne konsumiert werden und alle Profite investiert werden, befindet sich die Ökonomie auf einem nachhaltigen Wachstumspfad. Die Wachstumsrate entspricht hierbei dem Zinssatz. Insgesamt erinnert das Ergebnis an die Arbeiten von Wicksell und Hayek. Die formale Herleitung durch Phelps war aber zu diesem Zeitpunkt - im Jahre 1961 - eine bahnbrechende Erweiterung des Solow-Wachstumsmodells.

Das bisher in diesem Unterkapitel dargestellt entspricht noch vollständig dem keynesianischem Denken aus Kapitel \ref{Synthese}. Im Jahr 1966 wechselte Phelps von Yale auf die University of Pennsylvania. Damit konzentrierte sich seine Arbeit auf neue Themen, nämlich auf die theoretische Fundierung der Phillipskurve. 
Seine Arbeiten dazu sollten später die ersten Zweifel in das dominierende, keynesiansche Framework begründen. Im Nachhinein kann man getröst sagen, dass damit die Grundlagen für den "`Neu-Keynesianismus"' geschaffen wurden und damit - neben "`Monetarismus"', "`Neuer Klassischer Makroökonomie"' und "`Supply-Side-Economics"' - die vierte Front beim Angriff auf den Keynesianismus aufgemacht wurde.

Dies begann mit Phelps' Arbeiten zur Gültigkeit der Phillipskurve. Wie in Kapitel \ref{sec: Phillips} dargestellt, war der vermeintliche, negative Zusammenhang zwischen Inflation und Arbeitslosigkeit zwar nicht Bestandteil der ursprünglichen keynesianischen Theorie. Aber in weiterer Folge vor allem in der keynesianischen Wirtschaftspolitik ein fixer Bestandteil. Unabhängig voneinander waren Milton Friedman und eben Edmund Phelps bereits ab Mitte der 1960er Jahre die ersten Ökonomen, die den Zusammenhang zwischen Inflation und Arbeitslosigkeit in Frage stellten. Wohlgemerkt zu einer Zeit, in der der Zusammenhang empirisch noch recht gut beobachtet werden konnte. Das in den 1970er Jahren diese Korrelation weitgehend verschwand gab den Kritiker Friedman und Phelps natürlich gehörig Auftrieb. Phelps hatte seine Kritik dabei - im Gegensatz zu Friedman - formal sauber unterlegt. 

Zwei wesentliche Artikel:

Artikel: \textcite{Phelps1967}, \textcite{Phelps1968}

HIER WEITER: 
https://www.nobelprize.org/prizes/economic-sciences/2006/phelps/biographical/  : Hier: As I began my gradual departure from Penn I th

http://www.columbia.edu/~esp2/autobio1.pdf (Seite 94)
http://www.columbia.edu/~esp2/

Hierfür später den Nobelpreis 2006:
Bahnbrechende Teile darin:
Mikrofundierung der Makro
und Einführung der adaptiven Erwartungen (wenig später rationale Erwartungen durch die Neue Makro)



In den 1970er Jahren schließlich begründete er - als Antwort auf die Neue Klassische Makroökonomie - einen der wesentlichen Punkte der Neu-Keynesianer (mit Taylor und Calvo) unter anderem: "`NAIRU"'


ALT HIER:
dennoch leistete er, gemeinsam mit Milton Friedman die ersten Ideen zur Mikrofundierung der Makroökonomie, die später einer der zentralen Punkte der "`Neuen Klassik"' werden sollte. Er stellte den Zusammenhang von Inflation und Arbeitslosigkeit in Frage mit dem formalen Argument, dass eine Realgröße wie die Arbeitslosigkeit nicht systematisch mit einer Nominalgröße wie der Inflation korrelieren könne. 

Stattdessen ist die \textit{erwartete} Inflation von entscheidender Bedeutung:  "`expectations-augmented Phillips 
curve"' The intertemporal perspective implies that current inflation expectations affect the future 
tradeoff between inflation and unemployment. A higher current inflation rate typically leads to 
higher inflation expectations in the future, so that it then becomes more difficult to achieve the 
objectives of stabilization policy

A key result was that the long-run rate of unemployment cannot be influenced 
by monetary or fiscal policy affecting aggregate demand. Phelps’s analysis thus identified 
important limitations on what demand-management policy can achieve. This view has become 
predominant among macroeconomic researchers as well as policymakers. As a result, 
macroeconomic policy is carried out very differently today from what it was forty years ago.

Dieses Ergebnis war natürlich "`schockierend"' für die Keynesianer. 
ALT BIS HIER





Zusammengefasst:

Erster Schritt: Phelps arbeitete zunächst in der Tradition der Keynesianer und entwickelte die "`Goldene Regel der Akkumumlation"'

Zweiter Schritt: Die Emanzipation vom Keynesianismus erfolgte mit seinen Arbeiten zur Revolution der Phillipskurve. Diese umfassten nämlich Konzepte, die bis heute in der Mainstream-Ökonomie State-of-the-Art sind. Erstens, war er ein Vorreiter bei der Mikrofundierung der Makroökonomie und zweitens, etablierte er adaptive "`Erwartungen"' in die Modelle der Ökonomie. Beides spielte später bei den "`Neuen Klassikern"' eine wesentliche Rolle, wenn auch in der Form der \textit{rationalen} statt der \textit{adaptiven} Erwartungen. 

Dritter Schritt: Gegenbewegung zur Kritik der "`Neuen Klassiker"' inklusive der Entwicklung der "`Natürlichen Arbeitslosigkeit"'

\section{Marktversagen als Teil der Ökonomie}
\label{Marktversagen}

\subsection{Informationsasymmetrie: Spence, Stiglitz und Akerlof}


\subsection{Natürliche Monopole oder Baumol's angreifbare Märkte}


\section{Arbeitslosigkeit als Suchproblem}
\label{Suchtheorie}

\subsection{Diamond, Mortensen und Pisaridis}

Widersprach rasch der "`Neuen Klassik"' indem sie die natürliche Arbeitslosigkeit erweiterte.




