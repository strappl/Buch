%%%%%%%%%%%%%%%%%%%%% chapter.tex %%%%%%%%%%%%%%%%%%%%%%%%%%%%%%%%%
%
% sample chapter
%
% Use this file as a template for your own input.
%
%%%%%%%%%%%%%%%%%%%%%%%% Springer-Verlag %%%%%%%%%%%%%%%%%%%%%%%%%%

\chapter{Neu Keynesianismus} \label{cha: Neu Keynes}

\section{Phelps: Mikrofoundation der Makroökonomie}
\label{micmac}

Edmund Phelps ist selbst nicht der "`Neuen Klassischen Makroökonomie"' zuzuordnen, dennoch leistete er, gemeinsam mit Milton Friedman die ersten Ideen zur Mikrofundierung der Makroökonomie, die später einer der zentralen Punkte der "`Neuen Klassik"' werden sollte. Er stellte den Zusammenhang von Inflation und Arbeitslosigkeit in Frage mit dem formalen Argument, dass eine Realgröße wie die Arbeitslosigkeit nicht systematisch mit einer Nominalgröße wie der Inflation korrelieren könne. 

Stattdessen ist die \textit{erwartete} Inflation von entscheidender Bedeutung:  "`expectations-augmented Phillips 
curve"' The intertemporal perspective implies that current inflation expectations affect the future 
tradeoff between inflation and unemployment. A higher current inflation rate typically leads to 
higher inflation expectations in the future, so that it then becomes more difficult to achieve the 
objectives of stabilization policy

A key result was that the long-run rate of unemployment cannot be influenced 
by monetary or fiscal policy affecting aggregate demand. Phelps’s analysis thus identified 
important limitations on what demand-management policy can achieve. This view has become 
predominant among macroeconomic researchers as well as policymakers. As a result, 
macroeconomic policy is carried out very differently today from what it was forty years ago.

Dieses Ergebnis war natürlich "`schockierend"' für die Keynesianer. 




\section{Marktversagen als Teil der Ökonomie}
\label{Marktversagen}

\subsection{Informationsasymmetrie: Spence, Stiglitz und Akerlof}


\subsection{Natürliche Monopole oder Baumol's angreifbare Märkte}


\section{Arbeitslosigkeit als Suchproblem}
\label{Suchtheorie}

\subsection{Diamond, Mortensen und Pisaridis}

Widersprach rasch der "`Neuen Klassik"' indem sie die natürliche Arbeitslosigkeit erweiterte.




