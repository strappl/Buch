%%%%%%%%%%%%%%%%%%%%% chapter.tex %%%%%%%%%%%%%%%%%%%%%%%%%%%%%%%%%
%
% sample chapter
%
% Use this file as a template for your own input.
%
%%%%%%%%%%%%%%%%%%%%%%%% Springer-Verlag %%%%%%%%%%%%%%%%%%%%%%%%%%

\chapter{Neu-Keynesianismus} \label{cha: Neu Keynes}

Der Neu-Keynesianismus ist leider (noch) wesentlich schwieriger von anderen Schulen abzugrenzen als etwa der Keynesianismus oder der Monetarismus. Dies gilt sowohl in inhaltlicher Sicht, also auch in zeitlicher Sicht. Inhaltlich lässt sich der Neu-Keynesianismus am ehesten negativ abgrenzen. Einige Ökonomen erkannten, dass die Theorien der Keynesianer nicht mehr zureichend waren. Sie akzeptierten aber auch nicht die starren Annahmen der "`Neuen Klassiker"', diese Ökonomen könnte man "`Neu-Keynesianer"' nennen. Die "`Neu-Keynesianer"' sind dementsprechend keine geschlossene Gruppierung von Ökonomen, sondern behandelten eher zerstreut einzelne brennende Fragen der Ökonomie. Dazu gehörten zum Beispiel Fragen der Inflation (Phillips-Kurve), der Arbeitslosigkeit (Suchproblem, Natürliche Arbeitslosigkeit) und des Marktversagens (Informationsasymmetrien, Natürliche Monopole). 
Auch was die Personen betrifft ist die Abgrenzung schwieriger. So muss der Erzliberale \textsc{John Taylor} - Präsidenten der Mont Pelerin Society von 2018 - 2020 - inhaltlich zweifelsohne als ein früher Vertreter des Neu-Keynesianismus gesehen werden. 

Positive Definition:
- Mikrofundierung (Abgrenzung zu Keynes)
- Rationale Erwartungen (Abgrenzung zu Keynes)
- Rigiditäten auf Märkten --> Wirksamkeit der Geldpolitik (Abgrenzung zur Neu Klassik)
- Monopolistische Konkurrenz, Imperfekte Märkte (Abgrenzung zur Neu Klassik)


Die schwierigste Abgrenzung erfolgt aber in zeitlicher Hinsicht. Schließlich wird die heutige Mainstream-Ökonomie häufig als "`Neu-Keynesianismus"' bezeichnet. In dieser Logik müsste man den "`Neu-Keynesianismus"' zumindest in zwei Generationen teilen. Zweifelsohne beginnt der "`Neu-Keynesianismus"' nämlich als Antwort auf die "`Neue Klassische Makroökonomie"' ab den frühen 1980er Jahren zu existieren. In Wahrheit sogar schon etwas früher, nämlich mit der Kritik an der Phillips-Kurve ab Mitte der 1960er Jahre. Dieser "`frühe"' Neu-Keynesianismus wird an dieser Stelle beschrieben und dauerte bis etwa Ende der 1980er Jahren. Die Hauptproponenten sind hier \textit{Edmund Phelps, Peter Diamond, Joseph Stiglitz, George Akerlof und William Baumol}. Diese Schule war der Gegenpol zur aufstrebenden "`Neuen Klassischen Makroökonomie"' um Lucas, Sargeant und Barro. Die Vertreter dieses frühen Neu-Keynesianismus liefern mit ihren Arbeiten vor allem "`Aufweichungen"' der zu starren Annahmen der "`Neuen Klassiker"'. Sie lehnen in diesem Sinn die Arbeiten der "`Neuen Klassiker"' ab, akzeptieren aber auch, dass der Keynesianismus veraltet ist. Von der Zuordnung der Personen her entwickelte sich der "`Neu-Keynesianismus"' eher aus den Salzwasser-Universitäten (vgl. die entsprechende Einteilung in Kapitel \ref{Neue Makro}), die die Neuen Klassiker ja strikt - und nicht nur auf inhaltlicher Ebene - ablehnten. Es wurden aber nicht alle Keynesianer zu "`Neu-Keynesianern"': James Tobin zum Beispiel bestand darauf ein "`Alt-Keynesianer"', nicht  "`Neu-Keynesianer"' zu sein \parencite[S. 45ff]{Tobin1993}. Edmund Phelps drückte dies so aus: "`I [had] warm personal relations with Jim [James] Tobin and Bob [Robert] Solow as well as with Bob [Robert] Lucas and Tom [Thomas] Sargent – relations that have survived our differences. But I belonged to neither school." \parencite{Phelps2006}


Ab Anfang der 1990er Jahre kam es zunehmend zu einer Verschmelzung von "`Neu-Keynesianismus"' und "`Neuer Klassischer Makroökonomie"'. Diese wird als "`Neue Neoklassische Synthese"' im nächsten Kapitel beschrieben. Da es eher eine Verdrängung der "`starren"' Neuen Klassischen Makroökonomie durch junge Vertreter des "`Neu-Keynesianismus"' ist, wird sie aber häufig auch einfach "`Neu-Keynesianismus"' genannt\footnote{Man könnte sie auch Zweite Generation des Neu-Keynesianismus nennen}. die Hauptvertreter sind hier \textsc{John Taylor}\footnote{der aber eigentlich auch zur ersten Generation der Neu-Keynesianer gezählt werden muss} \textsc{David Romer, Greg Mankiw, Blanchard und Paul Krugman}. Der Unterschied zwischen der ersten Generation der Neu-Keynesianer und der zweiten Generation ("`Neue Neoklassische Synthese"') ist, dass die Letztgenannte vor allem die Methoden der "`Neuen Klassiker"', insbesondere "`Dynamische Stochastische General Equilibrium"'-Modelle aus der "`Real Business Cycle"'-Theorie übernommen hat und um ursprünglich keynesianische Elemente, nämlich Monopolistische Konkurrenz, Rigide Löhne und Preise und Nicht-Neutralität der Geldpolitik (und Fiskalpolitik) in der kurzen Frist, übernommen hat. Mehr dazu aber im nächsten Kapitel

Allgemein aber täuscht der Name "`Neu-Keynesianismus"' auf jeden Fall: Er ist nicht etwa eine Weiterentwicklung des Keynesianismus. Schon die hier beschriebene "`Erste Generation der Neu-Keynesianer"' akzeptierte inhaltlich und methodologisch die Fortschritte durch die "`Neuen Klassiker"', bestand aber auf der Bedeutung von Fiskal- und vor allem Geldpolitik, sowie der Existenz von Marktversagen. 

 
 Unterschiede Neu-Keynes vs. Neu-Klassik: (Romer-Paper, Snowdon S. 363)
 
 Klassische Dichotomie (Kein Unterschied zwischen Real und Nominal, weil Anpassung)
 Monopolistische Konkurrenz vs. Walrasianischer Auktionator
 
 
 
 
 
 
 
 
 

\section{Phelps: Mikrofoundation der Makroökonomie}
\label{micmac}

Man findet wohl kaum einen Namen, der den Übergang von "`Keynesianismus"' zu "`Neu-Keynesianismus"' besser repräsentiert als \textsc{Edmund Phelps}. Ökonomisch geprägt wurde er in einem eindeutig keynesianischen Umfeld: Er verfasste bei James Tobin seine Dissertation und arbeitete Mitte der 1960er Jahre mit Robert Solow, Paul Samuelson und Franco Modigliani zusammen. Also alles eindeutig keynesianische Ökonomen, die wir aus Kapitel \ref{Synthese} kennen. Laut seines autobiografischen Artikels \textcite[S. 93]{Heertje1995} war diese Zeit, inklusive Gastprofessur am Massachusetts Institute of Technologie (MIT), die prägendste seiner Karriere. Er selbst war innerhalb weniger Jahre ein international anerkannter Ökonom. Schon 1961 veröffentlichte er sein erstes bedeutendes Werk: \textit{The Golden Rule of Accumulation} \parencite{Phelps1961}. Ein bemerkenswerter Artikel, den der gerade mal 28-jährige Phelps im American Economic Review veröffentlichte. Gerade einmal sieben Seiten lang, beginnt dieser - so wie im Englischen normalerweise Märchen  - mit "`Once upon a time"'. In weiterer Folge wechseln sich mathematische Formeln mit Dialogen zwischen dem König und dem Volk der Solovians ab \parencite[S. 640]{Phelps1961}. So witzig und amüsant die Geschichte des Artikels, so bahnbrechend ist auch deren Inhalt. Diese Arbeit kann als direkter Anschluss an die Wachstumstheorie Solow's gesehen werden und im Zentrum steht folgende hypothetische Überlegung: Wenn die gesamte aktuelle Wirtschaftsleistung für die Investition (Investition = Sparen!) in neue Produktionsgüter verwendet wird, dann wird nichts für den aktuellen Konsum ausgegeben. Wird hingegen die gesamte aktuelle Wirtschaftsleistung für Konsum verwendet, werden im Umkehrschluss keinerlei neuen Investitionen getätigt. Beide Extrembetrachtungen führen also zu keinem sinnvollen Gleichgewicht. Das heißt aber auch, dass dazwischen irgendein optimales Verhältnis zwischen Sparen/Investieren auf der einen Seite und Konsumieren auf der anderen Seite bestehen muss. Dieses erreicht man eben durch \textit{The Golden Rule of Accumulation}. Diese wird erreicht - solange man einige vereinfachenden Annahmen zulässt - wenn die Wachstumsrate des BIPs dem Zinssatz entspricht. Bereits Phelps nannte diese natürliche Wachstumsrate "`nachhaltig"' \parencite[S. 638]{Phelps1961}. Weiters zeigt Phelps formal, dass diese Wachstumsrate erzielt wird, wenn die Summe der Investitionen der Summe der Profite entspricht, also alle Profite investiert werden. Umgekehrt werden im Optimum alle Löhne konsumiert. Zusammengefasst: Wenn alle Löhne konsumiert werden und alle Profite investiert werden, befindet sich die Ökonomie auf einem nachhaltigen Wachstumspfad. Die Wachstumsrate entspricht dann dem Zinssatz. Insgesamt erinnert das Ergebnis an die Arbeiten von Wicksell und Hayek. Die formale Herleitung durch Phelps war aber zu diesem Zeitpunkt - im Jahre 1961 - eine bahnbrechende Erweiterung des Solow-Wachstumsmodells.

Das bisher in diesem Unterkapitel dargestellte, entspricht noch vollständig dem keynesianischem Denken aus Kapitel \ref{Synthese}. Im Jahr 1966 wechselte Phelps von Yale an die University of Pennsylvania (Penn). Mit dem Umzug konzentrierte er sich auf neue Themen, nämlich auf die theoretische Fundierung der Phillipskurve. Seine Arbeiten dazu sollten später die ersten Zweifel am dominierenden, keynesianschen Framework begründen. Im Nachhinein kann man getrost sagen, dass damit die Grundlagen für den "`Neu-Keynesianismus"' geschaffen wurden.

Wie in Kapitel \ref{sec: Phillips} dargestellt, war der vermeintliche, negative Zusammenhang zwischen Inflation und Arbeitslosigkeit zwar nicht Bestandteil der ursprünglichen keynesianischen Theorie. Aber in weiterer Folge vor allem in der keynesianischen Wirtschaftspolitik ein fixer Bestandteil. Unabhängig voneinander waren Milton Friedman und eben Edmund Phelps bereits ab Mitte der 1960er Jahre die ersten Ökonomen, die den Zusammenhang zwischen Inflation und Arbeitslosigkeit in Frage stellten. Wohlgemerkt zu einer Zeit, in der der Zusammenhang empirisch noch recht gut beobachtet werden konnte. Das in den 1970er Jahren diese Korrelation weitgehend verschwand gab den Kritikern Friedman und Phelps natürlich gehörig Auftrieb. Phelps hatte seine Kritik dabei - im Gegensatz zu Friedman - mathematisch-formal unterlegt. 

Der Artikel mit dem unscheinbaren Titel "`Money-Wage Dynamics and Labor-Market Equilibrium"' \parencite{Phelps1968} stellte die bis dahin unbestrittene Phillipskurve nicht nur infrage, sondern legte die Grundlage für eine ganz neue Sicht auf die Wirtschaftswissenschaften. Interessant ist, dass gleich mehrere Punkte, die natürlich ineinandergriffen, in diesem Artikel revolutionäre waren:
\begin{enumerate}
\item Die Mikrofundierung der Makroökonomie
\item Die formale Einführung der Erwartungen (als adaptive Erwartungen) als Notwendigkeit bei Unvollständiger Information
\item Die formale Einführung der Natürlichen Arbeitslosigkeit und "`Effizienzlöhne"'
\end{enumerate}
Bemerkenswert ist insbesondere, dass alle drei genannten Punkte bis heute fixer Bestandteil der Mainstream-Modelle sind. Die heutigen DSGE-Modelle sind mikrofundiert, beinhalten das Konzept der Erwartungen (wenn auch der rationalen statt der adaptiven) und akzeptieren einen gewissen Prozentsatz an Arbeitslosigkeit als Gleichgewichtszustand. Natürlich wurden alle drei Konzepte seit 1968 wesentlich erweitert, aber im Gegensatz zu den Arbeiten anderer großen Ökonomen, fällt auf, dass Phelps' Arbeiten bis heute, 50 Jahre später, kaum an Gültigkeit verloren. Keynes' Multiplikator ist heute höchst umstritten, Friedman's Geldmengensteuerung betreibt keine Zentralbank der Welt mehr und selbst die späteren Arbeiten von Robert Lucas wurden größtenteils von der Realität überholt. Phelps' bahnbrechende Erkenntnisse sind hingegen bis heute die Grundlage ökonomischer Modelle und kann daher als Geburtsstunde des "`Neu-Keynesianismus"' gesehen werden.

In seiner Nobelpreis-Biographie schreibt Phelps, dass es seit seiner College-Zeit das Gefühl hatte die wichtigste aktuelle Herausforderung der Wirtschaftswissenschaften sei die Integration der Mikroökonomie in die Makroökonomie \parencite{Phelps2006}. Heute nennen wir dies die Mikrofundierung der Makroökonomie.
Der inhaltliche Ausgangspunkt des oben genannten Artikels \parencite{Phelps1968} ist die Phillipskurve. Phelps beschreibt sie als Naivität der Keynesianer. Wobei er Keynes selbst ausdrücklich in Schutz nimmt: Keynes' Nachfragesteuerung wäre niemals soweit gegangen einen dauerhaft stabilen Zusammenhang zwischen Inflation und Arbeitslosigkeit anzunehmen \parencite{Phelps2006}. Phelps stellt stattdessen einen Zusammenhang zwischen der \textit{erwarteteten} Inflation und Arbeitslosigkeit her. Dieser Zusammenhang sei aber nur in der kurzen Frist stabil. Angenommen die erwartete Inflation läge bei 4\%. Arbeitgeber und Arbeitnehmer würden bei ihren Vertragsverhandlungen diese Inflationserwartung einfließen lassen und die Lohnhöhe entsprechend festlegen. Will die Zentralbank nun die Arbeitslosigkeit senken, kann sie Maßnahmen setzen, die die Inflation auf zum Beispiel 6\% erhöhen. Solange die erwartete Inflation unter der tatsächlichen Inflation liegt, wird die Arbeitslosigkeit sinken und sich somit wie von der Phillipskurve postuliert verhalten. Es ist aber klar, dass die Diskrepanz zwischen tatsächlicher und erwarteter Inflation nur kurzfristig aufrechterhalten werden kann, bevor sich die Erwartung dem tatsächlichen Wert anpasst. Die keynesianische, langfristige Phillipskurve wurde durch die neu-keynesianische, kurzfristige erwartungsgestützte Phillipskurve ersetzt. Als solche findet sie bis heute Eingang in die makroökonomischen Lehrbücher. Nebenbei etablierte Phelps dabei das Konzept der adaptiven Erwartungen, das aber später vom neuklassischen Konzept der rationalen Erwartungen abgelöst werden sollte.
Die zentrale Aussage in \textcite{Phelps1968} lautet, dass durch Geldpolitik die Arbeitslosigkeit nicht dauerhaft beeinflusst werden kann, sehr wohl aber unter Umständen in der kurzen Frist. Geldpolitik funktioniere außerdem über Inflations\textit{erwartungen} und diese passen sich recht schnell an die aktuelle Inflation an. Eine niedrige Inflation wird daher auch nicht langfristig zu höherer Arbeitslosigkeit führen\parencite{Phelps1967}. Daraus könnte man ableiten, dass die zentrale Aufgabe der Zentralbanken die Inflationssteuerung ist. Heute orientieren sich fast alle führenden Zentralbanken tatsächlich primär an den Inflationszielen, dies aber direkt auf Phelps' frühe Arbeiten zurückzuführen ginge aber zu weit, folgten doch noch weitere Arbeiten dazu von anderen Neu-Keynesianern und Neuen Klassikern.
Sehr wohl direkte Folge aus \textcite{Phelps1968} ist hingegen die Idee der "`natürlichen Arbeitslosenrate"', später häufig als NAIRU\footnote{non-accelerating inflation rate of unemployment} bezeichnet. Während die Klassiker davon ausgingen, dass es im Gleichgewicht keine Arbeitslosigkeit gäbe und die Neuen Klassiker meinten im Gleichgewicht gäbe es ausschließlich freiwillige Arbeitslosigkeit, verfolgten die Keynesianer den Ansatz Arbeitslosigkeit sei stets mit nachfrageorientierter Wirtschaftspolitik zu minimieren. Die Neu-Keynesianer gehen davon aus, dass es im Gleichgewicht ein gewisses Maß an unfreiwilliger Arbeitslosigkeit gäbe. Diese "`natürliche Arbeitslosenrate"' wird häufig Milton Friedman zugeschrieben, der einen sehr ähnlichen Ansatz ebenfalls 1968 veröffentlichte \parencite{Friedman1968}. Tatsächlich hatten Friedman und Phelps unterschiedliche Wege gewählt, die sie zu den gleichen Schlussfolgerungen führten. Der Begriff "`natürliche Arbeitslosenrate"' ist wohl Friedman zuzuschreiben, größeren Einfluss in der akademischen Welt hatte aber der Ansatz von Phelps \textcite[S. 9f]{Nobelpreis-Komitee2006}. Die theoretische Bearbeitung der Arbeitslosigkeit wurde später zu einem eigenen Forschungsgebiet innerhalb des "`Neu-Keynesianismus"' und ist in Kapitel \ref{Suchtheorie} dargestellt.
Die zentralen Aussagen der damals neuen Theorie wurden in einer Konferenz aufgearbeitet und schließlich gesammelt als Buch veröffentlicht \textcite{Phelps1970}. Dieses war in weiterer Folge einflussreich und erreichte unter Ökonomen einen hohen Bekanntheitsgrad unter dem Titel "`The Phelps volume"'.

Die Mikrofundierung der Makroökonomie wird ebenfalls häufig als wesentliche Neuerung der "`Neuen Klassischen Makroökonomie"' gesehen. Es ist aus methodischer Sicht \textit{der} große Bruch mit den Theorien der Keynesianer und auch Monetaristen. Tatsächlich ist nicht von der Hand zu weisen, dass die Neuen Klassiker diesen Ansatz als Standard in ökonomischen Modellen etablierten (vgl. Kapitel \ref{Neue Makro}). Aber auch hier gilt, dass die erstmalige Anwendung auf Phelps zurückgeht. In der nun schon häufig zitierten Arbeit \textcite{Phelps1968} verwendet Phelps ein mikroökonomisches Modell um den klar makroökonomischen Zusammenhang zwischen Inflation und Arbeitslosigkeit zu modellieren. Die dort notwendige intertemporale Betrachtung (im Kapitel \ref{Neue Makro} haben wir das schlicht "`dynamische Betrachtung"' genannt) der Interaktion der Kennzahlen war eine weitere Neuerung durch Phelps, die bis heute State-of-the-art in den Wirtschaftswissenschaften ist \parencite[S. 9]{Nobelpreis-Komitee2006}.

Die soeben dargestellten Ergebnisse wurden nicht unmittelbar begeistert aufgenommen, wie dies zum Beispiel mit Keynes' General Theory, oder der Theorie der Rationalen Erwartungen von Lucas passierte. Bei ihrem erscheinen Ende der 1960er-Jahre konkurrierten die Ideen von Phelps mit zahlreichen alternativen Ideen. Erst etwas später wurde durch die Stagflation sichtbar, dass die Phillipskurve in ihrer alten Form untragbar wurde und Phelps Erwartungsgestützte Phillipskurve als Alternative zielführender sei.

Für die Ende der 1060er Jahre noch unumschränkt dominierende keynesianische Theorie waren Phelps' Ergebnisse gewissermaßen schockierend: Die Phillipskurve war zwar wenig theoretisch begründet, spielte aber in der keynesianisch geprägten Wirtschaftspolitik eine wichtige Rolle. Das Ergebnis, dass Geldpolitik in der langen Frist als nachfrageorientierte Wirtschaftspolitik wirkungslos sei, beschränkte das keynesianische Framework. Die Mikrofundierung der Makroökonomie beschritt methodisch gänzlich neue Wege. 

Kurz zusammengefasst kann Phelps' frühes wirken so beschrieben werden: Er entwickelte noch in der Tradition der neoklassischen Synthese die "`Goldene Regel der Akkumumlation"'. Mit seinem Umzug an die University of Pennsylvania emanzipierte er sich aber vom Keynesianismus und es folgten Arbeiten die zur Revolution der Phillipskurve führen sollten. Diese Arbeiten umfassten Konzepte, die bis heute in der Mainstream-Ökonomie State-of-the-Art sind. Erstens, war er ein Vorreiter bei der Mikrofundierung der Makroökonomie und zweitens, etablierte er (adaptive) "`Erwartungen"' in den Modellen der Ökonomie. Beides spielte später bei den "`Neuen Klassikern"' eine wesentliche Rolle, wenn auch in der Form der \textit{rationalen} statt der \textit{adaptiven} Erwartungen. Er nahm also die Kritikpunkte der Neuen Klassiker am Keynesianismus vorweg. Man beachte, dass das bahnbrechende Werk \textcite{Phelps1968} noch vor der Neu-Klassischen Revolution erschien \parencite{Lucas1972, Lucas1976}. Er wurde aber kein Vertreter dieser "`Neuen Klassiker"', sondern war sich immer der Unvollständigkeit der Märkte bewusst, die sich in unfreiwilliger Arbeitslosigkeit, monopolistischer Konkurrenz und Rigiditäten auf Märkten ausdrückte. Er nahm der Wirtschaftspolitik damit die Illusion einer funktionierenden Feinsteuerung der Wirtschaft, ohne dabei aber einer vollkommenen Marktgläubigkeit zu verfallen. Ich denke man kann ihn daher getrost als eigentlichen Begründer des "`Neu-Keynesianismus"' bezeichnen.

Die eben genannten Arbeiten zur Unvollständigkeit der Märkte, unfreiwilliger Arbeitslosigkeit, monopolistischer Konkurrenz und Rigiditäten auf Märkten wurden später als Antwort auf die empirischen Defizite der "`Neuen Klassiker"' ausgearbeitet. Sie bilden den Kern der Ersten Generation des Neu-Keynesianismus.







\section{"'Kern"' des Neukeynesianismus}
\label{Kern}


Diese Arbeiten stellen den Kern der Neu-Keynesianer 1. Generation dar, weil hier erstmals die zwei wesentlichen Punkte des Neu-Keynesianismus zusammengefügt werden: Erstens, die Monopolistische Konkurrenz als Folge der Nominalen Rigiditäten und Menu Costs und zweitens, die Nicht-Neutralität der Geldpolitik aus dem Spannungsverhältnis Nominale vs Reale Rigiditäten und das daraus ableitbare Nicht-Vorhandensein der Klassischen Dichotomie. \parencite{RomerDavid1993}

\subsection{Nominale Rigiditäten}
\label{Nominale Rigiditäten}

Die Arbeiten von Phelps waren, wie gerade erwähnt, der Ursprung des Neu-Keynesianismus und wurden zeitlich vor der neu-klassischen Revolution formuliert. Die meisten neu-keynesianischen Arbeiten der 1. Generation entstanden allerdings als direkte Antworten auf die aufkommenden aber mit starren Annahmen unterlegten Arbeiten der "`Neuen Klassiker"'.

\textcite{Lucas1976} gab den Anstoß zur "`Neuen klassischen Makroökonomie"' und damit zur Theorie der rationalen Erwartungen. \textcite{Sargent1975} steuerten ihre berühmte "`policy-ineffectiveness proposition"' bei, also die Annahme, dass jegliche Wirtschaftspolitik ohne Effekt verpufft. Bereits 1977 folgten - als direkte Antwort - zwei Arbeiten \textcite{Taylor1977, Fischer1977} von Ökonomen, die eben nicht Teil der "`Neuen Klassik"' sein wollten, aber die Überlegenheit einzelner Elemente daraus akzeptierten. Das realisiert man bereits wenn man nur das Abstract der beiden Artikel liest. Sinngemäß steht da, dass aktive Geldpolitik sehr wohl eine Wirkung haben kann, da Löhne in der kurzen Frist rigide sind. Dies sei unabhängig von der Annahme rationaler Erwartungen. Zwei typisch neu-keynesianische Elemente kommen hier vor. Erstens, das Vorhandensein von (nominalen) Rigiditäten und damit die Wirksamkeit von aktiver Wirtschaftspolitik und somit die Ablehnung der "`policy-ineffectiveness proposition"' und zweitens, die implizite Akzeptanz der Annahme rationaler Erwartungen.
Die Annahme Adaptiver Erwartungen im Sinne von \textcite{Phelps1968} bedeutet, dass Entscheidungsträger, also zum Beispiel die Zentralbank, für einen Unterschied zwischen tatsächlicher Inflationsrate und erwarteter Inflationsrate sorgen kann. Die Rationalen Erwartungen nach \textcite{Lucas1976} gehen hier weiter und behaupten, dass es keine Differenz zwischen tatsächlicher und erwarteter Inflationsrate geben kann. Wenn nämlich die Zielinflation (oder zu dieser Zeit noch die Geldmengenziel - "`money supply rule"') bekannt ist, dann wissen die Haushalte ebensogut wie die Entscheidungsträger in welcher Form auf überschießende oder zu niedrige Inflation reagiert wird.  Sowohl \textcite{Fischer1977} als auch \textcite{Taylor1977} akzeptieren die Existenz von Rationalen Erwartungen. Die von den Neuen Klassikern daraus abgeleitete Wirkungslosigkeit von Geldpolitik hingegen lehnen sie hingegen strikt ab. Das Argument dafür ist, dass es langfristige Verträge gibt die Löhne (in \textcite{Fischer1977}), bzw. Preise (in \textcite{Taylor1977}) festsetzen. Geldpolitik hingegen kann laufend vorgenommen werden. Das Ergebnis ist, das diese nominale Lohn- und Preisrigidität, das Geldpolitik in der kurzen Frist sehr wohl wirksam ist. Die beiden genannten Werke gelten heute noch als ein Eckpfeiler der Neu-Keynesianischen Theorie. Das Ergebnis ist in gewisser Weise paradox, denn waren es nicht die "`alten"' Keynesianer, die behaupteten, dass Geldpolitik wirksam ist ("`money matters"'), wenn Preise und Löhne rigide sind? \textcite[S. 166]{Taylor1977} sind sich dessen bewusst. Aber Sie schränken ein, dass die postulierten Zusammenhänge ganz andere waren und nur ihre Theorie mit der Annahme Rationaler Erwartungen vereinbar sei. Oder wie es \textcite[S. 166]{Taylor1977} ausdrücken: \textit{"'By adopting the framework of rational expectations, we hope to have produced not a new wine but an old wine in a new and more secure bottle."'}
Die Akzeptanz der Gültigkeit der Annahme Rationaler Erwartungen ist übrigens bis heute ein Streitpunkt zwischen den "`alten"' Keynesianern (also den Vertretern der Neoklassischen Synthese) und den Neu-Keynesianern. Für viele Vertreter der erstgenannten Gruppe ist diese Annahme schlicht unrealistisch.

\textcite{Taylor1979, Taylor1980} verallgemeinerte die Aussagen dahingehend, dass Rigiditäten auch außerhalb der strengen Annahmen fixer Laufzeiten bei Arbeitsverträgen auftreten. In seinem Modell geht er davon aus, dass Verträge gestaffelt neu für eine gewisse Zeitperiode ausverhandelt werden. Das Modell wird dementsprechend als "`Staggered contracts"', oder "`Taylor contracts"' bezeichnet. Im einfachsten Fall kann man davon ausgehen, dass Löhne in Arbeitsverträgen nur alle zwei Jahre angepasst werden. Das heißt jedes Kalenderjahr wird eine Hälfte der Arbeitsverträge an die beobachtete Inflation angepasst. Taylor konnte so zeigen, dass durch diese künstlich modellierte Lohnrigidität eine Abweichung vom langfristigen Gleichgewicht entsteht. \textcite{Blanchard1983} wendete dieses Modell auf die Preissetzung von Waren an und stellte fest, dass bei längeren Herstellungsketten der Effekt der Rigidität größer ist. Diese frühen Modelle der nominalen Rigiditäten waren noch nicht mikroökonomisch fundiert \parencite[S. 194]{Fischer1977}, was auch von Seiten der Neuen Klassiker recht rasch zu entsprechender Kritik führte. Dies wurde später durch die Arbeit von \textcite{Rotemberg1987} behoben. Die Annahme nominaler Lohnrigiditäten (nicht nominaler Preisrigiditäten) wurde später als unzureichend kritisiert \parencite{Mankiw1990}, weil die Reallöhne während Rezessionen steigen würden, was einerseits empirischen Beobachtungen widerspricht und andererseits dazu führt, dass Wirtschaftskrisen bei Personen mit sicheren Jobs sehr populär wären \parencite[S. 371]{Snowdon2005}. Erst in Verbindung mit der Annahme "`Monopolistischer Märkte"' (Imperfekte Märkte), "`Nicht-kostenloser Preisanpassung"' (Menu Costs), und "`Friktionen auf den Arbeitsmärkten"', die allesamt ebenfalls als Neu-Keynesianische Markenzeichen in diesem Kapitel noch besprochen werden, lassen sich rigide Löhne rechtfertigen.
Die nominale Preisrigidität überlebte aber und ist bis heute Teil der aktuellen Mainstream-Modelle! In den  Neu-Keynesianischen DSGE-Gesamtmodellen, die in Kapitel \ref{Neue Neoklassische Synthese} beschrieben werden, wird nominale Preisrigidität mittels "`staggered price setting"'-Modell von \textcite{Calvo1983} modelliert. Er bezieht sich dabei direkt auf die Arbeiten von \textcite{Taylor1979, Taylor1980}, macht daraus aber ein stochastisches Modell. Das heißt, Unternehmen können die Preise und Löhne nicht mehr nach Ablauf einer gewissen Zeitspannen anpassen, sondern erst jeweils nach einem zufällig langem Zeitraum. Das heißt die Preisanpassungen nach einem exogenen Schock finden noch unregelmäßiger statt. Dies bildet empirische Beobachtungen noch besser ab. 

Überzeugende Empirische Evidenz für die Existenz von nominalen Preisrigiditäten konnte man erst mit dem Aufkommen von Mikro-Datensätzen um die Jahrtausendwende erstellen. \textcite{Nakamura2008} fanden heraus, dass nominale Preise etwa neun bis elf Monate im Durchschnitt Bestand haben, dass es also nominale Rigiditäten tatsächlich gibt. Die beiden kritisieren aber auch, dass in den am häufigsten zitierten Modellen von \textcite{Taylor1980} und \textcite{Calvo1983} bei Preisänderungen stets von Preis\textit{erhöhungen} ausgeht. In ihrer empirischen Studie fanden \textcite[S. 1442]{Nakamura2008} hingegen heraus, dass mehr als ein Drittel aller Preisänderungen im Beobachtungszeitraum aber Preissenkungen waren.


\subsection{Monopolistische Konkurrenz}
\label{Monopol}
Die Annahme, dass auf den Märkten generell "`Vollständige Konkurrenz"' ("`Perfekte Konkurrenz"') herrscht, ist grundsätzlich so alt wie die Wirtschaftswissenschaft selbst. Implizit ging bereits Adam Smith davon aus, dass sich auf Märkten eine große Zahl von Anbietern und Nachfragern treffen und einen Gleichgewichtspreis finden. Explizit ausgesprochen und analysiert wurde dies von Leon Walras und seinem Konzept des "`Allgemeinen Gleichgewichts"'. Dieses spielt ja bis heute - vor allem bei den Neuen Klassikern als "`Walrasianischer Auktionator"' eine große Rolle. In der Mikroökonomik analysierten Gerard Debreu und Kenneth Arrow in den 1950er Jahren das "`Allgemeine Gleichgewicht"' unter Einbeziehung der Finanzmärkte mit modernen mathematischen Methoden. Natürlich wusste man, dass Märkte in vielen Fällen keine perfekten Konkurrenzmärkte sind, sondern eher Monopolistischen Märkten gleichen. Man versetze sich dazu nur an folgendes Beispiel: Sie gehen in einen Supermarkt und wollen dort Güter des täglichen Bedarfs kaufen. Auf einem vollständigen Konkurrenzmarkt müssten Sie den Preis jeden Gutes mit dem Verkäufer verhandeln. Das wäre zeitaufwändig und damit auch teuer. Stattdessen akzeptieren Sie mit dem Besuch im Supermarkt implizit, dass der Verkäufer eine Preis festsetzt. Auch wenn Ihnen manchmal vielleicht bewusst ist, dass Sie eine gute Verhandlungsbasis für einen niedrigere Preis hätten (z.B.: Das gleiche Gut kostet bei Ihrem Konkurrenten weniger), werden Sie in der Regel den festgesetzten Preis akzeptieren. Der Supermarkt wiederum ist sich seines  HIER WEITER



Die Idee dieses Wissen in makroökonomischen Modellen umzusetzen ist \textit{das} Alleinstellungsmerkmal der Neu-Keynesianer. Dieses Element ist nämlich sowohl den Keynesianern, den Monetaristen als auch den Neuen Klassikern fremd. 
Die Existenz von Monopolistischen Märkten ist im engen Zusammenhang mit den im letzten Kapitel behandelten Rigiditäten. Tatsächlich führen diese erst dazu, dass es keine "`Perfekte Konkurrenz"' geben wird. 

HIER WEITER



\subsection{Menu Costs}
Die modelltheoretischen Grundlagen \textit{wie} man nominale Rigiditäten berücksichtigen kann waren also schon früh durch \textcite{Taylor1977, Fischer1977}, bzw. \textcite{Calvo1983} geschaffen worden, wie in Kapitel \ref{Nominale Rigiditäten} dargestellt. Unbeantwortet hingegen blieb bislang die Frage, \textit{warum} es zu diesen nominalen Rigiditäten kommen kann. Die Antwort darauf lieferten \textcite{Mankiw1985b, Akerlof1985, Parkin1986, RomerDavid1990} und \textcite{Ball1988}. 

Die Ausgangsannahme ist, dass die Anpassung von Preisen selbst Kosten verursacht. Daraus leitet sich der von \textcite{Mankiw1985b} geprägte Begriff der "`Menu Costs"', also "`Speisekarten-Kosten"', ab. Dahinter steht folgende exemplarische Idee: Das Drucken neuer Speisekarten kostet Geld. Gastronomen müssen also abschätzen, ab welchem Ausmaß von Preiserhöhungen Mehrerlöse entstehen, die den Druckkostenaufwand wieder ausgleichen\footnote{Die Idee entstand in den 1980er Jahren, also vor der großen digitalen Revolution. Heute wären diese "`Menu Costs"' im Wortsinn vermutlich wesentlich geringer als 1985.}. Eine sehr einfache Idee, die für die meisten individuellen Unternehmen kaum von Belang ist. \textcite{Akerlof1985, Mankiw1985b, Parkin1986} zeigten aber jeweils, dass diese kleinen individuellen Effekte zu großen makroökonomischen Effekten führen können. \textcite{Rotemberg1987} nannte diese Erkenntnis "`PAYM insights"', angelehnt an die Anfangsbuchstaben der vier Autoren. Konkret führt auf Märkten mit Monopolistischer Konkurrenz das individuell Nutzen-maximierende Verhalten der einzelnen Unternehmer\footnote{\textcite[S. 823]{Akerlof1985} nennen es \textit{"'Insignifikant"' suboptimales Verhalten}, bzw. \textit{Nahe-Rationales Verhalten}} dazu, dass Preisanpassungen an den Gleichgewichtspreis erst bei größeren Preissprüngen vorgenommen werden.  Kommt es also zu einem (geringen) Rückgang der aggregierten Nachfrage, werden Unternehmen bei monopolistischer Konkurrenz ihre Preise aufgrund der "`Menu Costs"' zunächst nicht anpassen, sondern stattdessen, trotz niedriger Nachfrage, den ehemaligen Gleichgewichtspreis verlangen. Gesamtwirtschaftlich optimal wäre es, wenn die Unternehmen ihren Preis senken würden. Das würde man auch in der neoklassischen Analyse von Monopolmärkten erwarten. Da Unternehmen aber "`Menu Costs"' ausgesetzt sind, werden die höheren Preis beibehalten. Für Unternehmen ist dieses Verhalten Nutzen-maximierend, weil die Preis-Anpassungskosten höher wären als der zusätzliche Gewinn, den Unternehmen bei geringerem Preis aber höherer abgesetzter Menge, erhalten würden \parencite[S. 372]{Snowdon2005}. Gesamtwirtschaftlich ist das Ergebnis aber suboptimal, weil die abgesetzte Menge beim rigiden Preis viel geringer ist als die theoretische Gleichgewichtsmenge.\footnote{Dieser Ansatz mit Monopolistischer Konkurrenz und de-facto Preissetzung ist nicht unähnlich frühen Post-Keynesianischen Ansätzen (vgl. \ref{Post-Keynes})! Auch wenn Neu-Keynesianer dies nur ungern zugeben würden.} 

Bleibt man auf Ebene eines einzelnen Unternehmens ist die Erkenntnis zwar durchaus interessant, sie scheint aber weitgehend folgenlos für die Gesamtwirtschaft. Das ist aber ein Trugschluss. Die wichtigste Erkenntnis der "`PAYM-insights"' ist, dass kleine - auf natürliche Schwankungen zurückzuführende - Rückgänge bei der aggregierten Nachfrage zu deutlichen Schwankungen beim gesamtwirtschaftlichen Output führen\parencite[S. 375]{Snowdon2005}. Da solche Schwankungen unerwünscht sind, argumentieren \textcite{Akerlof1985}, dass aktive Wirtschaftspolitik\footnote{\textcite[S.837]{Akerlof1985} schreiben konkret nur von Geldpolitik} sehr wohl einen stabilisierenden und damit wünschenswerten Effekt hat.


\subsection{Reale Rigiditäten auf Märkten}
\label{Reale Rigiditäten}

Anfang der 1990er Jahre wurde das Konzept der Rigiditäten verfeinert. Denn zwar konnte durch die "`PAYM-insights"' gezeigt werden, dass es theoretisch möglich ist, dass nominale Rigiditäten zu große Schwankungen im Gesamtoutput verursachen, aber eben nur theoretisch. \textcite[S. 183]{RomerDavid1990} zeigten, dass dies nur dann möglich wäre, wenn ganz bestimmte Bedingungen erfüllt sind. So würde es, zum Beispiel, beim Auftreten von nominalen Preisrigiditäten nur dann zu großen Effekten auf den Gesamtoutput kommen, wenn der Arbeitsmarkt gleichzeitig extrem elastisch wäre. Gastronomen würden demnach zwar bei Nachfragerückgängen ihre Preise auf den Speisekarten unverändert lassen, aber gleichzeitig sofort Köche und Kellner kündigen?! Ein eher unrealistisches Szenario. Stattdessen führten \textcite{RomerDavid1990} \textit{reale} Rigiditäten als notwendige Ergänzung zu \textit{nominalen} Rigiditäten ein.

Zunächst muss einmal abgegrenzt werden, wie sich reale Rigiditäten von nominalen Rigiditäten unterscheiden. Nominale Rigiditäten haben wir schon als "`Menu Costs"' kennengelernt. Allgemein könnte man diese definieren als die Geschwindigkeit mit der sich Löhne und Preise anpassen, wenn vorgelagerte Preise sich ändern \parencite[S. 270]{Blanchard2003}. Reale Rigiditäten sind schon seit Keynes (vgl. Kapitel \ref{Keynes}) bekannt: Nominale Löhne sind rigide, weil sie bei Deflation nicht nach unten angepasst werden können. Die \textit{realen} Löhne steigen also durch Deflation´. Man spricht von \textit{realen} Rigiditäten. In diesem Fall kommt der Effekt des Geldwertes ins Spiel. Glaubt man, wie die Neuen Klassiker, an die Klassische Dichotomie zwischen Geldmarkt und Gütermarkt auch in der kurzen Frist, ist dieser Effekt nicht vorhanden. Die Neu-Keynesianer lehnen diese Klassische Dichotomie zumindest für die kurze Frist ab. Damit akzeptieren sie das temporäre Auseinanderlaufen von realen und Nominalen Werten und eben auch die Wirksamkeit von Geldpolitik! Wie sind reale Rigiditäten aber konkret zu erklären? Das Beispiel von Keynes, dass reale Löhne in der Deflation steigen, liefert ja nur in Zeiten von Deflation eine Erklärung. Gerade in den 1980er Jahren gab es in den Industriestaaten keine deflatorischen Tendenzen. Damit beschäftigte sich ein ganzes Forschungsfeld (vgl. Kapitel \ref{drei Grunde}). Und zwar großteils unabhängig und sogar zeitlich schon vor den hier dargestellten Auswirkung von Rigiditäten. \textcite{RomerDavid1990} verbanden in ihrer Arbeit das Forschungsfeld der Rigiditäten mit den Forschungsfeldern, die in Kapitel \ref{drei Grunde} dargestellt sind und identifizierten diese als reale Rigiditäten. Mit ihren Überlegungen grenzten \textcite{RomerDavid1990} den Neu-Keynesianismus ein weiteres Mal entscheidend als eigene ökonomische Denkrichtung ab und trugen mit der Verbindung der einzelnen Elemente dazu bei, dass der Neu-Keynesianismus als einheitliches Gesamtmodell gesehen werden kann. Davor waren Beiträge stets als ablehnende Antwort gegenüber den Neuen Klassikern entstanden, die aber eher unabhängig voneinander gesehen werden mussten.

Zum Inhalt dieses für den Neu-Keynesianismus wichtigen Beitrags: \textcite[S. 183]{RomerDavid1990} heben dabei gleich zu Beginn ihres Artikels hervor, "`dass \textit{reale} Rigiditäten nicht das gleiche sind wie \textit{nominale} Rigiditäten"'. Bis Ende der 1980er Jahre entstanden zwar viele Forschungsarbeiten zu Rigiditäten, diese unterschieden aber nicht zwischen realen und nominalen Effekten. Im nächsten Schritt erstellen die beiden ein interessantes aber komplexes Modell. Dessen Grundaussage lautet wie folgt: Erstens, nominale Rigiditäten ("`Menu Costs"') können realistischer Weise nur zu kleinen gesamtwirtschaftlichen Schwankungen führen. Zweitens, reale Rigiditäten können alleinstehend kaum existieren: Auf vollständigen Konkurrenzmärkten (dies entspricht einem Markt ohne jegliche nominale Rigidität), kommt es immer zur Anpassung an das Marktgleichgewicht. Würde man hier aufhören, wäre die Essenz: Rigiditäten spielen keine Rolle. Aber jetzt kommt der Clou aus \textcite{RomerDavid1990}: Treten nominale Rigiditäten auf, so können auch reale Rigiditäten existieren. In diesem Fall verstärken die realen Rigiditäten die nominalen Rigiditäten und es kann zu großen Schwankungen im Gesamtoutput kommen. Diese wiederum rechtfertigen - wie schon im Paper von \textcite{Akerlof1985} - den Einsatz aktiver Wirtschaftspolitik.

HIER WEITER MIT BEISPIEL

Noch zusätzlich: \parencite{Ball1988}

Steht in \textcite{Mankiw1991} Seite 4 unten. und S. 83 oben
Steht in \textcite[S. 378]{Snowdon2005}.


Verbindung zu Coordination Failures und 


Bringt auch die Nicht-Neutralität des Geldes ins Spiel



Ball and Romer, 1990; Woodford, 2003). Hier aus \parencite[S. 33]{Gali2007}


Eröffnete ein riesiges Forschungsfeld: Arbeitsmarkt, Kreditmarkt, Gütermarkt. (bzw. verband dieses mit dem Kern: Rigiditäten \textcite[S. 4]{Mankiw1991})



Taylor (auch als Übergang zur Neuen Synthese)


\subsection{Arbeitsmarkt, Kreditmarkt und Gütermarkt Rigiditäten}
\label{drei Grunde}

Zunächst einzelne Antworten auf die Neuen Klassiker. Erst durch \textcite{RomerDavid1990} verwendet für Theorie der Gesamtwirtschaft.

\section{Arbeitslosigkeit als Suchproblem}
\label{Suchtheorie}


Coordination Failures
Diamond 1982: Coconut Modell

Efficiency Wages Theorie


Direkt angestoßen von Phelps. Aber erst wirklich interessant als Gegenposition zur "`Neuen Klassik"'.
\textcite[S. 683]{Phelps1968}

Microeconomic foundations for wage and price setting
Phelps (1968a, 1970b) derived aggregate wage-setting behavior from detailed modeling of the 
behavior of individual agents. Jobs and workers are heterogeneous and there is imperfect 
information on both sides of the market. Markets are assumed to be almost atomistic, but there is 
no “Walrasian auctioneer” that instantaneously finds the wages (and prices) that clear all markets 
(as went the metaphor used in earlier general equilibrium analysis). Instead, wages are set by 
firms that are able to exercise temporary monopsony power. Workers and firms meet randomly 
at a rate determined by the number of unemployed workers searching for a job and the number of 
vacancies, according to a function that would today be recognized as a matching function. 
Phelps’s work here is a precursor of the search and matching theory of unemployment, where 
Peter Diamond, Dale Mortensen, and Christopher Pissarides have made especially important 
contributions. \parencite{Nobelpreis-Komitee2006}  See, for example, Diamond (1984), Mortensen (1982a, b), Mortensen and Pissarides (1994), and Pissarides 
(2000).


\subsection{Diamond, Mortensen und Pisaridis}

Widersprach rasch der "`Neuen Klassik"' indem sie die natürliche Arbeitslosigkeit erweiterte.




\section{Nicht-Neutralität von Geldpolitik}


\textcite[S. 823]{Akerlof1985}





\section{Wirkung und Bedeutung des Neu-Keynesianismus}

Interessant, dass Fiskalpolitik praktisch keine Rolle spielt. Aussage dazu von Mankiw in: \parencite[S. 446]{Snowdon2005}

Gewachsen: Anfang Nominale Rigiditäten --> Monopolisitsiche Konkurrenz (Imperfect Competition) --> Menu Costs --> Real Rigidtäten --> Fehlende Klassische Dichotomie --> Nicht-Neutralität des Geldes.

Dazu Speziall-Forschungsgebiete: 

1. Reale Rigiditäten: Arbeitsmarkt, Kreditmarkt, Gütermarkt
2. Monopolistische Konkurrenz: Coordination Failures (Überbegriff)
