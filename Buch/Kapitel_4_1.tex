%%%%%%%%%%%%%%%%%%%%% chapter.tex %%%%%%%%%%%%%%%%%%%%%%%%%%%%%%%%%
%
% sample chapter
%
% Use this file as a template for your own input.
%
%%%%%%%%%%%%%%%%%%%%%%%% Springer-Verlag %%%%%%%%%%%%%%%%%%%%%%%%%%

\chapter{Neu-Keynesianismus} \label{cha: Neu Keynes}

Der Neu-Keynesianismus ist leider (noch) wesentlich schwieriger von anderen Schulen abzugrenzen als etwa der Keynesianismus oder der Monetarismus. Dies gilt sowohl in inhaltlicher Sicht, also auch in zeitlicher Sicht. Inhaltlich lässt sich der Neu-Keynesianismus am ehesten negativ abgrenzen. Einige Ökonomen erkannten, dass die Theorien der Keynesianer nicht mehr zureichend waren. Sie akzeptierten aber auch nicht die starren Annahmen der "`Neuen Klassiker"', diese Ökonomen könnte man "`Neu-Keynesianer"' nennen. Die "`Neu-Keynesianer"' sind dementsprechend keine geschlossene Gruppierung von Ökonomen, sondern behandelten eher zerstreut einzelne brennende Fragen der Ökonomie. Dazu gehörten zum Beispiel Fragen der Inflation (Phillips-Kurve), der Arbeitslosigkeit (Suchproblem, Natürliche Arbeitslosigkeit) und des Marktversagens (Informationsasymmetrien, Natürliche Monopole). 

Auch was die Personen betrifft ist die Abgrenzung schwieriger. So muss der Erzliberale \textsc{John Taylor} - Präsidenten der Mont Pelerin Society von 2018 - 2020 - inhaltlich zweifelsohne als ein früher Vertreter des Neu-Keynesianismus gesehen werden. 

Die schwierigste Abgrenzung erfolgt aber in zeitlicher Hinsicht. Schließlich wird die heutige Mainstream-Ökonomie häufig als "`Neu-Keynesianismus"' bezeichnet. In dieser Logik müsste man den "`Neu-Keynesianismus"' zumindest in zwei Generationen teilen. Zweifelsohne beginnt der "`Neu-Keynesianismus"' nämlich als Antwort auf die "`Neue Klassische Makroökonomie"' ab den frühen 1980er Jahren zu existieren. In Wahrheit sogar schon etwas früher, nämlich mit der Kritik an der Phillips-Kurve ab Mitte der 1960er Jahre. Dieser "`frühe"' Neu-Keynesianismus wird an dieser Stelle beschrieben und dauerte bis etwa Ende der 1980er Jahren. Die Hauptproponenten sind hier \textit{Edmund Phelps, Peter Diamond, Joseph Stiglitz, George Akerlof und William Baumol}. Diese Schule war der Gegenpol zur aufstrebenden "`Neuen Klassischen Makroökonomie"' um Lucas, Sargeant und Barro. Die Vertreter dieses frühen Neu-Keynesianismus liefern mit ihren Arbeiten vor allem "`Aufweichungen"' der zu starren Annahmen der "`Neuen Klassiker"'. Sie lehnen in diesem Sinn die Arbeiten der "`Neuen Klassiker"' ab, akzeptieren aber auch, dass der Keynesianismus veraltet ist. Von der Zuordnung der Personen her entwickelte sich der "`Neu-Keynesianismus"' eher aus den Salzwasser-Universitäten (vgl. die entsprechende Einteilung in Kapitel \ref{Neue Makro}), die die Neuen Klassiker ja strikt - und nicht nur auf inhaltlicher Ebene - ablehnten. Es wurden aber nicht alle Keynesianer zu "`Neu-Keynesianern"': James Tobin zum Beispiel bestand darauf ein "`Alt-Keynesianer"', nicht  "`Neu-Keynesianer"' zu sein \parencite[S. 45ff]{Tobin1993}. Edmund Phelps drückte dies so aus: "`I [had] warm personal relations with Jim [James] Tobin and Bob [Robert] Solow as well as with Bob [Robert] Lucas and Tom [Thomas] Sargent – relations that have survived our differences. But I belonged to neither school." \parencite{Phelps2006}


Ab Anfang der 1990er Jahre kam es zunehmend zu einer Verschmelzung von "`Neu-Keynesianismus"' und "`Neuer Klassischer Makroökonomie"'. Diese wird als "`Neue Neoklassische Synthese"' im nächsten Kapitel beschrieben. Da es eher eine Verdrängung der "`starren"' Neuen Klassischen Makroökonomie durch junge Vertreter des "`Neu-Keynesianismus"' ist, wird sie aber häufig auch einfach "`Neu-Keynesianismus"' genannt\footnote{Man könnte sie auch Zweite Generation des Neu-Keynesianismus nennen}. die Hauptvertreter sind hier \textsc{John Taylor}\footnote{der aber eigentlich auch zur ersten Generation der Neu-Keynesianer gezählt werden muss} \textsc{David Romer, Greg Mankiw, Blanchard und Paul Krugman}. Der Unterschied zwischen der ersten Generation der Neu-Keynesianer und der zweiten Generation ("`Neue Neoklassische Synthese"') ist, dass die Letztgenannte vor allem die Methoden der "`Neuen Klassiker"', insbesondere "`Dynamische Stochastische General Equilibrium"'-Modelle aus der "`Real Business Cycle"'-Theorie übernommen hat und um ursprünglich keynesianische Elemente, nämlich Monopolistische Konkurrenz, Rigide Löhne und Preise und Nicht-Neutralität der Geldpolitik (und Fiskalpolitik) in der kurzen Frist, übernommen hat. Mehr dazu aber im nächsten Kapitel

Allgemein aber täuscht der Name "`Neu-Keynesianismus"' auf jeden Fall: Er ist nicht etwa eine Weiterentwicklung des Keynesianismus. Schon die hier beschriebene "`Erste Generation der Neu-Keynesianer"' akzeptierte inhaltlich und methodologisch die Fortschritte durch die "`Neuen Klassiker"', bestand aber auf der Bedeutung von Fiskal- und vor allem Geldpolitik, sowie der Existenz von Marktversagen. 

 

\section{Phelps: Mikrofoundation der Makroökonomie}
\label{micmac}

Man findet wohl kaum einen Namen, der den Übergang von "`Keynesianismus"' zu "`Neu-Keynesianismus"' besser repräsentiert als \textsc{Edmund Phelps}. Ökonomisch geprägt wurde er in einem eindeutig keynesianischen Umfeld: Er verfasste bei James Tobin seine Dissertation und arbeitete Mitte der 1960er Jahre mit Robert Solow, Paul Samuelson und Franco Modigliani zusammen. Also alles eindeutig keynesianische Ökonomen, die wir aus Kapitel \ref{Synthese} kennen. Laut seines autobiografischen Artikels \textcite[S. 93]{Heertje1995} war diese Zeit, inklusive Gastprofessur am Massachusetts Institute of Technologie (MIT), die prägendste seiner Karriere. Er selbst war innerhalb weniger Jahre ein international anerkannter Ökonom. Schon 1961 veröffentlichte er sein erstes bedeutendes Werk: \textit{The Golden Rule of Accumulation} \parencite{Phelps1961}. Ein bemerkenswerter Artikel, den der gerade mal 28-jährige Phelps im American Economic Review veröffentlichte. Gerade einmal sieben Seiten lang, beginnt dieser - so wie im Englischen normalerweise Märchen  - mit "`Once upon a time"'. In weiterer Folge wechseln sich mathematische Formeln mit Dialogen zwischen dem König und dem Volk der Solovians ab \parencite[S. 640]{Phelps1961}. So witzig und amüsant die Geschichte des Artikels, so bahnbrechend ist auch deren Inhalt. Diese Arbeit kann als direkter Anschluss an die Wachstumstheorie Solow's gesehen werden und im Zentrum steht folgende hypothetische Überlegung: Wenn die gesamte aktuelle Wirtschaftsleistung für die Investition (Investition = Sparen!) in neue Produktionsgüter verwendet wird, dann wird nichts für den aktuellen Konsum ausgegeben. Wird hingegen die gesamte aktuelle Wirtschaftsleistung für Konsum verwendet, werden im Umkehrschluss keinerlei neuen Investitionen getätigt. Beide Extrembetrachtungen führen also zu keinem sinnvollen Gleichgewicht. Das heißt aber auch, dass dazwischen irgendein optimales Verhältnis zwischen Sparen/Investieren auf der einen Seite und Konsumieren auf der anderen Seite bestehen muss. Dieses erreicht man eben durch \textit{The Golden Rule of Accumulation}. Diese wird erreicht - solange man einige vereinfachenden Annahmen zulässt - wenn die Wachstumsrate des BIPs dem Zinssatz entspricht. Bereits Phelps nannte diese natürliche Wachstumsrate "`nachhaltig"' \parencite[S. 638]{Phelps1961}. Weiters zeigt Phelps formal, dass diese Wachstumsrate erzielt wird, wenn die Summe der Investitionen der Summe der Profite entspricht, also alle Profite investiert werden. Umgekehrt werden im Optimum alle Löhne konsumiert. Zusammengefasst: Wenn alle Löhne konsumiert werden und alle Profite investiert werden, befindet sich die Ökonomie auf einem nachhaltigen Wachstumspfad. Die Wachstumsrate entspricht dann dem Zinssatz. Insgesamt erinnert das Ergebnis an die Arbeiten von Wicksell und Hayek. Die formale Herleitung durch Phelps war aber zu diesem Zeitpunkt - im Jahre 1961 - eine bahnbrechende Erweiterung des Solow-Wachstumsmodells.

Das bisher in diesem Unterkapitel dargestellte, entspricht noch vollständig dem keynesianischem Denken aus Kapitel \ref{Synthese}. Im Jahr 1966 wechselte Phelps von Yale an die University of Pennsylvania (Penn). Mit dem Umzug konzentrierte er sich auf neue Themen, nämlich auf die theoretische Fundierung der Phillipskurve. Seine Arbeiten dazu sollten später die ersten Zweifel am dominierenden, keynesianschen Framework begründen. Im Nachhinein kann man getrost sagen, dass damit die Grundlagen für den "`Neu-Keynesianismus"' geschaffen wurden.

Wie in Kapitel \ref{sec: Phillips} dargestellt, war der vermeintliche, negative Zusammenhang zwischen Inflation und Arbeitslosigkeit zwar nicht Bestandteil der ursprünglichen keynesianischen Theorie. Aber in weiterer Folge vor allem in der keynesianischen Wirtschaftspolitik ein fixer Bestandteil. Unabhängig voneinander waren Milton Friedman und eben Edmund Phelps bereits ab Mitte der 1960er Jahre die ersten Ökonomen, die den Zusammenhang zwischen Inflation und Arbeitslosigkeit in Frage stellten. Wohlgemerkt zu einer Zeit, in der der Zusammenhang empirisch noch recht gut beobachtet werden konnte. Das in den 1970er Jahren diese Korrelation weitgehend verschwand gab den Kritikern Friedman und Phelps natürlich gehörig Auftrieb. Phelps hatte seine Kritik dabei - im Gegensatz zu Friedman - mathematisch-formal unterlegt. 

Der Artikel mit dem unscheinbaren Titel "`Money-Wage Dynamics and Labor-Market Equilibrium"' \parencite{Phelps1968} stellte die bis dahin unbestrittene Phillipskurve nicht nur infrage, sondern legte die Grundlage für eine ganz neue Sicht auf die Wirtschaftswissenschaften. Interessant ist, dass gleich mehrere Punkte, die natürlich ineinandergriffen, in diesem Artikel revolutionäre waren:
\begin{enumerate}
\item Die Mikrofundierung der Makroökonomie
\item Die formale Einführung der Erwartungen (als adaptive Erwartungen)
\item Die formale Einführung der Natürlichen Arbeitslosigkeit (später NAIRU)
\end{enumerate}
Bemerkenswert ist insbesondere, dass alle drei genannten Punkte bis heute fixer Bestandteil der Mainstream-Modelle sind. Die heutigen DSGE-Modelle sind mikrofundiert, beinhalten das Konzept der Erwartungen (wenn auch der rationalen statt der adaptiven) und akzeptieren einen gewissen Prozentsatz an Arbeitslosigkeit als Gleichgewichtszustand. Natürlich wurden alle drei Konzepte seit 1968 wesentlich erweitert, aber im Gegensatz zu den Arbeiten anderer großen Ökonomen, fällt auf, dass Phelps' Arbeiten bis heute, 50 Jahre später, kaum an Gültigkeit verloren. Keynes' Multiplikator ist heute höchst umstritten, Friedman's Geldmengensteuerung betreibt keine Zentralbank der Welt mehr und selbst die späteren Arbeiten von Robert Lucas wurden größtenteils von der Realität überholt. Phelps' bahnbrechende Erkenntnisse sind hingegen bis heute die Grundlage ökonomischer Modelle und kann daher als Geburtsstunde des "`Neu-Keynesianismus"' gesehen werden.

In seiner Nobelpreis Biographie schreibt Phelps, dass es seit seiner College-Zeit das Gefühl hatte die wichtigste aktuelle Herausforderung der Wirtschaftswissenschaften sei die Integration der Mikroökonomie in die Makroökonomie \parencite{Phelps2006}. Heute nennen wir dies die Mikrofundierung der Makroökonomie.
Der inhaltliche Ausgangspunkt des oben genannten Artikels \parencite{Phelps1968} ist die Phillipskurve. Phelps beschreibt sie als Naivität der Keynesianer. Wobei er Keynes selbst ausdrücklich in Schutz nimmt: Keynes' Nachfragesteuerung wäre niemals soweit gegangen einen dauerhaft stabilen Zusammenhang zwischen Inflation und Arbeitslosigkeit anzunehmen \parencite{Phelps2006}. Phelps stellt stattdessen einen Zusammenhang zwischen der \textit{erwarteteten} Inflation und Arbeitslosigkeit her. Dieser Zusammenhang sei aber nur in der kurzen Frist stabil. Angenommen die erwartete Inflation läge bei 4\%. Arbeitgeber und Arbeitnehmer würden bei ihren Vertragsverhandlungen diese Inflationserwartung einfließen lassen und die Lohnhöhe entsprechend festlegen. Will die Zentralbank nun die Arbeitslosigkeit senken, kann sie Maßnahmen setzen, die die Inflation auf zum Beispiel 6\% erhöhen. Solange die erwartete Inflation unter der tatsächlichen Inflation liegt, wird die Arbeitslosigkeit sinken und sich somit wie von der Phillipskurve postuliert verhalten. Es ist aber klar, dass die Diskrepanz zwischen tatsächlicher und erwarteter Inflation nur kurzfristig aufrechterhalten werden kann, bevor sich die Erwartung dem tatsächlichen Wert anpasst. Die keynesianische, langfristige Phillipskurve wurde durch die neu-keynesianische, kurzfristige erwartungsgestützte Phillipskurve ersetzt. Als solche findet sie bis heute Eingang in die makroökonomischen Lehrbücher. Nebenbei etablierte Phelps dabei das Konzept der adaptiven Erwartungen, das aber später vom neuklassischen Konzept der rationalen Erwartungen abgelöst werden sollte.
Die zentrale Aussage in \textcite{Phelps1968} lautet, dass durch Geldpolitik die Arbeitslosigkeit nicht dauerhaft beeinflusst werden kann, sehr wohl aber unter Umständen in der kurzen Frist. Geldpolitik funktioniere außerdem über Inflations\textit{erwartungen} und diese passen sich recht schnell an die aktuelle Inflation an. Eine niedrige Inflation wird daher auch nicht langfristig zu höherer Arbeitslosigkeit führen\parencite{Phelps1967}. Daraus könnte man ableiten, dass die zentrale Aufgabe der Zentralbanken die Inflationssteuerung ist. Heute orientieren sich fast alle führenden Zentralbanken tatsächlich primär an den Inflationszielen, dies aber direkt auf Phelps' frühe Arbeiten zurückzuführen ginge aber zu weit, folgten doch noch weitere Arbeiten dazu von anderen Neu-Keynesianern und Neuen Klassikern.
Sehr wohl direkte Folge aus \textcite{Phelps1968} ist hingegen die Idee der "`natürlichen Arbeitslosenrate"'. Während die Klassiker davon ausgingen, dass es im Gleichgewicht keine Arbeitslosigkeit gäbe und die Neuen Klassiker meinten im Gleichgewicht gäbe es ausschließlich freiwillige Arbeitslosigkeit, verfolgten die Keynesianer den Ansatz Arbeitslosigkeit sei stets mit nachfrageorientierter Wirtschaftspolitik zu minimieren. Diese "`natürliche Arbeitslosenrate"' wird häufig Milton Friedman zugeschrieben, der einen sehr ähnlichen Ansatz ebenfalls 1968 veröffentlichte \parencite{Friedman1968}. Tatsächlich hatten Friedman und Phelps unterschiedliche Wege gewählt, die sie zu den gleichen Schlussfolgerungen führten.

Die Mikrofundierung der Makroökonomie wird ebenfalls häufig als wesentliche Neuerung der "`Neuen Klassischen Makroökonomie"' gesehen. Es ist auch tatsächlich so, dass die Neuen Klassiker diesen Ansatz als Standard in ökonomischen Modellen etablierten und somit eine wesentliche Neuerung gegenüber den alten Schulen Keynesianismus und Monetarismus einführte. Aber auch hier gilt, dass die erstmalige Anwendung auf Phelps zurückgeht. 










Dieses Ergebnis war natürlich "`schockierend"' für die Keynesianer: Die Phillipskurve war zwar wenig theoretisch begründet, spielte aber in der keynesianisch geprägten Wirtschaftspolitik eine wichtige Rolle. Das Ergebnis, dass Geldpolitik in der langen Frist als nachfrageorientierte Wirtschaftspolitik wirkungslos sei, beschränkte das keynesianische Framework auf Fiskalpolitk.

Die Mikrofundierung der Makroökonomie war ebenfalls ein Schlag für die "`alten"' Keynesianer, da sie ganz andere Wege beschritt-





Zwei wesentliche Artikel:

Artikel: \textcite{Phelps1967},

HIER WEITER: 
https://www.nobelprize.org/prizes/economic-sciences/2006/phelps/biographical/  : Hier: As I began my gradual departure from Penn I th

http://www.columbia.edu/~esp2/autobio1.pdf (Seite 94)
http://www.columbia.edu/~esp2/

Hierfür später den Nobelpreis 2006:
Bahnbrechende Teile darin:

Mikrofundierung der Makro



In den 1970er Jahren schließlich begründete er - als Antwort auf die Neue Klassische Makroökonomie - einen der wesentlichen Punkte der Neu-Keynesianer (mit Taylor und Calvo) unter anderem: "`NAIRU"'


ALT HIER:
dennoch leistete er, gemeinsam mit Milton Friedman die ersten Ideen zur Mikrofundierung der Makroökonomie, die später einer der zentralen Punkte der "`Neuen Klassik"' werden sollte. Er stellte den Zusammenhang von Inflation und Arbeitslosigkeit in Frage mit dem formalen Argument, dass eine Realgröße wie die Arbeitslosigkeit nicht systematisch mit einer Nominalgröße wie der Inflation korrelieren könne. 

Stattdessen ist die \textit{erwartete} Inflation von entscheidender Bedeutung:  "`expectations-augmented Phillips 
curve"' The intertemporal perspective implies that current inflation expectations affect the future 
tradeoff between inflation and unemployment. A higher current inflation rate typically leads to 
higher inflation expectations in the future, so that it then becomes more difficult to achieve the 
objectives of stabilization policy

A key result was that the long-run rate of unemployment cannot be influenced 
by monetary or fiscal policy affecting aggregate demand. Phelps’s analysis thus identified 
important limitations on what demand-management policy can achieve. This view has become 
predominant among macroeconomic researchers as well as policymakers. As a result, 
macroeconomic policy is carried out very differently today from what it was forty years ago.


ALT BIS HIER





Zusammengefasst:

Erster Schritt: Phelps arbeitete zunächst in der Tradition der Keynesianer und entwickelte die "`Goldene Regel der Akkumumlation"'

Zweiter Schritt: Die Emanzipation vom Keynesianismus erfolgte mit seinen Arbeiten zur Revolution der Phillipskurve. Diese umfassten nämlich Konzepte, die bis heute in der Mainstream-Ökonomie State-of-the-Art sind. Erstens, war er ein Vorreiter bei der Mikrofundierung der Makroökonomie und zweitens, etablierte er adaptive "`Erwartungen"' in die Modelle der Ökonomie. Beides spielte später bei den "`Neuen Klassikern"' eine wesentliche Rolle, wenn auch in der Form der \textit{rationalen} statt der \textit{adaptiven} Erwartungen. Er nahm also die Kritikpunkte der Neuen Klassiker am Keynesianismus vorweg.

Dritter Schritt: Gegenbewegung zur Kritik der "`Neuen Klassiker"' inklusive der Entwicklung der "`Natürlichen Arbeitslosigkeit"'

\section{Marktversagen als Teil der Ökonomie}
\label{Marktversagen}

Die Arbeiten von Phelps waren, wie gerade erwähnt, der Ursprung des Neu-Keynesianismus und wurden zeitlich vor der neu-klassischen Revolution formuliert. Die meisten neu-keynesianischen Arbeiten der 1. Generation entstanden allerdings als direkte Antworten auf die aufkommenden aber mit starren Annahmen unterlegten Arbeiten der "`Neuen Klassiker"'.


\subsection{Informationsasymmetrie: Spence, Stiglitz und Akerlof}


\subsection{Natürliche Monopole oder Baumol's angreifbare Märkte}


\section{Arbeitslosigkeit als Suchproblem}
\label{Suchtheorie}

\subsection{Diamond, Mortensen und Pisaridis}

Widersprach rasch der "`Neuen Klassik"' indem sie die natürliche Arbeitslosigkeit erweiterte.




