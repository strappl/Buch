%%%%%%%%%%%%%%%%%%%%% chapter.tex %%%%%%%%%%%%%%%%%%%%%%%%%%%%%%%%%
%
% sample chapter
%
% Use this file as a template for your own input.
%
%%%%%%%%%%%%%%%%%%%%%%%% Springer-Verlag %%%%%%%%%%%%%%%%%%%%%%%%%%

\chapter{Institutionsökonomik}
\label{Institut}

\section{Der Zyniker: Veblen}
Die Institutionsökonomik - man glaubt es heute kaum - war vor 1900 quasi die Mainstream-Ökonomie in den USA \parencite[S. 97]{Persky2000}. Zwischen Thorstein Veblen und dem "`Vater der amerikanischen Neoklassik"', John Bates Clark (vgl. Kapitel \ref{Neoklassik}) entwickelte sich um die Jahrhundertwende so etwas wie der "`Amerikanische Methodenstreit \parencite[S. 100]{Persky2000}.

\section{Die Entdeckung der Empirie: Mitchell \& Kuznets}






