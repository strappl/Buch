%%%%%%%%%%%%%%%%%%%%% chapter.tex %%%%%%%%%%%%%%%%%%%%%%%%%%%%%%%%%
%
% sample chapter
%
% Use this file as a template for your own input.
%
%%%%%%%%%%%%%%%%%%%%%%%% Springer-Verlag %%%%%%%%%%%%%%%%%%%%%%%%%%

\chapter{Neoklassische Finanzierungstheorie}
\label{Finance}

Neoklassische Finance, Modern Finance, schlicht Finance

Bis heute hat die Finance eine interessante Entwicklung durchgemacht. Der Forschungsaufwand war in dieser Disziplin stärker in fast allen anderen ökonomischen Feldern, vor allem im privaten Bereich. Dies ist wenig überraschend, schließlich erhoffen sich viele bis heute durch Finanzanlagen reich zu werden. Dies ist insofern interessant, als die - bis heute gültige - grundlegende Annahme davon ausgeht, dass man zukünftige Kursentwicklungen von Assests nicht vorhersehen kann. Sich diese stattdessen entlang eines reinen Zufallspfades entwickeln. Und auch in der Praxis ist das Angebot an Finanzmarktprodukten zweigeteilt. Zum einen gibt es quantitativ-wissenschaftlich geführte Fonds. Aber daneben gibt es noch immer einen erheblichen Zulauf zu "`Gurus"' oder Anlageberatern, die überzeugt davon sind den Markt schlagen zu können. Die neoklassische Finance wurde und wird aber auch von wissenschaftlicher Seite regelmäßig in Zweifel gezogen. Auch dies ist nicht überraschend: Schließlich hat die wissenschaftliche Weiterentwicklung in diesem Gebiet in keinster Weise dazu beigetragen, die Anzahl von Kursstürzen an den Börsen zu verringern. Weder der "`Schwarze Montag"' im Jahr 1987, noch die "`Dot-Com-Blase"', die im Jahr 2000 platzte, noch der Börsencrash zu Beginn der "`Great Depression"' 2007 passen so recht in das Konzept der "`Effizienten Finanzmärkte"'. Nicht zuletzt deshalb haben sich mit der Behavioral Finance (vgl. Kapitel \ref{Behavioral}) und innerhalb des Post-Keynesianismus (vgl. Kapitel \ref{Post-Keynes}) starke heterodoxe Ansichten zur Finanzierungstheorie gebildet, deren Bedeutung bis heute hoch einzuschätzen ist (vgl. Kapitel \ref{Finanzmarkt}). Die "`Modern Finance"' ist tatsächlich von vielen Seiten her angreifbar und - wie jede Theorie - weit weg davon die Realität vollständig abbilden zu können. Ihr wesentlicher Vorteil liegt aber darin die Konzepte von Nutzen, Ertrag und Risiko unter recht plausiblen Annahmen in mathematisch extrem eleganter Form miteinander zu vereinen. 



\section{Bawerk, Fisher, Marschak und Knight als Vorläufer} \label{FisherundKnight}

\section{Erwartungsnutzen und Arrow's Risikoaversion}
\label{Erwartungsnutzen}
Von-Neumann-Morgenstern-Nutzenfunktion (Querverweis auf Maurice Allais 1952)
St. Petersburg Paradoxon
Mean-Variance-Maß

\section{Modigliani Miller}

\section{Fama}
\label{Efficient}
Vorläufer bereits 1900 (Französischer Name???): log-normalverteilte Renditen und Querverweis Kritik daran durch Mandelbrot.

\section{Markowitz, Tobin, Sharpe, Mossin}
\label{Portfolio}

Chronologisch gesehen ist die wahrlich bahnbrechende Arbeit von \textcite{Markowitz1952} die erste Arbeit der neoklassischen Finanzierungstheorie. Neben \textcite{Modigliani1958} zählt sie somit zu deren "`Gründungsarbeiten"'. Zur Entstehung des Artikels "`Portfolio Selection"', der im wesentlichen auch die Doktorarbeit von Harry Markowitz darstellt, gibt es unzählige Geschichten. So erzählt Markowitz in einem Interview\footnote{Die Stelle findet sich hier: https://www.youtube.com/watch?v=RVWEhCd819E, Minute 00:50. In diesem Interview findet sich auch die Anekdote mit der Defensio bei Milton Friedman und  Jacob Marschak.}, dass eine Börsenhändler ihm den Tipp gegeben hätten sich einem Finanzthema zu widmen. Eine Anekdote von der abschließenden Defensio seiner Doktorarbeit gab Markowitz im Rahmen seiner Nobelpreis-Lectures zum Besten: "`Professor Milton Friedman argumentierte, dass Portfolio-Theorie kein Teil der Ökonomie sei und sie ihm daher keinen Doktortitel in Ökonomie für eine Dissertation [dafür] geben können."' \parencite[S. 286]{Markowitz1990}. Beide Geschichten machen deutlich wie bahnbrechend seine Arbeit im Jahr 1952 gewesen ist. Etwas das auch \textcite{Rubinstein2002} im Rahmen des 50-Jahr Jubiläums von "`Portfolio Selection"' hervorhob: "`Am beeindruckendsten an Markowitz' 1952 Artikel fand ich, dass er aus dem Nichts zu kommen scheint"'. Tatsächlich lautete, leicht übertrieben, die Prämisse auf den Finanzmärkten vor 1952: Suche die Aktie von der du dir die höchste Rendite erwartest und kaufe sie. Das Konzept der naiven Diversifikation ist schon seit jeher bekannt und auch das Konzept vom Risiko-Rendite-Trade-Off war nicht neu. Aber es gab keine quantitativ-mathematischen Ansätze zur Formalisierung dieser Konzepte. Dies ist einigermaßen überraschend. Wertpapierhandels gibt es schließlich schon seit Jahrhunderten.  Und - anders als in der Makroökonomie - haben selbst die "`Great Depression"', bzw. die Kursverluste in Folge des "`Schwarzen Donnerstags"' im Jahr 1929, keinen Durchbruch in diesem Bereich ausgelöst. Warum war die Arbeit nun so bahnbrechend? Nun, \textcite{Markowitz1952} war die erste rein \textit{mathematisch-quantitative} Arbeit im Bereich der Finanzmarktanalyse. Als solche wurden darin gleich drei wesentliche Konzepte etabliert, deren Bedeutung bis heute unumstritten ist: Erstens, die Varianz der Aktienrenditen wurde als Risikomaß. Zweitens, der Risiko-Rendite-Trade-Off - also die Annahme, dass höhere erwartete Rendite immer auch mit höherem Ausfallsrisiko verbunden ist, und drittens, Die Bedeutung der Korrelation von Aktienrenditen. Letzteres ist nichts anderes als die mathematische Fundierung der Diversifikation. Dieser letzte Punkt wird häufig als der wesentliche bezeichnet und tatsächlich basiert darauf die zentrale Idee des Artikels: Aktienportfolios aus der Kombination von Einzeltiteln zu bilden, die das optimale Verhältnis zwischen Rendite und Risiko abbilden. Was heißt das konkret? Wenn Sie eine Aktie kaufen so erwarten sie in Zukunft eine Rendite von x\%. Diese Erwartung bildet sich aus den vergangenen Renditen dieser Aktie. Da die Rendite bei Aktien aber nie konstant ist, sondern zufälligen vgl. Kapitel \ref{Efficient} - Schwankungen unterliegt, ist diese Rendite stets nur eine Erwartung. Aus den vergangenen Schwankungen lässt sich eine durchschnittliche Schwankung - die Standardabweichung - berechnen. Diese wird in weiterer Folge als Risikomaß herangezogen. Je stärker der Wert der Aktie schwankt, desto schwieriger ist deren zukünftiger Wert zu prognostizieren. Oder mit anderen Worten: Desto höher ist ihr Risiko. Stellen Sie sich nun \textit{zwei} Aktien vor. Für beide können sie einen Rendite-Erwartungswert, sowie eine Standardabweichung berechnen. Wenn Sie beide Aktien zu gleichen Teilen kaufen, so entspricht ihr Rendite-Erwartungswert dieses Portfolios dem Mittelwert der Renditen der beiden Aktien. Für die Berechnung der Standardabweichung stimmt dies aber \textit{nicht}! Die Renditen von Aktien verlaufen niemals genau gleich. Das heißt sie korrelieren niemals zu 100\%. Viele Aktien bewegen sich zwar tendenziell in die gleiche Richtung, aber manche Assets sind unabhängig von anderen, bzw. korrelieren sogar negativ. Das heißt bei der Berechnung des Risikos des Portfolios reicht es nicht aus einfach den Mittelwert der Risiko-Werte der Einzeltitel heranzuziehen. Stattdessen muss auch die Korrelation zwischen den beiden Titeln berücksichtigt werden. Wenn zwei Titel perfekt negativ miteinander korrelieren (was ebensowenig vorkommt wie perfekt positive Korrelation), dann steigt eine Aktie immer dann wenn die andere fällt. Dies ist gut für das Portfolio-Risiko: Da der Gewinn der einen Aktie den Verlust der zweiten Aktie immer zumindest teilweise ausgleicht. Die Standardabweichung des Portfolios ist daher stets geringer als der gewichtete Durchschnitt der Standardabweichung der einzelnen Aktien. \textcite{Markowitz1952} zeigte dies erstmals mathematisch. In der Folge kann man natürlich Portfolios aus vielen Einzeltitel zusammenstellen. Sogenannte "`Effiziente Portfolios"' sind aber nur solche, bei denen für eine gegebene Rendite keine niedrigere Standardabweichung erzielt werden kann. Das heißt, es gibt bei Markowitz nich das \textit{eine} optimale Portfolio, sondern eine Reihe von optimalen Portfolios. Diese liegen allesamt auf der "`Efficient Frontier"'. Rationale Individuen sollten nur solche Portfolios erwerben. Welches genau hängt bei \textcite{Markowitz1952} noch von der individuellen Risikoaversion des Investors ab. Die Arbeit gilt heute, 70 Jahre später, noch immer als Ausgangspunkt für quantitatives Asset-Management. 

Mit einer recht intuitiven Idee wurde die Markowitz-Portfoliotheorie durch \textcite{Tobin1958} erweitert. Und zwar indem er die Berücksichtigung eines risikolosen Assets einführte. Die Markowitz-Portfoliotheorie behandelt ausschließlich risikobehaftete Assets. Wenn man jetzt zum Beispiel eine risikolose Anleihe heranzieht, so hat diese einen bestimmten Erwartungswert und eine Standardabweichung von - definitionsgemäß - Null. Dieser Anleihe wird als Fixpunkt betrachtet. Ausgehend von diesem Fixpunkt wird nun eine Tangente an die "`Efficient Frontier"' gelegt. Per Definition berührt diese Gerade die Efficient Frontier nur in einem einzigen Punkt. Dieser Punkt stellt das tatsächlich einzige optimale Portfolio - genannt "`Marktportfolio"' - dar. Die Verbindungslinie zwischen risikoloser Anleihe und Marktportfolio nennt man die "`Kapitalmarktlinie"' (Capital Market Line). Das Marktportfolio ist in diesem theoretischen Konstrukt das einzig sinnvolle Portfolio. Unabhängig von der Risikoaversion kann nämlich aus der Kombination aus risikoloser Anleihe und Marktportfolio stets eine höhere Rendite-Erwartung (bei fixiertem Risiko) erzielt werden, als auf einem beliebigen Punkt auf der Efficient Frontier. Diese Erkenntnis wurde als die \textsc{Tobin-Separation} bekannt. Der Name Separation steht hierbei dafür, dass bei Finanzinvestitionen zwei voneinander unabhängige Entscheidungen getroffen werden müssen. Erstens, es muss das Marktportfolio ermittelt werden. Dieses ist allerdings für jeden risiko-adversen Investor identisch. Zweitens, abhängig von der individuellen Risikoaversion muss ein Investor entscheiden welchen Anteil seines Vermögens er in das risikobehaftete Marktportfolio steckt und welchen Anteil in die risikolose Anleihe. Auch dieses Verhältnis kann man übrigens - für jedermann individuell - quantitativ berechnen. Für jedes risiko-adverse Individuum lässt sich eine Nutzenfunktion ermitteln. Die individuelle Risikoaversion (vgl. Kapitel \ref{Erwartungsnutzen}) kann als "`Mean-Variance"'-Maß\footnote{"'Mean"' bezeichnet hierbei der Erwartungswert der Renditen und "`Variance"' die Varianz, also das Quadrat der Standardabweichung und damit das Risiko.} ausgedrückt werden. Die Nutzenfunktion lässt sich somit mittels Indifferenzkurven in das Portfolio-Diagramm überführen. Der Tangentialpunkt von Indifferenzkurve und Kapitalmarktlinie bestimmt das optimale, individuelle Verhältnis zwischen risikoloser Anleihe und Marktportfolio.

HIER WEITER: CAPM


Sharpe, Lintner, Mossin, Treynor
Black-Litterman







\section{Black-Scholes, Merton}


