%%%%%%%%%%%%%%%%%%%%% chapter.tex %%%%%%%%%%%%%%%%%%%%%%%%%%%%%%%%%
%
% sample chapter
%
% Use this file as a template for your own input.
%
%%%%%%%%%%%%%%%%%%%%%%%% Springer-Verlag %%%%%%%%%%%%%%%%%%%%%%%%%%

\chapter{Neue Institutionsökonomik}
\label{Neue Institut}

\section{Der "`alte"' Institutionalismus}

\subsection{Der Zyniker: Veblen}
Die Institutionsökonomik - man glaubt es heute kaum - war vor 1900 quasi die Mainstream-Ökonomie in den USA \parencite[S. 97]{Persky2000}. Zwischen Thorstein Veblen und dem "`Vater der amerikanischen Neoklassik"', John Bates Clark (vgl. Kapitel \ref{Neoklassik}) entwickelte sich um die Jahrhundertwende so etwas wie der "`Amerikanische Methodenstreit \parencite[S. 100]{Persky2000}.

\subsection{Die Entdeckung der Empirie: Mitchell \& Kuznets}


\section{Die Ursprünge: Transaktionen sind nicht gratis} \label{sec: Neue Inst}

Ronald Coase: Eigentumsrechte statt Pigou-Steuer: Kritik an Pigou \textcite[S. 243]{Cansier1989}. Coase stellt im Hinblick auf externe Effekte - wie die gerade aktuell diskutierten Umweltprobleme - Marktlösungen in den Vordergrund. Ein Staatseingriff ist ihm zufolge nicht unbedingt notwendig (Vergleiche dazu Kapitel \ref{sec: Pigou}).

Kenneth Arrow 1951: Unmöglichkeitstheorem und Social Choice Theorie



Es ist eine interessante Tatsache, dass als Ausgangspunkt für die "`Neue Institutionsökonomik"' immer wieder das Werk von Coase: \textit{Theory of the Firm} aus dem Jahr 1937 genannt wird. Interessant deshalb, weil zwischen diesem Ausgangspunkt und den weiteren Arbeiten im Bereich Transaktionskostentheorie -  oder Neue Institutionsökonomik überhaupt - ungefähr 30 Jahre vergingen \parencite[S. 148]{Blaug2001}.

Coase, Williamson, North, Olson (Querverbindung auch zu Neuer Politischer Ökonomie)

Prinzipal Agent Theorie


Ostrom (Querverbindung auch zu Neuer Politischer Ökonomie dort erwähnt!)

\section{Acemoglu: Kein Wohlstand ohne Institutionen}
Verbindung zu Endogener Wachstumstheorie \textcite[S. 633ff]{Snowdon2005}

