%%%%%%%%%%%%%%%%%%%%% chapter.tex %%%%%%%%%%%%%%%%%%%%%%%%%%%%%%%%%
%
% sample chapter
%
% Use this file as a template for your own input.
%
%%%%%%%%%%%%%%%%%%%%%%%% Springer-Verlag %%%%%%%%%%%%%%%%%%%%%%%%%%

\chapter{Neue Institutionenökonomik}
\label{Neue Institut}

\section{Der "`alte"' Institutionalismus des Zynikers Thorstein Veblen}

Mit dem Übergang von der Klassik zur Neoklassik, ging ein ganz wesentliches Element verloren, nämlich die Analyse der Bedeutung von Regeln, Gewohnheiten, Behörden und dergleichen. Diese Begriffe in ihrem weitesten Sinne zusammengefasst, bezeichnete man später als Institutionen. Damit umfasst sind also nicht nur Institutionen im heutigen Wortsinn, wie etwa Behörden und Gerichte, sondern auch zum Beispiel aufgestellte Regeln wie Gesetze oder aber auch gewachsene, ungeschriebene Regeln, wie Usancen. In diesem Sinne sind alle Nicht-Individuen in der ökonomischen Theorie Institution. Also auch Unternehmen, das Geld, Gesetze oder Gebietskörperschaften. In \textcite[S. 179]{Hodgson1998} werden Institutionen definiert als "`Denk- oder Handlungsweise von gewisser Prävalenz und Dauer, die in den Gewohnheiten einer Gruppe oder in den Bräuchen eines Volkes verankert ist."' \textcite[S. 239]{Veblen1919} definierte Institutionen noch allgemeiner als "`Feste Denkgewohnheiten ('Habits'), die der Allgemeinheit der Menschen gemeinsam sind."'

Die Neoklassik reduzierte die Ökonomie weitgehend auf die Mechanik von Angebot und Nachfrage. Wir haben bereits in Kapitel \ref{Neoklassik} festgestellt, dass dabei wichtige Schritte als gegeben angenommen werden. So bestimmt sich in der Neoklassik der Marktpreis aus dem Gleichgewicht von Angebot und Nachfrage. Aber was heißt "`bestimmt sich"'? Wie soll man sich diesen "`spontanen Prozess"' vorstellen? Walras - der Urvater der Gleichgewichtstheorie - verwendete recht schwammig den Begriff Tatonnement dafür. Es war und ist auch den Neoklassikern stets bewusst, dass der Prozess der Gleichgewichtsfindung nicht spontan, sondern durch handelnde Personen und Prozesse erfolgt und damit Kosten verursacht. Allerdings können diese in den mechanischen, neoklassischen Modellen nicht so einfach integriert werden, sodass die Bedeutung von Institutionen in der Neoklassik lange ignoriert wurde. Der Konflikt zwischen Neoklassikern und anderen Wirtschaftswissenschaftlern, ob diese Nicht-Berücksichtigung jeglicher "`Nebenkosten"' zulässig ist, entstand aber praktisch schon gleichzeitig mit der Marginalistischen Revolution. So behielten unter anderem bereits Carl Menger selbst - also einer der drei "`Hauptdarsteller"' der Marginalistischen Revolution -, vor allem aber seine Nachfolger in der Österreichischen Schule die Problematik institutioneller Prozesse stets auf der Agenda. Institutionen waren daher Anfang des 20. Jahrhunderts kein gänzlich neues ökonomisches Forschungsgebiet. Vielmehr wurden wesentliche Elemente davon gerade im deutschsprachigen Raum schon zuvor, aber auch später, häufig wissenschaftlich diskutiert. Die Historische Schule der Nationalökonomie (vgl. Kapitel \ref{Historisch}), aber auch die frühen Vertreter der Österreichischen Schule (vgl. \ref{Austria}), sowie später die deutschen Ordoliberalen (vgl. \ref{Ordoliberalismus}) griffen ganz ähnliche Fragestellungen auf, wie um die Jahrhundertwende der Institutionalismus. Der Begriff "`Institutionalismus"' wurde erstmals von \textcite{Hamilton1919} verwendet und später, zur Abgrenzung gegenüber dem "`Neuen Institutionalismus"', zu "`alter (oder amerikanischer) Institutionalismus"' weiterentwickelt. Bekannt unter diesem Namen wurde im frühen 20. Jahrhundert eine kleine Gruppe amerikanischer Ökonomen, die die Bedeutung von Institutionen so weit in den Vordergrund stellte, dass sie ihre Erkenntnisse mit den Modellen der Neoklassik für vollkommen unvereinbar hielt. Diese Frage der Vereinbarkeit ist übrigens eine Frage, die den Institutionalismus bis heute beschäftigt und deren Vertreter zuweilen spaltet: Können institutionelle Gegebenheiten in neoklassischen Modellen berücksichtigt werden? Die Neuen Institutionalisten beantworten diese Frage weitgehend mit ja. Demnach sind die Modelle und Annahmen der Neoklassik zutreffend, müssen aber um institutionelle Erkenntnisse erweitert werden. Der alten Institutionalismus lehnte die Neoklassik hingegen als Ganzes weitgehend ab. Den neoklassischen, mathematisch-physikalisch-mechanischen Ansätzen stellten die alten Institutionalisten Analogien aus der Biologie oder den Rechtswissenschaften gegenüber. Demnach entwickelten sich wirtschaftliche Prozesse und Regeln über Jahrtausende aus evolutionären Prozessen zu dem weiter, was sie heute sind. 

Der hier dargestellte "`alte Institutionalismus"' hat aus diesem Grund mit dem eigentlichen Hauptthema des Kapitels, dem "`Neuen Institutionalismus"', nicht allzu viel gemeinsam, abgesehen vom Namen. Aus heutiger Sicher kann der Hauptvertreter Thorstein Veblen und dessen Arbeit eher als Kuriosität der Wirtschaftsgeschichte angesehen werden. Obwohl er letztendlich gar kein eigenständiges Theorie-Gebilde zum Institutionalismus schaffen konnte, sondern eigentlich "`nur"' die Schwächen der Neoklassik offenlegte, gehört er zu den bekanntesten amerikanischen Ökonomen des frühen 20. Jahrhunderts. Mehr noch: Die alte Institutionsökonomik - man glaubt es heute kaum - war zwischen 1900 und 1920 quasi die Mainstream-Ökonomie in den USA \parencite[S. 97]{Persky2000} \parencite[S. 166]{Hodgson1998}. Zwischen Thorstein Veblen und dem "`Vater der amerikanischen Neoklassik"', John Bates Clark (vgl. Kapitel \ref{FisherandClark}) entwickelte sich um die Jahrhundertwende so etwas wie der "`Amerikanische Methodenstreit"', der in Form mehrere methodischer Artikel beider Seiten, vor allem im damals noch jungen Quarterly Journal of Economics, ausgetragen wurde \parencite[S. 100]{Persky2000}. In den USA hatte die Neoklassik damals noch einen schweren Stand, während der Institutionalismus sich im Aufwind befand \parencite[S. 100]{Persky2000}. Eine Entwicklung, die aber rasch eine Ende fand. 

Der bekannteste Vertreter des alten Institutionalismus ist sicherlich Thorstein Veblen. Die Inhalte, die er beschrieben hat, sind aber nicht so einfach zu fassen, bzw. in ein einheitliches Denkmuster zu bringen. Unumstritten ist seine klare Ablehnung und fundamentale Kritik der neoklassischen Ansätze \parencite[S. 703]{Blaug1962}. Dies alleine macht aber noch keine ökonomische Denkrichtung aus. Veblen's Ideen waren eindeutig geprägt von der Historischen Schule der Nationalökonomie (vgl. Kapitel \ref{Historisch}). Veblen sprach, als Nachfahre norwegischer Einwanderer, Deutsch und auch Französisch \parencite[S. 418]{Hodgson1998b}. Europäische Arbeiten waren ihm dementsprechend leichter zugänglich. Von den deutschen Ökonomen übernahm er auch die Idee jegliches ökonomisches Handeln aus einer evolutionstheoretischen Perspektive zu betrachten. Ebenso wie Darwin die Natur als Evolution erklärte, entwickelten sich demnach menschliche Handlungsweisen über die Jahrhunderte hinweg. Institutionen als zentrales Ergebnis dieser Evolutionsprozesse, sind also gewachsene Strukturen, die sich stets weiterentwickeln \parencite[S. 424]{Dugger1979}. Preise ergeben sich demnach nicht primär aus Angebot und Nachfrage, sondern aus Konventionen, Gewohnheiten und formellen Institutionen. \textcite{Veblen1899} erklärte damit die oft Generationen-übergreifende Existenz von Reichtum und Armut. Nachkommen aus armen Familien wären es demnach gewöhnt arm zu sein und agierten zeitlebens dementsprechend. Umgekehrt agieren reiche Personen ebenso ihrem Stand entsprechend und weitgehend unabhängig vom tatsächlichem Einkommen. \textcite{Duesenberry1949} entwickelte darauf aufbauend nach dem Zweiten Weltkrieg eine Konsumtheorie, die dieses Verhalten abbildet. Obwohl diese empirisch gute Ergebnisse lieferte \parencite[S. 170]{Hodgson1998}, geriet sie weitgehend in Vergessenheit. In den Sprachschatz der modernen Ökonomie haben es in diesem Zusammenhang allerdings die Begriffe "`Veblen-Effekt"' und "`Geltungskonsum"' geschafft. Der "`Veblen-Effekt"' liefert einen Erklärungsversuch für das selten auftretende Phänomen, wenn steigende Preise zu steigender Nachfrage führen. Demnach ist mit dem Besitz dieser speziellen Güter ein gewisser Status verbunden. Die höheren Preise weiten deren Position als Statussymbol noch aus. Der eigentlich zu erwartende negative Effekt auf die Nachfrage durch Preiserhöhungen wird vom gegenläufigen Veblen-Effekt übertroffen, sodass steigende Preise in speziellen Fällen eben sogar höhere Nachfrage verursachen. Jeder von uns erinnert sich in diesem Zusammenhang an \textit{das eine} Produkt aus seiner Kindheit oder Jugend, als man die bestimmten Schuhe, Hosen oder Accessoires einfach haben \textit{musste}, wenn man \textit{in} sein wollte. Damit eng verbunden ist die Theorie des "`Geltungskonsums"', die Veblen ebenso in seinem bekanntesten Werk "`The Theory of the Leisure Class"' \parencite{Veblen1899} entwickelte. Diese Theorie behauptet, dass Personen viele Güter primär deshalb konsumieren, weil sie ihren Status öffentlich darstellen wollen, bzw. um ihrem Status entsprechend aufzutreten. 

Genau dieses Werk \parencite{Veblen1899} wurde ein unmittelbarer Erfolg und vor allem auch außerhalb der Wirtschaftswissenschaften in intellektuellen Kreisen ein weit rezipiertes Werk. Die Kernthesen seines "`Institutionalismus"' finden sich aber primär gar nicht in seinem Hauptwerk, sondern vor allem in den wesentlich weniger beachteten Werken \textcite{Veblen1904} und \textcite{Veblen1923}, sowie verschiedenen Artikeln, die später auch in Sammelwerken zusammengefasst wurden (\textcite{Veblen1919}). Der Ausgangspunkt seines "`Evolutionären Institutionalismus"' findet sich in seinem Artikel \textcite{Veblen1898}: "`Why is Economics not an Evolutionary Science?"'. Veblen kritisiert darin die Individuen-Bezogenheit der Neoklassik. Dort ist der nutzenmaximierende Agent der Ausgangspunkt jeder Analyse. Dieser Kritikpunkt blieb zentral im "`alten Institutionalismus"': Die völlige Fehleinschätzung der menschlichen Natur ("`human nature"'). Dies warf Veblen nicht nur den Neoklassikern, sondern auch den anderen führenden Schulen jener Zeit, der historischen Schule der Nationalökonomie und der Österreichischen Schule, vor \parencite[S. 389]{Veblen1898}. Aber auch den Marxismus mit dessen Fokus auf das Kollektiv kritisierte er als unrealistisch. Er lehnte also sowohl den methodischen Individualismus wie auch den methodischen Kollektivismus strikt ab \parencite[S. 426]{Hodgson1998b}. Dem nutzenmaximierenden Individuum der Neoklassik stellt Veblen die Institutionen, die sich evolutionär entwickelt haben, als zentrales Element gegenüber \parencite[S. 422]{Hodgson1998b}. In Bezug auf die Individuen gehen die "`alten Institutionalisten"' ebenso von einer Evolution aus. Im Gegensatz zur Neoklassik unterliegen Individuen und deren Präferenzen einem ständigen Anpassungsprozess, was vor allem mit den Modellen zu Nachfrage- und Konsumverhalten der Neoklassik unvereinbar ist \parencite[S. 701]{Blaug1962}. Als Resultat beeinflussen sich individuelles Verhalten und institutionelle Strukturen gegenseitig: Institutionen formen Individuen, werden aber umgekehrt auch von Individuen geformt \parencite[S. 181]{Hodgson1998}. In \textcite{Veblen1904} und in seinem letzten Werk \textcite{Veblen1923} legt er den Fokus auf eine etwas andere Thematik, wobei er interessanterweise den technischen Fortschritt in den Mittelpunkt stellt. Etwas, das die neoklassische Theorie damals noch nicht kannte. Während die Unternehmer ausschließlich die Profitmaximierung in den Vordergrund stellen, sorgen die Entwickler für wirtschaftlichen Fortschritt. Die Institutionen wiederum bremsen diesen zumindest kurzfristig, da diese durch Gewohnheiten und gewachsene Handlungen geprägt sind und Neuerungen nur verzögert zulassen \parencite[S. 34]{Erlei2016}. Insgesamt ist Veblen's Spätwerk aber schwer verständlich und blieb zu Veblens Lebensende wenig gelesen und ohne Einfluss auf die zeitgenössische Ökonomie.

Veblen war, nicht nur was seine wirtschaftswissenschaftlichen Inhalte anging, sehr speziell, sondern auch seine Person betreffend. Mehrmals musste er Universitäten wegen "`unpässlichem Verhalten"' - konkret wegen außerehelicher Beziehungen - verlassen. 1906 die Universität Chicago und 1909 Stanford. Danach war er noch bis 1918 an der weit weniger bedeutenden Universität Missouri tätig. Zudem war sein Schreibstil höchst ungewöhnlich. \textcite{Veblen1899} zum Beispiel ist voller Ironie und Satire. Beim Lesen der Inhalte stellt man deren Ernsthaftigkeit eher in Frage. Gegen Ende seiner Karriere gerieten er und sein Werk in den Wirtschaftswissenschaften in Vergessenheit und er starb 1929 weitgehend unbeachtet. Random Fact am Rande: Laut \textcite{Aspromourgos1986} geht der allgegenwärtige Begriff "`Neoklassik"' ursprünglich auf Thorstein Veblen zurück. Zumindest diesbezüglich wirkt seine Arbeit bis heute nach. Aber auch abgesehen davon, erlebte Veblen's Arbeit im Rahmen der Weltwirtschaftskrise nach 1929 ein Revival. Er genießt heute einen verhältnismäßig hohen Bekanntheitsgrad, was doch überraschend ist, angesichts der Tatsache, dass er es nicht schaffte eine geschlossene Theorie des Institutionalismus zu verfassen. Im Gegensatz dazu ist etwa die "`Freiburger Schule"', die durchaus eigene Theorie-Gebilde in inhaltlich ähnlichen Bereichen schuf, heute international vollkommen unbedeutend. 

Als zweiter Begründer des "`alten Institutionalismus"' gilt John Rogers Commons, dessen Hauptwerke (\textcite{Commons1924}, \textcite{Commons1934}) allerdings erst nach Veblen's Wirken publiziert wurden. Obwohl Commons nur fünf Jahre später als Veblen geboren wurde, überschnitten sich ihre Schaffensperioden in Bezug auf Arbeiten zum Institutionalismus nicht. Bei Commons stand dennoch ebenso die Entwicklung von Institutionen im Sinne von gewachsenen Gewohnheiten im Vordergrund. So argumentierte \textcite[S. 45]{Commons1934}, dass erlernte Fähigkeiten, die von einer Gruppe oder sozialen Gemeinschaft ausgeübt werden, zu Routinen wachsen und umgekehrt, Routinen und Gewohnheiten von dieser Gruppe wieder weitergegeben werden, wodurch sie sich als Institutionen etablieren \parencite[S. 180]{Hodgson1998}. Commons ergänzte dabei das Konzept der individuellen "`Gewohnheiten"' (habits) um jenes der kollektiven "`Sitten und Gebräuche"' (customs) \parencite[S. 556ff]{Hodgson2003}. Davon abgesehen unterschied sich das Werk Commons' deutlich von Veblen's Beitrag. Commons ist heute für seine Verbindung zwischen den Wirtschafts- und den Rechtswissenschaften bekannt. Zudem wird sein Werk von "`Neuen Institutionalisten"' weniger stark abgelehnt als jenes von Veblen \parencite[S. 547]{Hodgson2003}. Aber auch er schaffte es in keinster Weise eine einheitliche Theorie des Institutionalismus zu begründen. Ein wesentliches Merkmal in Commons' Werk ist, dass er den Begriff Institutionen etwas enger sah als etwa Veblen und dafür die Bedeutung von Gesetzen als institutionellen Rahmen in den Mittelpunkt stellte. Formales Recht entsteht für ihn zwar häufig aus informellen Sitten und Gebräuchen oder sozialen Abmachungen heraus, allerdings hob er in diesem Zusammenhang die stabilisierende und regulatorische Wirkung des Rechtssetzung-Aktes hervor \parencite[S. 566]{Hodgson2003}. Er grenzte Recht damit von "`spontanen Ordnungen"' ab, akzeptiert zwar, dass es beide Formen - formale Regeln (formale Institutionen) und ungeschriebene Gesetze ("`organische"' Institutionen) - gibt, stellte aber formale Institutionen in den Vordergrund \parencite[S. 568]{Hodgson2003}. Commons sah Institutionen damit primär als "`strukturierte Organisation individueller Willen, die in einem sich entwickelnden Rechtsapparat agieren"' \parencite[S. 568]{Hodgson2003}.

Commons wird oftmals auch als Pionier der Labor-Economics (Arbeits-Ökonomie) bezeichnet und forschte zunächst vor allem zur Bedeutung von Gewerkschaften, Kollektivvertrags-Verhandlungen und sozialer Sicherheit \parencite[S. 44]{Barbash1989}. Im Gegensatz zur modernen Arbeits-Ökonomie lag der Fokus seiner Arbeit allerdings darauf, wie Arbeitnehmerrechte in der Gesetzgebung am besten zu verankern seien. Der Institutionalismus und die Arbeits-Ökonomie dieser Prägung stehen somit nicht nebeneinander als zwei unterschiedliche Forschungsrichtungen, sondern greifen ineinander. Die Karriere von Commons verlief in ihrer Frühphase alles andere als linear. Er arbeitete zunächst als Lehrer, entschied sich aber dann doch dazu bei Richard T. Ely Ökonomie zu studieren. Dieser genoss sein Ökonomie-Studium in Deutschland und war daher geprägt von der "`Historischen Schule der Nationalökonomie"' \parencite{Watkins}, was wiederum Commons Werk beeinflusste. Commons Ideen zu Labor-Economics galten im frühen 20. Jahrhundert in den USA als zumindest unorthodox, wenn nicht sogar links-radikal. Er hatte daher lange Zeit Probleme eine fixe universitäre Anstellung zu bekommen \parencite{Watkins}. Nach mehreren kurzen Intermezzos erhielt er schließlich 1904 eine Professur an der University of Wisconsin, die er bis 1932 behielt. Dort verfasste er auch seine Hauptwerke (\textcite{Commons1924} und \textcite{Commons1934}). Seine Karriere nahm - im Gegensatz zu jener von Veblen - mit fortschreitendem Alter an Fahrt auf. Ein Glücksfall für Commons war, dass er eine Zusammenarbeit mit dem Gouverneur von Wisconsin, Robert M. Folletten, etablieren konnte. Dieser gehörte zu den progressiven Kräften in der amerikanischen Politik zu jener Zeit. Wisconsin wurde so etwas wie ein Labor für Sozialpolitik \parencite[S. 44]{Barbash1989}. Commons etablierte Interessenvertretungen für Arbeitnehmerangelegenheiten ("`American Association for Labor Legislation"') und die US-Politik erkannte zunehmend die Bedeutung der Sozialgesetzgebung. Er war Mitglied der in den 1920er-Jahren von Präsident Wilson gegründeten "'Industrial Relations Commission"' \parencite[S. 45]{Barbash1989}. Seine Ausarbeitungen wurden als Vorlage für die ersten Gesetze zum Arbeitslosengeld herangezogen. Als im Rahmen der "`Great Depression"' wirtschaftsfördernde Maßnahmen allgemeine an Bedeutung zunahmen, wurden auch Commons' Ideen salonfähig. Er und seine Schüler gelten als eine treibende Kraft bei der Entwicklung der Sozialgesetze, die im Rahmen des "`New Deals"' umgesetzt wurden. In den 1930er-Jahren war die Neoklassik auch in den USA bereits zur Mainstream-Ökonomie aufgestiegen. Diese konnte mit Commons' Ideen zu Gewerkschaften, Kollektivverträgen und Arbeitsgesetzen wenig anfangen. Auf der anderen Seite war Commons aber auch weit von marxistischen Ideen entfernt \parencite[S. 47]{Barbash1989}. Seine Arbeit blieb zeitlebens der heterodoxen Ökonomie zuzuordnen. Er unterschätzte die Erklärungskraft des Konzepts der Nutzenmaximierung und verzettelte sich stattdessen in vagen Konzepten wie dem Voluntarismus und seinem Glauben, dass sich langfristig "`vernünftige"' Lösungen durchsetzen würden \parencite[S. 48]{Barbash1989} Sein politischer Einfluss ist damit aus heutiger Sicht höher einzuschätzen als sein Beitrag zur ökonomischen Theorie. 

Den "`alte Institutionalismus"' vereinten folgende Grundsätze: Die Ablehnung des individuellen Nutzenmaximierers, die Interdisziplinarität, die Ansätze aus Politik, Biologie, Soziologie und Rechtswissenschaften übernahm. Außerdem nahmen, im Gegensatz zur Neoklassik, mathematisch-mechanische Modelle einen geringen Stellenwert ein. Stattdessen wurde aus "`stylized facts"' und historisch-empirischen Daten Schlüsse gezogen. Kritiker des Ansatzes bezeichneten dieses Vorgehen als rein deskriptive Analyse von Daten. Das ganze Konzept sei rein induktiv und ohne Fokus darauf Theorien bilden zu \textit{wollen} \parencite[S. 703]{Blaug1962}, was Historiker versuchten zu widerlegen \parencite[S. 424]{Hodgson1998b} \parencite[S. 174]{Hodgson1998}. Faktum ist, dass der "`alte Institutionalismus"' kein eigenes Theoriegebilde darstellt. Ganz im Gegenteil, selbst die unterschiedlichen Beiträge der beiden Hauptvertreter Veblen und Commons verbindet nur wenig gemeinsames \parencite[S. 701f.]{Blaug1962}. 

In Bezug auf die gegenwärtigen Wirtschaftswissenschaften interessanter ist die Tatsache, dass die heute nicht mehr wegzudenkende Bedeutung der empirischen Wirtschaftsforschung - vor allem im Sinne der Sammlung und Erhebung von empirischen Daten - auf einen Vertreter der "`alten Institutionenökonomik"' zurückgeht, nämlich Wesley Clair Mitchell. Mitchell gilt nicht als "`typischer"' Vertreter dieser Schule, wird aber als Schüler Veblen's meist dieser zugerechnet. Mitchell ist Begründer des National Bureau of Economic Research (NBER) - einer der bis heute wichtigsten wirtschaftswissenschaftlichen Institutionen der USA. In einer Zeit, die reich an bedeutenden Theoretikern war, aber in der Empirie kaum eine Rolle in der Ökonomie spielte - erkannte er die Relevanz hochwertiger Daten. Im 1920 gegründeten NBER war er bis 1945 als Forschungsdirektor tätig. Inhaltlich, lehnte Mitchell den evolutionären Ansatz des alten Institutionalismus seines Lehrers Veblen weitgehend ab \parencite[S. 552]{Hodgson2003}. Stattdessen machte er sich einen Namen als Theoretiker im Bereich der Konjunkturzyklen \parencite{Mitchell1913, Mitchell1946}. Wiederum ein Schüler Mitchell's setzte dessen Pionierarbeit im Bereich der empirischen Wirtschaftsforschung fort: Simon Kuznets. Der aus Russland stammende Kuznets arbeitete gemeinsam mit Mitchell am NBER und beschäftigte sich bahnbrechend mit einem Thema, das bis heute allgegenwärtig ist in der Volkswirtschaftslehre \parencite[S. 172]{Hodgson1998}: Er prägte den Begriff des Bruttonationalproduktes. Mit \textcite{Kuznets1937} etablierte er das BIP als standardisiertes Berechnungskonzept für die wirtschaftliche Gesamtleistung einer Ökonomie \parencite{Nobelpreis-Komitee1971}. Später erweiterte er seine empirische Arbeiten auf die Erforschung des wirtschaftlichen Wachstums \parencite{Kuznets1967}. Im Jahr 1971 erhielt er für seine Arbeiten zur empirischen Wirtschaftsforschung den Nobelpreis für Wirtschaftswissenschaften, der damals erst zum dritten Mal vergeben wurde. Heute ist er vielen im Zusammenhang mit der nach ihm benannten "`Kuznets-Kurve"' ein Begriff. \textcite[S. 26]{Kuznets1955} stellte - selbst sehr vorsichtig formulierend: "`The paper is perhaps 5 per cent empirical information and 95 per cent speculation"' - die Theorie auf, dass die personelle Einkommenskonzentration in der Frühphase der wirtschaftlichen Entwicklung von Nationen zunächst stark zunimmt. Später, wenn die Industrialisierung einer Ökonomie weit fortgeschritten ist, dreht sich dieser Prozess um und die Einkommenskonzentration nimmt wieder ab. Diese Arbeit stellt wohl die erste bedeutende empirische Arbeit zur personellen Einkommensverteilung dar. Heute gilt die Annahme allerdings als widerlegt (vgl. Kapitel \ref{Ungleichheit}).

Aus deutschsprachiger Sicht interessant ist die Einordnung Karl Polanyi's als Vertreter des "`alten Institutionalismus"'. Der aus Wien stammende ungarische Soziologe und Wirtschaftswissenschaftler ist in der Soziologie durchaus noch ein Begriff. Sein eigentlich umfangreicheres, ökonomisches Werk ist hingegen im Mainstream in Vergessenheit geraten. Die Hervorhebung der Wichtigkeit von Institutionen in \textcite{Polanyi1944}, erinnert vor allem an den Institutionalismus von John Commons \parencite{Maucourant1995}. Ökonomen diskutieren Polanyi's Rolle als "`alten Institutionalismus"' immer wieder \parencite[S. 183]{Hodgson1998}, \parencite{Cangiani2011}. Insgesamt ist aber sowohl das Werk der "`alten Institutionalisten"', als auch jenes Polanyi's zu heterogen und mögliche Verbindungen sind zu wenig erforscht \parencite{Frerichs2024}, sodass eine eindeutige Zuordnung Polanyi's zum "`alten Institutionalismus"' schlicht nicht möglich ist.


\section{Die Ursprünge des "`Neuen Institutionalismus"': Wozu gibt es Unternehmen?} \label{sec: Neue Inst}

Sowohl der alte als auch der neue Institutionalismus stellen die Bedeutung von Institutionen ins Zentrum ihrer Analysen. Worin unterscheiden sich die beiden Schulen dann aber so deutlich, dass die Vertreter des "`Neuen Institutionalismus"' heute fast keine Verweise auf ihre Namensvetter aus der Vergangenheit zulassen? Einer der fundamentalsten Gründe dafür liegt im unterschiedlich akzeptierten "`Menschenbild"'. Der "`Alte Institutionalismus"' berücksichtigt, dass sich Präferenzen von Individuen im Laufe ihres Lebens - und nicht zuletzt durch den Einfluss von Institutionen - ändern können. Dies ist nicht vereinbar mit dem Bild des rational handelnden "`homo oeconomicus"', der auf Grundlage seiner gegebenen Präferenzen seinen Nutzen maximiert. Die Theorien der Mainstream-Ökonomie folgen strikt einem individualistischen Ansatz, der Individuen und ihre Präferenz als gegeben ansieht. Dazu gehört auch die Annahme des methodologischen Individualismus, wonach nur Individuen handeln. Unternehmen, der Staat, oder andere Organisationen handeln nicht als Kollektiv, sondern nur in Form von ihnen zuzurechnenden Individuen. Im Gegensatz zu den "`Alten Institutionalisten"' akzeptieren die (meisten) Vertreter des "`Neuen Institutionalismus"' diesen Ansatz \parencite[S. 177]{Hodgson1998}. Veblen, Commons und Co lehnten die neoklassischen Theorien daher vollumfänglich ab. Die (meisten) Vertreter des "`Neuen Institutionalismus"' argumentieren hingegen, dass die Berücksichtigung von Institutionen die neoklassischen Theorien erweitern. Der "`Neue Institutionalismus"' steht der modernen Mainstream-Ökonomie damit wesentlich näher, als der "`Alte Institutionalismus"'. Zwar kam es noch zu keiner vollständigen Synthese zwischen Neoklassik und Neuem Institutionalismus, dennoch kann man behaupten, dass letzterer heute Teil der Mainstream-Ökonomie ist. Ein Indiz ist, dass unter den meistzitierten wirtschaftswissenschaftlichen Journal-Artikel überraschend viele dem Neuen Institutionalismus zuzuordnen sind. Früher wurde diesbezüglich \textcite{Coase1960}: "`The Problem of Social Cost"' als höchstgereit genannt \parencite{Coase1991}, in aktuelleren Listen \parencite{Kim2006, Mergio2016} finder sich neben methodischen Artikeln regelmäßig die Arbeit von \textcite{Jensen1976}: "`Theory of the Firm: Managerial Behavior, Agency Costs and Ownership Structure"' unter den ganz vorderen Rängen, aber auch \textcite{Alchian1972}: "`Production, Information Costs, and Economic Organization"'. Zudem gingen die Wirtschaftsnobelpreise der Jahre 1991 (Ronald Coase), 1993 (Douglass C. North und Robert W. Fogel), 2009 (Oliver E. Williamson, Elinor Ostrom) und 2024 (Daron Acemoglu, Simon Johnson, James A. Robinson) an Vertreter des Neuen Institutionalismus\footnote{Andere Wirtschaftsnobelpreisträger, wie die Regulierungstheoretiker George J. Stigler (1982), bzw. Jean Tirole (2013), der politische Ökonom James M. Buchanan (1986), die Vertragstheoretiker Oliver Hart und Bengt Holmström (2016) und die Auktionstheoretiker Paul R. Milgrom und Robert B. Wilson (2020) können indirekt wohl auch dem Neuen Institutionalismus zugeordnet werden.}. Interessant ist auch, dass diese ökonomische Schule sich erst viel später und nur teilweise dem sonst üblichen, strikten Formalismus zuwendete. Stattdessen spielt bis in die Gegenwart der Historismus eine bedeutende Rolle. Sp ist einer der Hauptvertreter des Neuen Institutionalismus, Douglass North, eigentlich Wirtschaftshistoriker. Aber auch in den aktuellen Arbeiten von Daron Acemoglu, Simon Johnson und James A. Robinson sind historische Beispiele und Ereignisse allgegenwärtig. Wie in vielen anderen Schulen, hatte die Entwicklung der Spieltheorie auch entscheidenden Einfluss auf den Neuen Institutionalismus. Vielleicht sogar stärker als in allen anderen Schulen. Die Einhaltung oder Nicht-Einhaltung von institutionellen Regeln ist wie prädestiniert für die Anwendung spieltheoretischer Konzepte. Interessant ist auch, dass der Neue Institutionalismus eine recht junge Schule ist. Erst seit den 1960er Jahren gewann sie an Bedeutung. Dann aber recht rasch und außerdem sehr vielfältig. Als einzelnen Ausgangspunkt kann man die Arbeit von Ronald \textcite{Coase1960} sehen. Direkt darauf aufbauend entwickelten sich zumindest drei unterschiedliche Forschungsrichtungen: Die "`Transaktionskostentheorie"', die "`Principal-Agent"'-Theorie (Theorie der Unternehmen im engeren Sinn) und der "`Property-Rights-Ansatz"', aus dem schließlich die "`Law and Economics"'-Schule entstand. Praktisch alle Vertreter dieser frühen Phase des Neuen Institutionalismus kann man als ausgeprägt wirtschaftsliberal bezeichnen. Das ist einigermaßen überraschend, handelt es sich bei Institutionen doch oft um staatliche Behörden, meistens aber zumindest um staatliche Eingriffe in den freien Markt. Aber es geht eben auch darum, wann man diese Institutionen überhaupt benötigt und wie man sie effizient gestalten kann. In diesem Sinne grenzt sich der frühe Neue Institutionalismus auch aktiv gegen absolut freie Märkte in jeder Hinsicht, also Anarchismus, ab. Es liegt auf der Hand, dass so große Felder wie die Politische Ökonomie, die Regulierungsökonomie, die Wettbewerbsökonomie aber auch die Umweltökonomie und die gesamte Theorie des Marktversagens nicht ohne institutionelle Überlegungen auskommen. Direkt aus dem Neuen Institutionalismus ableitbar sind aber auch die Theorie der (unvollständigen) Verträge und die Auktionstheorie. Erst etwas später entwickelte sich so etwas wie ein makroökonomischer Zweig des "`Neuen Institutionalismus"'. 
Dabei gab es schon früh Überlegungen, warum Institutionen für die gesamtwirtschaftliche Entwicklung eines Landes von entscheidender Bedeutung seien (z.B.: \textcite[S. 65ff]{North1990}). Die Grundidee dabei ist, dass langfristige Investitionen nur dann lohnend sind, wenn man sich als Investor darauf verlassen kann, dass abgeschlossene Verträge eingehalten werden und das gesellschaftliche Umfeld auf diese Einhaltung vertraut. Explizit mit der ökonomischen Wachstumstheorie hat dieses Thema aber erst Daren Acemoglu um die Jahrtausendwende. Seine Arbeiten stellen momentan den wohl aktuellsten Zweig des Neuen Institutionalismus dar. Alle nun angeschnittenen Themen werden in den Folgekapiteln näher betrachtet. Davor kommen wir aber zurück zu den Anfängen dieser Schule und der Frage: Was sind Institutionen überhaupt?

Seit dem Entstehen des "`Neue Institutionalismus"' wurde die Definition von "`Institutionen"' immer wieder diskutiert. \textcite[S. 4]{Riker1980} bezeichnete Institutionen allgemein als Spielregeln (bei der Entscheidungsfindung). \textcite{Schotter1981} argumentiert, dass Institutionen die Markt-Gleichgewichte sind, die sich aufgrund der Spielregeln ergeben. Im Mittelpunkt der Analysen stehen bei ihm die Ergebnisse, die aus der Anwendung der Regeln und Gesetze resultieren, nicht die Regeln und Gesetze selbst. \textcite{Ostrom1986} beschäftigt sich ausführlich mit der Frage was Regeln überhaupt sind, nämlich anerkannte Vorschriften, die bei wiederholten Transaktionen angewendet werden \parencite[S. 5]{Ostrom1986} und fügt die Komponente "`Durchsetzbarkeit"' hinzu \parencite[S. 6]{Ostrom1986}. Institutionen umfassen demnach gesellschaftliche Spielregeln und wie die Missachtung dieser sanktioniert wird \parencite[S. 26]{Voigt2009} kann man daraus ableiten.

Die zitierte Elinor Ostrom wird uns später im Kapitel \ref{Neue_Politik} noch unterkommen. Überhaupt ist die Abgrenzung zwischen der "`(Neuen) Politischen Ökonomie"' und der "`Neuen Institutionenökonomik"' schwierig bis verschwimmend. Lehrbücher zur Institutionenökonomik \parencite{Erlei2016, Voigt2009} umfassen auch Inhalte der "`Neuen Politischen Ökonomie"' - insbesondere die Konstitutionen-Ökonomie. Institutionen sind für die Umsetzung wirtschaftspolitischer Agenden unumgänglich. So können vor allem die frühen Vertreter der "`Public Choice"'-Literatur auch dem Institutionalismus zugeordnet werden. Umgekehrt gibt es aber durchaus Denkrichtungen in den beiden Schulen, die mit der jeweils anderen wenig am Hut haben. Daher werden die beiden Themenbereiche in diesem Werk in jeweils einem eigenen Kapitel dargestellt.

Als Geburtsstunde des "`Neuen Institutionalismus"' wird aus heutiger Sicht der Economica-Artikel "`The Nature of the Firm"' von Ronald \textcite{Coase1937} angesehen. Er konfrontierte die Neoklassiker mit einer offensichtlichen und nahezu immer auftretenden Inkonsistenz in deren Theorien: Die Neoklassik kennt Haushalte auf der Nachfrageseite und Unternehmen (Firmen) auf der Angebotsseite. Die Firmen verwandeln Inputs mittels verschiedener Technologien über die Produktionsfunktion in Outputs (vgl. Kapitel \ref{sec: Cobb-Douglas-Produktionsfunktion}). Wie das geschieht, welche Prozesse geschehen und warum es Unternehmen überhaupt gibt, bleibt dabei unberücksichtigt. Unternehmen werden in der Neoklassik wie eine "`Black-Box"' gesehen. Was darin vorgeht sei Sache der Betriebswirtschaftslehre, nicht der Volkswirtschaftslehre, könnte man argumentieren.

Allerdings ist gerade die Frage, wie die Existenz von Unternehmen überhaupt gerechtfertigt wird, auch in der VWL von entscheidender Bedeutung: In einer freien Marktwirtschaft übernimmt der Marktpreis die Koordinationsfunktion zwischen Angebot und Nachfrage. Inputfaktoren werden durch die Marktkräfte und die resultierenden Preis-Änderungen stets dahin verschoben, wo sie am effizientesten eingesetzt werden. Diese Koordinationsfunktion scheint aber innerhalb von Unternehmen außer Kraft gesetzt zu sein. Dort gibt es bekannterweise Hierarchien und Manager, die den Arbeitskräften ihre Tätigkeiten zuteilen: "`Wenn ein Arbeiter von Abteilung Y in Abteilung X wechselt, so macht er das nicht, aufgrund sich verändernder relativer Preise, sondern weil ihm jemand gesagt hat, dass er dies tun soll"', schreibt \textcite[S. 387]{Coase1937}. Die neoklassische Theorie beharrt also, auf der einen Seite, dass freien Märkte effizient sind, beachtet dabei auf der anderen Seite aber nicht, dass auf unterster Produktionsebene - also in den Unternehmen - ganz unumstritten nicht Marktkräfte die Koordinierungsaufgaben übernehmen, sondern geschaffene Hierarchien. Weiter gedacht, bräuchte es laut neoklassischer Theorie gar keine Unternehmen. Stattdessen müssten Arbeitskräfte und ihr Arbeitskräfte-Angebot (Input-Faktor Arbeit) alleine durch die Marktkräfte und die resultierenden relativen Preise zu den Maschinen (Input-Faktor Kapital) finden, mit denen sie den höchsten Output schaffen können. Wir wissen, dass letzteres Utopie ist und in Unternehmen Markt-Transaktionen ersetzt werden durch Manager, die die Prozesse koordinieren \parencite[S. 388]{Coase1937}. Im Gegensatz zu den "`alten Institutionalisten"' schließt Coase daraus aber nicht, dass die neoklassischen Theorien falsch sind. Stattdessen erweitert er die Theorien um einen neuen Faktor: Transaktionskosten. Wobei er diese noch nicht als Transaktionskosten, sondern als Marketingkosten bezeichnete (vgl. z.B.: \textcite[S. 392]{Coase1937}). Wie aber kann die Einführung dieser gerechtfertigt werden? \textcite[S. 390]{Coase1937} argumentiert, dass die Nutzung des Marktes selbst Kosten verursacht. Er begründete damit die "`Transaktionskostentheorie"', die bis heute ein aktives Forschungsfeld ist. Da Markttransaktionen Kosten verursachen, kann es für Individuen sinnvoll sein, sich zu Organisationen zusammenzuschließen um diese Kosten gemeinsam zu stemmen. Diese Organisationen sind eben Unternehmen. Innerhalb der Unternehmen kommt es zu "`vertikalen"' \parencite[S. 388]{Coase1937} Hierarchien, über die sich Individuen innerhalb der Organisation koordinieren. Auch das ist nicht kostenlos, sondern verursacht die sogenannten Organisations-Kosten.

Damit ist aber noch nicht erklärt warum es \textit{verschiedene} Unternehmen gibt. Wenn der Zugang zum Markt mit Kosten verbunden ist, warum übernimmt nicht ein einzelnes Unternehmen all diese Kosten und bietet die gesamte Produktion an \parencite[S. 394]{Coase1937}? Mit anderen Worten: Wenn jedes Unternehmen Transaktionskosten für den Zugang zum Markt bezahlt, sollten größere Unternehmen einen Vorteil gegenüber kleinen Unternehmen haben, weil die relativen Kosten für den Marktzugang für diese geringer sind. Eine Tendenz zu Monopol-Bildung wäre die Folge. Historisch gesehen ist dies aber auch im Hinblick auf die damals noch herrschende Systemfrage relevant: Lenin meinte einst in Bezug auf die das real-sozialistische Wirtschaftssystem, es "`werde laufen, wie eine große Fabrik."' \parencite[S. 113]{Warsh}. Wäre nicht das sozialistische System der freien Marktwirtschaft überlegen, wenn man Transaktionskosten in die Analyse miteinbezieht? Könnte nicht ein großes, staatliches Unternehmen, viel effizienter Transaktionskosten reduzieren, als viele kleine Unternehmen? \textcite[S. 395]{Coase1937} gibt auch darauf die Antwort: Transaktionskosten sind nicht einfach fixe Kosten, oder Kosten, die linear mit der Unternehmensgröße steigen. Stattdessen gibt es so etwas wie "`abnehmende Management-Erträge"'. Ab einer gewissen Unternehmensgröße würde das Management eines Unternehmens schwieriger und teurer. Analog zur neoklassischen Produktionstheorie, wo die optimale Output-Menge erreicht ist, wenn die Grenzkosten mit dem Grenzertrag identisch sind, wäre die optimale Unternehmens-Größe demnach erreicht, wenn die internen Organisations-Kosten identisch sind mit den Transaktionskosten. Also die Kosten, die man für die Markt-Teilnahme bezahlen muss. \textcite{Coase1937} lieferte damit einen ersten Beitrag zur Theorie, warum es Unternehmen gibt, der mit der neoklassischen Theorie vereinbar ist. Sein Artikel blieb allerdings lange Zeit weitgehend unbeachtet \parencite{Coase1991a} und seine Idee wurde erst nach dem Zweiten Weltkrieg weiter entwickelt. Aufmerksamen Leser ist bestimmt aufgefallen, dass zwischen der Veröffentlichung von \textcite{Coase1937}, die als Geburtsstunde der "`Neuen Institutionenökonomik"' gilt, und den späten Werken der "`Alten Institutionalisten"', also zum Beispiel \textcite{Commons1934}, nur wenige Jahre liegen. Daraus eine zeitliche Nähe zwischen alter und neuer Institutionenökonomik zu vermuten, wäre aber falsch. Interessanterweise vergingen zwischen der Veröffentlichung von \textcite{Coase1937} mehr als zwanzig Jahre, bis weitere Arbeiten zur "`Neuen Institutionsökonomik"' folgten \parencite[S. 148]{Blaug2001}.


\section{Private Eigentumsrechte statt Staatseingriffe: Das Coase-Theorem}
\label{Coase-Theorem}

Der zweite bahnbrechende Artikel von Ronald Coase erschien im Jahr 1960. Im Gegensatz zu seinem Beitrag aus dem Jahre 1937, der im letzten Kapitel behandelt wurde, war \textcite{Coase1960}: "`The Problem of Social Cost"' unmittelbarer Erfolg vergönnt. Tatsächlich erhielt Ronald Coase im Jahr 1991 den Nobelpreis für Wirtschaftswissenschaften im Wesentlichen alleine für diese beiden Artikel. Das Nobelpreis-Komitee schreibt sogar wörtlich: "`Ronald Coases Schriften waren spärlich, doch sein Einfluss auf die Wirtschaftswissenschaften war tiefgreifend"' \parencite{Coase1991a}. Wie bereits kurz angedeutet, greift \textcite[S. 1]{Coase1960} die damals unter Ökonomen gängige Meinung an, dass nur staatliche Eingriffe im Sinne von \textcite{Pigou1920} die negativen Folgen von externen Effekten beseitigen können. Coase zeigt, dass es auch ohne staatlichen Eingriffen zu einer effizienten Zuweisung der Produktionsfaktoren kommt (effiziente Allokation). Selbst wenn negative externe Effekte auftreten, führen Marktlösungen zu pareto-optimalen Gesamtergebnissen. Allerdings passiert dies nur wenn Nebenbedingungen erfüllt sind: Wie in der Neoklassik üblich muss ein vollkommener Wettbewerbsmarkt vorliegen. Gewichtiger sind aber jene zwei Nebenbedingungen, die auch zentral für die frühe Phase des "`Neuen Institutionalismus"' sind. Erstens, damit der Markt ein effizientes Gleichgewicht durch Verhandlungen findet, müssen die Transaktionskosten gleich Null sein. Zweitens, muss es klar definierte und durchsetzbare Eigentumsrechte geben. Die Ergebnisse dieser Arbeit wurden als "`Coase-Theorem"' recht rasch zu einem gängigen, wenn auch umstrittenen Konzept der Wirtschaftswissenschaften. Der Name "`Coase-Theorem"' wurde dabei übrigens nicht von Coase selbst, sondern von George Stigler \parencite{Coase1991} in den 1960er-Jahren eingeführt. \textcite{Coase1960} plädiert in seiner Arbeit also dafür, anstatt direkter staatlicher Eingriffe und Regulierungen, die Eigentumsrechte klar zu definieren und davon abgesehen auf die Kräfte des freien Marktes zu vertrauen. 

\textcite{Coase1960} geht dabei das Problem der Internalisierung externer Effekte ganz anders an als \textcite{Pigou1920}. Bei letztgenanntem gibt es stets einen Schädiger, der dem Geschädigten Kosten verursacht, ohne einen adäquaten Preis dafür zu bezahlen. Diese Kosten nennt man negative externe Effekte. \textcite{Pigou1920} findet die Lösung für dieses Problem darin, dem Schädiger die Kosten direkt als Ausgleichszahlung in Rechnung zu stellen, oder eine Steuer in Höhe des Schadens einzuführen. Der Geschädigte bekommt seinen Verlust also entweder durch eine Zahlung ausgeglichen, oder - wenn die Allgemeinheit benachteiligt wird  - es wird eine Pigou-Steuer verhängt. \textcite{Coase1960} gibt das Konzept des Schädigers und des Geschädigten auf. Für ihn ist das Problem externer Effekte reziprok. Das heißt, es wird nicht danach getrachtet dem Schädiger die Kosten aufzubürden, sondern die Parteien werden als gleichberechtigt dargestellt. Dies ist grundsätzlich kontra-intuitiv. Es ist ja durchaus sinnvoll einem Schädiger sämtlichen verursachten Schaden auch zu verrechnen. Coase weist dies aber zurück. Anstatt schädigendes Verhalten vorab zu verbieten oder einseitig zu sanktionieren, sollte zunächst das maximale potentielle Gesamtergebnis berechnet werden, das erzielt wird, wenn die Schädigung nicht vorab verboten wird \parencite[S. 293]{Erlei2016}. 

Wie argumentiert \textcite[S. 2ff]{Coase1960} nun? Eines seiner Beispiele lautet, stark vereinfacht, wie folgt: Ein Getreidebauer verdient durch den Verkauf seiner Ernte sein Geld. Die Herde eines benachbarten Rinder-Züchters zerstört allerdings regelmäßig einen Teil dieser Ernte. Wobei die Größe seiner Herde schwankt und eine größere Herde mehr Schaden anrichtet. Angenommen der Rinder-Züchter muss den Schaden, den seine Kühe verursachen, bezahlen. Der Getreidebauer erhält also stets Zahlungen, die zumindest dem entstandenen Schaden entsprechen. Der Viehzüchter wird in diesem Fall, unabhängig vom Schaden, den seine Herde verursacht, seinen Gesamtgewinn maximieren. Solange der Mehrerlös aus einer größeren Herde größer ist als die fälligen Schadenszahlungen, wird der Rinderzüchter seine Herde ausweiten. Was passiert aber, wenn der Vieh-Züchter nicht zur Entschädigung verpflichtet ist? Angenommen er muss also keinerlei Entschädigungszahlungen leisten. Der Vieh-Züchter könnte nun seine Herde ausweiten, ohne auf die dadurch verursachten negativen externen Effekte, nämlich die Zerstörung des Getreide des Nachbarn, Rücksicht zu nehmen. Der Getreidebauer kann dies aber verhindern, indem er dem Vieh-Züchter Zahlungen anbietet, wenn dieser seine Herde auf eine gewisse Größe einschränkt. Hier gilt analog: Der Getreidebauer wird seine Zahlungen maximal solange ausweiten, bei sein Gewinn aus dem Getreideanbau Null wird. Der Viehzüchter hingegen wird sich auf die Ausgleichszahlungen so lange einlassen (und die Herde entsprechend klein belassen), bis die Zahlungen geringer sind als der zusätzliche Gewinn durch eine größere Herde. Natürlich wäre es dem Getreidebauern in jedem Fall lieber, wenn sein Nachbar keine Rinder züchten würde und dem Vieh-Bauern wäre es umgekehrt lieber, wenn das Nachbar-Grundstück unbewohntes Brachland wäre. Gegeben den beschriebenen Nachbarschaftsverhältnissen, ist dies aber eben nicht realistisch. \textcite{Coase1960} zeigt - und das ist der entscheidende Punkt -, dass es unabhängig von den Eigentumsverhältnissen zu einer optimalen Allokation der Ressourcen kommt. Egal ob der Viehbauer den Getreidebauern entschädigen muss oder nicht, die resultierende Gesamtproduktion wird in beiden Fällen die gleiche sein. Die reine Marktlösung führt demnach immer zum optimalen Gleichgewicht. Das gilt natürlich nur für das Gesamtergebnis. Für die beiden Parteien ist es aus individueller Sicht entscheidend, welche Eigentumsverhältnisse herrschen. Also im konkreten Fall ob der Viehbauer eine Entschädigung bezahlen muss, oder nicht. Unterlegen wir das Beispiel mit Zahlen: Angenommen der maximale Gesamtertrag der beiden Landwirte kann 150€ betragen. Die Herde umfasse dann vier Kühe, die 100€ Gewinn einbringen. Der Getreidebauer könne aus dem Verkauf der Ernte 50€ erlösen. Die Herde verursacht 20€ Schaden, ohne Herde hätte er also 70€ Erlös. Wenn die Landwirte sich bei Verhandlungen auf 30€ Schadenersatz seitens des Viehzüchters einigen, erzielen beide einen positiven Ertrag aus ihrer Landwirtschaft (Viehzüchter: 100€-30€ = 70€, Getreidebauer: 50€+30€ = 80€). Wenn der Viehzüchter nicht schadenersatzpflichtig ist, könnte er seine Herde auf fünf Kühe ausweiten, die dann (angenommen) 120€ Gewinn bringt. Der Schaden würde dann auf 45€ steigen. Der Getreidebauer könnten nun eine Zahlung von 22€ anbieten, wenn der Rinderzüchter die Herde bei vier Kühen belässt. Der Rinderbauer würde das Angebot annehmen: Er erhält nun 100€ aus der Viehzucht und 22€ als Ausgleichszahlung (122€ statt 120€ bei fünf Kühen). Der Getreidebauer erlöst 50€ muss aber nun 22€ bezahlen und hat einen Gewinn von 28€. Seine Alternative wäre keine Zahlungen anzubieten. Dann hätte er aber einen noch höheren Schaden als bei vier Kühen nämlich 45€ statt 20€. Der Erlös würde sich nochmal um 25€ schmälern und er hätte nur 25€ Gewinn. Auch diese Ausgleichszahlung ist also für beide Seiten optimal. Fazit: Die Allokation von Ressourcen durch freiwillige Verhandlungen zwischen den beiden beteiligten Parteien führt zu einer effizienten, gesamtwirtschaftlich optimalen, Lösung.  Im Beispiel führt diese optimale Allokation über Marktergebnisse, also ohne Eingriffe, immer zu insgesamt 150€ Gewinn. Für den Gewinn jener Partei, die die Ausgleichszahlungen erhält ist es unerheblich, ob sie ihren Gewinn aus Verkaufserträgen generiert, oder aus Ausgleichszahlungen. Die Partei, die Ausgleichszahlungen erhält, ist jene Partei in deren Eigentumsrechte (eigentlich Verfügungsrechte) eingegriffen wird. Die andere muss Ausgleichszahlungen leisten und ihr Gewinn kann sich im Vergleich zur Ausgangssituation verschlechtern.

Wie bereits erwähnt ist das Coase-Theorem selbst in der neoklassischen Modellwelt nur dann gültig, wenn es keine Transaktionskosten gibt und die Eigentumsrechte eindeutig geklärt sind. Coase wusste natürlich, dass Transaktionskosten in Höhe Null unrealistisch sind. Wie er selbst in seiner Nobelpreis-Rede 1991 behauptete, wollte er mit seinem Artikel nicht zeigen, dass die Ansätze von Pigou falsch sind, sondern, dass die Berücksichtigung von Transaktionskosten unumgänglich ist: "`My conclusion: let us study the world of positive transaction costs"' \parencite{Coase1991}. Sein Artikel selbst liest sich allerdings deutlich giftiger \parencite[S. 243]{Cansier1989} in Richtung Pigou: Gleich zu Beginn nennt er den Ansatz von Pigou schlicht falsch \parencite[S. 2]{Coase1960} um sich gegen Ende des Artikels darüber verwundert zu zeigen, wie ein so falscher Ansatz, wie jener von Pigou, so einflussreich werden konnte \parencite[S. 39]{Coase1960}. Daneben begründete \textcite{Coase1960} aber auch die Bedeutung von Eigentumsrechten ("`Property Rights"').  Allgemeiner formuliert könnte man sogar argumentieren, dass \textcite{Coase1960} die enorme Wichtigkeit der Institution Rechtssystem als erster aufzeigte. Je klarer die Rechte und Pflichten von Parteien geregelt sind, desto eher wird ein stabiles Gleichgewicht entstehen \parencite[S. 19]{Coase1960}. Wenn im Beispiel oben Getreidebauer und Viehzüchter von Anfang an vollends über die Entschädigungspflichten (also die Eigentumsrechte) informiert sind, werden sie selbst zur aufgezeigten Verhandlungslösung finden können. Sind die Eigentumsrechte hingegen unklar, werden Gerichte für Klarheit sorgen müssen und in diesem Zusammenhang werden Gerichtskosten, also Transaktionskosten, anfallen. 

Der Werdegang der Theorien von Coase ist aus vielen Gesichtspunkte interessant. Der Artikel \textcite{Coase1960} ist mit 44 Seiten recht umfangreich, aber sehr klar geschrieben. Die Argumente sind allesamt rein verbal formuliert, Mathematik hatte Coase schon während seiner Studienzeit vermieden \parencite{Coase1991a}. Obwohl dies vollkommen gegen den Zeitgeist war - Die quantifizierte Ökonomie erlebte gerade in den 1950er und 1960er Jahren enormen Zulauf - und obwohl sich die Theorien von Coase grundsätzlich sehr gut formal darstellen lassen, hatte Coase mit seinem rein verbalen Ansatz Erfolg. Auch empirische Arbeiten findet man nicht im Repertoire von Coase. Ein Kollege von ihm meinte einmal sogar: "`Er hat sein Leben lang keine einzige Zahl verwendet"' \parencite[S. 111]{Warsh}. Seine akademische Karriere begann zunächst nicht vielversprechend. Ein etwas kurios anmutendes Event, änderte dies schlagartig. Coase veröffentlichte 1959 einen Journal-Artikel \parencite{Coase1959}, in welchem er Vorschlug, dass Radio-Frequenzen für Sender zukünftig an den best-bietenden versteigert werden sollten. Die damalige Rechtslage sah vor, dass eine staatliche Stelle entsprechende Anträge von Radio-Sendern per Kommissionsbeschluss prüft und Übertragungsrechte (Radio-Frequenzen) vergibt. Diese staatliche Stelle hat also Einfluss darauf wer einen Radio-Sender betreiben darf und somit indirekt auch darauf, was im Radio zu hören ist \parencite[S. 879]{Coase1959}. Radio-Frequenzen sind ein Allmende Gut wie in Kapitel \ref{Offentliche Guter} beschrieben: Es gibt nur eine bestimmte Anzahl geeigneter Frequenzen (Rivalität), aber man kann niemanden davon ausschließen ein Signal in einer bestimmten Frequenz zu senden. Überlagerte Signale mit der gleichen Frequenz führen aber dazu, dass bei den Empfängern keine sinnvollen Informationen mehr ankommen. Für Ökonomen war es zu der Zeit unumstritten, dass in so einem Fall der Staat eingreifen muss und zum Wohle der Allgemeinheit mittels Regulierungen festlegt, wer auf welcher Frequenz senden darf. Die Theorie der Öffentlichen Güter, wie in Kapitel \ref{Offentliche Guter} beschrieben, war auch unter liberalen Ökonomen absolut "`State of the Art"'. Als \textcite{Coase1959} nun die Versteigerung vorschlug, war dies vor allem für die liberalen Ökonomen, deren geistiges Zentrum schon damals an der University von Chicago lag, reizvoll. Anstatt bürokratischer, staatlicher Kommissionen, würde der Markt in Form von Versteigerungen in Zukunft die Radio-Frequenzen vergeben. Die inhaltliche Argumentation von \textcite{Coase1959} vermochte aber nicht zu überzeugen. In Chicago wurde die Idee eben als reizvoll, aber vor allem auch als wahrscheinlich einfach falsch angesehen \parencite{Coase1991a}. Coase wurde in der Folge von den Granden der liberalen Ökonomen in Chicago - darunter Aaron Director, George Stigler und Milton Friedman - nach Chicago eingeladen um seine Idee vorzustellen. Es bildete sich eine wahre Legende \parencite[S. 113]{Warsh} darüber, wie Coase bei einem Dinner die etablierten Ökonomen von seinen Ansätzen überzeugte. \textcite[S. 45]{Schlafly2007} etwa schreibt vom "`berühmtesten Dinner in der Geschichte der Ökonomie"'. "`Zu Beginn des Abends waren alle 20 Ökonomen gegen die Coase, aber am Ende hatte er alle "`konvertiert"' [...] und die Teilnehmer waren überzeugt davon, dass an diesem Abend intellektuelle Geschichte geschrieben wurde"', schreibt \parencite[S. 115]{Warsh}. Erst in Folge der Gespräche dieses Abends, fasste Coase seine Ideen neu zusammen, was in der Publikation \textcite{Coase1960} mündete. Der anfängliche Widerstand gegen seine Arbeit könnte erklären, warum "`The Problem of Social Cost"' in einfachster Sprache verfasst und mit eine ganzen Reihe von Beispiele ausgeschmückt ist. Diesen Erzählungen folgend, könnte man meinen, dem beschriebenen Dinner folgte eine unmittelbare ökonomische Revolution. Tatsächlich meinte das Nobelpreis-Komitee anlässlich der Verleihung des Preises an Ronald Coase im Jahr 1991, dass dieser etwas grundlegend neues entdeckt habe, ähnlich der Entdeckung neuer Elementarteilchen in der Physik. Eine schlagartige Revolution nach dem genannten Dinner fand in diesem Sinne zwar nicht statt, der Einfluss von Coase' Theorie war aber enorm. Welche neuen Wege Coase' Ideen ebneten, wird in den folgenden Kapiteln nochmals aufgegriffen.

Der Artikel \textcite{Coase1960} galt lange Zeit als der meistzitierte wirtschaftswissenschaftliche Journal-Beitrag, wie \textcite{Coase1991a} in seiner Nobelpreisrede selbst anmerkte. Auch heute befindet sich der Artikel noch in den vorderen Rängen in entsprechenden Zählungen \parencite[S. 400]{Mergio2016}. Nichtsdestotrotz gilt das Coase-Theorem seit jeher als umstritten. Eine Tatsache, die der Anzahl der Zitierungen nicht abträglich ist. Aus theoretischer Sicht fehlt bei \textcite{Coase1960} eine Analyse von möglichem strategischem Verhalten: So könnten alle Nachbarn des oben genannten Getreidebauern, also auch jene ohne jede landwirtschaftliche Ambition, androhen eine Viehherde anzuschaffen, nur um sich durch Kompensationszahlungen schließlich "`davon abhalten"' zu lassen. Wichtiger ist aber der Einwand, dass es eben ein sehr theoretisches Modell ist. Wie Coase selbst anführt, existieren keine Märkte auf denen Transaktionskosten keine Rolle spielen. Die Voraussetzungen für die Anwendung des Coase-Theorems sind damit nie zu 100\% gegeben. Das Coase-Theorem ist somit seit seiner Entwicklung Teil eines ökonomischen Richtungsstreits, der nicht allgemein beigelegt werden kann: Wie soll man negativen externen Effekten beikommen? Entweder durch staatliche Regulierungen und Entschädigungen über eine Pigou-Steuer, wie eher links-orientierte Ökonomen und Politiker fordern, oder doch durch Marktlösungen im Sinne des Coase-Theorems, bei dem die Regelung der Eigentumsrechte im Vordergrund steht, wie Markt-liberale plädieren? Coase war, wenig überraschend, einer der Hauptvertreter zweiterer Gruppe. Auf die Frage ob er ein Beispiel für eine schlechte Regulierung nennen könne, antwortete er: "`I can't remember one that's good."' \parencite[S. 45]{Schlafly2007}. Auch wenn \textcite{Coase1960} die Ansätze von \textcite{Pigou1920} vehement Angriff, lösten Coase' Ideen jene von Pigou nicht ab. Beide Konzepte bestehen heute nebeneinander.

Insgesamt war Coase Teil jener konservativen Bewegung, die ab 1970 die Wirtschaftswissenschaften wieder Richtung Markt-Liberalismus steuerte. Vor allem  Bereich der "`Public Economy"' wurden durch seine Arbeiten und die Arbeiten der Neuen Politischen Ökonomie (vgl. Kapitel \ref{Neue_Politik}) grundlegend verändert. Marktversagen rechtfertigte fortan nicht mehr automatisch Staatseingriffe. Dem Marktversagen wurde das Staatsversagen gegenübergestellt. Die Feststellung, dass der Markt bestimmte Güter nicht befriedigend zur Verfügung stellt, war fortan kein ausreichender Grund mehr dafür, staatliche Eingriffe vorzunehmen. Stattdessen müsse man feststellen, ob der Staat diese Güter nicht noch schlechter zur Verfügung stelle. Man verbindet häufig die Namen Milton Friedman, August Friedrich Hayek und mit Abschlägen Robert Lucas mit der wirtschaftsliberalen Wende um 1970. Ebenso einflussreich waren aber die damals entstandenen Ideen für Marktlösungen, wo in den Jahren zuvor selbst konservative Ökonomen und Politiker Staatseingriffe für unumgänglich hielten: Im Bereich der externen Effekte wurde der Pigou-Steuer eben das Coase-Theorem gegenübergestellt. Bei den Allmende-Gütern forderte \textcite{Buchanan1965} die Einführung von "`Klubs"' anstelle staatlicher Eingriffe (vgl. Kapitel \ref{Neue_Politik}). Und im Bereich der natürlichen Monopole ersetzten die "`angreifbaren Märkte"' von \textcite{Baumol1982} die staatlichen Unternehmen (vgl. Kapitel \ref{Disease}). 

Dass die Arbeiten von \textcite{Coase1937, Coase1960} enorm einflussreich waren, ist aus den vergangenen zwei Kapiteln wohl hervorgegangen. Konkret bildeten die beiden Journal-Beiträge die Grundlage für gleich drei verschiedene Forschungszweige, die in den Folgejahren intensiv bearbeitet wurden:
\begin{itemize}
	\item Die Transaktionskostentheorie
	\item Die Theorie der Unternehmen im engeren Sinn und das Prinzipal-Agent-Problem
	\item Der Property-Rights-Ansatz und die "`Law and Economics"'-Schule
\end{itemize}

\subsection{Transaktionskostentheorie}

Das Thema "`Transaktionskosten"' spielt in der Diskussion um das Coase-Theorem \textcite{Coase1960} eine entscheidende Rolle. Um 1960 war die Annahme, dass die Probleme, dass die durch das Auftreten externer Effekte entstehen, ausschließlich durch staatliche Eingriffe geheilt werden können, wie von \textcite{Pigou1920} beschrieben, vollkommen unumstritten. In \textcite{Coase1960} beschreibt dieser zunächst, dass dies aus Sicht der Neoklassik nicht stimmt. Wenn nämlich, wie in der Neoklassik üblich, die Transaktionskosten ignoriert werden, dann führen Verhandlungen zwischen Verursachendem und Geschädigtem zu effizienten Lösungen. Nun wurden Transaktionskosten zwar in der Neoklassik beständig ignoriert, gleichzeitig war sich aber jeder Ökonom bewusst, dass es so etwas wie Transaktionskosten in der Realität bei jeder wirtschaftlichen Handlung gibt. In weiterer Folge herrschte in der Ökonomie Einigkeit über die hohe Bedeutung von Transaktionskosten und in weiterer Folge über die Wichtigkeit entsprechender Forschung dazu. Allerdings entwickelten sich ganz unterschiedliche Ansätze dazu:



Zunächst folgte der Artikel von \textcite{Arrow1969}. Er hob die Bedeutung der "`Vertikalen Integration"' hervor. Also die Tatsache, dass firmeninterne Organisationskosten in Kauf genommen werden, um die Kosten, die für die Marktteilnahme (Transaktionskosten) anfallen, zu umgehen. Er führte auch den Begriff der "`Transaktionskosten"' ein \parencite[S. 39]{Erlei2016}


Später wurde der "`Washington Ansatz"' \parencite[S. 33]{North1990} von Steven Cheung (1974, 1983) entwickelt und später von Douglass North weiterentwickelt. Letzterer versuchte auch die Transaktionskosten zu quantifizieren (Wallis und North 1986).

Der bedeutendste Ansatz war aber wohl jener von Oliver Williamson. Durchbruch der Transaktionkostentheorie: aber anderer Ansatz Die Arbeiten von Oliver E. Williamson: Market and Hierarchies (Voigt S. 87)

Ähnliche Ansätze in der Freiburger Schule (Kapital \ref{Neoliberalismus})

HIER WEITER






\subsection{Theorie der Unternehmen und Moral Hazard}
\label{sec: Theorie_Unternehmen}

In diesem Unterkapitel kommen wir zurück zur Frage: Warum gibt es überhaupt Unternehmen? Wie oben in Kapitel \ref{sec: Neue Inst} beschrieben, kann man dies durch das Auftreten von Transaktionskosten erklären. Dieser Ansatz wurde im letzten Unterkapitel weiter ausgeführt. Auch dieses Kapitel könnte man unter dem Titel Transaktionskostentheorie führen. Allerdings ist der Fokus in dieser Forschungsrichtung ein etwas anderer. Parallel zur gerade dargestellten Transaktionskostentheorie entwickelte sich in den 1970er Jahren ein volkswirtschaftlicher Forschungszweig, der die Vorgänge \textit{innerhalb} von Unternehmen genauer unter die Lupe nahm. Etwas salopp formuliert, wurde hier der Frage nachgegangen, welche Probleme aus streng neoklassisch-nutzenmaximierender Sicht innerhalb Firmen auftreten sollten. Darunter fallen insbesondere die Probleme der Messkosten bei Teamwork, Probleme zwischen Eigentümer und Angestellten und Probleme der Risikoverteilung und der Anreize \parencite[S. 67]{Erlei2016}. Die Frage warum es Unternehmen gibt, wurde also quasi von der anderen Seite her angegangen: Warum sollten Unternehmen eigentlich scheitern und was machen Firmen-Eigentümer dagegen? 

Einen ersten bahnbrechenden Artikel dazu verfassten \textcite{Alchian1972}. Die beiden Ökonomen werden uns auch im nächsten Kapitel noch prominent begegnen. Ausgangspunkt ist die Feststellung, dass Unternehmen an sich keinerlei Autoritätsmacht oder andere Verfügungsgewalt darüber haben, dass ihre Arbeitnehmer im Sinne des Unternehmers handeln. Der Grund warum Arbeitnehmer dennoch meist recht langfristig für einen Arbeitgeber tätig sind, liegt stattdessen an einem von \textcite[S. 778]{Alchian1972} postulierten "`Team-Prozess"'. Das Team ist mittels (Arbeits-)verträgen miteinander verbunden und nutzt gemeinsam die anderen Inputfaktoren (also nicht Arbeit) zum Erstellen eines Outputs. Wenn die anderen Produktionsfaktoren nicht beliebig teilbar sind, was üblicherweise der Fall ist, dann macht die Zusammenarbeit von Individuen in Teams Sinn. Schließlich ergibt erst eine gewissen Team-Größe in Kombination mit den anderen, nicht teilbaren Inputfaktoren, den optimalen Output. Man stelle sich in diesem Zusammenhang folgendes vor: Die Stahlproduktion erfordert auf der einen Seite den Hochofen und auf der anderen Seite mehrere \textit{zusammenarbeitende} Individuen, also ein Team. Nur dann kann mit dem Hochofen sinnvoll gearbeitet werden. In diesem Zusammenhang taucht aber das zentrale Problem auf: Der Beitrag eines einzelnen Teammitglieds zum entstehende Output ist oft nicht so ohne weiteres zu ermitteln. Man kann nicht messen welcher Arbeitsschritt schlussendlich welchen Gewinn verursacht hat. Entscheidend war ja die gemeinsame Erbringung der Leistung. Es ergibt sich also ein Messproblem - von \textcite[S. 778]{Alchian1972} "`Metering Problem"' genannt - das darin besteht, dass den einzelnen Arbeitnehmer ihre Leistung nicht vollständig zugerechnet werden kann. Folglich kann auch der Lohn nicht zu vollständig leistungsgerecht bezahlt werden. Es ergibt sich also die Situation, dass Zusammenarbeit im Team für jedes Individuum vorteilhaft ist, weil der erzielte Gesamtoutput so höher ist als die Summe der Outputs, die jedes Individuum alleine erreicht hätte. Dies ist eine mögliche Erklärung für die Entstehung von Unternehmen. Demgegenüber steht aber die Tatsachen, dass der Gesamtoutput nicht so ohne weiteres den einzelnen Individuen zugerechnet werden kann. Wird nun gar nicht versucht zu messen wer welchen Beitrag zum Output geleistet hat entsteht für die Individuen ein Anreiz "`sich vor der Arbeit zu drücken"' (Shirking)\footnote{Wir kennen den Begriff bereits aus Kapitel \ref{cha: Neu Keynes} und den dort beschriebenen Unvollkommenheiten am Arbeitsmarkt. Die dort damit in Verbindung gebrachte Arbeit \textcite{ShapiroStiglitz1984} wurde zeitlich allerdings wesentlich später verfasst, als die hier beschriebene Arbeit}. Gelingt es einem Arbeiter zum Beispiel sich vollkommen vor der Arbeit zu drücken, wird seine Freizeit auf 100\% steigen, der Gesamtoutput, und damit die Bezahlung des Arbeiters, wird hingegen nur um dessen Beitrag fallen. Individuell nutzenmaximierend wäre in diesem Fall eine Reduktion der individuellen Arbeit auf Kosten des Teams. Da diese individuelle Nutzenmaximierung für jedes Individuum gilt, hätte niemand Anreiz zu arbeiten, es würde aber auch kein Output entstehen. Es entsteht also ein klassisches Gefangenendilemma, das wir aus der Spieltheorie (Kapitel \ref{Spieltheorie}) kennen. Obwohl diese in den 1970er Jahren schon entwickelt war, wendeten \textcite{Alchian1972} diese in ihrem Journal-Beitrag interessanterweise nicht explizit an. Realistischerweise wird es aber nicht möglich sein, sich vollständig vor der Arbeit zu drücken, bzw. werden die Arbeitgeber eben versuchen die Leistung der Individuen doch zu messen. Wäre dies ohne Aufwand möglich, so könnte das Shirking-Problem vollständig eliminiert werden, üblicherweise entsteht allerdings ein Aufwand, zum Beispiel in Form von Personalkosten. So könnte zum Beispiel ein Individuum ausschließlich mit der Aufgabe betreut werden, die anderen zu überwachen und Shirking zu sanktionieren. Heute verwendet man auch im Deutschen dafür das harmloser klingende Wort "`Monitoring"'. Ist dieser Überwacher gleichzeitig der Eigentümer der anderen Inputfaktoren (außer Arbeit), so haben wir hier eine weitere Rechtfertigung für die Entstehung von Unternehmen. Der Firmenchef sorgt durch sein Team-Monitoring dafür, dass die einzelnen Arbeiter tatsächlich produktiv sind. Da der Gesamtoutput durch die Teamarbeit höher ist (unter der Voraussetzung, das Monitoring gelingt) als es die Summe der Outputs bei Einzelarbeit wäre, bleibt die Differenz dem Firmenchef als Unternehmensgewinn. Dadurch hat der Firmenchef auch Anreiz zum Monitoring, schließlich führt "`Shirking"' direkt zur Reduktion seines Anteils am Output. Alternativ kann der Überwacher aber auch ein Angestellter sein. Dann müssen die Erträge seines Monitorings - die entstehen durch die resultierende Reduktion des Shirkings - zumindest so hoch sein wie die Kosten des Monitorings - das ist der Lohn des Überwachers. Dann bleibt die Frage wer den Überwacher überwacht. So lassen sich Hierarchien in Unternehmen erklären. Der Überwacher eines Überwachers wäre demnach die zweite Führungsebene. Auch der Job des Managers - also der Führungskraft, die keine Eigentümerstellung hat - lässt sich so erklären. Jede Führungsebene muss dabei den individuellen Anreiz haben Shirking seiner Untergebenen zu unterbinden. Ein Unternehmen ist also bei \textcite{Alchian1972} ein Netzwerk aus Verträgen, das schlussendlich beim Eigentümer zusammenläuft.

Wie bereits erwähnt, war der Journal-Artikel \textcite{Alchian1972}: "`Production, Information Costs, and Economic Organization"' sehr einflussreich und einer der meistzitierten Artikel \parencite[S. 192]{Kim2006}. Allerdings war die Arbeit auch rasch recht umstritten. Zwar behaupten \textcite[S. 783]{Alchian1972}, dass ihr Erklärungsansatz nicht in Konkurrenz zum Transaktionskostenansatz von Coase (und Williamson) steht, gleichzeitig behaupten sie aber, dass langfristige Arbeitsverträge \parencite[S. 784]{Alchian1972} oder Autorität kaum eine Rolle bei der Erklärung der Existenz von Firmen spielen. Insgesamt bleibt ihre Theorie ein wichtiger Baustein zur Erklärung der Existenz von Unternehmen. Noch bedeutender und weitverbreiteter war aber ein recht ähnlicher Ansatz, der das Verhältnis zwischen Unternehmenseigentümer und Arbeitnehmer von mehreren Blickwinkeln betrachtete: \textcite{Jensen1976}. Der Artikel verbindet die Theorie der Unternehmen zum einen mit der Theorie der asymmetrischen Information und begründet in diesem Zusammenhang die "`Principal-Agent-Theory"', die bereits im Kapitel \ref{Info} kurz erwähnt wurde. Zum anderen stellt die Arbeit eine Erweiterung der Kapitalstrukturtheorie dar (vgl. Kapitel \ref{Struktur}). Einer der beiden Autoren, Michael Jensen, war auch eine prägende Figur in der neoklassischen Finanzierungstheorie. Studierenden der Corporate Finance ist sein Name wohl als Erfinder von "`Jensen's Alpha"' bekannt, einer Kennzahl die im CAPM (vgl. Kapitel \ref{Portfolio}) für das Maß von Überrenditen zu, Beispiel von Investmentfonds, herangezogen werden kann. Der überzeugte Wirtschaftsliberale, der an der Chicago School promovierte, propagierte Bonussysteme für Manager wie Aktienoptionen oder Golden Parachutes, also enorme Abfertigungen für Manager im Falle einer feindlichen Übernahme durch konkurrierende Unternehmen \parencite{Rosenwald2024}. Während der "`Great Recession"' kamen diese Managerbonusse massiv in Verruf und auch Jensen gab zu, dass seine ursprünglichen Ideen vollkommen aus dem Ruder gelaufen waren \parencite{Jensen2010}. Die in \textcite{Jensen1976} entwickelte "`Principal-Agent-Theory"' wurde während der "`Great Recession"' der breiten Öffentlichkeit ein Begriff.

Zum Artikel selbst: Eine Prinzipal-Agenten-Beziehung ist ein Vertrag, in dem festgelegt wird, dass der Prinzipal einen Agenten beauftragt für ihn Führungsaufgaben auszuführen. Ein typisches Beispiel dafür ist das Delegieren von Unternehmensführungsaufgaben durch Eigentümer an Manager. Ein rationaler Agent wird primär seinen eigenen Nutzen maximieren. Das muss nicht unbedingt auch den Nutzen des Prinzipals maximieren. Da dieser allerdings Auftraggeber ist, kann er allerdings darauf bestehen, dass der Agent in seinem Sinne handelt. Dazu muss er dem Agenten entsprechende Anreize bieten, oder aber in Überwachungsmaßnahmen (Monitoring) investieren \parencite[S. 310]{Jensen1976}. Der  Wohlstand des Prinzipals wird bei Beauftragung eines Agenten auf jeden Fall reduziert: Entweder durch die Monitoring-Kosten, oder durch die Verfolgung des Eigennutzes durch den Agenten ("`Residual-Kosten"'). \textcite{Jensen1976} nennen die Summe dieses Wohlstandsverlusts "`Agency-Costs"'. Diese sind recht ähnlich dem "`Shirking-Monitoring"'-Problem von \textcite{Alchian1972}, das auf den letzten Seiten dargestellt wurde. \textcite{Jensen1976} gehen aber mit ihrem Modell weitere Schritte. Ihre Theorie zur "`Trennung von Eigentum und Management"' (separation of ownership and control) führt - wie bereits angedeutet - zu einer Theorie der optimalen Kapitalstruktur von Unternehmen. \textcite{Jensen1976} beschreiben, dass "`owner-manager"' - also Unternehmer, die Eigentümer und Manager in einer Person sind - ihren Gesamtnutzen maximieren, indem sie den optimalen Mix aus finanziellen und nicht-finanziellen Nutzen maximieren. Der größte finanzielle Nutzen wird durch (nach-Steuer) Wohlstandsmaximierung erreicht. Aber auch der nicht-finanzielle Nutzen wird maximiert. Darunter führen \textcite[S. 316]{Jensen1976} zum Beispiel attraktives Office-Personal, Spenden an wohltätige Vereine, oder die private Nutzung von Unternehmenseigentum an. Dabei gibt es einen Trade-off zwischen finanziellem und nicht-finanziellem Nutzen: Eine Unternehmerin sollte im Sinne der Gewinnmaximierung den kompetentesten Bewerber für den Sekretärsposten wählen. Im Sinne ihrer persönlichen Nutzenmaximierung wählt sie aber vielleicht nur den dritt- oder viertbesten Bewerber, der dafür mit einem knackigen Hinterteil ausgestattet ist. Je höher nun der Anteil an "`externem Eigenkapital"' - diese Anteile werden gehalten von Eigenkapital-Gebern, die keinen Einfluss auf die Geschäftsführung haben - desto höher ist der Anreiz des owner-managers seinen Gesamtnutzen aus nicht-finanziellen Quellen zu ziehen. Schließlich bezieht er als alleiniger Manager den nicht-finanziellen Nutzen zur Gänze, während er den finanziellen Nutzen mit den externen Eigenkapitalgebern teilen muss. Diese haben wiederum den Anreiz den owner-manager in Richtung rein finanzieller Nutzenmaximierung zu bewegen. Erreichen können sie das - wahrscheinlich nicht vollständig aber zumindest teilweise - durch finanzielle Anreize, wie Bonusse, oder aber durch Monitoring-Maßnahmen. Beides kostet Geld und vermindert so die Gewinne des Unternehmens ebenfalls. Potentielle rationale externe Eigenkapitalgeber werden zudem Vorhersehen wie owner-manager ihr Verhalten ändern, wenn sie vom 100\%-Eigentümer zum Teilhaber werden und ihr Preisangebot für Eigenkapital-Anteile entsprechend anpassen \parencite[S. 316]{Jensen1976}. Dies allerdings reduziert wiederum den Unternehmenswert. Der "`owner-manager"' hat daher auch Anreize potenziellen Investoren glaubhaft zu signalisieren, dass er nicht-finanzielle Vorteile nicht überstrapaziert. So verhindert er, dass Monitoring-Kosten den Unternehmenswert über Gebühr reduzieren. Der "`owner-manager"' optimiert also bei der Aufnahme von externem Eigenkapital seinen eigenen Nutzen, indem er eine Abwägung zwischen Unternehmenswert, nicht-finanziellen Anreizen und den Monitoring-Kosten vornimmt.

Festzuhalten bleibt, dass externes Eigenkapital immer mit Kosten verbunden ist, gleichgültig ob diese als Monitoring-Kosten oder in Form von Unternehmenswert-Reduktion auftauchen. Daher bleibt die Frage, warum "`owner-manager"' überhaupt externes Eigenkapital aufnehmen wollen? Nach allem was wir bisher analysiert haben, wäre es besser das Manager-Eigenkapital ausschließlich mit Fremdkapital zu ergänzen. In der Empirie müsste man demnach Unternehmen jeder Größe sehen, die fast 100\% mit Fremdkapital finanziert sind. \textcite[S. 345]{Jensen1976} liefern drei Gründe. Erstens, so hohe Fremdkapital-Anteile würden den "`owner-manager"' zu extrem riskanten handeln motivieren. Da dies rationalen Fremdkapitalgebern bewusst ist, käme es wiederum zum "`Prinzipal-Agent-Problem"' und damit analog zur Reduktion des Unternehmenswertes. Zweitens - und auch hier handelt es sich um eine Analogie des oben beschrieben "`Prinzipal-Agent-Problems"' - könnte der "`owner-manager"' signalisieren, dass er vollends im Sinne der Fremdkapitalgeber handelt. Aber auch hier würden die Kosten dieses Monitorings den Unternehmenswert entsprechend reduzieren. Drittens, extreme Verschuldungsgrade gehen mit einem hohen Ausfallsrisiko einher, was wiederum Kosten verursacht und dementsprechend den Unternehmenswert reduziert. Abschließend leiten \textcite{Jensen1976} eine Theorie der optimalen Kapitalstruktur ab, bestehend aus "`owner-Manager"', externen Eigenkapitalgeber und Fremdkapitalgebern ab. Diese wird erreicht, indem die Manager darauf abzielen, dass die Agency-Costs minimiert werden. Insgesamt bleibt dieser Ansatz allerdings recht theoretisch, da schon in der wissenschaftlichen Untersuchung des Konzepts die Quantifizierung der verschiedenen genannten Kosten schwierig ist, in der Praxis sind solche noch schwerer zu berechnen. \textcite{Jensen1976} wenden in ihrer Arbeit eine streng positive und individualistische Betrachtungsweise an, die wenige Jahre davor von den Vertretern der "`Neuen Politischen Ökonomie"' entwickelt wurde (vgl. Kapitel \ref{Neue_Politik}). Sie legen in diesem Zusammenhang Wert darauf, dass Organisationen und Unternehmen niemals selbst als solche handeln, sondern stets von nutzenmaximierenden Personen vertreten werden müssen. Außerdem wenden die beiden Autoren auch bereits die Theorie der rationalen Erwartungen an (vgl. Kapitel \ref{Neue Makro}). So gehen sie davon aus, dass potenzielle Anteilseigner beim Unternehmens-Anteils-Kauf das Prinzipal-Agenten-Problem berücksichtigen und ihr Kaufangebot um entsprechende Kosten von Gegenmaßnahmen reduzieren. Wie bereits von \textcite[S. 311]{Jensen1976} vorhergesehen, ist das Prinzipal-Agent-Problem auf sehr viele Bereiche der Zusammenarbeit anzuwenden. Dementsprechend weit hat sich die Idee in den Folgejahren verbreitet. Das Prinzipal-Agent-Problem kann heute wohl als das bekannteste Moral-Hazard-Problem in den Wirtschaftswissenschaften angesehen werden.

Gemeinsam mit Eugene Fama, der wesentliche Beiträge zu Finanzierungstheorie geliefert hat (vgl. Kapitel \ref{Finance}), eröffnete Michael Jensen Anfang der 1980er-Jahre einen Forschungsbereich, der an die Arbeit \textcite{Jensen1976} anschließt. Anstatt jedoch die Auswirkungen der Agency-Probleme auf die Kapitalstruktur zu untersuchen, behandelt dieser Ansatz die Auswirkungen der Trennung von Eigentümerschaft und Management auf die interne Organisation von Unternehmen \parencite[S. 86]{Erlei2016}. 




HIER WEITER

Erlei-Buch S. 71

Daraus entwickelte sich Prinzipal-Agent (Moral-Hazard-Problem): Jensen und Meckling 1976.

Fama, E. F., Jensen, M. C. (1983) Separation of Ownership and Control

Erlei-Buch Seite 113: Weiterentwicklung






Vertragstheorie Incomplete contract: Sanford J. Grossman, Oliver D. Hart, vielleicht auch  Bengt Holmström and Paul Milgrom

Eine wesentliche Weiterentwicklung der Property-Rights-Theorie gelang schließlich Oliver Hart und Sanford Grossman. Wesentlich formalerer Ansatz und vor allem spieltheoretisch formuliert.
Unvollständige Verträge
Hier Überschneidungen zu Spieltheorie, Asymmetrischer Information und Property Rights.

\subsection{Property-Rights-Ansatz}

Das Coase-Theorem besagt, dass beim Fehlen von Transaktionskosten der Markt auch bei vorliegen externer Kosten das gesamtwirtschaftlich beste Ergebnis hervorbringen wird. Welche von den Parteien dabei von den externen Effekten profitiert und welche dafür aufkommen muss, ist für das aggregierte Gesamtergebnis zwar egal, für die einzelnen Parteien hingegen ganz entscheidend. Diesem Problem widmet sich die Theorie der Verfügungsrechte ("`Property rights"'). Diese stellen den Ausgangspunkt eines Forschungszweiges dar, der sich schließlich verbreiterte zum Bereich "`Law and Economics"'. Basierend auf den bahnbrechenden Arbeiten von Ronald Coase, etablierten zunächst vor allem die beiden "`University of California, Los Angeles (UCLA)"'-Ökonomen Armen Alchian und Harold Demsetz diesen Forschungszweig. Insgesamt kann man die Forschungsrichtung aber der wirtschaftsliberalen "`Chicago-School"' zurechnen (vgl. u.a. Kapitel \ref{Monetarismus}). Auch der sehr breite Bereich der Vertragstheorie könnte man als Teilbereich von "`Law and Economics" ansehen. Überschneidungen gibt es mit der Konstitutionen-Ökonomie, einem Teilbereich der Public-Choice-School, die James McGill Buchanan etablierte (vgl. Kapitel \ref{Pol_Econ}). 

Kommen wir zurück zum Beginn des Property-Rights-Ansatz aus historischer Sicht: Wir haben im Beispiel oben argumentiert, dass sich der Getreidebauer und der Viehzüchter ohne Zutun einer Obrigkeit darauf einigen können, wie viele Rinder die Viehherde haben soll. Die Frage ob der Viehzüchter dem Getreidebauern eine Entschädigung zahlen muss, oder nicht, hatten wir beiseite geschoben. Für den individuellen wirtschaftlichen Erfolg ist diese aber entscheidend. \textcite{Demsetz1967} griff Probleme dieser Art erstmals systematisch auf. Bei privatwirtschaftlichen Gütern, also solchen bei denen Rivalität und Ausschließbarkeit gegeben sind (vgl. Kapitel \ref{Offentliche Guter}), sind die Verfügungsrechte klar zugeordnet. Schwieriger wird dies bei Allmende-Gütern, wo die Nicht-Ausschließbarkeit bei gegebener Rivalität zu Übernutzung führt und in weiterer Folge zu negativen externen Effekten. Dies verdeutlicht den engen Zusammenhang zwischen Verfügungsrechten und externen Effekten. Die Verfügungsrechte bestimmen im Endeffekt, wer von Allmende-Gütern profitiert und wer durch ihre Übernutzung geschädigt wird. Durch Verfügungsrechte sollten die externen Effekte idealerweise vollständig internalisiert werden \parencite[S. 348]{Demsetz1967}. Demsetz argumentiert, dass sich Verfügungsrechte evolutionär entwickelt haben. Bei der Nutzung von öffentlichen Gütern (keine Rivalität!) werden keine Verfügungsrechte benötigt. Steigt der Grad an Rivalität, weil zum Beispiel die Nachfrage nach dem ursprünglich frei verfügbarem Gut stark ansteigt, werden öffentliche Güter zu Allmende-Gütern und Verfügungsrechte werden notwendig um Konflikte zu vermeiden. \textcite{Demsetz1967} verwendet ein historisches Beispiel, um dies zu untermauern. Wir werden im Laufe dieses Kapitels bemerken, dass historische Forschungsansätze im "`Neuen Institutionalismus"' allgemein und bis in die Gegenwart eine bedeutende Rolle spielen. Konkret greift Demsetz die Frage auf, wie Naturvölker mit Verfügungsrechten umgehen. Demsetz lehnt sich an die Forschungsarbeiten von Anthropologen an, die feststellten, dass sich Land-Verfügungsrechte und der kommerzielle Pelz-Handel unter Ur-amerikanischen Völkern praktisch gleichzeitig entwickelte. In jenen Gegenden, wo sich Anfang des 18. Jahrhunderts der kommerzielle Pelz-Handel etablierte, kam es wenig später zu abgegrenzten Jagdterritorien und Vereinbarungen zu Jagd-Beschränkungen. Oder mit anderen Worten: Die Nachfrage nach dem Gut Pelze ließ dessen Wert steigen. Dadurch wurde vermehrt Jagd auf das öffentliche Gut Wild gemacht. Ohne Beschränkungen wurde bald mehr Wild erlegt als durch die natürliche Reproduktion ausgeglichen werden konnte. Der unregulierte Jagd verursachte nun den negativen externen Effekt mangelnden Wildbestandes, den alle Einwohner der Region zu spüren bekamen. Aus dem öffentlichen Gut "`Wild"' wurde ein Allmende-Gut. In der Folge einigte man sich darauf, dass bestimmte Stämme nur an bestimmten Orten jagen durften. Den Bestand im eigenen Revier nicht übermäßig zu dezimieren, war nun im Eigeninteresse. Es hatten sich also Verfügungsrechte gebildet, welche die negativen externen Effekte weitgehend internalisierten, also aus den Allmende-Gütern schlussendlich private Güter machten \parencite{Demsetz1967}. Evolutionär, also ohne Eingriff einer externen Institution, entwickeln sich Verfügungsrechte demnach, wenn die Internalisierung von externen Effekten für jeden einzelnen Beteiligten geringere Kosten verursacht als Erträge (oder Nutzen) bringt. Eine recht ähnlichen Ansatz vertritt James McGill Buchanan, der in Kapitel \ref{Pol_Econ} behandelt wird. \textcite{Buchanan1975} vergleicht den Zustand absoluter Freiheit, also Anarchie, mit dem Zustand absoluter staatlicher Allmacht. Letzteres nennt Buchanan "`Leviathanische Zustände"'. Leviathan ist in verschiedenen Religionen in verschiedener Weise ein Vertreter des Bösen. In Thomas Hobbes gleichnamigen Werk ist er schließlich eben die allmächtige staatliche Instanz. In \textcite{Hobbes1651} wird unterschieden in private und kollektive Handlungen. \textcite{Buchanan1975} schließt daran an und unterscheidet in weiterer Folge, wie Spielregeln (die Verfassung) geschaffen werden und in einer zweiten Ebene, wie nach diesen Spielregeln gehandelt wird. Buchanan kommt zum Schluss, dass sowohl Anarchie als auch vollkommene staatliche Allmacht suboptimale Ergebnisse liefern. Das ist wenig überraschend, entscheidend ist aber in weiterer Folge, wie rationale Individuen zu einem Vertragsregelwerk gelangen, das allgemein akzeptiert wird, also für stabile Verhältnisse sorgt. Der Prozess dahin ist etwas weniger naiv als jener von \textcite{Demsetz1967}, aber dennoch evolutionär getrieben: Ausgehend von anarchischem Urzustand entwickeln sich Regeln, die von allen als vorteilhaft empfunden werden. Die Gemeinschaft installiert in der Folge Institutionen, die die Regeln sanktionieren kann. Interessanterweise kommt der ultrakonservative Buchanan zu dem Schluss, dass der Staat nicht nur Schutz, sowohl im Sinne von körperlicher Unversehrtheit, als auch im Sinne des Rechtsschutzes, gewähren soll, sondern auch bestimmte öffentliche Güter zur Verfügung stellen soll. Der Zustand der Anarchie scheitert, weil die absolute individuelle Freiheit dort enden muss, wo sie in die individuelle Freiheit anderer eingreift. Im Ergebnis muss durch die Zuordnung von Verfügungsrechte die individuelle Freiheit eingeschränkt werden. Erst wenn die Verfügungsrechte und deren Umsetzung geregelt sind, können Individuen langfristig planen und investieren \parencite[S. 285]{Erlei2016}. Eine Erkenntnis, die obwohl sie recht banal erscheint, auf die Forschung im Neuen Institutionalismus großen Einfluss genommen hat. Gerade die makroökonomischen Überlegungen zum Zusammenhang zwischen Institutionen und Wirtschaftswachstum erklären Wohlstandsunterschiede dadurch, dass in bestimmten Ländern Institutionen entstanden sind, die die Einhaltung von Verfügungsrechten zusicherten und somit die Grundlage für langfristige Planung und Investitionen durch privatwirtschaftlich organisierte Unternehmen bildeten. In Ländern, in denen solche Strukturen nicht entstanden sind, blieb die Entwicklung rückständig. Bekannt wurden die Forschungen in diese Richtung von \textcite{Soto2000} und \textcite{Acemoglu2012}, dessen Arbeiten in diesem Kapitel noch betrachtet werden.


HIER WEITER
Harold Demsetz durch seinen gemeinsamen Artikel mit Armen Alchian \parencite{Alchian1973}.


\parencite[S. 284]{Erlei2016}
Property-Rights-Theorie + Voigt-Buch: Kapitel 6 (S. 157): Marktversagen aus dieser Sicht.










Private Geschäfte, einmalig: Coase-Theorem (S. 55ff). Moderne Beispiele S. 68f, Höhe von Transaktionskosten: S. 71f


Weitere Kapitel: 

Empirie zu internen vs. externen Institutionen: Ellickson (1994), Stone, Levy und Paredes (1996) und Formlose Beschränkungen \parencite[S. 43]{North1990}: Richard Posner (Auch Beitrag in \textcite{Warsh})

Österreichische Sicht: Schätzung des informellen Sektors und Schwarzarbeit (Schneider)

Stigler 1971: Theory of Economic REgulation

Holmstrom, B. (1979) Moral Hazard and Observability

Williamson, O. E.(1979) Transaction-Cost Economics: Governance of Contractual Relations



Grossman, S. J. Hart, O. D. (1986) The Costs and Benefits of Ownership: A Theory of Vertical and Lateral Integration

Klein, B., Crawford, R. G., Alchian, A. A. (1978) Vertical Integration, Appropriable Rents, and the Competitive Contracting Process

Jensen, M. C. (1986) Agency Costs of Free Cash Flow, Corporate Finance, and Takeovers


 Wie \textit{entstehen} dann aber Institutionen, wenn nicht durch die Anpassung von Präferenzen? Ein typisches Henne-Ei-Problem: Was existierte früher, die Individuen mit ihren gegebenen Präferenzen, oder doch die Institutionen? Bereits Carl Menger nutzte dafür einen Kreislauf-Ansatz bei der Erklärung der Institution Geld: "`Um mühsamen Tauschhandel zu verhindern, wählt man als Alternative die bequemere Verwendung von Geld und umgekehrt ist die Verwendung von Geld bequem, wenn es von möglichst vielen Individuen verwendet wird \parencite[S.176]{Hodgson1998}



\section{North: Die Entstehung und Entwicklung von Institutionen und Staaten}
North
Wettbewerb der Institutionen?



\section{Acemoglu: Kein Wohlstand ohne Institutionen}

Dieses Kapitel behandelt, im Gegensatz zu den bisherigen beiden, die makroökonomische Komponente des "`Neuen Institutionalismus"'. 
Voigt-Buch S. 120ff
Jeffrey Sachs vs. Rodrik und Acemoglu



Verbindung zu Endogener Wachstumstheorie \textcite[S. 633ff]{Snowdon2005} und zur pol. Ökonomie \textcite[S. 562]{Snowdon2005}

Der Neue Institutionalismus ist aktuell eines \textit{der} Themen schlechthin in der Ökonomie. Die Ursprünge kommen dabei zu einem guten Teil aus der Neuen Politischen Ökonomie (vgl. Kapitel \ref{Neue_Politik}). Dementsprechend häufig wurde der Institutionalismus im letzten Kapitel auch genannt.





Kenneth Arrow 1951: Unmöglichkeitstheorem und Social Choice Theorie




Coase, Williamson, North, Olson (Querverbindung auch zu Neuer Politischer Ökonomie)
Meckling und Jensen: Principal Agent Theorem



Ostrom (Querverbindung auch zu Neuer Politischer Ökonomie dort erwähnt!)



