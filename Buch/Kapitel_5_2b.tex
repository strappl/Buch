%%%%%%%%%%%%%%%%%%%%% chapter.tex %%%%%%%%%%%%%%%%%%%%%%%%%%%%%%%%%
%
% sample chapter
%
% Use this file as a template for your own input.
%
%%%%%%%%%%%%%%%%%%%%%%%% Springer-Verlag %%%%%%%%%%%%%%%%%%%%%%%%%%

\chapter{Neue Institutionsökonomik}
\label{Neue Institut}

\section{Der "`alte"' Institutionalismus des Zynikers Thorstein Veblen}

Gleich vorab: Der hier dargestellte "`alte"' (amerikanische) Institutionalismus hat mit dem Hauptthema dieses Kapitels dem "`Neuen Institutionalismus"' nicht allzu viel gemeinsam, abgesehen vom Namen. Aus heutiger Sicher kann deren Hauptvertreter Thorstein Veblen und dessen Arbeit eher als Kuriosität der Wirtschaftsgeschichte angesehen werden. Historisch betrachtet sind aber so wohl seine eigenen Beiträge, vor allem aber jene seiner Schüler hoch interessant. Der Begriff "`Institutionalismus"' wurde erstmals von \textcite{Hamilton1919} verwendet. Auch hier gilt aber, dass diese ökonomische Schule auch Anfang des 20. Jahrhunderts nicht so neu war, wie heute oft dargestellt. Vielmehr wurden wesentliche Elemente davon gerade im deutschsprachigen Raum schon zuvor, aber auch später häufig wissenschaftlich diskutiert. Die Historische Schule der Nationalökonomie (vgl. Kapitel \ref{Historisch}), aber auch die frühen Vertreter der Österreichischen Schule (vgl. \ref{Austria}), sowie später die deutschen Ordoliberalen (vgl. \ref{Ordoliberalismus}) griffen ganz ähnliche Fragestellungen wie der alte Institutionalismus auf. Dieser war vielmehr eine amerikanische Gegenbewegung zur damals noch jungen Neoklassik. 
Die alte Institutionsökonomik - man glaubt es heute kaum - war vor 1900 quasi die Mainstream-Ökonomie in den USA \parencite[S. 97]{Persky2000}. Zwischen Thorstein Veblen und dem "`Vater der amerikanischen Neoklassik"', John Bates Clark (vgl. Kapitel \ref{FisherandClark}) entwickelte sich um die Jahrhundertwende so etwas wie der "`Amerikanische Methodenstreit"' \parencite[S. 100]{Persky2000}.
HIER WEITER.



Veblens Theorie des Institutionalismus nicht in seinem Hauptwerk sondern vor allem im späteren Werk \textcite{Veblen1914} worin er postuliert, dass nicht funktionierende Institutionen das durch Innovationen angetriebene Wachstum ständig bremsen.
HIER WEITER


Neben dem, wie bereits erwähnt, aus heutiger Sicht etwas kurios anmutenden Hauptvertreter Veblen und dessen Theorie, hinterließ der Institutionalismus durchaus nachhaltige Spuren in der Volkswirtschaftslehre. Zunächst - nicht mehr als eine Randnotiz - ist der "`Veblen-Effekt"' nach dem Hauptvertreter benannt: Dieser liefert einen Erklärungsversuch für das selten auftretende Phänomen, wenn steigende Preise zu steigender Nachfrage führen. Demnach ist mit dem Besitz dieser speziellen Güter ein gewisser Status verbunden. Die höheren Preise weiten den Position als Statussymbol noch aus. Der eigentlich zu erwartende negative Effekt auf die Nachfrage durch Preiserhöhungen wird vom Veblen-Effekt übertroffen, sodass steigende Preise eben sogar höher Nachfrage verursachen können. Jeder von uns erinnert sich in diesem Zusammenhang an eine Produkt aus seiner Kindheit oder Jugend, als man die bestimmten Schuhe, Hosen oder Accessoires einfach haben \textit{musste}, wenn man \textit{in} sein wollte. Damit verbunden ist die Theorie des "`Geltungskonsums"', die Veblen in seinem bekanntesten Werk "`The Theory of the Leisure Class"' \parencite{Veblen1899} entwickelte. Diese Konsumtheorie behauptet, dass Personen viele Güter primär deshalb konsumieren, um ihren Status öffentlich darzustellen.

In Bezug auf die gegenwärtigen Wirtschaftswissenschaften wichtiger ist allerdings die Tatsache, dass die heute nicht mehr wegzudenkende Bedeutung der empirischen Wirtschaftsforschung - vor allem im Sinne der Sammlung und Erhebung von empirischen Daten - auf einen Vertreter der Institutionenökonomik zurückgeht, nämlich Wesley Clair Mitchell. Der Schüler Veblen's ist Begründer des National Bureau of Economic Research (NBER) - einer der bis heute wichtigsten wirtschaftswissenschaftlichen Institutionen der USA. In einer Zeit, die reich an bedeutenden Theoretikern war, aber in der Empirie kaum eine Rolle in der Ökonomie spielte - erkannte er die Relevanz hochwertiger Daten. Im 1920 gegründeten NBER war er bis 1945 als Forschungsdirektor tätig. Inhaltlich machte er sich einen Namen als Theoretiker im Bereich der Konjunkturzyklen \parencite{Mitchell1913, Mitchell1946}. Wiederum ein Schüler Mitchell's setzte dessen Pionierarbeit im Bereich der empirischen Wirtschaftsforschung fort: Simon Kuznets. Der aus Russland stammende Kuznets arbeitete gemeinsam mit Mitchell am NBER und beschäftigte sich bahnbrechend mit einem Thema, das bis heute allgegenwärtig ist in der Volkswirtschaftslehre: Er prägte den Begriff des Bruttonationalproduktes \parencite{Kuznets1937} indem er für das Maß der Leistung einer Gesamtwirtschaft standardisierte Berechnungskonzepte einführte \parencite{Nobelpreis-Komitee1971}. Später erweiterte er seine empirische Arbeiten auf die Erforschung des wirtschaftlichen Wachstums \parencite{Kuznets1967}. Im Jahr 1971 erhielt er für seine Arbeiten zur empirischen Wirtschaftsforschung den Nobelpreis für Wirtschaftswissenschaften, der damals erst zum dritten Mal vergeben würde. 

Heute ist er vielen im Zusammenhang mit der nach ihm benannten "`Kuznets-Kurve"' ein Begriff. \textcite[S. 26]{Kuznets1955} stellte - selbst sehr vorsichtig formulierend: "`The paper is perhaps 5 per cent empirical information and 95 per cent speculation"' - die Theorie auf, dass die personelle Einkommenskonzentration in der Frühphase der wirtschaftlichen Entwicklung von Nationen zunächst stark zunimmt. Später, wenn die Industrialisierung einer Ökonomie weit fortgeschritten ist, dreht sich dieser Prozess um und die Einkommenskonzentration nimmt wieder ab. \parencite{Kuznets1955}. Diese Arbeit stellt wohl die erste bedeutende empirische Arbeite zur personellen Einkommensverteilung dar. Heute gilt die Annahme allerdings als widerlegt (vgl. Kapitel \ref{Ungleichheit}). 

Mitchell und Kuznets werden heute häufig als Vertreter des (alten) Institutionalismus bezeichnet. Ihre Bedeutung fußt aber vor allem auf deren Pionierarbeiten im Bereich der empirischen Wirtschaftsforschung, mit den Arbeiten Thorstein Veblens verbindet sie inhaltlich nur wenig.



\section{Die Ursprünge: Transaktionen sind nicht gratis} \label{sec: Neue Inst}

Ronald Coase: Eigentumsrechte statt Pigou-Steuer: Kritik an Pigou \textcite[S. 243]{Cansier1989}. Coase stellt im Hinblick auf externe Effekte - wie die gerade aktuell diskutierten Umweltprobleme - Marktlösungen in den Vordergrund. Ein Staatseingriff ist ihm zufolge nicht unbedingt notwendig (Vergleiche dazu Kapitel \ref{sec: Pigou}).


\section{Acemoglu: Kein Wohlstand ohne Institutionen}
Verbindung zu Endogener Wachstumstheorie \textcite[S. 633ff]{Snowdon2005}

Der Neue Institutionalismus ist aktuell eines \textit{der} Themen schlechthin in der Ökonomie. Die Ursprünge kommen dabei zu einem guten Teil aus der Neuen Politischen Ökonomie (vgl. Kapitel \ref{Neue_Politik}). Dementsprechend häufig wurde der Institutionalismus im letzten Kapitel auch genannt.

Kenneth Arrow 1951: Unmöglichkeitstheorem und Social Choice Theorie



Es ist eine interessante Tatsache, dass als Ausgangspunkt für die "`Neue Institutionsökonomik"' immer wieder das Werk von Coase: \textit{Theory of the Firm} aus dem Jahr 1937 genannt wird. Interessant deshalb, weil zwischen diesem Ausgangspunkt und den weiteren Arbeiten im Bereich Transaktionskostentheorie -  oder Neue Institutionsökonomik überhaupt - ungefähr 30 Jahre vergingen \parencite[S. 148]{Blaug2001}.

Coase, Williamson, North, Olson (Querverbindung auch zu Neuer Politischer Ökonomie)

Prinzipal Agent Theorie


Ostrom (Querverbindung auch zu Neuer Politischer Ökonomie dort erwähnt!)



