%%%%%%%%%%%%%%%%%%%%% chapter.tex %%%%%%%%%%%%%%%%%%%%%%%%%%%%%%%%%
%
% sample chapter
%
% Use this file as a template for your own input.
%
%%%%%%%%%%%%%%%%%%%%%%%% Springer-Verlag %%%%%%%%%%%%%%%%%%%%%%%%%%

\chapter{Neue Institutionsökonomik}
\label{Neue Institut}

\section{Die Ursprünge: Transaktionen sind nicht gratis} \label{sec: Neue Inst}

Es ist eine interessante Tatsache, dass als Ausgangspunkt für die "`Neue Institutionsökonomik"' immer wieder das Werk von Coase: \textit{Theory of the Firm} aus dem Jahr 1937 genannt wird. Interessant deshalb, weil zwischen diesem Ausgangspunkt und den weiteren Arbeiten im Bereich Transaktionskostentheorie -  oder Neue Institutionsökonomik überhaupt - ungefähr 30 Jahre vergingen \parencite[S. 148]{Blaug2001}.

Coase, Williamson, North

Prinzipal Agent Theorie


\section{Acemoglu: Kein Wohlstand ohne Institutionen}
Verbindung zu Endogener Wachstumstheorie

