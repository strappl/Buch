%%%%%%%%%%%%%%%%%%%%% chapter.tex %%%%%%%%%%%%%%%%%%%%%%%%%%%%%%%%%
%
% sample chapter
%
% Use this file as a template for your own input.
%
%%%%%%%%%%%%%%%%%%%%%%%% Springer-Verlag %%%%%%%%%%%%%%%%%%%%%%%%%%

\chapter{Neue Institutionenökonomik}
\label{Neue Institut}

\section{Der "`alte"' Institutionalismus des Zynikers Thorstein Veblen}

Mit dem Übergang von der Klassik zur Neoklassik, ging ein ganz wesentliches Element verloren, nämlich die Analyse der Bedeutung von Regeln, Gewohnheiten, Behörden und dergleichen. Diese Begriffe am weitesten zusammengefasst, bezeichnete man später als Institutionen. Das umfasst nicht nur offensichtliche Institutionen wie etwa Behörden und Gerichte, sondern auch zum Beispiel aufgestellte Regeln wie Gesetze oder gewachsene, ungeschriebene Regeln, wie Usancen. In diesem Sinne sind alle Nicht-Individuen Institution, also auch Unternehmen, Geld oder Gebietskörperschaften. In \textcite[S. 179]{Hodgson1998} werden Institutionen definiert als "`Denk- oder Handlungsweise von gewisser Prävalenz und Dauer, die in den Gewohnheiten einer Gruppe oder in den Bräuchen eines Volkes verankert ist."' \textcite[S. 239]{Veblen1919} definierte Institutionen noch allgemeiner als "`Feste Denkgewohnheiten ('Habits'), die der Allgemeinheit der Menschen gemeinsam sind."'

Die Neoklassik reduzierte die Ökonomie weitgehend auf die Mechanik von Angebot und Nachfrage. Wir haben bereits in Kapitel \ref{Neoklassik} festgestellt, dass dabei wichtige Schritte als gegeben angenommen werden. So bestimmt sich in der Neoklassik der Marktpreis aus dem Gleichgewicht aus Angebot und Nachfrage. Aber was heißt "`bestimmt sich"'? Wie soll man sich diesen sponaten Prozess vorstellen. Walras - der Urvater der Gleichgewichtstheorie - verwendete recht schwammig den Begriff Tatonnement dafür. Es war und ist auch den Neoklassikern stets bewusst, dass der Prozess der Gleichgewichtsfindung nicht spontan, sondern durch handelnde Personen und Prozesse erfolgt und damit Kosten verursacht. Allerdings können diese in den mechanischen neoklassischen Modellen nicht so einfach integriert werden, sodass die Bedeutung von Institutionen in der Neoklassik lange ignoriert wurde. Der Konflikt zwischen Neoklassikern und anderen Wirtschaftswissenschaftlern, ob diese Nicht-Berücksichtigung jeglicher "`Nebenkosten"' zulässig ist, entstand praktisch schon gleichzeitig mit der Marginalistischen Revolution. So behielten unter anderem bereits Carl Menger selbst - also einer der drei "`Hauptdarsteller"' der Marginalistischen Revolution -, vor allem aber seine Nachfolger in der Österreichischen Schule die Problematik institutioneller Prozesse stets auf der Agenda. Institutionen waren daher Anfang des 20. Jahrhunderts kein gänzlich neues ökonomisches Forschungsgebiet. Vielmehr wurden wesentliche Elemente davon gerade im deutschsprachigen Raum schon zuvor, aber auch später häufig wissenschaftlich diskutiert. Die Historische Schule der Nationalökonomie (vgl. Kapitel \ref{Historisch}), aber auch die frühen Vertreter der Österreichischen Schule (vgl. \ref{Austria}), sowie später die deutschen Ordoliberalen (vgl. \ref{Ordoliberalismus}) griffen ganz ähnliche Fragestellungen auf, wie um die Jahrhundertwende der Institutionalismus. Der Begriff "`Institutionalismus"' wurde erstmals von \textcite{Hamilton1919} verwendet und wurde später zur Abgrenzung zu "`alter (oder amerikanischer) Institutionalismus"' weiterentwickelt. Bekannt unter diesem Namen wurde im frühen 20. Jahrhundert eine kleine Gruppe amerikanischer Ökonomen, die die Bedeutung von Institutionen so weit in den Vordergrund stellte, dass sie deren Erkenntnisse mit den Modell der Neoklassik für vollkommen unvereinbar hielt. 

Dies ist übrigens eine Frage, die den Institutionalismus bis heute beschäftigt und deren Vertreter zuweilen spaltet: Können institutionelle Gegebenheiten in neoklassischen Modellen berücksichtigt werden? Die Neuen Institutionalisten beantworten diese Frage weitgehend mit ja. Demnach sind die Modelle und Annahmen der Neoklassik zutreffend, müssen aber um institutionelle Erkenntnisse erweitert werden. Der alten Institutionalismus lehnte die Neoklassik hingegen als Ganzes weitgehend ab. Deren mathematisch-physikalisch-mechanischen Ansätzen stellten die alten Institutionalisten Analogien aus der Biologie oder den Rechtswissenschaften entgegen. Demnach entwickelten sich wirtschaftliche Prozesse und Regeln über Jahrtausende aus evolutionären Prozessen zu dem weiter was sie heute sind. 

Der hier dargestellte "`alte Institutionalismus"' hat aus diesem Grund mit dem eigentlichen Hauptthema des Kapitels dem "`Neuen Institutionalismus"' nicht allzu viel gemeinsam, abgesehen vom Namen. Aus heutiger Sicher kann der Hauptvertreter Thorstein Veblen und dessen Arbeit eher als Kuriosität der Wirtschaftsgeschichte angesehen werden. Obwohl er letztendlich gar kein eigenständiges Theorie-Gebilde zum Institutionalismus schaffen konnte, sondern eigentlich "`nur"' die Schwächen der Neoklassik offenlegte, gehört er zu den bekanntesten amerikanischen Ökonomen des frühen 20. Jahrhunderts. Mehr noch: Die alte Institutionsökonomik - man glaubt es heute kaum - war zwischen 1900 und 1920 quasi die Mainstream-Ökonomie in den USA \parencite[S. 97]{Persky2000} \parencite[S. 166]{Hodgson1998}. Zwischen Thorstein Veblen und dem "`Vater der amerikanischen Neoklassik"', John Bates Clark (vgl. Kapitel \ref{FisherandClark}) entwickelte sich um die Jahrhundertwende so etwas wie der "`Amerikanische Methodenstreit"', der in Form mehrere methodischer Artikel beider Seiten, vor allem im damals noch jungen Quarterly Journal of Economics ausgetragen wurde \parencite[S. 100]{Persky2000}. In den USA hatte die Neoklassik damals noch einen schweren Stand, während der Institutionalismus sich im Aufwind befand \parencite[S. 100]{Persky2000}. 

Der bekannteste Vertreter des alten Institutionalismus ist sicherlich Thorstein Veblen. Die Inhalte, die er beschrieben hat, sind aber gar nicht so einfach zu fassen, bzw. ein einheitliches Denkmuster zu bringen. Unumstritten ist seine klare Ablehnung und fundamentale Kritik der neoklassischen Ansätze. Dies alleine macht aber noch keine ökonomische Denkrichtung aus. Veblen's Ideen waren eindeutig geprägt von der Historischen Schule der Nationalökonomie (vgl. Kapitel \ref{Historisch}). Veblen sprach Französisch und Deutsch \parencite[S. 418]{Hodgson1998b} und die Arbeiten waren ihm dementsprechend zugänglich. Von den deutschen Ökonomen übernahm er auch die Idee jegliches ökonomisches Handeln aus einer evolutionistischen Perspektive zu betrachten. Ebenso wie Darwin die Natur als Evolution erklärte, entwickelten sich demnach menschliche Handlungsweisen über die Jahrhunderte hinweg. Institutionen als zentrales Ergebnis dieser Evolutionsprozesse, sind also gewachsene Strukturen, die sich stets weiterentwickeln \parencite[S. 424]{Dugger1979}. Preise ergeben sich demnach nicht primär aus Angebot und Nachfrage, sondern aus Konventionen, Gewohnheiten und formellen Institutionen. \textcite{Veblen1899} erklärte damit die oft Generationen-übergreifende Existenz von Reichtum und Armut. Nachkommen aus armen Familien wären es demnach gewöhnt arm zu sein und agierten zeitlebens dementsprechend. Umgekehrt agieren reiche Personen ebenso ihrem Stand entsprechend und weitgehend unabhängig vom tatsächlichem Einkommen. \textcite{Duesenberry1949} entwickelte darauf aufbauend nach dem Zweiten Weltkrieg eine Konsumtheorie, die dieses Verhalten abbildet. Obwohl diese empirisch gute Ergebnisse lieferte \parencite[S. 170]{Hodgson1998}, geriet sie weitgehend in Vergessenheit. In die moderne Ökonomie haben es in diesem Zusammenhang allerdings die Begriffe "`Veblen-Effekt"' und "`Geltungskonsum"' geschafft. Ersterer ist in heutigen Ökonomie-Lehrbüchern nicht mehr als eine Randnotiz. Der "`Veblen-Effekt"' liefert einen Erklärungsversuch für das selten auftretende Phänomen, wenn steigende Preise zu steigender Nachfrage führen. Demnach ist mit dem Besitz dieser speziellen Güter ein gewisser Status verbunden. Die höheren Preise weiten deren Position als Statussymbol noch aus. Der eigentlich zu erwartende negative Effekt auf die Nachfrage durch Preiserhöhungen wird vom gegenläufigen Veblen-Effekt übertroffen, sodass steigende Preise in speziellen Fällen eben sogar höhere Nachfrage verursachen. Jeder von uns erinnert sich in diesem Zusammenhang an \textit{das eine} Produkt aus seiner Kindheit oder Jugend, als man die bestimmten Schuhe, Hosen oder Accessoires einfach haben \textit{musste}, wenn man \textit{in} sein wollte. Damit eng verbunden ist die Theorie des "`Geltungskonsums"', die Veblen ebenso in seinem bekanntesten Werk "`The Theory of the Leisure Class"' \parencite{Veblen1899} entwickelte. Diese Theorie behauptet, dass Personen viele Güter primär deshalb konsumieren, um ihren Status öffentlich darzustellen, bzw. ihrem Status entsprechend aufzutreten.

Die Kernthesen seines "`Institutionalismus"' finden sich nicht nur in seinem Hauptwerk, sondern in verschiedenen Artikeln, die später auch in Sammelwerken zusammengefasst wurden (z.B.: \textcite{Veblen1919}). Der Ausgangspunkt seines "`Evolutionären Institutionalismus"' findet sich in seinem Artikel \textcite{Veblen1898}: "`Why is Economics not an Evolutionary Science?"'. Veblen kritisiert die Individuen-Bezogenheit der Neoklassik. Dort ist der nutzenmaximierende Agent der Ausgangspunkt jeder Analyse. Dieser Kritikpunkt blieb zentral im "`alten Institutionalismus"': Die völlige Fehleinschätzung der menschlichen Natur ("`human nature"'). Dies warf Veblen nicht nur den Neoklassikern, sondern auch den anderen führenden Schulen jener Zeit: Der historischen Schule der Nationalökonomie und der Österreichischen Schule vor \parencite[S. 389]{Veblen1898}. Aber auch den Marxismus mit dessen Fokus auf das Kollektiv kritisierte er als unrealistisch. Er lehnte also sowohl den methodischen Individualismus wie auch den methodischen Kollektivismus strikt ab \parencite[S. 426]{Hodgson1998b}. Dem Fokus auf das nutzenmaximierenden Individuum der Neoklassik stellt Veblen die Institutionen, die sich evolutionär entwickelt haben, als zentrales zu analysierendes Element gegenüber \parencite[S. 422]{Hodgson1998b}. In Bezug auf die Individuen gehen die "`alten Institutionalisten"' ebenso von einer Evolution aus. Im Gegensatz zur Neoklassik unterliegen Individuen und deren Präferenzen einem ständigen Anpassungsprozess was vor allem mit den Modellen zu Nachfrage- und Konsumverhalten der Neoklassik unvereinbar ist \parencite[S. 701]{Blaug1962}. Als Resultat beeinflussen sich individuelles Verhalten und institutionelle Strukturen gegenseitig: Institutionen formen Individuen, werden aber umgekehrt auch von Individuen geformt \parencite[S. 181]{Hodgson1998}. 


HIER WEITER.


Veblen war nicht nur was seine wirtschaftswissenschaftlichen Inhalte anging sehr speziell, sondern auch seine Person betreffend. Mehrmals musste er Universitäten wegen "`unpässlichem Verhalten"' verlassen. 
Veblens Theorie des Institutionalismus nicht in seinem Hauptwerk sondern vor allem im späteren Werk \textcite{Veblen1914} worin er postuliert, dass nicht funktionierende Institutionen das durch Innovationen angetriebene Wachstum ständig bremsen.

Heute sind seine Beiträge, vor allem aber jene seiner Schüler vor allem aus historischer Sicht hoch interessant. 



Den "`alte Institutionalismus"' vereinten folgende Grundsätze: Die Ablehnung des individuellen Nutzenmaximierers, die Interdisziplinarität, die Ansätze aus Politik, Biologie, Soziologie und Rechtswissenschaften übernahm. Außerdem nahmen, im Gegensatz zur Neoklassik, mathematisch-mechanische Modelle einen geringen Stellenwert ein. Stattdessen wurde aus "`stylized facts"' und historischen-empirischen Daten Schlüsse gezogen. Kritiker des Ansatzes bezeichneten dieses Vorgehen als rein deskriptive Analyse von Daten. Das ganze Konzept sei rein induktiv und ohne Fokus darauf Theorien bilden zu \textit{wollen} \parencite[S. 703]{Blaug1962}, was Historiker versuchten zu widerlegen \parencite[S. 424]{Hodgson1998b} \parencite[S. 174]{Hodgson1998}. Faktum ist, dass der "`alte Institutionalismus"' kein eigenes Theoriegebilde darstellt. Ganz im Gegenteil, selbst die unterschiedlichen Beiträge der drei Hauptvertreter, Veblen, Commons und Mitchell verbindet nur wenig gemeinsames \parencite[S. 701f.]{Blaug1962}

Übrigens: Laut \textcite{Aspromourgos1986} geht der Begriff "`Neoklassik"' ursprünglich auf Thorstein Veblen zurück.


Als zweiter Begründer des "`alten Institutionalismus"' gilt John Commons, dessen Hauptwerke allerdings erst nach Veblen's Wirken publiziert wurden. Bei ihm stand ebenso die Entwicklung von Institutionen im Sinne von gewachsenen Gewohnheiten im Vordergrund. So argumentierte \textcite[S. 45]{Commons1934}, dass erlernte Fähigkeiten, die von einer Gruppe oder sozialen Gemeinschaft ausgeübt werden zu Routinen wachsen und umgekehrt Routinen und Gewohnheiten von dieser Gruppe wieder weitergegeben werden, wodurch sich diese Routinen als Institutionen festigen \parencite[S. 180]{Hodgson1998}. 

Ideen von Commons einbauen!



HIER WEITER



 Neben dem, wie bereits erwähnt, aus heutiger Sicht etwas kurios anmutenden Hauptvertreter Veblen und dessen Theorie, hinterließ der Institutionalismus durchaus nachhaltige Spuren in der Volkswirtschaftslehre.

In Bezug auf die gegenwärtigen Wirtschaftswissenschaften wichtiger ist allerdings die Tatsache, dass die heute nicht mehr wegzudenkende Bedeutung der empirischen Wirtschaftsforschung - vor allem im Sinne der Sammlung und Erhebung von empirischen Daten - auf einen Vertreter der Institutionenökonomik zurückgeht, nämlich Wesley Clair Mitchell. Der Schüler Veblen's ist Begründer des National Bureau of Economic Research (NBER) - einer der bis heute wichtigsten wirtschaftswissenschaftlichen Institutionen der USA. In einer Zeit, die reich an bedeutenden Theoretikern war, aber in der Empirie kaum eine Rolle in der Ökonomie spielte - erkannte er die Relevanz hochwertiger Daten. Im 1920 gegründeten NBER war er bis 1945 als Forschungsdirektor tätig. Inhaltlich machte er sich einen Namen als Theoretiker im Bereich der Konjunkturzyklen \parencite{Mitchell1913, Mitchell1946}. Wiederum ein Schüler Mitchell's setzte dessen Pionierarbeit im Bereich der empirischen Wirtschaftsforschung fort: Simon Kuznets. Der aus Russland stammende Kuznets arbeitete gemeinsam mit Mitchell am NBER und beschäftigte sich bahnbrechend mit einem Thema, das bis heute allgegenwärtig ist in der Volkswirtschaftslehre \parencite[S. 172]{Hodgson1998} Er prägte den Begriff des Bruttonationalproduktes \parencite{Kuznets1937} indem er für das Maß der Leistung einer Gesamtwirtschaft standardisierte Berechnungskonzepte einführte \parencite{Nobelpreis-Komitee1971}. Später erweiterte er seine empirische Arbeiten auf die Erforschung des wirtschaftlichen Wachstums \parencite{Kuznets1967}. Im Jahr 1971 erhielt er für seine Arbeiten zur empirischen Wirtschaftsforschung den Nobelpreis für Wirtschaftswissenschaften, der damals erst zum dritten Mal vergeben würde. 

Heute ist er vielen im Zusammenhang mit der nach ihm benannten "`Kuznets-Kurve"' ein Begriff. \textcite[S. 26]{Kuznets1955} stellte - selbst sehr vorsichtig formulierend: "`The paper is perhaps 5 per cent empirical information and 95 per cent speculation"' - die Theorie auf, dass die personelle Einkommenskonzentration in der Frühphase der wirtschaftlichen Entwicklung von Nationen zunächst stark zunimmt. Später, wenn die Industrialisierung einer Ökonomie weit fortgeschritten ist, dreht sich dieser Prozess um und die Einkommenskonzentration nimmt wieder ab. \parencite{Kuznets1955}. Diese Arbeit stellt wohl die erste bedeutende empirische Arbeite zur personellen Einkommensverteilung dar. Heute gilt die Annahme allerdings als widerlegt (vgl. Kapitel \ref{Ungleichheit}). 

Mitchell und Kuznets werden heute häufig als Vertreter des (alten) Institutionalismus bezeichnet. Ihre Bedeutung fußt aber vor allem auf deren Pionierarbeiten im Bereich der empirischen Wirtschaftsforschung, mit den Arbeiten Thorstein Veblens verbindet sie inhaltlich nur wenig.



\section{Die Ursprünge: Transaktionen sind nicht gratis} \label{sec: Neue Inst}

Unterschied zum "`alten Institutionalismus"': \parencite[S. 176]{Hodgson1998}

Ronald Coase: Eigentumsrechte statt Pigou-Steuer: Kritik an Pigou \textcite[S. 243]{Cansier1989}. Coase stellt im Hinblick auf externe Effekte - wie die gerade aktuell diskutierten Umweltprobleme - Marktlösungen in den Vordergrund. Ein Staatseingriff ist ihm zufolge nicht unbedingt notwendig (Vergleiche dazu Kapitel \ref{sec: Pigou}).


\section{Acemoglu: Kein Wohlstand ohne Institutionen}
Verbindung zu Endogener Wachstumstheorie \textcite[S. 633ff]{Snowdon2005}

Der Neue Institutionalismus ist aktuell eines \textit{der} Themen schlechthin in der Ökonomie. Die Ursprünge kommen dabei zu einem guten Teil aus der Neuen Politischen Ökonomie (vgl. Kapitel \ref{Neue_Politik}). Dementsprechend häufig wurde der Institutionalismus im letzten Kapitel auch genannt.

Kenneth Arrow 1951: Unmöglichkeitstheorem und Social Choice Theorie



Es ist eine interessante Tatsache, dass als Ausgangspunkt für die "`Neue Institutionsökonomik"' immer wieder das Werk von Coase: \textit{Theory of the Firm} aus dem Jahr 1937 genannt wird. Interessant deshalb, weil zwischen diesem Ausgangspunkt und den weiteren Arbeiten im Bereich Transaktionskostentheorie -  oder Neue Institutionsökonomik überhaupt - ungefähr 30 Jahre vergingen \parencite[S. 148]{Blaug2001}.

Coase, Williamson, North, Olson (Querverbindung auch zu Neuer Politischer Ökonomie)

Prinzipal Agent Theorie


Ostrom (Querverbindung auch zu Neuer Politischer Ökonomie dort erwähnt!)



