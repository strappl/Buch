%%%%%%%%%%%%%%%%%%%%% chapter.tex %%%%%%%%%%%%%%%%%%%%%%%%%%%%%%%%%
%
% sample chapter
%
% Use this file as a template for your own input.
%
%%%%%%%%%%%%%%%%%%%%%%%% Springer-Verlag %%%%%%%%%%%%%%%%%%%%%%%%%%

\chapter{Neue Institutionenökonomik}
\label{Neue Institut}

\section{Der "`alte"' Institutionalismus des Zynikers Thorstein Veblen}

Mit dem Übergang von der Klassik zur Neoklassik, ging ein ganz wesentliches Element verloren, nämlich die Analyse der Bedeutung von Regeln, Gewohnheiten, Behörden und dergleichen. Diese Begriffe in ihrem weitesten Sinne zusammengefasst, bezeichnete man später als Institutionen. Damit umfasst sind also nicht nur Institutionen im heutigen Wortsinn, wie etwa Behörden und Gerichte, sondern auch zum Beispiel aufgestellte Regeln wie Gesetze oder aber auch gewachsene, ungeschriebene Regeln, wie Usancen. In diesem Sinne sind alle Nicht-Individuen in der ökonomischen Theorie Institution. Also auch Unternehmen, das Geld, Gesetze oder Gebietskörperschaften. In \textcite[S. 179]{Hodgson1998} werden Institutionen definiert als "`Denk- oder Handlungsweise von gewisser Prävalenz und Dauer, die in den Gewohnheiten einer Gruppe oder in den Bräuchen eines Volkes verankert ist."' \textcite[S. 239]{Veblen1919} definierte Institutionen noch allgemeiner als "`Feste Denkgewohnheiten ('Habits'), die der Allgemeinheit der Menschen gemeinsam sind."'

Die Neoklassik reduzierte die Ökonomie weitgehend auf die Mechanik von Angebot und Nachfrage. Wir haben bereits in Kapitel \ref{Neoklassik} festgestellt, dass dabei wichtige Schritte als gegeben angenommen werden. So bestimmt sich in der Neoklassik der Marktpreis aus dem Gleichgewicht von Angebot und Nachfrage. Aber was heißt "`bestimmt sich"'? Wie soll man sich diesen "`spontanen Prozess"' vorstellen? Walras - der Urvater der Gleichgewichtstheorie - verwendete recht schwammig den Begriff Tatonnement dafür. Es war und ist auch den Neoklassikern stets bewusst, dass der Prozess der Gleichgewichtsfindung nicht spontan, sondern durch handelnde Personen und Prozesse erfolgt und damit Kosten verursacht. Allerdings können diese in den mechanischen, neoklassischen Modellen nicht so einfach integriert werden, sodass die Bedeutung von Institutionen in der Neoklassik lange ignoriert wurde. Der Konflikt zwischen Neoklassikern und anderen Wirtschaftswissenschaftlern, ob diese Nicht-Berücksichtigung jeglicher "`Nebenkosten"' zulässig ist, entstand aber praktisch schon gleichzeitig mit der Marginalistischen Revolution. So behielten unter anderem bereits Carl Menger selbst - also einer der drei "`Hauptdarsteller"' der Marginalistischen Revolution -, vor allem aber seine Nachfolger in der Österreichischen Schule die Problematik institutioneller Prozesse stets auf der Agenda. Institutionen waren daher Anfang des 20. Jahrhunderts kein gänzlich neues ökonomisches Forschungsgebiet. Vielmehr wurden wesentliche Elemente davon gerade im deutschsprachigen Raum schon zuvor, aber auch später, häufig wissenschaftlich diskutiert. Die Historische Schule der Nationalökonomie (vgl. Kapitel \ref{Historisch}), aber auch die frühen Vertreter der Österreichischen Schule (vgl. \ref{Austria}), sowie später die deutschen Ordoliberalen (vgl. \ref{Ordoliberalismus}) griffen ganz ähnliche Fragestellungen auf, wie um die Jahrhundertwende der Institutionalismus. Der Begriff "`Institutionalismus"' wurde erstmals von \textcite{Hamilton1919} verwendet und später, zur Abgrenzung gegenüber dem "`Neuen Institutionalismus"', zu "`alter (oder amerikanischer) Institutionalismus"' weiterentwickelt. Bekannt unter diesem Namen wurde im frühen 20. Jahrhundert eine kleine Gruppe amerikanischer Ökonomen, die die Bedeutung von Institutionen so weit in den Vordergrund stellte, dass sie ihre Erkenntnisse mit den Modellen der Neoklassik für vollkommen unvereinbar hielt. Diese Frage der Vereinbarkeit ist übrigens eine Frage, die den Institutionalismus bis heute beschäftigt und deren Vertreter zuweilen spaltet: Können institutionelle Gegebenheiten in neoklassischen Modellen berücksichtigt werden? Die Neuen Institutionalisten beantworten diese Frage weitgehend mit ja. Demnach sind die Modelle und Annahmen der Neoklassik zutreffend, müssen aber um institutionelle Erkenntnisse erweitert werden. Der alten Institutionalismus lehnte die Neoklassik hingegen als Ganzes weitgehend ab. Den neoklassischen, mathematisch-physikalisch-mechanischen Ansätzen stellten die alten Institutionalisten Analogien aus der Biologie oder den Rechtswissenschaften gegenüber. Demnach entwickelten sich wirtschaftliche Prozesse und Regeln über Jahrtausende aus evolutionären Prozessen zu dem weiter, was sie heute sind. 

Der hier dargestellte "`alte Institutionalismus"' hat aus diesem Grund mit dem eigentlichen Hauptthema des Kapitels, dem "`Neuen Institutionalismus"', nicht allzu viel gemeinsam, abgesehen vom Namen. Aus heutiger Sicher kann der Hauptvertreter Thorstein Veblen und dessen Arbeit eher als Kuriosität der Wirtschaftsgeschichte angesehen werden. Obwohl er letztendlich gar kein eigenständiges Theorie-Gebilde zum Institutionalismus schaffen konnte, sondern eigentlich "`nur"' die Schwächen der Neoklassik offenlegte, gehört er zu den bekanntesten amerikanischen Ökonomen des frühen 20. Jahrhunderts. Mehr noch: Die alte Institutionsökonomik - man glaubt es heute kaum - war zwischen 1900 und 1920 quasi die Mainstream-Ökonomie in den USA \parencite[S. 97]{Persky2000} \parencite[S. 166]{Hodgson1998}. Zwischen Thorstein Veblen und dem "`Vater der amerikanischen Neoklassik"', John Bates Clark (vgl. Kapitel \ref{FisherandClark}) entwickelte sich um die Jahrhundertwende so etwas wie der "`Amerikanische Methodenstreit"', der in Form mehrere methodischer Artikel beider Seiten, vor allem im damals noch jungen Quarterly Journal of Economics, ausgetragen wurde \parencite[S. 100]{Persky2000}. In den USA hatte die Neoklassik damals noch einen schweren Stand, während der Institutionalismus sich im Aufwind befand \parencite[S. 100]{Persky2000}. Eine Entwicklung, die aber rasch eine Ende fand. 

Der bekannteste Vertreter des alten Institutionalismus ist sicherlich Thorstein Veblen. Die Inhalte, die er beschrieben hat, sind aber nicht so einfach zu fassen, bzw. in ein einheitliches Denkmuster zu bringen. Unumstritten ist seine klare Ablehnung und fundamentale Kritik der neoklassischen Ansätze \parencite[S. 703]{Blaug1962}. Dies alleine macht aber noch keine ökonomische Denkrichtung aus. Veblen's Ideen waren eindeutig geprägt von der Historischen Schule der Nationalökonomie (vgl. Kapitel \ref{Historisch}). Veblen sprach, als Nachfahre norwegischer Einwanderer, Deutsch und auch Französisch \parencite[S. 418]{Hodgson1998b}. Europäische Arbeiten waren ihm dementsprechend leichter zugänglich. Von den deutschen Ökonomen übernahm er auch die Idee jegliches ökonomisches Handeln aus einer evolutionstheoretischen Perspektive zu betrachten. Ebenso wie Darwin die Natur als Evolution erklärte, entwickelten sich demnach menschliche Handlungsweisen über die Jahrhunderte hinweg. Institutionen als zentrales Ergebnis dieser Evolutionsprozesse, sind also gewachsene Strukturen, die sich stets weiterentwickeln \parencite[S. 424]{Dugger1979}. Preise ergeben sich demnach nicht primär aus Angebot und Nachfrage, sondern aus Konventionen, Gewohnheiten und formellen Institutionen. \textcite{Veblen1899} erklärte damit die oft Generationen-übergreifende Existenz von Reichtum und Armut. Nachkommen aus armen Familien wären es demnach gewöhnt arm zu sein und agierten zeitlebens dementsprechend. Umgekehrt agieren reiche Personen ebenso ihrem Stand entsprechend und weitgehend unabhängig vom tatsächlichem Einkommen. \textcite{Duesenberry1949} entwickelte darauf aufbauend nach dem Zweiten Weltkrieg eine Konsumtheorie, die dieses Verhalten abbildet. Obwohl diese empirisch gute Ergebnisse lieferte \parencite[S. 170]{Hodgson1998}, geriet sie weitgehend in Vergessenheit. In den Sprachschatz der modernen Ökonomie haben es in diesem Zusammenhang allerdings die Begriffe "`Veblen-Effekt"' und "`Geltungskonsum"' geschafft. Der "`Veblen-Effekt"' liefert einen Erklärungsversuch für das selten auftretende Phänomen, wenn steigende Preise zu steigender Nachfrage führen. Demnach ist mit dem Besitz dieser speziellen Güter ein gewisser Status verbunden. Die höheren Preise weiten deren Position als Statussymbol noch aus. Der eigentlich zu erwartende negative Effekt auf die Nachfrage durch Preiserhöhungen wird vom gegenläufigen Veblen-Effekt übertroffen, sodass steigende Preise in speziellen Fällen eben sogar höhere Nachfrage verursachen. Jeder von uns erinnert sich in diesem Zusammenhang an \textit{das eine} Produkt aus seiner Kindheit oder Jugend, als man die bestimmten Schuhe, Hosen oder Accessoires einfach haben \textit{musste}, wenn man \textit{in} sein wollte. Damit eng verbunden ist die Theorie des "`Geltungskonsums"', die Veblen ebenso in seinem bekanntesten Werk "`The Theory of the Leisure Class"' \parencite{Veblen1899} entwickelte. Diese Theorie behauptet, dass Personen viele Güter primär deshalb konsumieren, weil sie ihren Status öffentlich darstellen wollen, bzw. um ihrem Status entsprechend aufzutreten. 

Genau dieses Werk \parencite{Veblen1899} wurde ein unmittelbarer Erfolg und vor allem auch außerhalb der Wirtschaftswissenschaften in intellektuellen Kreisen ein weit rezipiertes Werk. Die Kernthesen seines "`Institutionalismus"' finden sich aber primär gar nicht in seinem Hauptwerk, sondern vor allem in den wesentlich weniger beachteten Werken \textcite{Veblen1904} und \textcite{Veblen1923}, sowie verschiedenen Artikeln, die später auch in Sammelwerken zusammengefasst wurden (\textcite{Veblen1919}). Der Ausgangspunkt seines "`Evolutionären Institutionalismus"' findet sich in seinem Artikel \textcite{Veblen1898}: "`Why is Economics not an Evolutionary Science?"'. Veblen kritisiert darin die Individuen-Bezogenheit der Neoklassik. Dort ist der nutzenmaximierende Agent der Ausgangspunkt jeder Analyse. Dieser Kritikpunkt blieb zentral im "`alten Institutionalismus"': Die völlige Fehleinschätzung der menschlichen Natur ("`human nature"'). Dies warf Veblen nicht nur den Neoklassikern, sondern auch den anderen führenden Schulen jener Zeit, der historischen Schule der Nationalökonomie und der Österreichischen Schule, vor \parencite[S. 389]{Veblen1898}. Aber auch den Marxismus mit dessen Fokus auf das Kollektiv kritisierte er als unrealistisch. Er lehnte also sowohl den methodischen Individualismus wie auch den methodischen Kollektivismus strikt ab \parencite[S. 426]{Hodgson1998b}. Dem nutzenmaximierenden Individuum der Neoklassik stellt Veblen die Institutionen, die sich evolutionär entwickelt haben, als zentrales Element gegenüber \parencite[S. 422]{Hodgson1998b}. In Bezug auf die Individuen gehen die "`alten Institutionalisten"' ebenso von einer Evolution aus. Im Gegensatz zur Neoklassik unterliegen Individuen und deren Präferenzen einem ständigen Anpassungsprozess, was vor allem mit den Modellen zu Nachfrage- und Konsumverhalten der Neoklassik unvereinbar ist \parencite[S. 701]{Blaug1962}. Als Resultat beeinflussen sich individuelles Verhalten und institutionelle Strukturen gegenseitig: Institutionen formen Individuen, werden aber umgekehrt auch von Individuen geformt \parencite[S. 181]{Hodgson1998}. In \textcite{Veblen1904} und in seinem letzten Werk \textcite{Veblen1923} legt er den Fokus auf eine etwas andere Thematik, wobei er interessanterweise den technischen Fortschritt in den Mittelpunkt stellt. Etwas, das die neoklassische Theorie damals noch nicht kannte. Während die Unternehmer ausschließlich die Profitmaximierung in den Vordergrund stellen, sorgen die Entwickler für wirtschaftlichen Fortschritt. Die Institutionen wiederum bremsen diesen zumindest kurzfristig, da diese durch Gewohnheiten und gewachsene Handlungen geprägt sind und Neuerungen nur verzögert zulassen \parencite[S. 34]{Erlei2016}. Insgesamt ist Veblen's Spätwerk aber schwer verständlich und blieb zu Veblens Lebensende wenig gelesen und ohne Einfluss auf die zeitgenössische Ökonomie.

Veblen war, nicht nur was seine wirtschaftswissenschaftlichen Inhalte anging, sehr speziell, sondern auch seine Person betreffend. Mehrmals musste er Universitäten wegen "`unpässlichem Verhalten"' - konkret wegen außerehelicher Beziehungen - verlassen. 1906 die Universität Chicago und 1909 Stanford. Danach war er noch bis 1918 an der weit weniger bedeutenden Universität Missouri tätig. Zudem war sein Schreibstil höchst ungewöhnlich. \textcite{Veblen1899} zum Beispiel ist voller Ironie und Satire. Beim Lesen der Inhalte stellt man deren Ernsthaftigkeit eher in Frage. Gegen Ende seiner Karriere gerieten er und sein Werk in den Wirtschaftswissenschaften in Vergessenheit und er starb 1929 weitgehend unbeachtet. Random Fact am Rande: Laut \textcite{Aspromourgos1986} geht der allgegenwärtige Begriff "`Neoklassik"' ursprünglich auf Thorstein Veblen zurück. Zumindest diesbezüglich wirkt seine Arbeit bis heute nach. Aber auch abgesehen davon, erlebte Veblen's Arbeit im Rahmen der Weltwirtschaftskrise nach 1929 ein Revival. Er genießt heute einen verhältnismäßig hohen Bekanntheitsgrad, was doch überraschend ist, angesichts der Tatsache, dass er es nicht schaffte eine geschlossene Theorie des Institutionalismus zu verfassen. Im Gegensatz dazu ist etwa die "`Freiburger Schule"', die durchaus eigene Theorie-Gebilde in inhaltlich ähnlichen Bereichen schuf, heute international vollkommen unbedeutend. 

Als zweiter Begründer des "`alten Institutionalismus"' gilt John Rogers Commons, dessen Hauptwerke (\textcite{Commons1924}, \textcite{Commons1934}) allerdings erst nach Veblen's Wirken publiziert wurden. Obwohl Commons nur fünf Jahre später als Veblen geboren wurde, überschnitten sich ihre Schaffensperioden in Bezug auf Arbeiten zum Institutionalismus nicht. Bei Commons stand dennoch ebenso die Entwicklung von Institutionen im Sinne von gewachsenen Gewohnheiten im Vordergrund. So argumentierte \textcite[S. 45]{Commons1934}, dass erlernte Fähigkeiten, die von einer Gruppe oder sozialen Gemeinschaft ausgeübt werden, zu Routinen wachsen und umgekehrt, Routinen und Gewohnheiten von dieser Gruppe wieder weitergegeben werden, wodurch sie sich als Institutionen etablieren \parencite[S. 180]{Hodgson1998}. Commons ergänzte dabei das Konzept der individuellen "`Gewohnheiten"' (habits) um jenes der kollektiven "`Sitten und Gebräuche"' (customs) \parencite[S. 556ff]{Hodgson2003}. Davon abgesehen unterschied sich das Werk Commons' deutlich von Veblen's Beitrag. Commons ist heute für seine Verbindung zwischen den Wirtschafts- und den Rechtswissenschaften bekannt. Zudem wird sein Werk von "`Neuen Institutionalisten"' weniger stark abgelehnt als jenes von Veblen \parencite[S. 547]{Hodgson2003}. Aber auch er schaffte es in keinster Weise eine einheitliche Theorie des Institutionalismus zu begründen. Ein wesentliches Merkmal in Commons' Werk ist, dass er den Begriff Institutionen etwas enger sah als etwa Veblen und dafür die Bedeutung von Gesetzen als institutionellen Rahmen in den Mittelpunkt stellte. Formales Recht entsteht für ihn zwar häufig aus informellen Sitten und Gebräuchen oder sozialen Abmachungen heraus, allerdings hob er in diesem Zusammenhang die stabilisierende und regulatorische Wirkung des Rechtssetzung-Aktes hervor \parencite[S. 566]{Hodgson2003}. Er grenzte Recht damit von "`spontanen Ordnungen"' ab, akzeptiert zwar, dass es beide Formen - formale Regeln (formale Institutionen) und ungeschriebene Gesetze ("`organische"' Institutionen) - gibt, stellte aber formale Institutionen in den Vordergrund \parencite[S. 568]{Hodgson2003}. Commons sah Institutionen damit primär als "`strukturierte Organisation individueller Willen, die in einem sich entwickelnden Rechtsapparat agieren"' \parencite[S. 568]{Hodgson2003}.

Commons wird oftmals auch als Pionier der Labor-Economics (Arbeits-Ökonomie) bezeichnet und forschte zunächst vor allem zur Bedeutung von Gewerkschaften, Kollektivvertrags-Verhandlungen und sozialer Sicherheit \parencite[S. 44]{Barbash1989}. Im Gegensatz zur modernen Arbeits-Ökonomie lag der Fokus seiner Arbeit allerdings darauf, wie Arbeitnehmerrechte in der Gesetzgebung am besten zu verankern seien. Der Institutionalismus und die Arbeits-Ökonomie dieser Prägung stehen somit nicht nebeneinander als zwei unterschiedliche Forschungsrichtungen, sondern greifen ineinander. Die Karriere von Commons verlief in ihrer Frühphase alles andere als linear. Er arbeitete zunächst als Lehrer, entschied sich aber dann doch dazu bei Richard T. Ely Ökonomie zu studieren. Dieser genoss sein Ökonomie-Studium in Deutschland und war daher geprägt von der "`Historischen Schule der Nationalökonomie"' \parencite{Watkins}, was wiederum Commons Werk beeinflusste. Commons Ideen zu Labor-Economics galten im frühen 20. Jahrhundert in den USA als zumindest unorthodox, wenn nicht sogar links-radikal. Er hatte daher lange Zeit Probleme eine fixe universitäre Anstellung zu bekommen \parencite{Watkins}. Nach mehreren kurzen Intermezzos erhielt er schließlich 1904 eine Professur an der University of Wisconsin, die er bis 1932 behielt. Dort verfasste er auch seine Hauptwerke (\textcite{Commons1924} und \textcite{Commons1934}). Seine Karriere nahm - im Gegensatz zu jener von Veblen - mit fortschreitendem Alter an Fahrt auf. Ein Glücksfall für Commons war, dass er eine Zusammenarbeit mit dem Gouverneur von Wisconsin, Robert M. Folletten, etablieren konnte. Dieser gehörte zu den progressiven Kräften in der amerikanischen Politik zu jener Zeit. Wisconsin wurde so etwas wie ein Labor für Sozialpolitik \parencite[S. 44]{Barbash1989}. Commons etablierte Interessenvertretungen für Arbeitnehmerangelegenheiten ("`American Association for Labor Legislation"') und die US-Politik erkannte zunehmend die Bedeutung der Sozialgesetzgebung. Er war Mitglied der in den 1920er-Jahren von Präsident Wilson gegründeten "'Industrial Relations Commission"' \parencite[S. 45]{Barbash1989}. Seine Ausarbeitungen wurden als Vorlage für die ersten Gesetze zum Arbeitslosengeld herangezogen. Als im Rahmen der "`Great Depression"' wirtschaftsfördernde Maßnahmen allgemeine an Bedeutung zunahmen, wurden auch Commons' Ideen salonfähig. Er und seine Schüler gelten als eine treibende Kraft bei der Entwicklung der Sozialgesetze, die im Rahmen des "`New Deals"' umgesetzt wurden. In den 1930er-Jahren war die Neoklassik auch in den USA bereits zur Mainstream-Ökonomie aufgestiegen. Diese konnte mit Commons' Ideen zu Gewerkschaften, Kollektivverträgen und Arbeitsgesetzen wenig anfangen. Auf der anderen Seite war Commons aber auch weit von marxistischen Ideen entfernt \parencite[S. 47]{Barbash1989}. Seine Arbeit blieb zeitlebens der heterodoxen Ökonomie zuzuordnen. Er unterschätzte die Erklärungskraft des Konzepts der Nutzenmaximierung und verzettelte sich stattdessen in vagen Konzepten wie dem Voluntarismus und seinem Glauben, dass sich langfristig "`vernünftige"' Lösungen durchsetzen würden \parencite[S. 48]{Barbash1989} Sein politischer Einfluss ist damit aus heutiger Sicht höher einzuschätzen als sein Beitrag zur ökonomischen Theorie. 

Den "`alte Institutionalismus"' vereinten folgende Grundsätze: Die Ablehnung des individuellen Nutzenmaximierers, die Interdisziplinarität, die Ansätze aus Politik, Biologie, Soziologie und Rechtswissenschaften übernahm. Außerdem nahmen, im Gegensatz zur Neoklassik, mathematisch-mechanische Modelle einen geringen Stellenwert ein. Stattdessen wurde aus "`stylized facts"' und historisch-empirischen Daten Schlüsse gezogen. Kritiker des Ansatzes bezeichneten dieses Vorgehen als rein deskriptive Analyse von Daten. Das ganze Konzept sei rein induktiv und ohne Fokus darauf Theorien bilden zu \textit{wollen} \parencite[S. 703]{Blaug1962}, was Historiker versuchten zu widerlegen \parencite[S. 424]{Hodgson1998b} \parencite[S. 174]{Hodgson1998}. Faktum ist, dass der "`alte Institutionalismus"' kein eigenes Theoriegebilde darstellt. Ganz im Gegenteil, selbst die unterschiedlichen Beiträge der beiden Hauptvertreter Veblen und Commons verbindet nur wenig gemeinsames \parencite[S. 701f.]{Blaug1962}. 

In Bezug auf die gegenwärtigen Wirtschaftswissenschaften interessanter ist die Tatsache, dass die heute nicht mehr wegzudenkende Bedeutung der empirischen Wirtschaftsforschung - vor allem im Sinne der Sammlung und Erhebung von empirischen Daten - auf einen Vertreter der "`alten Institutionenökonomik"' zurückgeht, nämlich Wesley Clair Mitchell. Mitchell gilt nicht als "`typischer"' Vertreter dieser Schule, wird aber als Schüler Veblen's meist dieser zugerechnet. Mitchell ist Begründer des National Bureau of Economic Research (NBER) - einer der bis heute wichtigsten wirtschaftswissenschaftlichen Institutionen der USA. In einer Zeit, die reich an bedeutenden Theoretikern war, aber in der Empirie kaum eine Rolle in der Ökonomie spielte - erkannte er die Relevanz hochwertiger Daten. Im 1920 gegründeten NBER war er bis 1945 als Forschungsdirektor tätig. Inhaltlich, lehnte Mitchell den evolutionären Ansatz des alten Institutionalismus seines Lehrers Veblen weitgehend ab \parencite[S. 552]{Hodgson2003}. Stattdessen machte er sich einen Namen als Theoretiker im Bereich der Konjunkturzyklen \parencite{Mitchell1913, Mitchell1946}. Wiederum ein Schüler Mitchell's setzte dessen Pionierarbeit im Bereich der empirischen Wirtschaftsforschung fort: Simon Kuznets. Der aus Russland stammende Kuznets arbeitete gemeinsam mit Mitchell am NBER und beschäftigte sich bahnbrechend mit einem Thema, das bis heute allgegenwärtig ist in der Volkswirtschaftslehre \parencite[S. 172]{Hodgson1998}: Er prägte den Begriff des Bruttonationalproduktes. Mit \textcite{Kuznets1937} etablierte er das BIP als standardisiertes Berechnungskonzept für die wirtschaftliche Gesamtleistung einer Ökonomie \parencite{Nobelpreis-Komitee1971}. Später erweiterte er seine empirische Arbeiten auf die Erforschung des wirtschaftlichen Wachstums \parencite{Kuznets1967}. Im Jahr 1971 erhielt er für seine Arbeiten zur empirischen Wirtschaftsforschung den Nobelpreis für Wirtschaftswissenschaften, der damals erst zum dritten Mal vergeben wurde. Heute ist er vielen im Zusammenhang mit der nach ihm benannten "`Kuznets-Kurve"' ein Begriff. \textcite[S. 26]{Kuznets1955} stellte - selbst sehr vorsichtig formulierend: "`The paper is perhaps 5 per cent empirical information and 95 per cent speculation"' - die Theorie auf, dass die personelle Einkommenskonzentration in der Frühphase der wirtschaftlichen Entwicklung von Nationen zunächst stark zunimmt. Später, wenn die Industrialisierung einer Ökonomie weit fortgeschritten ist, dreht sich dieser Prozess um und die Einkommenskonzentration nimmt wieder ab. Diese Arbeit stellt wohl die erste bedeutende empirische Arbeit zur personellen Einkommensverteilung dar. Heute gilt die Annahme allerdings als widerlegt (vgl. Kapitel \ref{Ungleichheit}).

Aus deutschsprachiger Sicht interessant ist die Einordnung Karl Polanyi's als Vertreter des "`alten Institutionalismus"'. Der aus Wien stammende ungarische Soziologe und Wirtschaftswissenschaftler ist in der Soziologie durchaus noch ein Begriff. Sein eigentlich umfangreicheres, ökonomisches Werk ist hingegen im Mainstream in Vergessenheit geraten. Die Hervorhebung der Wichtigkeit von Institutionen in \textcite{Polanyi1944}, erinnert vor allem an den Institutionalismus von John Commons \parencite{Maucourant1995}. Ökonomen diskutieren Polanyi's Rolle als "`alten Institutionalismus"' immer wieder \parencite[S. 183]{Hodgson1998}, \parencite{Cangiani2011}. Insgesamt ist aber sowohl das Werk der "`alten Institutionalisten"', als auch jenes Polanyi's zu heterogen und mögliche Verbindungen sind zu wenig erforscht \parencite{Frerichs2024}, als dass eine eindeutige Zuordnung Polanyi's zum "`alten Institutionalismus"' schlicht nicht möglich ist.


\section{Die Ursprünge des "`Neuen Institutionalismus"': Transaktionen sind nicht gratis} \label{sec: Neue Inst}

Vorab: Der "`Neue Institutionalismus"' steht der modernen Mainstream-Ökonomie wesentlich näher, als der "`Alte Institutionalismus"'. Der Grund dafür liegt im unterschiedlich akzeptierten "`Menschenbild"'. Der "`Alte Institutionalismus"' akzeptiert, dass sich Präferenzen von Individuen im Laufe ihres Lebens und nicht zuletzt durch den Einfluss von Institutionen ändern können. Dies ist nicht vereinbar mit dem Bild des rational handelnden "`homo oeconomicus"', der auf Grundlage seiner gegebenen Präferenzen seinen Nutzen maximiert. Die Theorien der Mainstream-Ökonomie folgen strikt einen individualistischem Ansatz, der Individuen und ihre Präferenz als gegeben sieht. Im Gegensatz zu den "`Alten Institutionalisten"' akzeptieren die (meisten) Vertreter des "`Neuen Institutionalismus"' diesen Ansatz \parencite[S. 177]{Hodgson1998}.
Der "`Neue Institutionalismus"' liefert dementsprechend auch eine Definition von "`Institutionen"': 

Definition: \textcite{North1990}, \textcite{Ostrom1986}, \textcite{Riker1980} 


 Wie \textit{entstehen} dann aber Institutionen, wenn nicht durch die Anpassung von Präferenzen? Ein typisches Henne-Ei-Problem: Was existierte früher, die Individuen mit ihren gegebenen Präferenzen, oder doch die Institutionen? Bereits Carl Menger nutzte dafür einen Kreislauf-Ansatz bei der Erklärung der Institution Geld: "`Um mühsamen Tauschhandel zu verhindern, wählt man als Alternative die bequemere Verwendung von Geld und umgekehrt ist die Verwendung von Geld bequem, wenn es von möglichst vielen Individuen verwendet wird \parencite[S.176]{Hodgson1998}


Ronald Coase: Eigentumsrechte statt Pigou-Steuer: Kritik an Pigou \textcite[S. 243]{Cansier1989}. Coase stellt im Hinblick auf externe Effekte - wie die gerade aktuell diskutierten Umweltprobleme - Marktlösungen in den Vordergrund. Ein Staatseingriff ist ihm zufolge nicht unbedingt notwendig (Vergleiche dazu Kapitel \ref{sec: Pigou}).


\section{Acemoglu: Kein Wohlstand ohne Institutionen}
Verbindung zu Endogener Wachstumstheorie \textcite[S. 633ff]{Snowdon2005}

Der Neue Institutionalismus ist aktuell eines \textit{der} Themen schlechthin in der Ökonomie. Die Ursprünge kommen dabei zu einem guten Teil aus der Neuen Politischen Ökonomie (vgl. Kapitel \ref{Neue_Politik}). Dementsprechend häufig wurde der Institutionalismus im letzten Kapitel auch genannt.

Kenneth Arrow 1951: Unmöglichkeitstheorem und Social Choice Theorie



Es ist eine interessante Tatsache, dass als Ausgangspunkt für die "`Neue Institutionsökonomik"' immer wieder das Werk von Coase: \textit{Theory of the Firm} aus dem Jahr 1937 genannt wird. Interessant deshalb, weil zwischen diesem Ausgangspunkt und den weiteren Arbeiten im Bereich Transaktionskostentheorie -  oder Neue Institutionsökonomik überhaupt - ungefähr 30 Jahre vergingen \parencite[S. 148]{Blaug2001}.

Coase, Williamson, North, Olson (Querverbindung auch zu Neuer Politischer Ökonomie)
Meckling und Jensen: Principal Agent Theorem



Ostrom (Querverbindung auch zu Neuer Politischer Ökonomie dort erwähnt!)



