%%%%%%%%%%%%%%%%%%%%% chapter.tex %%%%%%%%%%%%%%%%%%%%%%%%%%%%%%%%%
%
% sample chapter
%
% Use this file as a template for your own input.
%
%%%%%%%%%%%%%%%%%%%%%%%% Springer-Verlag %%%%%%%%%%%%%%%%%%%%%%%%%%

\chapter{Neue Politische Ökonomie}
\label{Neue_Politik}

Der Ökonomie-Zweig "`Politische Ökonomie"' ist nicht so einfach zu erfassen. Wie der Begriff selbst schon ausdrückt, handelt es sich hierbei um die Beschreibung gesamtwirtschaftlicher Zusammenhänge unter dem Einfluss von zentralen Entscheidungsträgern. Die meisten Klassiker (vgl. Kapitel \ref{Klassik}) interpretierten die Volkswirtschaftslehre in genau diesem Sinne und bezeichneten oder beschrieben ihre Werke als Arbeiten in "`Politischer Ökonomie"'. Paradebeispiele hierfür sind vor allem die Arbeiten von David Ricardo, John Stuart Mill, aber auch Karl Marx. Mit  dem Aufstieg der Neoklassik seit 1870 wurde die Volkswirtschaftslehre immer stärker zu einer positiven Wissenschaft und ebenso zu einer Volkswirtschafts\textit{theorie}, in welcher der politische Einfluss weniger bedeutsam war. Damit einher ging die wachsende Marktgläubigkeit, also die Überzeugung, dass die allermeisten Märkte gut funktionieren und keinen staatlichen Eingriff benötigen. Damit verbunden waren tatsächlich niedrige Staatsquoten die Jahrhundertwende. Die Staatsquote drückt aus, wie hoch der Anteil der Staatsausgaben am Bruttoinlandsprodukt ist. Damit ist die Staatsquote ein Maß dafür, an welchem Anteil der gesamten wirtschaftlichen Tätigkeiten die öffentliche Hand beteiligt ist. Um 1900 betrug dieser Wert für die damaligen Industriestaaten zwischen 10\% und 15\%. Den Einfluss des Staates zu vernachlässigen, war in der hohen Zeit der Neoklassik also eine nicht allzu unrealistische Vereinfachung. Vor der Corona-Krise\footnote{Während der Corona-Krise stiegen die Staatsquote stark an. Zum Einen, weil das BIP sank, vor allem aber aufgrund der teilweise sehr hohen Staatshilfen.} lag die Staatsquote in den westlichen Industriestaaten zwischen 40\% in den angelsächsischen Ländern und 50\% in den kontinental-europäischen Sozialstaaten. Der Anteil - aber auch der Einfluss - der öffentlichen Hand an der gesamten Wertschöpfung ist in den letzten hundert Jahren also enorm gestiegen. Dafür gibt es verschiedene Erklärungsansätze. \textcite{Wagner1892} bereits hatte eine fortlaufend steigende Staatsquote theoretisch postuliert. Mit wirtschaftlicher Entwicklung steige nämlich der Bedarf an den typischen Leistungen des Sozialstaates. Die Vorhersage Wagner's ist beeindruckend und bewahrheitete sich zwischen 1900 und 1975 für praktisch alle entwickelten Länder. Die Gründe hierfür waren allerdings nicht nur wachsende Sozialausgaben, sondern auch die beiden Weltkriege, in Folge deren der Staat wesentliche Aufgaben übernahm. Aus unserer wirtschafts-theoretischen Sicht allerdings, ist vor allem der Aufstieg des Keynesianismus (vgl. Kapitel \ref{Keynes}) von Interesse. Mit der Geburt der Makroökonomie durch die Veröffentlichung von \textcite{Keynes1936}  wurden der öffentlichen Hand nämlich auch wichtige wirtschaftspolitische Aufgaben in Form von aktiver Fiskal- und Geldpolitik zugeschrieben. In der folgenden hohen Zeit der Neoklassischen Synthese (vgl. Kapitel \ref{Synthese}) wurde der aktive Eingriff in die Gesamtwirtschaft sogar noch weiter forciert. Erst ab den 1970er Jahren, mit dem Auftreten von Stagflation und dem Wiedererstarken \textit{theoretischer}, liberaler Wirtschaftsschulen (vgl. Kapitel \ref{Monetarismus} und Kapitel \ref{Neue Makro}), kamen auch Fragen zur uneingeschränkten wirtschafts\textit{politischen} Sinnhaftigkeit staatlicher Eingriffe auf.

Dieser Fragen nahm sich die "`Neue Politische Ökonomie"' an. Es handelt sich hierbei um keine geschlossene ökonomische Denkrichtung, sondern um eine recht lose Sammlung verschiedener Ansätze. Die Neue Politische Ökonomie blickt sozusagen hinter die Kulissen der Staatseingriffe. Sie untersucht nicht, \textit{ob} Staatseingriffe in einer gewissen Situation zu befürworten sind, sondern, \textit{wie} deren Durchführung zustande kommt und ob zu erwarten ist, dass durch den Eingriff schlussendlich tatsächlich das reine Marktergebnis verbessert wird. Interessanterweise hatte Keynes selbst Zeit seines Lebens eine recht bescheidene Meinung über Politiker \parencite[S. 519]{Snowdon2005}. Obwohl er mehr als die meisten anderen Ökonomen mit Politikern eng zusammenarbeitete und sein erstes berühmtes Werk (\textcite{Keynes1919}: "`The Economic Consequences of the Peace"') direkt auf das Versagen von Politikern verwies, postulierte er, dass durch staatliche Eingriffe die reinen Marktlösungen verbessert werden können, ohne daran zu denken, dass auch diese staatlichen Eingriffe letztendlich von Menschen durchgeführt werden müssen, die wiederum Interessen verfolgen könnten, die nicht der Allgemeinheit dienen, sondern ausschließlich ihnen selbst. Richtigerweise relativiert Alberto Alesina - einer der führenden Vertreter der Neuen Politischen Ökonomie - allerdings, dass niemand, auch nicht Keynes, bei einer revolutionären Ausarbeitung alle Eventualitäten bis zum Ende durchdenken könne \parencite[S. 569]{Snowdon2005}. 

HIER WEITER:
Drei Richtungen: Medianwähler + Folgen: Ideologie, oder Opportunismus? Folgen: Rules rather than discretion? Unabhängige Zentralbanken und Schuldenbremse.

Theoretisches Modell: Arrow Paradox

Buchanan


Außerdem: Institutionen (eigenes Kapitel), Fragen der Ungleichheit im 21. Jahrhundert.

\section{Downs: Das Medianwähler-Modell}

\section{Arrow: Social Choice Theory}

Arrow: Unmöglichkeitstheorem.
Verbindung mit Wohlfahrtsökonomien vgl. Kapitel \ref{Wohlfahrt}.


\section{Buchanan: Die Public-Choice Theorie}
\label{Pol_Econ}

Eine neue \textit{makroökonomische} Theorie der politischen Ökonomie prägte in den 1970er Jahren James Buchanan. Er griff vor allem die Tatsache auf, dass durch die Keynesianische Revolution aktiver Wirtschaftspolitik eine große Bedeutung auch in sonst marktwirtschaftlich organisierten Ökonomien zukommt. Dabei blieb stets die Frage unbehandelt, ob denn der Staat, beziehungsweise dessen Repräsentanten, überhaupt besser als der Markt ungewollten Entwicklungen, erstens, gegensteuern \textit{kann}, und zweitens, \textit{will}.

Eventuell als eigenes Kapitel!


\section{Politik als Optimierungsaufgabe}
Snowdon-Buch: Kapitel Politische Ökonomie

\textcite{Nordhaus1975} und Hibbs.

Nach Theorie der rationalen Erwartungen:

Alberto Alessina
Unabhängigkeit der Zentralbanken Romer-Buch: S 638, berühmtes Paper Alesina1988

Friedrich Schneider


George Stigler: Regulierung.