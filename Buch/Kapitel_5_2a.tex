%%%%%%%%%%%%%%%%%%%%% chapter.tex %%%%%%%%%%%%%%%%%%%%%%%%%%%%%%%%%
%
% sample chapter
%
% Use this file as a template for your own input.
%
%%%%%%%%%%%%%%%%%%%%%%%% Springer-Verlag %%%%%%%%%%%%%%%%%%%%%%%%%%

\chapter{Neue Politische Ökonomie}
\label{Neue_Politik}

\section{Social Choice Theory}

Arrow: Unmöglichkeitstheorem.
Verbindung mit Wohlfahrtsökonomien vgl. Kapitel \ref{Wohlfahrt}.


\section{Buchanan: Die Public-Choice Theorie}
\label{Pol_Econ}

Eine neue \textit{makroökonomische} Theorie der politischen Ökonomie prägte in den 1970er Jahren James Buchanan. Er griff vor allem die Tatsache auf, dass durch die Keynesianische Revolution aktiver Wirtschaftspolitik eine große Bedeutung auch in sonst marktwirtschaftlich organisierten Ökonomien zukommt. Dabei blieb stets die Frage unbehandelt, ob denn der Staat, beziehungsweise dessen Repräsentanten, überhaupt besser als der Markt ungewollten Entwicklungen, erstens, gegensteuern \textit{kann}, und zweitens, \textit{will}.

Eventuell als eigenes Kapitel!


\section{Politik als Optimierungsaufgabe}
Snowdon-Buch: Kapitel Politische Ökonomie

Nordhaus und Hibbs.

Nach Theorie der rationalen Erwartungen:

Alberto Alessina
Unabhängigkeit der Zentralbanken Romer-Buch: S 638, berühmtes Paper Alesina1988

Friedrich Schneider