%%%%%%%%%%%%%%%%%%%%% chapter.tex %%%%%%%%%%%%%%%%%%%%%%%%%%%%%%%%%
%
% sample chapter
%
% Use this file as a template for your own input.
%
%%%%%%%%%%%%%%%%%%%%%%%% Springer-Verlag %%%%%%%%%%%%%%%%%%%%%%%%%%

\chapter{Neue Politische Ökonomie}
\label{Neue_Politik}

Der Ökonomie-Zweig "`Politische Ökonomie"' ist nicht so einfach zu erfassen. Wie der Begriff selbst schon ausdrückt, handelt es sich hierbei um die Beschreibung gesamtwirtschaftlicher Zusammenhänge unter dem Einfluss von zentralen Entscheidungsträgern. Die meisten Klassiker (vgl. Kapitel \ref{Klassik}) interpretierten die Volkswirtschaftslehre in genau diesem Sinne und bezeichneten oder beschrieben ihre Werke als Arbeiten in "`Politischer Ökonomie"'. Paradebeispiele hierfür sind vor allem die Arbeiten von David Ricardo, John Stuart Mill, aber auch Karl Marx. Mit dem Aufstieg der Neoklassik seit 1870 wurde die Volkswirtschaftslehre immer stärker zu einer positiven Wissenschaft und ebenso zu einer Volkswirtschafts\textit{theorie}, in welcher der politische Einfluss weniger bedeutsam war. Damit einher ging die wachsende Marktgläubigkeit, also die Überzeugung, dass die allermeisten Märkte gut funktionieren und keinen staatlichen Eingriff benötigen. Damit verbunden waren tatsächlich niedrige Staatsquoten die Jahrhundertwende. Die Staatsquote drückt aus, wie hoch der Anteil der Staatsausgaben am Bruttoinlandsprodukt ist. Damit ist die Staatsquote ein Maß dafür, an welchem Anteil der gesamten wirtschaftlichen Tätigkeiten die öffentliche Hand beteiligt ist. Um 1900 betrug dieser Wert für die damaligen Industriestaaten zwischen 10\% und 15\%. Den Einfluss des Staates zu vernachlässigen, war in der hohen Zeit der Neoklassik also eine nicht allzu unrealistische Vereinfachung. Vor der Corona-Krise\footnote{Während der Corona-Krise stiegen die Staatsquote stark an. Zum Einen, weil das BIP sank, vor allem aber aufgrund der teilweise sehr hohen Staatshilfen.} lag die Staatsquote in den westlichen Industriestaaten zwischen 40\% in den angelsächsischen Ländern und 50\% in den kontinental-europäischen Sozialstaaten. Der Anteil - aber auch der Einfluss - der öffentlichen Hand an der gesamten Wertschöpfung ist in den letzten hundert Jahren also enorm gestiegen. Dafür gibt es verschiedene Erklärungsansätze. \textcite{Wagner1892} bereits hatte eine fortlaufend steigende Staatsquote theoretisch postuliert. Mit wirtschaftlicher Entwicklung steige nämlich der Bedarf an den typischen Leistungen des Sozialstaates. Die Vorhersage Wagner's ist beeindruckend und bewahrheitete sich zwischen 1900 und 1975 für praktisch alle entwickelten Länder. Die Gründe hierfür waren allerdings nicht nur wachsende Sozialausgaben, sondern auch die beiden Weltkriege, in Folge deren der Staat wesentliche Aufgaben übernahm. Aus unserer wirtschafts-theoretischen Sicht allerdings, ist vor allem der Aufstieg des Keynesianismus (vgl. Kapitel \ref{Keynes}) von Interesse. Mit der Geburt der Makroökonomie durch die Veröffentlichung von \textcite{Keynes1936}  wurden der öffentlichen Hand nämlich auch wichtige wirtschaftspolitische Aufgaben in Form von aktiver Fiskal- und Geldpolitik zugeschrieben. In der folgenden hohen Zeit der Neoklassischen Synthese (vgl. Kapitel \ref{Synthese}) wurden der aktive Staatseingriffe in die Gesamtwirtschaft sogar noch weiter forciert. Wer aber dieser Staat ist und ob seine Handlungen tatsächlich immer optimal für die Gesellschaft sind, wurde dabei nicht thematisiert. Interessanterweise hatte Keynes selbst Zeit seines Lebens eine recht bescheidene Meinung über Politiker \parencite[S. 519]{Snowdon2005}. Obwohl er mehr als die meisten anderen Ökonomen mit Politikern eng zusammenarbeitete und sein erstes berühmtes Werk (\textcite{Keynes1919}: "`The Economic Consequences of the Peace"') direkt auf das Versagen von Politikern verwies, postulierte er, dass durch staatliche Eingriffe die reinen Marktlösungen verbessert werden können, ohne daran zu denken, dass auch diese staatlichen Eingriffe letztendlich von Menschen durchgeführt werden müssen, die wiederum Interessen verfolgen könnten, die nicht der Allgemeinheit dienen, sondern ausschließlich ihnen selbst. Richtigerweise relativiert Alberto Alesina - einer der führenden Vertreter der Neuen Politischen Ökonomie - allerdings, dass niemand, auch nicht Keynes, bei einer revolutionären Ausarbeitung alle Eventualitäten bis zum Ende durchdenken könne \parencite[S. 569]{Snowdon2005}. Erst ab den 1970er Jahren, mit dem Auftreten von Stagflation und dem Wiedererstarken \textit{theoretischer}, liberaler Wirtschaftsschulen (vgl. Kapitel \ref{Monetarismus} und Kapitel \ref{Neue Makro}), kamen auch Fragen zur uneingeschränkten wirtschafts\textit{politischen} Sinnhaftigkeit staatlicher Eingriffe auf. 

Aber auch auf mikroökonomischer Ebene wurden Fragen zur Bedeutung des Staates immer wichtiger. Wie in Kapitel \ref{sec: Pigou} bereits dargestellt, wurde in den 1920er Jahren innerhalb der Neoklassischen Schule die Notwendigkeit von Staatseingriffen im Falle von Marktversagen erkannt und diskutiert. Daneben entwickelte sich die Disziplin der "`Wohlfahrtsökonomie"', bzw. nach dem Zweiten Weltkrieg die "`Neue Wohlfahrtsökonomie"', dargestellt in Kapitel \ref{Wohlfahrt}. Die Bedeutung dieser Schule ist umstritten, weil es sich dabei um eine normative Theorie handelt. Wohlfahrtsökonomen rangen immer damit, ob Werturteile in der Ökonomie zulässig sind. Beide Schulen haben aber gemeinsam, dass sie - wie der Keynesianismus - recht unbestimmt von staatlichen Eingriffen sprechen. 

Genau hier setzt die "`Neue Politische Ökonomie"' ("`Public Choice Theory"') an. Konkret stellt sich nämlich die Frage "`Wer ist denn eigentlich der Staat?"' und "`Wie trifft dieser seine Entscheidungen?"' Schon anhand dieser beiden Fragen sieht man: \textit{Den Staat} als kollektives Entscheidungsorgan, gibt es in demokratischen Gesellschaften nicht. Stattdessen gibt es viele Individuen, die über nutzenmaximierendes Verhalten ihre Volksvertreter wählen, die wiederum den Staat repräsentieren. Dieser Prozess von "`individuellen Werten"' zu "`politischen Entscheidungen"' kann nicht einfach durch die Annahme eines "`Staates"' übergangen werden \parencite[S. 11]{Buchanan1962}. Genau diesem Problem und den daraus resultierenden Fragen nahm sich die "`Neue Politische Ökonomie"' an. Es handelt sich hierbei um keine geschlossene ökonomische Denkrichtung, sondern um eine recht lose Sammlung verschiedener Ansätze. Die Neue Politische Ökonomie blickt sozusagen hinter die Kulissen der Staatseingriffe. Sie untersucht nicht, \textit{ob} Staatseingriffe in einer gewissen Situation zu befürworten sind, sondern, \textit{wie} deren Durchführung zustande kommt und erst in weiterer Folge, ob zu erwarten ist, dass durch den Eingriff schlussendlich tatsächlich das reine Marktergebnis verbessert wird.

Die Neue Politische Ökonomie behandelt damit ähnliche Fragen wie die Wohlfahrtstheorie (vgl. Kapitel \ref{Wohlfahrt}), steht aber gleichzeitig im Spannungsverhältnis mit dieser. Beide Schulen suchen nach Antworten darauf, wie der gesamtgesellschaftliche Nutzen maximal wird. Die Wohlfahrtstheorie ist dabei allerdings eher eine normative Theorie, gibt also vor wie die Dinge idealerweise sein \textit{sollen} damit maximale Wohlfahrt für die Gesellschaft entsteht. Die Neue Politische Ökonomie sieht sich, zumindest überwiegend, als positive Theorie. Auch ihr geht es darum zu analysieren, wie in Gesellschaften nutzenmaximierende Entscheidungen getroffen werden. Allerdings ist die Neue Politische Ökonomie nüchterner was den Weg dorthin betrifft und geht davon aus, dass Wähler, Politiker und Institutionen primär ihren individuellen Nutzen maximieren, was zu gesamtgesellschaftlich nicht optimalen Ergebnissen führen kann. Die Schule, die dieses Spannungsverhältnis zwischen Wohlfahrtsökonomie und Neuer Politischer Ökonomie konkret behandelt ist die "`Social Choice Theorie"', die von \textcite{Arrow1950, Arrow1951} begründet wurde und bereits im Kapitel \ref{Wohlfahrt} behandelt wurde.

Die Neue Politische Ökonomie hat keinen alleinigen Begründer, oder Hauptvertreter. Als Vorläufer wird häufig Joseph Schumpeter genannt (\textcite[S. 519]{Snowdon2005}, \textcite[S. 95]{Warsh}), der bereits früh nach dem Erscheinen von Keynes' General Theory darauf hinwies, dass sich Politiker in Demokratien  um ihre Jobs immer wieder bei den Wählern bewerben müssen und dies ihre Handlungen - nicht immer zum Wohlergehen der Gesellschaft - beeinflusst \parencite{Schumpeter1942}. Bereits noch früher schlug der, seiner Zeit in vielen Belangen voraus gewesene, schwedische Ökonom Knut Wicksell (vgl. Kapitel \ref{Wicksell}) in eine ähnliche Kerbe. In seinen "`Finanztheoretischen Untersuchungen"' \parencite{Wicksell1896} bezeichnet er Wahlen als "`quid pro quo"'-Geschäft, also als Tauschgeschäft, in dem die Wähler von den Politikern etwas zurück bekommen wollen. Seine theoretischen Ansätze nannten \textcite[S. 8]{Buchanan1962} später als sehr inspirierendes Werk für ihre eigene bahnbrechende Arbeit.

Als moderne Begründer der Neuen Politischen Ökonomie werden heute verschiedene Ökonomen und deren Werke angeführt \parencite[S. 31]{Grofman2004}:
\begin{itemize}
	\item \textcite{Arrow1951, Arrow1950}, der mit der Begründung der "`Social Choice Theory"' und dem darin postulierten "`Unmöglichkeitstheorem"' die grundlegenden Konzepte der damaligen Wohlfahrtstheorie, wie die Existenz einer "`Sozialen Wohlfahrtsfunktion"', widerlegte. Dieser Zweig wurde bereits in Kapitel \ref{Wohlfahrt} behandelt.
	\item \textcite{Black1948a, Black1958}, der neben Arrow als Begründer der "`Social Choice Theory"' gilt, aber erst durch dessen Arbeiten bekannt wurde. Er griff als erster Probleme beim Abstimmungsverhalten theoretisch und begründete das Medianwähler-Modell \parencite{Black1948a}.
	\item \textcite{Downs1957b, Downs1957}, der als erster demokratische Wahlen als Markt interpretierte, auf dem Politiker Wahlversprechen anbieten um die nachfragenden Wähler zu überzeugen. Diese Schule etablierte einen sehr umfangreichen Forschungszweig, der bis heute recht aktiv ist und sich damit beschäftigt, inwieweit Politiker an Regeln gebunden werden sollen \parencite[S. 523]{Snowdon2005}.
	\item \textcite{Buchanan1962} sind die vielleicht bekanntesten Vertreter der Neuen Politischen Ökonomie. Sie vertieften die Idee, dass Politiker an bestimmte Regeln gebunden werden sollen und etablierten damit die Ideen einer Wirtschaftsverfassung "`constitutional economics"'. Gordon Tullock analysierte als erster das Problem des "`rent seekings"', also das Bestreben durch Lobbyismus selbst aufwandsloses Einkommen zu beziehen (Der Begriff selbst wurde durch \textcite{Krueger1974} geprägt).
\end{itemize}

HIER WEITER
Eventuell: Olson, Becker, Baumol
Eventuell: Ungleichheit?

\section{Black: Das Medianwähler-Modell}

Ausgangspunkt: \textcite{Hotelling1929}


\section{Downs: Wahlergebnisse als Marktergebnisse}

Anthony \textcite{Downs1957b, Downs1957} mit seinen einflussreichen Werken zur ökonomischen Theorie politischer Handlungen in Demokratien, kritisierte, dass Staatseingriffe in der ökonomischen Theorie stets als exogene Variable herangezogen werden, ohne zu Bedenken, dass Staatseingriffe von nutzenmaximierenden Personen vollzogen werden müssen \parencite[S. 135]{Downs1957}. Er lieferte in der Folge ein Modell, dass staatliche Handlungen endogenisierte, indem er individuell-nutzenmaximierende Politiker als Umsetzer der Staatseingriffe berücksichtigte. \textcite[S. 137]{Downs1957} stellt dazu die Hypothese auf, dass Politiker ihre Tätigkeit einzig und alleine darauf ausrichten, maximal viele Stimmen zu erhalten. Politikern geht es weder um Parteiprogramme, noch um Interessenvertretungen, sie bedienen sich derselben nur um ihr einziges Ziel die (Wieder-)Wahl zu erreichen. Die Durchführung politischer Handlungen ist dabei nur ein Nebenprodukt ihrer privaten Nutzenmaximierung: Hohes Einkommen, Prestige und Macht. Das hört sich zunächst recht negativ an. Doch es ist in Wirklichkeit nichts anderes als politisches Verhalten als ökonomisch-marktwirtschaftliches Verhalten zu interpretieren. Ein Politiker handelt eben nicht anders wie alle anderen Menschen auch. So schürft ein Arbeiter in einer Kohle-Mine ja auch nicht deswegen nach Kohle, um die Gesellschaft mit Energie zu versorgen, sondern um damit Geld zu verdienen \parencite[S. 137]{Downs1957}. Das entsprechende Vorgehen von Politiker ist daher \textit{nicht grundsätzlich} verwerflich, ganz im Gegenteil es widerspricht nicht dem gesamtwirtschaftlichen Ziel die "`Soziale Wohlfahrt"' zu maximieren. In einer Welt mit perfekter Information werden die Wähler jene Partei wählen, die ihnen den höchsten persönlichen Nutzen in der zukünftigen Wahlperiode liefert. Dieser erwartete Nutzen wird aus den bisherigen Handlungen der Politiker - sowohl jener, die regieren, als auch jener, die in Opposition sind - abgeleitet. Aufgrund der perfekten Information der Wähler scheint unredliches (verwerfliches) Verhalten der Politiker nicht sinnvoll, weil dadurch Wählerstimmen verloren gehen. Erst durch die realistischere Annahme unvollständiger Information beider Seiten - die Wähler wissen nicht vollständig, welche Handlungen welche politische Partei liefern wird, aber auch die politischen Parteien wissen nicht vollständig, welche Handlungen die Wähler sehen wollen - bringt kostenintensiver Wahlkampf für Politiker zusätzliche Stimmen. Außerdem bringt Lobbyismus bei unvollständiger Information bestimmten Wählergruppen Einfluss auf die Handlungen von Bewerbern um politische Positionen. Damit verbunden ist auch die Möglichkeit, dass Bestechung und anderes unredliches Verhalten zu rationalem Verhalten wird. Unvollständige Information bei politischen Entscheidungen ist also \textit{der} Hauptfaktor im Rahmen von politischen Entscheidungsprozessen. Daraus leitet \textcite[S. 141]{Downs1957} ab, dass politische Parteien bestimmte Ideologien verfolgen. Diese können als eine Art "`Signaling"' (vgl. Kapitel \ref{Info}) verstanden werden: Dem Wähler wird rasch gezeigt, welche Motive die Partei verfolgt, ohne dass sich der Wähler kostenintensiv diese Information besorgen muss.

In weiterer Folge greift \textcite[S. 142]{Downs1957} das Medianwähler-Modell auf, das im vorherigen Kapitel dargestellt wurde. Es ist sogar so, dass dieses Modell erst durch die Arbeiten von \textcite{Downs1957, Downs1957b} den großen Bekanntheitsgrad erfuhr, den es bis heute unter Ökonomen genießt. Interessanterweise wurde später vor allem die Ausführungen von \textcite{Black1948a, Black1948b} bekannt, während sich Downs selbst auf das ältere Werk von \textcite{Hotelling1929} bezog. Allerdings führt \textcite[S. 142]{Downs1957} zusätzliche Annahmen ein, die dazu führen, dass sich in einem Zwei-Parteien-System im Medianwähler-Modell die beiden Parteien nicht notwendigerweise im Zentrum aneinander annähern müssen um eben den Medianwähler und damit die Mehrheit von sich zu überzeugen. Dies ist dann der Fall, wenn die Verteilung der Wähler im Spektrum von "`links-radikal"' - also überzeugte Kommunisten -  bis "`rechts-radikal"' - also Befürworter eines Nachtwächter-Staates - nicht eingipfelig, wie bei etwa bei einer Normalverteilung verläuft, sondern zwei- oder mehrgipfelig. Zum Beispiel könnte es in einer gespaltenen, radikalisierten Gesellschaft die meisten Wähler jeweils an den Rändern geben. In solchen Situationen besinnen sich die Parteien ihrer Ideologie und werden nicht zu "`Parteien der Mitte"'. Der wahrscheinlichste Outcome ist aber, dass sich in einem Zwei-Parteien-System beide um den Medianwähler in der Mitte bemühen, während in einem Mehrparteien-System verschiedene Ideologien vorherrschen, weil es keinen Sinn macht sich als Partei ausschließlich an der Mitte zu orientieren, da ja die Gefahr besteht von anderen Parteien links oder rechts "`überholt"' zu werden. Dies war ein wichtiger, wenn auch letztlich bis heute umstrittener, Ansatz. Schließlich lässt sich die in der Folge in den 1970er Jahren entstandene Forschung einteilen in einen Zweig, der "`Opportunistisches Verhalten"' der Parteien unterstellte und damit eine Annäherung der großen Parteien an die Mitte wie im Medianwähler-Modell, und einem zweiten Zweig, der von "`Ideologischen Parteien"' ausgeht. An dieser Stelle sei noch ein Hinweis zu Zwei-Parteien und Mehr-Parteien-Systemen angebracht. Heute verbindet man erstere vor allem mit dem Angel-sächsischem Raum, wo es ja, aufgrund des Mehrheitswahlrechts, im wesentlichen tatsächlich nur zwei große Parteien gibt. In Kontinentaleuropa hingegen gibt es seit jeher Verhältniswahlrecht und dementsprechend auch wesentlich mehr Parteien. Ende der 1960er Jahre, als \textcite{Downs1957} publiziert wurde, gab es aber auch in Kontinentaleuropa in den meisten Staaten nur zwei bedeutende politische Großparteien. Oft als "`dritte Lager"' bezeichnete Parteien, sowie Öko-Parteien, kamen erst ab den 1980er Jahren zu bedeutenden Stimmumfängen.

\section{Makroökonomie und Politische Ökonomie}

In den 1950er und 1960er Jahren wa die Neoklassischen Synthese (vgl. Kapitel \ref{Synthese}) die alleinige und unangefochtene Mainstream-Ökonomie. Damit verbunden war ein nicht hinterfragter Glaube daran, dass "`der Staat"' die Konjunktur durch belebende Maßnahmen fördern sollte und auch bei Marktversagen eingreifen sollte \parencite[S. 522]{Snowdon2005}. Ab Ende der 1960er Jahre folgten erste theoretische Zweifel an der Neoklassischen Synthese (vgl. Kapitel \ref{Monetarismus} und \ref{micmac}). Ab Mitte der 1970er Jahre kam es schließlich auch auf dem Gebiet der Politischen Ökonomie zu bahnbrechenden Arbeiten, die untersuchten, wie die Politische Ökonomie mit der Makroökonomie interagiert. Konkret ging es darum, wie die Politik die beiden zentralen makroökonomischen Kennzahlen zu dieser Zeit, Arbeitslosigkeit und Inflation zu beeinflussen versuchten. Mitte der 1970er Jahre war erwartungsgestützte Phillipskurven-Zusammenhang nach \textcite{Phelps1968} und \textcite{Friedman1968} noch State of the Art.

William Nordhaus hob die Politische Ökonomie auf eine neue Ebene, indem er mit seiner Arbeit zum politischen Konjunkturzyklus \parencite{Nordhaus1975} eine formal-mathematische Analyse zum Zusammenhang zwischen Politik und Ökonomie vorlegte. Dazu griff er zunächst die Arbeit von \textcite{Downs1957, Downs1957b} auf und ging davon aus, dass es in einem Zwei-Parteien-System beide Kräfte zur Mitte tendieren und Politiker außerdem vorwiegend ihre eigene Wiederwahl anstreben und weitgehend ideologiefrei sind. Wähler wollen sowohl niedrige Arbeitslosigkeit, als auch niedrige Inflationsraten. Sie messen die Leistung der amtsführenden Politiker daran, in welchem Ausmaß diese beiden Ziele erreicht wurden. Dabei sind die Wähler aber recht kurzsichtig, vergessen also schnell welche was zu Beginn der Legislaturperiode war und berücksichtigen bei Wahlen die Kennzahlen der jüngsten Monate wesentlich stärker. Wie bereits erwähnt, ist die erwartungsgestützte Phillipskurve State of the Art. Das heißt kurzfristig gibt es einen negativen Zusammenhang zwischen Arbeitslosigkeit und Inflation und die Inflationserwartungen werden aus der vergangenen Inflation abgeleitet. Langfristig gibt es hingegen keinen Zusammenhang. Politiker können nun Geld- und Fiskalpolitik einsetzen, um Inflation und Arbeitslosigkeit zu steuern. Und das machen sie auch: Vor Wahlen verstärken Politiker fiskalpolitische Maßnahmen um das BIP zu steigern und die Arbeitslosigkeit zu senken. Die daraus resultierende, höhere Inflation tritt erst nach den Wahlen auf. Nach erfolgreicher Wiederwahl müssen Inflationserwartungen und die Inflation selbst gesenkt werden, dafür werden Sparmaßnahmen und höhere Arbeitslosenraten in Kauf genommen. So entsteht der von \textcite{Nordhaus1975} postulierte politische Konjunkturzyklus: Durch "`Wahlzuckerl"' vor den Urnengängen gibt es hohes BIP-Wachstum, niedrige Arbeitslosigkeit bei gleichzeitig (noch) niedriger Inflation, nach den Wahlen gibt es Rezessionen bei gleichzeitig hoher Inflation und steigenden Arbeitslosenzahlen.
\textcite{Nordhaus1975} zeigt aber weiters das langfristige Problem mit diesem Ansatz: Bei den Wahlen selbst sind Inflation und Arbeitslosigkeit entscheidend. Nach den Wahlen steigt die Inflation stets an. Langfristig ist die durchschnittliche Inflationsrate damit stets höher als im Optimum bei der Natürlichen Arbeitslosenquote. Dementsprechend ist auch die Inflationserwartung langfristig zu hoch mit dem Ergebnis, dass aus dem kurzfristig Wahl-optimierenden Verhalten von Entscheidungsträgern eine Politik resultiert, die langfristig eine zu hohe Inflation hervorbringt, gleichzeitig liegt die Arbeitslosenraten unter der natürlichen Arbeitslosenrate \parencite{Nordhaus1975}. 
Das Modell wird als auch als "`Opportunistisches Modell"' bezeichnet, weil Politiker darin keine Ideologien verfolgen, sondern opportunistisch handeln um ihren Stimmenanteil zu optimieren. Das Modell überzeugt durch seinen soliden formalen Aufbau. Empirisch ließ es sich nur teilweise erfolgreich anwenden. Als problematisch wurde von manchen allerdings die Annahme angesehen, dass Politiker keine ideologischen Ziele verfolgen. Dies erscheint doch recht unrealistisch, da auch in Zwei-Parteien-Systemen doch recht verfestigte Überzeugungen beobachtet werden können.

Als Gegenmodell entstand daher wenig später das "`Ideologische Modell"' (engl.: "`Partisan Model"') von \textcite{Hibbs1977}. HIER WEITER
\parencite[S. 532]{Snowdon2005}

Synthese aus beiden Modellen



Alle drei genannten Ansätze - also der Opportunistische, der Ideologische, sowie die Synthese daraus, wurden viel beachtet. Vor allem das hoch-formale Werk von \textcite{Nordhaus1975} war wegweisend in der Neuen Politischen Ökonomie. Allerdings war diesen Modellen nur kurzer Erfolg gegönnt. Denn die "`Lucas-Kritik"' und der Aufstieg der Theorie der Rationale Erwartungen spülte innerhalb der akademischen Welt nicht nur die Neoklassische Synthese und den Monetarismus innerhalb kurzer Zeit weg, sondern damit auch die Grundlagen der Modelle von \textcite{Nordhaus1975}, \textcite{Hibbs1977} und HIER ZITAT FREY1979 einfügen. Unter der Annahme rationaler Erwartungen gibt es für Politiker keine Möglichkeit die Wähler kurzfristig zu täuschen und damit Stimmen zu generieren. Erste Ende der 1980er Jahre schließlich fand die Politische Ökonomie Wege rationaler Erwartungen in ihre Modelle aufzunehmen HIER WEITER





Nach Theorie der rationalen Erwartungen:

Alberto Alessina
Unabhängigkeit der Zentralbanken zunächst von \textcite{Friedman1968} vorgeschlagen, dann bekannt durch \textcite{Kydland1977} bekannt geworden.

Romer-Buch: S 638, berühmtes Paper Alesina1988

Friedrich Schneider

Ideologie, oder Opportunismus? Folgen: Rules rather than discretion? Unabhängige Zentralbanken und Schuldenbremse.



\section{Buchanan: Die Public-Choice Theorie}
\label{Pol_Econ}

Eine neue \textit{makroökonomische} Theorie der politischen Ökonomie prägten in den 1970er Jahren James Buchanan und Gordon Tullock. Die beiden griffen vor allem die Tatsache auf, dass durch die Keynesianische Revolution aktiver Wirtschaftspolitik eine große Bedeutung auch in sonst marktwirtschaftlich organisierten Ökonomien zukommt. Dabei blieb stets die Frage unbehandelt, ob denn der Staat, beziehungsweise dessen Repräsentanten, überhaupt besser als der Markt ungewollten Entwicklungen, erstens, gegensteuern \textit{kann}, und zweitens, \textit{will}.

George Stigler: Regulierung.