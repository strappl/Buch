%%%%%%%%%%%%%%%%%%%%% chapter.tex %%%%%%%%%%%%%%%%%%%%%%%%%%%%%%%%%
%
% sample chapter
%
% Use this file as a template for your own input.
%
%%%%%%%%%%%%%%%%%%%%%%%% Springer-Verlag %%%%%%%%%%%%%%%%%%%%%%%%%%

\chapter{Supply-Side-Economics}
\label{Supply-Side-Economics}


\section{Laffer \& Mundell}
Großer politischer Einfluss. Credo: Steuern runter! (Als Gegenmaßnahme zur Stagflation, aber wenig Problem mit Staatsschulden)
Trotzdem gewisse Ähnlichkeit zu Österreichischer Schule.
Friedman war eigentlich Gegenspieler in Sachen Inflation. Aber weil gleiche Intention:



\section{Laffer \& Mundell: Supply-Side-Economics}

Die kurzzeitige politisch sehr einflussreiche "`Supply-Side-Economics"' wurde von Arthur Laffer und Mundell gegründet. Wir sind wieder einmal an einer Stelle angelangt, wo die Positionierung des Themas nicht wirklich richtig ist. Die "`Supply-Side-Economics"' war ein Teil der Anti-Keynesianischen Gegenrevolution. Sie war zwar ebenso konservativ wie der "`Monetarismus"' und die "`Neue Klassische Makroökonomie"', allerdings kann sie nicht als Teil einer der beiden Schulen gesehen werden, da sie inhaltlich - vor allem im Vergleich zum Monetarismus - und methodisch - vor allem im Vergleich zur "`Neuen Klassik"' anders ausgerichtet ist. Aber sie etablierte sich auch nicht als eigenständige ökonomische Schule.
Die "`Supply-Side-Economics"' wird zwar selten direkt mit der "`Österreichischen Schule"' in Verbindung gebracht, wenn man die Inhalte betrachtet, stechen die Parallelen aber sofort ins Auge. Die Orientierung am "`Say'schen Gesetz"' also an der "`Angebotsseite"' ist nicht nur - wie der Name schon sagt - bei den "`Supply-Side-Economics"' zentral, sondern auch bei der "`Österreichischen Schule"'. Ebenso die zentrale Forderung nach niedrigen Steuern.
Der erste bekannte Vertreter ist Arthur Laffer. Bekannt durch seine "`Laffer-Kurve"'. Damit konnte er den Politiker (Ronald Reagan???) davon überzeugen, dass Steuersenkungen Wirtschaftswachstum anregen können und in weiterer Folge die Gesamtsteuerleistung trotz niedrigerer Steuerbelastung höher ist, weil der Effekt des Wirtschaftswachstums den Effekt der Steuersenkung übertrifft. Die Schule auf diese Idee zu reduzieren, wäre aber eine unzulässige Vereinfachung.
Zweiter Robert Mundell: Er forderte während der Stagflation, dass (Dritter Weg neben Monetaristischer Geldmengensteuerung und keynesianischer Preisregulierung) die Steuern und die Staatsausgaben gesenkt werden sollten. (Kapitel 3??? in Stunde der Ökonomen)