%%%%%%%%%%%%%%%%%%%%% chapter.tex %%%%%%%%%%%%%%%%%%%%%%%%%%%%%%%%%
%
% sample chapter
%
% Use this file as a template for your own input.
%
%%%%%%%%%%%%%%%%%%%%%%%% Springer-Verlag %%%%%%%%%%%%%%%%%%%%%%%%%%

\chapter{Supply-Side-Economics}
\label{Supply-Side-Economics}

Der Begriff "`Supply-Side-Economics"', in deutsch meist mit "`Angebotsorientierte Wirtschaftspolitik"' übersetzt, wird häufig sehr allgemein für konservative Wirtschaftspolitik herangezogen. Im hier herangezogenen engeren Sinn, umfassen die "`Supply-Side-Economics"' ausschließlich die Arbeiten von Arthur Laffer und vor allem Robert Mundell in den 1960er- und 1970er-Jahren, die politisch großen Einfluss erlangten.
Die "`Supply-Side-Economics"' waren ein Teil der Anti-Keynesianischen Gegenrevolution. Sie war zwar ebenso konservativ wie der "`Monetarismus"' und die "`Neue Klassische Makroökonomie"', allerdings kann sie nicht als Teil einer der beiden Schulen gesehen werden, da sie inhaltlich - vor allem im Vergleich zum Monetarismus - und methodisch - vor allem im Vergleich zur "`Neuen Klassik"' -  anders ausgerichtet ist. Aber sie etablierte sich nicht als eigenständige wirtschafts\textit{wissenschaftliche} Schule. Berühmt geworden ist sie vielmehr durch ihre politische Umsetzung. Während sowohl der Keynesianismus, der Monetarismus, als auch die Neue Klassische Makroökonomie (und andere ökonomischen Schulen) auf grundlegenden theoretischen Werken aufgebaut wurden, fehlt die schriftliche wissenschaftliche Abhandlung der Supply-Side-Economics weitgehend. 

Als Hauptvertreter der "`Supply Side Economics"' kann zweifelsohne Robert Mundell angesehen werden. Eine in vielerlei Hinsicht "`interessante"' Persönlichkeit. Der in Kanada geborene Mundell war zeitlebens umtriebig, unorthodox und umstritten. Mit 24 Jahren promovierte er im Jahr 1956 am MIT um gleich danach an die University of Chicago als Post-Doktorand zu gehen. Schon 1957 zog es ihn weiter, nach Stanford und später nach Italien - dort kaufte er später ein Schlösschen - um 1961 bei Internationalen Währungsfonds (IWF) anzuheuern \parencite{Mundell1999}. In der Zeit schrieb er seine - aus wirtschafts\textit{wissenschaftlicher} Sicht bedeutendsten Werke. Heute zählen diese Werke längst zur Standard-Ökonomie und mit dem Feld der internationalen Makroökonomie begründete Mundell ein ganzes Forschungsgebiet. Anfang der 1960er Jahre war aber selbst diese Arbeit noch umstritten, weil das Bretton-Woods-System \footnote{Das Bretton-Woods-System wurde 1944 für die Zeit nach dem Zweiten Weltkrieg geschaffene und war ein internationales Währungssystem, dass stabile Wechselkurse sicherte. In dem System wurde der US-Doller an den Goldpreis gebunden. Eine Unze Gold kostete 35 US-Dollar. Staaten konnten sich dem Währungssystem anschließen und ihre Währung an den Dollar binden. Die US-Dollar konnten bei der Federal Reserve jederzeit gegen Gold eingetauscht werden} fixe Währungskurse sicherte und ein System mit international flexiblen Währungskursen zwischen den Industrieländern zu dieser Zeit noch undenkbar waren. In seinem Werk \textcite{Mundell1963} erweiterte er das keynesianische IS-LM-Modell um Gesichtspunkte des internationalen Handels und entwickelte damit das \textit{Mundell-Fleming-Modell}\footnote{Der Brite Marcus Fleming hatte unabhängig davon ein ähnliches Modell entwickelt, daher der Name.}. Das Modell führt in das IS-LM-Modell eine dritte Funktion, die ZZ-Kurve, ein. Diese bildet den internationalen Handel und damit das Außenhandelsdefizit (bzw. -überschuss) ab. Im Falle uneingeschränkter Kapitalmobilität und flexiblen Wechselkursen kann man mittels geldpolitischen Maßnahmen Wirtschaftspolitik betreiben, während Fiskalpolitik relativ unbedeutend wird, da zusätzliche Staatsausgaben zu einem großen Teil im Ausland verpuffen. Bei fixierten Wechselkursen hingegen muss man Geldpolitik betreiben um das Kursverhältnis im Gleichgewicht zu halten. Aktive Wirtschaftspolitik kann dann nur mittels Fiskalpolitik betrieben werden. Daraus leitete er außerdem das "`Impossible-Trinity-Problem"' ab: Bei freiem Kapitalverkehr kann ein Staat entweder unabhängige Geldpolitik betreiben, oder den Wechselkurs zu anderen Währungen fix halten. Beide Ziele zu erreichen ist unmöglich.
Bereits 1961 veröffentlichte er ein Werk zu "`optimalen Währungsräumen"' \parencite{Mundell1961}. Damit legte er die theoretische Grundlage für grenzübergreifende Währungsunionen. Er wird in diesem Zusammenhang immer wieder "`Vater des Euro"' genannt \parencite{Mundell2006}.
Sein drittes bemerkenswertes Paper in dieser Zeit wurde 1962 veröffentlicht und behandelte den "`Policy Mix"' aus Fiskal- und Geldpolitik \parencite{Mundell1962}. Darin kam schon seine unorthodoxe Sichtweise zum Ausdruck. In diesem Artikel plädierte er nämlich dafür, dass die Politik zur Stimulierung der Wirtschaft aktive Fiskalpolitik betreiben sollte, aber gleichzeitig zur Bekämpfung der Inflation die Zinsen erhöhen sollte \parencite[S. 121]{Appelbaum2019}. Er nahm darin zum Teil schon seine höchst umstrittenen angebotsorientierten Politikempfehlungen der 1970er Jahre voraus, wobei nicht klar ist, ob er mit de geforderten aktiven Fiskalpolitik Staatsausgaben, oder, wie später ausschließlich, Steuersenkungen meinte. Mundell wurde im Jahr 1962 übrigens 30 Jahre alt. 

Diese beeindruckende frühe Biographie Robert Mundell's ist im Grunde genommen ein Exkurs in diesem Kapitel. Der Forschungsbereich internationale Makroökonomie war zwar neu, aber doch Teil des keynesianischen Mainstreams. Im Jahr 1966 erhielt er eine Professur an der University of Chicago. Und passte seine Sichtweise scheinbar an das neue Umfeld an. "`Es war als ob er die klassische Ökonomie und die Sicht auf die lange Frist entdeckt hatte"', heißt es dazu in \textcite[S. 194]{Warsh}.
Während seiner Zeit in Chicago (er blieb nur bis 1971) entwickelte er sein höchst umstrittenes Konzept der "`Supply-Side-Economics"', ohne dabei aber auch nur eine wissenschaftliche Abhandlung darüber zu veröffentlichen \parencite[S. 192]{Warsh}. In den USA stieg die Inflationsrate in der zweiten Hälfte der 1960er Jahre auf die 5\%-Marke und in wissenschaftlichen Kreisen wurde erstmals darüber diskutiert, wie Inflation am besten entgegenzutreten war. Die damalige Standardantwort war jene der Keynesianer: Steuererhöhungen gegen die Inflation, dafür niedrige Zinsen und zusätzliche Staatsausgaben um Wirtschaftswachstum und Beschäftigung hoch zu halten. Die Monetaristen um Milton Friedman sahen in der Inflation ein rein monetäres Problem und wollten gegen Inflation schlicht das Geldmengenwachstum einschränken. Damit waren sie zwar noch nicht im Mainstream angekommen, aber die Idee war durch die Quantitätsgleichung theoretisch hinterlegt und fand immer mehr Zulauf. Und dann gab es eben Robert Mundell, der meinte gegen Inflation müsste man die Steuern senken und die Zinsen erhöhen. Die Idee Mundell's war Inflation damit zu Bekämpfen die Produktion durch niedrigere Steuern zu fördern. Mehr Produktion bringt mehr Waren und damit ein steigendes Angebot, das bei gleichbleibender Nachfrage zu sinkenden Preisen (oder zumindest zu geringerer Inflation) führt. Beide Lager, Keynesianer wie auch Monetaristen fanden dies geradezu lächerlich. Ende der 1960er-Jahre geriet das Bretton-Woods-System unter Druck. In den USA gab es zwar gute Wachstumszahlen, aber auch steigende Inflation und hohe Außenhandelsdefizite. In dieser Zeit hatte auch Robert Mundell einen schweren Stand. Als Professor in Chicago wurden seine Ideen von der Kollegenschaft nicht ernst genommen. Er versagte als Präsident der "`Hemispherical Economic Society"' und als Herausgeber des renommierten "`Journal of Political Economy"'\parencite[S. 195]{Warsh}. Die Wirtschaftspolitik unter den Präsidenten Johnson und Nixon, Ende der 1960er Jahre, hielt nichts von seinen Vorschlägen. Schließlich floh er 1971 in seine kanadische Heimat an die University of Waterloo in Kitchener, Ontario. "`At last, Waterloo meets its Napoleon"', waren angeblich die hämischen Kommentare \parencite[S. 195]{Warsh}.

Aber der Siegeszug der "`Supply-Side-Economics"' stand erst bevor. Und eine Reihe glücklicher Zufälle verhalf ihr auf die Sprünge. Im Frühjahr 1974 kehrte Mundell zurück in die USA, an die Columbia University. "`Im Mai oder Juni"', wie Mundell in einem Interview \parencite{Mundell1998}  sagte, gab es eine Konferenz des "`American Enterprise Institute"' organisiert von Arthur Laffer. Im Rahmen der Konferenz stellte Laffer Mundell den Journalisten Jude Wanniski vor. Die drei sollten gemeinsam ein höchst einflussreiches Gespann werden. Zunächst aber prognostizierte Mundell noch in beeindruckender Weise die kommende Rezession. 1973 schon war das Bretton-Woods-Systems endgültig zusammengebrochen. Auf der eben genannten Konferenz argumentierte Mundell, erneut dass nur Steuersenkungen die vor der Tür stehende Kombination aus Inflation und Rezession (Stagflation) vermeiden könne. "`Everyone was laughing at him"', soll Wanniski später über die Szenerie gesagt haben \parencite[S. 195]{Warsh}.

Aber nicht mehr lange! Im Herbst 1974 trat tatsächlich der Fall ein, dass US-Präsident Ford der steigenden Inflation mit Steuererhöhungen entgegentreten wollte. Wenig später fanden sich die USA in einer tiefen Rezession wieder \parencite[S. 195]{Warsh}. Damit war der Weg für die "`Supply-Side-Economics"' geebnet.

Bei der Umsetzung selbst spielte Mundell eine untergeordnete Rolle. Politisch einflussreich waren vielmehr seine Mitstreiter Arthur Laffer und Jude Wanniski. Wie deren Einfluss auf die US-Spitzenpolitik begann ist eine jener Geschichten um die sich zahllose Mythen ranken: Es ist dies die verwegene Geschichte der \textit{Laffer-Kurve}.
Arthur Laffer erhielt bereits 1968 mit 28 Jahren eine Professur an der Universität in Chicago \footnote{Selbst das ist umstritten, da er erst 1971 an der Stanford seine Promotion abschloss}, danach ging er nach Washington als Beamter einer Regierungsbehörde.
Die Entstehungsgeschichte der Laffer-Kurve lautet wie folgt: Wanniski und Laffer luden im November 1974 Donald Rumsfeld (Stabschef bei Präsident Ford) - die Quellen, ob er wirklich dabei waren gehen auseinander -  und Dick Cheney (Vizestabschef) zu einem Drink (oder Dinner) in ein Restaurant ein um die Inflationsbekämpfungsstrategie von Präsident Ford zu diskutieren. Dabei präsentierte Laffer einen Zusammenhang, der ebenso einfach wie offensichtlich grundsätzlich richtig ist: Auf der y-Achse bildete er die Steuerquote (Steuersatz) ab. Auf der x-Achse die Steuereinnahmen \footnote{Die Darstellung ist untypisch, heute werden die Achsen üblicherweise vertauscht.}. Wenn die Steuerquote 0\% beträgt sind die Einnahmen selbstverständlich auch 0. Wenn der Steuersatz aber 100\% beträgt, wird niemand arbeiten, weil er ja das gesamte Einkommen abgeben muss und die Einnahmen betragen ebenso 0. Dazwischen liegen positive Steuerquoten und positive Steuereinnahmen. Verbindet man alle Steuerquoten-Steuereinnahme-Beobachtungen so erhält man eine Kurve, die wie eine Flugzeugnase, oder ein auf nach hinten gefallenes U aussieht. Und jetzt kommt der Clou: Es gibt einen Steuersatz bei dem die Steuereinnahmen maximal sind. Dieser einzelne Punkt befindet sich ganz rechts in der gedachten Grafik, wäre also die Spitze der Flugzeugnase. Für alle anderen beobachteten Steuereinnahmen auf der x-Achse \textit{muss} es in weiterer Folge aber zwei dazugehörige Steuerquoten geben: Eine hohe und eine niedrige. Befindet sich die aktuelle Steuerquote in der oberen Hälfte der Kurve, so steigen die Steuereinnahmen, wenn die Steuerquote gesenkt wird.

"`Ich schnappte mir angeblich meine Serviette und einen Stift und skizzierte eine Kurve auf der Serviette, die den Zusammenhang zwischen Steuersätzen und Steuereinnahmen illustrierte"', so Laffer in einem Artikel \parencite[S. 1]{Laffer2004} und ergänzte: "`Ich persönlich erinnere mich nicht an die Details dieses Abend, aber Wanniskis Version könnte durchaus wahr sein."'
Der dargestellte Zusammenhang wurde später als "`Laffer Kurve"' bekannt. Angestoßen vor allem durch den Artikel von \textcite{Wanniski1978} wurde das Konzept unter Politikern und sogar der allgemeinen Bevölkerung bekannt. \textcite[S. 979ff]{Shiller2017} analysierte das Phänomen der Ausbreitung sogar wissenschaftlich und zeigte, dass die Verwendung des Artikels zwischen 1977 und 1981 in Zeitungen und Büchern geradezu explodierte. 
Damit ist die "`Laffer-Kurve"' ein Beispiel dafür, dass ein wissenschaftlich unbedeutender und recht einfacher Zusammenhang, zu einem unerklärliche Popularität und zum anderen enormen politischen Einfluss erlangen kann. Vor allem Politiker - und hier vor allem Ronald Reagan in den USA und Margaret Thatcher im Vereinigten Königreich - untermauerten ihre Steuersenkungspläne Anfang der 1980er  mit diesem Konzept, vermeintlich wissenschaftlich.

Die Popularität des Konzepts in der relativ kurzen Zeitspanne zwischen 1977 und 1981 \parencite[S. 980]{Shiller2017} ist auch deshalb bemerkenswert, weil Laffer selbst die Entdeckung desselben niemals für sich beanspruchte\footnote{Bedeutende wissenschaftliche Artikel veröffentlichte Laffer kaum zu dem Thema, außer vielleicht \textcite{Laffer1981}} und es schon mehrmals zuvor in der Geschichte behandelt wurde, zum Beispiel auch von Keynes \parencite[S. 2]{Laffer2004}. Der Haken an dem Konzept ist auch nicht besonders schwer zu finden: Intuitiv geht man davon aus, dass die Laffer-Kurve wie ein umgekehrtes U (oder nach alter, originaler Darstellung, wie ein seitlich betrachtetes U) aussieht. Damit wäre der Verlauf symmetrisch und die maximalen Steuereinnahmen würden bei einem Steuersatz von 50\% erzielt. In Wirklichkeit ist der Verlauf höchstwahrscheinlich nicht symmetrisch und vor allem empirisch schwer festzustellen und außerdem von Land zu Land verschieden und zeitlich nicht konstant. Wahrscheinlich ist aber, dass der Steuersatz, ab dem es tatsächlich zu einer Reduktion der Arbeitsleistung und folglich zu sinkenden Steuereinnahmen bei steigenden Steuerquoten kommt, recht hoch ist, wahrscheinlich bei ca. 70\%. Dann könnte man nur bei extrem hohen Steuersätzen durch Steuersenkungen höhere Steuereinnahmen generieren.

Der große politische Einfluss des Konzepts ist dennoch bis heute eindeutig sichtbar. Ende der 1970er Jahre kann Laffer durchaus als informeller Berater des Republikaners Ronald Reagen angesehen werden \parencite[S. 142]{Appelbaum2019}. Ende der 1970er Jahre war aber auch Milton Friedman zu einem inoffiziellen Berater von Ronald Reagen geworden \ref{Monetarismus}. Und der hielt als Monetarist aus wirtschaftswissenschaftlicher Sicht nichts von den Konzepten von Mundell und Laffer durch Steuersenkungen die Inflation einzudämmen. Allerdings war Friedman ein überzeugter Liberaler und das Konzept niedriger Steuern passte zu seiner Überzeugung der Staat solle sich aus wirtschaftlichen Handlungen weitgehend heraushalten, also keine Fiskalpolitik betreiben. Friedman war überzeugt, dass die Steuersenkungen zu geringeren Staatseinnahmen führen würden, er war also überzeugt die Anwendung des Konzept der Laffer-Kurve würde nicht funktionieren. Aber genau das war für Friedman kein Problem, weniger Steuereinnahmen bedeuteten nämlich, dass sich der Staat langfristig keine aktive Wirtschaftspolitik leisten konnte. Obwohl ihre Beweggründe diametral verschieden waren, traten die "`Monetaristen"' und die "`Supply-Siders"' gemeinsam für Steuersenkungen ein. Und ihr Mann in der Politik war eben Ronald Reagan. Dieser brachte seine Partei auf einen wirtschaftsliberalen Kurs und setzte sich 1980 in der US-Präsidentschaftswahl gegen Jimmy Carter durch.

In weiterer Folge wurden beide Konzepte, jene der "`Monetaristen"' und jene der "`Supply-Siders"' umgesetzt. Heute werden beide Konzepte deshalb häufig in einen Topf geworfen. Und beide werden dementsprechend - eigentlich fälschlicherweise - als "`Suppy-Side-Econmics"' bezeichnet. Aber die grundlegenden Ideen beider Schulen unterschieden sich nicht unwesentlich. Die "`Monetaristen"' um Milton Friedman wollten den staatlichen Einfluss weitgehend zurückdrängen und die Wirtschaft deregulieren. Die Inflation wollten sie ausschließlich mittels restriktiver Geldpolitik bekämpfen. Niedrige Steuern waren nur das Nebenprodukt eines schlanken und deregulierten Staates.
Die "`Supply-Siders"' wollten primär die Steuern senken um das Wirtschaftswachstum zu fördern. Insbesondere die Besteuerung hoher Einkommen und der Unternehmen sollte gesenkt werden. Dadurch wurden zwar ausschließlich die Spitzenverdiener besser gestellt, durch den sogenannten "`Trickle-Down-Effekt"' sollten die positiven Effekte aber auf alle Bevölkerungsschichten durchsickern. Die Inflation sollte als Nebenprodukt des höheren Wirtschaftswachstums und vor allem der gestiegenen Investitionen automatisch sinken. Geldpolitik und ein deregulierter Staat waren nicht die Hauptthemen der "`Supply-Siders"' Mundell und Laffer. 

Ronald Reagen setzte die genannten Ideen allesamt um. Seine Wirtschaftspolitik der \textit{Reaganomics} wird heute eher mit dem Monetaristen Friedman in Verbindung gebracht. Der alleinige Effekt der "`Supply-Side-Economics"' und der Laffer-Kurve im engeren Sinn dürfte wohl überschaubar sein. Ihre angestrebten Ziele, nämlich Steuersenkungen und hier vor allem Abflachung der Steuerprogression wurden in den Folgejahren aber dennoch nachhaltig umgesetzt. Der Spitzensteuersatz in den USA sank von 70\% auf unter 30\% Anfang der 1990er Jahre. In praktisch allen Industriestaaten wurde der Spitzensteuersatz in diesem Zeitraum ebenfalls deutlich gesenkt \parencite[S. 148]{Appelbaum2019}. 

In der langen Frist werden die Umsetzung der Ideen der "`Supply-Siders"' aber fast durchaus negativ bewertet. Zwar sank die Inflation in den USA von 13,5\% im Jahr 1980 auf unter 5\% ab den Jahren 1984. Allerdings wird dieser Erfolg alleine den Monetaristen zugeschrieben. Und entgegen der Vorhersage der "`Supply-Siders"' stiegen weder Steuereinnahmen noch das Wirtschaftswachstum  aufgrund der Steuersenkungen. Einzig das Budgetdefizit geriet unter den Regierungen Reagan und Bush senior in den USA außer Kontrolle. Damit verloren die "`Supply-Siders"' ihren Rückhalt in der Reagan-Regierung. Ein langfristiger Erolg blieb ihnen aber: Bis heute sind die Spitzensteuersätze weltweit deutlich unter dem Niveau der frühen 1980er Jahre. Ein wahrscheinlich daraus abgeleitetes Problem beschäftigt uns in der Gegenwart mehr denn je. Das Problem der steigenden Ungleichheit (Vergleiche: Kapitel \ref{Ungleichheit})

Wie kann man die "`Supply Side Economics"' zusammenfassen? Der Begriff wird heute fälschlicherweise oftmals sehr allgemein für wirtschaftsliberale Schulen verwendet und mit der Chicago School gleichgestellt. In der ökonomischen Diskussion spielte die Ideen der "`Supply-Siders nie wirklich eine bedeutende Rolle. Dazu fehlt auch die ökonomische Literatur. \textcite[S. 377]{Samuelson1998} zum Beispiel fragt in seinem berühmten Lehrbuch "`wie konnte [...] eine obskure Idee, die vom Berufsstand der Ökonomen verworfen wurde [...], einen derart durchschlagenden politischen Erfolg erzielen?"'. Die Antwort liegt in der Vermarktung ihrer Ideen und den ausgezeichneten politischen Verbindungen.
Der Hauptvertreter Robert Mundell wurde in \textcite{Warsh} als "`seltsames Genie"' bezeichnet. Seine frühen Werke zu Währungsräumen und der offenen Ökonomie waren bahnbrechend. Dafür erhielt er 1999 den Nobelpreis für Wirtschaftswissenschaften. Als Vertreter der "`Supply-Siders"' wurden seine Tätigkeit zunehmend obskur. In seiner Nobelpreisrede etwa, erzählte er seine ganz eigene Geschichte über seinen Einfluss auf die Erfolge der Reagen-Regierung, die schlussendlich ein Mitgrund dafür waren, dass die Sowjetunion die osteuropäischen Staaten aus ihrem Regime entließ \parencite[S. 337]{Mundell1999}. Er selbst war übrigens nie im engsten Beraterkreis der US-Ökonomie, stattdessen beriet er die Zentralbank Uruguays \parencite[S. 195]{Warsh}.
In der Ökonomie blieb die "`Supply-Side-Economics"' nicht mehr als eine Randerscheinung, die unter Ökonomen oft eher belächelt wurde. Ihr nachhaltiger politischer Einfluss war aber alles andere als lachhaft.













