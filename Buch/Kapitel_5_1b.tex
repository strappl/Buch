%%%%%%%%%%%%%%%%%%%%% chapter.tex %%%%%%%%%%%%%%%%%%%%%%%%%%%%%%%%%
%
% sample chapter
%
% Use this file as a template for your own input.
%
%%%%%%%%%%%%%%%%%%%%%%%% Springer-Verlag %%%%%%%%%%%%%%%%%%%%%%%%%%

\chapter{Supply-Side-Economics}
\label{Supply-Side-Economics}

Der Begriff "`Supply-Side-Economics"', in deutsch meist mit "`Angebotsorientierte Wirtschaftspolitik"' übersetzt, wird häufig sehr allgemein für konservative Wirtschaftspolitik herangezogen. Im hier herangezogenen engeren Sinn, umfassen die "`Supply-Side-Economics"' ausschließlich die Arbeiten von Arthur Laffer und vor allem Robert Mundell in den 1960er- und 1970er-Jahren, die politisch großen Einfluss erlangten.
Die "`Supply-Side-Economics"' waren ein Teil der Anti-Keynesianischen Gegenrevolution. Sie war zwar ebenso konservativ wie der "`Monetarismus"' und die "`Neue Klassische Makroökonomie"', allerdings kann sie nicht als Teil einer der beiden Schulen gesehen werden, da sie inhaltlich - vor allem im Vergleich zum Monetarismus - und methodisch - vor allem im Vergleich zur "`Neuen Klassik"' -  anders ausgerichtet ist. Aber sie etablierte sich nicht als eigenständige wirtschafts\textit{wissenschaftliche} Schule. Berühmt geworden ist sie vielmehr durch ihre politische Umsetzung. Während sowohl der Keynesianismus, der Monetarismus, als auch die Neue Klassische Makroökonomie (und andere ökonomischen Schulen) auf grundlegenden theoretischen Werken aufgebaut wurden, fehlt die schriftliche wissenschaftliche Abhandlung der Supply-Side-Economics weitgehend. 

Als Hauptvertreter der "`Supply Side Economics"' kann zweifelsohne Robert Mundell angesehen werden. Eine in vielerlei Hinsicht "`interessante"' Persönlichkeit. Der in Kanada geborene Mundell war zeitlebens umtriebig, unorthodox und umstritten. Mit 24 Jahren promovierte er im Jahr 1956 am MIT um gleich danach an die University of Chicago als Post-Doktorand zu gehen. Schon 1957 zog es ihn weiter, nach Stanford und später nach Italien - dort kaufte er später ein Schlösschen - um 1961 bei Internationalen Währungsfonds (IWF) anzuheuern \parencite{Mundell1999}. In der Zeit schrieb er seine - aus wirtschafts\textit{wissenschaftlicher} Sicht bedeutendsten Werke. Heute zählen diese Werke längst zur Standard-Ökonomie und mit dem Feld der internationalen Makroökonomie begründete Mundell ein ganzes Forschungsgebiet. Anfang der 1960er Jahre war aber selbst diese Arbeit noch umstritten, weil das Bretton-Woods-System \footnote{Das Bretton-Woods-System wurde 1944 für die Zeit nach dem Zweiten Weltkrieg geschaffene und war ein internationales Währungssystem, dass stabile Wechselkurse sicherte. In dem System wurde der US-Doller an den Goldpreis gebunden. Eine Unze Gold kostete 35 US-Dollar. Staaten konnten sich dem Währungssystem anschließen und ihre Währung an den Dollar binden. Die US-Dollar konnten bei der Federal Reserve jederzeit gegen Gold eingetauscht werden} fixe Währungskurse sicherte und ein System mit international flexiblen Währungskursen zwischen den Industrieländern zu dieser Zeit noch undenkbar waren. In seinem Werk \textcite{Mundell1963} erweiterte er das keynesianische IS-LM-Modell um Gesichtspunkte des internationalen Handels und entwickelte damit das \textit{Mundell-Fleming-Modell}\footnote{Der Brite Marcus Fleming hatte unabhängig davon ein ähnliches Modell entwickelt, daher der Name.}. Das Modell führt in das IS-LM-Modell eine dritte Funktion, die ZZ-Kurve, ein. Diese bildet den internationalen Handel und damit das Außenhandelsdefizit (bzw. -überschuss) ab. Im Falle uneingeschränkter Kapitalmobilität und flexiblen Wechselkursen kann man mittels geldpolitischen Maßnahmen Wirtschaftspolitik betreiben, während Fiskalpolitik relativ unbedeutend wird, da zusätzliche Staatsausgaben zu einem großen Teil im Ausland verpuffen. Bei fixierten Wechselkursen hingegen muss man Geldpolitik betreiben um das Kursverhältnis im Gleichgewicht zu halten. Aktive Wirtschaftspolitik kann dann nur mittels Fiskalpolitik betrieben werden. Daraus leitete er außerdem das "`Impossible-Trinity-Problem"' ab: Bei freiem Kapitalverkehr kann ein Staat entweder unabhängige Geldpolitik betreiben, oder den Wechselkurs zu anderen Währungen fix halten. Beide Ziele zu erreichen ist unmöglich.
Bereits 1961 veröffentlichte er ein Werk zu "`optimalen Währungsräumen"' \parencite{Mundell1961}. Damit legte er die theoretische Grundlage für grenzübergreifende Währungsunionen. Er wird in diesem Zusammenhang immer wieder "`Vater des Euro"' genannt \parencite{Mundell2006}.
Sein drittes bemerkenswertes Paper in dieser Zeit wurde 1962 veröffentlicht und behandelte den "`Policy Mix"' aus Fiskal- und Geldpolitik \parencite{Mundell1962}. Darin kam schon seine unorthodoxe Sichtweise zum Ausdruck. In diesem Artikel plädierte er nämlich dafür, dass die Politik zur Stimulierung der Wirtschaft aktive Fiskalpolitik betreiben sollte, aber gleichzeitig zur Bekämpfung der Inflation die Zinsen erhöhen sollte \parencite[S. 121]{Appelbaum2019}. Er nahm darin zum Teil schon seine höchst umstrittenen angebotsorientierten Politikempfehlungen der 1970er Jahre voraus, wobei nicht klar ist, ob er mit de geforderten aktiven Fiskalpolitik Staatsausgaben, oder, wie später ausschließlich, Steuersenkungen meinte. Mundell wurde im Jahr 1962 übrigens 30 Jahre alt. 

Diese beeindruckende frühe Biographie Robert Mundell's ist im Grunde genommen ein Exkurs in diesem Kapitel. Der Forschungsbereich internationale Makroökonomie war zwar neu, aber doch Teil des keynesianischen Mainstreams. Im Jahr 1966 erhielt er eine Professur an der University of Chicago. Und passte seine Sichtweise scheinbar an das neue Umfeld an. "`Es war als ob er die klassische Ökonomie und die Sicht auf die lange Frist entdeckt hatte"', heißt es dazu in \textcite[S. 194]{Warsh}.
Während seiner Zeit in Chicago (er blieb nur bis 1971) entwickelte er sein höchst umstrittenes Konzept der "`Supply-Side-Economics"', ohne dabei aber auch nur eine wissenschaftliche Abhandlung darüber zu veröffentlichen \parencite[S. 192]{Warsh}. In den USA stieg die Inflationsrate in der zweiten Hälfte der 1960er Jahre auf die 5\%-Marke und in wissenschaftlichen Kreisen wurde erstmals darüber diskutiert, wie Inflation am besten entgegenzutreten war. Die damalige Standardantwort war jene der Keynesianer: Steuererhöhungen gegen die Inflation, dafür niedrige Zinsen und zusätzliche Staatsausgaben um Wirtschaftswachstum und Beschäftigung hoch zu halten. Die Monetaristen um Milton Friedman sahen in der Inflation ein rein monetäres Problem und wollten gegen Inflation schlicht das Geldmengenwachstum einschränken. Damit waren sie zwar noch nicht im Mainstream angekommen, aber die Idee war durch die Quantitätsgleichung theoretisch hinterlegt und fand immer mehr Zulauf. Und dann gab es eben Robert Mundell, der meinte gegen Inflation müsste man die Steuern senken und die Zinsen erhöhen. Die Idee Mundell's war Inflation damit zu Bekämpfen die Produktion durch niedrigere Steuern zu fördern. Mehr Produktion bringt mehr Waren und damit ein steigendes Angebot, das bei gleichbleibender Nachfrage zu sinkenden Preisen (oder zumindest zu geringerer Inflation) führt. Beide Lager, Keynesianer wie auch Monetaristen fanden dies geradezu lächerlich. Ende der 1960er-Jahre geriet das Bretton-Woods-System unter Druck. In den USA gab es zwar gute Wachstumszahlen, aber auch steigende Inflation und starke Außenhandelsdefizite. In dieser Zeit hatte auch Robert Mundell einen schweren Stand. Als Professor in Chicago wurden seine Ideen von der Kollegenschaft nicht ernst genommen \parencite[S. 195]{Warsh}. Er versagte als Präsident der "`Hemispherical Economic Society"' und als Herausgeber des renommierten "`Journal of Political Economy"'. Die Wirtschaftspolitik unter den Präsidenten Johnson und Nixon, Ende der 1960er Jahre, hielt nichts von seinen Vorschlägen. Schließlich floh er 1971 in seine kanadische Heimat an die University of Waterloo in Kitchener, Ontario. "`At last, Waterloo meets its Napoleon"', waren angeblich die hämischen Kommentare \parencite[S. 195]{Warsh}.

Aber der Siegeszug der "`Supply-Side-Economics"' stand erst bevor. Und eine Reihe glücklicher Zufälle verhalf ihr auf die Sprünge. Im Frühjahr 1974 kehrte Mundell zurück in die USA, an die Columbia University. "`Im Mai oder Juni"', wie \textcite{Mundell1998} in einem Interview sagte, gab es eine Konferenz des "`American Enterprise Institute"' organisiert von Arthur Laffer, dem zweiten berühmt gewordenen Vertreter der "`Supply-Side-Economics"'. Dort stellte Laffer außerdem den Journalisten Jude Wanniski vor. Die drei sollten gemeinsam ein höchst einflussreiches Gespann werden. Zunächst aber prognostizierte Mundell noch in beeindruckender Weise die kommende Rezession. 1973 schon war das Bretton-Woods-Systems endgültig zusammengebrochen. Auf der eben genannten Konferenz argumentierte Mundell, dass nur Steuersenkungen die vor der Tür stehende Kombination aus Inflation und Rezession (Stagflation) vermeiden könne. Im Herbst 1974 trat tatsächlich der Fall ein, dass Präsident Ford der steigenden Inflation mit Steuererhöhungen entgegentreten wollte. Wenig später fanden sich die USA in einer tiefen Rezession wieder \parencite[S. 195]{Warsh}. Damit war der Weg für die "`Supply-Side-Economics"' geebnet.

Serviette Laffer-Kurve s 126
reagan s. 142
Friedman S. 133 Friedman war eigentlich Gegenspieler in Sachen Inflation. Aber weil gleiche Intention


Nobelpreis: Eigentlich für Internationale Makroökonomie. Selbstvertrauen blieb: Zusammenhang seine Theorie Zusammenhang zum Ende der Sowjetunion.





Der erste bekannte Vertreter ist Arthur Laffer. Bekannt durch seine "`Laffer-Kurve"'. Damit konnte er den Politiker (Ronald Reagan???) davon überzeugen, dass Steuersenkungen Wirtschaftswachstum anregen können und in weiterer Folge die Gesamtsteuerleistung trotz niedrigerer Steuerbelastung höher ist, weil der Effekt des Wirtschaftswachstums den Effekt der Steuersenkung übertrifft. Die Schule auf diese Idee zu reduzieren, wäre aber eine unzulässige Vereinfachung.
Zweiter Robert Mundell: Er forderte während der Stagflation, dass (Dritter Weg neben Monetaristischer Geldmengensteuerung und keynesianischer Preisregulierung) die Steuern und die Staatsausgaben gesenkt werden sollten. (Kapitel 3??? in Stunde der Ökonomen)










