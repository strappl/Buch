%%%%%%%%%%%%%%%%%%%%% chapter.tex %%%%%%%%%%%%%%%%%%%%%%%%%%%%%%%%%
%
% sample chapter
%
% Use this file as a template for your own input.
%
%%%%%%%%%%%%%%%%%%%%%%%% Springer-Verlag %%%%%%%%%%%%%%%%%%%%%%%%%%

\chapter{Supply-Side-Economics}
\label{Supply-Side-Economics}

Der Begriff "`Supply-Side-Economics"', in deutsch meist mit "`Angebotsorientierte Wirtschaftspolitik"' übersetzt, wird häufig sehr allgemein für konservative Wirtschaftspolitik herangezogen. Im hier herangezogenen engeren Sinn, umfassen die "`Supply-Side-Economics"' ausschließlich die Arbeiten von Arthur Laffer und vor allem Robert Mundell in den 1960er- und 1970er-Jahren, die politisch großen Einfluss erlangten.
Die "`Supply-Side-Economics"' waren ein Teil der Anti-Keynesianischen Gegenrevolution. Sie war zwar ebenso konservativ wie der "`Monetarismus"' und die "`Neue Klassische Makroökonomie"', allerdings kann sie nicht als Teil einer der beiden Schulen gesehen werden, da sie inhaltlich - vor allem im Vergleich zum Monetarismus - und methodisch - vor allem im Vergleich zur "`Neuen Klassik"' -  anders ausgerichtet ist. Aber sie etablierte sich nicht als eigenständige wirtschafts\textit{wissenschaftliche} Schule. Berühmt geworden ist sie vielmehr durch ihre politische Umsetzung. Während sowohl der Keynesianismus, der Monetarismus, als auch die Neue Klassische Makroökonomie (und andere ökonomischen Schulen) auf grundlegenden theoretischen Werken aufgebaut wurden, fehlt die schriftliche wissenschaftliche Abhandlung der Supply-Side-Economics weitgehend.
Ausgangspunkt für die Entwicklung der "`Supply-Side-Economics"' war die bereits häufig genannte Stagflation in den USA.
Während die damals vorherrschende Lehre, der Keynesianismus, Inflation für das geringere Problem im Gegensatz zur Alternative Arbeitslosigkeit. 







Der erste bekannte Vertreter ist Arthur Laffer. Bekannt durch seine "`Laffer-Kurve"'. Damit konnte er den Politiker (Ronald Reagan???) davon überzeugen, dass Steuersenkungen Wirtschaftswachstum anregen können und in weiterer Folge die Gesamtsteuerleistung trotz niedrigerer Steuerbelastung höher ist, weil der Effekt des Wirtschaftswachstums den Effekt der Steuersenkung übertrifft. Die Schule auf diese Idee zu reduzieren, wäre aber eine unzulässige Vereinfachung.
Zweiter Robert Mundell: Er forderte während der Stagflation, dass (Dritter Weg neben Monetaristischer Geldmengensteuerung und keynesianischer Preisregulierung) die Steuern und die Staatsausgaben gesenkt werden sollten. (Kapitel 3??? in Stunde der Ökonomen)


Großer politischer Einfluss. Credo: Steuern runter! (Als Gegenmaßnahme zur Stagflation, aber wenig Problem mit Staatsschulden)
Trotzdem gewisse Ähnlichkeit zu Österreichischer Schule.
Friedman war eigentlich Gegenspieler in Sachen Inflation. Aber weil gleiche Intention:







