%%%%%%%%%%%%%%%%%%%%%% pref.tex %%%%%%%%%%%%%%%%%%%%%%%%%%%%%%%%%%%%%
%
% sample preface
%
% Use this file as a template for your own input.
%
%%%%%%%%%%%%%%%%%%%%%%%% Springer-Verlag %%%%%%%%%%%%%%%%%%%%%%%%%%

\preface

%% Please write your preface here
Ich habe mich entschlossen dieses Buch zu schreiben, als ich das erste Mal bemerkte, dass ich Dinge vergesse. Dinge von denen ich wüsste, ich könnte diese früher einmal erklären, kann mich aber jetzt nicht mehr erinnern.

Früher gehörte ich zu den Personen, die die Einleitung ausließen und gleich mit dem ersten Kapitel in ein Buch starteten. Heute weiß ich: Das ist ein großer Fehler. Der Autor will in der Einleitung erklären was er mit dem Buch aussagen wollte. Warum manche Kapitel offensichtlich unterrepräsentiert sind, was er also nicht aussagen wollte. Und er möchte seine Motivation mitteilen: Warum schreibt man heute noch ein Buch?

Zu letzterem Punkt: Die Motivation liegt am ehesten darin Wissen für sich selbst festzuhalten. Sehr rational ist die Entscheidung nicht: Es ist kein wissenschaftliches Werk, dass eine wissenschaftliche Karriere fördern könnte. Es ist aber auch kein Buch für die breite Masse. Dafür ist der Kreis der Interessenten zu klein.

Es gibt sehr viele sehr gute Bücher zur Wirtschaftsgeschichte. Die meisten enden aber spätestens mit der neuen klassischen Makroökonomie, oder sogar mit dem Monetarismus. Dabei ist der Aufschwung und Niedergang (oder die Etablierung?) der neuen klassischen Makroökonomie 40 Jahre her. Tatsächlich wirkt es so als hätte die Ökonomie in den letzten Jahrzehnten kaum grundlegende neue Ideen hervorgebracht. Das ist überraschend: Schließlich hatten wir Anfang der 2000er Jahre eine enorme Bubble auf den Finanzmärkten zu verkraften und ein paar Jahr später - sogar noch schlimmer - die größte Wirtschaftskrise seit der Great Depression zu überstehen. Beide Schocks scheinen an den Mainstream-ökonomischen Ideen abzuprallen.


%% Please "sign" your preface
\vspace{1cm}
\begin{flushright}\noindent
Wien,\hfill {\it Stefan Trappl}\\
März 2020\hfill
\end{flushright}


