%%%%%%%%%%%%%%%%%%%%%% pref.tex %%%%%%%%%%%%%%%%%%%%%%%%%%%%%%%%%%%%%
%
% sample preface
%
% Use this file as a template for your own input.
%
%%%%%%%%%%%%%%%%%%%%%%%% Springer-Verlag %%%%%%%%%%%%%%%%%%%%%%%%%%

\preface

%% Please write your preface here

Früher gehörte ich zu jenen Personen, die beim Lesen eines Buches die Einleitung auslassen und gleich mit dem ersten Kapitel startet. Heute weiß ich: Das ist ein großer Fehler. Der Autor will in der Einleitung erklären was er mit dem Buch aussagen möchte. Warum manche Kapitel offensichtlich unterrepräsentiert sind, was er also nicht aussagen wollte. Und er möchte seine Motivation mitteilen: Warum schreibt man heute noch ein Buch?

Zu letzterem Punkt: Die Motivation liegt am ehesten darin Wissen für sich selbst festzuhalten. Sehr rational ist die Entscheidung nicht: Es ist kein wissenschaftliches Werk, dass eine wissenschaftliche Karriere fördern könnte. Es ist aber auch kein Buch für die breite Masse. Dafür ist der Kreis der Interessenten zu klein.

Eine wichtige Entscheidung war welche Inhalte das Buch umfassen sollte. Konkret ob nur Makroökonomie aufgenommen werden sollte, oder aber auch Mikroökonomie. Ich entschied mich schließlich dazu beide Gebiete der Ökonomie aufzunehmen. Erstens, weil viele mikroökonomische Entwicklungen auch auf die Makroökonomie Einfluss nahmen. So spielt zum Beispiel die mikroökonomische Spieltheorie eine große Rolle in den Methoden der modernen Makroökonomie ab der "`Neuen Klassik"'. Zweitens lassen sich viele Wirtschaftswissenschaftler nicht in das Konzept der Mikro- und Makroökonomie einordnen. Man denke zum Beispiel an die vielfältigen Arbeiten von \textsc{James Tobin}, oder \textsc{Kenneth Arrow}. Drittens, wird als Geburtsstunde der Makroökonomie oftmals die Veröffentlichung von \textsc{John Maynard Keynes'} "`General Theory"' herangezogen. Wie wären dann aber die Arbeiten der Klassiker wie \textsc{David Ricardo} oder \textsc{Francois Quesnay} einzuordnen?
Daher wurde auf eine Einschränkung auf Makroökonomie verzichtet. Mit dem Ergebnis, dass manche - eindeutig mikroökonomischen - Kapitel schwer \textit{"'richtig"'} in das Inhaltsverzeichnis einzuordnen sind. Eindeutig mikroökonomische Themen mussten schließlich in das insgesamt eher zeitlich- und makroökonomisch-orientierte Inhaltsverzeichnis eingeordnet werden. Dies bringt - zugegeben - manchmal einen Bruch in der Gliederung mit sich.

Die Gliederung des Buches war für sich eine kritische Entscheidung. Auch hier gäbe es mehrere sinnvolle Möglichkeiten. Jeweils die wirtschaftswissenschaftlichen Erkenntnisse in zeitlicher Abfolge aneinandergereiht - mit dem Nachteil, dass keine thematische Gruppierung erfolgt. Oder aber streng thematisch geordnet - mit dem Nachteil, dass die zeitlich-historische Abfolge verloren geht. Daher wurde auch hier ein Mittelweg eingeschlagen: Die "`Überkapitel"' - hier Teile genannt - stellen von Teil I-IV die vier "`großen Brüche"' im ökonomischen Denken dar. In diesen Teilen werden jene wirtschaftswissenschaftliche Schulen dargestellt, die man heute als die \textsc{Mainstream}-Ökonomie bezeichnen könnte:

Zunächst die Geburt der Wirtschaftswissenschaften mit der Entstehung der \textsc{Klassik} durch die Veröffentlichung von \textsc{Adam Smith'} Klassiker: "`Wealth of Nations"' im Jahre 1776. Danach die "`marginalistische Revolution"', die an drei Orten - Wien, London und Lausanne - um 1870 praktisch zeitgleich stattfand und der daraus entspringenden \textsc{Neo-Klassik}. Im 20. Jahrhundert sorgte \textsc{John Maynard Keynes} im Jahre 1936 mit der Veröffentlichung seiner "`General Theory"' für einen Paukenschlag und der Etablierung des \textsc{Keynesianismus}. Und schließlich die Konterrevolution durch \textsc{Neue Klassischen Makroökonomie} um 1975.

Der letztere große Block ist in zweierlei Hinsicht erklärungsbedürftig. Erstens könnte man argumentieren, dass die große Gegenrevolution zum Keynesianismus bereits durch den "`Monetarismus"' von \textsc{Milton Friedman} eingeläutet wurde. Wir werden aber später in Kapitel \ref{Neue Makro} sehen, dass aus \textit{wissenschaftlicher} Sicht die Ideen der "`Neuen Klassiker"' viel eher revolutionär gegenüber dem Keynesianismus (ja, sogar auch gegenüber dem Monetarismus) waren, als jene der Monetaristen. Zweitens könnte man argumentieren, dass sich die Ideen der "`Neuen Klassiker"' in vielen Punkten recht rasch als nicht haltbar erwiesen und wieder verworfen wurden. Das stimmt zu einem gewissen Teil. Die \textsc{Neue Klassischen Makroökonomie} in ihrer Reinform wurde bald wieder abgelöst. Aber es waren erst ihre Ideen -  vor allem die Theorie der Rationalen Erwartungen -, die auch überzeugte Keynesianer dazu brachte, Ökonomie gänzlich neu zu denken. So blieben die damaligen Keynesianer zwar überzeugte Gegner der "`Neuen Klassiker"', ihre Ideen übernahmen und erweiterten sie aber ohne Zweifel.

Teil V schließlich ist nicht mehr zeitlich konsistent, sondern stellt in Einzelkapiteln jeweils große \textsc{Heterodoxe} Schulen dar. Ganz konsistent ist diese Einteilung nicht. So könnte man auch die Kapitel \ref{Marx} den \textsc{Marxismus} bzw. Kapitel \ref{Historisch},  \textsc{Historische Schule} als heterodoxe Schulen ansehen und in Teil V verfrachten. Umgekehrt wäre die frühe \textsc{Österreichische Schule} in gewissem Ausmaß durchaus dem Mainstream zuzuordnen. Insgesamt aber denke ich, dass in der Abwägung zwischen formal richtiger Einteilung und "`fließendem Großen und Ganzen"', die Gliederung wie vorgenommen ganz gut gelungen ist.

Es gibt unzählige wirtschaftsgeschichtliche Bücher, die sich mit Smith, Marx den Neoklassiker, Keynes und Friedman beschäftigen. Aber kaum welche aber der Konterrevolution durch die Entstehung der \textsc{Neuen Klassischen Makroökonomie} um 1975. Insgesamt wurde daher Wert darauf gelegt, den "`späteren"' Kapiteln einen höheren Umfang einzuräumen als den "`früheren"'. Dies bringt allerdings den Nebeneffekt mit sich, dass verschiedenen Unterkapiteln nicht unbedingt die gleich historische Bedeutung zukommt.

Eine weitere wichtige Einschränkung ist noch zu erwähnen. Auf die Darstellung der \textsc{Ökonometrie}, als großen und teilweise eigenständigen Teil der Ökonomie, wurde verzichtet. Nicht weil ich diese nicht für wichtig erachte, sondern im Gegenteil, weil die Entwicklungen in der Ökonometrie so enorm sind, dass sie als zusätzlichen Teil den Umfang dieses Buches gesprengt hätte.

Warum \textit{noch} ein Buch zur Wirtschaftsgeschichte? Es gibt sehr viele, sehr gute Bücher zur Wirtschaftsgeschichte. Die meisten enden aber spätestens mit der neuen klassischen Makroökonomie, oder sogar schon mit dem Monetarismus. Dabei ist der Aufschwung und Niedergang (oder die Etablierung?) der neuen klassischen Makroökonomie bereits 40 Jahre her. Tatsächlich wirkt es so als hätte die Ökonomie in den letzten Jahrzehnten kaum grundlegende neue Ideen hervorgebracht. Das ist überraschend: Schließlich hatten wir Anfang der 2000er Jahre eine enorme Bubble auf den Finanzmärkten zu verkraften und ein paar Jahr später - sogar noch schlimmer - die größte Wirtschaftskrise seit der Great Depression zu überstehen. Beide Schocks scheinen an den Mainstream-ökonomischen Ideen abzuprallen.


%% Please "sign" your preface
\vspace{1cm}
\begin{flushright}\noindent
Wien,\hfill {\it Stefan Trappl}\\
März 2020\hfill
\end{flushright}


