%%%%%%%%%%%%%%%%%%%%% chapter.tex %%%%%%%%%%%%%%%%%%%%%%%%%%%%%%%%%
%
% sample chapter
%
% Use this file as a template for your own input.
%
%%%%%%%%%%%%%%%%%%%%%%%% Springer-Verlag %%%%%%%%%%%%%%%%%%%%%%%%%%

\chapter{Sisyphus-Ökonomie}

1:1 aus Blanchard

Anfang des neuen Jahrtausends aber schien sich eine Synthese herauszubilden.
Methodisch baute sie auf dem Ansatz der Real-Business-Cycle-Theorie auf mit ihrer
exakten Beschreibung des Optimierungsverhaltens von Haushalten und Unterneh-
men. Sie berücksichtigte die Bedeutung von Änderungen in der Rate des technischen
Fortschritts, die sowohl von der „RBC-Theorie“ wie von der Neuen Wachstumstheorie
betont wird. Aber sie integrierte auch wesentliche Elemente des neu-keynesianischen
Ansatzes – sie integrierte viele Friktionen wie Suchprozesse am Arbeitsmarkt, unvollständige
Information auf Kreditmärkten und die Rolle nominaler Rigiditäten für die
aggregierte Nachfrage. Es gab zwar keine Konvergenz zu einem einzigen einheitlichen
Modell oder einer einheitlichen Liste relevanter Friktionen, aber es herrschte Übereinstimmung
über den Forschungsrahmen und die analytische Vorgehensweise.
Ein gutes Beispiel dafür ist die Forschung von Michael Woodford (Columbia Universität
New York) und Jordi Gali (Pompeu Fabra in Barcelona). Sie entwickelten gemeinsam
mit Koautoren das Neue Keynesianische Modell, das Nutzen- und Gewinnmaximierung
mit nominalen Rigiditäten kombiniert – der Kern dieses Modells wurde in
einfacher Form in Kapitel 17 vorgestellt. Dieser Modellansatz hat sich als höchst einflussreich
bei der Neugestaltung der Geldpolitik erwiesen – angefangen von Inflationssteuerung
bis zu Zinsregeln, die in Kapitel 25 behandelt wurden. Der Ansatz bildet die
Basis einer neuen Klasse großer Modelle, die auf der einfachen Struktur aufbauen, aber
eine große Zahl von weiteren Friktionen einbauen. Sie lassen sich nur mehr numerisch
lösen. Diese Modelle, DSGE-Modelle (dynamic general equilibrium analysis) genannt,
sind mittlerweile zum Standardinstrument der Zentralbanken geworden.


\section{Die Great Recession}

Doug
Diamond (Universität Chicago) und Philip Dybvig (Universität Washington) erforschten
schon in den 1980er-Jahren die Mechanismen von Bank Runs (vgl. Kapitel 4): Weil
Aktiva illiquide, Passiva aber liquide sind, unterliegen selbst solvente Banken dem
Risiko eines Runs. Dieses Problem kann nur vermieden werden, wenn die Zentralbank
in einem solchen Fall Liquidität bereitstellt. Bengt Holmström (MIT) und Jean Tirole
Toulouse) zeigten, dass Fragen der Liquidität in modernen Volkswirtschaften zentrale
Bedeutung zukommt. Nicht nur Banken, selbst Unternehmen können durchaus in die
Lage geraten, zwar solvent, aber trotzdem illiquide zu sein – also nicht in der Lage, sich
zusätzliche Mittel zu beschaffen, um an sich rentable Projekte fertigzustellen. Andrej
Shleifer (Harvard Universität) und Robert Vishny (Universität Chicago) wiesen in ihrer
Arbeit über die Grenzen der Arbitrage nach, dass als Folge asymmetrischer Information
Investoren Arbitragemöglichkeiten nicht ausnutzen können, wenn der Vermögenswert
unter den Fundamentalwert sinkt. Im Gegenteil, sie können sogar genötigt werden, auch
selbst solche Vermögenswerte zu verkaufen und so zu einem Preisverfall beizutragen.
Die Forschungsrichtung der Verhaltensökonomie (etwa von Richard Thaler, Chicago
Universität) hat aufgezeigt, in welcher Weise Individuen vom Modell des rationalen
Agenten abweichen und welche Implikationen sich daraus für Finanzmärkte ergeben.