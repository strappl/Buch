%%%%%%%%%%%%%%%%%%%%% chapter.tex %%%%%%%%%%%%%%%%%%%%%%%%%%%%%%%%%
%
% sample chapter
%
% Use this file as a template for your own input.
%
%%%%%%%%%%%%%%%%%%%%%%%% Springer-Verlag %%%%%%%%%%%%%%%%%%%%%%%%%%

\chapter{Kollektives statt individuelles Glück}
\label{Marx}

\section{Marx}
Der Name Karl Marx hat bis heute eine enorme Strahlkraft, vor allem in seinem Geburtsland Deutschland, wo er noch zur Jahrtausendwende unter die Top 3 der wichtigsten Deutschen Personen aller Zeiten gewählt wurde. 
Das kommunistische Manifest
Das Kapital I - III

\section{Die Revolution frisst ihre Eltern}
\label{cha: Moderner Sozialismus}
Normalerweise sagt man die Revolution frisst ihre Kinder, in diesem Fall hingegen passt der Ausspruch "`die Revolution frisst ihre Eltern"' aber wohl besser. Von der ökonomischen Theorie des Karl Marx und auch dessen wirtschaftspolitischen Ideen blieb im real existierenden Sozialismus wenig über. Schon die Sowjets unter Lenin, sowie die Oktoberrevolution in Russland 1917 entsprachen wohl kaum den Vorstellungen Marx' wenn er von kommunistischer Revolution träumte. Das Russland des frühen 20. Jahrhunderts war wohl der am wenigsten für eine kommunistische Revolution geeignete Staat in Europa. Russland war zu dieser Zeit überwiegend ein Bauernstaat, kein Industriestaat, in dem die Arbeiter unter dem ungezügelten Kapitalismus litten. Von den späteren stalinistischen Methoden in Russland, aber auch die Ausprägungen in Kuba und schließlich in Nordkorea ganz zu schweigen. Und dass sich China heute noch als kommunistischen Staat bezeichnet, würde Marx bestenfalls vielleicht als Scherz in unlustigen Zuständen bezeichnen.