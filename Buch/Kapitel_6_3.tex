%%%%%%%%%%%%%%%%%%%%% chapter.tex %%%%%%%%%%%%%%%%%%%%%%%%%%%%%%%%%
%
% sample chapter
%
% Use this file as a template for your own input.
%
%%%%%%%%%%%%%%%%%%%%%%%% Springer-Verlag %%%%%%%%%%%%%%%%%%%%%%%%%%

\chapter{Ungleichheit als wachsendes Problem}
\label{Ungleichheit}

Ricardo.
Produktionsfaktoren

funktionale vs. personelle Einkommensverteilung
Vermögensverteilung vs. Einkommensverteilung
Woher kommt das:
\textcite[S. 1620]{Aghion1999}
\textcite[S. 359]{Alesina1994a}
Die Anfänge: Bowley's Law
Kaldor und Lewis \textcite[S. 557]{Snowdon2005}


DAs in das Kapitel Ungleichheit:
Ungleichheit und Wachstum: Kaldor und Kuznets \parencite{Alesina1994a} und \parencite[S. 556]{Snowdon2005}: Grundsätzlich positiver Zusammenhang: Mehr Ungleichheit (in der Form der funktionalen Einkommensverteilung) führt zu mehr Kapitalakkumulation und daher zu mehr Wachstum. Ansatz der politischen Ökonomie: Zunächst \textcite{Hirschman1973}: Toleranz gegen Einkommensungleichheit. Ab gewissem Level fällt diese Toleranz. Schließlich: Politische Instabiliät.
Ansatz von Alesina et al.: Hohe Ungleichheit: Median-Wähler hat hohes Bedürfnis nach Umverteilung. Hohe Steuerlast reduziert Investitionen und drückt das Wachstum.

\textcite{Deininger1996}


%\section{Atkinson}

%\section{Piketty}
Great Recession


