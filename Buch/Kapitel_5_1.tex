%%%%%%%%%%%%%%%%%%%%% chapter.tex %%%%%%%%%%%%%%%%%%%%%%%%%%%%%%%%%
%
% sample chapter
%
% Use this file as a template for your own input.
%
%%%%%%%%%%%%%%%%%%%%%%%% Springer-Verlag %%%%%%%%%%%%%%%%%%%%%%%%%%

\chapter{Österreichische Schule}
\label{Austria}

Als Begründer der "`Österreichischen Schule der Nationalökonomie"' gilt gemeinhin Carl Menger. Als Mitbegründer der "`Neoklassik"' wurde er bereits in Kapitel \ref{Wiener Schule} behandelt. Aus dem Werk \textcite{Menger1871}: "`Die Grundsätze der Volkswirtschaftslehre"' wird häufig die Preis- und Werttheorie hervorgehoben. Tatsächlich spielte diese lange in der Österreichischen Schule eine große Rolle. So auch noch bei Hayek ("`Preise und Produktion"'). Früh, nämlich bereits durch den direkten Schüler Menger's, Eugen von Böhm-Bawerk wurde die Kapitalmarkttheorie Forschungsgegenstand der "`Österreicher"'. Diese wurde später - durch die sogenannten "`Zweite Generation"', vor allem durch deren Hauptvertreter Ludwig von Mises und dessen fundamentale Geldtheorie - weiterentwickelt. Bis hierher war die "`Österreichische Schule"' nicht weit weg von der damaligen "`Mainstream"'-Ökonomie. Von Anfang an waren die "`Österreicher"' vehemente Kritiker des um 1880 aufkommenden Sozialismus. Die Ideen von Marx waren zur damaligen Zeit die wichtigsten heterodoxen Ansichten - alle anderen Schulen könnte man heute als "`Mainstream"' bezeichnen. Die fast schon radikale Marktgläubigkeit ließ die Österreichische Schule nach 1945 aus dem Mainstream abgleiten. War diese bei Mises noch als Antwort auf den Sozialismus zu verstehen, ist sie bei Hayek die vehemente Ablehnung des Keynesianismus. Dieser wurde jedoch, wie wir wissen, zur führenden Schule nach 1945, daneben blieb nur wenig Platz für die Österreichische Schule, die sich in weiterer Folge auch von der "`Freiburger Schule"' (vgl. Kapitel \ref{Neoliberalismus}) abgrenzte, die in der deutschen Wirtschaftspolitik kurz großen Einfluss hatte. Auch von Milton Friedman und dessen Monetarismus grenzten sich die "`Österreicher"' stark ab. Diese Schule stellte ebenso den Markt in den Mittelpunkt. Hayek, der Hauptvertreter der dritten Generation, wirkte zwar im Zentrum der "`Monetaristischen Gegenrevolution ab 1970"', nämlich an der Universität in Chicago, aber er schaffte es nicht, im Kreis der Ökonomen um Friedman eine Rolle zu spielen. Wissenschaftlich verlor die Österreichische Schule damit nach dem Zweiten Weltkrieg rasch an Bedeutung. Neben der radikalen Marktgläubigkeit ist dies vor allem auf die fast gänzliche Ablehnung formal-mathematischer Methoden - die die "`Österreichische Schule"' durchgehend seit deren Begründung durch Menger vertrat - zurückzuführen. Spätestens seit der Etablierung der Theorie der rationalen Erwartungen in der "`Neuen Klassik"', gibt es kaum noch Mainstream-Publikationen, die auf die formal-mathematische Methode verzichten. Nach der "`Great Depression"' 2008 und auch mit dem Etablierung populistischer Politiker in  führenden Positionen, feierte die Österreichische Schule wieder ein Comeback. Allerdings eher aus wirtschafts-\textit{politischer} Sicht als aus wirtschafts-\textit{theoretischer}.

\section{Wieser: Der Schüler Mengers}
Wieser: Begriff des Grenznutzens. 

\section{Mises: Der Hauptvertreter}

\section{Hayek: Weg vom Mainstream, aber rein in die Politik}

\section{Rothbard, Kirzner \& die Tea Party}
Buchanan wird manchmal als Österreicher bezeichnet.

\section{Schumpeter: Der ganz eigene Weg}
Eigentlich hätte sich \textit{Joseph Alois Schumpeter} ein eigenes Kapitel verdient. Die Zurechnung zur Österreichischen Schule hinkt ebenfalls gewaltig und ist am ehesten durch seine Herkunft gerechtfertigt, wobei auch das nur bedingt richtig ist. Aber der Reihe nach.

Schumpeter wurde 1883 in Trest geboren, das heute im südlichen Tschechien liegt, geboren. Trest (deutsch Triesch) gehörte damals zum österreichischen Teil der Doppelmonarchie Österreich-Ungarn. Er machte in Österreich früh eine beeindruckende Karriere: Jüngster Professor der Monarchie, später Finanzminister und Bankvorstand

Und trotzdem wirkt seine Biographie eher traurig als erfolgreich: Mit vier Jahren verstarb sein Vater. Die Mutter wurde zur wichtigen Bezugsperson. Seine Frau -- vielleicht die einzige, die er wirklich liebte, denn er galt sonst als Frauenheld und lebte dies für die damalige Zeit in skandalöser Weise aus -- starb bei der Geburt seines Kindes. Das Kind wenige Stunden später. 
Als Finanzminister trat er nach wenigen Monaten zurück. Die Bank, die er als Vorstand leitete, ging Pleite und führte ihn selbst an den Rand des Bankrotts. Wissenschaftlich war er geprägt von der Konkurrenz zu Keynes. Und wenn man diese Konkurrenz als Wettbewerb sehen will, zog er ständig den Kürzeren: Ein begonnenes geldpolitisches Werk vollendete Schumpeter nicht mehr, nachdem Keynes seine "`Treatise on Money"' veröffentlichte. Sein monumentales Wert zur Zyklentheorie "`Business Cycles"' stellte er 1939 fertig. Eine Unzeit wenn man bedenkt, dass in diesem Jahr der Zweite Weltkrieg begann, Konjunkturzyklen also wenig Interesse hervorriefen. Nach 1945 wiederum interessierte sich ebenso fast niemand für das Werk. Durch die "`General Theory"' von Keynes schien es, als gehörten Konjunkturzyklen für alle Zeit der Vergangenheit an: Keynes's Deficit Spendings sollten doch gerade diese Konjunkturzyklen glätten.
Und doch gilt Schumpeter heute als einer der bedeutensten Ökonomen des 20. Jahrhundert. Ich habe zu Anfang des Kapitels erwähnt, dass man Schumpeter nur schwierig einer ökonomischen Schule zuordnen kann. Vielleicht ein Grund warum er sich in seiner Gegenwart nur schwer durchsetzte. Vielleicht aber auch ein Grund dafür warum er heute unverändert aktuell ist: Seine Theorie ist praktisch zeitlos. Insbesondere die auf ihn zurückzuführende Idee der "`schöpferischen Zerstörung"' wird immer wieder zitiert. Und, wesentliche bedeutender: Kaum jemand widerspricht dieser Idee der schöpferischen Zerstörung! Keynes hatte seine Blütezeit, aber er wurde sehr bald auch kritisiert und später von \textit{Robert Lucas} wissenschaftlich für tot erklärt. Auf Schumpeter berufen sich noch heute -- fast 100 Jahre nach dem Erscheinen seines Werkes "`The Theory of Development"' -- Ökonomen unterschiedlichster Richtungen (Acemoglu, Aghion und Paul Romer, früher Baumol und Minsky).



\section{Die Stockholmer Schule}
Wicksell: Konzept des natürlichen Zinssatzes. Praktisch Wiederentdeckung von Wicksell durch Taylor-Rule, bzw. in DSGE-Modellen (daher vereinzelt auch Neo-Wicksellianische Schule genannt)