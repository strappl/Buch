%%%%%%%%%%%%%%%%%%%%% chapter.tex %%%%%%%%%%%%%%%%%%%%%%%%%%%%%%%%%
%
% sample chapter
%
% Use this file as a template for your own input.
%
%%%%%%%%%%%%%%%%%%%%%%%% Springer-Verlag %%%%%%%%%%%%%%%%%%%%%%%%%%

\chapter{Österreichische Schule}
\label{Austria}

Als Begründer der "`Österreichischen Schule der Nationalökonomie"' gilt gemeinhin Carl Menger. Als Mitbegründer der "`Neoklassik"' wurde er bereits in Kapitel \ref{Wiener Schule} behandelt. Aus dem Werk \textcite{Menger1871}: "`Die Grundsätze der Volkswirtschaftslehre"' wird häufig die Preis- und Werttheorie hervorgehoben. Tatsächlich spielte diese lange in der Österreichischen Schule eine große Rolle. So auch noch bei Hayek ("`Preise und Produktion"'). Früh, nämlich bereits durch den direkten Schüler Menger's, Eugen von Böhm-Bawerk wurde die Kapitalmarkttheorie Forschungsgegenstand der "`Österreicher"'. Diese wurde später - durch die sogenannten "`Zweite Generation"', vor allem durch deren Hauptvertreter Ludwig von Mises und dessen fundamentale Geldtheorie - weiterentwickelt. Bis hierher war die "`Österreichische Schule"' nicht weit weg von der damaligen "`Mainstream"'-Ökonomie. Von Anfang an waren die "`Österreicher"' vehemente Kritiker des um 1880 aufkommenden Sozialismus. Die Ideen von Marx waren zur damaligen Zeit die wichtigsten heterodoxen Ansichten - alle anderen Schulen könnte man heute als "`Mainstream"' bezeichnen. Die fast schon radikale Marktgläubigkeit ließ die Österreichische Schule nach 1945 aus dem Mainstream abgleiten. War diese bei Mises noch als Antwort auf den Sozialismus zu verstehen, ist sie bei Hayek die vehemente Ablehnung des Keynesianismus. Dieser wurde jedoch, wie wir wissen, zur führenden Schule nach 1945, daneben blieb nur wenig Platz für die Österreichische Schule, die sich in weiterer Folge auch von der "`Freiburger Schule"' (vgl. Kapitel \ref{Neoliberalismus}) abgrenzte, die in der deutschen Wirtschaftspolitik kurz großen Einfluss hatte. Auch von Milton Friedman und dessen Monetarismus grenzten sich die "`Österreicher"' stark ab. Diese Schule stellte ebenso den Markt in den Mittelpunkt. Hayek, der Hauptvertreter der dritten Generation, wirkte zwar im Zentrum der "`Monetaristischen Gegenrevolution ab 1970"', nämlich an der Universität in Chicago, aber er schaffte es nicht, im Kreis der Ökonomen um Friedman eine Rolle zu spielen. Wissenschaftlich verlor die Österreichische Schule damit nach dem Zweiten Weltkrieg rasch an Bedeutung. Neben der radikalen Marktgläubigkeit ist dies vor allem auf die fast gänzliche Ablehnung formal-mathematischer Methoden - die die "`Österreichische Schule"' durchgehend seit deren Begründung durch Menger vertrat - zurückzuführen. Spätestens seit der Etablierung der Theorie der rationalen Erwartungen in der "`Neuen Klassik"', gibt es kaum noch Mainstream-Publikationen, die auf die formal-mathematische Methode verzichten. Nach der "`Great Depression"' 2008 und auch mit dem Etablierung populistischer Politiker in  führenden Positionen, feierte die Österreichische Schule wieder ein Comeback. Allerdings eher aus wirtschafts-\textit{politischer} Sicht als aus wirtschafts-\textit{theoretischer}.

\section{Wieser: Der Schüler Mengers}
Wieser: Begriff des Grenznutzens. 

\section{Mises: Der Hauptvertreter}

\section{Hayek: Weg vom Mainstream, aber rein in die Politik}

\section{Rothbard, Kirzner \& die Tea Party}
Buchanan wird manchmal als Österreicher bezeichnet.


\section{Die Stockholmer Schule}
\label{cha:Stockholm}
Wicksell: Konzept des natürlichen Zinssatzes. Praktisch Wiederentdeckung von Wicksell durch Taylor-Rule, bzw. in DSGE-Modellen (daher vereinzelt auch Neo-Wicksellianische Schule genannt)