%%%%%%%%%%%%%%%%%%%% book.tex %%%%%%%%%%%%%%%%%%%%%%%%%%%%%
%
% Ausgangsfile für das Buch
%
% Hier die Inputs einfügen
%
%%%%%%%%%%%%%%%% Springer-Verlag %%%%%%%%%%%%%%%%%%%%%%%%%%


% RECOMMENDED %%%%%%%%%%%%%%%%%%%%%%%%%%%%%%%%%%%%%%%%%%%%%%%%%%%
\documentclass[envcountsame,envcountchap]{svmono}

% choose options for [] as required from the list
% in the Reference Guide, Sect. 2.2

\usepackage{makeidx}         			% allows index generation
\usepackage{graphicx}        			% standard LaTeX graphics tool
                             			% when including figure files
\usepackage{multicol}        			% used for the two-column index
\usepackage[bottom]{footmisc}			% places footnotes at page bottom
\usepackage[ngerman]{babel}				% Neue deutsche Rechtschreibung
\usepackage[babel, german=guillemets]{csquotes}
\usepackage[backend = bibtex8, style = authoryear, citestyle=authoryear]{biblatex} 
\usepackage[T1]{fontenc}
\usepackage{xcolor}
\usepackage{tikz}
\usetikzlibrary{calc,matrix}

\addbibresource{Literatur.bib}  



% etc.
% see the list of further useful packages
% in the Reference Guide, Sects. 2.3, 3.1-3.3

\makeindex             % used for the subject index
                       % please use the style svind.ist with
                       % your makeindex program


%%%%%%%%%%%%%%%%%%%%%%%%%%%%%%%%%%%%%%%%%%%%%%%%%%%%%%%%%%%%%%%%%%%%%

\begin{document}

\author{Stefan Trappl}
\title{Mainstream Ökonomie}
\subtitle{Wie wurde die Volkswirtschaftslehre zu dem was sie heute ist? Ein historischer Abriss von 1870 bis heute}
\maketitle

\frontmatter%%%%%%%%%%%%%%%%%%%%%%%%%%%%%%%%%%%%%%%%%%%%%%%%%%%%%%

\include{dedic}
%%%%%%%%%%%%%%%%%%%%%% pref.tex %%%%%%%%%%%%%%%%%%%%%%%%%%%%%%%%%%%%%
%
% sample preface
%
% Use this file as a template for your own input.
%
%%%%%%%%%%%%%%%%%%%%%%%% Springer-Verlag %%%%%%%%%%%%%%%%%%%%%%%%%%

\preface

%% Please write your preface here
Ich habe mich entschlossen dieses Buch zu schreiben, als ich das erste Mal bemerkte, dass ich Dinge vergesse. Dinge von denen ich wüsste, ich könnte diese früher einmal erklären, kann mich aber jetzt nicht mehr erinnern.

Früher gehörte ich zu den Personen, die die Einleitung ausließen und gleich mit dem ersten Kapitel in ein Buch starteten. Heute weiß ich: Das ist ein großer Fehler. Der Autor will in der Einleitung erklären was er mit dem Buch aussagen wollte. Warum manche Kapitel offensichtlich unterrepräsentiert sind, was er also nicht aussagen wollte. Und er möchte seine Motivation mitteilen: Warum schreibt man heute noch ein Buch?

Zu letzterem Punkt: Die Motivation liegt am ehesten darin Wissen für sich selbst festzuhalten. Sehr rational ist die Entscheidung nicht: Es ist kein wissenschaftliches Werk, dass eine wissenschaftliche Karriere fördern könnte. Es ist aber auch kein Buch für die breite Masse. Dafür ist der Kreis der Interessenten zu klein.

Es gibt sehr viele sehr gute Bücher zur Wirtschaftsgeschichte. Die meisten enden aber spätestens mit der neuen klassischen Makroökonomie, oder sogar mit dem Monetarismus. Dabei ist der Aufschwung und Niedergang (oder die Etablierung?) der neuen klassischen Makroökonomie 40 Jahre her. Tatsächlich wirkt es so als hätte die Ökonomie in den letzten Jahrzehnten kaum grundlegende neue Ideen hervorgebracht. Das ist überraschend: Schließlich hatten wir Anfang der 2000er Jahre eine enorme Bubble auf den Finanzmärkten zu verkraften und ein paar Jahr später - sogar noch schlimmer - die größte Wirtschaftskrise seit der Great Depression zu überstehen. Beide Schocks scheinen an den Mainstream-ökonomischen Ideen abzuprallen.


%% Please "sign" your preface
\vspace{1cm}
\begin{flushright}\noindent
Wien,\hfill {\it Stefan Trappl}\\
März 2020\hfill
\end{flushright}




\tableofcontents


\mainmatter%%%%%%%%%%%%%%%%%%%%%%%%%%%%%%%%%%%%%%%%%%%%%%%%%%%%%%%
%%%%%%%%%%%%%%%%%%%%%%%%% part.tex %%%%%%%%%%%%%%%%%%%%%%%%%%%%%%%%%%
%
% sample part title
%
% Use this file as a template for your own input.
%
%%%%%%%%%%%%%%%%%%%%%%%% Springer-Verlag %%%%%%%%%%%%%%%%%%%%%%%%%%


\part{1776 -- 1870\\Klassik: Geburt (k)einer Wissenschaft}
		% Klassik (inkl. Marx)
%%%%%%%%%%%%%%%%%%%%%% chapter.tex %%%%%%%%%%%%%%%%%%%%%%%%%%%%%%%%%
%
% sample chapter
%
% Use this file as a template for your own input.
%
%%%%%%%%%%%%%%%%%%%%%%%% Springer-Verlag %%%%%%%%%%%%%%%%%%%%%%%%%%

\chapter{Jeder ist seines eigenen Glückes Smith}
\label{Smith}


\section{Vorarbeiten}
Aristoteles
Cantillon
Merkantilismus

\section{Adam Smith}

Hume:
Hume, David. 1742 [1987]. “Of Money,” Part II,
Essay III.7, in Essays, Moral, Political, and Literary,
edited by Eugene F. Miller. Liberty Fund. http://
www.econlib.org/library/LFBooks/Hume/
hmMPL26.html.


Quesnay


%%%%%%%%%%%%%%%%%%%%%% chapter.tex %%%%%%%%%%%%%%%%%%%%%%%%%%%%%%%%%
%
% sample chapter
%
% Use this file as a template for your own input.
%
%%%%%%%%%%%%%%%%%%%%%%%% Springer-Verlag %%%%%%%%%%%%%%%%%%%%%%%%%%

\chapter{Die Klassiker schlechthin}
\label{Klassik}

\section{Ricardo}

\section{Say und Malthus}

\section{Mill}
%%%%%%%%%%%%%%%%%%%%%% chapter.tex %%%%%%%%%%%%%%%%%%%%%%%%%%%%%%%%%
%
% sample chapter
%
% Use this file as a template for your own input.
%
%%%%%%%%%%%%%%%%%%%%%%%% Springer-Verlag %%%%%%%%%%%%%%%%%%%%%%%%%%

\chapter{Kollektives statt individuelles Glück}
\label{Marx}

\section{Marx}

\section{Die Revolution frisst ihre Eltern}


%%%%%%%%%%%%%%%%%%%%%%%% part.tex %%%%%%%%%%%%%%%%%%%%%%%%%%%%%%%%%%
%
% sample part title
%
% Use this file as a template for your own input.
%
%%%%%%%%%%%%%%%%%%%%%%%% Springer-Verlag %%%%%%%%%%%%%%%%%%%%%%%%%%


\part{1870 -- 1930\\Neoklassik: Revolutionen auf Ökonomisch}

Innerhalb der Wirtschaftswissenschaften kam es immer wieder zu schlagartigen Umbrüchen. Die "`Marxistischen Revolution"' (vgl. Kapitel \ref{Marx}) war - zumindest aus heutiger Sicht - kein nachhaltiger Erfolg beschieden. Ganz anders der sogenannten "`Marginalistischen Revolution"' in deren Folge sich die Ökonomie als Wissenschaft grundlegend änderte. Die in der Folge entstehende "`Neoklassik"' enthält viele Elemente, die bis heute den "`State of the Art"' der Wirtschaftswissenschaften bilden. Tatsächlich wird die heutige "`Mainstream"'-Ökonomie heute noch häufig einfach als Neoklassik bezeichnet. Aus wirtschaftshistorischer Sicht ist dies allerdings ungünstig, da damit zu viele moderne Elemente mitgemeint wären. Betreffend Mikroökonomie wurden die Grundlagen für das heute verwendete Instrumentarium aber tatsächlich durch die (späten) Neoklassiker entwickelt (vgl. Kapitel \ref{Neoklassik}). Die Neoklassik hat viele verschiedene Bezeichnung, die mehr oder weniger gleichwertig benutzt werden. So spricht man häufig vom "`Marginalismus"' oder der "`Grenznutzenschule"'.

Die Entstehung der Neoklassik ist damit verbunden, dass die Klassiker kein befriedigendes Konzept hatten um den Unterschied zwischen Preis und Wert eines Gutes zu bestimmen. Damit konnten sie nicht erklären warum bestimmte Güter für verschiedene Personen unterschiedlich wertvoll sind. Die Neoklassik fügte dazu das Konzept des Nutzens ein und damit den Homo Oeconomicus - den nutzenmaximierenden Menschen. Die Nutzentheorie stellt bis heute einen wesentlichen Teil der mikroökonomischen Haushaltstheorie dar und ist bis heute ein hoch umstrittenes und viel beforschtes Thema. Mit Hilfe des Nutzenkonzepts lässt sich erklären, warum der Wert einer Ware unterschiedlich hoch bewertet wird - je nach vorhandener Menge, bzw. auch von verschiedenen Personen. Der Wert einer Ware ist also nicht mehr objektiv bestimmbar, sondern subjektiv. Man spricht daher auch von subjektiver Wertlehre. Diese "`Subjektivität"' ist einer der wesentlichen Punkte in der Neoklassik. Allerdings gab es recht unterschiedliche Ansätze damit umzugehen. In Wien\footnote{Zur Bedeutung der genannten Orte folgt gleich mehr.} war man der Meinung, dass man so etwas subjektives wie den Nutzen schlicht nicht quantitativ bewerten kann und soll. Ganz anders in Cambridge, wo die höhere Mathematik Einzug in die Ökonomie fand. Wissenschaftlich gesehen ist die Entstehung der Neoklassik mit dem Einzug der höheren Mathematik - konkret der hundert Jahre zuvor entwickelten Infinitesimalrechnung -  in die Ökonomie gleichzusetzen. Der Wert eines Gutes wird in der Neoklassik als subjektiv angesehen. Eine zusätzliche Menge eines Gutes, bringt keinen konstanten Nutzenzuwachs. Die Ökonomie war damit als Spielwiese für die Mathematik entdeckt worden. Bis heute ist umstritten inwieweit sich wirtschaftliches Verhalten durch mathematische Modelle abbilden lässt. Es ist interessant, dass die Österreichische Schule der Nationalökonomie, deren Ursprung auf die marginalistische Revolution in Wien zurückgeht, bis heute der höheren Mathematik in der Ökonomie abgeneigt ist.

Die Geburtsstunde der Neoklassik wird um das Jahr 1870 angesiedelt, obwohl bedeutende "`Vorläufer"' schon deutlich früher sehr ähnliche Konzepte vorweggenommen haben (vgl. Kapitel \ref{Vorläufer}). Interessant ist der "`Geburtsort"': Tatsächlich wurden sehr ähnliche Grundkonzepte in drei verschiedenen Städten entwickelt. An allen drei Orten entwickelte sich in der Folge eine rege wirtschaftswissenschaftliche Forschungstätigkeit und daraus entstanden drei verschiedene Schulen, die bis heute existieren, wenn sie sich auch in ganz unterschiedliche Richtungen entwickelt haben. In Wien entwickelte Carl Menger sein nicht quantitatives Konzept des Grenznutzens und bildete damit die Grundlage der bis heute existierenden "`Österreichischen Schule der Nationalökonomie"' (vgl. Kapitel \ref{Austria}). In Cambridge begründete Stanley Jevons die Theorie der subjektiven Wertlehre und begründete damit die "`Cambridge School"', die mit ihren späten Vertretern Alfred Marshall, Irving Fisher und Arthur Pigou die "`Vollendung der Neoklassik"' darstellt (vgl. Kapitel \ref{Neoklassik}. Die Lausanner Schule wurde von Leon Walras begründet. Dieser ist heute noch im Begriff des "`Walras-Gleichgewicht"' allgegenwärtig. Tatsächlich stellte er sich als erster die Frage ob alle einzelnen Märkte gemeinsam im Gleichgewicht sein können oder müssen. Eine Frage, die im 20. Jahrhundert stark beforscht wurde (vgl. Kapitel \ref{Arrow-Debreu}).

Das politische und wirtschaftliche Umfeld der 1870er-Jahre war geprägt von den späten Jahren der Monarchien und der langsamen Entstehung des Nationalismus in Zentraleuropa. Damit verbunden die späte Zeit des Kolonialismus und damit die Zeit des Hochimperialismus. 

HIER WEITER


		% Neoklassik bis Irving Fisher (inkl. Methodenstreit)
%%%%%%%%%%%%%%%%%%%%% chapter.tex %%%%%%%%%%%%%%%%%%%%%%%%%%%%%%%%%
%
% sample chapter
%
% Use this file as a template for your own input.
%
%%%%%%%%%%%%%%%%%%%%%%%% Springer-Verlag %%%%%%%%%%%%%%%%%%%%%%%%%%

\chapter{Drei Orte, die gleiche Idee! - oder doch nicht so einfach?}
\label{Marginalismus}

Die Geburtsstunde der Neoklassik wird um das Jahr 1870 angesiedelt, obwohl bedeutende "`Vorläufer"' schon deutlich früher sehr ähnliche Konzepte vorweggenommen haben (vgl. Kapitel \ref{Vorläufer}). Interessant ist der "`Geburtsort"': Tatsächlich wurden sehr ähnliche Grundkonzepte in drei verschiedenen Städten entwickelt. An allen drei Orten entwickelte sich in der Folge eine rege wirtschaftswissenschaftliche Forschungstätigkeit und daraus entstanden drei verschiedene Schulen, die bis heute existieren, wenn sie sich auch in ganz unterschiedliche Richtungen entwickelt haben. In Wien entwickelte Carl Menger sein nicht-quantitatives Konzept des Grenznutzens und bildete damit die Grundlage der bis heute existierenden "`Österreichischen Schule der Nationalökonomie"' (vgl. Kapitel \ref{Austria}). In London begründete Stanley Jevons die Theorie der subjektiven Wertlehre und untermauerte dies als erster mit mathematischen Formeln. Die "`Lausanner Schule"' wurde von Leon Walras begründet. Dieser ist heute noch im Begriff des "`Walras-Gleichgewicht"' allgegenwärtig. Tatsächlich begründete er die Theorie des allgemeinen Gleichgewichts, die sich mit der Frage beschäftigt, ob alle einzelnen Märkte gemeinsam im Gleichgewicht sein können. Eine Frage, die im 20. Jahrhundert stark beforscht wurde (vgl. Kapitel \ref{Arrow-Debreu}) und bis heute umstritten ist.

Der Übergang von "`Klassik"' zu "`Neoklassik"' wird bis heute an verschiedenen Elementen festgemacht \parencite[S. 198]{Ekelund2002}.
\begin{enumerate}
	\item Ein entscheidendes Element ist sicherlich die Etablierung höherer Mathematik in der Ökonomie. Insbesondere die Implementierung der Differenzialrechnung ist für "`Grenzbetrachtungen"' und daraus entstehenden Maximierungsaufgaben notwendig.
	\item Der Marginalismus selbst, die Betrachtung von Grenzwerten, also den Werten des "`letzten"' Gutes, ist eine wesentliche Erweiterung gegenüber der Klassik, die diesbezüglich statisch nur von einheitlichen Gütern sprach. Damit verbunden ist die Optimierungsbetrachtung der "`Statischen Effizienz"': Der maximale Gewinn wird dann erzielt wenn das nächste produzierte Gut gleich hohe Kosten verursacht als es Ertrag einbringt. 
	\item Der Subjektivismus, also die Zuweisung eines individuellen Nutzens, kann ebenfalls als \textit{das} wesentliche Abgrenzungskriterium zur "`Klassik"' gesehen werden. In der Klassik sind die Güter objektiv bewertbar, meist aus dem Herstellungskosten heraus. Die beiden Bewertungszugänge unterscheiden sich fundamental.
\end{enumerate}

Natürlich lassen sich diese drei Element nicht streng voneinander trennen - im Gegenteil in gewissem Ausmaß bedingen sie einander eher. Gemeinsam bilden sie allerdings eine recht saubere Abgrenzung zwischen klassischer Ökonomie und der Neoklassik.


\section{Gossen: Der unbelohnte, hochmütige Vorläufer der Neoklassik}
\label{Vorläufer}

Wie bereits erwähnt wurde die Neoklassik an drei verschiedenen Orten unabhängig voneinander und fast gleichzeitig entwickelt. Man nennt deren Entstehung daher häufig "`Marginalistische \textit{Revolution}"'. In vielen Wissenschaften gibt es das Phänomen, dass neue Erkenntnisse von verschiedenen Forschern gleichzeitig entwickelt wurden - berühmt ist in diesem Zusammenhang vor allem die Entwicklung der Infinitesimalrechnung, die von Leibniz und Newton unabhängig entwickelt wurde. Häufig finden Historiker einen gemeinsamen "`Auslöser"' für solch parallele Fortschritte. Für die "`Marginalistische Revolution"' hingegen, finden sich keine solchen Auslöser. Dafür waren die gesellschaftlichen und wissenschaftlichen Rahmenbedingungen in Wien, Lausanne und London - die drei Orte wo die Neoklassik entwickelt wurde - zu unterschiedlich \parencite[S. 269]{Blaug1973}. Vielmehr kann man davon ausgehen, dass die oben beschriebenen, zentralen Ideen der "`Neoklassik"' immer wieder punktuell "`erfunden"', aber auch wieder vergessen wurden. \textcite[S. 274]{Blaug1973} zählt gleich neun Ökonomen auf, die ähnliche Ideen beschrieben \footnote{Nur die bekanntesten davon werden in weiterer Folge als "`Vorläufer"' beschrieben.}. Außerdem weiß man heute, dass die "`Marginalistische Revolution"' als solche von den Zeitgenossen nicht wahrgenommen wurde, sondern erst später - gegen Ende des 19. Jahrhunderts - als solche bezeichnet wurde. \textcite[S. 338]{Hicks1934}, \textcite[S. 516]{Jaffe1976} schließlich weisen auch darauf hin, dass die zentralen Fragestellungen der Arbeiten von \textcite{Menger1871}, \textcite{Jevons1871} und \textcite{Walras1874} nicht so nahe beieinander liegen, wie man vermuten würde, wenn man von "`derselben Entdeckung"' spricht. Mittlerweile weiß man also, dass es sich nicht um eine Revolution im Sinne einer sprunghaften Veränderung handelte, sondern dass einzelne Elemente der Neoklassik bereits seit Anfang des 19. Jahrhunderts entwickelt wurden. \textcite{Ekelund2002} führen dazu für Großbritannien, Frankreich, Deutschland, Italien und auch die USA zahlreiche Beispiele im Detail an. Die meisten dieser Beiträge brachten nur eingeschränkt und isoliert Elemente der späteren Neoklassik hervor. Drei Autoren stechen allerdings als "`Vorreiter der Neoklassik"' heraus. Ihre Beiträge, die um 1850 entstanden, nehmen die wesentlichen Bausteine der Neoklassik in umfangreichem Ausmaß vorweg \parencite[S. 205]{Ekelund2002}. 

Hermann Heinrich Gossen ist eine der tragischsten Figuren in der Geschichte der Ökonomie. Heute wissen wir, dass seine Erkenntnisse geradezu bahnbrechend waren. Zu seinen Lebzeiten blieb seine Arbeit hingegen vollkommen unbekannt und er erhielt keinerlei Wertschätzung. Laut \textcite{Kurz2009} gibt es von Gossen weder Foto noch sonstiges Bildnis\footnote{Man findet unter Google einige Bilder, die Gossen darstellen sollen. Inwieweit dies falsche Informationen sind, oder doch unbekannte Quellen mit seinem wahren Bildnis vorhanden sind, kann nicht verifiziert werden.}. Interessant ist auch, dass es - wenn auch posthum - dem Ökonomen \textcite{Walras1885} zu weitgehend verdanken ist, dass Gossen doch noch zu Ruhm gekommen ist. Dies ist insofern bemerkenswert, als Walras selbst ja als einer der Väter der Neoklassik gilt. Nachdem er von der Existenz von Gossens' Werk gehört hat, bezeichnete er dieses als "`allgemeiner und ausführlicher"' als sein eigenes \parencite[S. 1]{Kurz2009}. Die Geschichte, wie Gossen's Werk wiederentdeckt wurde, ist in \textcite{Ikeda2000} ausführlich beschrieben. Sie zeigt auch, wie hoch die späteren Neoklassiker die Leistung Gossen's einschätzten.
Zu seinen Lebzeiten wurde sein einziges Werk, \textcite{Gossen1854}: "`Entwicklung der Gesetze des menschlichen Verkehrs [...]"', praktisch nicht gelesen \parencite[S. 282]{Rosner2012}. Er hatte keine akademisch-ökonomische Ausbildung, war stattdessen diesbezüglich Autodidakt \parencite[S. 3]{Kurz2009} und dafür studierter Jurist. Für ihn gilt - ebenso wie in Kürze für Johann Heinrich von Thünen und die anderen Vorläufer dargestellt - dass er einer der ersten war, der die Differentialrechnung in seinen Modellen anwendete. Die höhere Mathematik war in ökonomischen Arbeiten zur damaligen Zeit noch nicht angekommen. Dies und die Tatsache, dass seine Arbeit \parencite{Gossen1854} recht unstrukturiert aufgebaut war \parencite[S. 20]{Kurz2009}, sowie der Umstand, dass Gossen bereits vier Jahr nach Erscheinen des Buchs mit 47 Jahren verstarb, führte dazu, dass er zu Lebzeiten - wie oben beschrieben - vollkommen unbekannt blieb. Nach eigenen Angaben im Buch, arbeitete Gossen 20 Jahre lang \parencite[S. 3]{Kurz2009}, also fast sein ganzes Erwachsenenleben an seinem Werk. Als er es 1953 fertigstellt, ist kein Verlag an dem Manuskript interessiert. Gossen lässt es auf eigene Kosten drucken, aber fast niemand kauft das Werk. Die akademische Welt nimmt praktisch keine Notiz davon. Gossen erkrankt kurz darauf schwer und stirbt schon 1858 \parencite[S.4]{Kurz2009}. Er war zu diesem Zeitpunkt schwer verbittert über seinen Misserfolg, aber dennoch stets überzeugt, dass er eine bahnbrechende Arbeit erstellt habe. So vergleicht er seine eigenen Leistungen mit den umwälzenden Arbeiten von Kopernikus \parencite{Kurz2009, Gossen1854}. Welch hartes Schicksal, dass erst Jahrzehnte später seine tatsächliche Fortschrittlichkeit erkannt wurde.

Heute gilt Gossen als der eigentliche Begründer des quantitativen Nutzenkonzepts in der Ökonomie. Er ging davon aus, dass Menschen nach dem "`maximalen Lebensgenuss"' streben \parencite[S. 284]{Rosner2012}. Diese Idee war auch damals nicht bahnbrechend, sondern wirkt selbstverständlich. Aber diese Idee in eine wirtschaftswissenschaftliche Analyse überzuführen und dies auch noch zu quantifizieren ist ein entscheidender Schritt. Ob und inwieweit dies überhaupt möglich ist, ist bis heute umstritten, wie in einigen der folgenden Kapiteln diskutiert werden wird. Der Utilitarismus, also das in den Vordergrund stellen des "`Nutzens"', war ursprünglich bereits um 1790 als philosophische Schule in England von \textcite{Bentham1789} begründet worden. Es ist aber unwahrscheinlich, dass Gossen von diesen Lehren etwas gehört hat, geschweige denn von diesen geprägt wurde. 

In der akademischen Lehre stößt man auf die Arbeit von Gossen meist in Form der Gossenschen Gesetze: Erstens, das Gesetz des abnehmenden Grenznutzens: Jedes zusätzliche Gut liefert einen geringeren Nutzen als das (identische) Gut davor. Für ökonomische Modelle hat dies heute wenig Bedeutung, da es ja solange zum Austausch verschiedener Güter kommt, bis der Gesamtnutzen maximiert ist. Aber die Idee des abnehmenden Grenznutzen an sich ist bis heute von Bedeutung. Bei Entscheidungen unter Risiko wird eine konkave Nutzenfunktion herangezogen. Diese Risikofunktion wird heute "`Von Neumann-Morgenstern-Nutzenfunktion"' genannt. Der abnehmende Grenznutzen bei Gossen ist hierbei identisch mit dem später entwickelten Konzept der Risikoaversion (vgl. \ref{Erwartungsnutzen}).
Zweitens: Jedes Individuum wird seine Güter solange gegen andere Güter tauschen, bis alle Güter im eigenen Bestand einen gleich hohen Nutzen liefern. Später wurde dieses Prinzip in ähnlicher Form in den "`Wohlfahrtstheoremen"' formalisiert (vgl. Kapitel \ref{Neoklassik_nach1945}). Gossen liefert dazu auch den Versuch eines mathematischen Beweises: Solange ein Individuum durch einen einzigen Gütertausch seinen Nutzen erhöhen kann, solange kann der individuelle Nutzen nicht maximal sein.
Drittens: Einen ökonomischen Wert haben nur Güter, die nicht uneingeschränkt verfügbar sind. Dieses dritte Gossensche Gesetz, wird übrigens häufig nur im englischsprachigen Raum so genannt (vgl. z.B.: \parencite{Blaug1973}). Im Deutschen wird hingegen oft von zwei Gossenschen Gesetzen gesprochen.
Der Fokus auf seine "`Gesetze"' schränkt die wahre Leistung Gossen's ein. Vor allem in Kombination mit seinen mathematischen Formulierungen kann das Werk als Vorläufer der modernen Konsumtheorie gesehen werden.

Dem zweiten großen deutschen Vorläufer der Neoklassik wurde zu Lebzeiten durchaus Ehre zuteil, wenn auch nicht für seine eigentlichen Leistungen. Die Rede ist von Johann Heinrich von Thünen. In seinem Werk "`Der isolirte Staat [...]"' \textcite{Thunen1826} erstellte er ein ökonomisches Modell am Beispiel der landwirtschaftlichen Produktion. In Abhängigkeit vom geografischen Abstand zur zentral gelegenen Stadt, sollten verschiedene Güter in verschiedenen Zonen produziert werden. Gemüse und Milch zum Beispiel näher als Getreide. Das Konzept wird heute noch als die "`Thünschen Kreise"' gelehrt. Er gilt vielen aufgrund dieser Aspekte seiner Arbeit als Begründer der Wirtschaftsgeographie \parencite[S. 283]{Kurz2009}. Seine größere Leistung liegt aber in der von ihm verwendeten Methodik. Er war wohl der erste, der so formal präzise die verschiedenen Produktionsfaktoren zusammenführte und als Grenzprodukte analysierte. Er nahm damit die moderne Produktionstheorie vorweg: Er ging, erstens, von sinkenden Grenzerträgen für jeden Produktionsfaktor aus. Zweitens, analysierte er als erster Grenzprodukte: Also was passiert mit dem Gesamtoutput wenn ein Produktionsfaktor verändert wird und alle anderen gleich bleiben \parencite[S. 282]{Rosner2012}. Nichts anderes wird heute noch als komparativ statische Analyse in der Mikroökonomie gelehrt. Bezeichnend für seinen Fortschritt ist auch, dass er zur Berechnung seiner Modelle bereits die Infinitesimalrechnung, konkret die Differentialrechnung, heranzog. Diese wurde bis dahin praktisch ausschließlich in den Naturwissenschaften angewendet \parencite[S. 202]{Ekelund2002}. Seine konkreten Modelle, die mehrere verschiedene Produkte umfassten, konnte er damit aber dennoch nicht lösen. Die dazu nötigen Methoden - Input-Output-Analyse bzw., der dynamische Optimierung - wurden erst im 20. Jahrhundert entwickelt. Seine Arbeit war damit seiner Zeit weit voraus, keiner der zeitgenössischen Ökonomen baute zu seinen Lebzeiten - er starb im Jahr 1850 - auf seinen Arbeiten auf. Dennoch wurde sein Werk zumindest gewürdigt, die Arbeit wurde laut \textcite[S. 283]{Rosner2012} gelesen und gelobt und ihm wurde auch der Ruhm einer Ehrendoktorwürde und einer Ehrenbürgerschaft zuteil.

Auch in Frankreich gab es einen interessanten Vorläufer zur Neoklassik: Augustin Antoine Cournot. Er war aber Vorläufer auf eine ganz andere Art als die eben genannten deutschen Thünen und Gossen. Erstens, war er anerkannter Professor an der Universität Lyon und zweitens, beschäftigte er sich nicht mit dem Nutzenbegriff, sondern mit Ertrags- und Kostenfunktionen \parencite[S. 287f]{Rosner2012}. Das für die Neoklassik so zentrale Element des Nutzens fehlte also. Was machte Cournot dennoch zu einem Vorläufer der Neoklassik und nicht einfach zu einem Klassiker? In seiner Arbeit \textcite{Cournot1838} nahm er die Analyse der Profitmaximierung vorweg. Mit Hilfe der Differentialrechnung analysierte er zunächst die Nachfragefunktion und bestimmte eine Funktion zur Erlösmaximierung. Danach wendete er dieses wissen bereits für die verschiedenen Marktformen Monopol, Duopol und den "`uneingeschränkten Wettbewerb"', also die heutige vollkommenen Konkurrenz an und ermittelte bereits die Bedingungen der Profitmaximierung \parencite[S. 289]{Rosner2012}. Diese gelten bis heute praktisch unverändert.  Nicht nur in der Volkswirtschaft, sondern auch in der betriebswirtschaftlichen Kostenrechnung spricht man vom "`Cournotschen Punkt"' als jene Preis-Mengen-Kombination bei der der Monopolist seinen Gewinn maximiert. Bereits Cournot definierte diesen Punkt als "`Grenzkosten gleich Grenzerlös"'. Es ist interessant, dass Cournot der erste war, der vollkommene Konkurrenz in der heute gültigen Definition als Marktform definierte. Nämlich als Markt auf dem es so viele Anbieter gibt, dass Produzenten den Preis nicht beeinflussen können \footnote{"'Vollkommene Konkurrenzmärkte"' werden oft mit "`Freien Märkten"' in der Definition von Adam Smith verwechselt. Smith meinte allerdings, dass es keine künstlichen Zugangsbeschränkungen und andere Marktbarrieren geben sollte. Die Anzahl der Marktteilnehmer nannte Smith bei seiner Bestimmung von "`Monopolen"' oder "`Wettbewerbsmärkten"' nicht \parencite{Blaug2001}.}. Besonders bemerkenswert ist auch seine Profitmaximierungslösung im Duopol-Fall, also wenn es am Markt nur zwei Anbieter gibt. Tatsächlich kommt Cournot bereits auf die Lösung, dass es ein stabiles Gleichgewicht gibt, das allerdings nicht Pareto-optimal sein muss. Das heißt, im Duopol-Fall können beide Anbieter jeweils kurzfristig einen höheren Gewinn erzielen, wenn sie ihre Produktion ausweiten. Da diese Möglichkeit wechselseitig besteht, würde durch diese Produktionsausweitung der Preis und in weiterer Folge der Gewinn für beide Duopol-Anbieter sinken. Cournot nahm mit seiner Lösung die bahnbrechenden Arbeiten von \textcite{Nash1950} zur Spieltheorie in diesem Punkt vorweg \parencite{Leonard1994}. Tatsächlich entspricht die Lösung, die Cournot gefunden hat, bereits einem Nash-Gleichgewicht (vgl. Kapitel \ref{Spieltheorie})!

\section{Menger: Der Vater des Subjektivismus}
\label{Wiener Schule}

Carl Menger, wird häufig als "`Vater des Subjektivismus' bezeichnet. Dieser Subjektivismus bezieht sich bei ihm auf die Güterpreise. Er revolutionierte also die Wert- und Preistheorie. Zu seiner Zeit war die Preistheorie der Klassiker vorherrschend. Diese erklärten Güterpreise aus den Produktionskosten\footnote{Ein Ansatz der übrigens bis heute in der Kostenrechnung teilweise suggeriert wird, indem man in vielen Lehrbüchern zunächst häufig die Herstellungskosten berechnet und danach einen Gewinnaufschlag hinzurechnet.}, was aber häufig alltäglichen, empirischen Beobachtungen widerspricht. 

Es wurde - konkret für die Arbeit von Menger - von \textcite{Streissler1990} gezeigt und auch von anderen Autoren beschrieben (\textcite{Blaug1973}, \textcite{Ekelund2002}), dass die allermeisten Ideen, die \textcite{Menger1871} umfasst, in Kontinentaleuropa schon deutlich vor 1870 bekannt waren. Vor allem in Deutschland gab es dahingehend einige, bis heute weitgehend unbekannt gebliebene Autoren, die Menger's Ideen ab dem frühen 19. Jahrhundert vorwegnahmen. \textcite[S. 159]{Blaug2001} führt diesbezüglich neben Thünen vor allem Karl Heinrich Rau an, der  Produktions-seitig schon Nutzen und Grenzbetrachtungen behandelte. Menger's große Leistung bestand laut \textcite[S. 295]{Rosner2012} aber vor allem darin, alle Bausteine zu einer einheitlichen Theorie zusammenzuführen. Denn nicht nur die Preistheorie der Klassiker hatte offensichtliche, empirische Probleme. Auch die eben erwähnten Ansätze der deutschen Autoren um Rau, konnten einige Fragen bezüglich Preise und Werte von Gütern nicht lösen. Die Preistheorie der Klassiker scheitert vor allem schon an den ständig schwankenden Preisen. Die Produktionskosten bleiben üblicherweise - zumindest bei Gütern, die keine Vorleistungen benötigen - kurzfristig konstant. Es wäre also eine Korrelation zwischen Güterpreisen und Produktionskosten notwendig um die klassische Theorie aufrechtzuerhalten, diese ist aber offensichtlich nicht durchgehend zu finden. Die Klassiker hatten Probleme bei der Unterscheidung zwischen Preis und Wert eines Gutes. Der Preis eines Gutes ergab sich demnach aus einem \textit{einheitlichen, objektiven} Wert dieses Gutes. Offensichtlich unterschiedliche Bewertung verschiedener Güter durch verschiedene Menschen lässt sich damit nicht in Einklang bringen. Damit verbunden ist das fehlende Verständnis von Nutzen, den ein Gut stiftet: Nahrungsmittel stillen Hunger, Schmuck stillt das Bedürfnis nach Geltung. Dennoch sind Nahrungsmittel günstiger zu erwerben als Schmuck - ein fundamentales Problem - genannt das "`Wertparadoxon"' - welches die Preistheorie der Klassiker nicht lösen kann, aber in der Lebensrealität eine Rolle spielt.

\textcite{Menger1871} - wenig überraschend, schließlich nennt man ihn nicht umsonst den "`Vater des Subjektivismus"' - stellt die subjektiven Werteinschätzungen der Menschen in den Vordergrund seiner Arbeit. Er beginnt damit festzustellen, dass - sinngemäß -  Güter als Dinge definiert werden, die einen Nutzen\footnote{Wörtlich eine \textit{Nützlichkeit}} stiften um menschliche Bedürfnisse befriedigen können \parencite[S. 1f]{Menger1871}. In Kapitel drei schließlich, beschreibt er, dass Güter einen Wert haben, der davon abhängt, wie stark das Bedürfnis nach einem Gut ist und die Verfügbarkeit des Gutes \parencite[S. 78]{Menger1871}. Güter werden also, individuell subjektiv (nach gestiftetem Nutzen) bewertet, womit Menger die Frage, wodurch Güter einen \textit{Wert} haben, behandelt, noch nicht aber die Frage nach dem \textit{Preis} dieser Güter. Außerdem beschreibt er, dass das Ausmaß der Bedürfnisbefriedigung von der verfügbaren Menge des Gutes abhängt. Der Nutzen eines Gutes ist also nicht konstant, mit mit zunehmender Verfügbarkeit des Gutes, nimmt die Nutzenstiftung einer zusätzlichen Einheit ab. Heute spricht man diesbezüglich vom abnehmenden Grenznutzen eines Gutes. Schon alleine damit lässt sich das oben beschriebene "`Wertparadoxon"' lösen. Außerdem wird fährt Menger fort, dass bei ausreichender Verfügbarkeit eines Gutes, das Bedürfnis danach gestillt werden kann - heute sprechen wir von "`Sättigung"'. 
Wesentlich für seine Preistheorie ist seine "`Lehre vom Tausche"'. Aufbauend auf der eben dargestellten Werttheorie ist diese elegant herzuleiten. Menschen wollen Bedürfnisse befriedigen und haben eine gewisse Menge an Gütern (heute würde man Budgetrestriktion sagen), es liegt also nahe, die eigenen Güter nach deren individuell-subjektiver Nutzenstiftung (=Wert) zu bewerten und anschließend Güter anderer Personen ebenso zu bewerten. Kommt zwischen zwei Individuen ein Tausch zustande, hatten beide Vertragspartner das Gut des jeweils anderen offensichtlich höher bewertet, als das eigene Gut \parencite[S. 156]{Menger1871}. Das klingt heute banal, war aber damals in der ökonomischen Theorie bahnbrechend. Damit wird nämlich unterstellt, dass der Wert eines Gutes alleine durch den Tauschvorgang steigt - für die Klassiker (inklusive Marx) war dies undenkbar (vgl. Kapitel \ref{Klassik}). Möglich ist dies nur durch die Anerkennung des "`Subjektivismus"': Der Wert eines Gutes ist individuell von Mensch zu Mensch unterschiedlich.
\textcite[S. 172]{Menger1871} selbst schreibt sinngemäß, dass mit dem Wissen über den Tausch, die Theorie zur Bildung von Marktpreisen eigentlich bereits mit-umfasst ist: "`Die Preise [...] sind doch nichts weniger als das Wesentliche der ökonomischen Erscheinung des Tausches"'. Marktpreise ergeben sich also aus dem Tauschverhältnis verschiedener Güter zueinander. Er schneidet in weiterer Folge auch noch die Auswirkung der verschiedenen Marktformen auf den Marktpreis an, bleibt aber hier - wahrscheinlich auch, weil er auf mathematische Beispiele fast gänzlich verzichtet\footnote{Carl Menger's Abneigung gegenüber Mathematik in der Ökonomie ist ex post einigermaßen amüsant, war sein Sohn Karl Menger doch später ein bedeutender Mathematiker, der unter anderem widerlegte, dass dem St. Petersburg Paradoxon, durch die Annahme konkaver Nutzenfunktionen (=Nutzenfunktion mit abnehmenden Grenznutzen) entgegengetreten werden kann.} - hinter den Analysen von \textcite{Cournot1838} inhaltlich zurück \parencite[S. 304]{Rosner2012}.

Liest man \textcite{Menger1871} findet man ohne Probleme an vielen Stellen die engen Verbindungen zur heutigen Mainstream-Mikroökonomie. Aber Menger beschrieb dies eben alles ohne Rückgriff auf mathematische Formeln und durchaus auch mit bildlichen Vergleichen. So vergleicht er Preisschwankungen mit Wellen, die entstehen, "`wenn man die Schleussen zwischen zwei ruhig stehenden Gewässern [...] wegräumt \parencite[S. 172]{Menger1871}. Dadurch sind die Aussagen nicht immer ganz leicht nachzuvollziehen. Vielleicht ist dies ein Grund dafür, warum die unmittelbare Wirkung des Werkes eingeschränkt blieb und seine Fortschrittlichkeit erst im Nachhinein festgestellt wurde \parencite[S. 304]{Rosner2012}. 

Insgesamt wird die Leistung von Carl Menger - vor allem hinsichtlich ihrer Neuartigkeit - heute sehr unterschiedlich eingeschätzt. Auf der einen Seite gilt \textcite{Menger1871} eben als eines von drei Werken, dass  zur "`Marginalistischen Revolution"' geführt hat. Auf der anderen Seite weiß man heute, dass praktisch alle dort festgehaltenen Ideen schon zuvor Bestand hatten und Menger vor allem die - selbstverständlich keinesfalls zu unterschätzende - Leistung erbracht hat, diese Ideen geeint darzustellen. Auch gilt er als Begründer der "`Österreichischen Schule"' der Nationalökonomie (vgl. Kapitel \ref{Austria}), deren wesentliche Vertreter allerdings seine Schüler, bzw. Schüler seiner Schüler waren. Interessant ist auch seine Rolle im berühmt gewordenen Methodenstreit mit den Vertretern der "`Historischen Schule"' (vgl. Kapitel \ref{Historisch}). 

\section{Jevons: Die Überwindung der Klassik}
\label{Jevons}

Im selben Jahr wie Menger, nämlich 1871, veröffentlichte Auch Stanley Jevons sein Hauptwerk: \textit{Theory of Political Economy}. Der Vergleich zwischen Menger und Jevons ist interessant, da es viele Gemeinsamkeiten, aber gleichzeitig deutliche Unterschiede in ihren Werken und deren Entstehungsgeschichte gibt. Im deutschsprachigen Raum fehlte um 1870 eine weitgehend akzeptierte Werttheorie. Während Menger die vereinzelten Arbeiten der Vorläufer im deutschsprachigen Raum (vgl. Kapitel \ref{Vorläufer}) zusammentrug, musste Jevons in England die gängige Werttheorie erst überwinden. Schließlich gab es in dort eine ausgeprägte Tradition klassischer Ökonomen. Die Werttheorie von Ricardo war anerkannter State of the Art \parencite[S. 320]{Rosner2012}. \textcite{Jevons1871} baute auf die, oben bereits erwähnte, philosophische Schule des "`Utilitarismus"' auf und verwendete auch dessen Sprache. Er versuchte wörtlich "`Ökonomie als Berechnung von Vergnügen und Schmerz (pleasure and pain)"' zu sehen \parencite[S. 321]{Rosner2012}. Darin versteckt sich auch schon der zweite wesentliche Unterschied zu Menger: Jevons war ein großer Verfechter der Anwendung höherer Mathematik in der Ökonomie. Er war wohl der erste, der die Differentialrechnung nachhaltig in der Ökonomie verankerte. Sein Argument war, dass wirtschaftliche Aktivität ja in Zahleneinheiten (z.B.: Mengen und Preise) ausgedrückt wird und daher Mathematik die logischste Darstellungsform sei \parencite[S. 71]{Jevons1871}. Ganz ähnlich wie Menger verwendet er das Konzept des Nutzens. Bei \textcite[S. 106]{Jevons1871} ist der Nutzen der subjektive Vorteil, den ein Gut bei einem Menschen auslöst. Er leitet daraus den abnehmenden Grenznutzen ab und stellt den Nutzen in der heute noch üblichen Form dar: $u(x)$. Also der Nutzen$u$ als eine Funktion des Gutes$x$ \parencite{Rosner2012}. Das Konzept des Grenznutzens - als wesentliches Konzept seiner Arbeit und der ganzen "`Marginalistischen Revolution"' - nannte er "`final degree of utility"' und ist bei ihm eben die erste Ableitung des Gesamtnutzens. Er wendet seine Erkenntnis zu Grenznutzen an, um Marktpreise bestimmen zu können und kommt - ebenso wie Menger, aber mit mathematischer Präzision - zur Bestimmung des Haushaltsoptimums. Das Austauschverhältnis zweier Güter ist reziprok zum Verhältnis der Grenznutzen zweier Güter. Dies entspricht dem oben bereits erwähnten "`Zweiten Gossenschen Gesetz"'. Das Konzept der Grenzbetrachtung wendete er in weiterer Folge auch auf die Theorie der Rente, des Kapitals und der Arbeit an. Das Verrichten von Arbeit hat hierbei einen negativen Nutzen, ähnlich wie dies auch noch in den modernen Gleichgewichtsmodellen berücksichtigt wird.

Ein (einigermaßen) "`bekannter Mann"' \parencite[S. 227]{Jevons1934} wurde der Ökonom übrigens bereits mit der Veröffentlichung seines Artikels "`The Coal Question"' \parencite{Jevons1865}. Darin beschrieb er ein Phänomen - mittlerweile genannt das Jevons-Paradoxon -, das im Hinblick auf den Klimawandel heute aktueller denn je ist: Höhere Effizienz in der z.B. Kohle-Förderung führt zu höherem Wohlstand bei gleichem Kohleverbrauch und nicht zu geringerem Kohleverbrauch bei konstant-bleibenden Wohlstand.

Jevons war ein extrem vielseitiger Wissenschaftler. Er war begeistert von den Naturwissenschaften \parencite[S. 225]{Jevons1934} und publizierte auch Beiträge in anderen Disziplinen, unter anderem zu Logik. Die Praxis der Messung und Darstellung von naturwissenschaftliche, konkret meteorologischen Daten, wollte auf die Ökonomie übertragen \parencite[S. 524]{Keynes1936a}. Statistische Darstellungen von wirtschaftlichen Zeitreihen stellte er als erster mittel Index-Zahlen dar. Auch behandelte er in diesem Zusammenhang die Frage der richtigen Anwendung von arithmetischem und geometrischem Mittelwert \parencite[S. 525]{Keynes1936a} wie heute üblich: Das durchschnittliche Wachstum entspricht dem geometrischen Mittel der vergangenen, jährlichen Wachstumsraten. Als Prognose für das Wachstum im nächsten Jahr wird hingegen der arithmetische Mittelwert herangezogen.


Ein umstrittener Beitrag war jener, in dem er einen kausalen Zusammenhang zwischen Sonnenflecken und Wirtschaftszyklen unterstellte. Eine hohe Anzahl an Sonnenflecken sollte demnach zu Abkühlung, schlechteren Ernteerträgen und in weiterer Folge Rezessionen führen. Die Theorie erwies sich rasch als falsch. Sein Sohn argumentierte in \textcite[S. 229, S. 232]{Jevons1934}, dass Jevons selbst nicht wirklich an einen derartigen kausalen Zusammenhang glaubte \footnote{In \textcite[S. 232]{Jevons1934} schreibt sein Sohn, dass Jevons als einer der ersten Ökonomen Konjunkturzyklen analysierte. Dabei nahm er an, dass ungefähr alle zehn Jahre eine Wirtschaftskrise stattfindet. Demnach fand er eine \textit{Korrelation} zwischen Sonnenflecken und Rezessionen, er glaubte aber demzufolge nie an einen \textit{kausalen} Zusammenhang, was ihm häufig unterstellt wird}, Der Originaltext, zitiert nach \textcite[S. 529]{Keynes1936a}, lässt allerdings Gegenteiliges vermuten. Wie auch immer, der Begriff "`Sunspot Equilibrium"' schaffte es in das Vokabular der modernen Ökonomie. \textcite{Cass1983} verwendeten den Begriff erstmals und beschrieben damit - in Anlehnung an Jevons - die Möglichkeit eines Marktgleichgewichtes, das Zustande kommt, weil die Marktteilnehmer an die Bedeutung einer, in Wirklichkeit völlig bedeutungslosen, Variable glauben. Der allgemeine Glaube daran wird schließlich zur selbst-erfüllenden Prophezeiung. Im Neu-Keynesianismus wird diese Theorie vereinzelt zur Erklärung von Marktunvollkommenheiten herangezogen (vgl. \textcite{Woodford1990b}).

Wie die Geschichte manch anderer berühmter Wissenschaftler, hat auch jene von Stanley Jevons eine tragische Seite. Die eben erwähnte Vielseitigkeit könnte man heute auch so auslegen, dass er wohl ein "`Workaholic"' war, was ihn allerdings - laut seinem Sohn \parencite[S. 230]{Jevons1934} - körperlich überlastete. Mit 18 Jahren ging er für fünf Jahre nach Australien, wohl gegen seinen eigenen Willen, zugunsten finanzieller Probleme seiner Familie \parencite[S. 518]{Keynes1936a}. Nach seiner Rückkehr, litt er unter psychischen Problemen und fühlte gescheitert zu sein \parencite[S. 527]{Keynes1936a}. Auch noch nachdem er erste, erfolgreiche Publikationen vorweisen konnte.  1866, mit 29 Jahren erhielt er ein Professur in Manchester, aber schon mit 36 Jahren musste er krankheitsbedingt eine berufliche Auszeit nehmen und wenige Jahre später, 1875, die Professur aufgeben. Zwar nahm er kurz darauf am University College in London erneut eine Professur an, aber auch diese musste er 1880 krankheitsbedingt aufgeben \parencite[S. 230]{Jevons1934}. Schon 1882 schließlich, starb er bei einem Badeunfall in Südengland mit nur 46 Jahren.

Jevons gilt übrigens nicht als Vertreter der berühmten "`Cambridge School of Economics"', da er selbst nie in Cambridge tätig war. Erst Alfred Marshall (vgl. Kapitel \ref{Neoklassik}) begründete diese Schule, die bis ins 20. Jahrhundert eine Hochburg der Neoklassik blieb (vgl. Kapitel \ref{Neoklassik_nach1945}). Über den Einfluss von Jevons auf Marshall wurde schon zu deren Lebzeiten, und sogar von den beiden selbst, heftig diskutiert \parencite[S. 536]{Keynes1936a}. Häufig wird Jevons eine Generation älter eingeschätzt als Marshall, da er sein Hauptwerk 1971 veröffentlichte, Marshall hingegen erst 1890. In Wirklichkeit war Jevons nur knapp sieben Jahre älter als Marshall. Letzter lehnte es lange ab, die bahnbrechenden Leistungen Jevons anzuerkennen \parencite[S. 535]{Keynes1936a} und verwies stattdessen im Vorwort seines berühmten Werks "`Principles of Economics"' auf die Vorläufer Cournot und Thünen und erwähnt Jevons nur kurz \parencite[S. XXII]{Marshall1890}. 


\section{Walras: Ikone der Gleichgewichtstheorie}
\label{Walras}

Marie-Esprit Leon Walras war der dritte im Bunde jener Ökonomen, denen heute häufig die "`Marginalistische Revolution"' zugeschrieben wird. Sein Beitrag ging aber über die Entwicklung der "`Grenznutzentheorie"' hinaus, indem er im gleichen Werk die Theorie des "`allgemeinen Gleichgewichts"' etablierte, die bis heute zentral ist für die ökonomische Forschung. Der Begriff "`Walras-Gleichgewicht"' ist bis heute in der Volkswirtschaftslehre fest verankert. Unumstritten ist er von den drei genannten Ökonomen - Menger, Jevons und eben Walras - jener, der heute am bekanntesten ist und die umfassendste Wirkung auf die Ökonomie hatte. Er gilt als einer der einflussreichsten Ökonomen aller Zeiten. Für \textcite[S. 826]{Schumpeter1954} war er gar "`the greatest of all economists"'. Dabei war Walras eher ein "`Spätstarter"' in Bezug auf seine wirtschaftswissenschaftliche Laufbahn. Nachdem seine Aufnahme am renommierten "`Ecole Polytechnique"' nahe Paris zweimal wegen mangelnder Mathematikkenntnisse gescheitert war \parencite[S. 60]{Felderer1989}, arbeitete er als Journalist und verfasste nebenbei Romane. Anstatt technischer Fächer studierte er schließlich Ökonomie und arbeitet später auch in Banken und bei einer Eisenbahngesellschaft. Nach einigen gescheiterten Bewerbungen, erhielt er 1870 etwas überraschend \parencite[S. 63]{Streissler1990} eine Professur in Lausanne \parencite[S. 326]{Rosner2012}. Seine aus heutiger Sicht bedeutendste Arbeit veröffentlichte er 1874: "`Elements d'Economie Politique Pure ou Theorie de la Richesse Sociale"'. Dieses Werk ist die Grundlage der "`Lausanner Schule"' \parencite[S. 326]{Rosner2012}. Krankheitsbedingt musste er seine Lehrtätigkeit 1892 aufgeben. Bis zu seinem Tod im Jahre 1910 blieb er jedoch mit der wissenschaftlichen Community über Briefwechsel in regem Austausch, was für die damalige Zeit außergewöhnlich war und zudem in einer umfassenden Publikationstätigkeit resultierte. Bahnbrechend blieb aber vor allem sein Werk aus 1974 und hierbei vor allem der Versuch des Beweises der Existenz eines "`Allgemeinen Gleichgewichts"'. Sowohl sein mathematisch-formalistischer Zugang zur Ökonomie als auch die Frage nach einem "`Allgemeinen Gleichgewicht"' stießen bei seinen Zeitgenossen auf wenig Anerkennung. Die Bedeutung seines Werkes wurde erst nach seinem Tod entsprechend gewürdigt. \textcite[S. 75]{Felderer1989} meinte diesbezüglich, dass es "`wohl keinen anderen großen Ökonomen gibt, dessen zeitgenössische Bedeutung sich von seiner heutigen stärker unterscheidet"'. Vielleicht eine Ironie des Schicksals, dass es gerade \textcite{Walras1885} zu verdanken ist, dass die Arbeiten von Hermann Heinrich Gossen doch noch ihre Würdigung fanden. Dem wurde zu Lebzeiten schließlich noch weniger Anerkennung entgegengebracht (vgl. Kapitel \ref{Vorläufer}).  

\textcite{Walras1874} entwickelte in seinem Hauptwerk seine eigene Version der Theorie des Grenznutzens. Er war übrigens enttäuscht als er merkte, dass diese Theorie schon von \textcite{Jevons1871} und \textcite{Menger1871} beschrieben wurde und entwickelte vor allem gegenüber Jevons eine Rivalität, die sich aber wieder abschwächte als er erfuhr, dass \textcite{Gossen1854} bereits zwei Jahrzehnte zuvor ganz ähnliche Ideen hervorbrachte. Wie beschrieben wurde er in weiterer Folge ein großer Bewunderer Gossens und bemühte sich seine Verdienste bekannt zu machen. 

Aus heutiger Sicht, war allerdings ohnehin seine Forschung zur Thematik des allgemeinen Gleichgewichts von größerer Bedeutung. Interessanterweise arbeitet er am Thema des allgemeinen Gleichgewichts schon bevor er seine Version der Marginalbetrachtung erarbeitete und präsentierte diese auch schon außerhalb seines Hauptwerkes \parencite[S. 513ff]{Jaffe1976}. Ohne Grenzüberlegungen hing seine Gleichgewichtstheorie allerdings in der Luft.

In seinem Hauptwerk - \textcite{Walras1874} - stellt er dann zunächst seine Grenznutzentheorie dar - bei ihm wird das Konzept \textit{rareté} genannt. Ausführlich beschreibt \textcite[S. 21ff]{Walras1874}, dass der Wert von \textit{knappen} Gütern durch Tausch realisiert wird. Darauf aufbauend entwickelt er sein Konzept vom Tauschgleichgewicht. Zunächst für zwei Güter, wobei er - was wichtig ist - die Preise des ersten Gutes als Menge des zweiten Gutes ausdrückt \parencite[S. 47]{Walras1874}. Weiters leitet er daraus effektive Angebots- und Nachfragekurven ab. Hier ist die Verbindung zum abnehmenden Grenznutzen entscheidend: Nur dadurch steigt der Nutzen eines Gutes mit zunehmender Menge nicht linear. Das heißt, Personen sind - im hier eben dargestellten Zwei-Güter-Beispiel - geneigt eine Kombination aus beiden Gütern anzustreben. Nutzenmaximierung tritt dann ein, wenn "`das Verhältnis der 'raretés' gleich dem Preis ist"' \parencite[S. 89]{Walras1874} - eine Erkenntnis die noch heute in Mikroökonomie-Lehrbüchern genauso dargestellt wird, wenn man "`raretés"' durch "`Grenznutzen"' ersetzt. Im Maximum hat keine der Parteien einen Anreiz zu weiterem Tauschhandel, der Preis bleibt stabil, ein Gleichgewicht ist erreicht.

Das \textit{allgemeine} Gleichgewicht entwickelt Walras schließlich daraus, dass er diese Analyse vom Zwei-Güter-Fall zum Drei-Güter-Fall und schließlich zum beliebig großen m-Güter-Fall ausbaut. Wichtig ist hierbei sein Ansatz, dass der Preis eines Gutes jeweils als Tauschverhältnis zu den anderen Gütern ausgedrückt werden kann. So kommt er zur Erkenntnis, dass im Drei-Güter-Fall zwei Gleichungen ausreichen um das Preis-Mengen-Gleichgewicht darzustellen \parencite[S. 121]{Walras1874}, bzw. im m-Güter-Fall gilt: Es gibt $m$ Güter und dafür $m-1$ Möglichkeiten den Preis dafür auszudrücken, also $m*(m-1)$ Preise. Gleiches gilt für die Mengen. Das ergibt $2m*(m-1)$ Unbekannte \parencite[S. 124]{Walras1874}. Der darauf folgende Versuch eines mathematischen Beweises über den Ausschluss von Arbitrage-Möglichkeiten liest sich aus heutiger Sicht etwas naiv, aber der intuitive Zugang ist für die frühe Zeit der höheren Mathematik in der Ökonomie faszinierend:  Die grundsätzliche Existenz eines Allgemeinen Gleichgewichts wollte Walras schließlich beweisen indem er mit der Anzahl der Gleichungen und Unbekannten argumentierte. Wie bereits erwähnt, kann eines der $m$ Güter als "`Referenzgut"' - bei Walras \textit{numeraire} genannt - herangezogen werden, an dem sich die Preise aller anderen Güter als Austauschverhältnisse orientieren. Dadurch ergibt sich ein System, in dem die Anzahl der Unbekannten gleich der Anzahl der Gleichungen ist. Dieser "`Abzählbeweis"' reicht für Walras reicht aus, um die grundsätzliche Lösbarkeit des System zu zeigen.

Es bleibt aber noch ein Problem: Es besteht immer dann eine Arbitragemöglichkeit wenn der direkte Preis (noch immer als Austauschverhältnis der Mengen) zwischen zwei Gütern A und B dadurch gedrückt werden kann, dass der Tausch von A gegen C und anschließende Tausch C gegen B vorteilhafter ist als der direkte Tausch. Solange es Arbitragemöglichkeiten gibt, kann sich ein System nicht im Gleichgewichtszustand befinden. 

Für dieses Problem erfindet Walras den sogenannten "`Auktionator"' (Tatonnement -Prozess) \parencite[S. 229]{Walras1874}. Dieser sammelt solange Informationen der Angebots- und Nachfrageseite und erhöht bzw. senkt die Preise bis alle Märkte im Gleichgewicht sind. Erst danach kommt es zum Tauschprozess. Eine etwas konstruierte, aber elegante Lösung.

Insgesamt schuf \textcite{Walras1874} mit der Allgemeinen Gleichgewichtstheorie und dem Konzept des Auktionators ein beeindruckendes Forschungsfeld, das bis heute als eine der Grundlagen der Ökonomie angesehen werden kann. Die mikroökonomischen Voraussetzungen wie vollkommene Märkte, perfekte Konkurrenz und die Annahme reiner Tauschmärkte, also die Nicht-Einbeziehung von Geld, sowie die rein statische Analyse, machten das Modell zunächst zu einem reinen Theoriegebilde. 

Walras gelang es damit aber die Theorie des Allgemeinen Gleichgewichtes zu etablieren. Freilich ohne dessen Existenz nach heutigen Standards mathematisch formal beweisen zu können. Dazu war das mathematische Umfeld der Wirtschaftswissenschaften im 19. Jahrhundert aber auch noch gar nicht weit genug entwickelt. Er begründete damit aber eine Forschungsrichtung, die ab der Mitte des 20. Jahrhunderts - also Jahrzehnte nach seinem Tod - intensiv beforscht wurde. Zuerst wurde er von den Schweden, vor allem Wicksell (vgl.: \ref{cha:Stockholm}) wiederentdeckt. Den vermeintlichen Abschluss und Höhepunkt der Gleichgewichtsforschung  stellt sicherlich die Theorie von Kenneth Arrow und Gerard Debreu dar (vgl. Kapitel \ref{Neoklassik_nach1945}). Der Aufstieg des Keynesianismus schien die Bedeutung der allgemeinen Gleichgewichtstheorie einzuschränken. Zum Beispiel, weil bei Existenz von Arbeitslosigkeit eben offensichtlich nicht alle Märkte im Gleichgewicht sind. Durch das Aufkommen der Neu-Keynesianischen Gleichgewichtsmodelle Anfang der 1990er-Jahre und insbesondere durch die "`Neue Neoklassische Synthese"' (vgl. Kapitel \ref{Neue Neoklassische Synthese}) wurden die Ideen Walras' schließlich aber auch in der Makroökonomie übernommen. Tatsächlich sind daher alle gängigen Mainstream-makroökonomischen Modelle  "`allgemeine Gleichgewichtsmodelle"' - natürlich erweitert um all die Konzepte der Ökonomie des 20. Jahrhunderts - und werden in diesem Sinn ja auch heute noch in Journalartikeln und Lehrbüchern als "`walrasianisch"' bezeichnet.       % Vorläufer + Marginalistische Rev.		!!! KORREKTURLESEN
%%%%%%%%%%%%%%%%%%%%%% chapter.tex %%%%%%%%%%%%%%%%%%%%%%%%%%%%%%%%%
%
% sample chapter
%
% Use this file as a template for your own input.
%
%%%%%%%%%%%%%%%%%%%%%%%% Springer-Verlag %%%%%%%%%%%%%%%%%%%%%%%%%%

\chapter{Auf dem Historikerweg}
\label{Historisch}
Historische Schule

\section{List}

\section{Schmoller \& Hildebrand}
%%%%%%%%%%%%%%%%%%%%% chapter.tex %%%%%%%%%%%%%%%%%%%%%%%%%%%%%%%%%
%
% sample chapter
%
% Use this file as a template for your own input.
%
%%%%%%%%%%%%%%%%%%%%%%%% Springer-Verlag %%%%%%%%%%%%%%%%%%%%%%%%%%

\chapter{Das Ende der Ökonomie?}
\label{Neoklassik}

Die - wie wir mittlerweile wissen nur sogenannte - Marginalistische Revolution brauchte einige Zeit um sich zu etablieren. Gegen Ende des 19. Jahrhunderts setzten sich aber schließlich einige Gegebenheiten durch, die bis heute von Einfluss sind. Erstens, verlagerte sich die ökonomische Forschung fast ausschließlich in den universitären Bereich. Ökonomen wie Thünen oder Gossen, die gänzlich außerhalb des wissenschaftlichen Apparates arbeiteten, aber auch solche wie Jevons und Walras, die erst nach Tätigkeiten in der Privatwirtschaft im universitären Bereich Fuß fassten, sind seither eher die Ausnahme. Damit verlagerte sich, zweitens, die Publikationstätigkeit von gesamtheitlichen Wälzern auf hochspezialisierte Journalbeiträge. Ein Prozess, der zwar recht langsam voranschritt, die Gründungsjahre der heute noch führenden Journale deuten dies aber an: American Economic Review in 1911, das Economic Journal seit 1891, das Quarterly Journal of Economics 1886 und das Journal of Political Economy 1892 \parencite[S. 340]{Rosner2012}.
Drittens etablierte sich in der Ökonomie eine weitgehend einheitliche Sprache mit einheitlichen, fachspezifischen Ausdrücken. Einen wesentlichen Beitrag dazu leistete der britische Ökonom und Nachfolger Jevons in Cambridge Alfred Marshall.

\section{Marshall: Der Beginn der modernen Ökonomie}

\textcite{Marshall1890}: \textit{Principles of Economics - An Introductory Volume} gilt bis heute als eines der prägendsten Lehrbücher aller Zeiten. Es fasst nicht nur den State of the Art der Ökonomie zusammen, sondern erweiterte denselben auch. Der durchschlagende Erfolg dieses Buches hängt sicherlich auch damit zusammen, dass sich die mikroökonomische Theorie seit damals kaum mehr verändert hat. Natürlich wurde sie entscheidend und an vielen Stellen erweitert. Aber die damals schon bestehenden Theorien zur Mikroökonomie gelten bis heute unverändert und sind tatsächlich in einführenden Lehrbüchern praktisch identisch abgedruckt.

Was sein Privatleben anging, stammte Marshall aus einer hoch-religiösen Familie und war in seiner Kindheit vom "`tyrannischen"' Vater \parencite[S. 313]{Keynes1924} geprägt. Gegen den Willen seines Vaters und mit der Hilfe eines Darlehens von seinem Onkel studierte er Mathematik in Cambridge. Nach dem Abschluss seines Studiums nahm er 1868 bereits seine Lehrtätigkeit dort auf. Er publizierte in seinen frühen Jahren wenig, obwohl er viel verfasst und sich auf Reisen auch viel Wissen in den USA und Kontinental-Europa aneignete. 1877 musste er Cambridge verlassen, weil er eine ehemalige Studentin heiratete. 1885 aber konnte er zurückkehren. Ab dann stieg er rasch zum führenden englischen Ökonomen auf. Bereits 1879 hatte sein Mentor Henry Sidgwick einige seiner Werke veröffentlicht, die viel Aufsehen erregten. 1890 folgte sein Hauptwerk, die "`Principles of Economics"'. In weiterer Folge war er auch als Berater öffentlicher Stellen in England tätig. Ständig erweiterte er nebenbei seine "`Principles"'. Als Person, die empfindlich auf Kritik reagierte, wurde er zum "`Workaholic"', was schließlich seine Gesundheit angriff. 1908 wurde er emeritiert. In seinen letzten eineinhalb Lebensjahrzehnten publizierte er weiter, getrieben von seinem Perfektions-Drang \parencite[S. 145f]{Rieter1989}. Die inhaltliche Bedeutung seiner späten Arbeiten blieb aber beschränkt.

Mit Marshall änderte sich der Blick auf die Ökonomie grundlegend. Er gilt daher nicht umsonst als der "`Vollender der Neoklassik"'. Neben inhaltlicher Punkte, revolutionierte Marshall vor allem auch die ökonomische Methodologie und allgemein das Bild der Ökonomie als Wissenschaft:

Erstens, zunächst besticht sein Hauptwerk durch die Kombination von sprachlicher Verständlichkeit und mathematischer Präzision. Vergleicht man sein Werk mit jenen von, zum Beispiel, Menger, Walras oder Böhm-Bawerk, so merkt man sofort deutliche Unterschiede in Aufbau, sowie ein bessere Verständlichkeit. Dies ist zwar nicht unbedingt ausschließlich der Arbeit Marshalls eigen, sondern eher dem Zeitgeist zuzuschreiben und so auch zum Beispiel bei Irving Fisher zu finden, aber es besticht dennoch in \textcite{Marshall1890}. 

Zweitens, Marshall grenzte als erster die Volkswirtschaftslehre ("`Economics"') als Wissenschaft von den übrigen Sozialwissenschaften ab. Dazu "`kürzte"' er die politischen Entscheidungs- und Machtstrukturen aus seinem wirtschaftswissenschaftlichem Werk. Die bis dahin übliche Bezeichnung der "`Politischen Ökonomie"', wurde zur "`Volkswirtschaftslehre"'. Dies war für sein Werk notwendig. Die von ihm so elegant durchgezogene formale Herangehensweise ist nur dann möglich, wenn man Nebenbedingungen definiert und diese als gegeben annimmt. Dies ist eine der Stärken der Neoklassik. Es brachte ihr aber auch harsche Kritik ein, weil sich die Volkswirtschaftslehre damit von der erlebten Realität ein gutes Stück entfernt. Diese Kritik ist im Falle Marshalls ungerechtfertigt. Er war sich durchaus bewusst, dass seine "`Principles"' eine zu starke Vereinfachung der wirtschaftlichen Realität darstellen. Dementsprechend plante er, ja rang förmlich, um die Entstehung einer Fortsetzung seines Werkes \parencite[S. 146]{Rieter1989}. Tatsächlich ließ er erst mit der sechsten Auflage der "`Principles of Economics"' den Zusatz "`Volume I"' streichen. Mit 80 Jahren - 1920: "`Industry and Trade"', sowie 1923: "`Money Credit and Commerce"' - publizierte er schließlich seine Vorarbeiten zu weiteren Bänden, diese können aber nur mehr als fragmentierte Beiträge zu einzelnen Themenkomplexen gesehen werden. Die Abgrenzung der Volkswirtschaftslehre betrieb Marshall nicht nur inhaltlich, sondern auch organisatorisch. An der Universität von Cambridge setzte er durch, dass sie aus der Fakultät für "`Moral Science"' herausgelöst und stattdessen eine eigene Fakultät, mit eigenem Studiengang wurde \parencite[S. 141]{Rieter1989}. Nicht nur \textcite[S. 365]{Keynes1924} bezeichnete Marshall daher als "`Begründer der Cambridge School of Economics"'.

Drittens, Methodisch gilt die Arbeit Marshall's heute vielen als bahnbrechender als seine inhaltlichen Beiträge. Er führte die Ceteris-Paribus-Betrachtung in die Ökonomie ein. Also die Auswirkungen der Änderung eines einzelnen Einflussfaktors auf das Gesamtergebnis. Diese Betrachtung von statischen Gleichgewichten blieb in der Ökonomie lange der Standard. Marshall selbst sah darin einen Ausgangspunkt, war sich aber bewusst, dass eine dynamische Betrachtung besser wäre, aber in Modellen auch schwerer zu erfassen \parencite[S. 153]{Rieter1989}.

Viertens, Er war als einer der ersten Ökonomen extrem gut informiert über den "`State of the Art"' der Ökonomie. Heute ist es unumgänglich die wissenschaftlichen Arbeiten des Forschungszweiges, in welchem man selbst publizieren möchte, umfänglich zu kennen. Im 19. Jahrhundert allerdings sorgten Probleme der Sprache, Verfügbarkeit und bloßen Kenntnis dafür, dass oft Theorien entwickelt wurden, ohne dass die Urheber den Stand der Wissenschaft kannten. Wir erinnern uns zum Beispiel an Gossen. Aber auch Walras verfasste zunächst sein Hauptwerk und trat erst gegen Ende seiner Karriere in intensiven Austausch mit zeitgenössischen Ökonomen. Ganz anders war dies bei Marshall: Wie \textcite{Groenewegen1995} beschreibt, beschäftigte sich dieser bereits in den späten 1860er Jahren mit den deutschen Ökonomen wie Thünen, Roscher und Rau und kannte die Arbeiten der Franzosen Cournot und Dupuit. Selbstverständlich waren ihm die englischen Klassiker bekannt, aber auch die Historische Schule der Deutschen war ihm nicht fremd \parencite[S. 140]{Rieter1989}. Aufbauend auf all dem Wissen publizierte er sein Hauptwerk. Dies verhältnismäßig spät mit 48 Jahren im Jahr 1890. Die Erkenntnisse waren zu dieser Zeit bereits allesamt weitgehend bekannt und zirkulierten als Mitschriften aus seinen Vorlesungen. Er zitierte in den "`Principles"' auch nur wenig seinen Zeitgenossen Jevons, sondern eben vor allem die "`Vorläufer"'. Es gilt auch bis heute als umstritten, ob Marshall tatsächlich primär der \textit{Entwickler} bedeutender ökonomischer Theorien ist, oder doch eher der \textit{Vereiniger} ökonomischer Elemente, die bereits jeweils jemand anderer entwickelt hatte \parencite[S. 207ff]{Ekelund2002}. Marshall selbst behauptete in hohem Alter, dass er den Rahmen für seine "`Principles"' schon vor 1871, also dem Erscheinungsjahr der Arbeiten von Jevons und Menger, fertig hatte und er aus diesem Grund mehrheitlich die "`Vorläufer"' und kaum das Trio Jevons, Menger und Walras, zitiert hatte \parencite[S. 140]{Rieter1989}. 

Erst jetzt kommen wir zu seinen inhaltlichen Beiträgen: Allseits bekannt ist sicherlich das "`Marshall'sche Kreuz"', also seine Darstellung von Angebot und Nachfrage und dem daraus entstehenden Gleichgewicht. Darüber ob Marshall damit eine bahnbrechende Leistung erbracht habe, diskutierten Generationen von Ökonomen. Bekannt ist, dass die erste Darstellung von Angebot und Nachfrage als sich schneidende Kurven um das Jahr 1840 entstand und auf Antoine-Augustin Cournot zurückgeht \parencite[S. 3]{Humphrey1992}, oder eventuell von Karl Rau sogar schon noch etwas früher so dargestellt wurde \parencite[S. 159]{Blaug2001}. Definitiv auf Marshall geht damit aber die Verbindung zwischen marginalistischer Nachfragefunktion und eher klassischer Angebotsfunktion zurück. Die negativ verlaufende Nachfragefunktion leitet Marshall aus dem sinkenden Grenznutzen beim Konsum von Gütern ab. Die steigende Angebotsfunktion von den mit steigender Menge steigenden Produktionskosten. Dies findet sich bereits bei Ricardo. Wenn dieser auch davon ausging, dass die Produktionskosten deshalb stiegen, weil knappe Ressourcen in abnehmender Qualität zur Verfügung stehen würden.

Quasi als Nebenprodukte seiner Herleitung von Nachfrage- und Angebotsfunktion, schuf er Instrumente, die heute noch gängige Praxis sind. So zum Beispiel die Elastizität, die er konkret als relative Änderungen der nachgefragten Menge bei Änderung des Preises definiert. Schon Marshall führt ein Beispiel von Robert Giffen an, der darlegte, dass die Brot-Nachfrage in Irland im 19. Jahrhundert trotz steigender Preise anstieg. Bis heute lernt man das seltene Phänomen, dass trotz steigender Preise die Nachfrage steigt, als Giffen-Paradoxon (bzw. Giffen-Gut) kennen. Ebenfalls direkt auf \textcite{Marshall1890} geht die Analyse der Produzenten- und Konsumentenrente in heute üblicher Form zurück. Also die Differenz zwischen Preis, zu dem ein Produzent anbieten würde und dem Marktpreis, bzw. dem Preis, den ein Nachfragender bereit wäre zu bezahlen und dem Marktpreis. Marshall behandelte auch schon die Auswirkungen von Steuern auf diese die Konsumentenrente \parencite[S. 351]{Rosner2012}. 

Das Gesamtwerk Marshalls bildet bis heute die Grundlage der Mikroökonomie. Im idealisierten Bild des vollkommenen Marktes, auf dem vollständige Konkurrenz herrscht, ohne Zugangsschranken, mit streng nutzen-maximierenden Teilnehmern und ohne jegliches Marktversagen, gilt noch heute: "`It's all in Marshall!"'\footnote{Ein Zitat, das Arthur C. Pigou zugeschrieben wird, aber wohl nur sinngemäß tatsächlich geäußert wurde. \parencite{Audretsch2007, Pigou1925}}. Tatsächlich war die ursprüngliche "`neoklassische Theorie"' mit Marshall gewissermaßen abgeschlossen, wenn man dies in dem Sinne versteht, dass seine Erkenntnisse wie in seinem Werk "`Principles of Economics"' auch in modernen Mikroökonomie Büchern praktisch identisch dargestellt werden\footnote{Wichtige Ergänzungen etwa im Hinblick auf die Nutzendarstellung kamen etwas später noch von Vilfredo Pareto und Francis Edgeworth.}. Diese "`Vollendung der Neoklassik"' kann deshalb vielleicht als Ausgangspunkt der \textit{modernen} Ökonomie gelten.

Wie soeben dargestellt: Welcher Umfang von bedeutenden Beiträgen erst durch Marshall bekannt wurde, ist unumstritten, worin seine bahnbrechende Leistung nun tatsächlich lag, kann hingegen nicht eindeutig beantwortet werden. Nicht durch die \textit{eine} bedeutende Leistung gilt Marshall als einer der bedeutendsten Ökonomen, sondern durch sein Gesamtwerk. Wie \textcite{Rieter1989} es ausdrückt: "`Man empfindet ihn als Ganzes [...]. Ein komfortabler Neubau, errichtet auf alten Fundamenten."' 

Marshall war von verschiedensten Richtungen Kritik ausgesetzt. Wenig überraschend lehnten ihn die Sozialisten praktisch einfach grundsätzlich ab. Aber auch seine Zeitgenossen aus der "`Historischen Schule"'(vgl. Kapitel \ref{Historisch}) kritisierten ihn heftig, vor allem für die unrealistischer Annahmen, die für seine Modell notwendig sind und die gesetzmäßigen Zusammenhänge, die diese Modelle liefern. Die Amerikaner um Veblen (vgl. Kapitel \ref{Institut}) kritisierten, dass Marshall das Marktwirtschaftliche Wirtschaftssystem implizit als effizientes und gerechtes System akzeptierte. Sogar die österreichische Schule hielt wenig von Marshall's Lehre \parencite[S. 151]{Rieter1989}. Die Kritik traf Marshall hart und ungerechtfertigt. Er selbst stellte "`wirkliche"' ökonomische Probleme in den Vordergrund, hielt wenig vom starren Rationalprinzip und nannte die Bekämpfung der Armut als zentrales Ziel der Volkswirtschaftslehre \parencite{Rieter1989}. Heute wissen wir, dass sich Marshall's Lehren, entgegen aller Kritik, als Mainstream durchgesetzt haben. Interessant ist in diesem Zusammenhang, dass die Kritik an der Neoklassik bis heute eine ähnliche geblieben  ist. Auf wissenschaftlich stärker fundierter Ebene wurde und wird nach wie vor primär die Realität verschiedener Modellannahmen in Zweifel gezogen. Auch heute gibt es kaum einen Ökonomen, der die Modellannahmen für 100\% richtig hält und die gesetzmäßige Gültigkeit der Modellergebnisse als gegeben annimmt. Aber auf der anderen Seite hat sich bis heute in der Mikroökonomie keine umfassende Alternative zur Neoklassik durchgesetzt.  

Die moderne ökonomische Forschung \parencite{Ekelund2002, Blaug2001, Humphrey1992} sieht Marshall eher als "`Synthesizer"', denn als Entwickler, \parencite[S. 212]{Ekelund2002} der ökonomischen Theorien zum neoklassischen Gesamtwerk. Aber er gilt auch als Entwickler der modernen wirtschaftswissenschaftlichen Methodik. So schuf er den Rahmen für die lange vorherrschende statisch-komparative Analyse und verband induktive Theoriebildung mit deduktiver empirischen Überprüfung \parencite[S. 212]{Ekelund2002}. Insgesamt war er auf jeden Fall \textit{die} prägende Figur der frühen Neoklassik im England des späten 19. Jahrhunderts. Seine Arbeiten sind wohl die frühesten, die noch heute fast unverändert Teil der Mainstream-Ökonomie sind. Konkret wenn es um die mikroökonomische Analyse auf vollständigen Konkurrenzmärkten geht. 

Wir wissen aber natürlich, dass sich die Wirtschaftswissenschaften seither vielfältig weiterentwickelt haben. Sein Nachfolger in Cambridge, Arthur C. Pigou, machte sich als einer der ersten Gedanken über "`Marktversagen"', also Situationen, in dem eine rein rationale-mikroökonomische Analyse unerwünschte Marktergebnisse zum Vorschein bringt (vgl. Kapitel \ref{sec: Pigou}. Die Verbindung seiner Arbeiten mit der Gleichgewichtstheorie von Walras (vgl. Kapitel \ref{Arrow-Debreu}) und nicht zuletzt die neoklassische Wachstumstheorie (vgl. Kapitel \ref{sec: Solow-Modell}), sind Forschungsgebiete, die die Marshall'sche Neoklassik nach 1945 wesentlich weiterentwickelten.


\section{Edgeworth und Pareto: Die Lösung des Nutzenproblems, die bis heute hinkt}

Fassen wir zusammen: Eines, wenn nicht \textit{das}, die Neoklassik auszeichnende Element, ist das Nutzenkonzept, bzw. die Theorie des abnehmenden Grenznutzens. Dieses Konzept ist an und für sich intuitiv verständlich und leicht nachzuvollziehen: Nach einer langen Wanderung liefert mir das erste Bier einen enormen Nutzen, das fünfte Bier liefert ebenfalls einen Nutzen, doch ist dieser zweifelsohne deutlich geringer. Alleine dieses Konzept des abnehmende Grenznutzens lässt eine grafische Transformation von Geldeinheiten in Nutzeneinheiten schon zu: Der erste Euro liefert den höchsten Nutzen, der zweite einen etwas geringeren, der dritte Euro eine wiederum etwas geringeren, usw. Wenn man den Nutzen auf der y-Achse und die Geldeinheiten auf der x-Achse abträgt, erhält man eine Funktion die vom Ursprung ausgehend durchgehend einen positiven Anstieg aufweist. Der Anstieg verläuft dabei aber immer flacher. Vorausgesetzt eine Nutzenfunktion erfüllt diese Anforderung, dann können Geldeinheiten einfach in Nutzeneinheiten "`umgerechnet"' werden. Der entsprechende Nutzen wird dann in Zahlen ausgedrückt. Man spricht vom "`kardinalen Nutzenprinzip"'. Genau hier liegt aber ein Problem, das die frühen Neoklassiker\footnote{Walras und Menger behandelten das Problem tatsächlich nicht, Jevons meinte zwar, Nutzen sei nicht direkt messbar, er akzeptierte aber den Umweg über Geldeinheiten. Der Nutzen verschiedener Güter kann demnach im äquivalenten Geldwert ausgedrückt werden.} schlicht ignoriert haben \parencite[S. 328]{Blaug1962}: Nutzen ist in Wirklichkeit nicht direkt messbar. Es gibt keine sinnvolle Einheit in der man Nutzen quantifizieren könnte. Dementsprechend sind Rechenoperationen mit Nutzeneinheiten sinnlos.

Der erste, der dies ausführlich thematisierte war Francis Ysidro Edgeworth. Er war sowohl mit Stanley Jevons als auch mit Alfred Marshall befreundet und auch ein früher Verfechter der Mathematik in der Ökonomie. In seinen "`Mathematical Psychics"' \parencite{Edgeworth1881} kritisierte er, dass die Neoklassiker Nutzen unzulässigerweise als quantitative Variable behandelten. \textcite{Edgeworth1881} schlug vor nach Wegen zu suchen, den Nutzen tatsächlich direkt zu messen. Er verfolgte also auch ein kardinales Nutzenkonzept. Dazu wollte er einen "`Hedonimeter"' entwickeln, also ein Messgerät, dass den Nutzen direkt messen kann.  Übrigens verfolgte wenig später auch der junge Irving Fisher - der uns noch mehrmals unterkommen wird - in seiner Dissertation das Ziel einer kardinalen Nutzenmessung, allerdings schlug er vor diese indirekt vorzunehmen, also von getätigten Handlungen auf deren Nutzen zu schließen\parencite{Colander2007}. Das Konzept blieb schließlich in der Ökonomie ohne wesentliche Resonanz\footnote{Moderne Ansätze der Neuro-Ökonomie gehen allerdings wieder in Richtung kardinaler Nutzenmessung. Dabei wird in Magnetresonanz-Tomographen versucht Gehirnströme hinsichtlich Glücksgefühle zu messen}. Allerdings lieferte \textcite{Edgeworth1881} dennoch wichtige Bausteine für die Nutzentheorie. So entwickelte er darin das Konzept der - heute in der Ökonomie-Lehre nach wie vor omnipräsenten und ebenso beliebten - Indifferenzkurven. Abgeleitet können diese aus einer, wie oben beschriebenen, Nutzenfunktion. Interessant sind die Indifferenzkurven im Zwei-Güter-Fall. Angenommen ich bilde in einem Koordinatensystem die Menge von Gut A auf der x-Achse und die Menge von Gut B auf der y-Achse ab. Wenn ich mein ganzes Geld für Gut A ausgebe erreiche ich einen bestimmten Punkt direkt auf der x-Achse (und vice versa). Aus dem Konzept des abnehmenden Grenznutzens wissen wir, dass eine Güterkombination aus A und B gegenüber nur A (oder nur B) vorteilhaft ist. Oder mit anderen Worten: Ich bekomme für eine geringere Geldmenge, die ich für eine Güterkombination ausgebe den gleichen Nutzen, wie für ein höhere Geldmenge, die ich ausschließlich nur für A (oder nur für B) ausgebe. Verbindet man alle Güterkombinationen aus A und B, die den identischen Nutzen liefern, miteinander spricht man von einer Indifferenzkurve. Diese beginnt jeweils an einem Punkt auf der x- und y-Achse und ist zum Ursprung geneigt \parencite[S. 21ff]{Edgeworth1881}. Bildet also eine Linkskurve ab, bzw. verläuft konvex. \textcite{Edgeworth1881} beschreibt in weiterer Folge, wie zwei Personen miteinander über das Austauschverhältnis dieser zwei Güter verhandeln. Man stelle sich nun das soeben beschriebene Koordinatensystem mit zwei Personen vor. Zusätzlich zur Person A, deren Ausgangspunkt der Ursprung, also "`links unten"' ist, eine zweite Person B, deren Ausgangspunkt "`rechts oben"' ist. Seine Indifferenzkurven verlaufen spiegelverkehrt zu jenen von A, das heißt diese sind zum Punkt "`rechts oben"' geneigt. B hält in diesem Fall alle Güter 1 (aber kein 2), A hält die gesamte Menge 2 (aber keine 1). Beide wollen nun in einen Tauschprozess kommen. Mögliche "`Tauschpunkte"' sind überall dort wo sich die Indifferenzkurven der beiden Personen schneiden. Ein Tauschgleichgewicht und gleichzeitig eine maximale aggregierte Wohlfahrt (welfare) wird aber bei Edgeworth nur in einem Punkt erreicht, nämlich wo sich die Indifferenzkurve von A und B genau tangieren. Dies ist aber eine falsche Annahme seitens Edgeworth - es gibt mehrere Tangentialpunkte und vor allem keine Möglichkeit eine "`allemeines Nutzenmaximum"' zu identifizieren \parencite[S. 49]{Humphrey1996}. Das soeben beschriebene Tool hat dennoch extreme Bedeutung erlangt ist heute als die "`Edgeworth-Box"' bekannt\footnote{Edgeworth selbst stellte die Box leicht abweichend dar, nämlich mit den Personen A und B "`rechts unten"', bzw. "`links oben"'. Details zur reichhaltigen Geschichte der Edgeworth-Box sind in \textcite{Humphrey1996} dargestellt}. Sie ist ein wichtiges Element in der allgemeinen Gleichgewichtstheorie zu der wir gleich wieder kommen werden. In \textcite{Edgeworth1881} sind beide Konzepte, also Indifferenzkurven und Edgeworth-Box, mathematisch und verbal beschrieben. Bekannt gemacht und angewendet hat beides schließlich Vilfredo Pareto, wobei er der erste war, der beide Konzepte grafisch wie heute üblich darstellte.

Was Marshall nicht beachtete waren die Fragen nach dem Allgemeinen Gleichgewicht im Sinne Walras'. Laut \textcite[S. 360]{Rosner2012} war Marshall das Werk von \textcite{Walras1874} zwar bekannt, aber er hatte ihm offenbar nicht den Stellenwert beigemessen, des es später erhalten sollte. Damit die Verbindung zu einem weiter wichtigen Wegbereiter der "`älteren"' Neoklassik vollends hergestellt: Vilfredo Pareto. Interessanterweise, die Gründe sind unbekannt, wurde er auf den deutschen Namen Fritz Wilfried getauft \parencite[S. 158]{Eisermann1989}. Bekannt wurde der Sohn eines nach Frankreich emigrierten Italieners und einer Französin allerdings unter dem Namen Vilfredo. Er war zunächst als Techniker tätig, veröffentlicht aber immer wieder Artikel in einem italienischen, ökonomischen Journal. Er wird daraufhin, von einem italienischen Wirtschaftsprofessor empfohlen, 1893 nach Lausanne als Nachfolger des kränklichen Walras berufen. Mit Walras überwirft er sich in weiterer Folge allerdings rasch. Als Person wird er als radikaler Liberaler bezeichnet \parencite{Cirillo1983}, seine wissenschaftliche Herangehensweise als streng logisch-deduktiv.  Pareto's Werk ist insgesamt geprägt von seiner technischen Ausbildung. Seine Arbeiten behandeln recht enge ökonomische Themen. Insgesamt ähnelt sein wissenschaftlicher Stil damit bereits jenem der modernen Wirtschaftswissenschaften \parencite[S. 362]{Rosner2012}. Politisch wird er häufig als Vorläufer des Faschismus bezeichnet, da er dessen Aufstieg begrüßt haben soll. Vor alle sein soziologisches Werk - Pareto veröffentlichte neben wirtschaftswissenschaftlichen auch bedeutende soziologische Arbeiten - wurde häufig mit dem Faschismus in Verbindung gebracht. Dies wurde in der Literatur häufig diskutiert. \textcite[S. 162]{Eisermann1989} - offenbar ein großer Bewunderer Paretos - argumentierte dieser habe den Faschismus stets abgelehnt, vor allem mit dem Verweis er wäre ein großer Liberaler gewesen, dessen logisch deduktiver Zugang so weit gegangen sei, \textit{jegliche} politische Entwicklung stets nur als distanzierter Beobachter zu analysieren. \textcite{Cirillo1983} betrachtet Pareto's politische Haltung durchaus kritischer, so war dieser zweifellos kein Anhänger der Demokratie. Aber auch er meint, die italienischen Faschisten, insbesondere Mussolini selbst, seine Anhänger Pareto's gewesen und wollten sein Werk für sich vereinnahmen. Eine endgültige "`Wahrheit"' kann wohl auch hierzu nicht ermittelt werden. Aus wirtschaftswissenschaftlicher Sicht lieferte Pareto auf jeden Fall in mehrere Gebieten bedeutende Beiträge. Interessant ist, dass einige seiner Entdeckungen heute über die Wirtschaftswissenschaften hinaus verwendet werden und weithin bekannt sind. So wird er wohl in der Allgemeinheit am ehesten mit dem  "`Pareto-Prinzip"' in Verbindung gebracht: Diese 80-20-Regel, wonach 80\% der Leistung durch 20\% des Aufwands erbracht werden, lässt sich auf seine Untersuchung der Einkommensverteilung in Italien zurückführen \parencite{Pareto1896}. Er fand heraus, dass diese nicht normalverteilt, sondern rechtsschief ist. Die von ihm abgeleitete Verteilung wird seither als "`Pareto-Verteilung"' bezeichnet und spielt, nicht zuletzt durch das wieder aufgeflammte Interesse an Fragen der Einkommensverteilung\parencite{Persky1992}, heute noch eine große Rolle in Verteilungsanalysen.

Sein aus wirtschaftshistorischer bedeutendster Beitrag folgte aber ein Jahrzehnt später. In seiner heute als "`Theorie der Wahlakte"' bezeichneten hat Pareto das hinkende Nutzenkonzept aufgegriffen und in Verbindung mit den Elementen von Edgeworth auf gänzlich neue Beine gestellt. Wie gesagt sind Jevons, Walras, Menger aber auch noch Marshall implizit von einem kardinalen Nutzenkomzept ausgegangen. Betrachtet man ein einzelnes Gut, so spielt diese Betrachtung keine große Rolle, weil es egal ist, ob man dem Nutzen in diesem Fall einen numerischen Wert zuweist. Betrachtet man aber ein Güterbündel führt die numerische Betrachtung des Nutzens zu Problemen. Es verleitet nämlich dazu anzunehmen, die Nutzenwerte der einzelnen Güter im Bündel ließen sich zum Beispiel zu einem Gesamtnutzen addieren. Dies macht aber keinen Sinn, weil Nutzen eben nicht direkt gemessen werden kann. Edgeworth hatte dieses Problem erkannt und mit der Entwicklung von Indifferenzkurven wichtige Vorarbeiten geleistet. Er arbeitet aber daran einen Weg zu finden, wie man Nutzen doch direkt messen kann und dann gültig in Zahlen ausdrücken kann. \textcite{Pareto1906} griff nun die Idee der Indifferenkurven auf, wendete sie aber praktisch "`aus einer anderen Richtung kommend"' an: Wir wissen, dass auf einer Indifferenzkurve liegende Punkte jeweils denselben Nutzen liefern. "`Höher"' liegende Indifferenzkurven liefern aber immer einen höheren Nutzen, als "`tiefer"' Indifferenzkurve. Diese Information aus Indifferenzkurven ist bei Pareto die einzig gültige und die einzig wichtige. Der Nutzen wird also nicht mehr kardinal gemessen, sondern ordinal: Indifferenzkurven können nur hinsichtlich ihrer Rangfolge sortiert werden. Man kann aber nach wie vor bestimmen welches Güterbündel besser als ein anderes ist, verwendet dazu aber keine irreführenden Zahlenwerte für den Nutzen mehr. Pareto revolutionierte mit der Einführung des ordinalen Nutzen das Nutzenkonzept nachhaltig, es wird bis heute in ökonomischen Modellen verwendet. In \textcite{Pareto1906} wurde übrigens auch erstmals der Begriff "`homo \oe conomicus"' verwendet.

Den ordinalen Nutzen führte Pareto in die "`Edgeworth-Box"' über, die durch seine Art der Darstellung schließlich popularisiert wurde. Darin kommt es zu einem Güteraustausch, wenn die Indifferenzkurven von zwei Individuen sich tangieren. Dies lässt sich wiederum verallgemeinern zum Haushaltsoptimum: Das Verhältnis der Grenznutzen zweier Güter ist gleich dem Preisverhältnis\footnote{Dies kennen wir schon aus dem Kapitel \ref{Vorläufer}, entscheidend ist die Herleitung über das ordinale Nutzenkonzept.}. Nebenbei wird durch diese Herangehensweise auch das bis heute häufig verwendete "`Pareto-Effizienz"' begründet: Wenn niemand besser gestellt werden kann, ohne gleichzeitig jemand anderen schlechter zu stellen. Dies wird häufig als der Ausgangspunkt der "`Wohlfahrtsökonomie"', mit der sich später Arthur Cecil Pigou explizit ausführlich beschäftigte (vgl. Kapitel \ref{sec: Pigou}). 

Man könnte nun davon ausgehen, dass Pareto's Nutzenkonzept in der neoklassischen Theorie sofort zum "`State of the Art"' wurde. Dem ist aber nicht so. Stattdessen verschwand die Diskussion über das Nutzenkonzept für längere Zeit \parencite[S. 148]{Blaug2001} aus dem Fokus der ökonomischen Forschung. Zu einer erneuten Überarbeitung kam es erst in den 1930er Jahren. \textcite{Hicks1934b} bzw. \textcite{Hicks1934a} griffen ausdrücklich \textcite{Pareto1906} noch einmal auf und deckten kleinere mathematische Inkonsistenzen darin auf. Die wesentliche Änderung war, dass bei der Bestimmung des Haushaltsoptimums das "`Verhältnis der Grenznutzen"' zweier Güter durch deren "`Austauschverhältnis"' ersetzt wurde. Der "`Abnehmende Grenznutzen"' verliert dadurch an Bedeutung und wird durch das Konzept der "`Grenzrate der Substitution"' ersetzt. Eine im wesentlichen rein formale, weniger eine inhaltliche Änderung. Die Frage, die nun beantwortet wird lautet: \textit{Welche Menge an X kompensiert meinen Verlust der letzten Einheit Y?} statt: \textit{Das wievielte Stück X liefert mir den gleichen Nutzen wie das wievielte Stück Y?}.

Das neoklassische Nutzenkonzept fand damit einen Abschluss. Es wird in der Form wie von \textcite{Hicks1934b} dargestellt auch in modernen Mikroökonomie-Büchern noch verwendet. Fazit: Für die Gleichgewichtsfindung benötigt man keine kardinale Nutzenmessung. Es reicht das Konzept des ordinalen Nutzen, in dem man nur eine Rangordnung der individuell präferierten Güterbündel angeben kann, aber keinerlei Aussagen treffen kann \textit{um wie viel höher} der Nutzen von Güterbündel A gegenüber jenem von B ist. Diese Erkenntnis war auch wichtig für die nach dem Zweiten Weltkrieg aufkommende Forschung zur Theorie des "`Allgemeinen Gleichgewichts"' (vgl. Kapitel \ref{Arrow-Debreu}). 

Es gibt bis heute Stimmen, die meinen die ordinale Nutzenmessung sei nur eine Notlösung. Wie wir wissen ist das gesamte Konzept des homo \oe conomicus fortlaufender Kritik aus verschiedenen Richtungen ausgesetzt. Dem kann man entgegenhalten, dass jeder Versuch einer kardinalen Nutzenmessung bislang gescheitert ist. Ein besseres Konzept ist bislang nicht greifbar. Aber man kann durchaus die damit verbunden Konsequenzen durchdenken: Bei ordinalen Nutzen macht zum Beispiel die Betrachtung der Konsumentenrente keinen Sinn \parencite[S: 400]{Rosner2012}. Schließlich handelt es sich dabei um die Summe der Nutzen verschiedener Personen. Noch einschneidender: Sämtliche Aussagen zur gesamtwirtschaftlichen Wohlfahrt haben damit keine Grundlage. Wenn interpersonelle Nutzenvergleiche unzulässig sind, macht es keinen Sinn darüber zu diskutieren, ob Verteilungspolitik zu Verbesserungen führen. Demnach lässt sich nicht sagen, ob 100EUR einer alleinerziehenden Mutter mehr Nutzen stiften als dem wohlhabenden Millionär. Die Neoklassik mit ihrem ordinalen Nutzenkonzept lässt diesbezüglich keine wissenschaftlich fundierte objektive Aussage zu, es handelt sich immer Werturteile. Wirtschaftspolitisch ist dies natürlich unbefriedigend. Man könnte damit sogar in die Leibnitz'sche Interpretation verfallen, wir lebten in der besten aller Welten: Wäre nicht jedes Individuum in seinem Nutzenmaximum, würde kein Gleichgewicht bestehen und die Besitzverhältnisse würden sich ändern. Mit Blick auf die Realität kann dies aber nur eine zynische Aussage sein.

In einem Bereich wurden Nutzenfunktionen noch später neu definiert. Ohne es zu erwähnen handelte es sich bisher stets um die Abbildung von Nutzen unter Sicherheit. Das heißt, wenn ich Gut A gegen Gut B tausche, weiß ich, dass ich tatsächlich das mir bekannte Gut B in gewohnter Qualität erhalte. In einige Bereichen, vor allem in der Finanzwissenschaft, wurde nach dem Zweiten Weltkrieg eine Nutzentheorie unter Risiko von nöten. Hier reicht eine ordinale Nutzenbetrachtung nicht aus. \textcite{VonNeumann1944} zeigten allerdings, dass sich unter gewissen Voraussetzungen aus einer Lotterie ein kardinaler Nutzen ableiten lässt. Dies wird in Kapitel \ref{Erwartungsnutzen} detailliert dargestellt.

\section{Fisher and Clark: Economics goes USA}

Die Volkswirtschaftslehre wird heute - mehr noch als andere Disziplinen - von US-amerikanischen Beiträgen dominiert. Vor allem mit dem Beginn des Aufstiegs des Faschismus in Kontinentaleuropa und durch den Zweiten Weltkrieg erfuhr die Verlagerung der Wirtschaftswissenschaften von Europa in den Angel-sächsischen Raum Auftrieb. Was die Zeit vor 1900 angeht, ist es aber doch überraschend, dass in den USA praktisch keinerlei, aus heutiger Sicht, bedeutende Beiträge entstanden. Vor allem wenn man beachtet, dass die heute führenden Journale schon deutlich vor 1900 in den USA gegründet wurden:  Quarterly Journal of Economics 1886, das Journal of Political Economy 1892. Die American Economic Association wurde 1885 gegründet und publiziert seit 1911 den American Economic Review. Tatsächlich waren die - wiederum aus heutiger Sicht - ersten bedeutenden amerikanischen Ökonomen John Bates Clark und Irving Fisher.

Clark ist heute vor allem durch die John-Bates-Clark-Medaille bekannt, die mittlerweile jedes Jahr - bis 2009 war der Vergaberhythmus zwei Jahre - an einen herausragenden, in den USA tätigen Ökonomen oder eine Ökonomin unter 40 Jahren vergeben wird. Er wurde 1946 in den USA geboren und ging nach dem Abschluss seines Studiums nach Europa, wo er allerdings vor allem mit sozialistischen Ideen und der Historischen Schule in Kontakt kam. So war Karl Knies, ein Vertreter letztgenannter Schule, einer seiner Professoren, der ihn laut \textcite{Tobin1985} auch stark beeinflusste. Nach seiner Rückkehr in die USA vertrat Clark dann auch Kapitalismus-kritische Positionen. In \textcite{Clark1886} kritisierte er die Klassiker um Ricardo und auch deren Individualismus und die Verherrlichung des Wettbewerbsgedanken \parencite[S. 29]{Tobin1985}. Erst in weiterer Folge änderte er seine Ansichten um 180 Grad und wurde zu einem der führenden Neoklassiker jener Zeit. Dies ist bemerkenswert, da er zu diesem Zeitpunkt schon an die 50 Jahre alt war. Bekannt wurde er schließlich durch die "`Grenzproduktivitätstheorie der Einkommensverteilung"'. Diese stellte er bereits in \textcite{Clark1891} weitgehend dar, bekannt wurde aber vor allem sein Hauptwerk \textcite{Clark1899}: "`The Distribution of Wealth"'. Die frühen Neoklassiker hatte mit ihrer Nutzentheorie vor allem die Nachfrageseite behandelt. Die Angebotsseite blieb auch bei \textcite{Marshall1890} unzureichend behandelt: Zwar gab es bei ihm schon den abnehmenden Grenzertrag bei der Produktion, allerdings noch keine Aussagen zur Verteilung der Produktionsfaktoren. Auch in der allgemeinen Gleichgewichtstheorie, bei Walras und auch bei Pareto war die Angebotsseite - also die Produktion - noch unbefriedigend dargestellt. Dies änderte sich mit dem Beitrag von John Bates Clark. Der aufmerksame Leser wird bemerken, dass auch einer der Vorläufer der Neoklassik vor allem die Produktionsseite betrachtet hat, nämlich Johann Heinrich von Thünen (vgl. Kapitel \ref{Vorläufer}). Clark war sich dessen bewusst und ging auch auf die Arbeit von Thünen ein, erkannte darin aber so manche Unzulänglichkeit \parencite[S. 31]{Tobin1985}. Zur Theorie der Grenzproduktivität selbst: Clark argumentierte, dass ein Unternehmen nur solange zusätzliche Arbeitskräfte einstellen wird, bis die letzte eingestellte Arbeitskraft einen zusätzlichen Gewinn liefert. Im Optimum entspricht der Grenzprodukt der Arbeit also der Lohnhöhe. Gleiches gilt für den zweiten Produktionsfaktor, das Kapital. Das Grenzprodukt des Kapitals entspricht den Kapitalkosten (Zinssatz)- 

Clark schuf damit im Wesentlichen für die Produktionsseite ein Äquivalent zur "`Grenzrate der Substitution"', die wir gerade bei Pareto kennen gelernt haben: Die Produktionsfaktoren werden solange gegeneinander ausgetauscht, bis sie alle das gleiche Grenzprodukt liefern. Das heißt ein weiterer Tausch nicht mehr zu höherem Gewinn führen. Dies erklärt also primär wie bei gegebenen Faktorpreisen die Produktionsfaktoren im optimalen Verhältnis zueinander eingesetzt werden. Das Ergebnis ist auch deshalb so interessant, weil als Nebenprodukt daraus die funktionale Einkommensverteilung erklärt wird.

Die funktionale Einkommensverteilung zeigt, welchen Anteil des Nationaleinkommens der Faktor Arbeit im Form von Löhnen und welchen Anteil der Faktor Kapital im Form von Unternehmensgewinnen erhält. In der klassischen Ökonomie ging man davon aus - was zur damaligen Zeit auch noch eine realistische Annahme war -, dass Arbeitnehmer ausschließlich Löhne als Einkünfte generieren, während Arbeitgeber ausschließlich Gewinne als Einkünfte erhalten. Die Diskussion um die funktionale Einkommensverteilung war seit jeher eine Diskussion um Fairness und Gerechtigkeit und eine der zentralen Streitfragen zwischen Klassischen Ökonomen und den Marxisten (vgl. Kapitel \ref{Klassik} und Kapitel \ref{Marx}). Die gerade vorgestellt neoklassische Grenzproduktivitätstheorie der Verteilung war nun insofern auch bahnbrechend anders, als sie die funktionale Einkommensverteilung als reines Ergebnis der Marktkräfte darstellt und nicht mehr - wie in der Klassik üblich - als Ergebnis politischer und wirtschaftlicher Marktverhältnisse. Im Vorwort von \textcite{Clark1899} heißt es dementsprechend auch: "`Die Einkommensverteilung in der Gesellschaft wird durch ein Naturgesetz bestimmt, wenn [der Markt] ohne Friktionen arbeiten kann, erhält jeder Agent im Produktionsprozess jenen Anteil am Wohlstand, den er selbst kreiert."' Clark war sich aber sehr wohl bewusst, dass die tatsächliche funktionale Einkommensverteilung ungerecht sein könnte, nämlich dann wenn die Voraussetzung der vollkommenen Konkurrenz auf Märkten nicht erfüllt ist. Er setzte sich daher stets dafür ein, dass der Staat primär dafür sorgen muss, dass es zu keiner Monopolbildung auf Märkten kommt \parencite{Clark1907}. Der Ansatz, dass vor allem mangelnde Konkurrenzverhältnisse zu schlechten Verteilungsergebnisse führen, spielt auch heute noch eine gewichtige Rolle und wird unter anderem von Philippe Aghion vertreten.

Man darf nicht vergessen, dass diese Grenzproduktivitätstheorie der Verteilung die selben strengen Annahmen voraussetzt, wie die bereits bekannten neoklassischen Ansätze: Neben den perfekten Konkurrenzmärkten auch die Homogenität der Produktionsfaktoren, sowie deren perfekte Substituierbarkeit. Das heißt insbesondere, das die einzelnen Arbeitnehmer gleich effizient sind und sowohl untereinander als auch gegen Kapitalgüter jederzeit austauschbar sind. Zudem werden konstante Skalenerträge angenommen. Das heißt eine gleichzeitige Verdoppelung aller Produktionsfaktoren führt zu einer Verdoppelung des Outputs. Gleichzeitig nahm \textcite{Clark1899} abnehmende Grenzerträge an - also unterproportional steigenden Output, wenn nur ein Produktionsfaktor steigt. Gerade den letzten Punkt nahm Clark als ein "`Naturgesetz"' hin, ohne einen Beweis für die Richtigkeit zu liefern, wie \textcite[S. 32]{Tobin1985} kritisiert. Diese mikroökonomische Theorie auf die gesamtwirtschaftliche Kennzahl "`funktionale Einkommensverteilung"' überzuführen ist dementsprechend problematisch. Die gerade aufgezeigten Voraussetzungen kann man wahrscheinlich für einzelne Branchen rechtfertigen, aber wohl kaum für eine gesamte Ökonomie. Dies wurde dann in den 1960er Jahren in der "`Kapitaltheoretischen Kontroverse"' auch bekannt als "`War of the two Cambridges"' thematisiert (vgl. Kapitel \ref{Post-Keynes}).

Insgesamt wurde das Werk von Clark kontrovers diskutiert. Seine Zeitgenossen in den USA, allen voran \textcite{Veblen1909} lehnten die Neoklassische Schule noch lange Zeit ab \parencite[S. 97]{Persky2000}. Die Grenzproduktivitätstheorie wird heute fast immer \textcite{Clark1899} zugewiesen, obwohl Knut Wicksell und Philip Wicksteed diese - unabhängig von ihm - ebenso entwickelt hatten. \textcite{Clark1891, Clark1899} stellte seine Ausführungen gänzlich ohne mathematische Formeln dar, was für Neoklassiker sehr untypisch ist. \textcite{Tobin1985} kritisiert, dass Clark eine ähnliche und sehr ausführliche Arbeit von Stuart Wood (1889), die ihm bekannt sein musste, gänzlich ignorierte. Die Grenzproduktivitätstheorie stieß aber auch die Türe auf für viele wichtige Forschungslinien des 20. Jahrhunderts. Die Produktionsfunktion von Cobb und Douglas der 1920er Jahre schloss in neoklassischer Tradition an Clark's Grenzproduktivitätsmodell an (vgl. Kapitel \ref{sec: Cobb-Douglas-Produktionsfunktion}). Die späteren  Wachstumsmodelle von Harrod und Domar auf keynesianische Seite, bzw. das Solow-Wachstumsmodell in neoklassischer Tradition, waren - wie \textcite{Tobin1985} schreibt - die selben "`Ein Produkt, zwei Faktoren"'-Modelle wie jenes von Clark.

Wenn Clark heute als der erste bedeutende amerikanische Ökonom angesehen wird, dann gilt aus heutiger Sicht Irving Fisher als der bedeutendste amerikanische Ökonom des frühen 20. Jahrhunderts. Fisher wurde zwanzig Jahre später als Clark geboren. Obwohl Clark erst recht spät, nämlich wie beschrieben mit über 50 Jahren sein Hauptwerk vorlag, stand Fisher zu dessen Lebzeiten eher im Schatten von Clark. Er hat von seinen Zeitgenossen weit weniger Beachtung erlangt, als nach seinem Ableben. In historischen Beiträgen wird lebhaft darüber spekuliert, warum Fisher - der heutzutage als moderner und bahnbrechender Ökonom angesehen wird - von seinen Zeitgenossen kaum als solcher wahrgenommen wird. Einzig \textcite[S. 872]{Schumpeter1954} prognostizierte, dass "`Zukünftige Historiker Fisher als größten aller bisherigen Amerikanischen Ökonomen"' feiern werden. Derselbe nannte auch zwei Gründe für die erst später aufkommende Bewunderung von Fisher's Werken: Erstens, seine "`verrückten"' Ideen - dazu gleich mehr und zweitens, seine mathematischen Ansätze, für damalige Zeit in den USA noch immer die absolute Ausnahme.

Was sicherlich stimmt, ist, dass Fisher - ebenso wie Marshall - als einer der ersten Wirtschaftswissenschaftler moderner Prägung angesehen werden kann und zwar in vielerlei Hinsicht. Zum einen verwendete er eine klare, einwandfreie Mathematik gepaart mit eindeutigen sprachlichen Formulierungen. Wie später in Kapitel \ref{FisherundKnight} dargestellt, gelten Eugen von Böhm-Bawerk und Irving Fisher als Vorläufer der modernen Finanzierungstheorie. Tatsächlich liefern deren Arbeiten ähnliche Inhalte. Vergleich man aber \textcite{BohmBawerk1888} und \textcite{Fisher1930} direkt miteinander, merkt man sofort die Überlegenheit und methodische Modernität des Letztgenannten \parencite[S. 33 ]{Tobin1985}. Er war aber auch einer der ersten Ökonomen, der nicht nur theoretische Ansätze schuf, sondern diese auch empirisch überprüfte. Dazu entwickelte er auch eigens statistische Verfahren. Außerdem war er nicht nur als Wissenschaftler und Universitätsprofessor tätig, sondern er vermarktete seien Ideen auch als Unternehmer und Erfinder mit mehreren Dutzend Arbeitnehmern und Büros in New York und Washington \parencite[S. 215]{Monissen1989}. Heute würde man ihn wohl als (ersten) Begründer eines "`Think Tanks"' bezeichnen. Seine Geschäftstätigkeit führte aber auch zu einem prägenden Rückschlag für Fisher: Noch Mitte Oktober 1929 beteuerte er, dass die "`Aktienmärkte ein permanent hohes Niveau"' erreicht hätten \parencite[S. 11]{Dimand2005}. Dieses Zitat sollte ihm lang anhaften. Auch in den Monaten nach dem "`Schwarzen Donnerstag"' am 24. Oktober 1929 und trotz der fortlaufenden Kursverlusten, beteuerte Fisher stets seinen Glauben an eine rasche Erholung der Märkte. Fisher's eigene Vermögensverluste bezifferte sein Sohn später auf "`acht bis zehn Millionen Dollar"' \parencite[S. 216]{Monissen1989}. \textcite[S. 30]{Tobin1985} schreibt sogar davon, dass die Universität Yale - an der er sowohl studierte, als auch später lehrte, "`sein Haus für ihn retten"' musste. 
Sein grandioser Fehlschlag bei der Einschätzung der "`Great Depression"' war eine jener "`Spinnereien"', die \textcite{Schumpeter1954} meinte und Fisher's öffentlichem Ansehen als Wissenschaftler stark schadete. Außerdem wurde er von manchen dadurch als Geschäftsmann und "`skrupelloser Spekulant diskreditiert"' \parencite[S. 216]{Monissen1989}. Fisher war aber auch in mancherlei anderer Hinsicht speziell. Mit 30 Jahren erkrankte er an Tuberkulose, überwand diese allerdings durch einen dreijährigen Kuraufenthalt. Danach beschäftigte und engagierte er sich für öffentliche Gesundheit. Er lehnte jegliche Genussmittel ab und war Vegetarier, aber auch überzeugter Eugeniker. Die Prohibition unterstützte er nicht nur, er verfasst auch Arbeiten, die deren Sinnhaftigkeit im Sinne daraus resultierender höherer Produktivität zeigen sollten. Später stellte man darin grobe Datenmängel fest \parencite[S. 215]{Monissen1989}. Außerdem unterstützte er die Idee des "`Schwundgeldes"', also Geld, das mit fortlaufender Zeit an Wert verlieren sollte, sowie 100\% gedecktes Geld, was die Geldschöpfung durch Geschäftsbanken unmöglich machen sollte \parencite[S. 30]{Tobin1985}.

Sein Erklärungsansatz zur "`Great Depression"', die "`Debt-Deflation-Theory"' \parencite{Fisher1933} reihte sich ein in die zahlreichen Krisenerklärungstheorien, die nach der "`Great Deprssion"' publiziert wurden. Der Ansatz ist durchaus interessant und wurde im zeitlichen Umfeld der "`Great Recession"' nach 2007 wieder verstärkt zitiert. \textcite{Fisher1933} argumentiert darin, dass durch die Deflation während der Krisenjahre Anfang der 1930er Jahre die realen  Schuldenwerte deutlich stiegen. Die Verbindlichkeiten von Schuldnern nahmen durch die Deflation also selbst bei aufrechter Rückzahlung real zu. Das führte zu verstärkten Ausfällen und verstärkte in weiterer Folge Kreditzinsen, Vertrauensverlust und schlussendlich die gesamtwirtschaftliche Entwicklung. Der Zeitgeist - in Form des Aufstiegs des Keynesianismus - bescherte der "`Debt-Deflation-Theorie"' aber ein Dasein als eher wenig beachtetes Werk.

Kommen wir endlich zu den erfolgreichen wissenschaftlichen Beiträgen Fisher's. Seine bewundernswerte, enorme Schaffenskraft wurde schon angedeutet, und tatsächlich war Fisher in verschiedenen Bereichen tätig. Sein Beitrag zur Nutzenmessung, den er mit seiner Dissertation \parencite{Fisher1892} vorlag, wurde schon erwähnt. Heute gilt vor allem seine Arbeiten zur Kapitaltheorie als wichtigster Beitrag. Dazu legte \textcite{Fisher1907} "`The Rate of Interest"' vor. Die Erweiterung \textcite{Fisher1930} "`The Theory of Interest"' wird noch heute zitiert. Diese Arbeiten begründen die "`intertemporale Konsumentscheidung"', sowie das sogenannten Fisher-Separations-Theorem. Beide Theoreme stellen wichtige Grundlagen bei der Betrachtung der modernen Finanzierungstheorie dar und werden in Kapitel \ref{FisherundKnight} detaillierter dargestellt.

Zu seinen Lebzeiten galt Fisher's Arbeit zur Quantitätstheorie des Geldes \textcite{Fisher1911} als sein Hauptwerk. Darin behandelt er diese in jener Form, die noch bis heute üblich ist. Sie wird uns noch des öfteren unterkommen, weil sie bis in die Gegenwart eine Rolle spielt. Die Quantitätsgleichung des Geldes wurde dabei schon viel früher sinngemäß formuliert und zwar bereits als erste Erklärungstheorie von Inflation. Im 16. Jahrhundert wurde in Spanien und Frankreich ein Preisanstieg bei praktische allen verfügbaren Waren beobachtet. Geld bestand damals in Europa noch fast ausschließlich aus Gold- oder Silbermünzen. Es wurde als reines Tauschmittel angesehen, dessen Wert als konstant angenommen wurde. Die erste Erklärung für einen Wertverlust bei Münzen war, dass der Edelmetall-Anteil der Münzen abgenommen hatte. Dieses Phänomen, dass Münzen bewusst verschlechtert wurden, indem man deren Edelmetall-Anteil senkte, war bereits damals bekannt. Der französische Gelehrte Jean Bodin erwiderte dieser Theorie im Jahr 1568, dass nicht die Münzverschlechterung an den Preissteigerungen Schuld war, sondern der hohe Import von Edelmetallen aus Amerika durch die Spanier \parencite{OBrien2000}. Es verschlechterte sich also nicht das Geld, sondern es erhöhte sich die Geldmenge. Der uns bereits bekannte Klassiker David Hume (vgl. \ref{Klassik}) formulierte schließlich die erste ausdrückliche Form der Quantitätsgleichung. Aber es war \textcite{Fisher1911} der die \textit{moderne} Form der Quantitätsgleichung formulierte und daraus eine Quantitäts\textit{theorie} des Geldes machte. Geld spielte bei den Klassikern nur als Tauschmittel eine Rolle und auch die frühen Neoklassiker hatten Geld nicht jene Bedeutung zugestanden, die es in der modernen Wirtschaftswissenschaft hat. Kurzum: Die Geldtheorie war am Anfang des 20. Jahrhunderts noch wenig entwickelt \parencite[S. 32]{ Tobin2005}. Weitgehend etabliert und unumstritten war Anfang des 20. Jahrhunderts die Bindung von Gold an ein Edelmetall - konkret der "`klassische Goldstandard"', der zwischen 1873 und 1914 gut funktionierte. Die Quantitätsgleichung des Geldes kann in diesem Sinn einfach als nicht widerlegbare Geldtheorie interpretiert werden. Die Formel wird bis heute als $$M*V = P*T$$ dargestellt \footnote{Die Urheberschaft dieser Formel ist häufig diskutiert. Inwieweit die Aussagen von Bodin und Hume schon als äquivalent zur Formel betrachtet werden können ist umstritten. Bekannt ist auf jeden Fall, dass Simon Newcomb, ein amerikanischer Astronom und Ökonom, die Formel bereits ausformuliert hatte \parencite[S. 33]{Tobin2005}.}. Die Geldmenge$M$ mal der Umlaufgeschwindigkeit$V$, also die Häufigkeit mit der Geld den Besitzer wechselt, entspricht dem Preisniveau$P$ mal der Summe aller Transaktionen$T$, heute könnte man dies als Bruttoinlandsprodukt bezeichnen. Dazu ein einfaches Beispiel: Stellen sie sich eine beliebige Ware vor. Angenommen die Geldmenge$M$ würde sich verdoppeln. Und zwar in dem Sinn, dass alle Geldbestände plötzlich verdoppelt wären. Wie würde sich dann der Preis$P_i$ dieser "`beliebigen Ware"' (also auch aller Waren$P$) entwickeln? Rein intuitiv antwortet jeder, dass sich deren Preis ebenso plötzlich verdoppeln würde. Man spricht in diesem Fall von "`klassischer Dichotomie"'. Die Mengenänderung des Geldes hat keine reale Auswirkung auf die Wirtschaft. Es ändert sich einfach das Preisniveau. Die Geldtheorie Fisher's geht aber über die Simple Identität im Sinne der Quantitätsgleichung hinaus. Er analysierte die Preise und Mengen der einzelnen Waren und wie diese gewichtet werden müssten um eine realistische Darstellung zu erlangen. Seine bekannten Vorarbeiten zu Indexzahlen und statistischen Methoden fanden hierbei Anwendung. Seine Fragestellungen zur Quantitätgleichung, die diese schließlich zur Quantitätstheorie machten, sind teilweise bis heute nicht unumstritten. So kann die Gesamt-Geldmenge einer Ökonomie durch eine zentrale Institution - normalerweise eine Zentralbank - nur dann direkt gesteuert werden, wenn Geld zu 100\% gedeckt ist. Dies ist eben in modernen Ökonomien nicht der Fall. Geschäftsbanken können durch Kreditvergabe neues Geld erschaffen. Erhöht diese Giralgeldschöpfung nun die Geldmenge, oder wird dadurch einfach die Umlaufgeschwindigkeit erhöht? Fisher plädiert für zweiteres. Die Giralgeldschöpfung erkennt Fisher als problematisch für die Anwendbarkeit der Quantitätsgleichung. Dadurch wird das Geldangebot exogen durch Institutionen - nämlich Geschäftsbanken - gesteuert, wenn diese nicht ihren Rahmen zur Giralgeldschöpfung schon vollkommen ausgenutzt haben \parencite[S. 35]{Tobin1985}. Interessant ist aber, dass ausgerechnet Fisher, für dessen Werk Zinsen \parencite{Fisher1906, Fisher1930} von zentraler Bedeutung waren, deren Bedeutung bei in der Interpretation der Quantitätstheorie völlig vernachlässigte \parencite{Tobin2005}. Schließlich können bestimmte Wertpapiere zum Preis eines gewissen Zinsaufwandes, wie Geldmittel verwendet werden. Dies wirkt sich auf die Geldmenge aus. So gilt es heute als umstritten, welches der Geldmengenaggregate (M0, M1, M2, M3)\footnote{Die Geldmengenaggregate bestimmen welche Wertpapiere als Geld bezeichnet werden. Die Geldmenge M0 zum Beispiel besteht ausschließlich aus gedrucktem Geld und Münzen, sowie Geldbeständen der Geschäftsbanken bei der Zentralbank. M3 umfasst auch bestimmte Wertpapiere, die innerhalb einer bestimmten Frist zu Geld gemacht, also als Zahlungsmittel dienen, können.} in der Quantitätsgleichung herangezogen werden soll. 
Was ist nun die \textit{Theorie} in \textcite{Fisher1911}? Er leitet aus der Quantitätsgleichung eine Zyklentheorie ab. Eine steigende Geldmenge würde demnach schon in der kurzen Frist zu steigenden Preisen, aber noch schneller steigenden Unternehmensgewinnen führen \parencite[S. 59]{Fisher1911}. In der Folge investieren die Unternehmen mehr und die Banken vergeben mehr Kredite, bis sich herausstellt, dass es zu einer Abweichung vom Gleichgewicht kommt und der Wirtschaftszyklus sich ins negative dreht. Die Auslöser werden bei Fisher nicht weiter hinterfragt, sondern "`ad hoc"' angenommen. Auch spielen, wie bereits erwähnt, die Zinssätze keine Rolle. Insgesamt wirkt das Ganze aus heutiger Sicht recht konstruiert. Aber diese Geldtheorie im Sinne der Quantitätstheorie, sowie die daraus abgeleitete Zyklentheorie wird als eine der ersten makroökonomischen Theorien angesehen\footnote{Ob die Theorien zur Quantitätstheorie des Geldes des frühen 20. Jahrhunderts, oder doch erst die Publikation von Keynes' "`General Theory"' als Geburtsstunde der modernen Makroökonomie angesehen werden sollen, wird im nächsten Unterkapitel kurz andiskutiert.}. Fisher wird damit häufig als Vorläufer zum Monetarismus (vgl. Kapitel \ref{Monetarismus}) bezeichnet. Allerdings leitet er weit weniger stark wirtschaftspolitische Empfehlungen aus seiner Quantitätsgleichung ab als später Milton Friedman.

Der Lebensweg des überaus talentierten Irving Fishers ist in so vielen Facetten interessant. Vor allem \textcite{Dimand2005} übertreibt es aber wohl etwas bei seiner Lobhymne auf Fisher. Demnach könnte man in seinen Arbeiten praktisch alle bahnbrechenden Konzepte der Ökonomie bei Fisher finden: Die "`Phillips-Kurve"' und die Bedeutung der Geldpolitik während der "`Great Depression"' \parencite[S. 7]{Dimand2005}. Überhaupt den ganzen Monetarismus, sowie eigentlich vieles der modernen Makroökonomie  \parencite[S. 10]{Dimand2005}. Selbst sein großes persönliches Waterloo, die Fehleinschätzung der Entwicklung der Kapitalmärkte Ende der 1920er Jahre wird von \textcite[S. 9]{Dimand2005} als Entdeckung des "`Equity Premium Puzzles"' - ein Paradoxon bei dem es darum geht, dass die Differenz zwischen risikolosen und risikobehafteten Renditen zu unterschiedlich ist - interpretiert. Etwas realistischer ist wohl die Einschätzung, dass er vor allem mit seinen Arbeiten zu Kapitaltheorie, insbesondere zur intertemporalen Konsumentscheidungen und zur Zinstheorie, tatsächlich bahnbrechende Beiträge lieferte. Dass diese bis heute häufig zitiert werden, zeugt davon, dass er seiner Zeit weit voraus war. Bemerkenswert ist insbesondere sein moderner Stil, der sich kaum vom heute üblichen unterscheidet und jenem seiner Zeitgenossen deutlich überlegen ist. Was seine Quantitätstheorie des Geldes angeht hatte er mit Knut Wicksell einen intellektuellen Gegenspieler, für den Fisher's Quantitätstheorie viel zu einfach formuliert war. Wicksell's Ideen werden im nächsten Kapitel behandelt.

\section{Wicksell: Eine überraschend moderne Theorie}

Johan Gustav Knut Wicksell wird allgemein als Begründer der "`Schwedischen Schule"' - häufig auch "`Stockholmer Schule"' genannt - angesehen. Deren frühe Vertreter, wie eben Wicksell, werden oftmals auch der "`Österreichischen Schule"' zugerechnet. Dies scheint im Hinblick auf die gleich dargestellte Bedeutung der Zinsen zwar stimmig, insgesamt waren die abgeleiteten wirtschaftspolitischen Empfehlungen der Österreicher doch wesentlich andere als jene der Schwedischen Schule. So war Knut Wicksell ein überzeugter Sozialdemokrat und er wurde als "`politisch links"' bezeichnet \parencite[S. 210, S. 192]{Grossekettler1989}. Für die Vertreter der Österreichischen Schule wäre es eine Beleidigung als politisch links zu gelten. Seinen neoklassischen Arbeiten fügte er hinzu, dass zwar die Einkommensverteilung in der neoklassischen Theorie vom Markt bestimmt sei, dass man dies aber in der Realität nicht akzeptieren müsse und durch Vermögens- und Einkommenssteuern eine gerechtere Verteilung anstreben sollte \parencite[S. 196]{Grossekettler1989}.In gewisser Hinsicht dürfte Wicksell recht radikal gewesen sein, so wurde er nach einem gotteslästerlichen Vortrag zu zwei Monaten Gefängnis verurteilt. Verheiratet war er mit Anna Bugge, die eine der ersten Frauen in bedeutenden Positionen in Schweden war und Kopf der dortigen Emanzipationsbewegung war \parencite[S. 191]{Grossekettler1989} Interessant ist, dass er seine Arbeiten primär in deutscher Sprache veröffentlichte. Seine Arbeiten im Rahmen der Neoklassik wurden "`wiederentdeckt"', weniger bekannt sind seine Werke als Vorläufer der Neuen Politischen Ökonomie (vgl. Kapitel \ref{Pol_Econ}), die erst durch deren Hauptvertreter James M. Buchanan analysiert wurden. 

Wicksell wird häufig als Gegenspieler von Fisher im Hinblick auf die Quantitätstheorie bezeichnet. Dabei war Wicksell d'accord mit deren Grundannahmen. Allerdings geht Wicksell eben einen Schritt weiter und berücksichtigt Zinsen in seiner Form der Konjunkturtheorie (Zyklentheorie). Konkret schafft er das Konzept des "`natürlichen Zinssatzes"', das noch heute verwendet und diskutiert wird. Heute gibt es leider viele moderne Abhandlungen von selbsternannten "`Ökonomen"', die das Konzept des Zinses als Ganzes in Frage stellen. Die Herleitung des natürlichen Zinssatzes von \textcite{Wicksel1898} - die er vor über 100 Jahren verfasst hat - ist im Vergleich dazu ein wahrer Wohlgenuss. Auch wenn diese Herleitung, nicht sehr präzise und formal eindeutig ist. Zudem ist sie deutlich zu vereinfacht und verwirrend dargestellt\footnote{Im Rahmen der bereits erwähnten "`Kapitaltheoretischen Kontroverse"' wurde festgehalten, dass aus einzelwirtschaftlichen Analysen nicht auf gesamtwirtschaftliche Zinsen schließen kann ("`Wicksell Effekt"')  (vgl. Kapitel \ref{Post-Keynes})} \parencite[S. 639]{Blaug1962}: Vom Sozialprodukt zieht man die Löhne, Unternehmerlöhne und Abgeltungen für die Bodenbesitzer ab. Der Rest ist die Vermehrung des Kapitals, davon muss man noch die Abschreibungen abziehen - also welcher Teil davon muss investiert werden um den Kapitalstock zu erhalten - und man kommt auf die Verzinsung des Kapitals (\textcite[S. 113ff]{Wicksel1898} nach \textcite[S. 407]{Rosner2012}). Dies ist die natürliche Verzinsung des realen Kapitals (natürlicher Zinssatz). Daneben betrachtete \textcite{Wicksel1898} aber zusätzlich den Finanzmarkt. Auf diesem wird ebenfalls ein Zinssatz gebildet, man könnte diesen den Marktzins nennen.
Diese beiden Zinssätze wirken auf unterschiedlichen Märkten. Bekannt war vor Wicksell bereits, dass höhere Zinsen die Produktionskosten erhöhen, weil der Faktor Kapital teurer wird. Aber es gibt auch die Wirkung der Zinsen über die Finanzmärkte. \textcite[S. 73f]{Wicksel1898} beschreibt, dass niedrigere Zinsen zu höherer Kreditnachfrage und in weiterer Folge höherer Güternachfrage und steigender Inflation führen. Damit schafft es Wicksell die Geldtheorie in die Konjunkturtheorie, die zuvor ausschließlich den Gütermarkt betrachtete, einzubinden \parencite[S. 5]{Blanchard2000}. Oder mit anderen Worten er schafft es die Kapitaltheorie mit der Grenzproduktivitätstheorie der Einkommensverteilung zu verbinden. 

Demnach ist die Gesamtwirtschaft im Gleichgewicht, wenn die Kapitalgeber (Unternehmer) tatsächlich mit dem natürlichen Zins entlohnt werden. Ist der Marktzins niedriger als der natürliche Zinssatz, so werden die Unternehmen die Produktion ausweiten und in weiterer Folge werden die Preise steigen. Ein Marktzins, der über dem natürlichen Zinssatz liegt führt zu sinkenden Preisen. Zinserhöhungen (Zinssenkungen) wirken also bereits bei Wicksell erst dann konjunkturdämpfend (konjunkturfördernd), wenn der Marktzinssatz höher (niedriger) als der natürliche Zinssatz ist\footnote{Zinserhöhungen (Zinssenkungen) isoliert betrachtet haben also noch keine konjunkturdämpfende (konjunkturfördernde) Wirkung. Dieser Effekt wirkt nur im Verhältnis zum natürlichen Zinssatz.}. Wicksell hat diese Konjunkturtheorie übrigens nicht abschließend in \textcite{Wicksel1898} dargestellt, sondern laufend erweitert, sodass man deren Gesamtausmaß nur aus der Literatur seines Hauptwerkes und \textcite{Wicksell1922}: "`Vorlesungen über Nationalökonomie, Band 2: Geld und Kredit"' vollständig erfasst werden kann. 

Der Ansatz wirkt gerade heute ungemein modern. Wo doch die meisten Zentralbanken dazu übergegangen sind die Konjunktursteuerung mittels Inflation-Targeting vorzunehmen und dafür mutmaßlich einer "`Taylor-Rule"' folgen (vgl. Kapitel \ref{Taylor}), die genau das Verhältnis zwischen Marktzins, natürlichen Zins, neben erwarteter und tatsächlicher Inflation beinhaltet. Auch die Überinvestitionstheorie von Hayek (vgl. Kapitel \ref{Austria}) zur Erklärung der "`Great Depression"' ähnelt dem Konzept stark \parencite[S. 201]{Grossekettler1989}. Man darf aber nicht vergessen, dass Wicksell's Arbeit "`Geldzins und Güterpreise"' bereits 1898 publiziert wurde. Also noch lange vor der "`Great Depression"' (1929 - 1933) und auch lange vor Keynes' "`General Theorie"' (1936), die schließlich das Konzept der aggregierten Nachfrage in den Vordergrund rückte (vgl. Kapitel \ref{Keynes}). Dementsprechend geht Wicksell auch nicht weiter auf die Nachfrage-seitigen Effekte der Zinsänderungen ein, bzw. auf die Möglichkeit, dass der Marktzinssatz über längere Zeit unter dem natürlichen Zinssatz liegen könnte und dementsprechend in größeren Wirtschaftskrisen münden könnte.

Interessant ist der Vergleich zwischen Fisher's Ansatz und jenem von Wicksell aus moderner makroökonomischer Sicht. Während die Quantitätstheorie von den Monetaristen um Milton Friedman wiederbelebt wurde (vgl. Kapitel \ref{Monetarismus}), erlebten die Ideen von Wicksell in der "`Neuen neoklassischen Synthese"' (vgl. Kapitel \ref{Neue Neoklassische Synthese}) eine Renaissance. Einer deren Hauptvertreter, Michael Woodford, geht sogar soweit, diese Schule als "`Neo-Wicksellianische Schule"' zu bezeichnen. Dies ist auch der Grund für die "`Wiederentdeckung"' Wicksell's Ende des 20. Jahrhunderts. Mit der aufkommen der DSGE-Modelle und insbesondere der "`Zinssteuerung"' durch Zentralbanken (vgl. Kapitel \ref{Neue Neoklassische Synthese}) wurden die frühen Konzepte vom Wicksell wieder verstärkt beachtet.

Von Mark Blaug, einem der profiliertesten Wirtschaftshistoriker des 20. Jahrhunderts, wurde Wicksell sogar als der Begründer der modernen Makroökonomie angesehen \parencite[S. 274]{Blaug1986}. Die meisten führenden Lehrbücher (\textcite[S. 13]{Snowdon2005}, \textcite[S. 795]{Blanchard2003}, \textcite[S. 29]{Samuelson1998}) sprechen diese Rolle allerdings uneingeschränkt John Maynard Keynes zu

In diesem Buch wird dem pragmatische Ansatz von \textcite{Blanchard2000} gefolgt, wenn es darum geht, wann den nun die Makroökonomie entstanden ist: Vor 1936 gab es natürlich schon makroökonomische Ansätze. Abgesehen von früheren Arbeiten durch Ricardo, Marx und anderen, waren Wicksell und Fisher die zwei dominierenden Köpfe in diesem Bereich. Insgesamt gab es vor allem zur Geldtheorie in Form der Interpretation der Quantitätstheorie, aber auch zur Zyklentheorie, Forschungsansätze, die aber kein einheitliches Rahmenwerk darstellten \parencite[S. 2f]{Blanchard2000}. \textcite[S. 1]{Blanchard2000} nennt diese Phase: "`Eine Periode der Entdeckungen, in der Makroökonomie noch nicht Makroökonomie war."' Deren Geburt folgte eben erst 1936 mit der Veröffentlichung von Keynes' "`General Theory"'. Auf dieses bahnbrechende Werk kommen wir in Kürze, nämlich in Kapitel \ref{Keynes} zu sprechen. 
	% Neoklassik von Marshall bis Wicksell      !!! KORREKTURLESEN


%%%%%%%%%%%%%%%%%%%%%%%% part.tex %%%%%%%%%%%%%%%%%%%%%%%%%%%%%%%%%%
%
% sample part title
%
% Use this file as a template for your own input.
%
%%%%%%%%%%%%%%%%%%%%%%%% Springer-Verlag %%%%%%%%%%%%%%%%%%%%%%%%%%


\part{1936 -- 1975\\Die Geburt der Makroökonomie: Keynesianismus - Synthese - Monetarismus. Daneben wächst die Mikroökonomie aber weiter}

Die großen Ökonomen in der Mitte des 20. Jahrhunderts waren allesamt geprägt von der "`Great Depression"'. Der Weltwirtschaftskrise, die 1929 mit einem Börsenkrach an der New Yorker Börse, der \textsc{New York Stock Exchange} ihren Ausgang nahm. Es war wohl die erste Krise, die nicht auf kriegerische Auseinandersetzungen oder Umweltkatastrophen, wie Dürren oder Erdbeben, zurückzuführen war, aber dennoch für extremes Leid sorgte. Die Zahlen können nüchtern dargestellt werden: Die Wirtschaftsleistung in den (bis dorthin einzigen) Industrienationen ging um ein Viertel bis ein Drittel zurück. Die Arbeitslosenzahlen stiegen auf teilweise mehr als 25\%. Die persönlichen Schicksale dahinter kann man sich aber kaum vorstellen. Obwohl kein Krieg die Völker aufeinander losgehen ließ und obwohl keine Dürre die Felder verdorren ließ, gab es dennoch Menschen, die ihr Hab und Gut verloren und sogar Hunger litten. Es muss eine schreckliche, seltsam ruhige Zeit gewesen sein. Eine Krise, aber keinen Feind oder offensichtlich Auslöser oder Schuldigen. Dafür viele Menschen ohne Arbeit und ohne Einkommen.
Die Ökonomen hatten dafür ganz verschiedene Erklärungen:
\begin{itemize}
	\item Hayek: Überinvestition in den 1920er Jahren
	\item Keynes: Unterkonsumption in der Krise
	\item Schumpeter: Abwärtsbewegung von drei Wirtschaftszyklen gleichzeitig
	\item Friedman: Sinkende Geldmenge durch Goldstandard
	\item Fisher: Schuldenanstieg durch Deflation (Debt-Deflation)
\end{itemize}

HIER WEITER


Mit \textcite{Keynes1936} wurde nicht nur das Gebiet der Makroökonomie (neu) begründet, sondern das gesamte ökonomische Denken revolutioniert. Die Kontinentaleuropäischen Schulen - allen voran die Österreichische Schule, aber ebenso die Freiburger Schule - lehnten die Ideen von Keynes von Anfang an ab, rutschten aber in der Folge auch immer stärker an den Rand der ökonomischen Wahrnehmung und gelten heute als "`Heterodoxe Schulen"' (vgl. Teil \ref{sec: Heterodox}). Die modernere Kritik am Keynesianismus - in Form des Monetarismus (vgl. Kapitel \ref{Monetarismus}) und später in Form der Neuen klassischen Makroökonomie (vgl. Kapitel \ref{Neue Makro}) - entwickelte sich erst in den späten 1960er Jahren. 

Dennoch wäre es ein Irrtum zu glauben, dass es zwischen 1936 und 1960 nur den Keynesianismus als ökonomische Schule gab! 
Die davor entwickelten Erkenntnisse der Mikroökonomie - insbesondere der Neoklassik - wurden ja nicht per Se als falsch angesehen. Sie reichten nur nicht aus um \textit{gesamt}wirtschaftliche Phänomene zu erklären. Zwar stürzten sich viele junge Ökonomen in weiterer Folge auf die Ideen von Keynes und die Makroökonomie, aber die Mikroökonomie bestand ja als eigene Disziplin weiterhin. Nun zwar nicht mehr als die "`einzige"' ökonomische Theorie, sondern als gleichwertiger Partner neben der Makroökonomie. Dies machte sie als Forschungsgebiet aber nicht weniger wertvoll. Natürlich war die Koexistenz zwischen Makroökonomie und Mikroökonomie zu Beginn Konflikt-behaftet. Insbesondere weil in \textcite{Keynes1936} ja immer wieder auf die Fehler der Neoklassiker, hier insbesondere auf die Arbeiten von Pigou, als den damals führenden Neoklassiker, hinwies. Die Revolution der Ökonomie durch \textcite{Keynes1936} ist unumstritten. Aber die Arbeiten der späteren Neoklassiker - ausgehende von Pigou, hin zu Solow, sowie Arrow und Debreu - bildeten in weiterer Folge unumstritten das zweite Standbein der Ökonomie (vgl. Kapitel \ref{Neoklassik_nach1945}).		% Keynes - Synthese - Monetarismus
%%%%%%%%%%%%%%%%%%%%% chapter.tex %%%%%%%%%%%%%%%%%%%%%%%%%%%%%%%%%
%
% sample chapter
%
% Use this file as a template for your own input.
%
%%%%%%%%%%%%%%%%%%%%%%%% Springer-Verlag %%%%%%%%%%%%%%%%%%%%%%%%%%

\chapter{Keynesianismus: Nachfrage statt Angebot?}
\label{Keynes}

\textsc{John Maynard Keynes} war sicherlich der prägendste Ökonom des 20. Jahrhunderts. Sein Name ist über die Wirtschaftswissenschaften hinaus bekannt und wird im Begriff "`Keynesianismus"' häufig für jene ökonomische Schule verwendet, die nach 1945 für mehrere Jahrzehnte Volkswirtschaftstheorie und auch Wirtschaftspolitik prägte. Zumindest direkt nach dem Erscheinen seines Hauptwerkes - der "`General Theory"' - wurde er fast uneingeschränkt gefeiert. Heute gelten seine Theorien als umstritten. In der wirtschaftspolitischen Praxis sind jene Ansätze - die er als einer der ersten theoretisch beschrieben hat, und die bis heute nach ihm benannt werden - trotz aller formaler Bedenken weit verbreitet. Vor allem im Fall von Wirtschaftskrisen. Das Auftreten der "`Great Recession"' nach 2008 machte dies deutlich. Bis heute wird die Mainstream-Ökonomie, sowie deren Modelle, als neu-\textit{keynesianisch} bezeichnet. Und noch heute müssen sich Top-Ökonomen häufig dazu bekennen, ob sie pro oder contra Keynes sind. Tatsächlich definierten sich alle führenden makroökonomischen Wirtschaftsschulen seit 1946 auch an deren Einstellung zu den keynesianischen Ideen.
Was macht Keynes zu \textit{der} allgegenwärtigen Figur in der Volkswirtschaftslehre? Nun, er gilt als Begründer der modernen Makroökonomie. Wobei dies in dem Sinne zu verstehen ist, dass er als erster eine völlig neue Sichtweise auf die Wirtschaft als Ganzes lieferte und auch eine wirtschaftspolitische Steuerung vorschlug. Er betrachtete Wirtschaft nicht mehr als Summe der Leistungen einzelner Individuen, sondern als Gesamtsystem in dem sich Angebots- und Nachfrageseite gegenseitig bedingen. Der Wirtschaftspolitik wurde damit eine völlig neue Bedeutung gegeben. Dies alleine kann seine einzigartige Stellung aber nicht rechtfertigen. Vielmehr deutete der Zeitgeist nach der "`Great Depression"' bereits an, dass Ökonomie neu zu denken war. Keynes war in diesem Sinne eben nicht der einzige, der dies tat. Die Wirtschaftspolitik kannte die stimulierende Wirkung von Staatsausgaben schon vor 1936 \parencite{Fishback2010}. Es gab aber auch schon ähnliche \textit{theoretische} Ansätze, welche die Erkenntnisse Keynes' vorwegnahmen. Vor allem der polnische Ökonom Michal Kalecki arbeitete schon längere Zeit an einer solchen Theorie und publizierte diese \parencite{Kalecki1935} bereits Keynes auch in englischer Sprache. Diese wurde allerdings nur in Fachkreisen anerkennend aufgenommen. \textcite{Kahn1931} hatte schon deutlich zuvor den Multiplikator und dessen Wirkung vorgestellt und damit wichtige Vorarbeiten geliefert. Die beiden genießen aber bis heute nicht einmal einen Bruchteil des Ruhmes, den Keynes nach 1936 rasch erlangte. Noch dazu ist Keynes' Werk schwer verständlich und umständlich geschrieben: Ökonomen und Wirtschaftshistoriker sind sich recht einig über das Buch "`The General Theory"' selbst. \textcite [S. 31]{Mankiw2006} findet es "`erheiternd und frustrierend"' und rät Studierenden ab das Werk zu lesen. Robert Lucas antwortete auf die Frage, ob man das Werk heute noch lesen soll kurz und knapp mit "`Nein"' \parencite{Lucas2013}. \textcite[S. 429]{Rosner2012} bezeichnet den Charakter des Buches als "`eigenartig"' und selbst einer der eifrigsten Verfechter der keynesianischen Ideen Paul Samuelson findet das Werk selbst "`schlecht geschrieben, ärmlich organisiert"' und zudem "`höchst verwirrend"' \parencite[S. 190]{Samuelson1946}. Und doch zweifelt keiner der genannten an der hohen Bedeutung des Werks. Keynes' einzigartige Stellung in der Ökonomie erscheint angesichts dieser Tatsachen aus heutiger Sicht dennoch fast überraschend. Vor allem wenn man dazu noch selbst einen Blick in das Buch wirft. Seine bahnbrechende Leistung soll und darf aber auch nicht geschmälert werden. Die General Theory besticht nicht als Gesamtwerk. Die unumstrittene Genialität des Werks liegt vielmehr in den Einzelideen die dort an verschiedenen Stellen vorgebracht werden und in ihrer Summe zur damaligen Zeit revolutionär waren. Außerdem ist es aus heutiger Sicht schwierig die Tragweite der Veröffentlichung der General Theory abzuschätzen, wie \textcite[S. 187]{Samuelson1946} früh erkannte: "`Es ist für heutige Studierende fast unmöglich  den Einfluss der 'Keynesianischen Revolution' zu realisieren"'. Deren Inhalt ist spätestens nach dem Zweiten Weltkrieg in kürzester Zeit zum neuen Standard der Makroökonomie geworden. 

Das Bild des dominierenden Ökonomen, das von Keynes entstand, ist Resultat einer Kombination aus privilegierter Geburt und notwendigem Glück, aber vor allem Genialität gepaart mit einem enormen Tätigkeitsumfang. Zudem verfügte Keynes über ein gesundes Selbstvertrauen. Schon vor der Veröffentlichung seines Hauptwerkes schrieb er an den befreundeten Literaten George Bernard Shaw: "`I believe myself to be writing a book of economic theory which will largely revolutionize [...] the way the world thinks about economic problems"' \parencite[S.13]{Warsh}. 

Die Kombination aus seinem beruflicher Werdegang, seinem Lebensstil, sowie seinem relativ frühen Tod machten ihn später zu einer Art Mythos. Betrachten wir den Werdegang von Keynes. 

Keynes wurde 1883 geboren. Seine Familienverhältnisse scheinen wie gemacht für seine spätere Karriere. Sein Vater, John Neville Keynes, war ebenfalls Ökonom und später einer der Gründungsväter des "`Economic Journals"', das bis heute eine der führenden Fachzeitschriften der Wirtschaftswissenschaften ist. Seine war eine der ersten Student\textit{innen} in Cambridge und später Bürgermeisterin der Stadt. Sein Bruder Geoffrey wurde 1955 als berühmter Chirurg geadelt und seine Schwester Margret heiratete einen späteren Medizin-Nobelpreisträger \parencite[S. 275]{Scherf1989}. John Maynard Keynes studierte zunächst Mathematik und schien danach eine Art Beamtenlaufbahn einzuschlagen. Bei der Vorbereitung auf das "`Civil Service Exam"' genoss er den Unterricht der damals führenden Ökonomen Marshall und Pigou. Im Anschluss arbeitete er tatsächlich kurz, nämlich zwischen 1906 und 1908, als bei einer Verwaltungsbehörde, dem India Office in London, bevor er 1908 an die University of Cambridge zurückkehrte. Dort wurde er Lektor und erhielt die "`Fellowship"'-Auszeichnung - die dort bis heute für besondere wissenschaftliche Leistungen - vergeben wird, und zwar für seine frühen Arbeiten zur Wahrscheinlichkeitstheorie \parencite[S. 276]{Scherf1989}. Wirtschaftswissenschaftliche Artikel hatte er bis zu diesem Zeitpunkt noch keine publiziert, doch seine Leistungen, sein Engagement und seine Ausstrahlung müssen überdurchschnittlich gewesen sein, schließlich wurde er bereits 1911 Editor des Economic Journals. Während des Ersten Weltkrieges wechselt er ins englische Finanzministerium, wo er bereits 1917 zum Abteilungsleiter wird. Ein - aus wissenschaftlicher Sicht - unscheinbarer, aber wohl entscheidender Karriere-Schritt. Denn als solcher nimmt er 1919 an den Friedensverhandlungen von Versailles teil. Dort stellt er sich bekanntermaßen gegen die, seiner Meinung nach nicht erfüllbaren, Reparationsforderungen gegenüber Deutschland. Er reist ab und verfasst innerhalb kürzester Zeit sein Werk "`The Economic Consequences of the Peace"' \parencite{Keynes1919}, das ihm bereits damals zu Weltruhm verhalf. 

In diese Zeit fällt auch sein so häufig diskutiertes Privatleben als Teil der Bloomsbury Group, einer Gruppe junger Intellektueller, Philosophen und Künstler. Unter anderem war auch Virginia Woolf Mitglied. Über seine Offenheit gegenüber sexuellen Beziehungen - Keynes hatte in dieser Zeit häufige, offen homosexuelle Beziehungen - wird bis heute geschrieben. Ab 1921 war er schließlich mit der russischen Ballerina Lydia Lopokova liiert, die er 1925 heiratete. Eine Zeitung schrieb damals "`Will there ever be such a union of beauty and brains?"' \parencite[S. 13]{Warsh}. Diese Tatsachen, dass die Hochzeit des damals schon bekannten Ökonomen ein gesellschaftliches Ereignis war, und dass bis heute über die sexuelle Orientierung eines Ökonomen geschrieben wird, zeigt seine Berühmtheit über die Wirtschaftswissenschaften hinaus und erinnert eher an einen Musik- oder Filmstar. Vor allem trug es aber wohl bei zum Bild des genialen, aber auch lebensfrohen Winner-Typen, das von John Maynard Keynes bis heute gezeichnet wird.

Es ist interessant, dass der führende Ökonomie-Theoretiker des 20. Jahrhunderts bereits über 15 Jahre vor der Veröffentlichung seines bahnbrechenden Hauptwerkes den Bekanntheitsgrad eines Stars erreichte. Anstatt sich jetzt in seinem Ruhm zu sonnen und sich zurückzulehnen, beginnt nun eine Phase unglaublicher Produktivität. In Cambridge ist er nach wie vor als Editor des "`Economic Journals"' tätig, zusätzlich verwaltet er die Finanzen der Universität. Seine dabei erlangten Erfolge an den Kapitalmärkten trugen später auch zur Entstehung des Mythos um seine Person bei. Als Wissenschaftler veröffentlicht er zwischen 1920 und 1933 praktisch jedes Jahr ein ökonomisches Buch. Daneben ist er zu dieser Zeit gefragter Berater von Versicherungen, Investmentfonds und Politikern.

Gegen Ende der 1920er Jahre strebte Keynes immer mehr danach sein wissenschaftliches Gesamtwerk abzuschließen \parencite[S. 198]{Samuelson1946}, die darin bestehen sollte eine umfassende Geldtheorie vorzulegen. Dazu veröffentlichte er 1930 seine "`Treatise on Money"' \parencite{Keynes1930}. Aber schon mit der Veröffentlichung seines Werkes war er unzufrieden damit \parencite[S. 282]{Scherf1989},   \parencite[S. 198]{Samuelson1946}. Aus heutiger Sicht gesehen ist "`Treatise on Money"' nicht besonders revolutionär. Versetzt man sich zurück in das Erscheinungsjahr 1930 ist als kleiner Schritt heraus aus dem damals dominierenden makroökonomischem State of the Art zu sehen, deren wichtigster Baustein die Quantitätstheorie des Geldes war. Die "`Treatise on Money"' war ein erster Schritt aus diesem Dogma heraus \parencite[S. 282]{Scherf1989}, wenn auch kein hinreichend großer um bereits von einer ökonomischen Zeitenwende zu sprechen. Er verwendete darin schon die Argumentation anhand von Kreislaufzusammenhängen. Konjunkturzyklen entstehen demnach aus der Differenz zwischen Produktionskosten und Marktpreis, was in "`Übergewinne"', oder "`Verluste"' für die Unternehmen resultiert \parencite[S. 422]{Rosner2012}. Ist die Differenz Null so befindet sich die Ökonomie im Gleichgewicht, Sparen und Investitionen sind ident, der entsprechende Zinssatz ist jener, bei dem das Preisniveau konstant bleibt, wobei sich Keynes bereits auf den natürlichen Zinssatz von \textcite{Wicksel1898} (vgl. Kapitel \ref{Wicksell}). Übersteigen die Produktionskosten die Marktpreise kommt es zu Einschränkungen der Produktion und damit bei Keynes zu Arbeitslosigkeit. Ein Senken des Zinssatzes fördert Investitionen und die Wirtschaft nähert sich wieder dem Gleichgewicht. Übersteigen die Preise die Produktionskosten, kommt es zu einem Anstieg der Investitionen \parencite[S. 282]{Scherf1989}. Ein Erhöhen des Zinssatzes fördert Sparen und die Wirtschaft nähert sich auch hier dem Gleichgewicht. Warum es zu Abweichungen vom Gleichgewicht kommt, also warum die Märkte nicht sofort durch Preisanpassungen jederzeit vollständig geräumt werden, behandelt \textcite{Keynes1930} nicht. Der zweite Teil des Werks analysierte das Finanzsystem und dessen Wirkung auf die Realwirtschaft. Damit betrat \textcite{Keynes1930} zwar akademisches Neuland, aber  revolutionärer Durchbruch waren auch diese Ansätze nicht. \textcite{Scherf1989} bezeichnet die "`Treatise"' als "`Standardwerk der Geldlehre"', \textcite[S. 198]{Samuelson1946} bezeichnete das Werk als bedeutend, allerdings wenig aufregend und seine Gleichungen als "`Umweg und Sackgasse"'\footnote{Der später häufig als Duell auf Augenhöhe hochstilisierte wissenschaftliche Disput zwischen John Maynard Keynes und Friedrich August Hayek bezieht sich am ehesten auf die Zeit um 1930. \textcite{Keynes1930} "`Treatise on Money"' stellte \textcite{Hayek1931} "`Preise und Produktion"' gegenüber. Deren Diskussionen zur Geld- und Konjunkturtheorie fanden tatsächlich auf Augenhöhe statt. Die "`General Theory"' \parencite{Keynes1936} wurde zwar von Hayek von Beginn an abgelehnt. Ein wissenschaftliches Gegenmodell, das eine ähnliche Bedeutung wie die "`General Theory"' erlangte, konnte er aber nicht liefern. Sein Werk \textcite{Hayek1944} "`Der Weg zur Knechtschaft"' hat aus wirtschaftstheoretischer Sicht nicht den gleichen Stellenwert wie die "`General Theory"'. Die Theorie-Mängel des Keynesianismus deckten erst später \textcite{Friedman1963} und vor allem \textcite{Lucas1976} auf}.

Hätte Keynes zu diesem Zeitpunkt seine wissenschaftliche Tätigkeit eingestellt, würde er aus heutiger Sicht unzweifelhaft als wenig bedeutender Ökonom eingestuft. Dies, obwohl er damals, im Jahr 1930, schon lange international bekannt war. Erst in weiterer Folge, im Jahr 1936, erschien schließlich die "`General Theory"'. Dieses Werk ist unzweifelhaft das einflussreichste ökonomische Buch des 20. Jahrhunderts!
Es war der alleinige Auslöser der "`Keynesianischen Revolution"'. Die Paradoxien und Eigenheiten, die darin zu finden sind, wurden weiter oben schon beschrieben. Unabhängig davon wird es nach wie vor häufig diskutiert. Der Versuch sämtliche unterschiedliche Untersuchungen oder Interpretationen des Werks auch nur anzuführen, müsste scheitern. \textcite[S. 38ff]{Weintraub1979} benannte ein Kapitel seines Buches scherzeshalber "`The $4,827^{th}$ re-examination of Keynes's system"'. Die "`General Theory"' ist der direkte Ausgangspunkt dreier ganz verschiedener grober Forschungsrichtungen, die in weiterer Folge entstanden. Erstens, die Vertreter der "`Neoklassischen Synthese"' (Vergleich Kapitel \ref{Synthese}) griffen einzelne Ideen von Keynes auf und verschmolzen diese mit den bereits bestehenden (Neo)-Klassischen Arbeiten zur Mainstream-Ökonomie der nächsten Jahrzehnte. Zweitens, eine Gruppe von Ökonomen bezog und bezieht sich bis heute mehr oder weniger \textit{direkt} auf die "`General Theory"'. Diese Gruppe - häufig unter dem Begriff "`Post-Keynesianer"' zusammengefasst - sieht die Keynesianische Theorie durch die "`Neoklassischen Synthese"' verstümmelt und auf nur einzelne Punkte beschränkt. Eine dritte Gruppe von Ökonomen lehnte die "`General Theory"' von Anfang an ab. Dazu gehörten zunächst vor allem die kontinental-europäischen Liberalen (vgl. Kapitel \ref{Neoliberalismus}) zum Beispiel die Freiburger Schule und auch August Friedrich Hayek, aber auch schon der junge Milton Friedman. Dieser wurde später zum wirtschaftspolitischen Totengräber des Keynesianismus (vgl. Kapitel \ref{Monetarismus}). An seiner Wirkungsstätte - in Chicago - entwickelte sich schließlich der "`totale Widerpart"' zum Keynesianismus, die "`Neue Klassische Makroökonomie"' (vgl. Kapitel \ref{Neue Makro}).

Was ist nun der Inhalt der "`The General Theory	of Employment, Interest and Money"'? Es wurde bereits dargelegt, dass dieses Werk auf so unterschiedliche Art und Weise ausgelegt werden kann. Das liegt auch daran, dass die "`General Theory"' selbst einen Recht niedrigen Grad der Formalisierung aufweist. Insgesamt finden sich im Buch verhältnismäßig wenige Formeln und diese eher punktuell. Dies ist einigermaßen überraschend. Ging doch der Trend gerade zu dieser Zeit in Richtung höherer Formalisierung, außerdem war Keynes ja studierter Mathematiker. Dennoch sprach er sich explizit gegen eine zu hohe Formalisierung der Wirtschaftswissenschaften aus. Ein formalisiertes Gesamtmodell wird also in der "`General Theory"' auf jeden Fall nicht geliefert.

Keynes revolutioniert die Denkweise der Ökonomie, ohne aber den Fehler zu machen den bisherigen State of the Art komplett über den Haufen werfen zu wollen. So ist seine Theorie im wesentlichen eine Ungleichgewichtstheorie, dennoch bleibt er auf den etablierten Wegen von Marshall, seines großen Lehrers in Cambridge: der komparativ-statischen Analyse. Er ändert allerdings den Ausgangspunkt bei der Betrachtung ökonomischer Zusammenhänge. Dieser ist nämlich bei Keynes das Nationaleinkommen, welches stets mit der Gesamtbeschäftigung eng zusammenhängt. Entscheidend und neu ist der Begriff der "`Effektiven (Gesamt)-Nachfrage"', die in Kapitel 3 von \textcite[S. 25]{Keynes1936} entwickelt wurde. Dies ist jener Punkt an dem die aggregierte Nachfragefunktion mit der aggregierten Angebotsfunktion schneidet. Oder mit anderen Worten: Das ist die Gesamtnachfrage, die am Markt realisiert wird. Mit der Einführung der Effektiven Nachfrage greift Keynes auch früh das "`Say'sche Gesetz"' an: Wenn das Angebot die Nachfrage bestimmt, gibt es keinen Grund für die Unternehmen die Produktion nicht bis an jenen Punkt auszuweiten, an dem schlicht keine Arbeitskräfte mehr verfügbar sind. Demnach steht einer Vollbeschäftigung nichts im Wege. Erst wenn die aggregierte Nachfrage entscheidend ist für die Produktion, also das aggregierte Angebot, sind Situationen denkbar, in denen die Gesamtwirtschaft unter stabiler Unterbeschäftigung leidet. Keynes lehnt das "`Say'sche Gesetz"' um, die Gültigkeit seiner Theorie verlangt, dass die Nachfrage die Produktion bestimmt. Das ist die Grundlage für die später daraus abgeleiteten Empfehlungen für "`Nachfrageorientierte Wirtschaftspolitik"'. \textit{Der} Bruch mit der neoklassischen Mainstream-Ökonomie war die Annahme, dass die Preise - und damit auch die Löhne - gegeben sind\parencite[S. 58]{Snowdon2005}. Anpassungen erfolgen damit über die gehandelten Mengen. Dies macht die Theorie zu einer der kurzen Frist. Die Höhe der Nachfrage nach Arbeit - also die Beschäftigung - hängt damit direkt von der Effektiven Nachfrage ab.

Die Beschäftigungstheorie (Theorie zur Arbeitslosigkeit) von Keynes ist die erste bahnbrechende Erkenntnis, die wir aus der "`General Theory"' näher betrachten (In \textcite{Keynes1936} selbst ist die Beschäftigungstheorie am Ende des Buches angesiedelt). In der (Neo-)Klassik gibt es das Phänomen der Arbeitslosigkeit nicht wirklich. Funktionierende Märkte sorgen demnach immer für eine Bereinigung der Märkte - auch des Arbeitsmarktes. Damit kann unfreiwillige Arbeitslosigkeit nicht existieren. Dies war natürlich auch schon vor der "`Great Depression"' empirisch nicht zu halten und wurde während dieser zu blankem Hohn. Tatsächlich lieferte \textcite{Pigou1933} mitten in der "`Great Depression"' die erste umfassende Beschäftigungstheorie. Demnach widerspräche die beobachtbare Arbeitslosigkeit nicht grundsätzlich der klassischen Theorie. Er ging stattdessen davon aus, dass eine veränderte Nachfrage nach Arbeitskräften, sowie zu hohe Löhne dazu führten, dass es Arbeitslosigkeit gäbe, die aber nur von kurzfristiger Dauer sein könne (vgl. Kapitel \ref{sec: Pigou}). Das Problem zu hoher Reallöhne kannten die (Neo-)Klassiker bereits. Als einziges Mittel dagegen waren aber Lohnsenkungen. \textcite{Keynes1936} stellte dem seine fundamentale Beschäftigungstheorie entgegen. Zunächst akzeptiert \textcite[S. 258f]{Keynes1936} die Möglichkeit, dass durch Nominallohn-Kürzungen die Produktion angeregt wird und dadurch positive gesamtwirtschaftliche Effekte entstehen. Er schwenkt aber schnell dazu über, dass es ein Trugschluss sein muss, wenn man davon ausgeht, dass eine allgemeine Kürzung der Nominallöhne mit konstanter aggregierter Effektiver Nachfrage einhergehen würden. Aus heutiger Sicht klingt das vollkommen einleuchtend: Wenn man alle Löhne und Gehälter kürzt, muss man davon ausgehen, dass die Bezieher dieser ihre Konsumausgaben auch zurückfahren müssen \parencite[S. 269]{Keynes1936}. Aber den (Neo-)Klassikern blieb vor Keynes gar nichts anderes übrig als davon auszugehen, dass die Nachfrage von der Lohnhöhe unabhängig sei. Schließlich waren diese makroökonomisch in der Quantitätstheorie "`gefangen"' und darin ergibt sich die Gesamtnachfrage aus dem Produkt der Geldmenge und deren Umlaufgeschwindigkeit (vgl. Kapitel \ref{FisherandClark}). Eventuelle Lohnrückgänge müssten demnach durch entsprechende Gewinnsteigerungen ausgeglichen werden, wenn der Weg zurück zum Gleichgewicht dies erforderte. \textcite{Keynes1936} brach hier ganz ausdrücklich und durchaus verbal aggressiv aus den alten makroökonomischen Vorstellungen aus. Was schon vor Keynes bekannt war, ist die Tatsache, dass durch rigide Nominallöhne - also solche, die nach unten nicht flexibel sind - steigende Reallöhne entstehen können. Nämlich dann wenn das Preisniveau fällt, also Deflation herrscht. Dadurch sind die Reallöhne zu hoch und unfreiwillige Arbeitslosigkeit ist die Folge\footnote{\textcite{Keynes1936} behandelt den möglichen positiven Effekt steigender Reallöhne auf die Gesamtnachfrage nicht. Dies blieb später \textcite{Pigou1943} vorbehalten (vgl. Kapitel \ref{sec: Pigou}).}. \textcite{Keynes1936} akzeptiert diesen Effekt. Er postuliert aber auch, dass sinkende Nominallöhne die Gesamtnachfrage so weit fallen lassen können, dass dadurch wiederum das Preisniveau soweit fällt, dass insgesamt sogar steigende Reallöhne realisiert werden. Er plädiert daher dafür nicht die Nominallöhne zu senken, sondern das Preisniveau zu erhöhen. Dazu empfahl er allerdings nicht primär geldpolitische Maßnahmen, wie heutige Leser vermuten könnten, sondern eben fiskalpolitische. Durch eine Ausweitung der Aggregierten Nachfrage, durch staatliche Ausgaben, kommt es zu steigenden Preisen und damit sinkenden Reallöhnen und in weiterer Folge zu steigender Beschäftigung.

\textcite{Keynes1936} diskutierte auch die Möglichkeiten mit sinkenden Nominallöhnen wieder ein Vollbeschäftigungs-Gleichgewicht am Arbeitsmarkt herzustellen, lehnte diese aber durchwegs alle ab. Erstens, aus rein praktischen Gründen: Sinkende Nominallöhne sind in liberalen Demokratien schwer durchsetzbar - das klassische Problem rigider Löhne. Sinkende Reallöhne durch Inflation treffen hingegen auf weit weniger Widerstand wie \textcite[S. 14]{Keynes1936} schon postulierte. Er selbst hatte also - noch lange vor der Diskussion über die Gültigkeit der Phillips-Kurve (vgl. Kapitel \ref{sec: Phillips}) - wenig Berührungsängste mit mäßiger Inflation. Zweitens führte Keynes theoretische Gründe gegen Nominallohn-Senkungen an. Theoretisch, so Keynes, könnten diese zwar positive Effekte haben, indem der folgende Rückgang der aggregierten Nachfrage bei konstanter Geldmenge zu sinkenden Real-Zinsen führt. Dadurch werden Investitionen angeregt, die wiederum einen stimulierenden Effekt auf die Nachfrage haben und so schlussendlich zum Gleichgewicht zurück führen. Dieser Mechanismus wird heute Keynes-Effekt genannt, obwohl er selbst an dessen Wirkung zweifelte. Und zwar aus zwei Gründen. Erstens, wenn der nominale Zinssatz bereits bei Null ist, kann der Real-Zins nicht mehr fallen und Investitionen werden nicht angeregt. Eine Situation, die bis vor Kurzem als "`Liquiditätsfalle"' in aller Munde war. Zweitens, könnten Unternehmer, zum Beispiel aufgrund negativer Zukunftsaussichten, selbst dann nicht investieren wollen, wenn der Realzinssatz fällt. Dies wird heute als Investitionsfalle bezeichnet.

Keynes sah also die Gefahr einer stabilen Unterauslastung der Ökonomie und forderte aktive Wirtschaftspolitik im Sinne von Fiskalpolitik \textit{und} Geldpolitik um dieser Situation zu entkommen, wenn sie eintritt. Das war zu seiner Zeit revolutionär. (Neo-)klassische Ökonomen tendierten dazu Markteingriffe abzulehnen, zumindest dann wenn keine Marktunvollkommenheiten wie Monopole auftraten. Keynes akzeptierte interessanterweise die Annahme perfekter Märkte, argumentierte nun aber dennoch für aktive Wirtschaftspolitik, indem er Situationen aufzeigte, in denen ein Gleichgewicht trotz Unterbeschäftigung vorliegt. Dieser Zugang war natürlich im Angesicht der 1936 gerade überwundenen "`Great Depression"' unter sehr speziellen ökonomischen Umständen entstanden, oder wie \textcite[S. 199]{Samuelson1946} es ausdrückte: "`While Keynes did much for the Great Depression, it is no less true that the Great Depression did much for him"'. Trotzdem ist dieses Plädoyer für aktive Wirtschaftspolitik unerwartet. Wurde als ideologisch-ökonomisch größte Gefahr zu jener Zeit doch die Planwirtschaft in Form des real existierenden Sozialismus gesehen, der in den 1930er Jahren durchaus Erfolge verbuchen konnte. In Wirklichkeit aber wollte Keynes den Kapitalismus nicht an die Planwirtschaft annähern, sondern eben verhindern, dass diese für westeuropäische Bürger attraktiv wird.

Die zweite bahnbrechende Erkenntnis der "`General Theory"' ist die Funktionsweise \textit{wie} das Prinzip der Effektiven Nachfrage auf die Gesamtwirtschaft wirkt. Im Gleichgewicht der Effektiven Nachfrage entspricht die aggregierte Nachfrage $Z$ dem aggregierten Angebot (Produktion $P$) und damit dem Gesamteinkommen $Y$. Die aggregierte Nachfrage setzt sich zusammen aus dem Konsum der Haushalte $C$ und aus den Investitionen der Unternehmen $I$. Der Konsum hängt hierbei bei Keynes vom Einkommen ab. Welches Einkommen den Konsum wirklich bestimmt hat dann später zu verschiedenen "`Einkommenshypothesen"' und unterschiedlichen Interpretationen geführt, zum Beispiel von Modigliani (vgl. Kapitel \ref{Synthese}) und Friedman (vgl. Kapitel \ref{Monetarismus}). 

HIER WEITER Multiplikator \parencite[S. 59]{Snowdon2005}






Zeitgenossen litten darunter Pigou, Kalecki, Hayek, Schumpeter. 








Keynes war sowohl was seine persönliche Geschichte angeht, als auch seine wissenschaftliche Karriere betreffend ein Phänomen. Er hatte schon früh unkonventionelle und durchaus bedeutende Arbeiten wie \textcite{Keynes1919} "`The Economic Consequences of the Peace"' oder \textcite{Keynes1930} "`A Treatise on Money"' veröffentlicht. Der einschneidende Punkt war - wie für so viele Ökonomen, jedoch im Falle von Keynes noch ausgeprägter - die "`Great Depression"'. Die größte Wirtschaftskrise aller Zeiten,  die ab 1929 zunächst für Kursstürze an den Börsen und in weiterer Folge weltweit für BIP-Rückgänge und hohe Arbeitslosenraten sorgte, brachte Keynes dazu eine völlig neue Wirtschaftstheorie zu verfassen. Diese veröffentlichte er im bahnbrechenden Werk \textcite{Keynes1936} "`The General Theory of Employment, Interest and Money"', kurz meist einfach als "`General Theory"' bezeichnet. 


Für Keynes war die "`Vertrustung"' der Ökonomie, die sein Zeitgenosse Schumpeter thematisierte, kein Problem. Er war in dieser Hinsicht stark geprägt von Marshall und ging von funktionierendem Wettbewerb - im Sinne vollkommener Konkurrenzmärkte - auf den Einzelmärkten aus. 
Obwohl Keynes Politikern gegenüber allgemein skeptisch gegenüberstand \parencite[S. 291]{Scherf1989}, kam er nicht auf die Idee, dass aktive Wirtschaftspolitik auch falsch laufen könnte und Politiker damit ausschließlich ihren eigenen Nutzen maximierten, nicht jenen des Volkes. Diese Erkenntnis blieb viel später der "`Neuen Politischen Ökonomie"' vorbehalten. 




Sie mögen sich vielleicht fragen warum dieses Kapitel ein eher kurzes ist?! Schließlich ist doch der Keynesianismus die zentrale ökonomische Errungenschaft des 20. Jahrhunderts gewesen und deren Schöpfer \textsc{John Maynard Keynes} der größte Ökonom des 20. Jahrhunderts. Wenn man anstatt von "`der Größte"' die Notation "`einer der Größten"' verwendet, so kann man dies sicherlich bestätigen.

Warum ist aber das Kapitel nun so kurz ausgefallen? Nun, Keynes veröffentlichte sein bahnbrechendes Werk \textit{The General Theory of Employment, Interest and Money} im Jahr 1936. Bald darauf, 1939, brach der Zweite Weltkrieg aus und die Welt beschäftigte sich bis 1945 mit anderen Sachen, als rein ökonomischen Fragestellungen. 1946 starb Keynes. Wenig später startete "`sein"' Keynesianismus den ökonomischen Siegeszug um die Welt. Und hier muss man entscheidend einhaken! Was meist als "`Keynesianismus"' bezeichnet wird, ist in Wirklichkeit bereits eine Weiterentwicklung des eigentlichen, ursprünglichen Keynesianismus.

HIER WEITER
Inhalt Keynes:

Einer der wichtigsten Punkte für die spätere wirtschaftspolitische Anwendung war der "`Multiplikator-Effekt"'. Das zehnte Kapitel der "`General Theory"' widmet sich diesem Thema. Wie \textcite[S. 114]{Keynes1936} selbst beschreibt, war es Richard \textcite{Kahn1931}, der den "`Multiplikator-Effekt"' als erster beschrieb, und zwar im Sinne des Zusammenhangs zwischen (staatlichen) Investitionen und Arbeitslosigkeit.






Bereits 1937 veröffentlichte \textsc{John R. Hicks} den Artikel \textit{Mr. Keynes and the Classics: A Suggested Interpretation}. Das Werk von Keynes hat nämlich die Besonderheit, dass es schwer zu lesen ist, aber vor allem auf formale Darstellungen verzichtet. Hicks übernahm diese Formalisierung und verband einen Teil von Keynes' Theorie mit neoklassischen Elementen zum \textit{IS-LM-Modell}. Dieses Modell stellt noch heute den finalen Punkt in vielen Einführungslehrveranstaltungen zu Makroökonomie dar.

Diese Formalisierung durch Hicks enthält aber zwei extrem wichtige Punkte:
\begin{itemize}
	\item "`Er übernahm einen \textit{Teil} von Keynes' Theorie"': Bei dieser Formalisierung gingen im Gegenzug viele Teile der General Theory verloren. Eine Tatsache, die bis heute in der Mainstream-Ökonomie hingenommen wird. Diese "`verlorenen Teile"' sollten später von den \textit{Post-Keynesianern} aufgegriffen werden.
	\item "`Er verband diesen Teil mit neoklassischen Elementen"': Wenn wir lesen, dass die 1950er und 1960er Jahre die Hochzeit des "`Keynesianismus"' waren, dann meinen Ökonomen eigentlich, dass das "`alte"' Neoklassische Wissen herangezogen wurde und um "`Keynesianische"' Elemente erweitert wurde. Es entstand also eine "`Synthese"' aus zwei Wissensgebieten, folglich wird das Ganze unter Ökonomen die \textsc{Neoklassische Synthese} genannt.
\end{itemize}

Sie können natürlich jetzt argumentieren das sei Haarspalterei und zu behaupten die allgemeine Bezeichnung Keynesianismus müsste eigentlich Neoklassische Synthese heißen, sei Besserwisserisch und verwirrt mehr als sie Nutzen bringt. Das stimmt im Großen und Ganzen. Aber ich denke auch es ist wichtig zu erwähnen, dass die Theorien von Keynes praktisch von ihrem erscheinen weg unterschiedlich verwendet und interpretiert wurden. Die Interpretation, die sich in den Wirtschaftswissenschaften als am erfolgreichsten erwiesen hat, ist eben jene, die ich als \textsc{Neoklassische Synthese} im nächsten Kapitel vorstelle. 




%%%%%%%%%%%%%%%%%%%%% chapter.tex %%%%%%%%%%%%%%%%%%%%%%%%%%%%%%%%%
%
% sample chapter
%
% Use this file as a template for your own input.
%
%%%%%%%%%%%%%%%%%%%%%%%% Springer-Verlag %%%%%%%%%%%%%%%%%%%%%%%%%%

\chapter{Synthese: Ein bisschen Neoklassik, ein bisschen Keynes}
\label{Synthese}

\section{Hicks \& Samuelson}

IS-LM, 




Mundell-Fleming-Modell


Modigliani
Franco Modigliani belegt, die diesen wie folgt um-schreibt1): „Non-monetarists accept what 1 regard to be the fundamental practical message of the General Theory: that a private enterprise economy using an intangible money needs to be stabilized, can be stabilized and therefore should be stabilized by appropriate monetary and fiscal policies. Mone-tarists by contrast take the view that there is no serious need to stabilize the economy; that even if there were a need, it could not be done, for stabilizing policies would be more likely
Mone-tarists by contrast take the view that there is no serious need to stabilize the economy; that even if there were a need, it could not be done, for stabilizing policies would be more likely to increase than to decrease instability; and, at least some monetarists would, 1 believe, go so far as to hold that, even in the unlikely event that stabilization policies could on balance prove beneficial, the government should not be trusted with the necessary power."

1) American Economic Review, März 1977.



\section{Tobin}

Tobin war skeptisch gegenüber Neu-Keynesianern. \textcite[S. 398]{Snowdon2005}

\section{Philipskurve} \label{sec: Phillips}

Zusammenhang von Philips
Samuelson und Solow für die USA: Name: Philipskurve


Später: Konzept NAIRU


Dagegen Friedman und Phelps:
Friedman (1968): The Role of Monetary Policy
Phelps (1968): Money-Wage Dynamics and Labor-Marlet Equilibrium (viel formaler als Friedman)




%%%%%%%%%%%%%%%%%%%%% chapter.tex %%%%%%%%%%%%%%%%%%%%%%%%%%%%%%%%%
%
% sample chapter
%
% Use this file as a template for your own input.
%
%%%%%%%%%%%%%%%%%%%%%%%% Springer-Verlag %%%%%%%%%%%%%%%%%%%%%%%%%%

\chapter{Die Neoklassik neben Keynes}
\label{Neoklassik_nach1945}

Teil des Kapitels könnte auch Finanzierungstheorie, Spieltheorie sein (sind aber eigene Kapitel), aber auch die ganze Input-Output-Analyse.


\section{Pigou: Die Neoklassik erfindet sich neu}
\label{sec: Pigou}

Beginnend mit Arthur Cecil Pigou entwickelte sich die Neoklassik in eine neue Richtung. Wie in Kapitel \ref{Neoklassik} dargestellt, kann die Neoklassische Theorie mit Marshall in gewisser Hinsicht als abgeschlossen gelten. Dies vor allem in dem Hinblick, dass sich die mikroökonomischen Ausgangstheorien seit damals tatsächlich kaum noch geändert haben. Selbst in modernen Mikroökonomie-Büchern sind die ersten Kapitel im wesentlich identisch in den Arbeiten Marshall's zu finden. Die Mikroökonomie wurde seither nicht mehr grundlegend verändert, in dem Sinne, dass zuvor geltende Konzepte über den Haufen geworfen wurden, sondern sie wurde seither immer wieder erweitert. Marshall's Principles sind nach wie vor richtig. Aber man kann sie eben nicht mehr als Erklärung aller wirtschaftlichen Tätigkeiten heranziehen, sondern gelten mittlerweile nur mehr für einen sehr speziellen Fall. Heute wissen wir, dass die damals angenommenen Voraussetzungen, wie vollkommene Märkte und perfekte Konkurrenz nicht die Regel, sondern die Ausnahme sind.

Arthur Cecil Pigou ist der letzte große Ökonom in der Reihe der englischen Vertreter der Klassik und Neoklassik in Cambridge. Nach Pigou entwickelte sich die Ökonomie dort in Richtung Keynesianismus weiter, wie wir im letzten Kapitel (vgl. Kapitel \ref{Keynes}) bereits gelesen haben. Pigou ist wohl einer der meist unterschätzten Ökonomen des 20. Jahrhunderts. Schließlich war er der erste Mainstream-Ökonom, der postulierte und auch anerkannte, dass rein marktwirtschaftliche Ergebnisse nicht immer effizient sein müssen. Er erkannte also bereits vor Keynes die Sinnhaftigkeit von Staatseingriffe in bestimmten Situationen, wenn auch deren Wirkung und Berechtigung von ganz anderer Art und Weise sind. Während Keynes die makroökonomische Wirtschaftspolitik begründete, tat Pigou dies für die mikroökonomische Wirtschaftspolitik. Seine daraus unter anderem resultierenden Thesen zu den verschiedenen Formen des Marktversagen sind heute - Stichwort Klimawandel - aktueller denn je, wie gleich konkretisiert werden wird. Pigou's Karriere begann dann auch wie die eines ganz großen: Er war der Lieblingsschüler Marshall's der damals unangefochtenen Lichtgestalt der englischen Ökonomie. Mit erst 30 Jahre wurde Pigou schließlich dessen Nachfolger an der Universität in Cambridge. Und bereits 1912 legte er das bedeutende Werk "Wealth and Welfare" (\parencite{Pigou1912}) auf, das in der Neuauflage von 1920 als "The Economics of Welfare" (\parencite{Pigou1920}) ein bis heute bedeutendes Werk darstellt. Diese Jahreszahlen legen nahe, dass Pigou eigentlich auch in Kapitel \ref{Neoklassik} gut aufgehoben wäre. Inhaltlich von Bedeutung wurde der Forschungsbereich, den Pigou damit eröffnete, aber erst nach 1945. 

Mit zwei wesentlichen Punkten hat Pigou in seinem Hauptwerk die Neoklassik revolutioniert. Erstens, hat er mit dem Credo aufgeräumt, das seit Adam Smith die Neoklassik prägte, nämlich, dass individuell-nutzenmaximierendes Verhalten auch für die gesamte Gesellschaft erstrebenswert sei und das optimale gesamtwirtschaftliche Ergebnis hervorbringt \parencite[S. 111]{Pigou1920}. Das hat enorme Auswirkungen. Damit verbunden ist nämlich die Notwendigkeit staatlicher Eingriffe, immer dann, wenn der Markt darin versagt eine optimale Allokation hervorzubringen. Das riesige Gebiet der mikroökonomischen Wirtschaftspolitik war damit geboren. Zweitens, wollte er einen formalen Ansatz zur Analyse des \textit{gesamt}wirtschaftlichen Nutzens etablieren - was die Begründung der Wohlfahrtsökonomie bedeutete. Diese wurde aber recht rasch als gescheitert betrachtet. Dazu aber später mehr, werfen wir zunächst einen Blick auf seine Ansätze zum ersten Punkt. Ausgangspunkt sind die Gleichgewichtstheorien in der Tradition von Walras (vgl. Kapitel \ref{Walras}). Darin wird davon ausgegangen, dass alle Märkte im Gleichgewicht sind. Damit wurde stets auch impliziert, dass die Märkte jeweils das \textit{optimale} Ergebnis liefern. \textcite{Pigou1920} zeigte nun systematisch Beispiele, in denen das nicht der Fall ist. Der Ausgangspunkt sind hier stets Externalitäten \parencite[S. 115]. Um diese systematisch zu analysieren führt er den Vergleich zwischen privaten und sozialen Grenzprodukten ein \parencite[S. 114]{Pigou1920}. Nur wenn die sozialen und privaten Grenzerträge übereinstimmen, ist die Marktlösung auch eine gesamtwirtschaftlich optimale Lösung. Weichen die sozialen Grenzerträge von den privaten ab, liegt ein externer Effekt vor. Der könnte zum Beispiel darin bestehen, dass der Produzent A bei der Herstellung seiner Güter Kosten auf die Allgemeinheit überträgt, für die er nicht aufkommen muss. Zum Beispiel könnte seine Fabrik das Umland verschmutzen ohne, dass er eine entsprechende Reinigungsgebühr entrichten muss. Diese sozialen Kosten trägt stattdessen die Bevölkerung. Als Resultat sind die privaten Kosten des Fabrikanten zu niedrig, was wiederum in einer zu hohen Produktion und zu niedrigen Preisen führt. Nur wenn der Staat eingreift und durch Steuern die sozialen Kosten wieder auf den Produzenten überträgt, ist das gesamtwirtschaftliche Ergebnis tatsächlich optimal. Damit begründete er die heute wieder so oft zitierte Pigou-Steuer. Überhaupt erlangte Pigou mit dem Aufkommen der Umweltproblematik eine enorme Bekanntheit in den letzten Jahren. Seine Arbeiten zu Marktversagen begründeten die mikroökonomische Wirtschaftspolitik. Allerdings muss man einschränkend sagen, dass in \textcite{Pigou1920} zwar bereits verschiedene Marktversagensformen wie natürliche Monopole \parencite[S. 240]{Pigou1920}, Informationsmängel \parencite[S. 131]{Pigou1920} und eben externe Effekte, allerdings werden die Probleme eher anhand von Beispielen und nicht systematisch-analytisch behandelt. Auch finden sich noch theoretische Fehler in seinen Abhandlungen, wie zum Beispiel der fehlende Unterschied zwischen technologischen und pekuniären negativen Effekten \parencite[S. 242]{Cansier1989}. Letztgenannte sind definiert als Kosten, die einem Unternehmen erwachsen, wenn ein Konkurrent ein identisches Gut zu besseren Preisen anbieten kann. In diesem Fall ist keine Korrektur durch staatliche Einriffe sinnvoll, weil die "Kosten" darin bestehen, dass die Gewinne eines Unternehmens reduziert werden, und eben keine gesamtwirtschaftlichen Kosten entstehen. Pigou war relativ stur diesen Fehler einzugestehen \parencite[S. 153]{Johnson1960}. Er betrachtete externe Effekte außerdem stets eindimensional. Schließlich gibt es grundsätzlich zwei verschiedene Möglichkeiten, wie externe Effekte internalisiert werden können. Entweder zahlt der Verursacher der Gesellschaft einen Ausgleich für den Schaden, oder aber die Gesellschaft zahlt umgekehrt dem Verursacher die Kosten zur Vermeidung des Schadens. Ronald Coase (vgl. Kapitel \ref{Neue Institut}) kritisierte Pigou später vehement für diese einseitige Darstellung \parencite[S. 243]{Cansier1989}.

Der zweite neuartige Punkt in \textcite{Pigou1920} erscheint zwar reizvoll, wurde aber fast umgehend formal widerlegt: Der Versuch Pigou's eine gesamtwirtschaftliche Nutzenbetrachtung durchzuführen. Pigou beschäftigte sich dabei zunächst ausführlich damit den Begriff des Volkseinkommens, bzw. des Sozialprodukts zu definieren. In weiterer Folge wollte eine Funktion für die gesamtwirtschaftliche Wohlfahrt ableiten. Diese sei abhängig von der Höhe, der Verteilung und der Stabilität des Sozialprodukts \parencite[S. 42]{Pigou1920}. Der Ansatz klingt natürlich zunächst vielversprechend: Wir wissen ja, dass das Vermögen alleine nur eingeschränkte Aussagen über die tatsächliche Wohlfahrt, also den erlebten Nutzen, zulässt. Es war \textit{die} zentrale Errungenschaft der Neoklassik das Wertparadoxon der Klassiker zu überwinden und den Nutzen in den Vordergrund von Maximierungsentscheidungen zu stellen. Verlockend ist es daher eine "gesamtwirtschaftliche" Wohlfahrtsfunktion anzustreben. Das Bruttoinlandsprodukt (Volkseinkommen) als aggregierte Wertschöpfung der Bevölkerung in einem Staat, sagt wenig über die Wohlfahrt aus. Man könnte stattdessen die einzelnen Vermögen der Individuen als Nutzenwerte ausdrücken und das Aggregat dieser Werte als "nationalen Gesamtnutzen" - oder eben Wohlfahrt - interpretieren. \textcite[S. 48]{Pigou1920} legte also die individuelle Grenznutzen-Theorie auf die Gesamtwirtschaft um: "Das Gesetz des abnehmenden Grenznutzens lehrt uns, dass ein steigendes Sozialprodukt nur zu einem unterproportionalen Wohlfahrtsgewinn führt." Aus der Annahme des abnehmenden Grenznutzens lässt sich laut \textcite[S. 53]{Pigou1920} außerdem ableiten, dass eine zusätzliche Geldeinheit einer armen Person mehr Nutzen stiftet, als einer reichen Person. Oder mit anderen Worten: Neben einem möglichst hohem Gesamt-Volkseinkommen erhöht auch eine möglichst gleiche Verteilung der Einkommen zu einer höheren gesamtwirtschaftlichen Wohlfahrt. Daher sollte der Staat auch dahingehend eingreifen und zum Beispiel durch progressive Besteuerung die Einkommenshöhe angleichen. Pigou missachtete dabei allerdings die wesentlichen Erkenntnisse seiner Vorgänger: Nutzen ist nicht kardinal, also in Geldeinheiten ausgedrückt, messbar. Stattdessen behilft man sich mit ordinalen Nutzenkonzepten wie in Kapitel \ref{Pareto} dargestellt. Es ist ein Irrtum zu glauben, Pigou wäre sich dessen nicht bewusst gewesen. Er wusste, dass interpersonelle Nutzenvergleiche formal-mathematisch nicht möglich sind. Aber er war auch pragmatisch genug um anzuerkennen, dass zum Beispiel bei unverändertem Sozialprodukt eine gleichmäßigere Einkommensverteilung gesamtwirtschaftlich einen positiven Wohlfahrtseffekt hat. Auch eine bewusst vereinfachend einheitliche Einkommens-Nutzenfunktion dachte Pigou zumindest an \parencite[S. 237]{Pigou1920}. Das Konzept von \textcite{Pareto1906} plädiert hingegen für eine die Unzulässigkeit von interpersonellen Nutzenvergleichen. Man kann eben gerade \textit{nicht} behaupten, dass eine arme Person durch eine zusätzliche Geldeinheit mehr Nutzen generiert als ein Millionär. Der Effekt von Umverteilungsmaßnahmen auf die Gesamtwohlfahrt ist damit in der Neoklassik in keinster Weise abbildbar. Stark kritisiert \parencite[S. 123]{Robbins1932} wurde Pigou für seine Idee der gesamtwirtschaftlichen Wohlfahrtsfunktion eben, weil die formale Unzulänglichkeit seines Konzepts schon seit dem Werk von \textcite{Pareto1906} bekannt war. Interessant ist der Ansatz von Pigou aus heutiger Sicht aber möglicherweise dennoch. Natürlich, formal-mathematisch ist es anerkannt, dass Nutzen nicht kardinal gemessen werden kann. Das stattdessen angewendete  Konzept von \textcite{Pareto1906}, bzw. der modernere Ansatz mittels Grenzrate der Substitution nach \textcite{Hicks1934b} ist eine elegante und in formaler Hinsicht höchst erfolgreiche Lösung. Diese führt aber eben auch dazu, dass Fragen der Einkommensverteilung in der Neoklassik einfach keinen Platz haben. Gerade solche Fragen sind in den letzten Jahren aber wieder vermehrt in den Vordergrund getreten.

Mit der Veröffentlichung von "Wealth and Welfare" und gerade einmal 43 Jahren war Pigou allerdings bereits auf den Zenit seiner Karriere angekommen. Denn sein eigener Assistent, ein uns bereits aus dem letzten Kapitel bekannter, gewisser John Maynard Keynes, revolutionierte kurze Zeit später die Ökonomie. Pigou nahm während dieser Revolution eine eher unglückliche Position ein. Er war einer jener Ökonomen, der an der fast magischen Strahlkraft von Keynes und dessen Werk fast zerbrach. Ähnlich ging es übrigens Joseph Schumpeter, Michal Kalecki und in gewissen Maße auch Friedrich Hayek. Ab den 1930er Jahren wendete sich Pigou nämlich von der Weiterentwicklung der "Wohlfahrtsökonomie" ab und stattdessen anderen Inhalten zu. Getrieben wurde er dazu natürlich von der "Great Depression", später aber auch vom Aufstieg Keynes'.  
Pigou könnte als das genau Gegenteil von Keynes beschrieben werden. Nach erschütternden Ereignissen als Sanitäter im Ersten Weltkrieg, wurde er zum exzentrischen Einzelgänger \parencite[S. 153]{Johnson1960}. Der lockere Bloomsbury-Group-Lebemann Keynes trat schon diesbezüglich ganz anders auf. Auch wird Pigou in seiner Position als Berater als Vertreter veralteter Ideen beschrieben, der unter anderem die Wiedereinführung des Goldstandards empfahl \parencite[S. 232]{Cansier1989}. Eine Idee, die Keynes als "barbarisches Relikt" ansah und nach 1918 auch nicht mehr wirklich erfolgreich implementiert werden konnte \parencite[S. 232]{Cansier1989}. Zwar haben wir soeben gelesen, dass Pigou staatliche Eingriffe als erster moderner Ökonom in vielen Situationen befürwortete, in Bezug auf die "Great Depression" schlug er aber keine konjunkturfördernden Maßnahmen vor. Im Gegenteil, die hohe Arbeitslosigkeit erklärte er typisch neoklassisch als Missmatch zwischen Angebot und Nachfrage auf dem Arbeitsmarkt \parencite[S. 232]{Cansier1989}. Seine neoklassische Analyse der Arbeitslosigkeit (\textcite{Pigou1933}: The Theory of Unemployment) kam zur Unzeit, am Höhepunkt der "Great Depression". Es war die erste umfassende neoklassische Beschäftigungstheorie, aber gerade während der Weltwirtschaftskrise waren diese Erklärungsmuster unpassend. So wurde das Werk schließlich der Angriffspunkt schlechthin für Keynes. Überhaupt war Pigou, als führender Vertreter der neoklassischen Schule in den 1930er Jahren, \textit{die} Zielscheibe von Keynes' Kritik in der "General Theory" \parencite[S. 154]{Johnson1960}. Sehr häufig liest man darin über die "falschen Schlussfolgerungen von Prof. Pigou". Dieser reagiert trotzig und damit genau falsch. Anstatt sich auf Keynes' Theorien genauer einzulassen und diese dann eingehend zu analysieren, verfasste \textcite{Pigou1936} eine giftige Kritik über die Art und Weise wie Keynes seine Ideen darstellte, ohne dabei wirklich auf inhaltliche Unklarheiten tiefer einzugehen. Erst später änderte Pigou seine Meinung und akzeptierte die bahnbrechenden Erkenntnisse von \textcite{Keynes1936} \parencite[S. 154]{Johnson1960}. Sein späteres Werk \textcite{Pigou1941}: "Employment and Equilibrium", zum Beispiel, beinhaltet schon die Anerkennung keynesianischer Ideen, aber auch abweichende Meinungen, wie zum Beispiel die später als "Pigou-Effekt" beschriebene positive Wirkung sinkender Preise. Im Widerspruch zu Keynes führen sinkende Preise demnach zu einer höheren Nachfrage, weil die Kaufkraft des Geldes durch Deflation steigt. Der Pigou-Effekt ist nach wie vor in vielen Lehrbüchern zu finden, seine positive Wirkung blieb aber eher Minderheitenmeinung.

Angriffe auf das zweite Standbein von \textcite{Pigou1920}, die Notwendigkeit von Staatseingriffen bei Marktversagen, kamen nach dem Zweiten Weltkrieg von Seiten der neu aufkommenden Politischen Ökonomie (vgl. Kapitel \ref{Pol_Econ}). Zuerst kritisierte Coase die einseitige Betrachtung Marktversagen können nur durch Staatseingriffe beseitigt werden. Seiner Meinung nach wären marktwirtschaftlichen Lösungen ebenso möglich. Weiters wurde erstmals das Problem des möglichen Staatsversagens aufgegriffen. Zwar wurde anerkannt, dass es Situationen gibt, in denen der Markt zu nicht-effizienter Allokation führen würde, allerdings wurde zunehmend angezweifelt, dass Staatsvertreter für eine bessere Allokation sorgen würden. Für das Problem der natürlichen Monopole wurde von \textcite{Baumol1982} das alternative Konzept der angreifbaren Märkte entwickelt.

Pigou ging dennoch als revolutionärer Ökonom in die Geschichte ein. Er erweiterte die rein marktwirtschaftliche Analyse der neoklassischen Theorie um Aspekte des Marktversagens und der gesamtwirtschaftlichen Wohlfahrt. Er brachte damit den Staat als wichtigen Player ins Spiel, ohne aber von den grundsätzlichen Ideen der Mikroökonomie abzugehen. \textcite{Pigou1920} stellt damit den Beginn der mikroökonomischen Wirtschaftspolitik dar.


\section{Die moderne Wohlfahrtsökonomie}
\label{Wohlfahrt}

Mit dem Kapitel der Wohlfahrtsökonomie betritt die Volkswirtschaftslehre ganz neuen Boden. Schon rein methodisch unterscheidet sich die Wohlfahrtsökonomie von der klassischen und neoklassischen Ökonomie: Sie ist eine normative Theorie \parencite[S. 77]{Scitovsky1941}. Während positivistische Theorien die ökonomischen Vorgänge beobachten und daraus Gesetzmäßigkeiten ableiten, sind sich normative Theorien ihres Einflusses auf die ökonomischen Vorgänge bewusst. Positivistische Theorien sind nicht wertend und können dafür stets objektiv validiert werden. Die Wohlfahrtstheorie hingegen akzeptiert, dass Ökonomen wirtschaftspolitische Empfehlungen abgeben um "die Welt zu verbessern", sie enthalten stets auch "Werturteile". Ihre Fragestellungen können streng genommen nur subjektiv bewertet werden. Zum Beispiel: Ist eine gleichmäßigere Einkommensverteilung gerechter und besser für eine Gesellschaft?  Schon alleine diese Subjektivität machte die Wohlfahrtsökonomie von Anfang an umstritten und zwar auf zwei Ebenen. Erstens, die inhaltliche Ebene. Auf die Fragen der Wohlfahrtsökonomie gibt es oftmals plausible gegenteilige Antworten. Diese können zudem nicht abschließend, zum Beispiel mit empirisch-statistischen Untersuchungen, geklärt werden. Zweitens, gibt es eine übergeordnete Ebene, die in den Bereich der Philosophie vordringt. Macht es überhaupt Sinn, Fragen der Wohlfahrt, die man eben nie abschließend objektiv klären kann, in der Ökonomie zu behandeln? Bis heute ist die Wohlfahrtsökonomie ein stark beforschter Zweig der Wirtschaftswissenschaften. Wegen der fehlenden Möglichkeit einer Validierung allerdings wird bis heute heftig diskutiert, ob ihre Ergebnisse rein wissenschaftliche betrachtet einen wertvollen Beitrag liefern. 

Auch die Platzierung der Wohlfahrtsökonomie ist weder einfach noch klar. Die Einordnung in dieses Überkapitel macht vor allem wegen ihrer Ursprünge Sinn, alternativ könnte man sie auch als "Social Choice Theory" (Sozialwahl-Theorie)\footnote{Die "`Social Choice Theory"' ist eine Teildisziplin der Wohlfahrtsökonomie.} gemeinsam mit der "Public Choice Theory" im Kapitel \ref{Pol_Econ} darstellen. Wohlfahrtsökonomie und Public Choice Theorie beziehen sich auf ähnliche Grundkonzepte. Die Wohlfahrtstheorie ist allerdings Teil der Mikroökonomie, während die Public Choice Theorie der Neuen Politischen Ökonomie zugeordnet wird und damit auch in Teil \ref{Teil: NPO und Inst} behandelt wird. 

Die Arbeiten von \textcite{Pareto1906} und vor allem \textcite{Pigou1920} gelten bis heute als die Ursprünge der Wohlfahrtstheorie. In Hinblick auf Pigou's "`The Economics of Welfare"' wurde dies bereits im letzten Kapitel \ref{sec: Pigou} dargelegt. Kommen wir noch einmal kurz darauf zurück: Durch die Kritik von \textcite{Robbins1932} an den notwendigen interpersonellen Nutzenvergleichen galt dessen Ansatz rasch als widerlegt. Das Nutzenkonzept von \textcite{Pareto1906} und \textcite{Hicks1934a} galt als formal überlegen. Dieses etablierte sich als Teil der neoklassischen Mainstream-Theorie: Das Pareto-Prinzip und die ordinale Nutzenmessung waren der Kern dieser "`Neuen Wohlfahrtsökonomie"'. Anerkannt war, erstens, dass Wohlfahrtsökonomie die Gesamtwohlfahrt einer Gesellschaft zu bewerten als Ziel hat. Es war aber eben, nach der einflussreichen, kritischen Arbeit von \textcite{Robbins1932}, zweitens, auch bereits klar, dass die \textit{Summe} der empfundenen Nutzen (Wohlfahrt) nicht geeignet sei die Gesamtwohlfahrt zu messen (vgl. zum Beispiel \textcite{Lange1942}) Die Wohlfahrtsökonomie wurde daher ab Ende der 1930er Jahre auf neue Beine gestellt. Damals wurden die heute noch angeführten zwei Hauptsätze der Wohlfahrtstheorie als solche ausformuliert:
\begin{itemize}
	\item Erstes Wohlfahrtstheorem: Auf einem vollkommenen Markt - also mit vollkommener Information, bei vollkommener Konkurrenz und ohne externe Effekte - sind Gleichgewichtslösungen Pareto-optimal. Pareto-optimal (bzw. Pareto-effizient) sind Marktlösungen dann, wenn keine Person besser gestellt werden kann, ohne dass eine einzige Person schlechter gestellt wird.
	\item Zweites Wohlfahrtstheorem: Jedes Pareto-Optimum kann durch Marktgleichgewicht realisiert werden. Das heißt für jedes Pareto-Optimum existiert eine Einkommensverteilung, bei der alle Haushalte und Unternehmen ihre Nutzen bzw. Gewinne maximieren.
\end{itemize}
Die erstmalige Formulierung der Wohlfahrtstheoreme in ihrer modernen Form kann heute nicht mehr eindeutig zugeordnet werden. Das erste Theorem folgt im Prinzip schon direkt aus \textcite{Pareto1906}. Die erste mathematische Beweisführung wird häufig \textcite{Lange1942} zugeschrieben. 

Zwei verschiedene Ansätze prägten die frühe Zeit der "`Neuen Wohlfahrtsökonomie"'.  Der erste wurde von \textcite{Bergson1938} formuliert, aber erst durch \textcite{Samuelson1947} bekannt gemacht. Ähnlich wie bei Pigou gibt es hier eine "Soziale Wohlfahrtsfunktion", die den gesamtgesellschaftlichen Nutzen abbildet. Allerdings besteht dies nicht aus der \textit{Summe} der individuellen Nutzenwerte, sondern aus einem Vektor, der alle individuellen Nutzenwerte \textit{enthält}. Die Optimierungsaufgabe besteht nun nicht darin die Summe der Nutzen zu maximieren - was ja an den nicht-vergleichbaren und nicht-kardinalen individuellen Nutzenwerten scheitert - sondern darin, den optimalen Vektor zu bestimmen. Ein Vektor mit Nutzenwerten dominiert einen anderen Vektor immer dann, wenn er das Pareto-Kriterium erfüllt. Also kein Nutzen-Wert darf schlechter sein als im Vergleichsvektor \parencite[S. 9]{Suzumura2016}. Damit lässt sich eine "`Soziale Wohlfahrtsfunktion"' finden, die keine interpersonellen Nutzenvergleiche notwendig macht. Diese Bergson-Samuelson Sozial-Wohlfahrtsfunktion wurde von der ökonomischen Forschung bald wieder weitgehend verworfen. Anwendungsprobleme und Widersprüche zum Unmöglichkeitstheorem, auf das wir in Kürze eingehen werden, waren dafür verantwortlich. Weitgehend durchgesetzt hat sich hingegen der Ansatz, der auf \textcite{Kaldor1939} und \textcite{Hicks1940} zurückgeht. Noch heute wird oftmals auf das "`Kaldor-Hicks-Kriterium"', oder sogenannte "`Kompensation-Tests"' verwiesen, wenn es darum geht eine Kosten-Nutzen-Analyse vor Realisation eines Projektes durchzuführen. Worum geht es darin? Nun, das Pareto-Kriterium ist sehr streng wenn es darum geht Wohlfahrtsveränderungen herbeizuführen, da ja \textit{keine einzige} Person auch nur eine kleine Verschlechterung erfahren darf. \textcite{Hicks1940} und \textcite{Kaldor1939} argumentieren nun, dass eine Pareto-Verbesserung auch dann eintritt, wenn zwar der Wohlfahrtsverbesserung von Person A eine Wohlfahrtsverringerung bei Person B entgegensteht, die Wohlfahrtsverbesserung bei A aber zu einem Teil abgeschöpft wird um damit die Wohlfahrtsverringerung bei B zu kompensieren. Auch an diesem Konzept gibt es Kritikpunkte \parencite{Baumol1946}. Der erste ist inhaltlicher Natur. Während \textcite{Pigou1920} noch explizit darauf achtete bei seiner Form der Wohlfahrtsökonomie auch eine ethische Komponente zu umfassen, fehlt dies in der "Neuen Wohlfahrtsökonomie" gänzlich. Für Pigou war es klar, dass bei sonst konstanter Wohlfahrt eine Änderung der Einkommensverteilung zugunsten armer Haushalte eine höhere Gesamtwohlfahrt impliziert. Das darf man beim Kaldor-Hicks-Kriterium nicht annehmen. Wessen Wohlfahrt verringert wird um Kompensation bei einem Geschädigten zu erreichen und ob diese Kompensation auch tatsächlich realisiert wird, ist nicht Teil der Überlegungen bei Kaldor und Hicks \parencite[S. 11]{Suzumura2016}. Der zweite Kritikpunkt war ein rein formal-logischer. \textcite{Scitovsky1941} zeigte anhand eines einfachen Beispiels, dass es Fälle gibt, bei denen die Rückabwicklung einer ursprünglich Wohlfahrts-steigernden Maßnahme ebenfalls zu einer Wohlfahrtssteigerung führt. Das Kaldor-Hicks-Kriterium ist dann inkonsistent bei der Bestimmung der Wohlfahrtseffekte ökonomischer Maßnahmen. Dieses Phänomen wurde als Scitovsky-Paradoxon bekannt \parencite[S. 12]{Suzumura2016}.

Anfang der 1950er Jahre wurde die Wohlfahrtstheorie von Kenneth Arrow mit dessen Unmöglichkeitstheorem um eine vollkommen neue Problematik erweitert. Seine Arbeit war übrigens auch ein Anstoß für die neue Forschungsrichtung der "`Neuen Politischen Ökonomie"' ("`Public Choice Theory"'), die in Kapitel \ref{Neue_Politik} behandelt wird. Konkret auf die Wohlfahrtsökonomie angewendet in \textcite[S. 329]{Arrow1950} und wenig später in seinem bahnbrechendem Werk \parencite{Arrow1951} dargestellt, zeigt Arrow auf, dass rationales Wahlverhalten zur Unmöglichkeit konsistenter demokratischer Entscheidungen führt. Dieses Problem wurde schon von \textcite{Condorcet1785} erkannt: Wenn drei Personen bei drei alternativen Abstimmungsmöglichkeiten A, B und C jeweils eine andere Reihenfolge wählen, so bevorzugt eine Mehrheit von zwei Personen A gegenüber B und B gegenüber C. Aber es findet sich auch eine Mehrheit, die C gegenüber A bevorzugt. \textcite{Arrow1950} beschreibt, dass nicht nur spezielle Probleme, wie das Scitovsky-Paradoxon beim Kaldor-Hicks-Kriterium, oder das Condorcet-Paradoxon bei der Bergson-Samuelson Sozial-Wohlfahrtsfunktion, Probleme bereiten. Stattdessen gibt es für jedes Entscheidungskriterium, das darauf basiert, die individuellen Präferenzen von Individuen aggregiert heranzuziehen um eine demokratische Lösung zu finden, Beispiele, die Inkonsistenzen aufweisen \parencite[S. 330]{Arrow1950}. Das sogenannten "`Arrow'sche Unmöglichkeitstheorem"' - und damit auch die "`Social Choice Theory"' im engeren Sinn\footnote{\textcite{Fleurbaey2021} schreiben richtigerweise, dass die Social Choice Theory bereits durch zwei Journalbeiträge von Duncan Black im Jahr 1948 (\textcite{Black1948a} und \textcite{Black1948b}) begründet wurden. Darin werden Mehrheitsregeln und spezielle Mehrheitsregeln formal-logisch untersucht \parencite{Fleurbaey2021}.} - waren geboren. \textcite{Arrow1950} argumentiert damit, das nicht nur kardinale Nutzenmessung unmöglich ist, sondern auch ordinale Nutzenmessung - wie in der "Neuen Wohlfahrtsökonomie" angewendet - zumindest problematisch ist. Was hat es mit dem Unmöglichkeitstheorem auf sich? \textcite{Arrow1950} erstellt darin ein logisches System mit fünf Bedingungen. Diese beinhalten rein logische Statements, aber auch die Bedingung, dass die Individuen in einer Gesellschaft rational handeln, sowie frei über ihre Präferenzen entscheiden dürfen. Weiterhin gelten die Wohlfahrtstheoreme, also das Pareto-Prinzip, sowie, dass  interpersonelle Nutzenvergleiche nicht sinnvoll möglich sind. Die Aufgabe besteht nun darin eine Soziale Wohlfahrtsfunktion aus den Präferenzen der Individuen abzuleiten, die zudem die fünf genannten Bedingungen erfüllt \parencite[S. 339]{Arrow1950}. Mit einer konsequenten Beweisführung zeigt \textcite[S. 339ff]{Arrow1950}, dass demokratisches Abstimmungsverhalten zu keiner konsistenten "Sozialen Wohlfahrtsfunktion" führen wird. So eine Funktion muss entweder als gegeben angenommen werden, oder aber eine Autorität, also ein Diktator, übernimmt die Präferenzordnung für alle Individuen einer Gesellschaft. Zusammengefasst: Man kann aus individuellen Präferenzen keine Soziale Wohlfahrtsfunktion ableiten. Man kann damit in einer Demokratie keine gesamt-gesellschaftliche Nutzenmaximierung durchführen. Oder mit anderen Worten: Innerhalb des Konzepts der neoklassischen Mikroökonomie lässt sich keine befriedigende Wohlfahrtsökonomie etablieren. Tatsächlich schien die Wohlfahrtsökonomie in der Folge in einer Sackgasse. Und was deren weitere Entwicklung innerhalb der Mainstream-Ökonomie angeht, muss man das wohl auch bestätigten. 

Zwar gab es weiterhin verschiedenste Ansätze zur Wohlfahrtsökonomie, diese im Detail zu analysieren wäre hier jedoch nicht zielführend. \textcite{Fleurbaey2021} folgend sei aber erwähnt, dass es einen starken Zweig in Richtung Spieltheorie gibt. Wie schon erwähnt, sind wohlfahrtstheoretische Überlegungen mit Ansätzen der Neuen Politischen Ökonomie verwandt. Aber auch in den Gebieten Logik und "`Computergestützte Sozialwahl"' gibt es abgeleitete Forschungszweige. Alternativ dazu gibt es auch Ansätze, die vom Inhalt her mit der Wohlfahrtstheorie in Verbindung gebracht werden, allerdings klar im Bereich der Heterodoxen Ökonomie anzusiedeln sind. Erwähnenswert ist hier die Glücksforschung, die seit den 1950er Jahren vor allem von \textcite{Easterlin1974} durchgeführt wurde. Auf Basis von Umfragen wird versucht zu erheben, was in Menschen tatsächlich Glücksgefühle auslöst. Wohlfahrt wird nicht mehr ausschließlich anhand des Sozialprodukts gemessen. Diese Ansätze unterscheiden sich fundamental von den davor dargestellten. Zum einen sind diese interdisziplinär, eine Mischung aus Ökonomie, Psychologie und Verhaltenswissenschaften. Zum anderen sind die Ansätze empirisch getrieben und verzichten auf die neoklassischen Modellannahmen, wie zum Beispiel die Annahme von Rationalität. Als modernen Ansatz zur Wohlfahrtsökonomie könnte man auch die Neuroökonomie ansehen. Mit bildgebenden Verfahren werden hier Vorgänge im Gehirn hinsichtlich sozialer Präferenzen und bei ökonomischen Entscheidungen gemessen. Dieser Forschungsbereich wird der Verhaltensökonomie zugezählt. Möglicherweise bietet der Ansatz in Zukunft aber die Unmöglichkeit das Problem der interpersonellen Nutzenvergleiche zu überwinden. Dieser Zweig der Wissenschaft publizierte zuletzt sehr erfolgreich in hoch angesehenen Journalen \parencite{Fehr2000, Fehr2003}.

Als historisch bedeutendste Weiterentwicklung, aber auch klare Neuorientierung der Wohlfahrtsökonomie, müssen die Arbeiten von Amartya Sen genannt werden. Neben dem extrem einflussreichen Werk des Polit-Philosophen \textcite{Rawls1971}: "`A Theory of Justice"',  führte vor allem Sen die Wohlfahrtsökonomie in den 1970er Jahren auf neue Wege. Zunächst - Anfang der 1970er Jahre - in Richtung Erweiterung der "`Social Choice Theory"'. In einem kurzen Beitrag führt \textcite{Sen1970} das Pareto-Prinzip in gewissen Situationen ad absurdum. Er führt dazu das Entscheidungskriterium des "`Liberalismus"' ein. Er selbst gibt zu, dass der Name etwas unglücklich gewählt ist. Es geht hierbei einfach darum, dass es Entscheidungen gibt, die nur auf eine Person, nämlich den Entscheider selbst, Auswirkungen hat. Am Beispiel des Buches "`Lady Chatterley's Lover"' - einer der ersten Erotik-Romane, der 1928 erschien und in vielen Ländern zensiert wurde - zeigt \textcite{Sen1970}, dass eine individuelle Präferenzordnung im Sinne des ordinalen Nutzenprinzips, das Pareto-Prinzip und das von Sen eingeführte liberale Prinzip nicht miteinander vereinbar sind. \textcite{Sen1970} wollte damit vor allem die Unzulänglichkeiten des Pareto-Prinzips als ökonomisches Entscheidungsprinzip aufzeigen. 
\textcite{Sen1970b} weichte in weiterer Folge die seit \textcite{Robbins1932} uneingeschränkt geltende Unzulässigkeit von interpersonellen Nutzenvergleichen auf. Dabei ging er streng formal-mathematisch vor. Insgesamt vertrat er den Standpunkt, dass interpersonelle Nutzenvergleiche in gewissem Ausmaß durchaus realistisch sind. "`Wir sollten keine großen Zweifel daran haben, dass Kaiser Nero's Nutzen von der Feuerbrunst in Rom kleiner war, als die Summe der Nutzenverluste aller anderen Römer"', meinte in seiner Rede anlässlich der Verleihung des Nobelpreises 1998 \parencite[S. 356]{Sen1999}. Sen hatte wenig Berührungsängste mit philosophischen Ansätzen. Er vertritt den Standpunkt, dass Wohlfahrtsökonomie auch ganz bewusst ethische Fragen behandeln darf. Dementsprechend entdeckte er Verteilungsfragen und Fragen der Armut für die Wohlfahrtsökonomie wieder. Durch viele Publikationen in hochwertigen Journalen und natürlich nicht zuletzt durch die Vergabe des Nobelpreises an Sen im Jahr 1998 erlangte die Wohlfahrtsökonomie in der öffentlichen Wahrnehmung wieder an Bedeutung. Allerdings entfernte sich die Wohlfahrtsökonomie damit immer stärker von der Mainstream-Ökonomie. Derzeit spielt die Wohlfahrtsökonomie innerhalb der Mainstream-Forschung maximal eine Nebenrolle, wie \textcite{Atkinson2011, Atkinson2001} kritisierte. Sen schuf aber vor allem auch Verbindungen zu den Arbeiten zur Einkommensverteilung von Anthony Atkinson (vgl. Kapitel \ref{Ungleichheit}) und zu den Arbeiten zur Armut von Angus Deaton (vgl. Kapitel \ref{Armut}). Man kann diese Bereiche als Weiterentwicklung der Wohlfahrtstheorie sehen, so gesehen gewinnt diese Disziplin in letzter Zeit wieder an Bedeutung.


\section{Bäume, die in den Himmel wachsen: Die Neoklassische Wachstumstheorie}

Die Wachstumstheorie ist eine \textit{der} zentralen Disziplinen innerhalb der Ökonomie. Der Kern dieses Kapitels beschäftigt sich großteils mit dem historisch gesehen bedeutendsten Modell-theoretischem Erklärungsansatz: Der Exogenen Wachstumstheorie, häufig auch schlicht neoklassische Wachstumstheorie genannt. Diese entstand in den 1950er Jahren und wird bis heute in den meisten Lehrbüchern als Mainstream-Modell dargestellt. Auch wenn ihr die Endogenen Wachstumsmodelle diesen Rang in den letzten Jahrzehnten zunehmend ablaufen. 

Bevor wir uns den theoretischen Modellen zuwenden, gehen wir aber kurz fundamentaler auf das Phänomen Wirtschaftswachstum ein. Mehr als andere ökonomische Disziplinen beschäftigt es Ökonomen aus verschiedensten Richtungen. Tatsächlich ist die Geschichte der modernen Ökonomie - wenn man diese mit Adam Smith beginnen lässt - eine Geschichte des Wirtschaftswachstums. Seit der industriellen Revolution, die eben auch grob zusammenfällt mit der Publikation der "`Wealth of Nations"' in der zweite Hälfte des 18. Jahrhunderts, erlebt die wirtschaftliche Entwicklung auf einen beispiellosen Aufwärtstrend. Natürlich unterbrochen durch Wirtschaftskrisen, Kriege, Revolutionen und auch Seuchen. In keiner geschichtlichen Epoche davor konnte eine vergleichbare Entwicklung beobachtet werden. Ebenso interessant ist die Tatsache, dass der wirtschaftliche Aufstieg nicht gleichmäßig über den gesamten Globus erfolgte, sondern stattdessen einzelne Staaten und Wirtschaftsblöcke einen enormen Aufschwung erlebten, während dieser in anderen Teile der Erde bis heute ausblieb. Diese Thematik ist durchaus viel beforscht. Hier können die verschiedenen Erklärungsansätze dazu nur kurz angestreift werden: Bekannt sind die "`Fünf Stadien der Entwicklung"' - von der Tauschgesellschaft zum Massenkonsum - des insgesamt recht umstrittenen Politikers und Ökonomen \textcite{Rostow1960}. Der Wirtschaftshistoriker \textcite{Gerschenkron1962} nennt ebenso einen "`Modernisierungsanstoß"' als notwendigen Auslöser für einen dann folgenden großen Entwicklungssprung. Zwar wahrten die beiden Autoren eine kritische Distanz zueinander, ihre Arbeiten werden dennoch unter dem Begriff "`Modernisierungstheorie"` zusammengefasst. Im Gegensatz dazu entwickelte ab 1949 vor allem der Argentinier Raul Prebisch die "`Dependenztheorie"', die davon ausgeht, dass die Rückständigkeit der "`Peripherie-Länder"' durch deren Abhängigkeit von den Industriestaaten ("`Zentrum"') zustande kommt. Einflussreich schlug in dieselbe Kerbe die "`Welt-System-Theorie"' von \textcite{Wallerstein1974}. 

Die sogenannte "`Great Divergence"', also das Auseinanderdriften der Entwicklungsstadien in verschiedenen Staaten oder Wirtschaftsblöcken, spielte schließlich auch eine Rolle in der Evolution der Modell-theoretischem Erklärungsansätze, die wir in weiterer Folge im Detail betrachten. Das Solow-Modell postuliert eigentlich, dass es zu einer Angleichung des wirtschaftlichen Entwicklungsstandes verschiedener Länder kommen sollte. Allerdings ist dies eher nicht zu beobachten, was im "`Lucas Paradoxon"' formuliert wurde. Die Endogenen Wachstumstheorien (vgl. Kapite \ref{sec: endogene}) erklären die anhaltende Divergenz mittels unterschiedlichen Bildungsmöglichkeiten in unterschiedlichen Ländern. Einen anderen Ansatz verfolgt der "`Neue Institutionalismus"' von Daren Acemoglu (\ref{sec: Neue Inst}), der die Bedeutung von funktionierenden Institutionen als Voraussetzung für Wirtschaftswachstum hervorhebt. 

Die Erklärung langfristigen Wachstums, ist mehr als die meisten anderen ökonomischen Teilbereiche, eine Disziplin der Wirtschaft\textit{geschichte}, wie \textcite{Baumol1986} beschreibt. Zur Analyse der langfristigen, wirtschaftlichen Entwicklung sind aber auch entsprechende Datensätze notwendig. Bekannt geworden sind in diesem Zusammenhang die Arbeiten von Angus Maddison \parencite{Maddison2010}, der bis an sein Lebensende der Aufgabe nachging historische BIP-Daten zu sammeln oder zu rekonstruieren. Das Projekt wird seit seinem Tod im Jahr 2010 von ehemaligen Kollegen weitergeführt \parencite{Maddison2023}.

Nach diesem kurzen Exkurs zum eigentlichen Inhalt des Kapitels: Wirtschaftswachstum aus Modell-theoretischer Sicht. Die Vorgeschichte der Wachstumstheorien ist lang und vielfältig. Die bedeutendsten Ökonomen verschiedenster Richtungen hatten sich allesamt auch mit Wachstumstheorien beschäftigt. So etwa Adam Smith, der optimistisch im Hinblick auf langfristiges Wachstum war. Was die Klassiker angeht, wird auch die pessimistische Prognose Malthus' bis heute häufig zitiert, obwohl sich diese bislang nicht bewahrheitet hat. Nicht zuletzt widmete sich auch Karl Marx Wachstumstheorien. Mit der marginalistischen Revolution verlagerte sich die ökonomische Forschung auf die mikroökonomische Ebene. Allgemeines Wirtschaftswachstum wurde damit nicht mehr explizit im Rahmen einer eigenen Theorie thematisiert, aber natürlich implizit zum Beispiel im Rahmen der Produktivitätstheorien. Insbesondere die Arbeiten von \textcite{Wicksteed1894} und \textcite{Wicksell1922} waren in dieser Ära wichtige Wegbereiter der modernen Wachstumstheorie, wie wir gleich sehen werden. Während der "`Great Depression"' und der hohen Zeit des frühen Keynesianismus dominierten Theorien zur Krisenerklärung. Zwar entstanden in dieser Zeit die wichtigen (Vorläufer-) Arbeiten zur Wachstumstheorie - insbesondere die Cobb-Douglas-Produktionsfunktion, sowie das Ramsey-Modell - die hohe Bedeutung, die diese bis heute genießen, wurde ihnen allerdings erst nach dem Zweiten Weltkrieg zuteil. Betrachten wir zunächst die Entwicklung der Produktionsfunktion.

\subsection{Die Cobb-Douglas-Produktionsfunktion} \label{sec: Cobb-Douglas-Produktionsfunktion}
Cobb-Douglas-Produktionsfunktionen sind noch heute in jeden Ökonomie-Studium omnipräsent. Sie scheinen eines jener Konzepte, die die Evolution der Ökonomie unbeschadet überstehen. Interessant ist hierbei, dass der Name \textit{Cobb-Douglas}-Produktionsfunktion dabei nicht auf die eigentlichen Entwickler des \textit{theoretischen} Konzepts zurückgeht. Paul H. Douglas war ein Ökonomie-Lehrender an verschiedenen US-amerikanischen Universitäten, der, abgesehen von den Arbeiten zur Produktionsfunktion, keine bedeutenden wirtschaftswissenschaftlichen Forschungsarbeiten hervorbrachte. Douglas war allerdings als Politiker erfolgreich und lange Zeit für die Demokratische Partei im US-Senat. In \textcite{Douglas1976} beschreibt er, wie er im Jahr 1927, eher zufällig, die so langlebige Produktionsfunktion entwickelte. Er erstellte und verglich Indexzahlen für die Anzahl der Arbeitnehmer, sowie Höhe des eingesetzten Kapitals für amerikanische Produktionsunternehmen über die Jahre 1899 bis 1922. Als er die Werte in logarithmischer Form ins Verhältnis zu einem Produktionsindex setzte, bemerkte er, dass der Abstand der drei Funktionen über die Zeit annähernd konstant blieb \parencite[S. 904]{Douglas1976}. Er zog den befreundeten Mathematiker Charles Cobb zurate, wie denn dieses Ergebnis zu interpretieren sei. Dieser schlug vor den empirischen Zusammenhang in eine Formel zu gießen:

$$P = b*L^k*^{1-k}$$

Die Produktion$P$ ergibt sich also aus den Faktoren Arbeit$L$ und Kapital$K$, die jeweils mit $k$ gewichtet wurden. Diese Formel ist eben bis heute als Cobb-Douglas-Produktionsfunktion weltweit bekannt. Die Gewichtung der Faktoren mit $k$ und $(1-k)$ bildet hierbei konstante Skalenerträge ab. Das heißt, wenn beide Produktionsfaktoren verdoppelt werden, verdoppelt sich auch der Output der Produktion$P$. Die Annahme konstanter Skalenerträge ist bis heute umstritten. Später gingen Cobb und Douglas dazu über die Exponenten der Faktoren unabhängig voneinander zu bestimmen, stellten aber empirisch fest, dass die Summe dieser Exponenten ohnehin jeweils ungefähr eins beträgt \parencite[S. 904]{Douglas1976}.

Ihre Erkenntnisse veröffentlichten die beiden 1928 im Journal of Political Economy als "`A Theory of Production"' \parencite{Cobb1928}. Das der darin empirisch gezeigte Zusammenhang bereits zuvor von \textcite{Wicksteed1894} und \textcite{Clark1899} in ähnlicher Form postuliert wurde, war den beiden bekannt und sie zitierten entsprechend auch deren Arbeiten \parencite[S. 151]{Cobb1928}. Die Beiträge von \textcite{Wicksell1922}, der die Theorie bereits um 1890 darstellte und heute als der \textit{eigentliche} Urheber der Produktionsfunktion gilt, waren den beiden hingegen nicht bekannt. Zu Beginn wurde die Arbeit wenig akzeptiert, auch weil empirische Daten weitgehend fehlten um deren Aussagekraft überhaupt zu verifizieren. Vor allem der erste Ökonomie-Nobelpreisträger und damals führende quantitative Ökonom Ragnar Frisch kritisierte die Theorie als weitgehend nutzlos und die Aussagen, aufgrund der dünnen Datenlage als nicht haltbar \parencite[S. 905]{Douglas1976}. Cobb und Douglas konnten mit weiteren empirischen Untersuchungen die Gültigkeit ihrer Produktionsfunktion unterlegen. Ihre bahnbrechende Bedeutung erlangte sie aber erst später - dann bereits unter dem Namen "`Cobb-Douglas-Produktionsfunktion"' - vor allem als Ausgangspunkt der neoklassischen Wachstumstheorie. Interessantere Nebenaspekt: Die Cobb-Doulgas-Produktionsfunktion wird seit ihrer Entwicklung sowohl in der Mikro- als auch in der Makroökonomie herangezogen. Deren Anwendung kann also praktisch als Vorläufer der später - in den 1970er Jahren - etablierten Mikrofundierung der Makroökonomie gesehen werden.

\subsection{Solow: Technischer Fortschritt als Wachstumsquelle} \label{sec: Solow-Modell}

HIER WEITER


Solow-Swan-Modell

Ausgangspunkt ist eine Produktionsfunktion, wie jene, die wir gerade in Kapitel \ref{sec: Cobb-Douglas-Produktionsfunktion} kennen gelernt haben.  Der Output – gesamtwirtschaftlich das BIP – wird durch verschiedene Inputs – standardmäßig in der Neoklassik Arbeit und Kapital – hervorgebracht. Der Output ergibt sich also aus eine Kombination der beiden Inputfaktoren, Ökonomen würden sagen: "`Der Output ist eine Funktion der Inputfaktoren"'. 
Folgende Überlegung macht recht schnell klar, warum man mit diesem einfachen Modell stetige Wachstumsraten nicht erklären kann: Angenommen der Output ist einfach eine Addition der beiden Inputfaktoren Arbeit und Kapital. Möchte ich den Output verdoppeln, so müsste ich \textit{beide} Inputfaktoren Arbeit und Kapital jeweils verdoppeln. Ökonomen sprechen hier von konstanten Skalenerträgen. 
Was passiert aber wenn nur einer der beiden Input-Faktoren steigen kann? Gesamtwirtschaftlich könnte man argumentieren, dass der Produktionsfaktor Arbeit durch die Bevölkerungszahl begrenzt ist. Wenn man dies in unserer Überlegung berücksichtigt, würden wir folglich nicht beide Inputfaktoren gleichzeitig erhöhen, sondern nur einen, nämlich Kapital. Erhöhen wir diesen Inputfaktor nun um eine Einheit und der andere Inputfaktor bleibt gleich, so steigt der Gesamtoutput zwar selbstverständlich an, allerdings pro zusätzlicher Einheit um einen immer geringeren Prozentsatz. Intuitiv ist das leicht verständlich: Erhöhe ich bei gleichbleibender Mitarbeiterzahl ständig das Kapital – zum Beispiel die Anzahl der Computer – dann wird der erste eingesetzte Computer einen hohen Zuwachs an Produktivität bringen. Mit jedem weiteren Computer wird die Produktivität zwar weiter steigen, allerdings mit immer geringerer Zuwachsrate. Wenn jeder Mitarbeiter mehr als einen Computer besitzt, wird der Produktivitätszuwachs verschwindend gering werden. Diesen Zusammenhang bezeichnen Ökonomen als „Abnehmenden Grenzertrag“.
Das würde aber bedeuten, dass bei ungefähr gleich bleibender Arbeitsbevölkerung – eine Annahme, die man zumindest für die mittlere Frist in Industriestaaten, bedenkenlos machen kann – die Wirtschaftsleistung stagnieren sollte. Geht man davon aus, dass der Kapitaleinsatz ständig steigt, wäre zwar stetiges Wachstum möglich, aber nur mit immer geringer werdenden Wachstumsraten \footnote{Ständig steigender Kapitaleinsatz wäre nur mit steigenden Sparquoten erklärbar. In einer Ökonomie ohne Außenhandel gilt ja, dass das Sparen der Haushalten den Investitionen der Firmen entspricht. Investitionen wiederum bedeuten einen Aufbau von Kapital. Spezielle Beobachtungen von Fällen von Wirtschaftswachstum werden tatsächlich darauf zurückgeführt, dass die Sparquoten gestiegen sind. So ist zum Beispiel die Ökonomie in der stalinistischen Sowjetunion tatsächlich beträchtlich gewachsen. Da man aber keine wesentlichen technologischen Vorsprünge des Landes in dieser Zeit ausmachen kann, vermutet man, dieses Wachstum sei eben alleine auf den Anstieg der Sparquote zurückzuführen.}. Langfristig würden die Zuwachsraten aber gegen Null tendieren, womit auch in diesem Fall die Wirtschaftsleistung stagniert.

Bisher haben wir aber eine Möglichkeit außer Acht gelassen: Nämlich, dass die eingesetzten Maschinen (hier als Synonym für Kapital verwendet) immer besser werden. Tatsächlich wird eine Arbeitskraft mit zwei Computern nicht wesentlich produktiver sein, als mit einem Computer. Sie könnte aber mit einem \textit{besseren} Computer wesentlich produktiver sein. Es könnte sich also nicht nur die \textit{Menge} des Kapitals verändern, sondern auch dessen \textit{Qualität}. Dies nennen wir „technischen Fortschritt“ \footnote{Technischer Fortschritt umfasst nicht nur die Weiterentwicklung bestehender Produkte zu „besseren“ Produkten, sondern auch die Einführung neuer Produkte}.
Berücksichtigt man diesen Umstand, kommt man zu dem Ergebnis, dass stetiges Wirtschaftswachstum nur dann möglich ist, wenn sich die eingesetzten Kapitalgüter – also zum Beispiel Maschinen, Computer, Transportmittel, Kommunikationsmittel – immer weiter verbessern.
Der Inhalt des „technischen Fortschritts“, also was sich wie verbessert – ist allerdings ist nicht Teil der Ökonomie. Die ersten Wachstumstheorien haben sich also damit abgefunden festzustellen, dass technischer Fortschritt für Wachstum notwendig ist, dieser selbst allerdings nicht durch ökonomisches Handeln beeinflusst werden kann. Der technische Fortschritt wurde also als „exogen“ betrachtet. Daher der Name „exogene Wachstumstheorie“.


Nachteil: Konnte Divergenz zwischen armen und reichen Ländern nicht erklären (Kontra-Argument: Mankiw, der das exogene Wachstumsmodell noch befürwortet.)


Leontieff: Input-Output-Analyse.



Vorläufer: Von-Neumann Growth-Theory: 
"`Über ein ökonomisches Gleichungssystem und eine Verallgemeinerung des Brouwerschen Fixpunktsatzes"',  1937, in K. Menger, editor, Ergebnisse eines mathematischen Kolloquiums, 1935-36. [English 1945 trans. as "A Model of General Economic Equilibrium", RES].






Ramsey–Cass–Koopmans Modell:
Das Ramsey–Cass–Koopmans Modell (RCK) ähnelt dem Solow-Modell stark, jedoch wird die Dynamik der Wirtschaftsaggregate durch Entscheidungen auf mikroökonomischer Ebene bestimmt. Das RCK-Modell unterscheidet sich vom Solow-Modell in einem entscheidenden Punkt, es ist mikrofundiert.



Zusammengefasst: Zuerst keynesianisches HArrod-Domar. Dann Solow-Swan. Dann lange nichts. Dann endogene Wachstumstheorie, aber mit Verteidiung der Solow-Modelle vor allem durch Mankiw. Außerdem neu: Acemoglu-Ansätze. 
In den 1970er Jahren schließlich Endogene Wachstumstheorien (vgl Kapitel). Paul Romer. Zuletzt Ansätze aus dem Neuen Institutionalsimus vor alle Acemoglu



\section{Die Welt im Arrow-Debreu-Gleichgewicht}
\label{Arrow-Debreu}
Ursprünglichste Form: Walras.
Vorarbeiten von Neumann (1937, siehe oben) und Leontieff. Danach: \textit{Ramsey-Cass-Koopmans!}

Arrow-Debreu:
Das Arrow-Debreu Gleichgewichtsmodell (auch: Arrow-Debreu-McKenzie-Modell) ist ein mikroökonomisches Modell der gesamten Volkswirtschaft. Es ist nach Gérard Debreu und Kenneth Arrow sowie Lionel W. McKenzie benannt, stellt eine Weiterentwicklung des von Léon Walras entwickelten walrasianischen Gleichgewichtsmodells dar und untersucht einen gesamtwirtschaftlichen Gleichgewichtszustand. 
Das Modell erweitert das allgemeine Gleichgewichtsmodell um unsichere Erwartungen und zustandsabhängige Größen und ist damit für die Finanzierungstheorie von großer Bedeutung. Es zeigt, dass es in einer Marktwirtschaft unter idealisierenden Bedingungen nicht möglich ist, jemanden besserzustellen, ohne jemand anderen schlechterzustellen. Kurz gesagt ist ein Marktgleichgewicht ein Pareto-Optimum. 
















%%%%%%%%%%%%%%%%%%%%% chapter.tex %%%%%%%%%%%%%%%%%%%%%%%%%%%%%%%%%
%
% sample chapter
%
% Use this file as a template for your own input.
%
%%%%%%%%%%%%%%%%%%%%%%%% Springer-Verlag %%%%%%%%%%%%%%%%%%%%%%%%%%

\chapter{Monetarismus}
\label{Monetarismus}

\section{Friedman}

Zunächst: Steuerung der Wirtschaft über die Geldmenge.

Danach: adaptive Erwartungen: Leute erwarten Fortschreibung der derzeitigen Inflation. Daher wirksam nur Änderungen der Inflationsrate. Erweitert durch die rationalen Erwartungen, die das noch ausweiten
		% Monetarismus							!!! KORREKTURLESEN
%%%%%%%%%%%%%%%%%%%%% chapter.tex %%%%%%%%%%%%%%%%%%%%%%%%%%%%%%%%%
%
% sample chapter
%
% Use this file as a template for your own input.
%
%%%%%%%%%%%%%%%%%%%%%%%% Springer-Verlag %%%%%%%%%%%%%%%%%%%%%%%%%%

\chapter{Neoklassische Finanzierungstheorie}
\label{Finance}



Bis heute hat die Neoklassische Finanzierungstheorie (Neoklassische Finance, Modern Finance, oder schlicht Finance) eine interessante Entwicklung durchgemacht. Der Forschungsaufwand war in dieser Disziplin stärker in fast allen anderen ökonomischen Feldern, vor allem auch in privatwirtschaftlichen Institutionen wird hier viel Forschung betrieben. Dies ist wenig überraschend, schließlich erhoffen sich viele bis heute durch Finanzanlagen reich zu werden. Dies ist insofern interessant, als die - bis heute gültige - grundlegende Annahme davon ausgeht, dass man zukünftige Kursentwicklungen von Assets in keiner Weise vorhersehen kann, sich diese stattdessen entlang eines reinen Zufallspfades entwickeln. In der Praxis ist das Angebot an Finanzmarktprodukten zweigeteilt. Zum einen gibt es quantitativ-wissenschaftlich geführte Fonds. Aber daneben gibt es noch immer einen erheblichen Zulauf zu "`Gurus"' oder Anlageberatern, die überzeugt davon sind, den Markt schlagen zu können. Die neoklassische Finance ist und war aber auch von wissenschaftlicher Seite regelmäßig vehementer Kritik ausgesetzt. Auch dies ist nicht überraschend: Schließlich hat die wissenschaftliche Weiterentwicklung in diesem Gebiet in keinster Weise dazu beigetragen, die Anzahl von Kursstürzen an den Börsen zu verringern. Weder der "`Schwarze Montag"' im Jahr 1987, noch die "`Dot-Com-Blase"', die im Jahr 2000 platzte, noch der Börsencrash zu Beginn der "`Great Depression"' 2007 passen so recht in das Konzept der "`Effizienten Finanzmärkte"'. Nicht zuletzt deshalb haben sich mit der Behavioral Finance (vgl. Kapitel \ref{Behavioral}) und innerhalb des Post-Keynesianismus (vgl. Kapitel \ref{Post-Keynes}) starke heterodoxe Ansichten zur Finanzierungstheorie gebildet, deren Bedeutung bis heute nicht zu unterschätzen ist. Die "`Modern Finance"' ist tatsächlich von vielen Seiten her angreifbar und - wie jede Theorie - weit weg davon die Realität vollständig abbilden zu können. Ihr wesentlicher Vorteil liegt aber darin die Konzepte von Nutzen, Ertrag und Risiko unter recht plausiblen Annahmen in mathematisch extrem eleganter Form miteinander zu vereinen. 


\section{Vorläufer der Finanzierungstheorie}
\label{FisherundKnight}

Die erste theoretische Rechtfertigung des Zinses, die jener der Gegenwart entspricht, wird dem Vertreter der Österreichischen Schule (vgl. Kapitel \ref{Austria}) Eugen von Böhm-Bawerk zugeschrieben. So wie die meisten Ökonomen zur damaligen Zeit, versuchte er eine "`Gesamterklärung"' der Ökonomie zu liefern, was in seinem zweibändigen Hauptwerk "`Kapital und Zins"' mündete. Von Bedeutung bis in die Gegenwart ist heute vor allem seine frühe Theorie des Zinses (Agiotheorie). Er schreibt: "`Gegenwärtige Güter sind in der Regel mehr wert, als künftige Güter gleicher Art und Zahl"' \parencite[S. 248]{BohmBawerk1888}. Dieser Satz ist insofern bemerkenswert, da er heute noch im ersten Kapitel eines modernen Finanzierungsbuchs sinngemäß identisch abgedruckt ist. Die erste Lektion lautet dort eben, dass Zahlungen, die zu unterschiedlichen Zeitpunkten erfolgen nur dann miteinander verglichen werden können, wenn sie vorher auf ihren Barwert \textit{abgezinst} werden. Interessant ist auch, dass diese Grundaussage von  \textcite{BohmBawerk1888} bis heute Gültigkeit hat, seine drei angeführten Gründe dafür heute als falsch angesehen werden \parencite[S. 316]{Rosner2012}. \textcite[S. 258ff]{BohmBawerk1888} diskutierte zwar bereits den Einfluss von Unsicherheit und Risiko (Wahrscheinlichkeit der Rückzahlung), fügte aber hinzu, dass diese Risikoprämie nichts mit dem Zinssatz zu tun hat \parencite[S. 261]{BohmBawerk1888}. Dennoch lieferte er eben als erster die bis heute gültige Theorie, dass Einkommen aus Kapital darauf zurückgeht, dass man gegenwärtigen Konsum durch Sparen in zukünftigen Konsum tauscht. Eine allgemeine Anmerkung, die an dieser Stelle besonders gut passt: Wirtschaftswissenschaftliche Arbeiten um 1900 sehen grundsätzlich anders aus, als heutige Arbeiten. Sucht man heute in verschiedenen Quellen nach Eugen von Böhm-Bawerk, so bekommt man den Eindruck er hätte primär an einem sehr eingeschränkten Bereich der Kapitalmarkttheorie gearbeitet. Aber das stimmt so nicht. Die beiden Bände von "`Kapital und Zins"' umfassen jeweils um die 500 Seiten. So wie alle Ökonomen dieser Zeit, versuchte auch Böhm-Bawerk ein "`allumfassendes"' Werk zu schaffen. Als bahnbrechende Leistung hat es natürlich nur jeweils ein Bruchteil dieser Arbeiten ins Bewusstsein der Gegenwart geschafft. Aus heutiger Sicht auch nur mehr schwer nachvollziehbar, wenn auch eigentlich nicht sehr überraschend: Die meisten Arbeiten zielten darauf ab die Arbeiten von Karl Marx und dessen kommunistischen Theorien zu widerlegen. 

Der US-amerikanische Ökonom Irving Fisher - der ja bereits aus dem Kapitel \ref{Neoklassik} bekannt ist - schuf auf mehreren Ebenen die Grundlage für die moderne Kapitalmarkttheorie. Zunächst nahm er die Zinstheorie von Böhm-Bawerk auf und formalisierte sie zur Theorie der \textsc{Intertemporalen Konsumentscheidung}, die bis heute in der Mikroökonomie "`State-of-the-Art"' ist - jeder Student kennt das entsprechende "`Fisher-Diagramm"'. Demnach maximiert ein rationales Individuum seinen Nutzen aus aktuellem Konsum und zukünftigem Konsum in Abhängigkeit vom aktuellen Einkommen, zukünftigem Einkommen und dem Zinssatz. Je niedriger der Zinssatz, desto höher ist die Attraktivität des aktuellen Konsums \parencite{Fisher1930}. Daraus lässt sich herleiten, dass Zinssenkungen zu höherem Konsum führen und Zinserhöhungen den Konsum drosseln. Aus dieser Theorie der Zeitpräferenz lässt sich auf die Höhe des risikolosen Marktzinssatzes schließen: Angenommen ich schulde ihnen 100 Euro. Wenn sie frei wählen können ob sie diese 100 Euro heute oder in einem Jahr bekommen wollen, dann wird jeder die sofortige Übergabe bevorzugen. Wenn ich aber das Angebot ändere auf: "`Heute 100 Euro oder aber in einem Jahr 100+X Euro"', wird jeder einen Wert x finden bei dem die spätere Zahlung bevorzugt (oder eigentlich als gleichwertig einschätzt). Das x ist nichts anderes als der Zinssatz. Die Person, die sich mit dem niedrigsten x zufriedengibt, wird von mir diesen Kreditauftrag bekommen. In einer seiner ersten wissenschaftlich bedeutenden Arbeiten formalisierte Fisher explizit den Zusammenhang, der als "`Fisher-Gleichung"' bekannt wurde: Realzinssatz entspricht Nominalzinssatz minus Inflation. Außerdem beschrieb er darin den Unterschied zwischen diskreter und stetiger Verzinsung\parencite[S. 191ff]{Fisher1906}. Beides mag zwar heute banal klingen, man darf aber nicht vergessen, dass diese Konzept bis heute unverändert Bedeutung haben. Bedeutend für die Entwicklung einer alleinstehenden Kapitalmarkttheorie ist zudem das "`Fisher-Separationstheorem"', mit diesem legt \textcite[S. 125f]{Fisher1930} dar, dass rationale Indiviuden auf einem vollkommenen Kapitalmarkt ihre Investitions- und Finanzierungsentscheidungen völlig unabhängig voneinander treffen können. Investitionen werden demnach rein nach ihrem Kapitalwert bewertet, der wiederum vom Zinssatz abhängig ist. Ist der Kapitalwert positiv soll das Projekt auf jeden Fall realisiert werden. Erst danach - praktisch auf der nächsten Entscheidungsebene, daher Separation - wird entschieden \textit{wie} die Investition finanziert wird. Auf dieser Grundlage macht es Sinn den Kapitalmarkt als von den Realmärkten völlig unabhängigen Markt zu betrachten. Bisher - sowohl von Böhm-Bawerk, als auch von Fisher - außen vor gelassen wurden aber alle Überlegungen zum Zusammenhang zwischen Rendite (Zins) und Risiko.

Frank Knight wird häufig, gemeinsam mit Jacob Viner, als der Begründer der "`Chicago School"' bezeichnet. Tatsächlich wurde vor allem der makroökonomischen Zweig der Chicago School um Milton Friedman und dessen Monetarismus, sowie später die Neue Klassische Makroökonomie um Robert Lucas weltbekannt. Aber auch die mikroökonomische, neoklassische Finanzierungstheorie wurde zu einem erheblichen Teil in Chicago entwickelt. Aus wissenschaftlicher Sicht spielt Knight vor allem aufgrund seiner Unterscheidung zwischen fundamentaler Unsicherheit und Risiko eine Rolle \parencite{Knight1921}. Bei erstgenannter können keinerlei Informationen über die Eintrittswahrscheinlichkeiten zukünftiger Zahlungsströme gemacht werden. Bei Risiko hingegen kann man zukünftige Zahlungsströme mit Eintrittswahrscheinlichkeiten - meist in Form von Verteilungen - versehen. Damit kann man Erwartungswerte und Standardabweichungen berechnen. Dieses Konzept bildete später die Grundlage für die Entwicklung der Entscheidungstheorie.

Als Vorläufer der Finanzmathematik gilt heute \textcite{Bachelier1900}. Er ist de facto der Begründer der Random-Walk Theorie und verwendete als Erster stochastische Prozesse zur Darstellung von Aktienrenditen. Konkret ging schon Bachelier davon aus, dass stetige Renditen durch eine normalverteilte Zufallsvariable dargestellt werden können. Dies begründete auch die Wichtigkeit der Normalverteilung von Renditen in der Modern Finance. Anleger machen demnach die Entscheidung über ihr Investment ausschließlich von erwarteter Rendite und dem Risiko einer logarithmisch-normalverteilten Zufallsvariable abhängig. Diese Annahme erwies sich als extrem nützlich. Die Normalverteilung ist nämlich durch die ersten beiden Momente vollständig beschrieben. Das erste Moment, der Mittelwert $(\mu)$, kann dabei in der Finance als Erwartungswert der Renditen interpretiert werden. Das zweite Moment, die Standardabweichung $(\sigma)$, wird als Maß für das Risiko herangezogen. Das dritte Moment, die Schiefe der Verteilung, kann bei Normalverteilungen vernachlässigt werden, da diese symmetrisch sind und der Wert daher immer 0 ist. Für die Finance bedeutet dies, dass Gewinne und Verluste immer gleich wahrscheinlich sind, was perfekt zur Annahme effizienter Märkte passt. Das vierte Moment der Normalverteilungen, die Wölbung, nimmt immer den Wert 3 an. Dies ist der Pferdefuß der Normalverteilungsannahmen. Man weiß nämlich mittlerweile aus empirischen Untersuchungen, dass Aktienrenditen eine stärkere Wölbung aufweisen, also "`leptokurtisch"' sind. Die Normalverteilungsannahme ist daher nur näherungsweise erfüllt. Insbesondere der Mathematiker Mandelbrot verfasst dazu schon früh Arbeiten \parencite{Mandelbrot1963}. Er wendete das von ihm geprägte Konzept der Fraktalen Geometrie auch auf den Finanzmarkt an und schlug statt der Normalverteilung eine Verteilung mit nicht-endlicher Varianz, zum Beispiel eine Levy-Verteilung vor. Tatsächlich ist das zentrale Problem der Normalverteilungsannahme, dass in empirischen Untersuchungen fast immer zu "`schwere Ränder"', also zu viele Extremwerte, beobachtet werden. So zeigten \textcite{Dowd2008}, dass Tagesrenditen, die außerhalb von drei Standardabweichungen einer log-Normalverteilung liegen, nur einmal in drei Jahren vorkommen dürften. In der Realität beobachtete man aber gerade zu Beginn von großen Finanzkrisen vereinzelt Abweichungen, die außerhalb von sechst Standardabweichungen lagen, sogenannte "`Six-Sigma-Events"'. Diese dürften laut \textcite{Dowd2008} nur einmal in vier Millionen Jahren vorkommen. Trotz aller Kritik überwiegen für die meisten Ökonomen die Vorteile der Annahme log-normalverteilter Renditen, da sie - wie gleich dargestellt wird - eine Voraussetzung für die Anwendbarkeit bestimmter Modelle ist.

\section{Erwartungsnutzen und Pratt's Risikoaversion}
\label{Erwartungsnutzen}
Die \textsc{Erwartungsnutzentheorie} und das Konzept der Risikoaversion sind zentrale Voraussetzung für die neoklassische Finanzierungstheorie. Die Erwartungsnutzentheorie als Teil der Entscheidungstheorie ist eng verbunden mit dem Konzept des \textsc{Homo \oe conomicus}. Der "`rational handelnde Mensch"' ist eigentlich schon eine implizite Voraussetzung in der klassischen Ökonomie, auf jeden Fall aber in der Neoklassik. Auch in der modernen Makroökonomie gibt es das Konzept, wird dort aber "`repräsentativer Agent"' genannt. Für die Finanzierungstheorie von entscheidender Bedeutung ist die Erwartungsnutzentheorie für den Spezialfall der Entscheidungen unter Risiko. Das heißt, ein Zahlungsstrom in der Zukunft kann nicht mit Sicherheit vorhergesagt werden, sondern es gibt immer ein gewisses Risiko, ob überhaupt und wenn ja in welcher Höhe die Zahlung erfolgt. Ökonomen sprechen bei solchen unsicheren Zahlungsströmen häufig von Lotterie. Wir haben bereits im letzten Unterkapitel die "`Einteilung"' von Risiko nach Frank Knight kennen gelernt. Da man für Zahlungsströme unter "`fundamentaler Unsicherheit"' wenig Aussagen treffen kann, geht man davon aus, dass zumindest eine Risikoverteilung zu zukünftigen Zahlungsströmen angegeben werden kann. Dadurch können wir eine bestimmte Wahrscheinlichkeit angeben, in welcher Höhe die zukünftigen Zahlungsströme erfolgen. Häufig bedient man sich, der Einfachheit halber, diskreter Wahrscheinlichkeiten. Das heißt man zum Beispiel an, dass ein Kreditnehmer seine Schulden mit 80\% Wahrscheinlichkeit vollständig bezahlen kann und mit 20\% Wahrscheinlichkeit gar nicht. Vom Konzept her gleich, aber mathematisch etwas herausfordernder, ist die Annahme stetiger Wahrscheinlichkeitsverteilungen. Ebenfalls im letzten Unterkapitel haben wir die Arbeit von Louis Bachelier beleuchtet, der bereits um die Jahrhundertwende davon ausging, dass Aktienrenditen sich nach einem Zufallsprozess entwickeln. Die Vorteile der Annahme logarithmisch-normalverteilter Renditen haben wir oben schon beschrieben. 

Aus diesem Konzept, dass wir zukünftige Zahlungsströme als Wahrscheinlichkeitsverteilung betrachten können, lässt sich die Erwartungsnutzentheorie ableiten. Der Beginn unserer Überlegungen führt uns dabei zurück ins 18. Jahrhundert. Der Schweizer Mathematiker Daniel Bernoulli veröffentlichte 1738, sinngemäß, folgendes Gedankenexperiment: Es wird das Spiel "`Kopf oder Zahl?"' gespielt. Bei "`Kopf"' erhalten Sie 2 Euro und das Spiel geht weiter, wobei jede Runde ihr Gewinn verdoppelt wird. Nach der zweiten Runde Kopf steht ihr Gewinn also bei 4 Euro, nach der dritten bei 8 Euro und so weiter. Erscheint aber das erste mal "`Zahl"' ist das Spiel vorbei. Die Frage, die sich \textcite{Bernoulli1738} stellt, lautet, wie hoch der faire Preis für so eine Lotterie nun wäre? Als intuitive Entscheidungsregel denkt man grundsätzlich einmal an den Erwartungswert. Dazu würde man den Durchschnitt der Lotterie-Ergebnisse berechnen. Wenn also \textit{einmal} eine Münze geworfen wird und bei "`Kopf"' 2 Euro, bei "`Zahl"' hingegen nichts bezahlt wird, wäre der faire Preis 1 Euro. Wendet man diese Formel allerdings beim oben genannten Gedankenexperiment an, erhält man als Ergebnis unendlich! Sie haben zwar jede Runde eine Gewinnchance von nur 50\% aber dafür wird auch jede Runde der Gewinn verdoppelt. Das Spiel wurde als "`St. Petersburg Paradoxon"' bekannt und gilt vor allem unter angehenden Roulette-Spielern als \textit{die} Gewinnstrategie\footnote{Das ist natürlich nur scherzhaft gemeint. Als (angehender) Ökonom wissen Sie: Würde das Konzept funktionieren, wären Roulette-Tische schon ausgestorben. Tatsächlich ist das Wesentliche an der Strategie übrigens das ständigen verdoppeln der Einsätze, die Wahl der Farbe ist hingegen natürlich völlig unbedeutend. Dadurch erreichen Sie an Roulette-Tischen auch recht rasch das Tischlimit}. Die erste ökonomische Grundaussage von \textcite{Bernoulli1738} lautet damit, dass der Erwartungswert als Entscheidungsmodell unzureichend ist. Das war damals eine neue Erkenntnis. Die berühmten Begründer der Wahrscheinlichkeitstheorie - Pascal und Fermat - waren nämlich noch vom Erwartungswert als ausreichendes Entscheidungsprinzip ausgegangen. \textcite{Bernoulli1738} lieferte aber noch eine weitere Erkenntnis, die ebenfalls mit einem einfachen Münzwurfbeispiel erklärt werden kann. Versetzen Sie sich in das oben beschriebene, \textit{einmalige} "`Kopf-oder-Zahl"' Spiel. Das genannte Spiel um zwei Euro, bei 1 Euro Einsatz würden wir vielleicht ohne groß zu überlegen spielen, einfach weil es Spaß macht, oder um einem Kind  Freude zu bereiten. Würden Sie die Lotterie aber auch dann eingehen, wenn Gewinnchance und Einsatz jeweils verzehnfacht würden? Oder wenn es gar um einen Gewinn von 200.000 Euro ginge, bei einem Einsatz von 100.000 Euro? Zumindest im letzten Fall würde wohl jeder ablehnen. Beachten Sie aber, dass nach dem Prinzip des Erwartungswertes das Spiel um 200.000 Euro ebenso fair bewertet wird, wie das Spiel um 2 Euro. Wahrscheinlich würden die meisten von uns die letzte Wette aber sogar dann ablehnen, wenn der Einsatz von 100.000 Euro auf 90.000 Euro gesenkt würde. \textcite{Bernoulli1738} schloss daraus, dass Menschen offenbar neben dem Erwartungswert und auch das entstehende Risiko bei ihren Entscheidungen berücksichtigen und vor allem mit zunehmenden Einsatz immer vorsichtiger werden. Während bei Risikoneutralität der Erwartungswert als Entscheidungskriterium ausreicht, maximieren Menschen in der Realität ihren \textit{Nutzen}. Diese Erkenntnis ist bahnbrechend, aber es entsteht daraus auch ein Problem. Den Erwartungswert kann man leicht in Geldeinheiten ausdrücken. Für den Nutzen hingegen gibt es kein Maß. Mehr noch: Der Nutzen einer bestimmten Lotterie ist für jeden Menschen unterschiedlich. Zwar ist jeder Risiko grundsätzlich negativ eingestellt - Ökonomen bezeichnen dies als "`Risikoaversion"' - das Maß dieser Risikoscheu ist kaum quantifizierbar und von Individuum zu Individuum unterschiedlich. Es gibt also keine einheitliche Formel wie man Wohlstand in Nutzen umrechnen kann. 

Dieses Problem wurde erst im 20. Jahrhundert gelöst. Die Arbeit von \textcite{Bernoulli1738} war in Latein verfasst und seine Erkenntnis zwischendurch praktisch verlorengegangen. Erst Anfang des 20. Jahrhunderts wurde sie wiederentdeckt und schließlich 1954, in Englisch übersetzt, veröffentlicht. Eine wesentliche Weiterentwicklung stellte aber \textcite{Morgenstern1944} dar. Das Monumentalwerk revolutionierte nicht nur die Erwartungsnutzentheorie - tatsächlich werden Nutzenfunktionen bis heute meist "`Von Neumann-Morgenstern-Nutzenfunktionen"' genannt - es bildete auch die Grundlage der Spieltheorie (vgl. Kapitel \ref{Spieltheorie}). Das Buch wurde bereits 1944 erstmals veröffentlicht, wird aber meist in der Ausgabe von 1953 zitiert. Laut \textcite[S. 235]{Bernstein1996} war der Papiermangel im Zweiten Weltkrieg dafür verantwortlich, dass das Buch zwar 1944 erschien, aber erst 1953 in größerer Anzahl gedruckt wurde. Das Erwartungsnutzenkonzept, wie es bis heute Bestand hat, wurde dabei im kurzen Kapitel 3 (Seiten 15-29) des 640 Seiten-Wälzers entwickelt. Die Grundaussage dabei lautet, dass Nutzen nicht kardinal gemessen werden kann sondern nur ordinal. Das heißt, Nutzen kann weder in einer messbaren Einheit angegeben werden, noch kann ihm ein bestimmter Zahlenwert sinnvoll zugewiesen werden. Aber auch mit dem ordinalem Nutzenkonzept kann man quantitative Aussagen machen, solange bestimmte Voraussetzungen erfüllt sind. Diese wurden in einem Axiomensystem von \textcite[S. 26f]{Morgenstern1944} beschrieben. So müssen Entscheidungsträger alle verfügbaren Alternativen in eine Präferenz-Reihenfolge bringen können (Vollständige Ordnung) und Kombinationen aus verschiedenen Alternativen bilden können, die den gleichen Nutzen liefern wie andere Kombinationen (Stetigkeit). Schließlich müssen die Alternativen Unabhängigkeit aufweisen: Die Kombination schlechter individueller Alternativen darf nicht besser bewertet werden als die Kombination guter individueller Alternativen. Die Erwartungsnutzentheorie ist ebenso mächtig wie umstritten. \textcite{Allais1953} veröffentlichte eine Kritik, die im Journal Econometrica in französischer Sprache, nur mit einer englischen Zusammenfassung, abgedruckt wurde. Man konnte in empirischen Experimenten rasch zeigen, dass die Axiome aus \textcite{Morgenstern1944} nicht immer aufrecht zu erhalten sind. Diese Kritik an der Erwartungsnutzentheorie von \textcite{Allais1953} gilt heute als Grundlage der Behavioral Economics (vgl. Kapitel \ref{Behavioral}). Der Vorteil der Erwartungsnutzentheorie ist, dass durch bestimmte Transformationen aus dem Erwartungswert sehr einfach ein Erwartungsnutzen berechnet werden kann, der allen formalen Anforderungen laut \textcite{Morgenstern1944} erfüllt. Konkret eignen sich dazu Transformationen, die aus der linearen Erwartungswertfunktion (der Erwartungswert steigt im Gleichen Verhältnis wie die Lotterieauszahlungen) eine konkave Erwartungsnutzenfunktion machen (der Erwartungsnutzen steigt weniger stark an als die Lotterieauszahlungen). Diese Funktion mit ständiger kleiner werdender Steigung bildet die Tatsache ab, dass mit höheren Lotteriewerten die Risikoaversion steigt. Konkrete Nutzenfunktionen sind häufig Wurzelfunktionen, oder Logarithmusfunktionen. Zum Abschluss noch eine interessante Querverbindung: Die Risikoaversion führt also zu einer Nutzenfunktion, die zwar ständig ansteigt, aber deren Steigung immer geringer wird. Eine ähnliche Beobachtung haben wir bereits im Kapitel \ref{Neoklassik} gemacht. Der Vorläufer der Neoklassik Hermann Heinrich Gossen postulierte als erster den abnehmenden Grenznutzen: Je mehr Güter ich besitze, desto geringer ist der Nutzenzuwachs durch ein zusätzliches Gut - in Kapitel \ref{Neoklassik} nannten wir dies das Prinzip des abnehmenden Grenznutzens. Und tatsächlich stammt die Risikoaversion aus diesem Prinzip. Risikoaversion und Abnehmender Grenznutzen sind identisch, nur aus anderen Blickwinkeln betrachtet! Wenn ich 100.000 Euro besitze und 90.000 davon drohe ich durch ein einzelnes Ereignis zu verlieren, dann werde ich tunlichst versuchen diesen Verlust zu vermeiden. Wenn ich 100 Mio. Euro besitze, werden mir die 90.000 Euro eher egal sein - Mit höheren Vermögen nimmt die Risikoaversion ab. Wenn ich 100.000 Euro besitze und 90.000 gewinne, werde ich mich sprichwörtlich "`freuen wie ein Schneekönig"'. Wenn ich 100 Mio. Euro besitze und 90.000 Euro gewinne, wird meine Freude nicht ganz so euphorisch sein - Mit höherem Vermögen nimmt der Grenznutzen ab.

Unabhängig voneinander erweiterten drei Autoren die Anwendung von Erwartungsnutzenfunktionen. \textcite{DeFinetti1952}, \textcite{Arrow1963b} und \textcite{Pratt1964} entwickelten Ein Maß mit dem die absolute Risikoaversion (ARA) quantifiziert werden kann. Entscheidend hierbei ist, dass dieses sogenannte "`Arrow-Pratt-Maß"' direkt aus Erwartungsnutzenfunktionen abgeleitet wird und die Varianz die Höhe des Risikos darstellt. Die Kennzahl für die absolute Risikoaversion erlaubt uns den Wert von Zahlungen, die nur mit einer bestimmten Wahrscheinlichkeit eintreffen um die Risikoeinstellung zu "`bereinigen"'. Wie wir wissen hat jeder Mensch eine individuelle Risikoaversion und damit eine individuelle Nutzenfunktion. Wenn wir aber die Nutzenfunktion einer bestimmten Person kennen, können wir daraus ein Maß, also eine Zahl, für die absolute Risikoaversion ableiten. Mit Hilfe dieses Maßes können wir wiederum berechnen, welcher sichere Geldbetrag für diese Person gleich viel wert ist wie eine bestimmte Lotterie. Ökonomen nennen diesen Betrag "`Sicherheitsäquivalent"'. Wie wir oben bereits beschrieben haben, kennen wir von einer Lotterie die möglichen Zahlungen und die dazu gehörigen Wahrscheinlichkeiten, mit denen diese Zahlungen realisiert werden. Daraus lässt sich der Erwartungswert der Zahlungen $(\mu)$, sowie die durchschnittliche Abweichung von diesen Zahlungen berechnen, was wir als Risikomaß heranziehen werden $(\sigma)$. Das Sicherheitsäquivalent $(\phi)$ ergibt sich nun - leicht vereinfacht - aus dem Erwartungswert der Lotterien abzüglich des Risikos der Lotterien, gewichtet mit der individuellen absoluten Risikoaversion\footnote{Die tatsächliche Formel lautet $\phi = \mu - \frac{1}{2}*ARA*\sigma^2$.}. Man nennt dieses Konzept den "`Mean-Variance-Ansatz"', beziehungsweise das $\mu-\sigma$-Prinzip. Dieses hat eine große Bedeutung in der gesamten Neoklassischen Finanzierungstheorie. Wie wir im Kapitel \ref{Portfolio} in Kürze sehen werden, baut auch die Bewertung von Portfolios auf den gleichen Kennzahlen, also Erwartungswert und Standardabweichung (=Wurzel der Varianz) auf. Das Konzept der Erwartungsnutzentheorie ist daher vollständig integrierbar in die Modelle der Modern Finance, insbesondere die Portfoliotheorie und das CAPM. Dies gilt aber nur dann, wenn tatsächlich nur die ersten beiden Momente, also Erwartungswert (Mittelwert) und Standardabweichung relevant, der Verteilung der Lotterie eine Rolle spielen. Dies ist zum Beispiel dann der Fall, wenn die einzelnen Ausprägungen der Lotterie durch eine Normalverteilung abbildbar sind. Genau deshalb hält man in der Finanzierungstheorie an der umstrittenen Normalverteilungsannahme fest. Nur unter dieser Prämisse greifen die Konzepte der Erwartungsnutzentheorie und der Modern Finance so schön und mathematisch elegant ineinander.


\section{Die Relevanz der Irrelevanz}
\label{Struktur}
Die Arbeit von \textcite{Modigliani1958} zur Kapitalstrukturtheorie gilt bis heute als \textit{die} Grundlage der wissenschaftlichen "`Corporate Finance"'. Bis zu diesem Zeitpunkt wurde die Finanzierung von Unternehmen als reiner Teil der Betriebswirtschaft gesehen. Unternehmen wurden ausschließlich individuell betrachtet. Franco Modigliani und Merton Miller revolutionierten diese Ansicht, indem sie sich erstmals wissenschaftlich der Kapitalstruktur von Unternehmen annäherten. Sie gingen von einem perfekten Kapitalmarkt aus, auf dem es keine Steuern und Transaktionskosten gibt und sich Unternehmen zu einem konstanten Fremdkapitalzinssatz finanzieren können. Der Wert eines Unternehmens bestimmt sich ausschließlich aus den zukünftigen, abgezinsten Cashflows. Und ausgehend von diesen Rahmenbedingungen, zeigten \textcite{Modigliani1958} mit einem einfachen Arbitrage-Argument, dass die Zusammensetzung der Finanzierung eines Unternehmens, also der Anteil von Eigenkapital und Fremdkapital, keinerlei Auswirkungen auf den Unternehmenswert hat. Das Konzept wurde als "`Irrelevanztheorie"' bekannt und klingt nicht besonders spannend. Damit sind aber interessante Punkte verbunden. Zunächst war der Ansatz, mittels Arbitrageargument einen wissenschaftlichen Beweis zu führen, richtungsweisend. Der "`Law of one Price"'-Ansatz blieb in der Finanzierungstheorie bis heute das wichtigste Konzept und bildet auch die Grundlage für die Effizienzmarkthypothese (vgl. Kapitel \ref{Efficient}). Inhaltlich revolutionär war aber die Erkenntnis, dass der Wert eines Unternehmens gänzlich von seiner Kapitalstruktur unabhängig ist. Das Arbitrageargument lautet hierfür wie folgt: Man stelle sich zwei identische Unternehmen vor, die sich ausschließlich darin unterscheiden, dass Unternehmen A aus zwei Investoren besteht, die beide das Eigenkapital des Unternehmens stellen, während Unternehmen B aus einem Eigenkapitalgeber und einem Fremdkapitalgeber besteht. Von den Gewinnen muss Unternehmer B zunächst einmal die Fremdkapitalzinsen bezahlen. Der Rest des Gewinns gehört aber dem Eigenkapitalgeber. Bei Unternehmen A fließen die kompletten Gewinne den Eigenkapitalgebern zu, diese müssen sich die Gewinne aber teilen. Die beiden Unternehmer B verdienen also weniger als der eine Eigenkapitalgeber in A. Es scheint so als wäre Unternehmen A dann auch wertvoller, schließlich wirft es mehr Gewinn für seine Eigentümer ab. Das stimmt aber nicht. Man darf nämlich nicht vergessen, dass im Verlustfall der eine Eigentümer A den gesamten Verlust tragen muss, während die beiden B-Investoren sich auch die Verluste aufteilen. Die höheren Ertragschancen in Unternehmen A werden durch das höhere Risiko genau ausgeglichen. Daraus resultiert der berühmte Hebeleffekt, auch im deutschsprachigen Raum meist einfach "`Leverage"' genannt: Durch die Aufnahme von Fremdkapital kann ich die erwartete Rendite erhöhen, allerdings steigt damit auch das Risiko des Investments. Das lässt sich wiederum erweitern auf die Kapitalkosten. Die berühmte WACC-Formel - bis heute Teil jeder einführenden Finanzierungs-, Controlling- und Finanzierungsvorlesung - geht direkt auf \textcite{Modigliani1958} zurück: Die durchschnittlichen, gewichteten Kapitalkosten eines Unternehmens bleiben, unabhängig von der Kapitalstruktur, konstant.

Die Arbeit von \textcite{Modigliani1958} war zwar rasch wissenschaftlich etabliert, allerdings war sie auch der Kritik ausgesetzt, dass die starken Annahmen eines perfekten Kapitalmarktes in der Realität nie anzutreffen ist. Insbesondere die Annahme, dass es keine Steuern gäbe, ist schlicht falsch. Aus diesem Grund wurde mit \textcite{Modigliani1963} eine Erweiterung des Modells veröffentlicht. Fremdkapital ist steuerlich begünstigt, da die Fremdkapitalzinsen als Aufwand den Gewinn und damit die Steuerlast des Unternehmens verringern. Abhängig von der Höhe des Steuersatzes lässt sich somit die Kapitalstruktur optimieren.

\section{Fama: Nichts arbeitet so effizient wie der Markt}
\label{Efficient}

Die Theorie der Effizienten Märkte ist eine Grundlage der anderen Konzepte der modernen Finanzwirtschaft. Markteffizienz bedeutet, dass der Markt sämtliche relevante Informationen zur Verfügung hat und die Kräfte aus Angebot und Nachfrage zu jedem Zeitpunkt dafür sorgen, dass der Preis in seinem wahrem Gleichgewicht liegt. Es gibt ein einfaches Argument dafür, dass diese Theorie grundsätzlich Sinn macht. Stellen Sie sich vor Sie hätten eine zuverlässige Information darüber, dass ein bestimmtes Asset morgen um 10\% an Wert gewinnt. Es wäre nur rational dieses Asset heute zu kaufen, also nachzufragen, um den freien Gewinn - den "`Free Lunch"' - mitzunehmen. Die dadurch generierte Nachfrage würden den Preis aber ansteigen lassen. Das Asset würde also nicht erst morgen um 10\% steigen, sondern durch die erhöhte Nachfrage heute schon. Denkt man dieses Konzept konsequent durch, bleibt nur die Lösung, dass die Märkte ständig die wahren Preise abbilden. Also effizient sind, es gibt keinen "`Free Lunch"'. Dennoch ist keine Theorie innerhalb der Modern Finance so umstritten wie jene der Effizienten Märkte. Dafür gibt es vor allem empirische Gründe. Die großen Kursschwankungen, vor allem aber rasche Einbrüche an den Finanzmärkten, wie jene am "`Schwarzen Donnerstag"' 1929, oder am "`Schwarzen Montag"' im Jahr 1987, lassen sich kaum mit der Idee der Markteffizienz vereinbaren. Zur Kritik später mehr.

Die Effizienzmarkttheorie ist auch relativ schwer eindeutig zuzuordnen. Die Idee wird oft in Verbindung mit der "`Random-Walk-Theorie"' gesehen, die - wie oben beschrieben - ursprünglich auf \textcite{Bachelier1900} zurückgeht. Demnach repräsentieren Aktienkurse immer effiziente Märkte und zukünftige Kursschwankungen erfolgen rein zufällig\footnote{Im Umkehrschluss muss aber eben nicht gelten, dass Kursschwankungen, die einem zufälligen Pfad folgen auch tatsächlich effizient sind.}. Unumstritten ist auf jeden Fall, dass \textcite{Fama1970} die Effizienzmarkthypothese als erster operationalisierte und auch ausführliche empirische Untersuchungen dazu veröffentlichte. Eugene Fama unterscheidet zwischen schwacher, mittelstarker und starker Informationseffizienz. Die schwache Effizienz sagt aus, dass die aktuellen Kurse alle historischen Kursinformationen schon berücksichtigt hat, die mittelstarke Effizienz umfasst zusätzlich alle öffentlich zugänglichen Informationen, wie Bilanzen, Veröffentlichungen oder Presseberichte. Für die Gültigkeit dieser beiden Ausprägungen der Effizienzmarkthypothese fand \textcite{Fama1970} recht starke empirische Argumente. Damit ist übrigens verbunden, dass so weitverbreitete Methoden wie die technische Analyse, oder die Fundamentalanalyse keinerlei Informationsgehalt haben. Bei Vorliegen der starken Effizienz würden selbst Insider-Informationen schon im Preis enthalten sein, dafür fand aber selbst \textcite{Fama1970} keine überzeugenden empirischen Anhaltspunkte.
Kritik an der Theorie fand sich schon bald nach deren Publikation. Einen interessanten theoretischen Kritikpunkt veröffentlichten \textcite{Grossman1980}. Sie argumentierten, dass die Beschaffung von Marktinformationen Kosten verursacht. Im Zusammenhang mit dem Vorliegen der Effizienzmarkthypothese entsteht folgende paradoxe Situation. Wenn die Preise alle Informationen bereits enthalten und die Informationsbeschaffung Kosten verursacht, aber keinen Nutzen bringt, dann wird kein rationaler Marktteilnehmer den Aufwand der Informationsbeschaffung auf sich nehmen. Wenn aber niemand Informationen beschafft, dann können im Umkehrschluss die Preise nicht Informations-effizient sein. Schwankender Informationsbeschaffungsaufwand seitens der Marktteilnehmer könnte damit das Auftreten von Finanzblasen erklären. Über die Finanzwelt hinaus berühmt würde auch Robert Shiller, vor allem durch die Einführung des Case-Shiller-Index - einem Immobilienindex - und das populärwissenschaftliche Buch "`Irrational Exuberance"', das knapp vor dem Platzen der "`Dot-Com-Blase"' im Jahr 2000 veröffentlicht wurde. Seinen wissenschaftlichen Durchbruch schaffte Shiller bereits im Jahr 1981. Er argumentierte empirisch-statistisch begründet, dass Aktienkurse viel zu stark schwanken um mit der Effizienzmarkthypothese vereinbar zu sein \parencite{Shiller1981}. Punktuelle Angriffspunkte auf die Effizienzmarkthypothese lieferten seit den 1970er Jahren die Vertreter der Behavioral Finance, die aber in Kapitel \ref{Behavioral} näher behandelt wird.

Dass die Effizienzmarkthypothese selbst innerhalb der Wirtschaftswissenschaften umstritten ist, zeigt sich unter anderem auch an der doch recht amüsant anmutenden Tatsache, dass im Jahr 2013 mit Eugene Fama und Robert Shiller zwei Ökonomen mit dem Nobelpreis ausgezeichnet wurden, die geradezu gegenteilige Meinungen zur Effizienzmarkthypothese vertreten. Dennoch ist sie nach wie vor Bestandteil der Mainstream-Ökonomie. Man ist sich der Schwächen des Konzepts zwar durchaus bewusst, aber wie jedes Modell scheint es eine ausreichend gute Näherung an die Realität darzustellen. Vor allem aber ist es die Grundlage für die gesamte Neoklassische Finanzierungstheorie, die ein theoretisch gut fundiertes System darstellt, bei dem ein Rädchen wunderbar ins andere greift. Die Alternativen, wie die Behavioral Finance (vgl. Kapitel \ref{Behavioral}) oder die Finanzmarkt-Ansätze der Post-Keynesianer (vgl. Kapitel \ref{Post-Keynes}) konnten bislang zwar punktuell wichtige Kritikpunkte aufwerfen und Teillösungen anbieten, aber eben kein so elegantes Modell wie jenes der Neoklassischen Finance.


\section{Markowitz: Don't put all your eggs in one basket}
\label{Portfolio}

Chronologisch gesehen ist die wahrlich bahnbrechende Arbeit zur Portfoliotheorie von \textcite{Markowitz1952} die erste Arbeit der neoklassischen Finanzierungstheorie. Neben \textcite{Modigliani1958} zählt sie somit zu deren "`Gründungsarbeiten"'. Zur Entstehung des Artikels "`Portfolio Selection"', der im wesentlichen auch die Doktorarbeit von Harry Markowitz darstellt, gibt es unzählige Geschichten. So erzählt Markowitz in einem Interview\footnote{Die Stelle findet sich hier: https://www.youtube.com/watch?v=RVWEhCd819E, Minute 00:50. In diesem Interview findet sich auch die Anekdote mit der Defensio bei Milton Friedman und  Jacob Marschak.}, dass eine Börsenhändler ihm den Tipp gegeben hätten sich einem Finanzthema zu widmen. Eine Anekdote von der abschließenden Defensio seiner Doktorarbeit gab Markowitz im Rahmen seiner Nobelpreis-Lectures zum Besten: "`Professor Milton Friedman argumentierte [Anm.: Wahrscheinlich scherzhaft], dass Portfolio-Theorie kein Teil der Ökonomie sei und sie ihm daher keinen Doktortitel in Ökonomie für eine Dissertation [dafür] geben können."' \parencite[S. 286]{Markowitz1990}. Beide Geschichten machen deutlich wie bahnbrechend seine Arbeit im Jahr 1952 gewesen ist. Etwas das auch \textcite{Rubinstein2002} im Rahmen des 50-Jahr Jubiläums von "`Portfolio Selection"' hervorhob: "`Am beeindruckendsten an Markowitz' 1952 Artikel fand ich, dass er aus dem Nichts zu kommen scheint"'. Tatsächlich lautete, leicht übertrieben, die Prämisse auf den Finanzmärkten vor 1952: Suche die Aktie von der du dir die höchste Rendite erwartest und kaufe sie. Das Konzept der naiven Diversifikation ist schon seit jeher bekannt und auch das Konzept vom Risiko-Rendite-Trade-Off war nicht neu. Aber es gab keine quantitativ-mathematischen Ansätze zur Formalisierung dieser Konzepte. Dies ist einigermaßen überraschend. Wertpapierhandels gibt es schließlich schon seit Jahrhunderten.  Und - anders als in der Makroökonomie - haben selbst die "`Great Depression"', bzw. die Kursverluste in Folge des "`Schwarzen Donnerstags"' im Jahr 1929, keinen Durchbruch in diesem Bereich ausgelöst. Warum war die Arbeit nun so bahnbrechend? Nun, \textcite{Markowitz1952} war die erste rein \textit{mathematisch-quantitative} Arbeit im Bereich der Finanzmarktanalyse. Als solche wurden darin gleich drei wesentliche Konzepte etabliert, deren Bedeutung bis heute unumstritten ist: Erstens, die Varianz der Aktienrenditen wurde als Risikomaß. Zweitens, der Risiko-Rendite-Trade-Off - also die Annahme, dass höhere erwartete Rendite immer auch mit höherem Ausfallsrisiko verbunden ist, und drittens, Die Bedeutung der Korrelation von Aktienrenditen. Letzteres ist nichts anderes als die mathematische Fundierung der Diversifikation. Dieser letzte Punkt wird häufig als der wesentliche bezeichnet und tatsächlich basiert darauf die zentrale Idee des Artikels: Aktienportfolios aus der Kombination von Einzeltiteln zu bilden, die das optimale Verhältnis zwischen Rendite und Risiko abbilden. Was heißt das konkret? Wenn Sie eine Aktie kaufen so erwarten sie in Zukunft eine Rendite von x\%. Diese Erwartung bildet sich aus den vergangenen Renditen dieser Aktie. Da die Rendite bei Aktien aber nie konstant ist, sondern zufälligen vgl. Kapitel \ref{Efficient} - Schwankungen unterliegt, ist diese Rendite stets nur eine Erwartung. Aus den vergangenen Schwankungen lässt sich eine durchschnittliche Schwankung - die Standardabweichung - berechnen. Diese wird in weiterer Folge als Risikomaß herangezogen. Je stärker der Wert der Aktie schwankt, desto schwieriger ist deren zukünftiger Wert zu prognostizieren. Oder mit anderen Worten: Desto höher ist ihr Risiko. Stellen Sie sich nun \textit{zwei} Aktien vor. Für beide können sie einen Rendite-Erwartungswert, sowie eine Standardabweichung berechnen. Wenn Sie beide Aktien zu gleichen Teilen kaufen, so entspricht ihr Rendite-Erwartungswert dieses Portfolios dem Mittelwert der Renditen der beiden Aktien. Für die Berechnung der Standardabweichung stimmt dies aber \textit{nicht}! Die Renditen von Aktien verlaufen niemals genau gleich. Das heißt sie korrelieren niemals zu 100\%. Viele Aktien bewegen sich zwar tendenziell in die gleiche Richtung, aber manche Assets sind unabhängig von anderen, bzw. korrelieren sogar negativ. Das heißt bei der Berechnung des Risikos des Portfolios reicht es nicht aus einfach den Mittelwert der Risiko-Werte der Einzeltitel heranzuziehen. Stattdessen muss auch die Korrelation zwischen den beiden Titeln berücksichtigt werden. Wenn zwei Titel perfekt negativ miteinander korrelieren (was ebensowenig vorkommt wie perfekt positive Korrelation), dann steigt eine Aktie immer dann wenn die andere fällt. Dies ist gut für das Portfolio-Risiko: Da der Gewinn der einen Aktie den Verlust der zweiten Aktie immer zumindest teilweise ausgleicht. Die Standardabweichung des Portfolios ist daher stets geringer als der gewichtete Durchschnitt der Standardabweichung der einzelnen Aktien. \textcite{Markowitz1952} zeigte dies erstmals mathematisch. In der Folge kann man natürlich Portfolios aus vielen Einzeltitel zusammenstellen. Sogenannte "`Effiziente Portfolios"' sind aber nur solche, bei denen für eine gegebene Rendite keine niedrigere Standardabweichung erzielt werden kann. Das heißt, es gibt bei Markowitz nich das \textit{eine} optimale Portfolio, sondern eine Reihe von optimalen Portfolios. Diese liegen allesamt auf der "`Efficient Frontier"'. Rationale Individuen sollten nur solche Portfolios erwerben. Welches genau hängt bei \textcite{Markowitz1952} noch von der individuellen Risikoaversion des Investors ab. Die Arbeit gilt heute, 70 Jahre später, noch immer als Ausgangspunkt für quantitatives Asset-Management. 

Mit einer recht intuitiven Idee wurde die Markowitz-Portfoliotheorie durch \textcite{Tobin1958} erweitert. Und zwar indem er die Berücksichtigung eines risikolosen Assets einführte. Die Markowitz-Portfoliotheorie behandelt ausschließlich risikobehaftete Assets. Wenn man jetzt zum Beispiel eine risikolose Anleihe heranzieht, so hat diese einen bestimmten Erwartungswert und eine Standardabweichung von - definitionsgemäß - Null. Dieser Anleihe wird als Fixpunkt betrachtet. Ausgehend von diesem Fixpunkt wird nun eine Tangente an die "`Efficient Frontier"' gelegt. Per Definition berührt diese Gerade die Efficient Frontier nur in einem einzigen Punkt. Dieser Punkt stellt das tatsächlich einzige optimale Portfolio - genannt "`Marktportfolio"' - dar. Die Verbindungslinie zwischen risikoloser Anleihe und Marktportfolio nennt man die "`Kapitalmarktlinie"' (Capital Market Line). Das Marktportfolio ist in diesem theoretischen Konstrukt das einzig sinnvolle Portfolio. Unabhängig von der Risikoaversion kann nämlich aus der Kombination aus risikoloser Anleihe und Marktportfolio stets eine höhere Rendite-Erwartung (bei fixiertem Risiko) erzielt werden, als auf einem beliebigen Punkt auf der Efficient Frontier. Diese Erkenntnis wurde als die \textsc{Tobin-Separation} bekannt. Der Name Separation steht hierbei dafür, dass bei Finanzinvestitionen zwei voneinander unabhängige Entscheidungen getroffen werden müssen. Erstens, es muss das Marktportfolio ermittelt werden. Dieses ist allerdings für jeden risiko-adversen Investor identisch. Zweitens, abhängig von der individuellen Risikoaversion muss ein Investor entscheiden welchen Anteil seines Vermögens er in das risikobehaftete Marktportfolio steckt und welchen Anteil in die risikolose Anleihe. Auch dieses Verhältnis kann man übrigens - für jedermann individuell - quantitativ berechnen. Für jedes risiko-adverse Individuum lässt sich eine Nutzenfunktion ermitteln. Die individuelle Risikoaversion (vgl. Kapitel \ref{Erwartungsnutzen}) kann als "`Mean-Variance"'-Maß\footnote{"'Mean"' bezeichnet hierbei der Erwartungswert der Renditen und "`Variance"' die Varianz, also das Quadrat der Standardabweichung und damit das Risiko.} ausgedrückt werden. Die Nutzenfunktion lässt sich somit mittels Indifferenzkurven in das Portfolio-Diagramm überführen. Der Tangentialpunkt von Indifferenzkurve und Kapitalmarktlinie bestimmt das optimale, individuelle Verhältnis zwischen risikoloser Anleihe und Marktportfolio.

In den 1960er Jahren wurde Tobin's Modell schließlich zum \textsc{Capital Asset Pricing Model (CAPM)} (sprich: CÄP-M) weiterentwickelt. Gleich vier Autoren haben dessen Entwicklung parallel vorangetrieben: \textcite{Sharpe1964}, \textcite{Lintner1965}, \textcite{Mossin1966} und später wurde auch \textcite{Treynor1961} die Idee zugeschrieben. Der Ausgangspunkt ist, dass mittels Diversifikation - wie bei Markowitz dargestellt - Unternehmens-spezifisches Risiko eliminiert werden kann. Einfach deswegen, weil man in viele verschiedene Unternehmen investiert und der Anteil eines bestimmten Unternehmen mit steigender Anzahl an Assets gegen Null geht. Das Unternehmens-spezifische Risiko wird hierbei unsystematisches Risiko genannt. Nicht weg-diversifizieren kann man das systematische Risiko. Dieses wird auch Marktrisiko genannt, bzw. im CAPM als ($\beta$) bezeichnet. Da man im CAPM davon ausgeht, dass man das unsystematische Risiko durch Diversifikation vollständig eliminieren kann, wird man für die eventuelle Übernahme von unsystematischem Risiko (durch fehlende Diversifikation) nicht belohnt. Im CAPM wird daher nur das Marktrisiko betrachtet. Die entsprechenden Darstellungen zeigen daher stets den Trade-Off zwischen erwarteter Rendite und Beta (statt Standardabweichung bei Markowitz und Tobin). Das Marktrisiko wird dabei auf das "`Risiko des Gesamtmarktes"' bei 1 standardisiert. Natürlich ist umstritten was in der Praxis "`der Gesamtmarkt"' ist. Durchgesetzt haben sich hierbei aber breit gefasste Indizes wie der MSCI World. Es werden aber auch Indizes eines Einzelstaates durchaus als Gesamtmarkt herangezogen. Die Renditen der einzelnen Unternehmen (Aktien) werden ins Verhältnis zu diesem Gesamtmarkt gesetzt. Das heißt man schaut sich für einen gewissen Zeitraum, zum Beispiel fünf Jahre, wie sich die monatlichen Marktrenditen zu den monatlichen Einzelaktien-Renditen verhalten haben. Dies wird mathematisch analog zu einer univariaten, linearen Regression gemacht: Die gemeinsame Varianz (Covarianz) von Marktrendite und Einzelaktie-Rendite wird ins Verhältnis zur Marktrendite gesetzt. Das Ergebnis ist eben der $\beta$-Wert. Oder mathematisch: Die Steigung der Regressionsgerade, die bei Regressionen eben auch $\beta$ genannt wird. Ist dies Wert kleiner als eins, so unterliegt die Einzelaktie geringeren Schwankungen als der Gesamtmarkt und umgekehrt. Dieses Maß für das systematische Risiko eines Unternehmens hat bis heute eine enorme Bedeutung in der Unternehmensbewertung. Trotz aller Kritik und weiterentwickelten Methoden, wird in den überwiegenden Fällen der Eigenkapitel-Wert eines Unternehmens noch immer mittels $\beta$-Werten geschätzt. \textcite{Hamada1972} erweiterte das Modell schließlich noch um eine Bereinigung von Effekten der Unternehmens-Kapitalstruktur (vgl. Kapitel \ref{Struktur}). Bis heute sind das CAPM und vor allem die daraus abgeleiteten Risikomaße $\beta$, oder das \textsc{Sharpe-Ratio} - die Überrendite einer Aktie im Verhältnis zu ihrer Standardabweichung - zentrale Kennzahlen in der Finanzwelt. Das ist einigermaßen überraschend, da das CAPM seine Prognosen nur auf Grundlage einer einzigen Kennzahl, der vergangenen Rendite, trifft. Außerdem bleibt die Frage was das \textit{eine} Marktportfolio sei. Zudem kann die Heranziehung unterschiedlich langer Betrachtungszeiträume, zu recht unterschiedlichen $\beta$-Werten führen. Als Alternative veröffentlichte \textcite{Ross1976} sein Arbitragepreismodell. Ebenfalls Eingang in die Standardliteratur fand die CAPM-Erweiterung von \textcite{Fama1993}. Neben der vergangenen Rendite, werden hierbei die Unternehmensgröße, sowie das Kurs-Buchwert-Verhältnis als zusätzliche Einflussfaktoren herangezogen. Eine praktisch überaus bedeutende Erweiterung der Portfolio-Theorie präsentierten \textcite{Black1992}. Darin drehten sie das Konzept der Portfolio-Theorie um, indem sie nicht zukünftige Renditen schätzen, sondern lassen stattdessen die Portfolio-Zusammensetzung für gegebene Renditen vom Modell berechnen. Dies umgeht das praktische Problem, dass Schätzmethoden für Renditen immer mit großen Unsicherheiten behaftet sind.

Fassen wir noch einmal die Erkenntnisse der letzten Unterkapitel in einem Beispiel zusammen. Gesetzt es gibt so etwas wie effiziente Märkte. Dann kann man davon ausgehen, dass sich Kursänderungen, zum Beispiel von Aktienkursen, nur aus nicht vorhersehbaren, zufälliger Ereignissen ergeben. Die resultierenden Aktienrenditen verhalten sich also wie eine Zufallsvariable, die - wie empirische Beobachtungen zumindest näherungsweise bestätigen - durch eine logarithmische Normalverteilung abgebildet werden kann. Rendite und Risiko einer Aktie lassen sich also durch die Parameter Erwartungswert $(\mu)$ und Standardabweichung $(\sigma)$ einer log-Normalverteilung abbilden. Über das Konzept der Markowitz-Portfoliotheorie und deren Erweiterung der Tobin-Separation, werden Diversifikationseffekte, die durch die Korrelation von Aktien untereinander entstehen, berücksichtigt. Es gilt aber auch für gesamte Portfolios, dass diese mittels Erwartungswert und Standardabweichung vollkommen bewertet werden können.

Durch eine Von Neumann-Morgenstern-Nutzenfunktion (Erwartungsnutzenfunktion) lassen sich Erwartungswerte in Nutzen-Erwartungswerte transformieren. Zwar hat jeder Mensch eine individuelle Erwartungsfunktion, allerdings kann diese durch empirische Tests auch tatsächlich ermittelt werden. Das Arrow-Pratt-Maß liefert aus dieser Erwartungsnutzenfunktion ein Maß für die absolute Risikoaversion. Durch dieses Maß lässt sich die Erwartungsnutzenfunktion anhand der schon bekannten Parameter Erwartungswert $(\mu)$ und Standardabweichung $(\sigma)$ ausdrücken und in Form von Indifferenzkurven abbilden. Wir können als eine Koordinatensystem bilden in dem auf der y-Achse der Erwartungswert und auf der x-Achse die Standardabweichung abgebildet ist. Darin können wir sowohl die Efficient-Frontier, bzw. die Kapitalmarktlinie aus der Portfoliotheorie bzw. der Tobin-Separation einzeichnen, also auch die aus der Erwartungsnutzenfunktion abgeleiteten Indifferenzkurven für ein bestimmtes Individuum. Der Tangentialpunkt aus Kapitalmarktlinie und Indifferenzkurve bildet \textit{das} optimale Portfolio für einen bestimmten Menschen ab.

\section{Die Bepreisung von Optionen}
\label{Optionen}

Der eben genannte Fischer Black lieferte sein Meisterstück bereits in den frühen 1970er Jahren mit der nach ihm und seinem Forschungskollegen Myron Scholes benannten Optionspreisformel. Aber langsam. Zu Beginn der 1970er Jahre veränderten sich die Anforderungen an die internationalen Finanzmärkte grundlegend. Das Bretton-Woods-System, das geschaffen worden war um stabile Wechselkurse und somit finanzielle Planungssicherheit im internationalen Handel zu sichern, brach 1973 endgültig zusammen. Die Wechselkurse waren den Marktkräften ausgesetzt. Fixe Wechselkurse konnte man sich in der Folge nur noch durch Finanztermingeschäfte sichern. Im selben Jahr wurde in den USA das Chicago Board of Options Exchange (CBOE) gegründet. Bis heute die zentrale Terminbörse der USA. Der Markt für Termingeschäfte war also im Wachsen. Was aber fehlte war eine theoretisch fundierte Theorie zur Bepreisung von Optionsgeschäften. Und diese lieferten im gleichen Jahr \textcite{Black1973}. Die resultierende Optionspreisformel ist seither als "`Black-Scholes-Modell"', nur selten "`Black-Scholes-Merton-Modell"' bekannt. Die Arbeit von \textcite{Merton1973} erweiterte das Modell indem er die Existenz von Dividenden und schwankenden Zinssätze berücksichtigte \parencite{Scholes1997}.  Die faire Bepreisung von Optionen ist alles andere als ein banales Problem. Ausschlaggebend dafür ist die asymmetrische Pay-Off-Struktur von Optionen. Optionen gleichen einer Wette: Für einen bestimmten Einsatz - den Optionspreis - hat der Käufer einer Option das Recht - aber nicht die Verpflichtung - zu einem bestimmten Zeitpunkt in der Zukunft ein dahinter liegendes Asset (Underlying) zu kaufen. Wenn der Kurs dieses Assets zwischen Kauf der Option und dem festgelegten Ausübungszeitpunkt steigt, so macht der Käufer der Option einen Gewinn. Steigt der Kurs des Underlyings nicht, so lässt er seine Option einfach verfallen. Sein Verlust beschränkt sich dann auf den Optionspreis, den er bereits bezahlt hat.
Das heißt für die Bepreisung, dass man eine Wahrscheinlichkeitsverteilung finden muss, aus der man ablesen kann, welchen Wert das Underlying im Ausübungszeitpunkt der Option mit welcher Wahrscheinlichkeit hat. Zur Berechnung dieser bedienten sich \textcite{Black1973} einer Formel aus der Physik. Nämlich der Brownschen-Bewegung, mit der man zum Beispiel die Ausbreitung von Wärme berechnen kann. Man geht auch bei diesem Konzept davon aus, dass zukünftige Renditen einem Zufallsprozess folgen, der aber eine logarithmische Normalverteilung abbildet, dies bildet eben die Brownsche Bewegung ab. Die Black-Scholes-Formel - eine stochastische Differenzialgleichung - berechnet unter diesen Annahmen aus dem derzeitigen Kurs des Underlyings, dem Ausübungspreis, sowie der Standardabweichung des Underlyings\footnote{Außerdem benötigt man einen konstanten, risikolosen Zinssatz und die Prämissen, dass keine vorzeitige Ausübung möglich ist und keine Dividenden bezahlt werden, müssen nach \textcite{Black1973} ebenfalls gelten.} den fairen Optionspreis.
Ende der 1970er Jahre entwickelten \textcite{Rubinstein1979} übrigens das sogenannte Binominal-Modell zur Optionspreisbewertung.Es handelt sich hierbei aber um ein diskretes Modell. Das heißt man geht davon aus, dass das Underlying in jeder Periode mit einer gewissen Wahrscheinlichkeit steigt bzw. mit der Gegenwahrscheinlichkeit fällt. Erhöht man die Anzahl der Perioden, entsteht ein Baum an dessen Enden man jeweils den Wert der Option für eine bestimmte Wahrscheinlichkeit ablesen kann. Erhöht man die Anzahl der Perioden gegen unendlich liefert das Binominalmodell genau das gleiche Ergebnis wie die Black-Scholes-Formel. In den folgenden Jahrzehnten wurden Derivative Wertpapiere wie Optionen vielfach weiter entwickelt und deren Verbreitung steigerte sich enorm. Vor allem in der Nachbetrachtung der "`Great Depression"' gerieten Derivate in der Öffentlichkeit in Verruf. Ein Verbot bestimmter Derivate wurde gefordert und zweitweise auch in gewissen Bereichen eingeführt. Tatsächlich sind reine Finanz-Wetten auf Soft-Commodities wie zum Beispiel Weizen oder Schweinehälften verwerflich. Man muss aber so realistisch sein, dass man eben kaum unterscheiden kann zwischen dem sinnvollen Einsatz von Derivaten, zum Beispiel im Rahmen von Absicherungsgeschäften, und reinen Finanzspekulations-Geschäften. Faktum ist, dass viele moderne Derivate - zum Beispiel Knock-Out Zertifikate - eine komplizierte Auszahlungsstruktur haben und häufig nur mehr mittels numerischer Rechenmethoden bewertet werden können. Auch die Bewertungsmethoden haben sich in den letzten Jahrzehnten also wesentlich weiterentwickelt.

In seiner Nobelpreisrede \parencite{Scholes1997} und in \textcite[S. 311]{Bernstein1996} wird die Geschichte der drei Schöpfer der Optionspreistheorie, Fischer Black, Myron Scholes und Robert Merton, lebhaft erzählt. Besonders interessant ist die Tatsache, dass das Paper \textcite{Black1973} von drei renommierten Journals abgelehnt wurde, bis es schließlich - nach Interventionen - doch noch im \textit{Journal of Political Economy} veröffentlicht wurde \parencite[S. 136]{Scholes1997}. Ähnlich wie die oben angeführte Geschichte vom Zweifel Friedman's an Markowitz' Portfoliotheorie, zeigt auch diese Story, dass die damalige Einführung mathematisch-quantitativer Methoden in die Finanzmarkttheorie damals als völlig unkonventionell angesehen wurde. Im Nachhinein gesehen aber auf jeden Fall aber bahnbrechend. Die Finanzmarkttheorie war in den 1970er Jahren in doppelter Hinsicht eine "`junge"' Disziplin: Markowitz war bei der Veröffentlichung seiner "`Portfolio Selection"' gerade 25 Jahre alt, William Sharpe bei der Entwicklung des CAPMs knapp 30 und die drei Optionspreistheoretiker ebenso um die 30 bei der Veröffentlichung ihrer Formel. Eine oft zitierte Side-Story zu Myron Scholes und Robert Merton ist deren Engagement beim Hedge-Fund "`Long-Term Capital Management"' (LTCM). Die beiden waren Direktoren dieses Fonds. Das Geschäftsmodell bestand - grob gesagt - darin Assets mit ähnlichem Risiko, aber unterschiedlicher Bewertung zu identifizieren. Es wurden in weiterer Folge gehebelte Wetten darauf platziert, dass sich dieser "`Spread"' zwischen den Bewertungen schließen sollte - an sich eine erwartbare Entwicklung. Mit der russischen Finanzkrise 1998 geriet dieses Konzept allerdings vollkommen aus den Fugen. Der Fonds schlitterte in die Pleite und musste schließlich sogar staatlich gerettet werden, da Auswirkungen auf die gesamte Finanzwelt befürchtet wurden.

Die vier genannten Bereiche - die Kapitalstrukturtheorie, die Effizienzmarkthypothese, die Porfoliotheorie und die Optionspreistheorie - bilden bis heute das Rückgrat der "`Modern Finance"'. Vor allem in der Hochschulausbildung bilden diese vier Konzepte die theoretische Basis jedes "`(Corporate) Finance"'-Kurses. Die Entwickler dieser Disziplin wurden in den folgenden Jahrzehnten übrigens fast allesamt mit dem Nobelpreis geehrt\footnote{Wenn auch teilweise primär für andere Beiträge zur Ökonomie, wie zum Beispiel im Fall von Franco Modigliani und James Tobin.}: James Tobin 1981, Franco Modigliani 1985, Harry Markowitz, Merton Miller und William Sharpe 1990, Robert Merton und Myron Scholes 1997\footnote{Fischer Black war bereits zuvor verstorben} und schließlich Eugene Fama 2013. Dies ist doch einigermaßen überraschend, da die Finanzmarkttheorie ja nur ein kleiner Teilbereich der Ökonomie ist.





		% Finance                               !!! KORREKTURLESEN
%%%%%%%%%%%%%%%%%%%%% chapter.tex %%%%%%%%%%%%%%%%%%%%%%%%%%%%%%%%%
%
% sample chapter
%
% Use this file as a template for your own input.
%
%%%%%%%%%%%%%%%%%%%%%%%% Springer-Verlag %%%%%%%%%%%%%%%%%%%%%%%%%%

\chapter{Spieltheorie} \label{cha: Spieltheorie}
\label{Spieltheorie}

Sie wird häufig als Spezialfall und Weiterentwicklung der Entscheidungstheorie, oder als "`Interaktive Entscheidungstheorie"' bezeichnet und in der Ökonomie noch immer eher als Randthema behandelt, dabei ist sie wohl eine \textit{der} wesentlichen Weiterentwicklungen der Wirtschaftswissenschaften im 20. Jahrhundert: Die Spieltheorie. Ihre Bedeutung kann kaum überschätzt werden. Sowohl in der Mikroökonomie, als auch in der Makroökonomie ist die Spieltheorie Teil unzähliger Modelle. Auch diesbezüglich wandelte sich die Ökonomie: Ausgehend von den Neoklassikern, aber eben auch die Keynesianer und die Monetaristen, suchten nach "`Pareto-optimalen"' Lösungen. Deren Modelle gehen von vollkommenen Konkurrenzmärkten aus, alle Teilnehmer sind Preisnehmer. Sie optimieren also ihr individuelles Verhalten im Anbetracht eines Marktgleichgewichts. Vertreter der Neuen Neoklassischen Synthese hingegen sprechen stattdessen meist von "`Nash-Gleichgewichten"' - also einem spieltheoretischen Gleichgewichtszustand.  Diese neueren Modelle (vgl. Kapitel \ref{cha: Neu Keynes} und \ref{Neue Neoklassische Synthese}, aber auch der Bereich Politische Ökonomie) berücksichtigen, dass die Annahme vollkommener Konkurrenzmärkte häufig unrealistisch ist. Die entsprechenden Optimierungsaufgaben hängen also wechselseitig vom Verhalten der Marktgegenseite ab. 

Wie bereits beschrieben, fristet die Spieltheorie dennoch in gewisser Art und Weise eine Außenseiterrolle innerhalb der Ökonomie. Über die Gründe kann man hier nur spekulieren. Wahrscheinlich spielt es aber eine Rolle, dass die Spieltheorie keine volkswirtschaftliche Theorie im eigentlichen Sinne ist. Ganz im Gegenteil, ihre Aussagen sind auch in der Politik, Biologie, Betriebswirtschaft, Spiel und Sport von Interesse. Dazu passt auch, dass die Spieltheorie nicht von Ökonomen, sondern von Mathematikern entwickelt wurde. Tatsächlich findet man Vorlesungen zur Spieltheorie aber vor allem in wirtschaftswissenschaftlichen Curricula wieder. Zuletzt ist es wohl kaum zu bestreiten, dass die Spieltheorie - stärker als jede andere Disziplin - von bemerkenswerten und oft kontroversen Persönlichkeiten geprägt wurde, von denen uns in diesem Kapitel mehrere unterkommen werden.

Was ist Spieltheorie grundsätzlich? \textcite[S. 136]{Harsanyi1994} bringt es in einem Satz auf den Punkt: "`Spieltheorie ist eine Theorie der strategischen Interaktion. Das heißt, sie ist eine Theorie des rationalen Verhaltens in Situationen, in denen jeder Spieler die wahrscheinlichen Gegenzüge seines Gegenspielers bedenkt und darauf basierend seine eigenen Züge setzt."' Das erinnert zunächst an wirkliche Spiele wie Mühle, Schach oder Poker. Tatsächlich wird zum Beispiel beim Poker mittels spieltheoretischer Ansätze die Gewinnwahrscheinlichkeit einer "`Hand"' berechnet. Der Name \textit{Spiel}theorie ist dennoch etwas irreführend, weil sie in der Realität auf verschiedene Situationen angewendet werden kann, wie individuelle soziale Interaktionen, politische Konflikte, oder eben in verschiedenen wirtschaftlichen Situationen. Die frühe Spieltheorie bei Von Neumann und Morgenstern behandelte sogenannte Nullsummen-Situationen. Das sind Situationen in denen die Gewinne des einen Spielers betragsmäßig den Verlusten seines Gegenspielers stets entsprechen. Eine Situation, die eben tatsächlich vor allem bei wirklichen Spielen auftritt: Wenn Weiß beim Schach einen Läufer verliert kann man auch vom Gewinn eines Läufers durch Schwarz sprechen. Eine wesentliche Erweiterung erfuhr die Spieltheorie Anfang der 1950er Jahre durch John Forbes Nash, dessen Arbeit die Spieltheorie auch auf Nicht-Nullsummen-Situationen ausweitete \parencite[S. 163]{Nash1994}. Wirtschaftliche Kooperation zum Beispiel kann zum Beispiel dafür sorgen, dass beide "`Spielteilnehmer"' ihre Position zu verbessern. Berühmt geworden sind aber vor allem jene Beispiele, bei denen individuelle Nutzenmaximierung zu gesamtwirtschaftlich schlechten Ergebnissen führen - dazu aber später mehr, Stichwort: Gefangenendilemma.

Vorab machen wir das abstrakte Feld der Spieltheorie ein bisschen greifbarer. Die nachstehende Grafik zeigt eine typische Darstellung eines spieltheoretischen Problems. Konkret handelt es sich um ein \textit{nicht kooperatives} (das sieht man nicht aus der Darstellung), \textit{zwei Personen, Nicht-Nullsummen}-Spiel. \textit{Spieler 1} kann aus seinen beiden Strategien wählen \textit{Leugnen, $S_{1,1}$} oder \textit{Gestehen, $S_{1,2}$}. Das Gleiche gilt hier für \textit{Spieler 2}. Es entsteht eine $2x2$-Matrix mit jeweils einem Auszahlungspaar, wobei die erste Zahl jeweils als der Nutzen für Spieler 1 gelesen werden kann. In diesem Fall ist der Nutzen negativ angegeben, was nur ausdrückt, dass man seinen Nutzen maximiert, indem man den geringsten negativen Betrag anstrebt. Vorweggenommen: Rein intuitiv ist klar, welche Lösung gesamtwirtschaftlich (Spieler 1 und Spieler 2 stellen die Gesamtwirtschaft dar) angestrebt wird: Beide sollten die Strategie \textit{Leugnen} wählen. Gesamtwirtschaftlich tritt dann der größte Nutzen (=kleinster Schaden) ein und die Situation keines Spielers könnte verbessert werden \textit{ohne} die Situation des anderen zu verschlechtern. Definitionsgemäß ist ein Pareto-Optimum erreicht. Aber zu welcher Lösung kommt man mit spieltheoretischen Ansätzen?


\begin{tikzpicture}[element/.style={minimum width=2.85cm, minimum height=1.50cm}]
\matrix (m) [matrix of nodes,nodes={element},column sep=-\pgflinewidth, row sep=-\pgflinewidth,]{
	& Leugnen $S_{2,1}$  & Gestehen $S_{2,2}$  \\
	Leugnen $S_{1,1}$ & |[draw]|-2 / -2 & |[draw]|-10 / -1 \\
	Gestehen $S_{1,2}$ & |[draw]|-1 / -10 & |[draw]|-8 / -8 \\    };



\node[above=0.15cm] at ($(m-1-2)!0.5!(m-1-3)$){\textbf{Spieler 2}};
\node[rotate=90] at ($(m-2-1)!0.5!(m-3-1)+(-1.25,0)$){\textbf{Spieler 1}};
\end{tikzpicture}

Diese Form der Darstellung wird übrigens "`Normalform"' genannt. Die zweite übliche Darstellungsform nennt man "`extensive Form"'. Diese umfasst für den gesamten Spielverlauf alle notwendigen Informationen zu Entscheidungen und Auszahlungen und gleicht von der Darstellung her einem Entscheidungsbaum. Beide Formen wurden übrigens schon in der Geburtsstunde der Spieltheorie so verwendet \parencite{Selten2001}.


\section{Von Neumann Morgenstern}

Vereinzelte Ansätze, die Ideen der Spieltheorien vorwegnahmen gab es bereits im 19. Jahrhundert. Das bekannteste Beispiel ist wohl die Duopol-Theorie von \textcite{Cournot1836}. Weithin gilt aber die Veröffentlichung des fundamentalen Werks "`Theory of Games and Economic Behavior"' im Jahr 1944 durch Oskar Morgenstern und John von Neumann als Ursprung der Spieltheorie. \textcite{VonNeumann1928} behandelte bereits einen speziellen Ansatz der Theorie, im Buch von 1944 wurde dieser verbreitert und verallgemeinert. In der zweiten Auflage im Jahr 1947 wurde der Beweis für die Axiome der Erwartungsnutzentheorie erbracht. Diese haben wir bereits im Kapitel \ref{Erwartungsnutzen} kennen gelernt und sind eigentlich mehr Voraussetzung für die Spieltheorie als Teil derselben \parencite[S. 3]{Selten2001}. 

Als "`Vater der Spieltheorie"' gilt also John von Neumann. Ein Mathematik-Genie. Tätig in unzähligen Gebieten, neben Mathematiker und Ökonom gilt er als einer der Entwickler des modernen Computers und er entwickelte eine binäre Programmiersprache. Er arbeitete an der Entwicklung der Quantenmechanik und der Wasserstoffbombe mit. \textcite[S. 232]{Bernstein1996} zitiert, dass er "`während seiner Zeit beim Militär Admiräle gegenüber Generälen bevorzugte, da diese trinkfester waren"' und weitere Geschichten, die ihn als lebenslustiges Genie darstellen. Der Beitrag seines Koautors Oskar Morgenstern - der in Österreich das Institut für Höhere Studien mitbegründete - wird in der modernen Literatur häufig in Frage gestellt. In \textcite[S. 494]{Leonard1994} wird dargestellt, dass Morgenstern, erstens unter Ökonomen-Kollegen in den USA recht unbeliebt war und zweitens, einen\textit{inhaltlichen}-bescheidenen Beitrag zur hoch-mathematischen Spieltheorie von Neumann's beigetragen hat. Insbesondere im Artikel \textcite{Morgenstern1976} stellt der Autor seinen eigenen Beitrag zur Entstehung der "`Theory of Games and Economic Behavior"' gänzlich anders dar. Laut \textcite{Nash1994} ist es aber unzweifelhaft, dass dieses fundamentale Werk ohne die Zusammenarbeit von Oskar Morgenstern und John von Neumann in dieser Form nicht entstanden wäre.

HIER WEITER (aus Selten-Text) Vor allem Nullsummenspiele
Der Beitrag war speziell: Ein (nicht-kooperatives) zwei-Personen Nullsummenspiel und die Minimax-Lösung dafür.
Was war nun aber der Inhalt dieser frühen Spieltheorie?




\section{Nash}
Berührend ist die Geschichte von John Forbes Nash, die nicht zuletzt durch den Film "`A Beautiful Mind"' weit über wissenschaftliche Kreise hinaus bekannt wurde. Nash verfasste 1950 eine geniale und nur großzügige\parencite[S. 164]{Nash1994} 27 Seiten lange Dissertation \parencite{Nash1950}, die später als Journalbeitrag publiziert wurde \parencite{Nash1951}, und mit der er die Spieltheorie entscheidend weiterentwickelte. Ende der 1950er Jahre erkrankte er aber schwer an Schizophrenie und war die folgenden 25 Jahre stark eingeschränkt. Erst in seinen Fünfzigern erholte er sich. 1994 wurde ihm gemeinsam mit Reinhard Selten und John Harsanyi der Nobelpreis für Ökonomie zugesprochen. In einem interessanten Interview \parencite{Nash2004}\footnote{https://www.nobelprize.org/prizes/economic-sciences/1994/nash/interview/} im Rahmen des Ersten Nobelpreisträgertreffens im Jahr 2004 erzählt er, dass die Verleihung des Nobelpreise einen enormen Einfluss auf sein Leben hatte, war er doch zuvor schon lange Zeit arbeitslos, obwohl er schon längere Zeit bei guter Gesundheit war und er daher mehr oder weniger kaum noch am öffentlichen Leben teilnahm. Die Tragik in seinem Leben setzte sich übrigens bis zu seinem Tod fort: Im Jahr 2015 erhielt er den Abel-Preis, eine Auszeichnung für Mathematiker. Bei der Rückkehr aus Norwegen, wo der Preis verliehen wird, war das Taxi, das ihn vom Flughafen nach Hause bringen sollte in einen Autounfall verwickelt, der Nash und seiner Frau das Leben kostete. 

Nash erweiterte die Erweiterung in verschiedener Hinsicht. Am grundlegendsten ist wohl seine \textit{Unterscheidung} zwischen "`Kooperativer Spieltheorie"' und "`Nicht-kooperative Spieltheorie"'. In Letztgenannter geben Spieler kein Commitment zu einer bestimmten Strategie ab, während bei der "`Kooperativen Spieltheorie"' durchsetzbare Verträge vorhanden sein können.
Den größten nachhaltigen Beitrag stellt sicherlich seine Gleichgewichtslösung für nicht-kooperative Spiele dar \parencite{Nash1951, Nash1950b}. Studierende der Ökonomie kennen vor allem deshalb seinen Namen, selbst dann, wenn ihr Studium keine Vorlesung zur Spieltheorie enthält. Schließlich haben diese "`Nash-Gleichgewichte"' in der modernen Ökonomie einen Fixplatz in vielen Modellen eingenommen. Für "`Nicht-kooperative Spiele"' bewies Nash, dass es in solchen Situation immer zumindest ein Nash-Gleichgewicht gibt. Also ein Gleichgewicht, in dem kein Spieler seine Auszahlungen erhöhen kann, indem er einseitig seine Strategie verändert. HIER WEITER: Was ist genau ein Nash-Gleichgewicht!




Von Neumann war übrigens wenig beeindruckt von der Erweiterung "`seiner"' Spieltheorie durch Nash. Vielmehr sah er darin eine "`triviale Folge"' aus dem Fixpunkttheorem, das Brouwer einige Jahre zuvor bewiesen hatte. (Zitieren aus \textcite{Cassidy2015} )

Kollegen Kuhn und Tucker, die vor allem durch die nicht-lineare Programmierung "`Kuhn-Tucker-Bedingung"'.


GEfangenendilemma - Ursprung Tucker: \parencite[S. 161]{Nash1994} - auch das Beispiel von Alan Blinder (1982) mit den Zentralbanken vs. Staatsausgaben











\section{Hurwicz, Aumann, Harsanyi, Selten}

Harsanyi: Nach 1960: Einführung der Spiele mit inkompletter Information (davon abzugrenzen: Perfekte vs. Imperfekte Information). Harsanyi 1994. 

\section{Marktdesign und Auktionstheorie}

Marktdesign: Alvin Roth und Lloyd Shapley 
Auktionstheorie: \textit{Paul Milgrom} und \textit{Robert Wilson} (Seite 243 im Bernstein-Buch)



		% Spieltheorie							!!! KORREKTURLESEN


%%%%%%%%%%%%%%%%%%%%%%%% part.tex %%%%%%%%%%%%%%%%%%%%%%%%%%%%%%%%%%
%
% sample part title
%
% Use this file as a template for your own input.
%
%%%%%%%%%%%%%%%%%%%%%%%% Springer-Verlag %%%%%%%%%%%%%%%%%%%%%%%%%%


\part{Seit 1975\\Neue Klassische Makroökonomie -- Neu-Keynesianismus -- Neue Neoklassische Synthese}

In den 1970er Jahren steckte die ökonomische Forschung in der Sackgasse. Der Keynesianismus (in der Form der neoklassischen Synthese) galt, zumindest im angelsächsischen Raum, spätestens seit der Ölkrise von 1973 als überholt. Die Zeit spielte schon lange gegen die österreichische Schule: Ihre liberalen Theorien waren zu radikal um sie wirtschaftspolitisch umzusetzen. Und ihre konkreten Befürchtungen, etwa in Hinsicht auf internationale Geldsysteme erwiesen sich als zu pessimistisch. Zum Beispiel waren die Währungssysteme auch ohne Goldstandard relativ stabil. Zudem galt ihre Methodik als widerlegt: Mathematik und empirische Forschung setzten sich in der Wissenschaft immer mehr durch. Aber auch Milton Friedman's Monetarismus galt in seiner Grundidee als zu kurz gegriffen: Der Monetarismus wurde durchaus wirtschaftspolitisch praktiziert. Aber die Konzentration alleine auf Geldpolitik hatte nicht die erwartete stabilisierende Wirkung auf die Preise.

Wer aber dachte das Pendel würde nach dem wirtschaftsliberalen Monetarismus wieder nach links, in Richtung Keynesianismus ausschlagen, täuschte sich. Die Lucas-Kritik kam wie ein Paukenschlag über die Ökonomie und brachte nichts anderes als die Rückbesinnung auf die "`(Neo-)Klassik"'.

Mit dem Aufkommen der sogenannten "`Neuen Klassischen Makroökonomie"' ist vor allem der Name Robert Lucas und seine Lucas-Kritik \parencite{Lucas1976} verbunden. Tatsächlich kam es Mitte der 1970er Jahre zu einer Revolution und gleichzeitig zu einer Spaltung der Wirtschaftswissenschaften.

Beginnen wir den neuen Teil mit einem kurzen Abriss des letzten Teils: Die wirtschaftswissenschaftliche Community war während und nach dem Zweiten Weltkrieg praktisch zur Gänze von Europa in die USA umgezogen. Dort sammelten sich die unterschiedlichen Schulen rasch an verschiedenen Orten. Angeblich war es Robert Hall im Jahre 1976, der die Ökonomen - durchaus humorvoll - in "`Salzwasser-"' und "`Süßwasserökonomen"' einteilte: "`Needless to say, individual contributors vary across a spectrum of salinity. [...] A few examples: Sargent corresponds to distilled water, Lucas to Lake Michigan, Feldstein to Charles River above the dam, Modigliani to the Charles below the dam, and Okun to the Salton See"' \parencite[S. 1]{Hall1976}. Tatsächlich war der Keynesianismus nach 1945 als Mainstream-Ökonomie sowohl an der Westküste als auch an der Ostküste die dominierende Schule. In Chicago, also an einem der großen Süßwasser-Seen, formte sich hingegen um Frank Knight und später um Milton Friedman, sowie schließlich eben um Robert Lucas eine Clique von Ökonomen, die den wirtschaftlichen Liberalismus wieder zurück in den Mainstream brachte.

Direkt nach dem Zweiten Weltkrieg wandten sich die meisten Wirtschaftswissenschaftler dem Keynesianismus zu. Die Österreichische Schule lehnte dessen Ideen hingegen von Anfang an strikt ab und war nach 1945 auch noch ein wissenschaftlich bedeutender Gegenpol.\footnote{Mit dem Tod von Schumpeter und Mises und dem Aufkommen der Chicago School in der Makroökonomie, vor allem mit Milton Friedman's Monetarismus, verlor die Österreichische Schule in der akademischen Welt allerdings bald an Bedeutung.} Der Keynesianismus (in der Form der neoklassischen Synthese) war hingegen unumstritten bis Mitte der 1960er Jahre die dominierende Schule. Spätestens mit dem Schock der Ölkrise 1973 verlor er aber den Status als "`State-of-the-Art"' in der Ökonomie. 

Schon Mitte der 1960er Jahre begann der wissenschaftliche Aufstieg des Monetarismus. Milton Friedman's und Anna Jacobson Schwartz' Werk \textit{A Monetary History of the United States, 1867–1960} im Jahr 1963 kann als wissenschaftlicher Grundstein des Monetarismus gesehen werden, der Höhepunkt erfolgte gegen Ende der 1960er Jahre. 

Mitte der 1970er Jahre schließlich - hier ist die Veröffentlichung der Lucas-Kritik im Jahr 1976 ein typischer zeitlicher Startpunkt - wurden die beiden vorher genannten Schulen abgelöst, beziehungsweise eigentlich im Endeffekt erweitert, von der Neuen Klassischen Makroökonomie. Interessant und wichtig ist der Unterschied bezüglich des vorherrschenden ökonomischen Denkens zwischen der wirtschafts-wissenschaftlichen Community und den Entscheidungsträgern in der Wirtschaftspolitik. Gerade in den 1970er und 1980er Jahren gab es hier einen beträchtlichen Time-Lag. In Bezug auf realpolitische Umsetzung kam die Zeit der wirtschaftsliberalen Schulen, Monetarismus und Neue Klassische Makroökonomie, nämlich erst später. Bis in die späten 1960er Jahre war der Keynesianismus in den USA sowohl unter demokratischen als auch republikanischen Präsidenten als wissenschaftliche Grundlage ihres wirtschaftspolitischen Handelns anerkannt \parencite[S. 12]{Woodford1999}. Das änderte sich erst in den 1970er Jahren: Die Wirtschaft der USA war damals von einer Berg-und-Talfahrt geprägt. Häufig spricht man im Hinblick auf diese Phase von "`Stagflation"', also niedriger BIP-Wachstumsraten, bei gleichzeitig permanent hoher Inflation. Genau diese wurde von den Keynesianern (zu) lange Zeit als relativ unproblematisch hingenommen, schließlich ging mit hoher Inflation zumeist eine niedrige Arbeitslosigkeit einher. Tatsächlich wurde die allgemeine Teuerung allerdings zunehmend als Problem wahrgenommen. Ein Problem für das die liberalen Ökonomen die besseren Erklärungsmodelle und Bekämpfungsmethoden bereitstellten. 

Interessanterweise läutete ein demokratischer Präsident, nämlich Jimmy Carter, die Hinwendung zu liberaler Wirtschaftspolitik ein. Er bestellte Alfred Edward Kahn zu einem wichtigen wirtschaftspolitischen Berater, der in den Folgejahren vor allem die Deregulierung und Privatisierung weiter Teile der staatlichen Industrie (Luftfahrt) vorantrieb. 1979 wurde Paul Volcker zum Vorsitzenden der US-Notenbank Federal Reserve (Fed) bestellt. In weiterer Folge wurde Inflationsbekämpfung als primäres Ziel ausgegeben. Schon mit diesen beiden Schritten hatten sich die USA wirtschaftspolitisch deutlich dem Monetarismus zugewendet. Eindeutiger wurde dies in den 1980er Jahren. Die \textit{Reaganomics} von Präsident Ronald Reagan in den USA, sowie der \textit{Thatcherism} in Großbritannien der "`Eisernen-Lady"', Premierministerin Margaret Thatcher,  bezogen sich offen auf den liberalen Monetarismus, aber auch auf die Österreichischen Schule. Friedrich Hayek und Milton Friedman fungierten dabei sogar direkt als wirtschaftspolitische Berater. Der Zeit-Lag ist deutlich sichtbar: Der politische Einfluss von Hayek und Friedman erfolgte Jahre nachdem ihre ökonomischen Schulen in der Wissenschaft ihren Zenit erreicht hatten.

Schon ab den frühen 1970er Jahren waren die wissenschaftlichen Erkenntnisse der "`Neuen Klassischen Makroökonomie"' entwickelt worden. Zwar setzten sich deren Erkenntnisse nach und nach durch, der enorme und vor allem \textit{direkte} wirtschaftspolitische Einfluss, den Hayek und Friedman erlangten, blieb Lucas, Sargent und Co aber verwehrt.

		% Neue Klassik - Neu Keynes - Neue Synthese
%%%%%%%%%%%%%%%%%%%%% chapter.tex %%%%%%%%%%%%%%%%%%%%%%%%%%%%%%%%%
%
% sample chapter
%
% Use this file as a template for your own input.
%
%%%%%%%%%%%%%%%%%%%%%%%% Springer-Verlag %%%%%%%%%%%%%%%%%%%%%%%%%%

\chapter{Neue Klassische Makroökonomie}
\label{Neue Makro}

\section{Lucas' Kritik und Sargent's Beitrag}
In Bezug auf die dogmengeschichtliche Einordnung könnte man argumentieren, die "`Neue Klassische Makroökonomie"' wäre eine Weiterentwicklung des Monetarismus. Dafür sprechen aber eigentlich nur ideologische und geografische Gründe. Die Vertreter beider Schulen, also des Monetarismus und der Neuen Klassischen Makroökonomie, sind zumindest wirtschaftspolitisch dem Liberalismus zuzuordnen. Außerdem war Robert E. Lucas Student und später Professor an der University of Chicago, gehörte also auch dem großen und erfolgreichen Zirkel von Ökonomen an, die nach Frank Knight und Milton Friedman in Chicago lehrten. Allerdings kritisierte Robert Lucas den Monetarismus zu vehement \parencite[S. 121]{Lucas1972}, als dass man bei der Neuen Klassischen Makroökonomie von einer Erweiterung des Monetarismus sprechen könnte.  

Entscheidend ist aber ohnehin die inhaltliche Sichtweise und hier unterscheiden sich die beiden Schulen doch entscheidend. Vor allem in der verwendeten Methodik brachte die "`Neue Klassik"' eine Revolution und sie brach gleich an mehreren Stellen mit den bisherigen Usancen der Ökonomie und ging ökonomische Fragestellungen grundlegend anders an, als sowohl Keynesianer als auch Monetaristen. Am bekanntesten ist das Beispiel der Phillips-Kurve: Der negative Zusammenhang zwischen Inflation und Arbeitslosigkeit. Keynesianische Wirtschaftspolitik akzeptierte eine hohe Inflation, weil damit eine niedrige Arbeitslosigkeit verbunden sei. Tatsächlich ließ sich dieser Zusammenhang überraschend eindeutig bis Ende der 1960er Jahre feststellen. Danach aber folgten der Ölpreisschock und Jahre der "`Stagflation"'. Also Jahre in denen es zwar kaum Wirtschaftswachstum, aber sowohl hohe Inflation als auch hohe Arbeitslosigkeit gab. Die Monetaristen um Milton Friedman (aber auch der Neu-Keynesianer Edmund Phelps) griffen die Keynesianer in Bezug auf die Phillips-Kurve an und argumentierten es mache auch schon theoretisch keinen Sinn einen langfristigen Zusammenhang zwischen der nominalen Größe Inflation und der realen Größe Arbeitslosigkeit anzunehmen. Inflation kann daher nicht kausal für niedrige Arbeitslosigkeit sein kann. Die Stagflation der 1970er-Jahre deckte die Schwächen der damaligen Makroökonomie auf. Aber auch die Monetaristen konnten die empirischen Vorgänge nicht befriedigend erklären.

Stattdessen war die Geburtsstunde der "`Neuen Klassischen Makroökonomie"' gekommen. In seiner wahrlich bahnbrechenden "`Lucas-Kritik"' \parencite[S. 19ff]{Lucas1976} zeigte er, dass es geradezu naiv sei zu glauben Arbeitnehmer würden Verträge abschließen in denen Löhne festgeschrieben werden, die aus heutiger Sicht zwar "`fair"' sind, aber durch Inflation in einem Jahr deutlich weniger Kaufkraft hätten. Ebenso naiv sei es zu glauben, dass Arbeitgeber Verträge abschließen nur weil sie wissen, dass die darin festgelegten Löhne in einem Jahr ohnehin real viel geringer seien. Nein! Beide Parteien, sowohl Arbeitnehmer als auch Arbeitgeber, wissen um den Einfluss der Inflation auf die Kaufkraft Bescheid und lassen ihre entsprechenden \textit{Erwartungen} -- selbstverständlich -- auch in die Lohnverhandlungen einfließen. Aus diesem Beispiel lassen sich auch die, für die Neue Klassische Makroökonomie so charakteristischen, Grundannahmen ableiten: Erstens, die "`Rationalen Erwartungen"', zweitens, die Betonung des "`natürlichen Gleichgewichts"' und der daraus folgenden Wirkungslosigkeit von Fiskal- und Geldpolitik und drittens der Mikrofundierung der Makroökonomie.

\begin{enumerate}
	\item Rationale Erwartungen: Der Begriff der "`Rationalen Erwartungen"' wurde eigentlich schon durch \textcite{Muth1961} begründet und von \textcite{Lucas1972} bereits erstmals als für die gesamte Makroökonomie gültiges Theorem vorgeschlagen. Ab Mitte der 1970er Jahre, vor alle durch die Arbeiten von \textcite{Lucas1976} und \textcite{Sargent1975}, wurde das Konzept der Rationalen Erwartungen etabliert. Seit damals gelten sie als \textit{die} wesentliche Neuerung durch die Neue Klassische Makroökonomie und zählen - obwohl nicht unumstrittenen - zum Kern der modernen Mainstream-Ökonomie. So leicht sind die "`Rationalen Erwartungen"' gar nicht abzugrenzen. Den (mikroökonomischen) \textit{Homo Oeconomicus} -- also den rational \textit{entscheidenden} Mensch -- gab es in der Ökonomie schließlich schon lange. Auch die Bedeutung von Erwartungen war nicht neu. Bei Keynes zum Beispiel wurden Änderungen der Zukunftserwartungen als "`Animal Spirits"' bezeichnet. Diese waren bei Keynes ein wichtiges Konzept, dass aber als nicht modellierbar akzeptiert wurde. Die \textit{rationalen Erwartungen} umfassen eben mehr. Während der Homo Oeconomicus nutzen-maximierend vergangenheitsorientierte Information auswertet\footnote{Häufig wird zwischen "`Adaptiven Erwartungen"' und eben "`Rationalen Erwartungen"' unterschieden. Bei ersteren Konzept ziehen die Leute ausschließlich vergangene Daten heran und schreiben diese in die Zukunft fort. Beim zweiten Konzept leiten die Leute hingegen wahrscheinliche zukünftige Handlungen ab um ihre Zukunftserwartungen herzuleiten.}, umfassen die Rationalen Erwartungen die Tatsache, dass sich Menschen auch rational verhalten was die Informationsgewinnung betrifft und entsprechend rationale "`Vorhersagen"' zur zukünftigen Entwicklung wirtschaftlicher Aspekte treffen. Das heißt, eine bestimmte Wirtschaftspolitik bestimmt auch die Erwartungen der Menschen. Änderungen der Wirtschaftspolitik führen dementsprechend auch zu Änderungen der Erwartungshaltungen. Zusammengefasst: Erstens, Menschen machen vorhersagen, ohne dabei systematische Fehler zu begehen (\textcite{Lucas2013}: Interview mit Lucas: "`Die Leute sind nicht verrückt"'). Das heißt, der Staat kann seine Bewohner nicht systematisch "`austricksen"'. Zweitens, Menschen bauen alle verfügbaren Informationen und Wirtschaftstheorien in ihre Entscheidungen ein. Diese Annahme ist umstritten. Auf Beispiele herunter gebrochen bedeuten Rationale Erwartungen folgendes:
	Hohe Budgetdefizite durch expansive Fiskalpolitik führen dazu, dass die Menschen steigende Steuerbelastung erwarten und ihre Sparquoten erhöhen. Anstatt der von der Politik erhofften Konjunktur-belebenden Wirkung führt die Fiskalpolitik zu einer Verdrängung privater Ausgaben durch staatliche Ausgaben. Politiker, die eine hohe Inflation bewusst nutzen wollen um die Arbeitslosenraten zu senken, werden enttäuscht: Erwarten die Menschen höhere Inflationsraten, fordern sie in den Lohnverhandlungen eine entsprechende Abgeltung dafür, womit das bewusste Ausnutzen der hohen Inflation zugunsten niedriger Arbeitslosenzahlen scheitert. Die neue klassische Makroökonomie konnte damit elegant das konkrete Phänomen der "`Stagflation"' der 1970er Jahre erklären. 	 
	Diese beiden Beispiele zeigen den enormen Nebeneffekt der Rationalen Erwartungen: Nämlich, dass weder Fiskalpolitik noch Geldpolitik\footnote{Nur wenn die politischen Handlungen absolut unvorhergesehen erfolgen, können damit kurzfristig die erwünschten Effekt eintreten.}, einen stabilisierenden Einfluss auf die gesamtwirtschaftliche Entwicklung haben \parencite{Sargent1975, Barro1976}. Wirtschaftspolitik \textit{steuert} nicht länger die ökonomische Entwicklung, sondern ist nur ein \textit{Player} in einem Spiel zwischen Politik und den Markteilnehmern \parencite{Kydland1977}. Dies brachte die Spieltheorie in die Makroökonomie, was ebenfalls dem Zeitgeist entsprach und die Neue Klassische Makroökonomie noch "`sexier"' machte.	
	Abgesehen davon ist dies aber natürlich auch der absolute Bruch mit den Lehren des Keynesianismus.
	Die Wirkungslosigkeit jeglicher damals bekannter wirtschaftspolitischer Elemente, führt uns direkt zum zweiten wesentlichen Punkt der Neuen Klassischen Makroökonomie.
	
	\item Dynamisches natürliches Gleichgewicht: Sowohl keynesianische als auch monetaristische Modelle akzeptierten, dass Lohn- und Preisanpassungen auf den Märkten mit einer gewissen \textit{zeitlichen} Verzögerung eintraten. Eine Erhöhung der Geldmenge zum Beispiel führe demnach zunächst zu einer Erhöhung der Produktion, gleichzeitig zu einer Verringerung der Arbeitslosigkeit und erst in weiterer Folge zu höheren Nominallöhnen und höheren Preisen. Erst nach mehreren Runden dieser Anpassungsprozesse sind die Löhne und Preise wieder im Gleichgewicht.
	In der "`Neuen Klassischen Makroökonomie"' gibt es aufgrund der rationalen Erwartungen diese langfristigen Anpassungsprozesse nicht. Dementsprechend sind alle Märkte auch in der kurzen Frist im Gleichgewicht und vollständige Konkurrenzmärkte\footnote{Auf diese Annahmen ist auch der Name "`Neue \textit{Klassische} Makroökonomie"' zurückzuführen.}. 
	Dies wiederum impliziert, dass Konjunkturschwankungen ausschließlich durch exogene Schocks verursacht werden. Diese Implikation ist notwendig, da Konjunkturschwankungen ständig empirisch zu beobachten sind, bei vollständiger Konkurrenz mit flexiblen Löhnen und Preisen müssten schließlich die Lohn- und Preisanpassungen laufend zu Markträumung und Glättung der Konjunkturschwankungen führen. Die "`Real Business Cycle"'-Theorie von Edward Prescott formalisierte diese Annahmen (siehe Kapitel \ref{RBC}).
	
	Dieser zweite Kernpunkt der "`Neuen Klassischen Makroökonomie"' wurde allerdings zum "`Sargnagel"' dieser ökonomischen Schule. Anfang der 1980er Jahre sah es so aus, als würde die Neue Klassische Makroökonomie zur alleinigen Mainstream-Ökonomie aufsteigen. Aber dafür stellten sich die vorausgesetzten Annahmen als zu realitätsfern heraus. Dass alle Märkte vollkommene Konkurrenzmärkte sind und sich ständig im Gleichgewicht befinden, ist einfach zu weit weg von täglichen Beobachtungen: Erstens war diese Annahme nicht vereinbar mit der empirischen Beobachtung von Arbeitslosenraten von fast 10\% in den USA der frühen 1980er Jahre. Zweitens, zeigten der Neukeynesianer \textcite{Fischer1977} mittels formaler Modelle, dass sich Löhne und Preise auch bei Berücksichtigung der Theorie der Rationalen Erwartungen nur langsam an veränderte Arbeitslosenraten anpassen.
	Und drittens, erwiesen sich die "`Real Business Cycle"' Modelle bald als wenig treffsicher. Sie implementierten die oben beschriebenen Annahmen, dass es ständig zu Markträumung kommt und es keine natürliche Arbeitslosigkeit gibt. Die Konjunkturzyklen würden dann primär durch Schwankungen im technischen Fortschritt verursacht. Die Modelle konnten zwar vor allem methodisch überzeugen, aber die empirisch beobachtete Konjunkturschwankungen nicht erklären.
	Selbst innerhalb der Neuen Klassischen Makroökonomie akzeptierte man bald, dass die Annahmen zu starr sind und verwarf einige davon\footnote{Dies "`nutzten"' die Neu-Keynesianer, die pragmatisch die bahnbrechenden Erkenntnisse der Neuen Klassik übernahmen, aber realistischere Marktannahmen zu Märkten, Rigiditäten, Arbeitslosigkeit und Wirtschaftspolitik verwendeten}. Was exogene Wachstumsmodelle betrifft, überließen die Neuen Klassiker bald den Neu-Keynesianern das Feld. Innerhalb der Neuen Klassik wendete man sich den "`Endogenen Wachstumstheorien"' zu.
	
	\item Mikrofundierung der Makroökonomie. Die keynesianischen und monetaristischen Modelle verwendeten typische makroökonomische Kennzahlen, wie zum Beispiel Bruttoinlandsprodukt, Konsum, Investitionen und Sparen. Das Zusammenspiel dieser Kennzahlen wurde häufig mit sogenannten "`Ad-hoc-Annahmen"' modelliert. Das heißt, es wurden Zusammenhänge herangezogen, die nicht weiter begründet wurden. Ein Beispiel ist die Einkommenshypothese nach Keynes, wonach Der Konsum eine - nicht weiter spezifizierte - Funktion des derzeitigen Einkommens sei. Zum Teil wurden einzelne dieser Kennzahlen im Laufe der Zeit abgeändert dargestellt. So erweiterten die "`permanente Einkommenshypothese"' von Milton Friedman und die "`Lebenszyklushypothese"' von Franco Modigliani die eben genannte keynesianische Konsumtheorie. Aber es blieb immer bei der Heranziehung statischer makroökonomischer Faktoren. Natürlich wusste man auch vor Lucas' Kritik, dass jede Entscheidung nicht ausschließlich auf statischen, gegenwärtigen Fakten basierten und Vermutungen über die Zukunft blieben nicht völlig ausgeklammert. Aber doch war Lucas' Kritik ein entscheidender Anstoß für ein Umdenken innerhalb der Ökonomie von statischen Überlegungen zur Implementierung dynamischer Erwartungen. Dementsprechend wurden auch ökonometrische Modelle völlig neu gedacht. Lange Zeit dominierten zuvor in der Makroökonomie statische keynesiansche Totalmodelle, beziehungsweise die neoklassischen, mikroökonomischen Walrasianischen Gleichgewichtsmodelle. 
	
	Die Neue Klassische Makroökonomie brachte auch hier eine Revolution:\footnote{In den Ökonomie-Lehrbüchern sah man das daran, dass das Standardmodell der neoklassischen Synthese, das IS-LM-Modell, zunehmend durch das einfachste mikrofundierte Modell, das AS-AD-Modell ersetzt wurde.} Die Mikrofundierung der Makroökonomie. Heute sind praktisch alle Makroökonomischen Modelle mikroökonomisch fundiert. Konkret bedeutet dies, dass man die Konzepte aus der Mikroökonomie, also die Nutzenmaximierung aus der Haushaltstheorie und die Gewinnmaximierung aus der Unternehmenstheorie heranzieht. Man kann aber nicht alle Individuen auf allen Märkten beobachten und deren Verhalten zu einem "`Gesamtverhalten"' aggregieren, also aufsummieren. Stattdessen behilft man sich eines \textit{repräsentativen Agenten}, also ein "`Haushalt"' oder "`Unternehmen"', der/das typisches Verhalten zeigen würde. Mit den in der Mikroökonomie üblichen Techniken wird dann \textit{optimales} Verhalten des Agenten modelliert. Das heißt, man nimmt an, dass alle Modellgleichungen auf \textit{konsistenten Annahmen} beruhen. In diesem Umfeld optimiert der Agent sein Verhalten, wobei er \textit{rationale Erwartungen} über zukünftige Entwicklungen hat. Der Term rational ist hier im Sinne von "`stochastisch berechenbar"' zu verstehen. Die Modelle sind dynamisch, der repräsentative Agent passt also sein Optimierungsverhalten schlagartig auf Veränderungen in seinen Erwartungen oder den konsistenten Annahmen an und die einzelnen Gleichungen beeinflussen sich gegenseitig. Damit ist das Verhalten des repräsentativen Agenten konsistent mit den Vorhersagen des Modells. 
	
	Der Leser mag sich aufgrund der komplizierten Formulierungen und Vielzahl an Annahmen denken, dass die Ökonomie mit diesen Modellen den Bezug zur Realität vollkommen verloren hat. Höheren Mathematik war endgültig in ökonomischen Modellen angekommen. Dies entsprach und entspricht dem Zeitgeist. Die Ökonomie wurde in der Folge zunehmend als Naturwissenschaft oder gar Formalwissenschaft betrieben, immer weniger als Sozialwissenschaft. Ein Umstand der seither mit wechselnder Vehemenz kritisiert wird, ob zu Recht oder zu Unrecht muss jeder Leser für sich entscheiden. Im Mainstream haben sich diese Modelle auf jeden Fall fest etabliert. Neben der Neuen Klassischen Makroökonmie, setzten auch die Neu-Keynesianer auf diese Art von ökonomischen Modellen. Die heute so häufig herangezogenen neukeynesianischen "`Dynamischen stochastischen allgemeinen Gleichgewichtsmodelle"' unterscheiden sich zwar deutlich von den dynamischen Gleichgewichtsmodellen der "`Neuen Klassischen Makroökonomie"', basieren aber im wesentlichen auf deren Ideen.  
	Man könnte sogar so weit gehen, dass diese Art der ökonomischen Modelle den Mainstream von den heterodoxen Schulen trennt. Sowohl die österreichische Schule als auch die Post-Keynesianer und natürlich die Verhaltensökonomen, lehnen diesen stark formalisierten Zugang jedenfalls strikt ab.


\end{enumerate}	


Die Neuen Klassiker erlebten rasch einen enormen Aufschwung und enorme Beachtung. Rasch war klar, dass ihre formell tatsächlich sehr schön dargestellten und abgehandelten Modellen, den keynesianischen und monetaristischen Modellen formal überlegen waren.
Vor allem die Mikrofundierung der Makroökonomischen Modelle stellte einen deutlichen und nachhaltigen Fortschritt dar. Schließlich akzeptierten auch die Keynesianer die modelltheoretische Überlegenheit\footnote{Die Implementierung keynesianischer Ideen in die Modellannahmen der Neu Klassiker machte die Keynesianer schließlich zu Neu-Keynesianern}
Damit konnte sich die Annahme der Rationalen Erwartungen rasch etablieren. Schließlich war dieses Konzept nicht neu, aber eben technisch schwer umsetzbar, da bei rationalen Erwartungen eine Wechselwirkung zwischen erwarteten zukünftigen Entwicklungen und heutigem Verhalten besteht. Genau das Problem wurde mit den Modellen der "`Neuen Klassiker"' gelöst. Und so sah es Anfang der 1980er Jahre danach aus, als würde sich die "`Neue Klassische Makroökonomie"' als neue, alleinige Mainstream-Ökonomie etablieren. Aber auch das erwies sich rasch als Trugschluss. Die Annahmen der Modelle waren einfach zu stark und zu starr als dass damit die Realität beschrieben werden konnte. 
		
Im Gegensatz zu den Neukeynesianern, die sich zu dieser Zeit ebenfalls formierten und deren Hauptthemen die verschiedenen Formen von Marktversagen waren, meinten die neuen Klassiker, dass alle Märkte selbstständig ein Gleichgewicht finden. Und das auch in der kurzen Frist! Rückbesinnung auf die Klassik eben. Dieser Punkt erwies sich rasch als nicht haltbar. Dies impliziert, dass es keine unfreiwillige Arbeitslosigkeit gäbe, was wohl unrealistisch ist. Genau das war auch der zentrale Angriffspunkt auf die Neue Klassik. "`Wenn der Arbeitsmarkt immer ins Gleichgewicht findet, dann heißt das, dass die Neuen Klassiker davon ausgehen, dass sich mitten in der "`Great Depression"' Millionen Amerikaner dafür entschieden haben jetzt mehr Freizeit zu konsumieren"' (cf \cite{Stiglitz1987}, p. 119), lautete ein hämischer Kommentar der Neu-Keynesianer. 
Warum aber war die Neue Klassik zu deren Beginn so erfolgreich? Meines Erachtens liegt der Grund hierfür in ihrer methodischen Überlegenheit gegenüber Keynesianern und Monetaristen. Wenn alle Modellannahmen eingehalten sind, dann führen die Modelle unwiderlegbar und sehr elegant zu eindeutigen Ergebnissen. Allerdings stellte man rasch fest, dass viele der Modellannahmen empirisch nicht zu halten sind. Robert Solow in einem Interview, dass in \textcite[S. 146]{Klamer1984} veröffentlicht wurde, brachte dies wohl am besten auf den Punkt:

\textit{"'Angenommen, jemand [...] sagt zu ihnen, er sei Napoleon Bonaparte. Das Letzte, was ich möchte, ist, mich mit ihm auf eine technische Diskussion über die Kavallerietaktik in der Schlacht von Austerlitz einzulassen. Wenn ich das tue, werde ich stillschweigend anerkennen, dass er Napoleon ist. Nun, Bob Lucas und Tom Sargent mögen nichts lieber, als technische Diskussionen vorzunehmen, denn dann haben Sie sich stillschweigend auf ihre Grundannahmen eingelassen. Ihre Aufmerksamkeit wird von der grundlegenden Schwäche der ganzen Geschichte abgelenkt."'}

Gerade als sich Anfang der 1980er-Jahre die Neue Klassik als neue Mainstream Ökonomie durchzusetzen schien, stiegen die Arbeitslosenquoten in den USA auf 10\%. Das war nicht vereinbar mit der angeblich ausschließlich freiwilligen Arbeitslosigkeit. Mittlerweile rücken auch die meisten Vertreter der neuen Klassik davon ab, dass sich alle Märkte auch in der kurzen First im Gleichgewicht befinden (Zitat).
Man darf daraus jetzt aber nicht schließen, die Neue Klassische Makroökonomie sei widerlegt und verschwunden. Ganz im Gegenteil! Sie hat viele wichtige und richtige Erweiterungen der Mainstream-Ökonomie gebracht. Erstens, die Methodik wurde revolutioniert. Mikrofundierte, dynamische Gleichgewichtsmodelle wurden von den Neu-Keynesianern rasch aufgenommen, erweitert und für sich beansprucht. Die Anerkennung für die Überwindung der veralteten Makromodelle der Keynesianer und Monetaristen steht aber den Neuen Klassikern zu. Zweitens, die Theorie der Rationalen Erwartungen war - trotz aller Kritik - ein Meilenstein in der Ökonomie, der bis heute State-of-the-Art ist.

Die Veröffentlichung der Lukas-Kritik gilt als eine Revolution in der Ökonomie. Warum aber blieben deren Vertreter, allen voran Robert Lucas, in der öffentlichen Wahrnehmung eher blass? Ein Aspekt ist sicherlich, dass sowohl Keynes als auch Friedman und Hayek auch außerhalb der wissenschaftlichen Ökonomie auftraten, vor allem als Politikberater. Ein weiterer Aspekt ist aber auch die Art der Kommunikation der Vertreter der "`Neuen Klassischen Makroökonomie"'. Diese war ungewöhnlich scharf: \textit{That [the Keynesian] predictions were wildly incorrect and that the doctrine on which they were based is fundamentally flawed are now simple matters of fact}, schrieben Lucas und Sargent in ihrem Artikel \textit{After Keynesian Macroeconomics} \parencite[S. 1]{Lucas1979}. Die Ideen der Neuen Klassiker wurden schon nach wenigen Jahren in die Modelle des bisherigen Mainstreams integriert, nicht jedoch die Leute, die Stimmung innerhalb der wirtschaftswissenschaftlichen Community war in den 1970er und 1980er Jahren vergiftet, beschreibt \cite{Blanchard2003} in seinem Standardlehrbuch. Ähnlich, wenn auch etwas diplomatischer drückte sich \cite{Samuelson1998} aus. Lucas und Co kümmerten die etablierten Ökonomen wenig, es scheint als hielten sie so wirklich gar nichts von ihnen. Umgekehrt erkannten vor allem die Vertreter der neoklassischen Synthese die inhaltliche Sinnhaftigkeit der Ideen der Neuen Klassiker. Deren "`Schüler"' implementierten diese Ideen in ihre eigenen Modelle, deckten die vorhandenen Schwachpunkte der "`Neu Klassiker"' auf und wurden zu "`Neu-Keynesianern"'. Die Modelle näherten sich also an - vor allem weil die Mainstream-Ökonomen die Ideen der Neuen Klassiker aufnahmen - die dahinterstehenden Personen allerdings in keinster Weise. 

Um sich als Mainstream durchzusetzen, waren die Ideen der Neuen Klassiker zu radikal. Die gänzliche Ablehnung der Synthese aus Neoklassik und Keynesianismus erwies sich als vorschnell. Überhaupt zeigte sich der Hauptvertreter der Neuen Klassischen Makroökonomie, Robert Lucas, als wenig pragmatisch was seine ökonomische Sichtweise angeht. Im Jahre 2003 zum Beispiel veröffentlichte er einen seiner Artikel im \textit{American Economic Review} mit der These, dass \textit{macroeconomics in [the] original sense has succeeded: Its central problem of depression-prevention has been solved, for all practical purposes, and has in fact been solved for many decades} \parencite[S. 1]{Lucas2003}. 
Dass nur vier Jahre später mit der "`Great Recession"' die größte Krise seit den 1930er Jahren ausbrechen sollte, zeigte das Gegenteil. Bereits 1987 meinte er: \textit{The most poisonous [tendencies in economics], is to focus on questions of distribution} \parencite{Lucas1987}. Fragen der Einkommensverteilung sind aber seit damals tatsächlich gesellschaftlich wie ökonomisch immer bedeutender geworden. Auf die Frage, ob Ökonomie-Studierende heute noch Keynes lesen sollten, antwortete er 1998 mit einem schlichten "`No"' \parencite{Lucas2013}. Die "`Great Recession"' war somit so etwas wie die "`Widerlegung"' der reinen neuen klassischen Makroökonomie. Lucas sah die Krise, so wie zugegebenermaßen die meisten andern Ökonomen nicht nur nicht kommen, sondern, glaubte auch nicht, dass eine derart schwere Krise kommen könnte. Und während der Krise wendeten die Politiker schließlich gnadenlos keynesianisches "`Deficit Spending"' sowie eine extreme Geldpolitik - "`Quantitative Easing"' - an. 

Insgesamt darf man aus heutiger Sicht darf man aber nicht vergessen, dass diese ökonomische Schule die Wirtschaftswissenschaften tatsächlich revolutioniert hat. Viele der von ihr erstmals vorgebrachten Elemente wurden rasch vom Großteil der Ökonomen aller Richtungen übernommen und sind heute aus der Mainstream-Ökonomie nicht mehr wegzudenken. Robert Lucas zählt daher meines Erachtens zu den größten Ökonomen des 20. Jahrhunderts. Seine Arbeiten sind nicht so einprägend wie die "`General Theory"' von Keynes. Sein Auftritt ist nicht so überzeugend wie jener von Milton Friedman, der durch seine politischen Tätigkeiten auch weit außerhalb der wirtschaftswissenschaftlichen Community bekannt wurde. Aber Robert Lucas stand den beiden in nichts nach. Seine Ideen revolutionierten die Ökonomie des 20. Jahrhunderts und machten daraus eine andere Wissenschaft. Keynes wird oft als genialer "`Lebemann"' dargestellt. Er starb schon zehn Jahre nach der Veröffentlichung seine bahnbrechenden Werkes und musste es selbst nicht mehr gegen Angriffe verteidigen. Vielleicht ist er auch deshalb so populär. Friedman wird sehr kontrovers gesehen. Von den Liberalen noch im hohen Alter als Ikone gefeiert, durch seine Beratertätigkeit oft jedoch auch verhasst. Vor allem aber war er ein brillanter Redner mit charismatischen Auftritt. All das ist nicht die Stärke von Robert Lucas. Seine oben zitierten Aussagen scheinen eher unglücklich formuliert. In seinen Auftritten erscheint er sympathisch, aber nicht als der große Vortragende. Robert Lucas war dafür ein brillanter Wissenschaftler. Seine messerscharfen formalen Abwandlungen prägten Generationen von Studierenden und waren bei seinen Gegnern gefürchtet. Dafür gebührt ihm bleibende Anerkennung und Wertschätzung in der Ökonomie.


\section{Real Business Cycle Theorie}
\label{RBC}
Dass die Makroökonomie auf der Basis von Mikrofundierung ein Fortschritt gegenüber den alten keynesianischen und auch monetaristischen Modellen sei, geht bereits auf Edmund Phelps und eben Robert Lucas zurück. Aber \textit{entwickelt}, und somit in die praktische Anwendung umgesetzt, wurden diese Modelle von \textsc{Edward Prescott} und \textsc{Finn Kydland}. Diese veröffentlichten zusammen zwei bahnbrechende Artikel \parencite{Kydland1977, Kydland1982}. Die erstmalige Entwicklung dynamischer, mikroökonomisch basierter, ökonometrischer Modelle ist die wahre Errungenschaft, die die beiden getätigt haben. Diese Modelle waren die Vorläufer der "`Dynamischen, stochastischen, allgemeinen Gleichgewichtsmodelle (DSGE)"'. Diese gelten bis heute als der Goldstandard der Konjunkturprognose-Modelle, wenn auch die Parameter wesentlich erweitert wurden. (Siehe Kapitel \ref{Neue Neoklassische Synthese})

Kydland und Prescott lieferten ein Modellfundament für zwei wesentliche Bausteine der "`Neuen Klassischen Makroökonomie"': In \textcite{Kydland1977} formalisierten die beiden was ursprünglich Robert Lucas postuliert hatte: Nämlich, dass aktive Wirtschaftspolitik (also sowohl Geldpolitik als auch Fiskalpolitik) nicht den erwünschten stabilisierenden Effekt hat. Im Gegenteil durch rationale Erwartungen und eine Zeitverschiebung (lag) zwischen Beschluss, Umsetzung und Wirksamkeit wirtschaftspolitischer Maßnahmen käme es laut Kydland und Prescott sogar dazu, dass aktive Wirtschaftspolitik im Endeffekt destabilisierend wirke \parencite[S. 486]{Kydland1977}. Dieses "`Zeitinkonsistenz-Problem"' fand es rasch in die Lehrbücher zur Wirtschaftspolitik als "`Inside Lag"' - also die Zeit, die vergeht, bis sich eine Regierung oder eine Zentralbank durchringen kann wirtschaftspolitische Entscheidungen zu treffen - und "`Outside Lag"' - also der Zeitraum der notwendig ist, bis die fiskalpolitischen oder geldpolitischen Maßnahmen tatsächlich wirken. Als Antwort auf dieses Problem kommen die beiden im Artikel - im Einklang mit \textcite{Lucas1976} - zu dem Schluss, dass eben "`Regeln statt [wirtschaftspolitischer] Entscheidungen"' die wirtschaftliche Entwicklung stabilisiere.

In ihrer zweiten großen Arbeit \parencite{Kydland1982} etablierten sich die beiden schließlich als Hauptvertreter der "`Real Business Cycle"'-Theory. Eine ganz ähnliche Arbeit lieferten \textcite{Plosser1983}. Nachdem die Neuklassiker und eben die beiden selbst \parencite{Kydland1977} vorgeschlagen hatten eine langfristige, auf Regeln aufgebaute, Wirtschaftspolitik durchzuführen, benötigte man ein Modell, das prognostizierte wie sich die Ökonomie unter diesen Umständen entwickeln würde. In den Jahrzehnten davor, also die 1960er und die 1970er Jahre, herrschte auf der einen Seite die keynesianische Ansicht vor, dass es Aufgabe der Wirtschaftspolitik war die Konjunkturzyklen zu glätten. Auf der anderen Seite gab es seit 1956 relativ unumstritten das "`Solow-Wachstumsmodell"', mit dem langfristiges Wachstum fast ausschließlich durch technologischen Fortschritt generiert würde.
Kydland und Prescott argumentierten nun, entgegen dem vorherrschen keynesianischen Mainstream, dass Konjunktureinbrüche nicht aufgrund fehlender Nachfrage verursacht würden, sondern vor allem durch angebotsseitige Schocks, wie zum Beispiel plötzlich steigende Rohstoffpreise, oder exogene Faktoren wie Naturkatastrophen. Die Konjunkturzyklen wären demnach keine Folge, erstens keine Folge eine nicht-funktionierende Wirtschaft und zweitens, diese exogenen Schocks würden rein zufällig auftreten. Schließlich haben Naturkatastrophen zwar Auswirkung auf die Wirtschaft, aber umgekehrt kann man diese nicht mit ökonomischen Modellen vorhersagen oder diese in ökonomische Modelle implementieren. Die "`Real Business Cycles"' sind demnach Folgen von Faktoren, die außerhalb der Ökonomie liegen, haben aber einen Effekt auf diese Ökonomie. Das eben genannten impliziert auch, dass die Abschwünge im Konjunkturzyklus rein zufällig auftreten. Da die exogenen Faktoren, zum Beispiel Naturkatastrophen nicht von der Ökonomie abhängen und vorhergesagt werden, treffen sie eine Ökonomie gezwungenermaßen vollkommen unerwartet, also wie zufällig. Genau wie in der neoklassischen Finance (siehe Kapitel XXX) sich Aktienkurse nach einem "`Random Walk"' bewegen, bewegt sich der gesamte Konjunkturzyklus ebenso nach einem reinen Zufallspfad.

Dies waren die zusätzlichen, theoretischen Implikationen, die Kydland und Prescott im Artikel von 1982. Fassen wir zusammen, welche Bedingungen Kydland und Prescott festlegten:

Das Modell stand in der Tradition der "`Neuen Klassischen Makroökonomie"', dementsprechend gelten die drei bereits genannten Ausgangspunkte, die hiermit nur kurz angeführt werden:
\begin{enumerate}
\item Dynamisches Modell
\item Mikrofundiertes Modell
\item Rationale Erwartungen werden berücksichtigt
\end{enumerate}
Des weiteren wurden aber noch folgende Punkte zusätzlich festgelegt.
\begin{enumerate}
\item[4.] Es wird davon ausgegangen, dass die Märkte effizient sind. Das ist in Verbindung mit Rationalen Erwartungen zu sehen: Wenn keine systematischen Fehler bei den Marktbewertungen passieren, dann sind die Märkte eben effizient in der Markträumung. Dies führt uns zu den nächsten beiden Punkten.
\item[5.] Effiziente Märkte befinden sich im Gleichgewicht. Ähnlich wie Walrasianische Gleichgewichtsmodelle, befinden sich auch die mikrofundierten makroökonomischen Modelle im Gleichgewicht.
\item[6.] Prescott und Kydland nehmen dabei an, dass dabei keinerlei Rigiditäten gibt und die eben angesprochenen Gleichgewichte praktisch immer vorhanden sind, also auch in der kurzen Frist.
\item[7.] Alle Märkte in diesem Modell sind außerdem vollkommene Konkurrenzmärkte. Das heißt es gibt auf allen Märkten jeweils auf beiden Marktseiten eine große Anzahl von Teilnehmern. Dementsprechend verfügt kein einzelner Teilnehmer über eine bemerkenswerte Marktmacht.
\item[8.] Die Märkte werden, wie oben dargestellt, wenn dann durch exogene Schocks, angebotsseitig aus dem Gleichgewicht gebracht. Diese Schocks umfassen beispielsweise die schon angesprochenen Naturkatastrophen, explizit angesprochen sind aber technologische Fortschritte. \parencite[S. 1345]{Kydland1982}. Diese rücken in den Vordergrund bei der Analyse von Konjunkturzyklen.
\end{enumerate}

Um diese Annahmen herum bauen die beiden ein ökonometrisches Modell auf. Und dieses \textit{Modell} ist der bahnbrechende Beitrag von Kydland und Prescott. Es gilt heute als erstes "`Dynamischen Stochastisches General Equilibrium"'-Modell. Diese - heute meist abgekürzt genannten - DSGE-Modelle waren den bisherigen keynesianischen Modellen und vor allem den monetaristischen Ansätzen, modelltheoretisch haushoch überlegen. Sie ließen die Ökonomie eine ganz neue Richtung einschlagen. Ab diesem Zeitpunkt beherrschten formalisierte, quantitative Modelle die ökonomische Forschung, zumindest im Mainstream. Rein methodisch war dieses erste Modell der Startschuss für eine Unmenge an Forschungsarbeiten unter anderem im Rahmen der Zeitreihenmodelle. Durch die Vergabe des Ökonomie-Nobelpreises 2001 wurde in diesem Zusammenhang Christopher Sims\footnote{Er erhielt den Preis zusammen mit dem bereits genannten Thomas Sargent} und seine Entwicklung der "`Vektor-Autoregressive"'-Modelle bekannt.

Die modelltheoretische Überlegenheit wurde weitgehend akzeptiert. Damit sah es Mitte der 1980er Jahre so aus, als würde die "`Neue Klassische Makroökonmie"' mit der damit eng verwandten "`Real Business-Cycle"'-Theorie, als deren Umsetzung,  die neoklassische Synthese als Mainstream-Ökonomie ablösen.

Allerdings hatte die "`Real Business Cycle"'-Theorie Schwächen, die bald unübersehbar wurden. Die getätigten, oben beschriebenen Annahmen erwiesen sich rasch als teilweise empirisch nicht haltbar. 
Vor allem das nur halbherzig behandelte Problem der Arbeitslosigkeit wurde später immer wieder von Kritikern aufgegriffen. So sind sich Kydland und Prescott natürlich bewusst, das Arbeitslosigkeit ein zentrales Thema der Makroökonomie ist. Aber sie handeln das in ihrem Artikel extrem kurz ab. Sinngemäß schreiben sie, dass die Menschen nicht nur "`dem Konsum, sondern auch der Freizeit einen Wert zuweisen"' \parencite[S. 1345]{Kydland1982}. Das wurde dem Modell später zum Verhängnis, wenn man so will. Der Angriffspunkt war natürlich, dass das Modell damit unterstellt, dass Arbeitslosigkeit dadurch entsteht, dass Menschen nicht arbeiten, weil sie Freizeit höher einschätzten als Konsum.
Das ist aber nicht mit der empirischen Beobachtung, vor allem während Wirtschaftskrisen, vereinbar. Man könnte diese Ansicht gar als Zynismus auslegen.

HIER DIE KRITIKPUNKTE aus Romer, Advanced-Macro-Buch von Seite 227ff einbauen.

Außerdem wurde bald die Forscherkonkurrenz tätig: Die Vertreter der Neoklassischen Synthese akzeptierten rasch die methodische Überlegenheit, der "`Neuen Klassiker"' und übernahmen deren Modelle bald. Natürlich nicht, ohne Anpassungen vorzunehmen: So wurden die Märkte nicht mehr als vollkommene Konkurrenzmärkte gesehen und es wurden ein Arbeitsmarktmodelle entwickelt, welche die Berücksichtigung von Rigiditäten und somit unfreiwilliger Arbeitslosigkeit innerhalb der Theorie der rationalen Erwartungen ermöglichte. Zudem wurde zunehmend die Effizienz von verschiedenen Märkten in Frage gestellt. Diese drei Punkte können als Geburtsstunde der "`Neu-Keynesianer"' gesehen werden. Daher in Kapitel \ref{cha: Neu Keynes} mehr dazu.

Was aber blieb von der "`Real Business Cycle"'-Theorie? Kydland und Prescott können als Urväter mikrofundierter, dynamischer Modelle und damit auch heute noch weitverbreiteter DSGE-Modelle gesehen werden. Die beiden Autoren wurden daher zurecht mit dem Ökonomie-Nobelpreis 2004 ausgezeichnet.
Auch wenn der Inhalt der RBC rasch stark infrage gestellt wurde. In den beiden ökonomischen Lehrbüchern von Mankiw und Blanchard, wird die "`Real Business-Cycle"'-Theorie überraschend scharf als einfach falsch dargestellt. Tatsächlich überwogen bald die Kritiker an den Modellen und was blieb waren die formal eleganten und fortschrittlichen Modelle auf methodischer Seite. Aber inhaltlich setzten sich in den 1990er Jahren die "`Neu-Keynesianer"' als Mainstream-Ökonomie durch. Wohlgemerkt unter Berücksichtigung wesentlicher Elemente, die die "`Neue Klassik"' hervorgebracht hat. Man kann daher auch argumentieren, die neue Mainstream-Ökonomie der 1990er Jahre war eine Kombination aus "`Neu-Keynesianismus"' und "`Neuer Klassischer Makroökonomie"'. Dagegen spricht allerdings, dass die beiden Schulen in den 1990er Jahren noch stark nebeneinander statt miteinander aktiv waren.
Von einer Verschmelzung der "`Neuen Klassik"' und der "`Neuen Keynesianer"' zur neuen Mainstream-Ökonomie würde ich daher erst später sprechen.


\section{Barro: Ricardianische Äquivalenz}
Neben Robert Lucas und Thomas Sargent gilt auch \textsc{Robert Barro} als einer der Väter der Neuen Klassischen Makroökonomie. Sein Werdegang ist dahingehend interessant, dass er in seinen 20ern durchaus erfolgreich keynesianische Journalartikel verfasste \parencite{Barro1971}. Aber bereits mit 30 Jahren, also 1974, veröffentlichte er den einflussreichen Artikel \textins{Are Government Bonds Net Wealth?} \parencite{Barro1974}. Das Werk schlägt in dieselbe Kerbe wie die Arbeit von \textcite{Sargent1975} und allgemein der Neuen Klassiker, nämlich dass Wirtschaftspolitik keinen positiven Effekt auf das Bruttoinlandsprodukt (Net wealth) hat. Diese Arbeit war damit - ebenso wie die bereits genannten von Arbeiten von Robert Lucas und Thomas Sargent - ein Bruch mit den damals vorherrschenden keynesianischen Ideen. Schließlich war es bis dahin praktisch unumstritten, dass "`expansive Fiskalpolitik"', im Barro-Artikel als Government Debt bezeichnet, über den Multiplikatoreffekt zu einer Steigerung der aggregierten Nachfrage und somit zu einem Wohlfahrtsgewinn im Form eines steigenden BIPs, führt. \textcite[S. 336]{Blinder1973} hatten im Jahr davor gegen argumentiert, dass die keynesianische Fiskalpolitik sehr wohl ökonomisch sinnvoll ist und daher "`[will] survive[s] the monetarist challenge"'. Barro wiederum führt "`neue klassische"' Argumente ins Feld wenn man so will: Fiskalpolitik sei deshalb wirkungslos, weil Haushalte generationsübergreifend agieren würden und sich damit bewusst wären, dass Staatsschulden, die heute aufgenommen werden, in Zukunft nur durch entsprechend höhere Steuereinnahmen zurückbezahlt werden können \textcite[S. 1116]{Barro1974}. Oder mit anderen Worten: Die Menschen sind sich bewusst, dass heutige Budgetdefizite in Zukunft durch höhere Steuereinnahmen kompensiert werden müssen und agieren dementsprechend mit höheren Sparquoten, was dazu führt, dass Fiskalpolitik zu keiner höheren aggregierten Nachfrage führt. 
Das Paper argumentiert also es komme bei Fiskalpolitik immer zu einem vollständigen Crowding-Out-Effect: Der positive Effekt auf das BIP-Wachstum durch die zusätzlichen Staatsausgaben wird durch den negativen Effekt ausgeglichen, der dadurch entsteht, dass die Menschen den privaten Konsum einschränken, weil sie für die zukünftige Steuerbelastung sparen. Der keynesianische Multiplikator wäre demnach nicht größer als Eins. Der Ansatz von Barro wurde später als \textit{Ricardianische Äquivalenz} (oder Ricardo-Barro-Äquivalenz) bekannt. David Ricardo hatte sich nämlich schon 1820 Gedanken darüber gemacht, dass es keinen Unterschied mache, ob ein Staat einen Krieg durch einen Kredit mit den entsprechenden Zinszahlungen oder eine Steuererhöhung in Höhe der Zinszahlungen finanziere. Das Prinzip wurde auf jeden Fall kontrovers aufgenommen und bis heute dementsprechend diskutiert. In einem Interview mit der \textit{Minneapolis Fed} beschreibt Barro, dass sein Theorem zwar "`bis heute nicht Teil der Mainstream-Ökonomie ist, weil die meisten Ökonomen das Konzept nicht vollständig als richtig akzeptieren. Das Konzept aber dennoch einen enormen Einfluss darauf hatte wie in der Ökonomie über Fiskalpolitik gedacht wird."' Beides ist zweifelsfrei richtig: Die Ricardianische Äquivalenz erlitt dasselbe Schicksal wie so viele Konzepte der Neuen Klassik. Formal einwandfrei dargestellt, scheitert das Konzept an empirischen Beobachtungen und an den zu starren theoretischen Voraussetzungen.
Dennoch ist das Konzept nach wie vor nicht von der Bildfläche verschwunden. Zwar gilt die Ricardianische Äquivalenz in seiner reinen Form als widerlegt, aber der keynesianische Optimismus gegenüber der positiven Effekte von Fiskalpolitik wurde nicht zuletzt durch die Arbeit von Barro in Zweifel gezogen. Als State-of-the-Art gilt heute, dass Fiskalpolitik in der kurzen Frist durchaus positive Wirkungen auf das BIP-Wachstum hat, in der langen Frist hingegen sogar negative Effekte.

Robert Barro wird meist mit der umstrittenen "`Ricardianischen Äquivalenz"' in Verbindung gebracht, manchmal auch mit der endogenen Wachstumstheorie (die später in diesem Kapitel beschrieben wird). Relativ selten aber mit jenem Thema, das er ebenfalls stark mitgeprägt hat und das große praktische Auswirkungen hatte: Das "`Inflation Targeting"' der Zentralbanken. Wir erinnern uns, dass die 1970er Jahre in den USA von hohen Inflationsraten geprägt wurden. Die Monetaristen hatten zu deren Bekämpfung zunächst eine Geldmengensteuerung und später eine konstante Wachstumsrate der Geldmenge vorgeschlagen. Beides erwies sich aber als nicht besonders zielführend. Danach folgte - wie soeben beschrieben - die große Zeit der "`Neuen Klassischen Makroökonomie"'. In vielen Punkten wurde diese ökonomische Schule recht umstritten aufgenommen und erwies sich in ihrer Reinform oft für die Wirtschaftspolitik als zu theoretisch. Man denke nur an die angeblich völlige Wirkungslosigkeit der Fiskalpolitik. Was hingegen die Geldpolitik angeht, wenden die meisten Zentralbanken heute jene Ideen an, deren theoretische Grundlagen erstmals von den Neuen Klassikern beschrieben wurden! 
Konkret wandelte sich die primäre Rolle der Zentralbank von der Bekämpfung der Arbeitslosigkeit im Keynesianismus zur Bekämpfung der Inflation im Monetarismus. Die Ausführungen von Milton Friedman und dessen Monetaristen waren aber wenig theoretisch hinterlegt. Erst die Neuen Klassiker entwarfen fundierte wissenschaftliche Arbeiten, die zumindest die Grundlage der heutigen Politik der Zentralbanken schuf, nämlich dass Geldwertstabilität deren primär anzustrebendes Ziel ist. Das hat sich seit Anfang der 1990er Jahre bis heute unbestritten tatsächlich als das primäre Ziel der Zentralbanken in den Industriestaaten etabliert. Zwar spielt die "`Neue Philipskurve"' - also die Bekämpfung von Arbeitslosigkeit durch Inflationserwartungen -  in der kurzen Frist eine Rolle, aber langfristig steht die Geldwertstabilität im Fokus der Zentralbanken. Bei manchen - wie der EZB - ausdrücklich festgeschrieben, bei anderen - wie der Fed - de facto ebenso wichtig, aber formal weniger stark festgelegt\footnote{"'I think over the last two decades the Fed has come close to an inflation targeting regime even though it's not explicit"', Robert Barro 2005 in einem Interview mit der Minneapolis Fed: https://www.minneapolisfed.org/article/2005/interview-with-robert-barro}.

Die Grundlagen dafür lieferten zunächst die bereits besprochenen \textcite{Kydland1977}. Aber es war vor allem Robert Barro (gemeinsam mit David Gordon), der Anfang der 1980er-Jahre \parencite{Barro1976, Barro1983a, Barro1983b}, folgendes Dilemma der Zentralbanken theoretisch löste: In der langen Frist kann die Geldpolitik das BIP und auch die Arbeitslosigkeit nicht entscheidend beeinflussen. Im Gegenteil, führen Bestrebungen dahingehend langfristig zu Kosten in Form von Inflation und damit Vertrauensverlust in die Währung. Langfristig ist Preisstabilität das Beste, dass eine Zentralbank anstreben kann. In der kurzen Frist jedoch können sehr wohl Wachstums- und Beschäftigungsakzente gesetzt werden, vor allem wenn die Zentralbank überraschend von ihrer angekündigten Politik abweicht. Nachdem die Inflationserwartungen der privaten Akteure gebildet sind, ändert sich also das optimale Verhalten der Bank. Wenig überraschend leidet aber die Glaubwürdigkeit ("`reputation"') einer Zentralbank, wenn sie immer wieder kurzfristig von ihren langfristig gesteckten Zielen abweicht. Mittels spieltheoretischen Ansatzes analysierten \textcite{Barro1983a} im sogenannten \textsc{Barro-Gordon-Modell} unter welchen Umständen der langfristige Nachteil (der durch das Abweichen von der angekündigten Politik entsteht) durch den kurzfristigen Vorteil der stimulierenden Wirkung von "`Überraschungsinflation"', überwiegt.

Es ist heute recht unumstritten, dass politisch unabhängige Notenbanken dieses langfristige Ziel der Geldwertstabilität am besten umsetzen können. Politiker, die sich alle paar Jahre einer Wiederwahl stellen müssen, wären wohl eher versucht die kurzfristig sinnvollen Abweichungen vom Inflationsziel vorzunehmen. Diese Fragen wurden ungefähr zehn Jahre später aus institutioneller Sicht \parencite{Tabellini1993} diskutiert. Die ersten formalen Inflationsziele wurden übrigens ebenfalls Anfang der 1990er-Jahre festgelegt. Unter anderem in Kanada und Neuseeland (1991), Großbritannien (1992), Schweden und Finnland (1993) \parencite{Fischer1994}. 

Die erfolgreiche Bekämpfung der Inflation kann auf jeden Fall als Erfolgsgeschichte gesehen werden. Der jüngeren Generation in den westlichen Staaten ist das Problem hoher Inflationsraten gar kein Begriff mehr. Aber zumindest bis in die 1980er-Jahre war dieses Problem ein allgegenwärtiges. Es war für die Menschen ärgerlich, dass das heute verdiente Geld in einem Jahr vier-oder-mehr-Prozent an Wert verlor. Natürlich waren es verschiedene Aspekte, die dazu führten, dass die Zentralbanken das Problem heute weit besser im Griff haben. Zum Einen die rigorose Anti-Inflationspolitik der Federal Reserve in den USA unter Paul Volcker. Weiters die Aufgabe der fixierten Wechselkurse. Damit müssen Staaten nicht länger ihre Geldpolitik nicht mehr länger von der Entwicklung der Referenzwährung abhängig machen. In welchem Ausmaß die soeben dargestellten wissenschaftlichen Erkenntnisse hierbei eine direkte Rolle spielten oder zumindest die Akteure beeinflusste, kann man wohl nicht quantifizieren. Meines Erachtens können diese aber als unumstrittener Erfolg der Neuen Klassiker gesehen werden.

Ab Mitte der 1990er-Jahre setzte sich auch in Bezug auf die Geldpolitik der "`Neu-Keynesianismus"' weitgehend durch, der im nächsten Kapitel beschrieben wird. Vor allem die Arbeiten von \textsc{John Taylor}\footnote{John Taylor ist zwar ein sehr konservativer Ökonom (unter anderem ist er im Jahr 2020 Präsident der wirtschaftsliberalen Mont-Pèlerin-Gesellschaft), seine bedeutendsten Arbeiten sind aber dem Neu-Keynesianismus zuzuordnen und stehen teilweise im Widerspruch zu den Arbeiten der Neu-Klassiker (\textcite{Taylor1977} als Antwort auf \textcite{Sargent1975} } erweiterten die Arbeiten von Barro und Gordon.

Vielleicht geht man zu weit, wenn man die heute in Notenbanken weitverbreitete Praxis des "`Inflation-Targetings"' alleine auf Arbeiten der "`Neuen Klassiker"' zurückführt. Selbst Robert Barro blieb diesbezüglich in einem Interview \parencite{Barro2005} eher zurückhaltend. Fix ist allerdings, dass die Erkenntnisse der "`Neuen Klassische Makroökonomie"' in Bezug auf die Geldpolitik weit weniger umstritten sind, als in den meisten anderen Bereichen. 


\section{Lucas und Romer: Endogenes Wachstumsmodell} \label{sec: endogene}
Die "`Endogene Wachstumstheorie"' ist eines jener Themen in diesem Buch, dass an der "`falschen"' Stelle platziert ist. Sie ist zwar eindeutig ein makroökonomisches Thema, aber nicht wirklich eng verbunden mit der "`Neuen Klassischen Makroökonomie"'. Allerdings ist auch die "`Endogene Wachstumstheorie"' ein klarer Bruch mit der damals vorherrschenden Mainstream-Theorie dem "`Solow-Wachstumsmodell"'. Wie im gleichnamigen Kapitel beschrieben, wird ökonomisches Wachstum durch technologischen Fortschritt erklärt, der allerdings als exogene Variable betrachtet wird.
Die eigentliche Verbindung zur "`Neuen Klassik"' besteht vor allem über die handelnden Akteure: Der Begründer der "`Neuen Klassischen Makroökonomie"' ist auch einer der beiden Entwickler der "`Endogenen Wachstumstheorie"': Nämlich Robert Lucas. Der zweite Entwickler ist Paul Romer. Wie die "`typischen"' Vertreter der Neuen Klassik hatte auch er als Absolvent des Physikstudiums \textcite{Romer2018} einen stark quantitativen Background. Außerdem war Robert Lucas einer der Betreuer seiner Dissertation, die er in Chicago abschloss. Eigentlich scheint es also als wäre er selbst ein "`typischer"' Vertreter der "`Neuen Klassik"'. Erst bei genauerem hinsehen entdeckt man, dass er nicht der kompromisslosen mathematischen Argumentation erlegen ist wie eben die "`typischen Neuen Klassiker"'. Im Gegenteil: Im Jahr 2015 publizierte er eine Kritik an dieser strengen Mathematik-Gläubigkeit. \textit{Mathiness} nennt Romer die - seiner Meinung nach - häufig missbräuchliche Verwendung von Mathematik in wirtschaftswissenschaftlichen Journalartikeln. Inhaltlich schlechte oder falsche Annahmen würden hierbei überdeckt durch mathematische und damit scheinbar neutrale Abhandlungen. So weit so gut. Hauptziel seiner bemerkenswert scharfen Kritik\parencite{Romer2015} waren aber ausgerechnet die "`Neuen Klassiker"' Edward Prescott und Robert Lucas\footnote{Es wurden im selben Artikel aber auch Joan Robinson und Thomas Piketty - also zwei sehr "`linke"' Ökonomen ebenso scharf kritisiert.}. Man muss allerdings tatsächlich festhalten, dass sich gerade die Wirtschaftsmodelle und Prognosen von Lucas und Prescott in ihrer Reinform - an der beide stur festhielten - recht rasch als unzulänglich erwiesen. Die Kritik erinnert ein wenig an das oben beschriebene Zitat von Robert Solow, wonach man mit Robert Lucas nicht über ökonomische Modelle und mit Napoleon Bonaparte nicht über Kavallerietaktik diskutieren soll, weil man sich dann auf deren Spezialgebiet begeben hat. Das Problem seien vielmehr die Grundlagen dahinter.

Woher kam plötzliche Wiedererstarken des Interesses an Wachstumstheorien Mitte der 1980er Jahre? Nachdem Solow bereits 1956 sein Modell veröffentlicht hatte, war das Thema jahrzehntelang von den Bildschirmen verschwunden. Aber mit der Stagflation kamen eben nicht nur die schon beschriebenen Zweifel an keynesianischer Wirtschaftspolitik, sondern - da Wirtschaftswachstum eben nun ausblieb - stellte man wieder vermehrt Fragen woher dieses denn eigentlich komme? 
Natürlich spielte aber auch der sich ändernde Zeitgeist eine Rolle: Wachstums wurde seit den 1970er Jahren erstmals auch mit kritischen Augen gesehen. Sowohl die Frage ob Wachstum, wie wir es in den Jahren nach dem Zweiten Weltkrieg in den Industriestaaten gesehen hatten, auf lange Frist überhaupt möglich wäre. Als auch die Frage, ob Wirtschaftswachstum uneingeschränkt als positiv zu bewerten sei. Im Jahr 1972 schlug diesbezüglich ein Bericht des "`Club of Rome"' hohe Wellen: In "`Die Grenzen des Wachstums"' wurde prognostiziert, dass das wirtschaftliche Wachstum aufgrund der Ausbeutung von Rohstoffen spätestens Mitte des 21. Jahrhunderts kollabieren würde. Von der Allgemeinheit zu tragende Schäden, die durch zügelloses Wirtschaftswachstum verursacht wurden traten immer häufiger in den Mittelpunkt. Zunächst wurde die gesundheitsgefährdende Wirkung des Pflanzenschutzmittels DDT, dass erfolgreich zu Steigerung landwirtschaftlicher Erträge eingesetzt wurde, offensichtlich. Später rückten Themen wie der Saure Regen der das Waldsterben auslöste, ab Mitte der 1970er Jahre das Ozonloch in den Fokus der Öffentlichkeit. Bis heute beschäftigt uns der $C0_2$-Ausstoß und die damit verbundene globale Klimaveränderung.
In seiner Nobelpreis-Rede erinnerte sich \textcite{Romer2018} an dieses Umfeld, das die frühen 1980er-Jahre prägte, zurück. Es herrschte eher ein Pessimismus dahingehend vor, ob nachhaltiges Wachstum möglich sei. Als Doktoraststudent war auch für ihn als Absolvent eines Physikstudiums zunächst nur eine Wachstumstheorie sinnvoll: Jene von Thomas Malthus (vgl. Kapitel XXX), die bekanntermaßen eher apokalyptisch ist\parencite{Romer1986}. Seine Forschungen zu endogenem Wachstum sollte ihn schließlich davon überzeugen, dass nachhaltiges Wachstum sehr wohl möglich sei.

Die primäre Motivation für das Thema war, dass man nicht länger akzeptieren wollte, dass wirtschaftliches Wachstum ausschließlich durch technologischen Fortschritt verursacht wird \textit{und}, dass die Entstehung dieses technologischen Fortschritts nichts mit ökonomischen Prozessen zu tun habe \footnote{Dieses Problem brannte manchen Ökonomen schon länger unter den Fingernägeln wie das Zitat aus \textcite{Arrow1962} zeigt: \textit{Nevertheless a view of economic growth that depends so heavily on an exogenous variable, let alone one so difficult to measure as the quantity of knowledge, is hardly intellectually satisfactory.}}. In der exogenen Wachstumstheorie geht man davon aus, dass das BIP grundsätzlich dazu neigt auf gleichem Niveau zu bleiben, das Wachstum also stagniere. Nur durch technologischen Fortschritt - auf den aber nicht näher eingegangen wird, weil er eben als exogen betrachtet wird - kommt es also zu Wirtschaftswachstum.

\textcite{Romer1994} meint es gäbe zwei verschiedene Versionen über die Anfänge der "`Endogenen Wachstumstheorie"'. Die erste Version geht zurück auf die sogenannte "`Konvergenz-Kontroverse"'. Anfang der 1980er-Jahre erschien der erste Datensatz, der langfristige Zeitreihen für wichtige makroökonomische Kennzahlen für eine verschiedene Staaten zur Verfügung stellte \parencite{Maddison1982}. Darauf aufbauend kam es zu einer Debatte, ob es ärmeren Staaten im 20. Jahrhundert gelungen sei zum Wohlstand wohlhabender Staaten aufzuschließen. Man kam rasch zum Ergebnis, dass dies zumindest den meisten Staaten nicht gelang und die Entwicklung auch nicht darauf schließen lasse, dass dies in den nächsten Jahren gelingen würde. Robert Lucas und Paul Romer stellten fest, dass dies allerdings laut exogener Wachstumstheorie passieren hätte sollen: Technologischer Fortschritt ist in solchen Modellen nämlich eben exogen und sollte dementsprechend in armen wie in reichen Staaten gleichermaßen vorkommen. Außerdem gehen exogene Wachstumsmodelle davon aus, dass moderne Technologie auf der ganzen Welt angewendet werden kann. Warum aber sollte es laut "`exogenen Wachstumstheorien"' zu einer Konvergenz, also einem stärkeren Wirtschaftswachstum in armen Ländern als in reichen Ländern, kommen? Dazu müssen wir direkt an die Überlegungen aus Kapitel \ref{sec: Solow-Modell} anschließen. Dort wurde behauptet, dass sich langfristig Wirtschaftswachstum verlangsamen müsste, weil der Produktionsfaktor Arbeit stabil ist und eine ständige Erhöhung des Faktors Kapital zu immer geringeren Zuwachsraten beim Wachstum führt. Nur wenn sich die \textit{Qualität} des Faktors Kapital ständig erhöht, also technischer Fortschritt eintritt, kann dauerhaftes Wachstum entstehen. Und hier kommt die eigentlich notwendige Konvergenz ins Spiel: In einem armen Land ist per Definition das BIP pro Kopf niedriger als in einem reichen Land. Oder mit anderen Worten, da wir den Output ja "`pro Kopf"' betrachten, also um den Faktor Arbeit bereinigen, is der Kapitaleinsatz pro Kopf in einem armen Land geringer als in einem reichen Land. Nun haben wir im Kapitel \ref{sec: Solow-Modell} festgestellt, dass eine zusätzliche Input-Einheit den Gesamtoutput dann am stärksten erhöht, wenn diese Art des Inputs insgesamt noch unterrepräsentiert ist. Wir haben dies den "`Abnehmenden Grenzertrag"' genannt. Erinnern Sie sich an das Beispiel zurück, in dem der erste Computer einen höheren Produktivitätszuwachs bringt, als der x-te Computer. Wenn Technologie und technologischer Fortschritt nun auf der ganzen Welt, also in armen Ländern wie in reichen Ländern, in gleichem Ausmaß zur Verfügung stünde, wie die exogene Wachstumstheorie postuliert, dann wäre es doch viel sinnvoller zusätzliches Kapital in armen Ländern einzusetzen, als in reichen Ländern. In armen Ländern müsste dieses zusätzliche Kapital nämlich in viel höherem Ausmaß die Produktivität steigern, zu Wirtschaftswachstum führen und das BIP in armen Länder langsam auf das Niveau der reichen Ländern anwachsen lassen. Das BIP der beiden Staaten müsste eben konvergieren. Dies passte aber eben nicht mit den empirischen Beobachtungen zusammen. 
Beide, Romer und Lucas, versuchten sich in weiterer Folge darin, Modelle zu erstellen, welche die Entwicklung von technologischen Fortschritt nicht länger als Gott-gegeben annehmen, sondern in Wachstumsmodellen mit-erklären sollten. Technologischer Fortschritt ist also auch in den neuen Wachstumstheorien der wesentlicher Einflussfaktor auf Wirtschaftswachstum, allerdings wird dieser endogenisiert.

Zwei wesentliche Beobachtungen müssen festgehalten werden um die Richtigkeit und Wichtigkeit der Endogenisierung des technologischen Fortschritts zu begründen \parencite{Romer1994}. Erstens, technologischer Fortschritt ist kein Zufallsprozess, sondern das Ergebnis harter Arbeit von Menschen. Die Erfolgsaussichten der Forschungsarbeit hängen nicht unwesentlich mit den Umständen zusammen unter denen die Forscher arbeiten. Paul Romer machte dies mit einem einzigen Bild, dass er im Rahmen seiner Nobelpreis-Rede zeigte klar: Afrikanische Studenten sitzen auf diesem Bild unter dem Licht von Straßenlaternen um zu lernen. Offensichtlich war diese Lichtquelle die einzige, die sie nutzen konnten. Es ist klar, dass europäische oder nordamerikanische Studenten und Forscher einen Vorteil gegenüber diesen Studierenden haben, wenn es darum geht technologischen Fortschritt hervorzubringen. Zweitens, technologischer Fortschritt ist kein öffentliches Gut und damit eben nicht überall auf der Erde gleichermaßen verfügbar. In der Ökonomie spricht man von öffentlichem Gut, wenn ein Gut keine Rivalität und auch keine Ausschließbarkeit im Gebrauch aufweist. Ersteres ist bei wissenschaftlichen Erkenntnissen gegeben, weil es wissenschaftlichen Erkenntnissen nicht schadet, wenn sie von mehreren Personen gleichzeitig angewendet werden. Zweiteres ist aber nicht gegeben. Durch Geheimhaltung aber auch Patent- und Musterschutzrechte können Personen sehr wohl davon ausgeschlossen werden, neueste wissenschaftliche Erkenntnisse zu nutzen. In diesem Fall spricht man von Klubgütern anstatt von öffentlichen Gütern.
Diese zweite Beobachtung verändert die Anforderung an Wachstumsmodelle erheblich. Wenn technologischer Fortschritt für alle verfügbar ist, so haben alle Marktteilnehmer die gleichen Startvoraussetzungen. Wie auf einem Markt auf dem perfekter Wettbewerb herrscht. Die Ausschließbarkeit von Forschungsergebnissen führt aber dazu, dass Unternehmen, die über Forschungsergebnisse verfügen, dies auch verwerten können. Sie sind also Monopolanbieter für Güter, die aufgrund dieser Forschungsergebnisse produziert werden. 

Mitte der 1980er Jahre befand sich die Wachstumstheorie also in einem Dilemma: Nach dem etablierten Konzept abnehmender Grenzerträge, sollte Wirtschaftswachstum eigentlich durch Ressourcenknappheit beschränkt sein. Die exogene Wachstumstheorie, wonach stetiges Wachstum eben doch möglich sei, wenn man die Existenz von technologischem Fortschritt akzeptiere, dessen Entstehung aber nicht weiter nachverfolge, war zunehmend unbefriedigend. Dazu kam das Problem der fehlenden - aber eigentlich zu erwartenden - Konvergenz zwischen armen und reichen Ländern. 

Schritt für Schritt wurden in den kommenden Jahren die angesprochenen Unzulänglichkeiten berücksichtigt:

Zunächst wurden Modelle erstellt, die nicht mehr unbedingt von abnehmenden Grenzerträgen ausgingen. Diese basierten auf dem Vorläufer-Modell von \textcite{Arrow1962}, der die privaten Investitionen in Forschung und Entwicklung in den Vordergrund stellte. Dies Modelle werden heute häufig unter "`AK-Modelle"'-Modelle zusammengefasst. Vor allem die Arbeiten von \textcite{Romer1986} und \textcite{Rebelo1991} gehen auf diesen Ansatz zurück. Technisch gesehen bedienen sich diese Modelle eines kleinen Tricks: Die bereits in Kapitel \ref{sec: Cobb-Douglas-Produktionsfunktion} kennengelernte Cobb-Douglas-Funktion wird hier herangezogen - wie auch zum Beispiel beim exogenen Wachstumsmodell. Ohne auf die Details einzugehen wird hier ein Parameter so gesetzt, dass zunehmender Kapitaleinsatz nicht mehr zu abnehmenden Grenzerträgen führt. Inhaltlich wird hierbei argumentiert\parencite{Romer1994}, dass Forschungsausgaben von privaten Unternehmen den Inputfaktor Kapital so stark wachsen lassen, dass es dauerhaft positive Grenzerträge gibt, und dass über Spillover-Effekte diese privaten Forschungsausgaben auch gesamtwirtschaftlich zu technischem Fortschritt führen. Man konnte damit zumindest innerhalb des Modells erklären, warum dauerhaftes Wachstum möglich war, allerdings zeigte sich \textcite[S. 15]{Romer1994} selbst rasch nicht sehr glücklich mit der Rohform dieser Modelle.

Es dauerte auch nicht lange bis \textcite{Lucas1988} ein alternatives Modell vorstellte. Durch \textcite{Romer1986} konnte man zwar erklären wie es zu dauerhaftem Wachstum kommen kann, allerdings war die Frage ungeklärt, warum unterschiedliche Wachstumsraten in verschiedenen Entwicklungsländern auftraten. Also warum es nicht notwendigerweise zur oben beschriebenen Konvergenz kommt. Ein häufig wiedergegebenes Zitat aus \textcite[S. 5]{Lucas1988} bringt die Fragestellung auf den Punkt: \textit{"'Is there some action a government of India could take that would lead the Indian economy to grow like Indonesia's or Egypt's? If so, what, exactly? If not, what is it about the' nature of India' that makes it	so?"'}\footnote{Während in  und Indonesien Anfang der 1990er Jahre hohe Wachstumsraten zu verzeichnen waren, war Indien noch geprägt von geringem Pro-Kopf-Einkommen und niedrigen Wachstumsraten. Eine Entwicklung, die sich gerade wenn man Ägypten und Indien vergleicht nur wenige Jahre später umdrehen sollte.} Genau diese Fragestellung adressierte Lucas in weiterer Folge. Wobei das Paper selbst einigermaßen bemerkenswert ist: Es ist einer der am häufigsten zitierten Journal-Artikel\footnote{Zum Beispiel hier auf Platz 7: https://ideas.repec.org/top/top.item.nbcites.html, Stand 11.01.2020} in den Wirtschaftswissenschaften überhaupt. Was insofern überrascht als Robert Lucas im Kernbereich der "`Neuen Klassischen Makroökonomie"' vermeintlich wichtigere Beiträge geleistet hat. Robert Lucas selbst meint außerdem in den "`Acknowledgements"' \parencite[S. 41]{Lucas1988} des Artikels, dass er zwar ein berühmter Ökonom sei, aber von diesem Thema wenig verstehe und der Artikel außerdem eigentlich zu lang für einen Journal-Beitrag wäre.
Wie auch immer: Aus modelltechnischer Sicht war die Arbeit an \parencite{Uzawa1965} und auch \parencite{Romer1986}. Allerdings entwickelte Lucas in diesem Modell den Inputfaktor Kapital weiter, indem dieser nicht mehr ausschließlich aus physischem Kapital bestand, sondern er implementierte darin außerdem das von Gary S. Becker entwickelte Konzept des Humankapitals\footnote{Der Begriff Humankapital wird im nächsten Kapitel: \ref{sec: Becker} behandelt.}.
Die Erweiterung gegenüber der Arbeit von \textcite{Romer1986} besteht nun darin, dass die Forschungsbemühungen von privaten Unternehmen \textit{explizit betrachtet} werden. Konkret fließt ein Teil des Humankapitals natürlich in die Produktion der Güter, aber der zweite Teil wird dazu verwendet dieses individuelle Humankapital auszubauen, also Bildung und Ausbildung zu schaffen. Das heißt aber auch, dass man in der kurzen Frist auf Produktion verzichten muss um stattdessen mit den freigewordenen Zeit-Ressourcen Wissen (Ausbildung) aufzubauen. Erst langfristig führt dieses Wissen dazu, dass besser ausgebildetes Personal effizientere Arbeit leistet und so für ständig positive Grenzerträge sorgt. Wie auch in \textcite{Romer1986} kommt es also auch gesamtwirtschaftlich zu technischem Fortschritt, allerdings nicht mehr als Nebeneffekt nicht weiter spezifizierter Forschungsausgaben, sondern durch Investition in den Modellparameter Humankapital.
Durch unterschiedliche Bildungsniveaus in verschiedenen Regionen der Erde lassen sich die beobachteten Wachstumsraten also jetzt erklären. Denn die Investition in physisches Kapitel alleine führt nicht zu langfristigen Wachstum, wenn die Ausstattung mit Humankapital nicht gegeben ist. Einfach ausgedrückt: Eine mit moderner Technologie ausgestattete Fabrik alleine wird keine Erträge bringen, wenn die Arbeiter und Angestellten die dort arbeiten sollen nie die Möglichkeit hatten lesen und schreiben zu lernen.

Die beiden nun vorgestellten Modelle nennt man häufig auch "`Modelle mit konstanter Technologie"'. Dies ist etwas verwirrend, da wir ja ständig davon gesprochen haben, dass eben technologischer Fortschritt Wirtschaftswachstum verursacht. Allerdings beschränken sich sowohl \textcite{Lucas1988} als auch \textcite{Romer1986} darauf sich entweder auf Fortschritt beim Humankapital (Bildungsniveau) festzulegen, bzw. Fortschritt unspezifisch zu betrachten. Der fehlende Schritt besteht nun darin technischen Fortschritt im Sinne von neuen oder besseren Produkten und Produktionsverfahren zu integrieren. Dies ist deshalb wichtig, weil dieses Wissen durch Patente geschützt werden kann. Im Gegensatz zu "`Bildungsstand"' allgemein, also kein öffentliches Gut ist.

In \textcite{Romer1990} wurde technologischer Fortschritt als Produkt- bzw. Verfahrensweiterentwicklungen und mit Schutzmechanismen wie Patentregelungen und Urheberrechten modelliert. Damit erlangt man durch technischen Fortschritt eine Monopolstellung, da Produkte, die auf technologischem Fortschritt basieren, nur jenes Unternehmen anbieten kann, welches diese entwickelt hat. Das heißt aber auch, dass Modelle, die auf dem Prinzip der vollkommener Konkurrenz basieren durch Modelle mit monopolistischer Konkurrenz ersetzt werden\footnote{\textcite[S. 17]{Romer1994} nannte diese Modelle "`Neo-Schumpeter-Wachstumsmodelle. Schließlich spielte auch bei Schumpeter die Monopolstellung des Entrepreneurs eine wichtige Rolle bei der Entstehung von Wirtschaftswachstum. Der Begriff "`Neo-Schumpeter"' wird aber meines Erachtens nicht konsistent für diese Modelle verwendet}. In der historischen Entwicklung der "`Endogenen Wachstumstheorie"' ist dies ein interessanter Punkt. Denn diese Monopolmodelle wurden in anderen Bereichen vor allem von Neu-Keynesianern (vergleiche dazu Kapitel \ref{cha: Neu Keynes}) verwendet. In der "`Endogenen Wachstumstheorie"' forschten bislang vor allem Vertreter der "`Neuen Klassischen Makroökonomie"': Robert Lucas, den hier nicht speziell erwähnten Robert Barro und eben auch Paul Romer. Mit der Implementierung monopolistischer Modelle wendete sich Paul Romer nun aber ab von seinem Doktorvater Robert Lucas ab, der bis ausschließlich "`vollkommene Marktmodelle"' für richtig hält. In der bereits beschriebenen Mathiness-Debatte wird der Bruch offensichtlich: \textcite[S. 91f]{Romer2015} kritisiert Lucas mit scharfen Worten.

Zum Inhalt dieser Modelle: Erneut werden die Produktionsfaktoren Arbeit, Kapital und Humankapital als Inputs herangezogen. Weiters wird unterschieden zwischen dem "`allgemeinen Wissenstand"', und dem "`technischen Fortschritt"'. Letztgenannter ist durch Patente oder Urheberrechte geschützt. Dementsprechend kann dieser technische Fortschritt zur Generierung von Monopolgewinnen genutzt werden. Das formale Modell besteht schließlich aus drei Sektoren \parencite[S. 79]{Romer1990}
\begin{itemize}
	\item Der Forschungssektor produziert aus dem "`allgemeinen Wissenstand"' und Humankapital "`technischen Fortschritt"'. Die Ergebnisse aus dem Forschungssektor können nur zum Teil durch Patente gesichert werden, der andere Teil wird frei verfügbar. Dies kann man sich anhand eines Beispiels erklären: Bahnbrechende Entwicklungen, wie zum Beispiel der Mikrochip, bringen dem Erfinder hohe Erträge. Der gesamtwirtschaftliche Gewinn ist allerdings noch wesentlich größer. Ökonomen sprechen in diesem Fall von "`positiven externen Effekten"'. Man weiß, dass alle Tätigkeiten, die solche Effekte verursachen im privatwirtschaftlichen Sektor tendenziell unter-finanziert werden, da es eben keine volle Abgeltung der Gewinne daraus gibt.
	\item Der Zwischensektor erzeugt aus diesem "`technischen Fortschritt"', Kapital und Humankapital neue Produktionsverfahren. Da im Zwischensektor die Patente aus dem Forschungssektor verwendet werden, handelt es sich hierbei um den vorhin schon angesprochenen Monopol-Markt. Typischerweise treten hier negative externe Effekte auf: Der Monopolist wird das neue Produktionsverfahren in einem Ausmaß anwenden, der ihm erlaubt eine Monopolrente abzuschöpfen. Die Aussicht auf diese Monopolrente ist aber auch die treibende Kraft für Unternehmen innovativ tätig zu werden. Das langfristige endogene Wachstum wird also in diesem Sektor geschaffen.	
	\item Im dritten Sektor werden diese Verfahren schließlich eingesetzt um gemeinsam mit Humankapital und Kapital die Konsumgüter herzustellen. Von denen - wie schon aus den anderen Modellen bekannt - ein Teil konsumiert wird und ein Teil investiert wird.
\end{itemize} 
Als Conclusio kann aus diesem Paper gezogen werden, dass damit der dritte vorhin angesprochene Diskussionspunkt, nämlich monoplistische Konkurrenzmärkte, in die Wachstumstheorie eingebunden wurden. Die Implikationen aus dem Modell sind, dass steigendes Humankapital für stetiges Wachstum verantwortlich ist. Wie im Modell von \textcite{Lucas1988} lässt sich auch hier damit die fehlende Konvergenz zwischen armen und reichen Ländern, aber auch die anhaltenden hohen Wachstumsraten in Industriestaaten im 20. Jahrhundert erklären \parencite{Romer1990}. Zusätzlich kann aus dem Paper der Schluss gezogen werden, dass der Markt dazu tendiert tendenziell wenig Anreiz für Grundlagenforschung zu bieten. Staatliche Anreize könnten hier sinnvoll sein. Weiters kann man aus dem Modell ableiten, dass eine höhere Bevölkerungszahl - aufgrund im Modell technisch berücksichtigter positiver Skaleneffekte - mit höheren Wachstumszahlen einhergeht. Diese Implikation stieß früh auf Kritik. Da dies sowohl ökonomisch-inhaltlich als auch empirisch eine schwer haltbare Annahme ist.

Dieses Problem wurde durch \textcite{Jones1995} aufgegriffen und das Modell wurde technisch so verändert, dass das Problem eliminiert wurde. Insgesamt ist das Thema Wachstumstheorien eines an dem sehr lange intensiv geforscht wurde und teilweise noch immer geforscht wird. Wenn sich auch der Fokus seit der "`Great Recession"' ab dem Jahr 2007 etwas verschoben hat.

Vor allem aufbauend auf das Paper von \textcite{Romer1990} kam es bis Mitte der 1990er Jahre zu einigen bekannten Erweiterungen des endogenen Wachstumsmodells. Aufgrund des hohen Detailgrades der Diskussion sollen die wesentlichen Paper hier nur angeführt werden: \textcite{Lucas1990} brachte in einem Artikel noch einem die Diskussion ein warum nicht mehr Kapital in Entwicklungsländer fließt, da dieses dort ja eigentlich eine höheren Produktivitätszuwachs pro zusätzlich investierter Geldeinheit als in Industriestaaten haben sollte. Diese Diskussion wurde als \textit{Lucas-Paradoxon} bekannt.

Die frühen Vertreter des (späten) Neu-Keynesianismus \textcite{Mankiw1992} argumentierten mittels empirischer Untersuchung, dass das Solow-Modell sehr wohl geeignet sei die beobachteten Wachstumsraten zu erklären, wenn man um Effekte des Bevölkerungswachstums und die Effekte der Entwicklung des Inputfaktors Kapital bereinigt. Dieser Journalartikel stellte eine vielzitierte Gegenposition zum endogenen Wachstumsmodell dar. 

\textcite{Grossman1991a, Grossman1991b} legen den Fokus auf internationale Verflechtungen und analysieren die Auswirkungen von Außenhandel auf das Romer-Wachstumsmodell.

\textcite{Aghion1992} werfen den Blick - ebenfalls als Erweiterung zum Romer-Modell - noch genauer auf den Wachstumsprozess. Konkret in Anlehnung an Schumpeter auf den Prozess der "`kreativen Zerstörung"'. Welche Rolle spielt bei Entscheidungen ob in Forschung investiert werden soll die Gefahr, dass Vorsprung durch Forschung rasch wieder konkurrierende Forschung zunichte gemacht werden könnte. Die Forschung wurde von den beiden lange und erfolgreich weitergeführt und entsprechend vertieft \parencite{Aghion2005}.

Nicht vergessen werden sollte an dieser Stelle ein interessanter, aufstrebender Ansatz zu langfristigen Wachstumsmodellen: Daron Acemoglu führt langfristiges Wachstum auf die Existenz und Qualität von Institutionen zurück. Diesem Thema widmen wir uns etwas später im Rahmen des Kapitels \ref{sec: Neue Inst}. Zuvor gehen wir auf ein Thema ein, dass wir im aktuellen Kapitel unreflektiert eingeführt und ausgeschlachtet haben: Was ist überhaupt Humankapital?

\section{Becker: Rational Choice Theory}  \label{sec: Becker}

In der Mikroökonomie spielte rationales Entscheidungsverhalten schon beim Übergang von \textit{Klassik} zu \textit{Neoklassik} eine Rolle. Die "`Rational Choice Theory"' ist hierbei ein Überbegriff, der in der Humankapitaltheorie einen Höhepunkt erreicht hat, weil hier weitere Bereiche des menschlichen Zusammenlebens formal analysiert werden. Im Jahr 1947 wurde der \textit{Homo Oeconomicus} - also der rational handelnde Mensch - (vgl. Kapitel \ref{cha: Spieltheorie}) durch Von Neumann und Morgenstern in der Erwartungsnutzen-Theorie formalisiert \parencite{VonNeumann1944}. Wenig später allerdings kam es schon zu erheblichen Zweifeln an diesem Prinzip durch das \textit{Allais-Paradoxon}. Das rationale Verhalten in der Ökonomie ist seit jeher ein umstrittener Begriff, der die Wirtschaftswissenschaften geradezu spaltete. Die Kontroversen spielen sich auf verschiedenen Ebenen ab. Manche lehnen Rationalverhalten als menschliches Verhalten komplett ab (vgl. Behavioral Economics), wiederum andere unterscheiden akzeptieren das Rationalverhalten wenn es um rein wirtschaftliche Angelegenheiten geht, während andere das Rationalverhalten als Nutzenmaximierung auf alle Lebensbereiche ausdehnen. Am weitesten gingen auch hier die Vertreter der Chicago-School: Um den Beginn der 1960er Jahren stand der rational handelnde Mensch im Zentrum der neuen Humankapitaltheorie. Vorarbeiten leistete \textit{Theodore Schultz}, die Hauptprotagonisten aber waren \textit{Jacob Mincer} und vor allem \textit{Gary Stanley Becker}. Im der Humankapitaltheorie wurde wirklich alles als ökonomisches Problem betrachtet. Nicht nur das, ganz im Stile der "`Neuen Klassiker"' wurde auch alles formalisiert und entsprechend mit mathematischen Modellen erklärt.
In \textcite[Ausgabe von 1991: S. 108]{Becker1981} lautet der Titel des vierten Kapitels: "`Bildung von zueinander passenden Paaren auf dem Heiratsmarkt"'. Darin wimmelt es von mathematischen Formeln und es heißt unter anderem: "`In diesem Kapitel wird gezeigt, dass auf einem effizienten Heiratsmarkt in der Regel eine positive assortative Paarung vorliegt, bei der Männer mit hoher Qualität mit Frauen mit hoher Qualität und Männer mit niedriger Qualität mit Frauen mit niedriger Qualität zusammengebracht werden[...]."'
Man kann sich wahrscheinlich vorstellen, dass dieser Zugang innerhalb der Wirtschaftswissenschaften auf Widerstand stieß. In den Sozialwissenschaften - auf die Gary Becker seine ökonomische Theorie schließlich unweigerlich ausweitete - wurde er lange Zeit überhaupt ignoriert. "`Die meisten Ökonomen dachten nicht, dass [meine Arbeit] Ökonomie sind, Soziologen und Psychologen haben allgemein nicht akzeptiert, dass ich zu ihren Forschungsfeldern beitrage"' beschreibt \textcite{Becker1992}. Natürlich wirkt das Thema provokativ. Schon der Titel "`Humankapital"' deutet darauf hin, dass rein menschliche Eigenschaften kommerzialisiert wird, was natürlich umstritten ist.

Die Humankapitaltheorie ging in Chicago aus der Arbeitsmarktökonomie hervor. Wobei hier zunächst vor allem  \textcite{Schultz1961} als Pionier \parencite{Becker1992} tätig waren. \textcite{Becker1957, Becker1962} dehnte die Theorie aber schließlich auf das gesamte menschliche Verhalten aus. Seine frühen Arbeiten entstanden zeitlich eher parallel zu jenen der zweiten Generation der Chicago-School, zu der vor allem die Monetaristen um Milton Friedman gezählt werden. Dennoch wird die Humankapitaltheorie eher der dritten Generation der Chicago School zugeordnet. "`Gary [Becker] ist der Nachfolger von Milton Friedman in Mikroökonomie [...], Robert Lucas ist der Nachfolger in Makroökonomie"', brachte es der Ökonom Donald McCloskey in einem Interview über Gary Becker auf den Punkte \parencite[S. 137]{Warsh}. Tatsächlich wird die Humankapitaltheorie häufig mit der "`Neuen Klassik"' in Verbindung gebracht (weshalb sie auch in diesem Kapital platziert ist). Natürlich ist die "`Neue Klassik"' Teil der Makro-, die Humankapitaltheorie hingegen Teil der Mikroökonomie. Aber der streng mathematisch-analytische Zugang in Verbindung mit der strengen Marktorientierung zeigt deutlich die verbindenden Ähnlichkeiten beider Schulen.

Gary Becker verließ drei Jahre nach seiner Promotion die University of Chicago in Richtung Columbia University und National Bureau of Economic Research (NBER). Dort publizierte er sein Hauptwerk "`Human Capital"' \parencite{Becker1964} und arbeitete sehr produktiv mit Jacob Mincer zusammen. Hier entstanden jene Werke, die im Nachhinein den nachhaltigsten Einfluss auf die Wirtschaftswissenschaften hatten, weil sie später den Ausgangspunkt der neuen Wachstumstheorie darstellte, wie im letzten Kapitel \ref{sec: endogene} angeführt. Gemeinsam mit \textcite{Schultz1963} und \textcite{Mincer1974} lieferte \textcite{Becker1962} die theoretischen Grundlagen für die Auswirkungen von Bildung auf die wirtschaftliche Entwicklung. Schultz war hierbei der erste, der den Zusammenhang zwischen Investitionen in Bildung und Wirtschaftswachstum herstellte. Becker unterschied zwischen spezifische Ausbildung und Bildung allgemein. Mincer lieferte empirische Ergebnisse zum Thema.
Lange Zeit litt die Anerkennung der Arbeiten an der Kontroverse über die Inhalte. Schon der Begriff "`Humankapital"' wurde kritisiert, weil er den Menschen als Maschine darstellt und auf die Kennzahl Produktivität reduziert. Auch der Ansatz Bildung als "`Investition"' statt als "`kulturelle Erfahrung"' zu sehen, war neu und stieß auf breiten Widerstand. Gary Becker überlegte eigenen Aussagen zufolge lange, ob er sein Buch wirklich "`Human Capital"' nennen sollte, da der Begriff eben umstritten ist.

Erst nach seiner Rückkehr nach Chicago im Jahr 1970 erweiterte Becker seine Arbeit zunehmend auf weitere Bereiche des menschlichen Lebens, wie Familiengröße, Scheidung, Kinder und Altruistisches Verhalten. Diese Arbeiten resultierten schließlich in das bereits zitierte Buch \textcite{Becker1981}.

Die Forschung zu Humankapital hatte enorme Auswirkung auf die ökonomische Entwicklung. Zum einen im bereits genannten Forschungsfeld der "`endogenen Wachstumstheorie"', aber auch der riesige Bereich der mikroökonometrischen, empirischen Forschung, der um die Jahrtausendwende aufkam, behandelt wie selbstverständlich Themen, deren Etablierung in der Ökonomie auf die Arbeiten von Gary Becker zurückgehen.
Die Kontroversen darüber ob "`menschliches Verhalten"' Gegenstand wirtschaftswissenschaftlicher Forschung sein kann und darf, bestehen weiterhin. Allerdings sind in wirtschaftswissenschaftlichen Kreisen die Ressentiments dagegen schon wesentlich schwächer geworden. Längst wird die Forschung dazu nicht mehr ausschließlich von markt-gläubigen Ökonomen durchgeführt. Mit dem Nobelpreis 1992 erfuhr Gary Becker in gewisser Weise die Absolution, dass seine Forschung bedenkenlos und wertvoll ist.





\section{Wirkung und Bedeutung der Neuen Klassischen Makroökonomie}

Wurde die "`Neue Klassik"' zum neuen Mainstream? In den 1980er Jahren sah es vielleicht danach aus. Dennoch muss man dies entschieden ablehnen! Dazu war die "`Neue Klassik"' zu starr marktgläubig. Die Annahme, dass sich alle Märkte immer im Gleichgewicht befinden ist einfach nicht aufrechtzuerhalten. Ihre Vertreter waren (und sind bis heute) zu stur um die Modelle zugunsten empirisch sinnvollerer Annahmen abzuändern.
Wurde die "`Neue Klassik"' dann überwunden? Auch das muss man entschieden ablehnen! Dazu waren ihre Errungenschaften zu bahnbrechend und zu erfolgreich. Ihre Ideen wurden gerne aufgenommen von den "`Neu-Keynesianern"' (siehe nächstes Kapitel \ref{cha: Neu Keynes}). Diese lehnten zwar die Vertreter der "`Neuen Klassik"' strikt ab, nicht aber deren erfolgreichen Methoden. Ohne die starren Annahmen der "`Neuen Klassiker"' implementierten sie deren Ideen in eigene Modelle.
Die neue Mainstream-Ökonomie steht also nicht im Widerspruch zur Neuen Klassik. Vielmehr entstand aus den konkurrierenden Schulen der 1970er- und 1980er-Jahren - den "`Neuen Klassikern"' und den wenig miteinander verbundenen "`Neu-Keynesianern"' - Anfang der 1990er Jahre eine neue gemeinsame Mainstream-Ökonomie, eine "`Neue Synthese"'. (siehe Kapitel \ref{Neue Neoklassische Synthese})

Was ist von der "`Neuen Klassik"' geblieben? Geradezu revolutioniert wurde die Ökonomie durch die formalisierte Herangehensweise der "`Neuen Klassiker"' an ökonomische Fragestellungen. Heutige Publikationen enthalten zum größten Teil hochformalisierte Modelle. Zwar war die Mathematik auch schon bei den Keynesianern - weniger bei Friedman's Monetarismus - ein zentrales Element, aber die durchgehende Formalisierung einer Fragestellung in einen formalen Rahmen und unter Nebenbedingungen war neu und setzte sich im bis heute Mainstream durch.

Geblieben ist auch die Mikrofundierung der Makroökonomie. Diese ist verbunden mit der eben genannten Formalisierung der Ökonomie. Auch die Mikrofundierung der Makroökonomie war keine "Erfindung" der "Neuen Klassiker", sondern geht ursprünglich auf Edmund Phelps zurück. Aber durchgängig angewendet wurde das Konzept erstmals von Lucas, Sargent und Co.

Die Annahme der "`rationalen Erwartungen"' ist zwar umstritten, aber hat es dennoch in die ökonomischen Mainstream-Modelle geschafft. Unumstritten spielen rationale Erwartungen eine Rolle zum Beispiel in der Geldpolitik. Die Zentralbanken verfolgen mittlerweile nicht mehr primär ein Geldmengenziel, sondern ein Inflationsziel, außerdem arbeiten sie auch konkret mit Inflations\textit{erwartungen}.

Zumindest einen wesentlichen Einfluss auf die moderne, wissenschaftliche Ausrichtung der Zentralbanken hatten die Autoren, die mit der Neuen Klassik in Verbindung gebracht werden. Es würde zwar zu weit gehen, wenn man diese Entwicklung alleine der "`Neuen Klassischen Makroökonomie"' zuschreiben würde, aber vor allem die Arbeiten von Robert Barro können zumindest in gewissem Maße als theoretische Begründung dafür gesehen werden, dass die meisten modernen Notenbanken heute ein "`Inflation-Targeting"' betreiben und stärker als früher auch als tatsächlich politisch unabhängige Zentralbanken geführt werden.

Was hat sich als falsch erwiesen? Der größte Fehler war das Festhalten am perfekten Funktionieren der Märkte auch in der kurzen Frist. Dass es keine unfreiwillige Arbeitslosigkeit gäbe war rasch empirisch nicht zu halten und modelltheoretisch nicht notwendig wie spätere Modelle von "`Neu-Keynesianern"' zeigten. Außerdem entstanden rasch "`Neu-Keynesianische"' Modelle, die bewiesen, dass die Theorie der Rationalen Erwartung auch dann aufrecht erhalten werden kann, wenn sich Löhne und Preise nicht \textit{sofort} an das allgemeine Gleichgewicht anpassen, sondern langsam über mehrere "`Entscheidungsrunden"'. Damit fiel auch die Annahme, dass Geldpolitik und Fiskalpolitik komplett unwirksam seien. Die aktuelle Mainstream-Ökonomie geht davon aus, dass zumindest in der kurzen Frist beide wirksam sind.

		% Neue Klassische Makroökonomie			!!! KORREKTURLESEN
%%%%%%%%%%%%%%%%%%%%% chapter.tex %%%%%%%%%%%%%%%%%%%%%%%%%%%%%%%%%
%
% sample chapter
%
% Use this file as a template for your own input.
%
%%%%%%%%%%%%%%%%%%%%%%%% Springer-Verlag %%%%%%%%%%%%%%%%%%%%%%%%%%

\chapter{Neu-Keynesianismus} \label{cha: Neu Keynes}

Der Neu-Keynesianismus ist leider (noch) wesentlich schwieriger von anderen Schulen abzugrenzen als etwa der Keynesianismus oder der Monetarismus. Dies gilt sowohl in inhaltlicher Sicht, also auch in zeitlicher Sicht. Inhaltlich lässt sich der Neu-Keynesianismus am ehesten negativ abgrenzen. Einige Ökonomen erkannten, dass die Theorien der Keynesianer nicht mehr zureichend waren. Sie akzeptierten aber auch nicht die starren Annahmen der "`Neuen Klassiker"', diese Ökonomen könnte man "`Neu-Keynesianer"' nennen. Die "`Neu-Keynesianer"' sind dementsprechend keine geschlossene Gruppierung von Ökonomen, sondern behandelten eher zerstreut einzelne brennende Fragen der Ökonomie. Dazu gehörten zum Beispiel Fragen der Inflation (Phillips-Kurve), der Arbeitslosigkeit (Suchproblem, Natürliche Arbeitslosigkeit) und des Marktversagens (Informationsasymmetrien, Natürliche Monopole). 

Auch was die Personen betrifft ist die Abgrenzung schwieriger. So muss der Erzliberale \textsc{John Taylor} - Präsidenten der Mont Pelerin Society von 2018 - 2020 - inhaltlich zweifelsohne als ein früher Vertreter des Neu-Keynesianismus gesehen werden. 

Die schwierigste Abgrenzung erfolgt aber in zeitlicher Hinsicht. Schließlich wird die heutige Mainstream-Ökonomie häufig als "`Neu-Keynesianismus"' bezeichnet. In dieser Logik müsste man den "`Neu-Keynesianismus"' zumindest in zwei Generationen teilen. Zweifelsohne beginnt der "`Neu-Keynesianismus"' nämlich als Antwort auf die "`Neue Klassische Makroökonomie"' ab den frühen 1980er Jahren zu existieren. In Wahrheit sogar schon etwas früher, nämlich mit der Kritik an der Phillips-Kurve ab Mitte der 1960er Jahre. Dieser "`frühe"' Neu-Keynesianismus wird an dieser Stelle beschrieben und dauerte bis etwa Ende der 1980er Jahren. Die Hauptproponenten sind hier \textit{Edmund Phelps, Peter Diamond, Joseph Stiglitz, George Akerlof und William Baumol}. Diese Schule war der Gegenpol zur aufstrebenden "`Neuen Klassischen Makroökonomie"' um Lucas, Sargeant und Barro. Die Vertreter dieses frühen Neu-Keynesianismus liefern mit ihren Arbeiten vor allem "`Aufweichungen"' der zu starren Annahmen der "`Neuen Klassiker"'. Sie lehnen in diesem Sinn die Arbeiten der "`Neuen Klassiker"' ab, akzeptieren aber auch, dass der Keynesianismus veraltet ist. Von der Zuordnung der Personen her entwickelte sich der "`Neu-Keynesianismus"' eher aus den Salzwasser-Universitäten (vgl. die entsprechende Einteilung in Kapitel \ref{Neue Makro}), die die Neuen Klassiker ja strikt - und nicht nur auf inhaltlicher Ebene - ablehnten. Es wurden aber nicht alle Keynesianer zu "`Neu-Keynesianern"': James Tobin zum Beispiel bestand darauf ein "`Alt-Keynesianer"', nicht  "`Neu-Keynesianer"' zu sein \parencite[S. 45ff]{Tobin1993}. Edmund Phelps drückte dies so aus: "`I [had] warm personal relations with Jim [James] Tobin and Bob [Robert] Solow as well as with Bob [Robert] Lucas and Tom [Thomas] Sargent – relations that have survived our differences. But I belonged to neither school." \parencite{Phelps2006}


Ab Anfang der 1990er Jahre kam es zunehmend zu einer Verschmelzung von "`Neu-Keynesianismus"' und "`Neuer Klassischer Makroökonomie"'. Diese wird als "`Neue Neoklassische Synthese"' im nächsten Kapitel beschrieben. Da es eher eine Verdrängung der "`starren"' Neuen Klassischen Makroökonomie durch junge Vertreter des "`Neu-Keynesianismus"' ist, wird sie aber häufig auch einfach "`Neu-Keynesianismus"' genannt\footnote{Man könnte sie auch Zweite Generation des Neu-Keynesianismus nennen}. die Hauptvertreter sind hier \textsc{John Taylor}\footnote{der aber eigentlich auch zur ersten Generation der Neu-Keynesianer gezählt werden muss} \textsc{David Romer, Greg Mankiw, Blanchard und Paul Krugman}. Der Unterschied zwischen der ersten Generation der Neu-Keynesianer und der zweiten Generation ("`Neue Neoklassische Synthese"') ist, dass die Letztgenannte vor allem die Methoden der "`Neuen Klassiker"', insbesondere "`Dynamische Stochastische General Equilibrium"'-Modelle aus der "`Real Business Cycle"'-Theorie übernommen hat und um ursprünglich keynesianische Elemente, nämlich Monopolistische Konkurrenz, Rigide Löhne und Preise und Nicht-Neutralität der Geldpolitik (und Fiskalpolitik) in der kurzen Frist, übernommen hat. Mehr dazu aber im nächsten Kapitel

Allgemein aber täuscht der Name "`Neu-Keynesianismus"' auf jeden Fall: Er ist nicht etwa eine Weiterentwicklung des Keynesianismus. Schon die hier beschriebene "`Erste Generation der Neu-Keynesianer"' akzeptierte inhaltlich und methodologisch die Fortschritte durch die "`Neuen Klassiker"', bestand aber auf der Bedeutung von Fiskal- und vor allem Geldpolitik, sowie der Existenz von Marktversagen. 

 

\section{Phelps: Mikrofoundation der Makroökonomie}
\label{micmac}

Man findet wohl kaum einen Namen, der den Übergang von "`Keynesianismus"' zu "`Neu-Keynesianismus"' besser repräsentiert als \textsc{Edmund Phelps}. Ökonomisch geprägt wurde er in einem eindeutig keynesianischen Umfeld: Er verfasste bei James Tobin seine Dissertation und arbeitete Mitte der 1960er Jahre mit Robert Solow, Paul Samuelson und Franco Modigliani zusammen. Also alles eindeutig keynesianische Ökonomen, die wir aus Kapitel \ref{Synthese} kennen. Laut seines autobiografischen Artikels \textcite[S. 93]{Heertje1995} war diese Zeit, inklusive Gastprofessur am Massachusetts Institute of Technologie (MIT), die prägendste seiner Karriere. Er selbst war innerhalb weniger Jahre ein international anerkannter Ökonom. Schon 1961 veröffentlichte er sein erstes bedeutendes Werk: \textit{The Golden Rule of Accumulation} \parencite{Phelps1961}. Ein bemerkenswerter Artikel, den der gerade mal 28-jährige Phelps im American Economic Review veröffentlichte. Gerade einmal sieben Seiten lang, beginnt dieser - so wie im Englischen normalerweise Märchen  - mit "`Once upon a time"'. In weiterer Folge wechseln sich mathematische Formeln mit Dialogen zwischen dem König und dem Volk der Solovians ab \parencite[S. 640]{Phelps1961}. So witzig und amüsant die Geschichte des Artikels, so bahnbrechend ist auch deren Inhalt. Diese Arbeit kann als direkter Anschluss an die Wachstumstheorie Solow's gesehen werden und im Zentrum steht folgende hypothetische Überlegung: Wenn die gesamte aktuelle Wirtschaftsleistung für die Investition (Investition = Sparen!) in neue Produktionsgüter verwendet wird, dann wird nichts für den aktuellen Konsum ausgegeben. Wird hingegen die gesamte aktuelle Wirtschaftsleistung für Konsum verwendet, werden im Umkehrschluss keinerlei neuen Investitionen getätigt. Beide Extrembetrachtungen führen also zu keinem sinnvollen Gleichgewicht. Das heißt aber auch, dass dazwischen irgendein optimales Verhältnis zwischen Sparen/Investieren auf der einen Seite und Konsumieren auf der anderen Seite bestehen muss. Dieses erreicht man eben durch \textit{The Golden Rule of Accumulation}. Diese wird erreicht - solange man einige vereinfachenden Annahmen zulässt - wenn die Wachstumsrate des BIPs dem Zinssatz entspricht. Bereits Phelps nannte diese natürliche Wachstumsrate "`nachhaltig"' \parencite[S. 638]{Phelps1961}. Weiters zeigt Phelps formal, dass diese Wachstumsrate erzielt wird, wenn die Summe der Investitionen der Summe der Profite entspricht, also alle Profite investiert werden. Umgekehrt werden im Optimum alle Löhne konsumiert. Zusammengefasst: Wenn alle Löhne konsumiert werden und alle Profite investiert werden, befindet sich die Ökonomie auf einem nachhaltigen Wachstumspfad. Die Wachstumsrate entspricht dann dem Zinssatz. Insgesamt erinnert das Ergebnis an die Arbeiten von Wicksell und Hayek. Die formale Herleitung durch Phelps war aber zu diesem Zeitpunkt - im Jahre 1961 - eine bahnbrechende Erweiterung des Solow-Wachstumsmodells.

Das bisher in diesem Unterkapitel dargestellte, entspricht noch vollständig dem keynesianischem Denken aus Kapitel \ref{Synthese}. Im Jahr 1966 wechselte Phelps von Yale an die University of Pennsylvania (Penn). Mit dem Umzug konzentrierte er sich auf neue Themen, nämlich auf die theoretische Fundierung der Phillipskurve. Seine Arbeiten dazu sollten später die ersten Zweifel am dominierenden, keynesianschen Framework begründen. Im Nachhinein kann man getrost sagen, dass damit die Grundlagen für den "`Neu-Keynesianismus"' geschaffen wurden.

Wie in Kapitel \ref{sec: Phillips} dargestellt, war der vermeintliche, negative Zusammenhang zwischen Inflation und Arbeitslosigkeit zwar nicht Bestandteil der ursprünglichen keynesianischen Theorie. Aber in weiterer Folge vor allem in der keynesianischen Wirtschaftspolitik ein fixer Bestandteil. Unabhängig voneinander waren Milton Friedman und eben Edmund Phelps bereits ab Mitte der 1960er Jahre die ersten Ökonomen, die den Zusammenhang zwischen Inflation und Arbeitslosigkeit in Frage stellten. Wohlgemerkt zu einer Zeit, in der der Zusammenhang empirisch noch recht gut beobachtet werden konnte. Das in den 1970er Jahren diese Korrelation weitgehend verschwand gab den Kritikern Friedman und Phelps natürlich gehörig Auftrieb. Phelps hatte seine Kritik dabei - im Gegensatz zu Friedman - mathematisch-formal unterlegt. 

Der Artikel mit dem unscheinbaren Titel "`Money-Wage Dynamics and Labor-Market Equilibrium"' \parencite{Phelps1968} stellte die bis dahin unbestrittene Phillipskurve nicht nur infrage, sondern legte die Grundlage für eine ganz neue Sicht auf die Wirtschaftswissenschaften. Interessant ist, dass gleich mehrere Punkte, die natürlich ineinandergriffen, in diesem Artikel revolutionäre waren:
\begin{enumerate}
\item Die Mikrofundierung der Makroökonomie
\item Die formale Einführung der Erwartungen (als adaptive Erwartungen)
\item Die formale Einführung der Natürlichen Arbeitslosigkeit (später NAIRU)
\end{enumerate}
Bemerkenswert ist insbesondere, dass alle drei genannten Punkte bis heute fixer Bestandteil der Mainstream-Modelle sind. Die heutigen DSGE-Modelle sind mikrofundiert, beinhalten das Konzept der Erwartungen (wenn auch der rationalen statt der adaptiven) und akzeptieren einen gewissen Prozentsatz an Arbeitslosigkeit als Gleichgewichtszustand. Natürlich wurden alle drei Konzepte seit 1968 wesentlich erweitert, aber im Gegensatz zu den Arbeiten anderer großen Ökonomen, fällt auf, dass Phelps' Arbeiten bis heute, 50 Jahre später, kaum an Gültigkeit verloren. Keynes' Multiplikator ist heute höchst umstritten, Friedman's Geldmengensteuerung betreibt keine Zentralbank der Welt mehr und selbst die späteren Arbeiten von Robert Lucas wurden größtenteils von der Realität überholt. Phelps' bahnbrechende Erkenntnisse sind hingegen bis heute die Grundlage ökonomischer Modelle und kann daher als Geburtsstunde des "`Neu-Keynesianismus"' gesehen werden.

In seiner Nobelpreis Biographie schreibt Phelps, dass es seit seiner College-Zeit das Gefühl hatte die wichtigste aktuelle Herausforderung der Wirtschaftswissenschaften sei die Integration der Mikroökonomie in die Makroökonomie \parencite{Phelps2006}. Heute nennen wir dies die Mikrofundierung der Makroökonomie.
Der inhaltliche Ausgangspunkt des oben genannten Artikels \parencite{Phelps1968} ist die Phillipskurve. Phelps beschreibt sie als Naivität der Keynesianer. Wobei er Keynes selbst ausdrücklich in Schutz nimmt: Keynes' Nachfragesteuerung wäre niemals soweit gegangen einen dauerhaft stabilen Zusammenhang zwischen Inflation und Arbeitslosigkeit anzunehmen \parencite{Phelps2006}. Phelps stellt stattdessen einen Zusammenhang zwischen der \textit{erwarteteten} Inflation und Arbeitslosigkeit her. Dieser Zusammenhang sei aber nur in der kurzen Frist stabil. Angenommen die erwartete Inflation läge bei 4\%. Arbeitgeber und Arbeitnehmer würden bei ihren Vertragsverhandlungen diese Inflationserwartung einfließen lassen und die Lohnhöhe entsprechend festlegen. Will die Zentralbank nun die Arbeitslosigkeit senken, kann sie Maßnahmen setzen, die die Inflation auf zum Beispiel 6\% erhöhen. Solange die erwartete Inflation unter der tatsächlichen Inflation liegt, wird die Arbeitslosigkeit sinken und sich somit wie von der Phillipskurve postuliert verhalten. Es ist aber klar, dass die Diskrepanz zwischen tatsächlicher und erwarteter Inflation nur kurzfristig aufrechterhalten werden kann, bevor sich die Erwartung dem tatsächlichen Wert anpasst. Die keynesianische, langfristige Phillipskurve wurde durch die neu-keynesianische, kurzfristige erwartungsgestützte Phillipskurve ersetzt. Als solche findet sie bis heute Eingang in die makroökonomischen Lehrbücher. Nebenbei etablierte Phelps dabei das Konzept der adaptiven Erwartungen, das aber später vom neuklassischen Konzept der rationalen Erwartungen abgelöst werden sollte.
Die zentrale Aussage in \textcite{Phelps1968} lautet, dass durch Geldpolitik die Arbeitslosigkeit nicht dauerhaft beeinflusst werden kann, sehr wohl aber unter Umständen in der kurzen Frist. Geldpolitik funktioniere außerdem über Inflations\textit{erwartungen} und diese passen sich recht schnell an die aktuelle Inflation an. Eine niedrige Inflation wird daher auch nicht langfristig zu höherer Arbeitslosigkeit führen\parencite{Phelps1967}. Daraus könnte man ableiten, dass die zentrale Aufgabe der Zentralbanken die Inflationssteuerung ist. Heute orientieren sich fast alle führenden Zentralbanken tatsächlich primär an den Inflationszielen, dies aber direkt auf Phelps' frühe Arbeiten zurückzuführen ginge aber zu weit, folgten doch noch weitere Arbeiten dazu von anderen Neu-Keynesianern und Neuen Klassikern.
Sehr wohl direkte Folge aus \textcite{Phelps1968} ist hingegen die Idee der "`natürlichen Arbeitslosenrate"'. Während die Klassiker davon ausgingen, dass es im Gleichgewicht keine Arbeitslosigkeit gäbe und die Neuen Klassiker meinten im Gleichgewicht gäbe es ausschließlich freiwillige Arbeitslosigkeit, verfolgten die Keynesianer den Ansatz Arbeitslosigkeit sei stets mit nachfrageorientierter Wirtschaftspolitik zu minimieren. Diese "`natürliche Arbeitslosenrate"' wird häufig Milton Friedman zugeschrieben, der einen sehr ähnlichen Ansatz ebenfalls 1968 veröffentlichte \parencite{Friedman1968}. Tatsächlich hatten Friedman und Phelps unterschiedliche Wege gewählt, die sie zu den gleichen Schlussfolgerungen führten.

Die Mikrofundierung der Makroökonomie wird ebenfalls häufig als wesentliche Neuerung der "`Neuen Klassischen Makroökonomie"' gesehen. Es ist auch tatsächlich so, dass die Neuen Klassiker diesen Ansatz als Standard in ökonomischen Modellen etablierten und somit eine wesentliche Neuerung gegenüber den alten Schulen Keynesianismus und Monetarismus einführte. Aber auch hier gilt, dass die erstmalige Anwendung auf Phelps zurückgeht. 










Dieses Ergebnis war natürlich "`schockierend"' für die Keynesianer: Die Phillipskurve war zwar wenig theoretisch begründet, spielte aber in der keynesianisch geprägten Wirtschaftspolitik eine wichtige Rolle. Das Ergebnis, dass Geldpolitik in der langen Frist als nachfrageorientierte Wirtschaftspolitik wirkungslos sei, beschränkte das keynesianische Framework auf Fiskalpolitk.

Die Mikrofundierung der Makroökonomie war ebenfalls ein Schlag für die "`alten"' Keynesianer, da sie ganz andere Wege beschritt-





Zwei wesentliche Artikel:

Artikel: \textcite{Phelps1967},

HIER WEITER: 
https://www.nobelprize.org/prizes/economic-sciences/2006/phelps/biographical/  : Hier: As I began my gradual departure from Penn I th

http://www.columbia.edu/~esp2/autobio1.pdf (Seite 94)
http://www.columbia.edu/~esp2/

Hierfür später den Nobelpreis 2006:
Bahnbrechende Teile darin:

Mikrofundierung der Makro



In den 1970er Jahren schließlich begründete er - als Antwort auf die Neue Klassische Makroökonomie - einen der wesentlichen Punkte der Neu-Keynesianer (mit Taylor und Calvo) unter anderem: "`NAIRU"'


ALT HIER:
dennoch leistete er, gemeinsam mit Milton Friedman die ersten Ideen zur Mikrofundierung der Makroökonomie, die später einer der zentralen Punkte der "`Neuen Klassik"' werden sollte. Er stellte den Zusammenhang von Inflation und Arbeitslosigkeit in Frage mit dem formalen Argument, dass eine Realgröße wie die Arbeitslosigkeit nicht systematisch mit einer Nominalgröße wie der Inflation korrelieren könne. 

Stattdessen ist die \textit{erwartete} Inflation von entscheidender Bedeutung:  "`expectations-augmented Phillips 
curve"' The intertemporal perspective implies that current inflation expectations affect the future 
tradeoff between inflation and unemployment. A higher current inflation rate typically leads to 
higher inflation expectations in the future, so that it then becomes more difficult to achieve the 
objectives of stabilization policy

A key result was that the long-run rate of unemployment cannot be influenced 
by monetary or fiscal policy affecting aggregate demand. Phelps’s analysis thus identified 
important limitations on what demand-management policy can achieve. This view has become 
predominant among macroeconomic researchers as well as policymakers. As a result, 
macroeconomic policy is carried out very differently today from what it was forty years ago.


ALT BIS HIER





Zusammengefasst:

Erster Schritt: Phelps arbeitete zunächst in der Tradition der Keynesianer und entwickelte die "`Goldene Regel der Akkumumlation"'

Zweiter Schritt: Die Emanzipation vom Keynesianismus erfolgte mit seinen Arbeiten zur Revolution der Phillipskurve. Diese umfassten nämlich Konzepte, die bis heute in der Mainstream-Ökonomie State-of-the-Art sind. Erstens, war er ein Vorreiter bei der Mikrofundierung der Makroökonomie und zweitens, etablierte er adaptive "`Erwartungen"' in die Modelle der Ökonomie. Beides spielte später bei den "`Neuen Klassikern"' eine wesentliche Rolle, wenn auch in der Form der \textit{rationalen} statt der \textit{adaptiven} Erwartungen. Er nahm also die Kritikpunkte der Neuen Klassiker am Keynesianismus vorweg.

Dritter Schritt: Gegenbewegung zur Kritik der "`Neuen Klassiker"' inklusive der Entwicklung der "`Natürlichen Arbeitslosigkeit"'

\section{Marktversagen als Teil der Ökonomie}
\label{Marktversagen}

Die Arbeiten von Phelps waren, wie gerade erwähnt, der Ursprung des Neu-Keynesianismus und wurden zeitlich vor der neu-klassischen Revolution formuliert. Die meisten neu-keynesianischen Arbeiten der 1. Generation entstanden allerdings als direkte Antworten auf die aufkommenden aber mit starren Annahmen unterlegten Arbeiten der "`Neuen Klassiker"'.


\subsection{Informationsasymmetrie: Spence, Stiglitz und Akerlof}


\subsection{Natürliche Monopole oder Baumol's angreifbare Märkte}


\section{Arbeitslosigkeit als Suchproblem}
\label{Suchtheorie}

\subsection{Diamond, Mortensen und Pisaridis}

Widersprach rasch der "`Neuen Klassik"' indem sie die natürliche Arbeitslosigkeit erweiterte.




		% Neu-Keynes (1. Generation)            !!! KORREKTURLESEN
%%%%%%%%%%%%%%%%%%%%% chapter.tex %%%%%%%%%%%%%%%%%%%%%%%%%%%%%%%%%
%
% sample chapter
%
% Use this file as a template for your own input.
%
%%%%%%%%%%%%%%%%%%%%%%%% Springer-Verlag %%%%%%%%%%%%%%%%%%%%%%%%%%

\chapter{Neue Neoklassische Synthese}
\label{Neue Neoklassische Synthese}

Mit diesem Kapitel sind wir in der Gegenwart der Ökonomie angekommen. Man kann zwar durchaus argumentieren, dass die Makroökonomie nach der weltweiten Weltwirtschaftskrise ab 2008 und der immer noch praktizierten globalen Nullzinspolitik eine erneute "`Revolution"' nötig hätte, aber Stand 2022 ist die State-of-the-Art Mainstream-Ökonomie die \textit{Neue Neoklassische Synthese}. Der Begriff "`Neue Neoklassische Synthese"' ist (noch) nicht wirklich etabliert als Bezeichnung für den aktuellen wirtschaftswissenschaftlichen "`Mainstream"'. Meist spricht man stattdessen von "`Neu-Keynesianismus"' oder auch von "`Neoklassik"'\footnote{In Lehrbüchern wird häufig ohne Unterschied vom "`Neu-Keynesianismus"' gesprochen. Auch "`Neu-Keynesianismus der 2. Generation"', "`Neue Synthese"', "`Neue Keynesianische Synthese"'  kommen vor. Selten werden die Modelle auch als "`Neo-Wicksellianisch"' bezeichnet \parencite[S. 28]{Gali2007}, dies wegen der Ähnlichkeit zur Theorie von Wicksell der Abweichungen vom natürlichen Gleichgewicht beschreibt}. Beide Begriffe sind aber nicht eindeutig. Um Unklarheiten zu vermeiden wird hier der etwas holprige, aber eindeutige und inhaltlich meiner Meinung nach passende Begriff "`Neue Neoklassische Synthese"' (oder schlicht "`Neue Synthese"') verwendet.

Ungefähr um 1990 versuchten Ökonomen, die nicht vom Streit zwischen Neu-Klassikern und Neu-Keynesianern vorbelastet waren, unvoreingenommen das beste aus beiden Welten zu übernehmen und zu einer "`Neuen Synthese"' zusammen zuführen. Die Abgrenzung ist zwischen "`Neu-Keynesianismus"' und "`Neuer Synthese"' ist hierbei sowohl inhaltlich als auch zeitlich schwierig. Vor allem, weil viele Ökonomen, die uns im letzten Kapitel untergekommen sind, auch in diesem Kapitel die "`Hauptdarsteller"' sein werden. Es gibt aber auch durchaus Abspaltungen bei den Vertretern des "`Neu-Keynesianismus"': Paul Krugman, Joseph Stiglitz, lehnen vielen Ansätze der "`Neuen Synthese"' heute weitgehend ab. Mankiw, Blanchard und David Romer sind schwieriger einer der beiden Schulen zuzuordnen, sie stehen für den Übergang von "`Neu-Keynesianismus"' zur "`Neuer Synthese"'. Ein zentraler Vertreter der "`Neuen Sythese"' (ohne Vergangenheit im "`Neu-Keynesianismus) ist Jordi Gali. Ein spezieller Vertreter ist John Taylor. Er mitbegründete - gemeinsam mit \textcite{Phelps1968} und \textcite{Fischer1977} - den "`Neu-Keynesianismus"' \parencite{Taylor1977}. Aber auch für die "`Neue Synthese"' lieferte er einen der grundlegenden Journal-Beiträge \parencite{Taylor1993}.

HIER WEITER





Passend dazu verschwammen damit auch die ideologischen Unterschiede zwischen den verschiedenen ökonomischen Gruppen. Konnte man bis in die 1980er Jahre hinein die ökonomischen Richtungen einer politischen Richtung zuweisen, ist dies nun nicht mehr möglich. Sozialdemokraten (Kontinental-Europa), Labour-Party (UK) und Demokraten (USA) waren dem Keynesianismus zugeneigt. Christ-Demokraten (Kontinental-Europa), Conservative Party (UK) und Republikaner (USA)  
Die Ideen zur Geldpolitik des Erzliberalen John Taylor, die Arbeiten des bekennenden Republikaners Mankiw aber auch 



Elemente aus verschiedenen Schulen:
\begin{itemize}
	\item Keynesianismus: Rigiditäten. Teilweise Fiskalpolitik im Krisenfall
	\item Monetarismus: Zentrale Bedeutung der Geldpolitik und der Zentralbanken, Natürliche Arbeitslosigkeit
	\item Österreichische Schule: Konzept des Gleichgewichtszinssatzes (Wicksell)
	\item Neu-Keynesianismus: "`Nominale Rigiditäten"', "`Monopolistische Konkurrenz"' und "`Nicht-Neutralität der Geldpolitik in der kurzen Frist"'. Älter: NAIRU, Marktversagen
	\item Neue Klassische Makroökonomie: Annahme "`Rationale Erwartungen"', Real-Business-Cycle-Modelle.
\end{itemize}




Es gibt aber zwei Punkte, die sich seit Anfang der 1990er-Jahre entwickelt haben und die Makroökonomie und deren Wirtschaftspolitik seither eindeutig prägen und sehr wohl eine eindeutige Abgrenzung ermöglichen:
\begin{enumerate}
	\item Erstens, in der makroökonomischen Theorie: die Zusammenführung der formal-mathematischen Real-Business-Cycle-Gleichgewichtsmodelle mit Elementen der Neu-Keynesianer.
	\item Zweitens, in der Wirtschaftspolitik: Die Dominanz der Bedeutung der Geldpolitik und der Aufstieg der Zentralbanken zum wichtigsten wirtschaftspolitischen Player.
\end{enumerate}

\section{Die frühen 1990er-Jahre}
Bevor wir uns aber diese Entwicklungen im Detail betrachten, blicken wir auf das Umfeld: Arbeitslosigkeit statt Inflation als Herausforderung. 
Staatsverschuldung kommt in den Fokus, dafür Fiskalpolitik aus dem Fokus


\section{DSGE: Die Zweckehe zwischen "`Neuen Klassikern"' und "`Neu-Keynesianern"'}

Die "`Neu-Keynesianer"' entwickelten die Elemente "`Nominale Rigiditäten"', "`Monopolistische Konkurrenz"' und "`Nicht-Neutralität der Geldpolitik in der kurzen Frist"'. Die "`Neue Synthese"'  bettete diese Elemente in die ursprünglich "`neu-klassischen"' RBC-Modelle vollständig ein \parencite[S. 6]{Gali2015}. Die Kombination der "`neu-keynesianischen"' Ideen in die "`neu-klassischen"' Modelle begründet schließlich den Namen "`Neue Neoklassische Synthese"'. 


\subsection{HANK}
"`HANK"'!
Kapitel 9 im Gali-Buch!

\subsection{Neue Philips Kurve}

In Gali and Gertler (1999) and Gali, Gertler and Lopez-Salido (2005),


Jordi Gali usw nach 2008




Richard Clarida and Jordi Gali and Mark Gertler, 2000. "Monetary Policy Rules and Macroeconomic Stability: Evidence and Some Theory," The Quarterly Journal of Economics, Oxford University Press, vol. 115(1), pages 147-180.

Mark Gertler and Jordi Gali and Richard Clarida, 1999. "The Science of Monetary Policy: A New Keynesian Perspective," Journal of Economic Literature, American Economic Association, vol. 37(4), pages 1661-1707, December.



Inhaltlich spielt in der Wirtschaftspolitik fast ausschließlich nur mehr die Geldpolitik eine Rolle. Zur Fiskalpolitik haben die Vertreter der "`neuen Neoklassischen Synthese"' praktisch ausschließlich eine ablehnende Haltung.







\section{Taylor-Rule oder die Verwissenschaftlichung der Zentralbanken}



\subsection{Exkurs: Die Evolution der Zentralbanken}
Vom Verwalter des Goldstandards, zum Hüter der Wechselkurse, zum Spieler gegen Spekulanten, zur zentralen Player der Wirtschaftspolitik

\subsection{Taylor-Rule: Ein pragmatischer Zugang zur Geldpolitik}
Alle  in diesem Kapitel behandelten Themen umfassten Aspekte die begründen warum Märkte in der Regel nicht vollkommen reibungslos funktionieren. Wäre dies der Fall wäre aktive Wirtschaftspolitik wirkungslos und Konjunkturschwankungen wären rein zufällig, wie im Real-Business-Cycle-Framework dargestellt.
\textcite[S. 823]{Akerlof1985}

Taylor ist eigentlich eher dem erzkonservativem Spektrum zuzuordnen. Unter anderem ist er Vorsitzender der Mont-Pelerin-Gesellschaft. Aber die von ihm etablierte Idee der Zentralbankensteuerung war ein zentraler Schritt zur erneuten "`Wiedervereinigung"' der Ökonomie, also zur "`Neuen neoklassischen Synthese"'.

Die "`Neue Klassische Makroökonomie"' sorgte für viel Wirbel innerhalb der Ökonomie. Der Ton der vorgebrachten Kritik war ungewöhnlich scharf. Die wirtschafts-wissenschaftliche Community war deutlich zerstritten. Aber die "`Neue Klassik"' brachte eben auch viele neue Erkenntnisse und Wege aus den empirisch beobachteten Problemen. Eine gewisse Zeit lang sah es so aus, als würde diese Schule der neue "`Mainstream"' werden. Aber wieder erwiesen sich die Ideen als zu radikal. Empirisch hielt vor allem die Annahme, es gebe keine Preis- und Lohnrigiditäten nicht stand.
Und so kam es dazu, dass sich in der langen Frist die etablierte Mainstream-Ökonomie, also die neoklassische Synthese, durchsetzte. Nicht aber ohne jene Ideen aus der Neuen Klassischen Makroökonomie zu übernehmen, die sich als richtig erwiesen hatten. Dies waren vor allem:

Die Taylor-Rule veränderte die Geldpolitik. Von den Geldmengenzielen (Friedman) zu den Zinssatz-Regeln \parencite[S. 36]{Gali2007}


















Vier Quadrate nach \textcite{RomerDavid1993}




RBC + \textcite{RomerDavid1990}
Nominale Rigiditäten
Monopolistische Konkurrenz
Geldpolitik in der kurzen Frist (Fehlende Klassische Dichotomie)

Formulierung der Taylor-Rule (1993?) als  Übergangszeitpunkt 1. Generation --> 2. Generation
Zweiter Übergangspunkt: Ab 1990 viel stärker empirisch (bis dahin sehr theoretische Arbeiten der Neu-Keynesianer)

Problem der Arbeitslosigkeit trat in den Vordergrund.


(Taylor Rule)

Neue Phillips Kurve

Cost of Inflation (\textcite{Snowdon2005} ab S. 401) und Inflations-targeting

Erstes Neu-keynesianisches DSGE-Modell: Rotemberg und Woodford
Rotemberg, Julio; Woodford, Michael (1993), "Dynamic General Equilibrium Models with Imperfectly Competitive Product Markets", NBER Working Paper No. 4502

Wirtschaftspolitisch Dominanz der Geldpolitik
Fiskalpolitik selbst in Krisen umstritten (Paper zur empirischen Bestimmung der Wirksamkeit der Fiskalpolitik (Blanchard))


 








heutige Mainstream-Modell basieren auf dem "`Neu-Keynesianischen Ausgangsmodell"', das bis heute das Arbeitstier in der Analyse von wirtschaftspolitischen Maßnahmen, Gleichgewichtsabweichungen und Wohlstand ist \parencite[S. 52]{Gali2015}









\begin{itemize}
	\item die Annahme rationaler Erwartungen
	\item die Mikrofundierung der Makroökonomie
	\item (vergleiche Kapitel Neue Klassische Makro)
\end{itemize}






\section{Auf den Schultern von Giganten}
\label{Giganten}

\subsection{Krugman}

\subsection{Blanchard}

\subsection{David Romer \& Mankiv}




		% Neue Neoklassische Synthese			!!! KORREKTURLESEN	
%%%%%%%%%%%%%%%%%%%%% chapter.tex %%%%%%%%%%%%%%%%%%%%%%%%%%%%%%%%%
%
% sample chapter
%
% Use this file as a template for your own input.
%
%%%%%%%%%%%%%%%%%%%%%%%% Springer-Verlag %%%%%%%%%%%%%%%%%%%%%%%%%%

\chapter{Sisyphus-Ökonomie}

Die Geschichte scheint sich auch für die Ökonomie immer wieder zu wiederholen. Kaum scheint es so als wären alle Fragen und Herausforderungen der Disziplin wären weitgehend gelöst, sorgt ein Ausnahme-Event für die Zerstörung dieser Illusion. So sprach man bei Marshall vom Vollender der Neoklassischen Ökonomie, bis die "`Great Depression"' Ende der 1920er Jahre deren Unzulänglichkeiten aufzeigte. In den frühen 1970er-Jahren gab es so etwas wie einen Konsens zwischen Keynesianern und Monetaristen. Wieder schien ökonomische Forschung alle relevanten Fragen beantwortet zu haben. Bis in den 1970er Jahren das Phänomen der "`Stagflation"' auftrat und die Lucas-Kritik für einen neuen Paukenschlag sorgte. Und gerade als die "`Neue neoklassische Synthese"' die früheren Konkurrenten der Neuklassik und der Neu-Keynesianer in den DSGE-Modellen vereinte und just nachdem Robert Lucas nach der Jahrtausendwende verkündete, "`alle wesentlichen Probleme der Makroökonomie seien auf Jahrzehnte hinaus gelöst"', erschütterte die "`Great Recession"' die Welt. Zwar muss man der modernen Makroökonomie zugute halten, dass ihre Methoden schlimmeres verhindert hat - sowohl BIP-Einbrüche als auch Arbeitslosenzahlen fielen wesentlich geringer aus als in der "`Great Depression"' - allerdings war das resultierende historisch einmalige Niedrigzinsniveau der Folgejahre mit keiner der modernen makroökonomischen Theorien erklärbar.


1:1 aus Blanchard

Anfang des neuen Jahrtausends aber schien sich eine Synthese herauszubilden.
Methodisch baute sie auf dem Ansatz der Real-Business-Cycle-Theorie auf mit ihrer
exakten Beschreibung des Optimierungsverhaltens von Haushalten und Unterneh-
men. Sie berücksichtigte die Bedeutung von Änderungen in der Rate des technischen
Fortschritts, die sowohl von der „RBC-Theorie“ wie von der Neuen Wachstumstheorie
betont wird. Aber sie integrierte auch wesentliche Elemente des neu-keynesianischen
Ansatzes – sie integrierte viele Friktionen wie Suchprozesse am Arbeitsmarkt, unvollständige
Information auf Kreditmärkten und die Rolle nominaler Rigiditäten für die
aggregierte Nachfrage. Es gab zwar keine Konvergenz zu einem einzigen einheitlichen
Modell oder einer einheitlichen Liste relevanter Friktionen, aber es herrschte Übereinstimmung
über den Forschungsrahmen und die analytische Vorgehensweise.
Ein gutes Beispiel dafür ist die Forschung von Michael Woodford (Columbia Universität
New York) und Jordi Gali (Pompeu Fabra in Barcelona). Sie entwickelten gemeinsam
mit Koautoren das Neue Keynesianische Modell, das Nutzen- und Gewinnmaximierung
mit nominalen Rigiditäten kombiniert – der Kern dieses Modells wurde in
einfacher Form in Kapitel 17 vorgestellt. Dieser Modellansatz hat sich als höchst einflussreich
bei der Neugestaltung der Geldpolitik erwiesen – angefangen von Inflationssteuerung
bis zu Zinsregeln, die in Kapitel 25 behandelt wurden. Der Ansatz bildet die
Basis einer neuen Klasse großer Modelle, die auf der einfachen Struktur aufbauen, aber
eine große Zahl von weiteren Friktionen einbauen. Sie lassen sich nur mehr numerisch
lösen. Diese Modelle, DSGE-Modelle (dynamic general equilibrium analysis) genannt,
sind mittlerweile zum Standardinstrument der Zentralbanken geworden.


\section{Die Great Recession}

Doug
Diamond (Universität Chicago) und Philip Dybvig (Universität Washington) erforschten
schon in den 1980er-Jahren die Mechanismen von Bank Runs (vgl. Kapitel 4): Weil
Aktiva illiquide, Passiva aber liquide sind, unterliegen selbst solvente Banken dem
Risiko eines Runs. Dieses Problem kann nur vermieden werden, wenn die Zentralbank
in einem solchen Fall Liquidität bereitstellt. Bengt Holmström (MIT) und Jean Tirole
Toulouse) zeigten, dass Fragen der Liquidität in modernen Volkswirtschaften zentrale
Bedeutung zukommt. Nicht nur Banken, selbst Unternehmen können durchaus in die
Lage geraten, zwar solvent, aber trotzdem illiquide zu sein – also nicht in der Lage, sich
zusätzliche Mittel zu beschaffen, um an sich rentable Projekte fertigzustellen. Andrej
Shleifer (Harvard Universität) und Robert Vishny (Universität Chicago) wiesen in ihrer
Arbeit über die Grenzen der Arbitrage nach, dass als Folge asymmetrischer Information
Investoren Arbitragemöglichkeiten nicht ausnutzen können, wenn der Vermögenswert
unter den Fundamentalwert sinkt. Im Gegenteil, sie können sogar genötigt werden, auch
selbst solche Vermögenswerte zu verkaufen und so zu einem Preisverfall beizutragen.
Die Forschungsrichtung der Verhaltensökonomie (etwa von Richard Thaler, Chicago
Universität) hat aufgezeigt, in welcher Weise Individuen vom Modell des rationalen
Agenten abweichen und welche Implikationen sich daraus für Finanzmärkte ergeben.





\section{After Great Recession}

\subsection{Grenzen der Geldpolitik}
Geldpolitik im Angesicht der "`Zero Lower Bound on the Nominal Interest Rate"'. Romer-Buch S. 615.
Krugman 1998




\subsection{Wiederauferstehung der Fiskalpolitik oder Austerität?}



Wirtschaftspolitisch Dominanz der Geldpolitik
Inhaltlich spielt in der Wirtschaftspolitik fast ausschließlich nur mehr die Geldpolitik eine Rolle. Zur Fiskalpolitik haben die Vertreter der "`neuen Neoklassischen Synthese"' praktisch ausschließlich eine ablehnende Haltung.
Fiskalpolitik selbst in Krisen umstritten (Paper zur empirischen Bestimmung der Wirksamkeit der Fiskalpolitik (Diskussion in der Europäischen Währungskrise zum Multiplikator (Blanchard änderte die Meinung))

\textcite[S. 130]{Christiano2018}



Buch "`What have we learned"'
Fiskalpolitik \textcite[S. 131]{Christiano2018}


Seite 409 (Snowdon/Vane)
\parencite{Woodford2011}
JEEA 2007 von Gali, Lopez-Salido und Valles 

Könnte neuen Schwung durch \textcite{Kaplan2018} erhalten



















		% Das neue Jahrtausend


%%%%%%%%%%%%%%%%%%%%%%%% part.tex %%%%%%%%%%%%%%%%%%%%%%%%%%%%%%%%%%
%
% sample part title
%
% Use this file as a template for your own input.
%
%%%%%%%%%%%%%%%%%%%%%%%% Springer-Verlag %%%%%%%%%%%%%%%%%%%%%%%%%%


\part{Politische Ökonomie und Institutionalismus} \label{Heterodox}
		% Neue Politische Ökonomie und Institut.
%%%%%%%%%%%%%%%%%%%%%% chapter.tex %%%%%%%%%%%%%%%%%%%%%%%%%%%%%%%%%
%
% sample chapter
%
% Use this file as a template for your own input.
%
%%%%%%%%%%%%%%%%%%%%%%%% Springer-Verlag %%%%%%%%%%%%%%%%%%%%%%%%%%

\chapter{Institutionsökonomik}
\label{Institut}







		% Institutionsökonomie
%%%%%%%%%%%%%%%%%%%%% chapter.tex %%%%%%%%%%%%%%%%%%%%%%%%%%%%%%%%%
%
% sample chapter
%
% Use this file as a template for your own input.
%
%%%%%%%%%%%%%%%%%%%%%%%% Springer-Verlag %%%%%%%%%%%%%%%%%%%%%%%%%%

\chapter{Neue Institutionsökonomik}
\label{Neue Institut}

\section{Die Ursprünge: Transaktionen sind nicht gratis} \label{sec: Neue Inst}

Ronald Coase: Eigentumsrechte statt Pigou-Steuer: Kritik an Pigou \textcite[S. 243]{Cansier1989}. Coase stellt im Hinblick auf externe Effekte - wie die gerade aktuell diskutierten Umweltprobleme - Marktlösungen in den Vordergrund. Ein Staatseingriff ist ihm zufolge nicht unbedingt notwendig (Vergleiche dazu Kapitel \ref{sec: Pigou}).

Kenneth Arrow 1951: Unmöglichkeitstheorem und Social Choice Theorie



Es ist eine interessante Tatsache, dass als Ausgangspunkt für die "`Neue Institutionsökonomik"' immer wieder das Werk von Coase: \textit{Theory of the Firm} aus dem Jahr 1937 genannt wird. Interessant deshalb, weil zwischen diesem Ausgangspunkt und den weiteren Arbeiten im Bereich Transaktionskostentheorie -  oder Neue Institutionsökonomik überhaupt - ungefähr 30 Jahre vergingen \parencite[S. 148]{Blaug2001}.

Coase, Williamson, North

Prinzipal Agent Theorie


\section{Acemoglu: Kein Wohlstand ohne Institutionen}
Verbindung zu Endogener Wachstumstheorie

		% Neuer Institutionalismus
%%%%%%%%%%%%%%%%%%%%% chapter.tex %%%%%%%%%%%%%%%%%%%%%%%%%%%%%%%%%
%
% sample chapter
%
% Use this file as a template for your own input.
%
%%%%%%%%%%%%%%%%%%%%%%%% Springer-Verlag %%%%%%%%%%%%%%%%%%%%%%%%%%

\chapter{Neue Politische Ökonomie}
\label{Neue_Politik}

Der Ökonomie-Zweig "`Politische Ökonomie"' ist nicht so einfach zu erfassen. Wie der Begriff selbst schon ausdrückt, handelt es sich hierbei um die Beschreibung gesamtwirtschaftlicher Zusammenhänge unter dem Einfluss von zentralen Entscheidungsträgern. Die meisten Klassiker (vgl. Kapitel \ref{Klassik}) interpretierten die Volkswirtschaftslehre in genau diesem Sinne und bezeichneten, oder beschrieben, ihre Werke als Arbeiten in "`Politischer Ökonomie"'. Paradebeispiele hierfür sind vor allem die Arbeiten von David Ricardo, John Stuart Mill, aber auch Karl Marx. Mit dem Aufstieg der Neoklassik seit 1870 wurde die Volkswirtschaftslehre immer stärker zu einer positiven Wissenschaft und ebenso zu einer Volkswirtschafts\textit{theorie}, in welcher der politische Einfluss weniger bedeutsam war. Damit einher ging die wachsende Marktgläubigkeit, also die Überzeugung, dass die allermeisten Märkte gut funktionieren und keinen staatlichen Eingriff benötigen. Damit verbunden waren niedrige Staatsquoten um die Jahrhundertwende. Die Staatsquote drückt aus, wie hoch der Anteil der Staatsausgaben am Bruttoinlandsprodukt ist. Damit ist die Staatsquote ein Maß dafür, an welchem Anteil der gesamten wirtschaftlichen Tätigkeiten die öffentliche Hand beteiligt ist. Um 1900 betrug dieser Wert für die damaligen Industriestaaten zwischen 10\% und 15\%. Den Einfluss des Staates zu vernachlässigen, war in der hohen Zeit der Neoklassik also eine nicht allzu unrealistische Vereinfachung. Vor der Corona-Krise\footnote{Während der Corona-Krise stiegen die Staatsquote stark an. Zum Einen, weil das BIP sank, vor allem aber aufgrund der teilweise sehr hohen Staatshilfen.} lag die Staatsquote in den westlichen Industriestaaten zwischen 40\% in den angelsächsischen Ländern und 50\% in den kontinental-europäischen Sozialstaaten. Der Anteil - aber auch der Einfluss - der öffentlichen Hand an der gesamten Wertschöpfung ist in den letzten hundert Jahren also enorm gestiegen. Dafür gibt es verschiedene Erklärungsansätze. \textcite{Wagner1892} bereits hatte eine fortlaufend steigende Staatsquote theoretisch postuliert. Mit wirtschaftlicher Entwicklung steige nämlich der Bedarf an den typischen Leistungen des Sozialstaates. Die Vorhersage Wagner's ist beeindruckend und bewahrheitete sich zwischen 1900 und 1975 für praktisch alle entwickelten Länder. Die Gründe hierfür waren allerdings nicht nur wachsende Sozialausgaben, sondern auch die beiden Weltkriege, in Folge deren der Staat wesentliche Aufgaben übernahm. Aus unserer wirtschafts-theoretischen Sicht allerdings, ist vor allem der Aufstieg des Keynesianismus (vgl. Kapitel \ref{Keynes}) von Interesse. Mit der Geburt der Makroökonomie durch die Veröffentlichung von \textcite{Keynes1936}  wurden der öffentlichen Hand nämlich auch wichtige wirtschaftspolitische Aufgaben in Form von aktiver Fiskal- und Geldpolitik zugeschrieben. In der folgenden hohen Zeit der Neoklassischen Synthese (vgl. Kapitel \ref{Synthese}) wurden aktive Staatseingriffe in die Gesamtwirtschaft sogar noch weiter forciert (vgl. Kapitel \ref{cha: Marktversagen}). Wer aber dieser Staat ist und ob seine Handlungen tatsächlich immer optimal für die Gesellschaft sind, wurde dabei nicht thematisiert. Interessanterweise hatte Keynes selbst Zeit seines Lebens eine recht bescheidene Meinung über Politiker \parencite[S. 519]{Snowdon2005}. Obwohl er mehr als die meisten anderen Ökonomen mit Politikern eng zusammenarbeitete und sein erstes berühmtes Werk (\textcite{Keynes1919}: "`The Economic Consequences of the Peace"') direkt auf das Versagen von Politikern verwies, postulierte er, dass durch staatliche Eingriffe die reinen Marktlösungen verbessert werden können, ohne daran zu denken, dass auch diese staatlichen Eingriffe letztendlich von Menschen durchgeführt werden müssen, die wiederum Interessen verfolgen könnten, die nicht der Allgemeinheit dienen, sondern ausschließlich ihnen selbst. Richtigerweise relativiert Alberto Alesina - einer der führenden Vertreter der Neuen Politischen Ökonomie - allerdings, dass niemand, auch nicht Keynes, bei einer revolutionären Ausarbeitung alle Eventualitäten bis zum Ende durchdenken könne \parencite[S. 569]{Snowdon2005}. Erst ab den 1970er Jahren, mit dem Auftreten von Stagflation und dem Wiedererstarken \textit{theoretischer}, liberaler Wirtschaftsschulen (vgl. Kapitel \ref{Monetarismus} und Kapitel \ref{Neue Makro}), kamen auch Fragen zur uneingeschränkten wirtschafts\textit{politischen} Sinnhaftigkeit staatlicher Eingriffe auf. 

Aber auch auf mikroökonomischer Ebene wurden Fragen zur Bedeutung des Staates immer wichtiger. Wie in Kapitel \ref{sec: Pigou} bereits dargestellt, wurde in den 1920er Jahren innerhalb der Neoklassischen Schule die Notwendigkeit von Staatseingriffen im Falle von Marktversagen erkannt und diskutiert. Daraus entwickelte sich die Disziplin der "`Wohlfahrtsökonomie"', bzw. nach dem Zweiten Weltkrieg die "`Neue Wohlfahrtsökonomie"', dargestellt in Kapitel \ref{Wohlfahrt}. Die Aussagen dieser Schule waren in der wissenschaftlichen Community stets recht umstritten, weil es sich dabei um weitgehend normative Theorien handelt. Wohlfahrtsökonomen rangen immer mit der Frage, ob Werturteile in der Ökonomie zulässig sind. Im Hinblick auf wirtschaftlichen Tätigkeiten des Staates sprachen im Zeitraum nach 1945 alle führenden Ökonomen recht unbestimmt und unspezifisch von staatlichen Eingriffen. Wie im Kapitel \ref{cha: Marktversagen} dargestellt, war es unter Ökonomen recht unumstritten, dass im Falle von Marktversagen und insbesondere beim Problem Öffentlicher Güter (vgl. Kapitel \ref{Offentliche Guter}) der Staat in den Markt eingreifen und als Anbieter auftreten sollte. Genau hier setzt die "`Neue Politische Ökonomie"' an, indem sie sich dieser Ansicht entschieden gegenüberstellt. In der der "`Theorie der öffentlichen Wirtschaft"' wird der Staat als allmächtiger und allwissender Übervater dargestellt, der stets das beste für die Gesamtheit seiner Bewohner will. In Modellen wird der Staat also stets als exogener Heilsbringer implementiert. Das stellt die "`Neue Politische Ökonomie"' vehement in Frage. Sie stellt in Aussicht, dass Marktversagen nicht immer durch die öffentliche Hand behoben werden kann, dass also auch "`Staatsversagen"' möglich ist.

Konkret untersucht sie "`Wer ist denn eigentlich der Staat?"' und "`Wie trifft dieser seine Entscheidungen?"' Schon anhand dieser beiden Fragen sieht man: \textit{Den Staat} als kollektives Entscheidungsorgan gibt es in demokratischen Gesellschaften nicht. Stattdessen gibt es viele Individuen, die über individuell-nutzenmaximierendes Verhalten ihre Volksvertreter wählen, die wiederum den Staat repräsentieren. Dieser Prozess von "`individuellen Werten"' zu "`politischen Entscheidungen"' kann nicht einfach durch die Annahme eines "`Staates"' übergangen werden \parencite[S. 11]{Buchanan1962}. Genau diesem Problem und den daraus resultierenden Fragen nahm sich die "`Neue Politische Ökonomie"' an. Es handelt sich hierbei um keine geschlossene ökonomische Denkrichtung, sondern um eine recht lose Sammlung verschiedener Ansätze. Die Neue Politische Ökonomie blickt sozusagen hinter die Kulissen der Staatseingriffe. Sie untersucht nicht, \textit{ob} Staatseingriffe in einer gewissen Situation zu befürworten sind, sondern, \textit{wie} deren Durchführung zustande kommt und erst in weiterer Folge, ob zu erwarten ist, dass durch den Eingriff schlussendlich tatsächlich das reine Marktergebnis verbessert wird.

Die Neue Politische Ökonomie behandelt damit auch ähnliche Fragen wie die Wohlfahrtstheorie (vgl. Kapitel \ref{Wohlfahrt}), steht aber gleichzeitig auch zu dieser im Spannungsverhältnis. Beide Schulen suchen nach Antworten darauf, wie der gesamtgesellschaftliche Nutzen maximal wird. Die Wohlfahrtstheorie ist dabei allerdings eher eine normative Theorie, gibt also vor wie die Dinge idealerweise sein \textit{sollen} damit maximale Wohlfahrt für die Gesellschaft entsteht. Die Wohlfahrtstheorie sucht nach einer "`gesamtwirtschaftlichen Nutzenfunktion"' ("`Social Welfare Function"'). Die Neue Politische Ökonomie sah sich, zumindest in ihren Anfängen, überwiegend als positive Theorie. Auch ihr geht es darum zu analysieren, wie in Gesellschaften nutzenmaximierende Entscheidungen getroffen werden. Allerdings ist die Neue Politische Ökonomie nüchterner was den Weg dorthin betrifft und geht davon aus, dass Wähler, Politiker und Institutionen primär ihren individuellen Nutzen maximieren, was zu gesamtgesellschaftlich nicht optimalen Ergebnissen führen kann. Sie zeichnet der sogenannte "`Methodische Individualismus"' aus, das heißt es gibt keine kollektiven Entscheidungen und keine gesamtwirtschaftliche Nutzenfunktion. Jede Entscheidung, auch die von großen Gruppen, ist im Endeffekt das Ergebnis aus einer Summe individuell-nutzenmaximierenden Einzelentscheidungen. Die Schule, die dieses Spannungsverhältnis zwischen Wohlfahrtsökonomie und Neuer Politischer Ökonomie konkret behandelt, ist die "`Social Choice Theorie"', die von \textcite{Arrow1950, Arrow1951} begründet wurde und bereits im Kapitel \ref{Wohlfahrt} behandelt wurde.

Als Vorläufer wird häufig Joseph Schumpeter genannt (\textcite[S. 519]{Snowdon2005}, \textcite[S. 95]{Warsh}), der bereits früh nach dem Erscheinen von Keynes' General Theory darauf hinwies, dass sich Politiker in Demokratien  um ihre Jobs immer wieder bei den Wählern bewerben müssen und dies ihre Handlungen - nicht immer zum Wohlergehen der Gesellschaft - beeinflusst \parencite{Schumpeter1942}. Bereits noch früher schlug der, seiner Zeit in vielen Belangen voraus gewesene, schwedische Ökonom Knut Wicksell (vgl. Kapitel \ref{Wicksell}) in eine ähnliche Kerbe. In seinen "`Finanztheoretischen Untersuchungen"' \parencite{Wicksell1896} bezeichnet er Wahlen als "`quid pro quo"'-Geschäft, also als Tauschgeschäft, in dem die Wähler von den Politikern etwas zurück bekommen wollen. Seine theoretischen Ansätze nannten \textcite[S. 8]{Buchanan1962} später als sehr inspirierendes Werk für ihre eigene bahnbrechende Arbeit.

Als moderne Begründer, bzw. verschiedene Zweige der Neuen Politischen Ökonomie werden daher heute verschiedene Ökonomen und deren Werke angeführt \parencite[S. 31]{Grofman2004} bzw. \parencite{Mitchell1988}:
\begin{itemize}
	\item \textcite{Arrow1951, Arrow1950}, der mit der Begründung der "`Social Choice Theory"' und dem darin postulierten "`Unmöglichkeitstheorem"' die grundlegenden Konzepte der damaligen Wohlfahrtstheorie, wie die Existenz einer "`Sozialen Wohlfahrtsfunktion"', widerlegte. Dieser Zweig wurde bereits in Kapitel \ref{Wohlfahrt} behandelt.
	\item \textcite{Black1948a, Black1958}, der neben Arrow als Begründer der "`Social Choice Theory"' gilt, aber erst durch dessen Arbeiten bekannt wurde. Er griff als erster Probleme beim Abstimmungsverhalten theoretisch und begründete das Medianwähler-Modell \parencite{Black1948a}.
	\item \textcite{Downs1957b, Downs1957}, der als erster demokratische Wahlen als Markt interpretierte, auf dem Politiker Wahlversprechen anbieten um die nachfragenden Wähler zu überzeugen. Diese Schule etablierte einen sehr umfangreichen Forschungszweig, der bis heute recht aktiv ist und sich damit beschäftigt, inwieweit Politiker an Regeln gebunden werden sollen \parencite[S. 523]{Snowdon2005}.
	\item \textcite{Buchanan1962} sind die vielleicht bekanntesten Vertreter der Neuen Politischen Ökonomie. Sie vertieften die Idee, dass Politiker an bestimmte Regeln gebunden werden sollen und etablierten damit die Ideen einer Wirtschaftsverfassung ("`constitutional economics"'). Gordon Tullock analysierte als erster das Problem des "`rent seekings"', also das Bestreben durch Lobbyismus selbst aufwandsloses Einkommen zu beziehen (Der Begriff selbst wurde durch \textcite{Krueger1974} geprägt).
	\item \textcite{Riker1962}, der politische Prozesses primär als spieltheoretische Anwendungsbeispiele verstand und einen beträchtlichen Beitrag zur Entwicklung des quantitativen Zweigs der Neuen Politischen Ökonomie leistete. Dieser Zweig verstand Politische Ökonomie zudem als streng positive Theorie.
	\item Das Ehepaar Ostrom erarbeitete gemeinsam mit Charles Tiebout Modelle, die zeigen, dass dezentralisierte Entscheidungen der öffentlichen Hand effizienter sein können, als zentrale Entscheidungen. Mit anderen Worten: Die öffentliche Hand sollte in nicht auf gesamtstaatlicher, sondern auf lokaler Ebene Entscheidungen treffen \parencite{Ostrom1961, Ostrom1971}.
\end{itemize}

Die Entwicklung der Neuen Politischen Ökonomie verlief nicht linear. Die genannten frühen Arbeiten von \textcite{Arrow1951}, \textcite{Black1948a} und \textcite{Downs1957} führten nicht unmittelbar zur Begründung der wissenschaftlichen Richtung. Also solche wurde sie eher ab Anfang der 1960er-Jahre wahrgenommen. Vor allem das Werk von \textcite{Buchanan1962} gilt als sehr einflussreich. Das liegt auch am Auftritt der Autoren James Buchanan und Gordon Tullock. Die beiden gelten als Begründer der "`Virginia School"' oder "`Public Choice Theory"'. Der zweitgenannte Begriff wird oft äquivalent mit dem Begriff "`Neue Politische Ökonomie"' verwendet. Ein Ansatz, dem hier nicht gefolgt wird. Als "`Public Choice Theory"' wird hier ausschließlich die Schule von Buchannan und Tullock bezeichnet. Beide waren extrem Markt-fundamental eingestellt, indem sie die Möglichkeit des "`Staatsversagens"' und des Scheiterns politischer Entscheidungsprozesse hervorhoben. Auch verließen beide bald den Weg die Neue Politische Ökonomie als positive Theorie zu betrachten \parencite[S. 105]{Mitchell1988}. Sie lehnten die Mainstream-Ökonomie vehement ab. Und zwar sowohl die makroökonomische dominierende "`Neoklassische Synthese"', wie auch die "`Neoklassik"', wobei sie hier vor allem die dort übliche mathematisch-quantitative Methodik ablehnten \parencite[S. 106]{Mitchell1988}. Vehement kritisierten sie die Ansätze der "`Wohlfahrtstheorie"' und der "`Theorie der Öffentlichen Wirtschaft"'. Letzteres mündete sogar in einer Streitschrift mit Richard Musgrave: \textcite{Musgrave1999}. Stattdessen näherte sich die "`Virginia School"' ideologisch und methodisch der "`Österreichischen Schule"' an. Dies führte allerdings dazu, dass ihre späteren Werke innerhalb der Mainstream-Ökonomie nicht akzeptiert wurden. Die Schüler von Buchanan und Tullock scheiterten weitgehend daran in der akademischen Welt Fuß zu fassen und wurden stattdessen in den 1980er-Jahren überproportional häufig zu Beratern und Mitarbeitern der wirtschaftsliberalen politischen Bewegung des späteren US-Präsidenten Ronald Reagan. Nichtsdestotrotz stellt ihr frühes Werk bis heute ein wichtiges Werk der "`Neuen Politischen Ökonomie"' dar, dass wichtige Fragen zum politischen Entscheidungsprozess  und zum "`Rent Seeking"' aufwirft und behandelt.

Ähnlich einflussreich, wenn auch weniger lautstark, war die fast parallel entstandene "`Rochester School"'. Deren früher Hauptvertreter William Riker behandelte die Neue Politische Ökonomie als streng positive Theorie. Er setzte sein Hauptwerk \parencite{Riker1962} auf den mathematisch-analytischen Arbeiten von \textcite{Black1948a} und \textcite{Arrow1951} auf. Tatsächlich erlangten die frühen Arbeiten von \textcite{Black1948a, Black1948b} erst durch Riker - und später noch verstärkt durch \textcite{Nordhaus1975} (siehe unten) - seinen heutigen Bekanntheitsgrad. Riker's vielleicht wichtigster Schritt war allerdings politische Prozesse als Anwendungsbeispiele der Spieltheorie zu sehen. Diese war Anfang der 1960er-Jahre im frühen Stadium seiner fortlaufenden Verbreitung. Mit ihrer mathematisch-quantitativen Ausrichtung der Neuen Politischen Ökonomie als positive Schule etablierte Riker eine formale und Mainstream-taugliche Schule der Politische Ökonomie.

Als extrem einflussreich erwies sich schließlich die Theorie der Politischen Konjunkturzyklen. Angestoßen wurde dieser Forschungszweig durch die Arbeit von \textcite{Nordhaus1975}. Aufbauend auf die Arbeiten von \textcite{Black1948a} und \textcite{Downs1957} argumentierte er, dass Politiker als individuell-nutzenmaximierende Individuen primär ihre Wiederwahl anstreben und daher ihre Tätigkeiten ausschließlich darauf ausrichten. \textcite{Hibbs1977} widersprach dem, da Nordhaus' Modell Politikern ja jegliche inhaltliche Überzeugung absprach und empirisch schwer haltbar schien. Stattdessen erstellte er ein "`ideologisches Modell"', wonach im Zwei-Parteien-System die Parteien jeweils ihre Präferenzen hinsichtlich makroökonomischer Variablen durchsetzen. Beide Modelle erhielten Mitte der 1970er für kurze Zeit sehr viel Zuspruch. Mit der Revolution in Folge der "`Lukas Kritik"' verschwanden die Modelle allerdings recht rasch wieder. Erst Ende der 1980er Jahre erlebte die Theorie der Politischen Konjunkturzyklen ein Revival. \textcite{Rogoff1986} und \textcite{Alesina1987} betteten die Modelle von Nordhaus bzw. Hibbs in die Theorie der rationalen Erwartungen ein. Vor allem Alberto Alesina wurde in den 1990er Jahren zu einer prägenden Figur der Neuen Politischen Ökonomie. Er brachte Geldpolitik - in der Form von Überlegungen zur Unabhängigkeit von Zentralbanken - und Fiskalpolitik - in der Form von Überlegungen zur Bedeutung von Staatsschulden - aus Sicht der politischen Ökonomie in das Zentrum der Forschung in diesem Bereich. Die Neue Politische Ökonomie behandelt - vor allem mit der Konzentration auf Geldpolitik - damit Themen, die auch in der Makroökonomie an zentraler Stelle stehen. Damit ist die Schule in diesem Bereich mitten in der Mainstream-Ökonomie angekommen.

Alesina erweiterte seine Analysen auch auf die Themen Einkommensverteilung \parencite{Hirschman1973} und politische Elemente des Wirtschaftswachstums, bevor er 2020 im Alter von 63 Jahren überraschend verstarb. Alle zuletzt genannten Bereiche zählen heute zu den aktivsten in der wirtschaftswissenschaftlichen Forschung. Allerdings spricht man in diesem Zusammenhang nicht mehr von "`Neuer Politischer Ökonomie"'. Aktuell findet man mit Dani Rodrik, Philippe Aghion, Kenneth Rogoff und Carmen Reinhart (u.a) eine ganze Liste von Ökonomen unter den meistzitierten Wirtschaftswissenschaftler, die allesamt in einem Bereich tätig sind, den man auch in der "`Neuen Politischen Ökonomie"' unterbringen könnte. Daneben entwickelte sich der "`Neue Institutionalismus"' (vgl. Kapitel \ref{sec: Neue Inst}) um Daron Acemoglu, den man auch innerhalb der Politischen Ökonomie ansiedeln könnte, auf jeden Fall aber mit diesem eng verbunden ist.

Betrachten wir nun die Entwicklungsschritte der Neuen Politischen Ökonomie im Detail.


\section{Black: Das Medianwähler-Modell}

Der Geburtsstunde der "`Public Choice Theory"' gelten heute weitgehend die frühen Arbeiten von Duncan \textcite{Black1948a, Black1948b} und Kenneth \textcite{Arrow1951}. Beide Arbeiten - vor allem jene von Arrow - sind uns schon aus Kapitel \ref{Wohlfahrt} bekannt. Gelten sie doch auch als Ausgangspunkt der "`Social Choice Theory"'. Dies ist insofern paradox als die späteren Vertreter der beiden Schulen "`Social Choice Theory"' und "`Neue Politische Ökonomie"' geradezu gegensätzliche ideologische Standpunkte vertraten.

Die Etablierung der Neuen Politischen Ökonomie verlief dabei alles andere als geradlinig. Die heute als bahnbrechend angesehenen Arbeiten von \textcite{Black1948a, Black1948b} blieben längere Zeit unentdeckt. Laut \textcite{Grofman2004} hatte Duncan Black auch Probleme seine Arbeiten in wissenschaftlichen Journalen zu platzieren. Wenn man diese frühen Arbeiten liest verwundert dies zunächst wenig. Black thematisiert die Problematiken, die im Rahmen von demokratischen Abstimmungen entstehen und verweist darauf, dass dies in ökonomischen Abhandlungen noch nicht thematisiert wurde. Er macht als eine Problem, das man wohl zunächst einmal in den Politikwissenschaften verorten würde zu einem ökonomischen Problem. Seine Zeitgenossen dürften den Zusammenhang zur Ökonomie zunächst unterschätzt haben \parencite[S. 33]{Grofman2004}. Ein weiterer amüsanter Aspekt ist, das bereits vor dem Zweiten Weltkrieg Harold \textcite{Hotelling1929} ganz ähnliche Fragen wie Duncan Black thematisiert hat und  schon im späten 18. Jahrhundert der Franzose Marie-Jean Marquis de Condorcet potentielle Probleme bei Abstimmungen identifiziert hatte, allerdings natürlich ohne den ökonomischen Zusammenhang zu thematisieren. Erst durch die Beiträge von Anthony Downs und William Riker, die in den nächsten Kapiteln behandelt werden, bekam auch Black die entsprechende Wertschätzung.
Als am bekanntesten gilt heute sein zusammenfassendes Werk \textcite{Black1958}: "`The Theory of Committees and Elections"'. Was waren die wesentlichen Erkenntnisse der Werke \textcite{Black1948a, Black1948b, Black1958}? Bereits in den früheren Werken entdeckt er das in Kapitel \ref{Wohlfahrt} angeführte Condorcet-Paradoxon (auch Paradox der zyklischen Mehrheiten) selbständig wieder, erweitert dies allerdings entscheidend um seine praktische Relevanz in politischen Entscheidungsprozessen. Erst später - und dann auch in seinem Werk \textcite{Black1958} entsprechend angeführt und aufgearbeitet - war ihm die Arbeit von \textcite{Condorcet1785} bekannt \parencite[S. 14]{Tullock1981}. Die Relevanz seiner Theorie wurde in weiterer Folge von ein großen Zahl von Ökonomen untersucht \parencite[S. 17f]{Tullock1981}.

Als noch wesentlich relevanterer Beitrag erwies sich allerdings das Medianwähler-Modell, das Black, unter der recht strikten Annahmen eingipfliger Präferenzen, in der gleichen Arbeit \parencite[S. 28ff]{Black1948a} ableitete. Dazu definiert \textcite[S. 26]{Black1948a} zunächst die "`Eingipfligkeit"'. Dazu muss jeder Wähler zu den verschiedenen Alternativen einer Abstimmung eine Rangordnung vornehmen, die seine Präferenz eindeutig anzeigt. Ein Beispiel verdeutlicht dies. Angenommen ein Wähler muss seine Präferenzen zur Bereitstellung eines öffentlichen Gutes durch den Staat angeben. Es gibt drei Alternativen. Ein Kommunist würde wohl folgende Rangordnung bevorzugen: 1. Viele öffentliche Güter, 2. durchschnittlich viele öffentlichen Güter, 3. keine öffentlichen Güter. Ein liberaler Bürger würde wahrscheinlich eine genau gegenteilige Rangordnung angeben. Und eine moderater Bürger würde vielleicht meinen, dass ihm durchschnittlich viele öffentliche Güter am liebsten sind, da er beide extreme ablehnt. Alle drei genannten Bürger haben eine eindeutige Präferenz der Rangordnung. Man spricht eben von eingipfligen Präferenzen. Ein vierter Bürger, der die "`Extreme liebt"' und am liebsten Viele öffentliche Güter bereitgestellt bekommt und am zweitliebsten gar keine öffentlichen Güter und erst an dritter Stelle seiner persönlichen Rangordnung  durchschnittlich viele öffentlichen Güter angibt, hat hingegen zweigipfelige Präferenzen. Da in der Realität die Anzahl der Alternativen nicht auf drei beschränkt ist, sprich man in diesem Fall von mehrgipfeligen Präferenzen. Der entscheidende Punkt ist nun, dass \textcite[S. 27]{Black1948a} mathematisch zeigen kann, dass - unter der Voraussetzung, dass alle Bürger eingipfelige Präferenzen aufweisen - demokratische Wahlen in größeren Gesellschaften (das heißt ohne größere Absprachemöglichkeiten) stets dazu führen, dass als demokratisches Ergebnis die "`mittlere"' Alternative gewählt wird. Die mittlere Alternative ist jene, bei der 50\% der Wähler eine "`Idealalternative"' haben, bei der - wenn man im oben angeführten Beispiel bleibt -  mehr öffentliche Güter bereitgestellt würden und umgekehrt. Dieses formale Ergebnis hatte - mit etwas zeitlicher Verzögerung - enorme Auswirkung auf die ökonomische Forschung. Die zentrale Voraussetzung, nämlich dass Individuen eingipflige Präferenzen haben, erscheint intuitiv als recht verständlich. Ab den späten 1960er-Jahren gab es viele Forschungsarbeiten dazu, ob die Voraussetzungen des Medianwähler-Modells in demokratischen Prozessen erfüllt sind \parencite[S. 21ff]{Tullock1981}. Obwohl die Ergebnisse dazu nicht eindeutig sind, ist das Modell bis heute Standard in vielen Ökonomie-Lehrbüchern. Die Implikation des Medianwähler-Modells ist, dass sich in einem Zwei-Parteien-System beide um den Medianwähler in der Mitte bemühen und ideologische Unterschiede zwischen den beiden Wahl-werbenden Gruppen daher zunehmend verschwimmen. Eine Annahme, die bis heute umstritten ist und die in Unterkapitel \ref{Der Politische Wirtschaftszyklus} zentral für die Gültigkeit eines der dort vorgestellten Modelle ist.

An dieser Stelle ist außerdem noch ein Hinweis zu Zwei-Parteien- und Mehr-Parteien-Systemen angebracht. Heute verbindet man erstere vor allem mit dem Angel-sächsischem Raum, wo es ja, aufgrund des Mehrheitswahlrechts, im wesentlichen tatsächlich nur zwei große Parteien gibt. In Kontinentaleuropa hingegen gibt es seit jeher Verhältniswahlrecht und dementsprechend auch wesentlich mehr Parteien. In den 1950er-Jahren, als die Arbeiten von \textcite{Black1948a, Black1958} zunehmend bekannt wurden, gab es aber auch in Kontinentaleuropa in den meisten Staaten nur zwei bedeutende politische Großparteien. Die häufig als "`das dritte Lager"' bezeichnete Parteien, sowie Öko-Parteien, kamen erst ab den 1980er Jahren zu bedeutenden Stimmenanteilen.


\section{Downs: Wahlergebnisse als Marktergebnisse}

Die Arbeiten von Duncan Black blieben zunächst recht unbekannt. Die ökonomische Schule der "`Neuen Politischen Ökonomie"' bekam Ende der 1950er so richtig Auftrieb. Anthony \textcite{Downs1957b, Downs1957} mit seinen einflussreichen Werken zur ökonomischen Theorie politischer Handlungen in Demokratien, kritisierte, dass Staatseingriffe in der ökonomischen Theorie stets als exogene Variable herangezogen werden, ohne zu Bedenken, dass Staatseingriffe von nutzenmaximierenden Personen vollzogen werden müssen \parencite[S. 135]{Downs1957}. Er lieferte in der Folge ein Modell, dass staatliche Handlungen endogenisierte, indem er individuell-nutzenmaximierende Politiker als Umsetzer der Staatseingriffe berücksichtigte. \textcite[S. 137]{Downs1957} stellt dazu die Hypothese auf, dass Politiker ihre Tätigkeit einzig und alleine darauf ausrichten, maximal viele Stimmen zu erhalten. Politikern geht es weder um Parteiprogramme, noch um Interessenvertretungen, sie bedienen sich derselben nur um ihr einziges Ziel die (Wieder-)Wahl zu erreichen. Die Durchführung politischer Handlungen ist dabei nur ein Nebenprodukt ihrer privaten Nutzenmaximierung: Hohes Einkommen, Prestige und Macht. Das hört sich zunächst recht negativ an. Doch es ist in Wirklichkeit nichts anderes als politisches Verhalten als ökonomisch-marktwirtschaftliches Verhalten zu interpretieren. Ein Politiker handelt eben nicht anders wie alle anderen Menschen auch. So schürft ein Arbeiter in einer Kohle-Mine ja auch nicht deswegen nach Kohle, um die Gesellschaft mit Energie zu versorgen, sondern um damit Geld zu verdienen \parencite[S. 137]{Downs1957}. Das entsprechende Vorgehen von Politiker ist daher \textit{nicht grundsätzlich} verwerflich, ganz im Gegenteil es widerspricht nicht dem gesamtwirtschaftlichen Ziel die "`Soziale Wohlfahrt"' zu maximieren. In einer Welt mit perfekter Information werden die Wähler jene Partei wählen, die ihnen den höchsten persönlichen Nutzen in der zukünftigen Wahlperiode liefert. Dieser erwartete Nutzen wird aus den bisherigen Handlungen der Politiker - sowohl jener, die regieren, als auch jener, die in Opposition sind - abgeleitet. Aufgrund der perfekten Information der Wähler scheint unredliches (verwerfliches) Verhalten der Politiker nicht sinnvoll, weil dadurch Wählerstimmen verloren gehen. Erst durch die realistischere Annahme unvollständiger Information beider Seiten - die Wähler wissen nicht vollständig, welche Handlungen welche politische Partei liefern wird, aber auch die politischen Parteien wissen nicht vollständig, welche Handlungen die Wähler sehen wollen - bringt kostenintensiver Wahlkampf für Politiker zusätzliche Stimmen. Außerdem bringt Lobbyismus bei unvollständiger Information bestimmten Wählergruppen Einfluss auf die Handlungen von Bewerbern um politische Positionen. Damit verbunden ist auch die Möglichkeit, dass Bestechung und anderes unredliches Verhalten zu rationalem Verhalten wird. Unvollständige Information bei politischen Entscheidungen ist also \textit{der} Hauptfaktor im Rahmen von politischen Entscheidungsprozessen. Daraus leitet \textcite[S. 141]{Downs1957} ab, dass politische Parteien bestimmte Ideologien verfolgen. Diese können als eine Art "`Signaling"' (vgl. Kapitel \ref{Info}) verstanden werden: Dem Wähler wird rasch gezeigt, welche Motive die Partei verfolgt, ohne dass sich der Wähler kostenintensiv diese Information besorgen muss.

In weiterer Folge greift \textcite[S. 142]{Downs1957} das Medianwähler-Modell auf, das im vorherigen Kapitel dargestellt wurde. Es ist sogar so, dass dieses Modell erst durch die Arbeiten von \textcite{Downs1957, Downs1957b} den großen Bekanntheitsgrad erfuhr, den es bis heute unter Ökonomen genießt. Interessanterweise wurden später vor allem die Ausführungen von \textcite{Black1948a, Black1948b} bekannt, während sich Downs selbst auf das ältere Werk von \textcite{Hotelling1929} bezog. Allerdings führt \textcite[S. 142]{Downs1957} zusätzliche Annahmen ein, die dazu führen, dass sich in einem Zwei-Parteien-System im Medianwähler-Modell die beiden Parteien nicht notwendigerweise im Zentrum aneinander annähern müssen um eben den Medianwähler und damit die Mehrheit von sich zu überzeugen. Dies ist dann der Fall, wenn die Verteilung der Wähler im Spektrum von "`links-radikal"' - also überzeugte Kommunisten -  bis "`rechts-radikal"' - also Befürworter eines Nachtwächter-Staates - nicht eingipfelig, wie bei etwa bei einer Normalverteilung verläuft, sondern zwei- oder mehrgipfelig. Zum Beispiel könnte es in einer gespaltenen, radikalisierten Gesellschaft die meisten Wähler jeweils an den Rändern geben. In solchen Situationen besinnen sich die Parteien ihrer Ideologie und werden nicht zu "`Parteien der Mitte"'. Dies war ein wichtiger, wenn auch letztlich bis heute umstrittener, Ansatz. Schließlich lässt sich die in der Folge in den 1970er Jahren entstandene Forschung zum Politischen Wirtschaftszyklus einteilen in einen Zweig, der "`Opportunistisches Verhalten"' der Parteien unterstellt und damit eine Annäherung der großen Parteien an die Mitte wie im Medianwähler-Modell, und einem zweiten Zweig, der von "`Ideologischen Parteien"' ausgeht. Diese Modelle werden weiter unten in diesem Kapitel (vgl. Unterkapitel \ref{Der Politische Wirtschaftszyklus}) vorgestellt. Zuvor betrachten wir - der chronologischen Entwicklung folgend - die Arbeiten der "`Public-Choice"'-Vertreter im engeren Sinn. 

\section{Buchanan: Die Public-Choice Theorie}
\label{Pol_Econ}

Eine neue Richtung verliehen der Theorie der Neuen Politischen Ökonomie in den 1960er Jahren James McGill Buchanan und Gordon Tullock. Vor allem dem stark polarisierenden Buchanan wird heute die Rolle zugeschrieben der Begründer Der "`Public Choice Theory"' im engeren Sinn zu sein. Er drückte ihr den Stempel auf, eine extrem wirtschaftsliberale ökonomische Schule zu sein. Und das, obwohl er akzeptierte, dass Märkte in vielen Fällen versagen können. Noch stärker war allerdings seine Überzeugung, dass "`der Staat"' in diesen Fällen keine bessere Lösung anbieten kann als die versagenden Märkte. Er war die längste Zeit seiner akademischen Karriere an Universitäten in Virginia tätig und gründete dort das "`Center for the Study of Public Choice"'. Fast zeitgleich begründete Gordon Tullock das wissenschaftliche Journal "`Papers in Non-Market Decision-Making"', das bald in "`Public Choice"' unbenannt wurde \parencite[S. 101]{Mitchell1988} und unter diesem Namen bis heute veröffentlicht wird. Obwohl Buchanan nie wirklich dem ökonomischen Mainstream zuzurechnen war, was er auch selbst stets betonte \parencite[S. 96]{Warsh}, hatten seine Ideen einen prägenden Einfluss auf die Wirtschaftswissenschaften und die Wirtschaftspolitik. 

Die bisher genannten Arbeiten von \textcite{Black1948a} und \textcite{Downs1957} waren interessante Weiterentwicklungen der Wirtschaftswissenschaften, weil sie Probleme der Politikwissenschaften mit ökonomischen Methoden behandelten und so politische Prozesse in die Ökonomie eingliederten. Allerdings standen ihre Arbeiten weitgehend neben der traditionellen Mikro- und Makroökonomie, ohne viel Überschneidungspunkte aufzuzeigen. Dies änderte sich durch die Arbeit von \textcite{Buchanan1962}. Diese behandelte Abstimmungsprozesse nicht alleinstehend, sondern erkannte, dass erst die rollierende Abfolge von Abstimmungsprozessen zu interessanteren ökonomischen Ergebnissen führt. Dabei spielen die damals jungen Erkenntnisse der Spieltheorie (vgl. Kapitel \ref{Spieltheorie}) eine große Rolle. Vor allem der "`Staatsbegriff"' der damaligen Mainstream-Ökonomie wurden dadurch in Frage gestellt.

Durch die Keynesianische Revolution spielte aktive Wirtschaftspolitik in den 1950er und 1960er-Jahren eine große Bedeutung in der Makroökonomie (vgl. Kapitel \ref{Synthese}). In der Mikroökonomie war dies ähnlich: Die Wohlfahrtsökonomie (vgl. Kapitel \ref{Wohlfahrt}) und die Theorien der Public Economy (vgl. Kapitel \ref{cha: Marktversagen}) waren viel beforschte Themenbereiche. In beiden Fällen übernahm der Staat die Position eines wohlwollenden, gütigen Despoten. Dabei blieb stets die Frage unbehandelt, ob denn der Staat, beziehungsweise dessen Repräsentanten, überhaupt besser als der Markt ungewollten Entwicklungen, erstens, gegensteuern \textit{kann}, und zweitens, \textit{will}. In dieser Zeit, in der wenige Ökonomen daran zweifelten, dass staatlich Eingriffe der wirtschaftlichen Entwicklung schaden könnten, griff \textcite{Buchanan1962} genau dieses Thema auf. Die beiden Autoren erhoben laute und uneingeschränkte Zweifel daran, dass Staatseingriffe die Rettung bei ökonomischen Missständen seien. Dem Begriff des Marktversagen stellten sie den Begriff des Staatsversagens gegenüber. Die zentrale Ausgangsthese lautete dabei, dass sich der Staat ja nichts anderes sei als die Summe seiner Einwohner. Und seine Repräsentanten und Bürokraten streben auch nach nichts anderem als individueller Nutzenmaximierung. Sehr eingängig verdeutlichte dies ein Schüler Buchanan's: "`Angenommen ein Finanzexperte wechselt von der Wallstreet ins Finanzministerium. Die Mainstream-Ökonomie geht davon aus, dass diese Person schlagartig vom Saulus zum heiligen Paulus wird, also seine bisher ausgelebte individuelle Nutzenmaximierung aufgibt und fortan selbstlos als Staatsdiener arbeitet."' \parencite[S. 96]{Warsh}. In den normativen Schulen der Public Economy (vgl. Kapitel \ref{cha: Marktversagen}) und der Wohlfahrtstheorie (vgl. Kapitel \ref{Wohlfahrt}) gingen die Ökonomen davon aus, dass sie wussten was am besten für eine Gesellschaft wäre und dass Entscheidungsträger dieses Wissen nutzen würden um die ökonomischen Gegebenheiten in einem Staat zu verbessern \parencite[S. 167]{Romer1988}. Dem stellte die positive Theorie der Neuen Politischen Ökonomie entgegen, dass es ausschließlich \textit{individuell}-nutzenmaximierende Personen gäbe und sich aus der Summe ihrer Entscheidungen im Rahmen von demokratischen Spielregeln die ökonomischen Gegebenheiten in einem Staat ergeben. Wie bereits erwähnt akzeptieren \textcite{Buchanan1962}, dass manche Produkte und Dienstleistungen auf rein privaten Märkten nicht angeboten werden. Die Entscheidung in welchem Umfang diese öffentlichen Güter und Dienstleistungen jetzt angeboten werden, wird aber nicht dem "`Staat"' als exogenen Entscheidungsträger überlassen, stattdessen gibt es individuell nutzenmaximierende Individuen, die ihre demokratischen Stimmrechte einsetzen um Regeln festzulegen, wie die öffentlichen Güter und Dienstleistungen angeboten werden. Dem exogenen Staat weicht also ein privatwirtschaftlicher Markt für politische Abstimmungen. Die Public Choice Theory folgt also einem radikalen Subjektivismus, der unterstellt, dass nur Individuen selbst wissen können, was am besten für sie ist. Dieses Wissen setzen sie in politischen Abstimmungsprozessen ein. 
\textcite{Buchanan1962} suchen daher nach einem demokratischen Prozess für die optimale Entscheidungsfindung in diesen politischen Abstimmungsprozessen. Sie wägen dabei zwei Kosten ab. Erstens, die Kosten, die Individuen erleiden, weil im politischen Abstimmungsprozess ihre Präferenzen überstimmt werden. Und zweitens, die Kosten der Entscheidungsfindung. Dabei kommt es zu folgendem Trade-Off: Verlangen die politischen Abstimmungsprozesse Einstimmigkeit, so kommt es zwar als Folge der einzelnen Abstimmungen zu keinerlei externen Kosten, weil ja jeder Abstimmungsberechtigte zugestimmt hat, allerdings sind die Kosten der Entscheidungsfindung enorm, weil ja jeder einzelne überzeugt werden muss. Wenn umgekehrt ein Diktator alleinig entscheiden kann, kommt es zu keinerlei Entscheidungskosten, dafür aber zu enormen externen Kosten, weil ja die individuellen Präferenzen der anderen Personen allesamt unbeachtet blieben. \textcite{Buchanan1962} argumentieren, dass selbst ein wohlwollender Diktator - als solchen könnte man den exogenen Staat, der im Sinne der Public Economy School stets das Beste für alle Individuen will, interpretieren - keine optimalen Entscheidungen treffen, weil er schlicht die individuellen Präferenzen nicht kennt. Das heißt, rationale Individuen werden eine optimale Entscheidungsregel wählen, bei der die Summe aus Kosten der Entscheidungsfindung und die externen Kosten, die entstehen weil Abstimmungen nicht wie erwünscht ausgegangen sind, minimiert wird \parencite[S. 71]{Buchanan1962}. \textcite{Buchanan1962} argumentieren damit schon gegen den Ansatz der damaligen Mainstream-Ökonomen öffentlichen Güter vom "`Staat"' zur Verfügung stellen zu lassen, indem sie zeigen, dass der Staat als solcher erst definiert werden muss. Die beiden sind sich aber auch bewusst, dass sie mit ihrer Analyse der optimale Entscheidungsregel das Problem wie öffentliche Güter nun zur Verfügung gestellt werden sollen nur auf eine höhere Ebene geschoben haben. Denn wie soll denn nun eine optimale Entscheidungsregel genau aussehen und vor allem, wie sollen sich individuell-nutzenmaximierende Wähler auf eine solche einigen? Diesbezüglich liefern sie einen interessanten aber umstrittenen Ansatz. Sie argumentieren, dass unterschiedliche individuelle Präferenzen bei \textit{einzelnen} Entscheidungen sehr unterschiedlich ausgeprägt sein können und ein demokratischer Abstimmungsprozess im Einzelfall zu keinen befriedigenden Ergebnissen führt. Viel einfacher ist es hingegen durch demokratische Prozesse, langfristige, allgemeingültige Gesetze zu schaffen, weil bei der Abstimmung darüber die Wähler nicht von kurzfristig nutzenmaximierenden Verhalten geblendet sind \parencite[S. 78]{Buchanan1962}. Die Aufgabe des Staates wäre es dennach vor allem die allgemeinen Entscheidungsregeln möglichst genau zu formulieren. Mit anderen Worten: Durch demokratische Wahlen eine "`Wirtschaftsverfassung"' festzulegen, die Regierungen und Bürokraten in der weiteren Folge bei Einzelentscheidungen nur einen engen Handlungsspielraum lässt. Dieser Ansatz ist eingängig und lässt sich mit einem einfachen Beispiel verdeutlichen: So macht es tatsächlich keinen Sinn alle Individuen eines Staates darüber abstimmen zu lassen, ob in einer Stadt im Osten des Landes ein Kongresszentrum gebaut werden soll, weil die Individuen im Rest des Landes daraus keinerlei Nutzen ziehen. Jede demokratische Abstimmung darüber würde daher zuungunsten des Kongresszentrums ausgehen. Stattdessen sollte in Gesetzen genau festgelegt werden, unter welchen Umständen der Bau von Gebäuden, Straßen, etc. der Bevölkerung des Staates insgesamt mehr Nutzen bringt als Kosten verursacht, und die Entscheidung über deren Bau alleine auf Grundlage dieser Gesetze vorgenommen werden. 

In weiterer Folge legen \textcite[S. 110]{Buchanan1962} ihre Erkenntnisse auf die damals schon etablierte Public Economy (vgl. Kapitel \ref{cha: Marktversagen}) und die Wohlfahrtsökonomie (vgl. Kapitel \ref{Wohlfahrt}) um. Die beiden Autoren kritisieren, dass in diesen beiden Schulen externe Effekte als Marktversagensform eine große Rolle spielen, aber gleichzeitig noch niemand festgestellt hat, dass bei Abstimmungen, bei denen keine Einstimmigkeit gefordert wird, immer ebenso externe Effekte auftreten. Schließlich erleidet jede Person, deren präferierte Option bei Abstimmungen nicht zum Zuge kommt, externe Kosten, in Form der Nutzen-Differenz zwischen der realisierten Option und der präferierten Option. Daraus folgern \parencite[S. 110]{Buchanan1962}, dass die Einstimmigkeitsregel die einzige Entscheidungsregel ist, die das Kriterium des "`Pareto-Optimums"', das ja eine entscheidende Rolle in der Wohlfahrtstheorie spielt, erfüllt. Die Anwendung dieser Regel verursacht aber enorme Entscheidungsfindungskosten und ist bekanntermaßen aus diesem Grund nicht effizient. Dies ist ein erster Angriff auf die Wohlfahrtsökonomie und auf die Public Economy, die vor allem Buchanan in weiterer Folge zeitlebens stark kritisieren wird. Die Idee der externen Kosten bei Abstimmungen für die "`Verlierer"' in Abstimmungsprozessen verfolgen \textcite[S. 115]{Buchanan1962} weiter und schließen daraus, dass politische Einheiten - also etwa Gebietskörperschaften - eher klein gehalten werden und außerdem möglichst homogen sein sollten. In kleineren politischen Einheiten, in welchen außerdem Konsens in vielen Themenbereichen herrscht, sind die externen Kosten bei demokratischen Abstimmungen geringer als in großen und/oder heterogenen Gemeinschaften. Eine Aussage, die allerdings durchaus problematisch ausgelegt werden kann und daher höchst umstritten ist. In weiterer Folge stellen \textcite[S. 117]{Buchanan1962} dar wie demokratische Entscheidungen im Detail gefällt werden. In den Mittelpunkt rückt hierbei das "`Logrolling"', das man ins Deutschen am ehesten mit "`politischen Stimmentausch"' oder "`Gefälligkeitstausch "`übersetzen könnte. \textcite{Tullock1959} hatte diesen Begriff schon wenige Jahre zuvor analysiert: Wenn politische Abstimmungen nicht als einzelne Events gesehen werden, sondern als rollierende Abfolge von Wahlen, wird Logrolling attraktiv. Politische Gruppen, die für ihre Interessen alleine keine Mehrheiten finden, können so Koalitionen mit anderen politischen Gruppen bilden. Im gegenseitigen Austausch der Stimmen, können so durch Mehrheitsabstimmungen Entscheidungen herbeigeführt werden, die eigentlich keine Mehrheiten repräsentieren. Noch komplizierter wird die Analyse unter Anwendung spieltheoretischer Überlegungen \parencite[S. 149]{Buchanan1962}. Insgesamt führen Mehrheitsabstimmungen häufig zu nicht-optimalen Entscheidungen. Es muss daher stets eine Abwägung vorgenommen werden, ob in bestimmten Bereichen der öffentlichen Wirtschaft Mehrheitsentscheidungen durch Einstimmigkeitsentscheidungen oder Privatisierungen - um beide Extreme zu nennen - ersetzt werden soll \parencite[S. 172]{Romer1988}.

Aufbauend auf das Problem des "`Logrollings"', bei dem stimmen zwischen politischen Einheiten gehandelt werden, entwickelte \textcite{Tullock1967} das Konzept des "`Rent-Seekings"', wobei der Begriff selbst erst später von \textcite{Krueger1974} verwendet wurde \parencite[S. 231]{Congleton2004}. Hierbei werden politische Entscheidungen gekauft. Das heißt, Interessensgruppen versuchen Politiker dazu zu bringen, Entscheidungen so zu fällen, dass sie selbst finanziell davon profitieren, ohne aber dafür neuen Wohlstand zu generieren. Dies kann zum Beispiel dadurch erfolgen, dass der Interessensgruppe durch ein Gesetz eine Monopolstellung auf einem bestimmten Markt eingeräumt wird. Dies führt nicht nur zu ungerechtfertigten Gewinnen bei den Vertretern der Interessensgruppe, sondern ist auch ökonomisch ineffizient und führt daher zu einem gesamtwirtschaftlichen Wohlstandsverlust \parencite[S. 228]{Tullock1967}. Im später geprägten Tullock-Paradoxon stellt der Ökonom fest, dass Rent-seeking den begünstigten Interessensgruppen weit weniger Kosten verursacht, als Gewinne einbringt. Was wiederum die Frage aufwirft, warum rationale Politiker für ihr Stimmverhalten nicht mehr von den Gewinnen verlangen \parencite[S. 217]{Congleton2004}. Insgesamt hat \textcite{Tullock1967} damit einen großen Forschungszweig angestoßen. Das daraus entstandene Gebiet des Lobbyismus ist heute aktueller denn je. Auch in der heute wieder aktuellen Ungleichheitsdebatte (vgl. Kapitel \ref{Ungleichheit}) spielt Rent-seeking eine Rolle und zwar in der Form, dass die wohlhabendsten Personen die Gesetzgebung stark zu ihren Gunsten beeinflussen.

Ebenfalls Mitte der 1960er-Jahre entwickelte \textcite{Buchanan1965} die Neue Politische Ökonomie in eine andere Richtung weiter. Mit "`An Economic Theory of Clubs"' baute er seine Alternative zur Wohlfahrtsökonomie und der Public Economy aus. Vor allem \textcite{Samuelson1954, Samuelson1955} etablierte ja schon zehn Jahre davor die Theorie der Öffentlichen Güter (vgl. Kapitel \ref{Offentliche Guter}) mit unzähligen Anwendungsfällen. Dieser Theorie setzte Buchanan den Begriff der "`Klubgüter"' entgegen. Während Öffentliche Güter durch ihre "`Nicht-Rivalität"' und die "`Nicht Ausschließbarkeit"' im Konsum definiert sind, gilt für Klubgüter nur der erstere Begriff. Durch die künstliche Schaffung von Klubs ist es eben doch möglich Personen von der Nutzung dieser Güter auszuschließen. Es ist dann nicht mehr notwendig, dass der Staat das Angebot an öffentlichen Gütern schafft, stattdessen agieren Klubs bei \textcite{Buchanan1965} als Institutionen, die die Zahlungsbereitschaft der potentiellen Kunden ermitteln und über den "`Klubbeitrag"' einen fairen Preis für das ursprünglich öffentlichen Gut einheben \parencite[S. 244]{Marciano2021}. \textcite{Buchanan1959} kritisierte bereits früher das Konzept der Sozialen Wohlfahrtsfunktion von \textcite{Samuelson1954} als problematisch. Zwar sei dieses formal schön dargestellt, allerdings sei er auch ein normativer Ansatz, der Werturteile darüber voraussetzt, was das "`beste für die Gesellschaft"' ist. \textcite{Buchanan1965} forderte stattdessen, dass die Zahlungsbereitschaft jeder Person für ein öffentliches Gut ermittelt werden sollte und gegen diesen Betrag die öffentlichen Güter der Person zur Verfügung gestellt werden. Ein eleganter Ansatz: Wenn jeder seinen individuellen Preis bezahlt, gibt es keine externen Kosten und das von der Wohlfahrtstheorie geforderte Pareto-Optimum ist erreicht. Problematisch ist allerdings, dass das zentrale Problem der Theorie der öffentlichen Güter dabei ungelöst bleibt: Wie bringt man Individuen dazu ihre wahre Zahlungsbereitschaft bekannt zu geben? Aus diesem Problem schlossen \textcite{Musgrave1939} und \textcite{Samuelson1954} ja überhaupt erst auf die Notwendigkeit staatlicher Eingriffe bei öffentlichen Gütern. Die nicht vollkommen befriedigende Antwort von \textcite{Buchanan1965} war eben die Gründung von Klubs. In kleinen Klubs würden Individuen kooperieren, also ihre wahren Präferenzen offenbaren und damit den fairen Preis für die Klubgüter bezahlen. Das Trittbrettfahrer-Problem würde in diesen kleinen Klubs dann nicht auftreten, wenn als Folge davon der Ausschluss aus dem Klub droht \parencite{Buchanan1965}. Als Klubgüter werden heute oft Mautstraßen, oder die Mitgliedschaft bei Fußballvereinen genannt. Beides mäßig geeignete Beispiel, schließlich wollte \textcite{Buchanan1965} zeigen, dass öffentliche Güter nicht unbedingt vom Staat angeboten werden müssen, sondern stattdessen privatwirtschaftlich organisierte Institutionen den Zugang zum grundsätzlich öffentlichen Gut regeln. Noch allgemeiner aber wollte Buchanan eine Alternative zur Wohlfahrtsökonomie und der darin verankerten Sozialen Wohlfahrtsfunktion im Sinne von \textcite{Samuelson1954, Samuelson1955} liefern.

\textcite{Buchanan1962} folgen in ihrem "`Calculus of Consent"' noch recht streng der Prämisse eine positive Theorie zu sein, oder weisen zumindest darauf hin, wenn ausnahmsweise normative Ansätze dargestellt werden. Auch die Errungenschaften der beiden Mitte der 1960er-Jahre - Das Rent-seeking von Tullock, sowie die Entwicklung der Klubgüter durch Buchanan - wurden in der wissenschaftlichen Community aufgenommen und haben sich bis heute dort verankert. In den späteren 1970er-Jahren lieferten die beiden hingegen zunehmend unkonventionelle Beiträge. So publizierte Tullock Arbeiten zur Ökonomie in Diktaturen, ein damals fast verbotenes Thema wie \textcite[S. 215]{Congleton2004} meint. Bekannter aber ebenso unkonventionell wurden die späteren Arbeiten seines Partners, wie zum Beispiel \textcite{Buchanan1977} oder \textcite{Buchanan1978}. In diesen fällt vor allem durch die radikale Ablehnung jeglicher Staatseingriffe auf. Auch können diese Arbeiten eindeutig nicht mehr als rein positivistisch gesehen werden \parencite[S. 105]{Mitchell1988}. Es waren so auch diese späteren Arbeiten, die das Bild von James McGill Buchanan als weitgehend kontroversen und extrem libertären Ökonomen etablierte. Da Politiker und Bürokraten primär ihren eigenen Nutzen maximieren, können von ihnen gesteuerte fiskalpolitische Eingriffe nicht zur gesamtwirtschaftlichen Wohlfahrtsmaximierung beitragen. Die individuelle Nutzenmaximierung von Politikern wird außerdem dazu führen, dass diese immer mehr Macht anstreben und daher danach streben die Staatsquote ständig zu erhöhen. Dies führt auch dazu, dass Unternehmen im Staatsbesitz extrem wichtig sind als Auftraggeber für privatwirtschaftliche Unternehmen, was wiederum zur Folge hat, dass diese zunehmend Einfluss auf politische Entscheidungsprozesse im Sinne des Rent-seekings vornehmen müssen um Aufträge zu erhalten. Buchanan ist also überzeugt davon, dass der Staat möglichst schlank gehalten werden muss. Keynesianischen Eingriffe lehnt er vollends ab, wobei seine Wortwahl nicht zimperlich ist, wenn er in \textcite{Buchanan1978} davon spricht, dass keynesianische Wirtschaftspolitik eine Krankheit ist, die langfristig das Bestehen der Demokratie gefährdet. Einflussreich ist auch seine Einstellung zu Budgetdefiziten, die er in jeglicher Form ablehnt. Laut \textcite{Buchanan1978} führte die Umsetzung der keynesianischen Ideen dazu, dass ständige Budgetdefizite als normal betrachtet werden und Politiker nicht mehr nach ausgeglichenen Budgets streben. Um die negativen Folge von Staatsverschuldung zu vermeiden, sollten Politiker mittels enger verfassungsrechtlicher Vorgaben daran gehindert werden defizitäre Budgets zu verursachen \parencite[S. 172]{Pressman1999}. Genau diese Idee wird heute in vielen westlichen Staaten als "`Schuldenbremse"' oder "`Schuldenobergrenze"' praktiziert. Im Gegensatz zur damals ebenfalls aufkommenden "`Neuen Klassischen Makroökonomie"' (vgl. Kapitel \ref{Neue Makro}), akzeptiert die "`Public Choice"'-Schule das Auftreten von Marktversagen. Die Lösung liegt aber nicht in Staatseingriffen, sondern gesetzliche Rahmenbedingungen, die das Funktionieren von Märkten fördern. Diese Rahmenbedingungen sollten durch enge verfassungsmäßige Regelungen geschaffen werden. Auch dieser Ansatz ist modernen Vertretern der "`Neuen Politischen Ökonomie"' (siehe dieses Kapitel weiter unten) und des "`Neuen Institutionalismus"' (vgl. Kapitel \ref{Neue Institut}) nicht unähnlich. Dieses sehen auch in "`zu wenig Wettbewerb"' ein aktuell zentrales Problem.

Aus wissenschaftlicher Sicht waren diesen späten Arbeiten von Buchanan weniger bedeutend. Allerdings hatte \textcite{Buchanan1978} große politische Wirkung. In den USA der frühen 1980er-Jahre gewannen libertäre Ideen wie der Ruf nach einem schlanken Staat, Steuersenkungen und Nulldefiziten enorm an Zulauf, die in der Präsidentschaft von Ronald Reagan mündeten. Obwohl Buchanan selbst Reagan heftig kritisierte \parencite[S. 177]{Romer1998}, weil dieser das Ziel eines ausgeglichenen Budgets nicht konsequent genug verfolgte, wird der Schule und insbesondere den Schülern Buchanan's großer Einfluss auf die Reagan-Präsidentschaft zugeschrieben. Tatsächlich waren viele Absolventen der "`Virginia School"' in der Reagan-Regierung aktiv \parencite[S. 96]{Warsh}. Die Hauptvertreter der "`Public Choice"'-Schule, James McGill Buchanan und Gordon Tullock, waren aus heutiger Sicht extrem unkonventionelle Ökonomen. Entgegen dem Zeitgeist verzichteten beide fast gänzlich auf hoch-mathematische Modelle, aber auch auf empirische Arbeiten. \textcite[S. 215]{Congleton2004} etwa meinte, dass Gordon Tullock geniale Ideen entwickelte, aber diese bewusst nicht besonders tief analysierte, sondern rein verbal formulierte und anschließend ohne viel Feinschliff rasch publizierte. \textcite[S. 178]{Romer1988} vertritt eine ähnliche Meinung zu Buchanan, wenn er schreibt, dass Buchanan ausschließlich Theorien erstellte und keine einzige empirische Arbeit verfasst.  Überraschend und auch sehr umstritten war die Verleihung des Wirtschafts-Nobelpreises an ihn im Jahr 1986. Der erzkonservative Buchanan war von 1984-1986 Präsident der ebenso erz-liberalen "`Mont Pelerin Gesellschaft"'. Unter anderem die Verleihung des Preises an Buchanan brachte dem Wirtschaftsnobelpreis-Komitee den Vorwurf ein, liberale Ökonomen zu bevorzugen und sogar vorwiegend solchen, die Mitglied der umstrittenen "`Mont Pelerin Gesellschaft"' sind, den Preis besonders häufig zu verleihen. Dazu kommt, dass Buchanan's Arbeiten zwar ganz bestimmt eine Forschungslücke identifizierten und auch gewisse Beachtung fanden, allerdings in der Community der führenden Ökonomen eher ignoriert wurden. Er unterrichtete weder an einer der Top-Wirtschaftsschulen in den USA und verweigerte - wie bereits erwähnt - seine Thesen in hoch-mathematische Theorien zu gießen \parencite[S. 173]{Pressman1999}. Überraschend war auch, dass sein kongenialer Partner Gordon Tullock bei der Nobelpreis-Vergabe unberücksichtigt blieb, was dieser selbst stark beklagte \parencite[S. 98]{Warsh}.

Innerhalb der Neuen Politischen Ökonomie übernahmen bald andere Köpfe die führenden Rollen. Die Forschung zum "`Politische Wirtschaftszyklus"' (vgl. Kapitel \ref{Der Politische Wirtschaftszyklus}) untersucht zwar ebenso individuell-nutzenmaximierende Wähler und Stimmen-maximierende Politiker, hat aber sonst nur mehr sehr wenig gemeinsam mit der hier dargestellten "`Public Choice Theory"' im engeren Sinn. Davor werfen wir aber noch einen Blick auf die "`Rochester-School"' und die Lehren der "`Ostroms"', die zeitlich fast parallel zur "`Public-Choice-Theory"' entwickelt wurden.

\section{Alternative Ansätze der Neuen Politischen Ökonomie: Riker, Olson und die Ostroms }

Ebenfalls im Jahr 1962 publizierte William Riker sein Buch "`The Theory of Political Coalitions"' \parencite{Riker1962}. Riker war bereits in den 1950er-Jahren überzeugt, das Politik und Ökonomie stark miteinander interagieren und gemeinsam als positive Theorie mit quantitativen Methoden analysiert werden sollten \parencite[S. 205]{Maske2003}. Dementsprechend inspirierten ihn die frühen Arbeiten zur Spieltheorie \parencite{Morgenstern1944}, sowie die in diesem Kapitel schon behandelten Arbeiten von \textcite{Downs1957b}, \textcite{Black1948a} und \textcite{Arrow1951} \parencite[S. 205]{Maske2003}. In seinem Hauptwerk \textcite{Riker1962} ist vor allem die Verwendung spieltheoretischer Ansätze und hoch-mathematischer Methoden bahnbrechend. Riker etablierte entsprechende Forschung an der University of Rochester und

Dieser Ansatz entspricht natürlich eher der Entwicklung der Mainstream-Ökonomie, in der sich mathematische Gleichgewichtsmodelle mehr und mehr durchsetzten. Dennoch beh 







HIER WEITER
\textcite[S. 176]{Romer1988}
\textcite[S. 102]{Mitchell1988}
\textcite{Maske2003}


Während Vincent Ostrom eher philosophische Überlegungen einbrachte, machte sich Elinor Ostrom einen Namen mit empirischen Studien \parencite[S. 110]{Mitchell1988}. Das Forscherehepaar Ostrom lebte praktisch gemeinsam für ihre Version der "`Neuen Politischen Ökonomie"', beide verstarben nach fast 50 Jahren Ehe im Jahr 2012 im Abstand von nur wenigen Wochen. 


\textcite[S. 110]{Mitchell1988}



\section{Der Politische Wirtschaftszyklus}
\label{Der Politische Wirtschaftszyklus}

In den 1950er und 1960er-Jahren wa die Neoklassischen Synthese (vgl. Kapitel \ref{Synthese}) die alleinige und unangefochtene Mainstream-Ökonomie. Damit verbunden war ein nicht hinterfragter Glaube daran, dass "`der Staat"' die Konjunktur durch belebende Maßnahmen fördern sollte und auch bei Marktversagen eingreifen sollte \parencite[S. 522]{Snowdon2005}. Ab Ende der 1960er Jahre folgten erste theoretische Zweifel an der Neoklassischen Synthese (vgl. Kapitel \ref{Monetarismus} und \ref{micmac}). Ab Mitte der 1970er Jahre kam es schließlich auch auf dem Gebiet der Politischen Ökonomie zu bahnbrechenden Arbeiten, die untersuchten, wie die Politische Ökonomie mit der Makroökonomie interagiert. Konkret ging es darum, wie die Politik die beiden zentralen makroökonomischen Kennzahlen zu dieser Zeit, Arbeitslosigkeit und Inflation zu beeinflussen versuchte. Mitte der 1970er Jahre war der Erwartungsgestützte Phillipskurven-Zusammenhang nach \textcite{Phelps1968} und \textcite{Friedman1968} noch State of the Art\footnote{Die bereits existierenden Ansätze der Neuen Klassiker (vgl. Kapitel \ref{Neue Makro}), wonach es gar keinen Zusammenhang zwischen Arbeitslosigkeit und Inflation gibt, waren bereits publiziert worden, aber noch nicht weitreichend anerkannt}. William Nordhaus lieferte darauf aufbauend seine Arbeit zum politischen Konjunkturzyklus \parencite{Nordhaus1975}, die eine formal-mathematische Analyse zum Zusammenhang zwischen Politik und Ökonomie vorlegte und damit die Politische Ökonomie auf eine neue Ebene hob. Dazu griff er zunächst die soeben behandelten Arbeiten von \textcite{Downs1957, Downs1957b} auf. Er ging also davon aus, dass es in einem Zwei-Parteien-System beide Kräfte zur Mitte tendieren und Politiker außerdem vorwiegend ihre eigene Wiederwahl anstreben und daher weitgehend ideologiefrei regieren. Wähler wollen sowohl niedrige Arbeitslosigkeit, als auch niedrige Inflationsraten. Sie messen die Leistung der amtsführenden Politiker daran, in welchem Ausmaß diese beiden Ziele erreicht wurden. Dabei sind die Wähler aber recht kurzsichtig, vergessen also schnell was zu Beginn der Legislaturperiode geschehen ist und berücksichtigen dafür bei Wahlen die Kennzahlen der jüngsten Monate sehr stark. Wie bereits erwähnt, war damals die Erwartungsgestützte Phillipskurve State of the Art. Das heißt kurzfristig gibt es einen negativen Zusammenhang zwischen Arbeitslosigkeit und Inflation und die Inflationserwartungen werden aus der vergangenen Inflation abgeleitet. Langfristig gibt es hingegen keinen Zusammenhang. Politiker können nun Geld- und Fiskalpolitik einsetzen, um Inflation und Arbeitslosigkeit zu steuern. Und das machen sie auch: Vor Wahlen verstärken Politiker fiskalpolitische Maßnahmen um das BIP zu steigern und die Arbeitslosigkeit zu senken. Die daraus resultierende, höhere Inflation tritt erst nach den Wahlen auf. Nach erfolgreicher Wiederwahl müssen Inflationserwartungen und die Inflation selbst gesenkt werden, dafür werden Sparmaßnahmen und höhere Arbeitslosenraten in Kauf genommen. So entsteht der von \textcite{Nordhaus1975} postulierte politische Konjunkturzyklus: Durch "`Wahlzuckerl"' vor den Urnengängen gibt es hohes BIP-Wachstum, niedrige Arbeitslosigkeit bei gleichzeitig (noch) niedriger Inflation, nach den Wahlen gibt es Rezessionen bei gleichzeitig hoher Inflation und steigenden Arbeitslosenzahlen.
\textcite{Nordhaus1975} zeigt aber weiters das langfristige Problem mit diesem Ansatz: Bei den Wahlen selbst sind Inflation und Arbeitslosigkeit entscheidend. Nach den Wahlen steigt die Inflation stets an. Langfristig ist die durchschnittliche Inflationsrate damit stets höher als im Optimum bei der Natürlichen Arbeitslosenquote. Dementsprechend ist auch die Inflationserwartung langfristig zu hoch mit dem Ergebnis, dass aus dem kurzfristig Wahl-optimierenden Verhalten von Entscheidungsträgern eine Politik resultiert, die langfristig eine zu hohe Inflation hervorbringt, gleichzeitig liegt die Arbeitslosenraten unter der natürlichen Arbeitslosenrate \parencite{Nordhaus1975}. 
Das Modell wird als auch als "`Opportunistisches Modell"' bezeichnet, weil Politiker darin keine Ideologien verfolgen, sondern opportunistisch handeln um ihren Stimmenanteil zu optimieren. Das Modell überzeugt durch seinen soliden formalen Aufbau. Empirisch ließ es sich nur teilweise erfolgreich anwenden. Als problematisch wurde von manchen allerdings die Annahme angesehen, dass Politiker keine ideologischen Ziele verfolgen. Dies erscheint doch recht unrealistisch, da auch in Zwei-Parteien-Systemen doch recht verfestigte Überzeugungen beobachtet werden können.

Als Gegenmodell entstand daher wenig später das "`Partei-Ideologien Modell"' (engl.: "`Partisan Model"') von \textcite{Hibbs1977}. Er geht stärker auf die politischen Gegebenheiten der Nachkriegszeit bis in die 1970er Jahre ein. Für diese Zeit ist es im wesentlichen unumstritten davon auszugehen, dass es in den westlichen Demokratien jeweils zwei große Parteien gibt. Eine davon ist eher links orientiert, wie die Demokraten in den USA, Labour in Großbritannien und sozialdemokratische\footnote{Diese hießen zu der Zeit in den meisten westeuropäischen Staaten noch "`Sozialistische Parteien"', grenzten sich aber doch recht stark vom Sozialismus im Sinn des real existierenden Sozialismus ab.} Parteien in Westeuropa. Die andere Großpartei war konservativ eingestellt, so wie die Republikaner in den USA und Frankreich, Conservatives in UK und  Christdemokraten oder einfach Volksparteien in den übrigen west-europäischen Staaten. Als wichtigsten wirtschaftspolitischen Trade-off identifiziert auch \textcite{Hibbs1977}, ähnlich wie \textcite{Nordhaus1975}, jenen zwischen Arbeitslosigkeit und Inflation. Tatsächlich dominierten diese beiden Themen die wirtschaftspolitische Diskussion der 1970er Jahre. Die zentrale Annahme bei \textcite[S. 1468]{Hibbs1977} ist nun, dass die linken Parteien als primäres Ziel niedrige Arbeitslosigkeit verfolgen und dafür höhere Inflation akzeptieren, während konservative Parteien primär eine Aversion gegen Inflation haben, die Arbeitslosigkeit hingegen als zweitrangig erachten. Damit unterstellt Hibbs den Parteien einen ideologischen Unterschied im Hinblick auf deren makroökonomischen Ziele. Dies zeigt er auch ausführlich indem er die Entwicklung der beiden Kennzahlen für zwölf verschiedene Länder im Zeitraum zwischen 1945 und 1969 dahingehend vergleicht, ob linke oder konservative Parteien an der Macht waren. Dies ist insgesamt recht eingängig, aber dennoch umstritten. Erstens reduziert Hibbs den Trade-Off auf die zwei Kennzahlen Arbeitslosigkeit und Inflation. Dessen ist er sich durchaus bewusst und er führt auch andere makroökonomischen Zielgrößen der Wirtschaftspolitik, wie Wirtschaftswachstum und Einkommensverteilung an \parencite[S. 1471]{Hibbs1977}. Zweitens geht Hibbs davon aus, dass linke Parteien eher von Personen mit niedrigem bis durchschnittlichem Einkommen gewählt werden, während Wohlhabende eindeutig Konservative Parteien bevorzugen. Dies scheint nachvollziehbar. Es gibt allerdings keine theoretische Begründung dafür, warum arme Haushalte Inflation weniger abneigend gegenüber stehen sollten als reiche Haushalte. Auch dies diskutiert \textcite[S. 1470]{Hibbs1977}. Er argumentiert schließlich empirisch: In Umfragen lehnten Bezieher geringer und mittlerer Einkommen hohe Arbeitslosigkeit tatsächlich stärker ab, als hohe Inflation, während für Bezieher hoher Einkommen das umgekehrte galt \parencite[S. 1470]{Hibbs1977}. Für die USA und Großbritannien erstellt er darauf aufbauend ein ökonometrisches Modell, dass die Arbeitslosigkeit in Abhängigkeit der regierenden Parteien prognostiziert und seine theoretischen Annahmen untermauert.

Sowohl \textcite{Nordhaus1975} als auch \textcite{Hibbs1977} gelten in der Disziplin der Politischen Ökonomie als extrem bedeutend. Auffällig ist aber, dass sich die beiden in einem wichtigen Punkt widersprechen: Im Gegensatz zu \textcite{Nordhaus1975} geht \textcite{Hibbs1977} nicht davon aus, dass sich politische Parteien hinsichtlich ihrer wirtschaftspolitischen Zielen aneinander annähern, sondern ihren ideologischen Prinzipien treu bleiben. Sind Politiker nun Opportunisten oder Ideologen? In dieser Zwickmühle lieferten \textcite{Schneider1978a, Schneider1978b} eine viel beachtete Erweiterung. Diese Arbeiten sind insbesondere auch aus deutschsprachiger Sicht bedeutend. Mit dem Schweizer Bruno Frey und dem Österreicher Friedrich Schneider lieferten schließlich zwei Ökonomen, die beide überwiegend in Zentraleuropa tätig waren und sind, einen bahnbrechenden Beitrag. Die beiden Arbeiten konzentrieren sich vor allem auf eine saubere ökonometrische Analyse der Thematik. In \textcite[S. 175]{Schneider1978a} führen die beiden Autoren eine "`Popularitäts-Funktion ein. Damit messen sie wie die ökonomischen Faktoren Arbeitslosigkeit, Inflation und Konsumwachstum, sowie die persönlichen Faktoren, wie die Beliebtheit, basierend auf Umfragewerten, die Popularität der US-Präsidenten zwischen 1953 und 1975 beeinflussten. Außerdem erstellen sie eine "`Reaktions-Funktion"', die zeigt, wie der jeweilige Präsident auf die ökonomischen Gegebenheiten reagierte um seine Macht zu sichern \parencite[S. 178]{Schneider1978a}. Regierende wollen primär ihren eigenen Nutzen maximieren. Dies bedeutet sie wollen ihre Ideologie umsetzen. Dabei unterliegen sie aber einigen Einschränkungen: Vor allem die Notwendigkeit der Wiederwahl. \textcite[S. 189f]{Schneider1978a} ziehen daraus den Schluss, dass Politiker primär ideologisch motiviert sind, wie in \textcite{Hibbs1977} postuliert. Wenn die eigene Popularität allerdings fällt und zudem Wahlen anstehen, werden Politiker zunehmend populistisch, wie von \textcite{Nordhaus1975} vorhergesagt, und versuchen durch beliebte Maßnahmen ihre Stimmen zu maximieren, ohne auf die eigene Ideologie zu achten. In \textcite{Schneider1978a, Schneider1978b} wird dies mittels ökonometrischem Modell für die USA, bzw. Großbritannien dargelegt.

Alle drei genannten Ansätze - also der Opportunistische, der Ideologische, sowie die Synthese daraus, wurden viel beachtet. Vor allem das hoch-formale Werk von \textcite{Nordhaus1975} war wegweisend in der Neuen Politischen Ökonomie. Allerdings war diesen Modellen nur kurzer Erfolg gegönnt. Denn die "`Lucas-Kritik"' (vgl. Kapitel \ref{Neue Makro}) und der rasche Aufstieg der Theorie der Rationale Erwartungen spülte innerhalb der akademischen Welt nicht nur die Neoklassische Synthese und den Monetarismus innerhalb kurzer Zeit weg, sondern damit auch die Grundlagen der Modelle von \textcite{Nordhaus1975}, \textcite{Hibbs1977} und \textcite{Schneider1978a}: Unter der Annahme rationaler Erwartungen gibt es für Politiker keine Möglichkeit die Wähler kurzfristig zu täuschen und damit Stimmen zu generieren. Außerdem wurde der Phillips-Kurven-Zusammenhang von den jetzt dominierenden Neuen Klassikern komplett abgelehnt. Für eine allgemein-gültige Theorie fehlte außerdem die Empirie für diese frühen Modelle der Politischen Ökonomie. Zwar waren \textcite{Hibbs1977} und \textcite{Schneider1978a} bemüht gerade empirisch und methodisch fortschrittlich zu arbeiten. Allerdings lieferte die Modelle unterschiedlich erfolgreiche Ergebnisse, wenn man sie auf alternative Zeiträume oder Staaten anwendete \parencite[S. 652]{Alesina1987}. Die Forschung zum Zusammenhang zwischen Makroökonomie und demokratischen Prozessen kam rasch zum Erliegen.

Erst Ende der 1980er Jahre schließlich fand die Politische Ökonomie Wege rationale Erwartungen in ihre Modelle aufzunehmen. Untrennbar mit dem Wiederentdecken des Themas verbunden ist der Name Alberto Alesina, der ab Ende der 1980er Jahre bis zu seinem frühen Tod im Jahr 2020 der prägendste Vertreter der Neuen Politischen Ökonomie war. Im wesentlichen baut seine Theorie auf jenem von \textcite{Hibbs1977} auf und wird daher auch als "`Rational Partisan Theory"' (dt.: Rationales Partei-Ideologien-Theorie) bezeichnet. Das Forschungsprogramm von \textcite{Alesina1987, Alesina1988, Alesina1989} ist jedoch umfangreicher und wesentlich moderner - Hibbs war im Jahr 1977 noch vom bereits damals veralteten ursprünglichen Phillips-Kurven-Zusammenhang ausgegangen. Zunächst wurde in der Rational Partisan Theory, wie der Name schon sagt, die Theorie der Rationalen Erwartungen der Neuen Klassiker (vgl. Kapitel \ref{Neue Makro}) aufgenommen. Wichtig ist auch, dass die Spieltheorie (vgl. Kapitel \ref{Spieltheorie}) als Entscheidungstheorie Eingang in der Neue Politische Ökonomie gefunden hat \parencite{Alesina1987} und in weiterer Folge auch dort eine wichtige Rolle spielte. Außerdem wurde zu dieser Theorie eine große Anzahl empirischer Studien durchgeführt (vgl. z.B.: \textcite{Alesina1988b}, \textcite{Alesina1992}).

Das "`Rational Partisan-Model"' wurde in \textcite{Alesina1987} entwickelt. Wie bereits erwähnt knüpft es an \textcite{Hibbs1977} an, und geht davon aus, dass es im Zwei-Parteien-System ideologische Unterschiede zwischen den beiden Parteien gibt. Er implementiert aber auch die Theorie der Rationalen Erwartungen, die in der Form der einflussreichen Arbeit von \textcite{Kydland1977} die Ökonomie revolutioniert hat (vgl. Kapitel \ref{Neue Makro}). Die Wähler wie auch die politischen Parteien sind in diesem Modell folglich zwei Spieler mit unterschiedlichen Zielen. Unter der Annahme Rationaler Erwartungen kann es aber zu keiner systematischen Täuschung der Wähler durch die Politik kommen. Die Erklärung für unterschiedliche Inflations-Arbeitslosigkeits-Gleichgewichte, also politische Konjunkturzyklen, wie von \textcite{Nordhaus1975} vorgeschlagen, muss daher scheitern. Wenn die Wähler, die gleichzeitig ja auch Arbeitnehmer sind, rationale Erwartungen haben, dann können sie von den politischen Parteien nicht systematisch getäuscht werden. Durch politische Entscheidungen ausgelöste Konjunkturzyklen existieren dann nicht, weil die Wähler ihr Verhalten stets an jenes der Politiker sofort anpassen. Hier findet \textcite{Alesina1987} seinen Ansatz für das "`Rational Partisan-Model"'. Wähler können nämlich ihre Präferenzen nur bei Wahlen Ausdruck verleihen, zwischen jeweils zwei Wahlterminen sind die politischen Machtverhältnisse hingegen fixiert. Man kann diesen Effekt mit einer "`nominalen Rigidität, wie in Kapitel \ref{Nominale Rigiditäten} dargestellt, vergleichen. Die Neu-Keynesianer argumentieren dort, dass Geldpolitik in der kurzen Frist eben doch wirksam ist, weil Lohnverhandlungen die Gehälter für die Arbeitnehmer für einen bestimmten Zeitraum fixieren. Analog argumentiert \textcite{Alesina1987}: Die Wähler und Arbeitnehmer verhandeln ihre Löhne mit dem Wissen, dass die beiden politischen Parteien unterschiedliche Präferenzen hinsichtlich Inflation haben. Das heißt: Sind linke Parteien an der Macht, ist die Inflation tendenziell höher, was höhere Lohnabschlüsse rechtfertigt. Umgekehrtes gilt, wenn konservative Parteien an der Macht sind. Rationalen Lohn-Verhandlern ist dies bekannt und wird dementsprechend eingepreist. Finden allerdings nach den Lohnverhandlungen Wahlen statt, so können selbst rationale Verhandler die zukünftigen politischen Maßnahmen nicht abschätzen, weil Unsicherheit darüber besteht, welche Partei überhaupt an der Macht ist \parencite[S. 653]{Alesina1987}. Kommt eine Konservative Partei an die Macht, so wird die Inflation gesenkt. Die zu hohen Inflationserwartungen führen zu einem Anstieg der Arbeitslosigkeit und sinkenden Wachstumsraten. Nach Anpassung der Inflationserwartungen werden sich Arbeitslosigkeit und Wachstum wieder an ihre natürlichen Raten annähern. Umgekehrt wird bei einem Wahlsieg linker Parteien die Inflation höher als erwartet sein. Die Arbeitslosigkeit sinkt und das Wachstum steigt zunächst. Zwischen zwei Wahlen kommt es also zu einem politischen Konjunkturzyklus. Nach den Wahlen weichen die Kennzahlen vom natürlichen Gleichgewicht ab und nähern sich diesem nach einiger Zeit wieder an. Da Abweichungen vom langfristigen Gleichgewicht immer unerwünscht sind, plädiert \textcite[S. 653]{Alesina1987} dafür, den Handlungsspielraum politischer Entscheidungsträger durch klar definierte Regeln einzuschränken.
In ähnlicher Weise - wie \textcite[S. 671]{Alesina1987} selbst anmerkt - hatten bereits ein Jahr früher \textcite{Rogoff1986} das "`Opportunistische Modell"' von \textcite{Nordhaus1975} zu einem "`Rationalen Opportunistischen Modell"' weiterentwickelt. Ihr Ausgangspunkt dabei war, dass die Wähler in ihrem Wahlverhalten zwar rational entscheiden, allerdings unter Informationsasymmetrie leiden. Der Aufwand sich über die Wahlprogramme der Parteien zu informieren, ist höher als der einzelwirtschaftliche Nutzen einer daraus abgeleiteten "`richtigen"' Wahl. Dementsprechend können Politiker durch opportunistisches Verhalten auch Wählern, die unter der Prämisse Rationaler Erwartungen handeln, Stimmen generieren.

HIER WEITER

\parencite[S. 542]{Snowdon2005} Empirische Ergebnisse.











\section{Moderne Politische Ökonomie: Die Unabhängigkeit von Zentralbanken}

Schieben wir an dieser Stelle einen längeren Exkurs über Zentralbanken im Allgemeinen ein: Texte über Funktion, Entstehung und Evolution von Zentralbanken könnten ganze Bibliotheken füllen. Hier soll nur ein ganz kurzer Abriss über Zentralbanken im allgemeinen gemacht werden. Der amerikanische Entertainer Will Rogers soll bereits 1920 gesagt haben: "`There have been three great inventions since the beginning of time: fire, the wheel, and central banking."' Das ist zwar deutlich übertrieben, aber tatsächlich scheint die Arbeit von Zentralbanken\footnote{Gemeint ist hier die Durchführung der Geldpolitik.} für überraschend viele Menschen ein Mysterium. Dass es ganz grundlegende Unterschiede gibt, ob eine Regierung oder aber eine Zentralbank Geld ausgibt, verstehen viele nicht. Umgekehrt glauben viele, dass Geld ausschließlich von der Zentralbank geschaffen wird. Und um ehrlich zu sein, ist das Geldsystem wesentlich komplizierter als es auf den ersten Blick wirkt. Auch dieses Buch muss diesbezüglich an der Oberfläche bleiben\footnote{Möglicherweise bleibt der Beitrag sogar zu stark an der Oberfläche, aber die Alternative wäre, dass der Fokus auf das eigentliche Thema verloren ginge.}. Wir betrachten hier die historische Entwicklung, die Zentralbanken vor allem im 20. Jahrhundert durchliefen. Zentralbanken werden häufig die "`Hüterinnen der Währung"' genannt. Geld ist evolutionär so entstanden, dass sich ein bestimmter "`Kreis an Personen"' auf bestimmte Gegenstände als Tauschmittel geeinigt hat. Das hat den Vorteil, dass zum Beispiel ein Fischer, der zur Abwechslung einmal Kartoffel erwerben wollte, nicht solange suchen musste, bis er einen Ackerbauern gefunden hat, der selbst Fische erwerben wollte. Stattdessen konnte der Fischer seine Ware an jeden Fisch-Liebhaber gegen Geld eintauschen und mit dem Geld Kartoffel erwerben. Der Nachteil des Ganzen: Der "`Kreis der Personen"' muss sich auf ein Gut einigen, das alle als Tauschmittel akzeptieren. Das Gut muss also einen "`allgemeinen Wert"' haben. Niemand kann genau sagen warum, aber bestimmte Edelmetalle, vor allem Gold, hat sich als zentrales Tauschmittel früh etabliert. Zur Einordnung: Wir reden hier von vorchristlichen Zeiten, also lange vor der Entstehung der Ökonomie als Wissenschaft. Die ständige Mitnahme seiner Geldreserven in Form von purem Gold erwies sich aber rasch als unpraktisch. Erstens, Gold ist schwer, zweitens es nutzt sich ab, verliert also an Wert und drittens, in der Praxis ist es recht schwer den \textit{reinen} Goldgehalt, zum Beispiel einer Münze, festzustellen. Die Heureka-Geschichte des Archimedes diesbezüglich ist ja recht gut bekannt. Daher ging man dazu über seine Edelmetalle einer zentralen Institution - genannt Bank - zu übergeben, die dafür wiederum ein Dokument (Wechsel) ausstellte. Die Idee ist, dass dieses - an sich wertlose Dokument - einem potentiellen Käufer signalisieren soll: Ich habe Gold bei der Institution Bank liegen, wenn du mir deine Güter gibst, so gebe ich dir das Dokument und du kannst dir mein Gold von der Bank holen. Im recht kleinen Personenkreis funktioniert das recht gut. Allerdings müssten alle Personen ihre Goldreserven bei der gleichen Bank einlagern, damit das System reibungslos funktioniert. Ist \textit{die eine} Bank räumlich weit weg, ist ein Dokument von dieser Bank umständlich gegen Gold einzutauschen. Es entstanden also lokal verschiedenste Banken, die verschiedenste Dokumente ausstellten. Um dennoch ein einheitliches und damit reibungsloses Geldsystem zur Verfügung zu stellen, müssten diese Banken wiederum eine zentrale Institution gründen, die ein einheitliches Dokument ausstellt. Meist trat spätestens hierbei die Politik in das Spiel ein. Die herrschende Macht - weitgehend egal übrigens ob König, Parlament oder Diktator - gründete diese "`Bank der Banken"' und damit das was wir heute "`Zentralbank"' nennen.

Das eben beschriebene System wäre ein "`reiner Goldstandard"'\footnote{Eigentlich müsste es "`Edelmetallstandard"' heißen, denn als hinterlegtes Gut fungierten zunächst häufig auch andere Edelmetalle, tatsächlich Silber häufiger als Gold. Namensgebend ist aber jenes System, dass sich 1870, ausgehende von Großbritannien, weltweit durchgesetzt hat. Im "`reinen Goldstandard"' im engeren Sinn wird das vorhandene Gold tatsächlich zu Münzen geprägt und als Zahlungsmittel verwendet. Ökonomisch gesehen ist es aber von untergeordneter Bedeutung ob das Gold tatsächlich im Umlauf ist, oder ob Geldscheine stattdessen verwendet werden. Wichtig ist, dass stets die \textit{gesamte} Geldmenge durch Gold repräsentiert ist.}. Die Bezeichnung von Währungen erinnert teilweise noch heute an dieses System: Zum Beispiel "`Pfund Sterling"'. Pfund ist eine Gewichtseinheit (Masseneinheit), Sterling eine Metalllegierung aus Silber und Kupfer. Auf den Geldscheinen steht ein Satz, der uns heute seltsam anmutet und auch nicht mehr wörtlich genommen werden darf: "`I promise to pay the bearer on demand the sum of 10 pounds"'. Also: "`Dem Inhaber der Banknote werden 10 Pfund bezahlt"'!? Der Inhaber der Banknote \textit{hat} ja schon den 10-Pfund-Schein. Gemeint ist aber eben, dass man für diesen Schein 10 Pfund Sterling erhält. Was aber eben auch nicht mehr stimmt. Die Englische Zentralbank "`Bank of England"' hat keine Eintauschverpflichtung gegen Silber. Auf den Pfund-Sterling-Banknoten ist außerdem die Queen abgebildet, was suggeriert, dass das Herrscherhaus das Geldsystem kontrolliert. Auch das ist überholt: Wir werden später sehen, dass die Unabhängigkeit der Zentralbanken bis heute ein zentrales Thema in der Geldpolitik ist. Die einzige Aufgabe einer Zentralbank besteht im "`reinen Goldstandard"' darin ein Edelmetall einzulagern und für eine gewisse Menge dieses Edelmetalls genau einen Geldschein auszugeben. Die Menge an Geldscheinen kann nur dann steigen, wenn auch die Menge an Edelmetall im selben Ausmaß steigt. Aktive Geldpolitik ist in diesem Fall nicht möglich. Diese Form des Geldsystems galt lange Zeit als das einzig stabile und denkbare System. Herrscher wichen meist nur deshalb davon ab, um sich selbst einen Vorteil zu verschaffen, indem sie zum Beispiel zusätzliches - nicht durch Gold gedecktes - Geld druckten um ihre Schulden zu bezahlen. Man nennt das "`Seignorage"' und es führt fast immer zu (Hyper-)Inflation, Vertrauensverlust in die Währung und schließlich eine Wirtschaftskrise. Papiergeld war daher lange Zeit sehr unbeliebt und musste der Bevölkerung aufgezwungen werden. Auch nach dem Zweiten Weltkrieg gab es aus diesem Grund in vielen Währungen noch Münzen mit Edelmetallgehalt - nicht ausschließlich zu Sammelzwecken, sondern es wurde damit tatsächlich bezahlt. Die klassischen Ökonomen hielten den "`reinen Goldstandard"' als einzig denkbares, stabiles Währungssystem und Geld als reines Tauschmittel. Heute gilt der Goldstandard als veraltetes System, wenngleich zum Beispiel in der "`Österreichischen Schule"' (vgl. Kapitel \ref{Austria}) nach wie vor immer wieder die Vorzüge des Goldstandards propagiert werden.  

Mit der ersten Globalisierungswelle, Ende des 19. Jahrhunderts, erhielten die Zentralbanken eine neue Aufgabe, nämlich das Aufrechterhalten stabiler Wechselkurse. Mit dem "`klassischen Goldstandard"' wurde dieses System etabliert. Es folgt dem "`Price-Specie-Flow"'-Mechanismus (vgl. Kapitel \ref{Klassik}): Werden Güter aus Land A exportiert, wird gleichzeitig Kapitel - und damit indirekt Gold - importiert. Die Warenmenge fällt durch den Export, die Geldmenge steigt. Im Importland B geschieht aber genau das Gegenteil hier gibt es nach dem Import mehr Güter aber weniger Gold. In diesem einfachen Modell sorgen die Marktkräfte dafür, dass ein Außenhandelsgleichgewicht entsteht. Durch unterschiedliche Zinssätze in verschiedenen Staaten konnte dieses Gleichgewicht gestört werden. Hier kommen die Zentralbanken ins Spiel: Diese müssen international abgestimmt ihre Zinssätze so anpassen, dass das Gleichgewicht beibehalten wird. Als führende Zentralbank gab hierbei die "`Bank of England"' den Ton an. Die politischen Umwälzungen führten schließlich dazu, dass der Goldstandard mit Beginn des Ersten Weltkriegs aufgegeben wurde. Schon zuvor wurde übrigens aus dem "`reinen Goldstandard"' ein "`Proportionalsystem"'. Hierbei geht man davon aus, dass niemals gleichzeitig das ganze Geld gegen Gold eingetauscht wird und man druckt in weiterer Folge für eine bestimmte Goldmenge eine größere Geldmenge. Ein vorab festgelegtes Verhältnis darf dabei aber niemals überschritten werden. Nach dem Ersten Weltkrieg wurde der Goldstandard wieder eingeführt, allerdings wenig erfolgreich. Deutschland und Österreich litten ab Anfang der 1920er-Jahre an großen wirtschaftlichen Problemen, die in Hyperinflation mündeten. Der britische Pfund war überbewertet und die Bank of England hatte Mühe ihre Verpflichtungen aufrechtzuerhalten. In den USA wiederum war das Wirtschaftswachstum hoch und dauernde Exportüberschüsse wurden verzeichnet. Mit der Weltwirtschaftskrise ab 1929 erfuhr der Goldstandard seinen Tiefpunkt. \textcite{Friedman1963} wurde später dafür berühmt, gezeigt zu haben wie die Federal Reserve während der "`Great Depression"' zu lange auf den Goldstandard gesetzt hatte und damit die Schwere der Krise verstärkt hatte. Noch während des Zweiten Weltkriegs wurde 1944 im gleichnamigen "`Bretton Woods"' die Grundlage für eine neues, einheitliches Währungssystem geschaffen. Diesmal ist der US-Dollar die Leitwährung und praktisch alle Industriestaaten binden ihre Währung an jene der Amerikaner. Diesmal sind es die Inflationstendenzen des US-Dollar bei gleichzeitiger guter wirtschaftlicher Entwicklungen der europäischen Staaten, die dazu führen, dass das System in den 1970er Jahren aufgegeben werden muss.

Damit begann die Zeit der Zentralbanken und der Aufstieg der Bedeutung der Geldpolitik. Hier spielten viele Faktoren zusammen: Bei den Keynesianern steht die Fiskalpolitik im Vordergrund, Geldpolitik wird zwar als wichtig anerkannt, aber eher als Ergänzung im Zusammenspiel mit der Fiskalpolitik. Bereits Mitte der 1960er-Jahre veröffentlichten \textcite{Mundell1963} und \textcite{Fleming1962} ihr "`Impossible Trinity"'-Modell (vgl. Kapitel \ref{Supply-Side-Economics}). Darin beschreiben Sie, dass ein Staat grundsätzlich flexible Wechselkurse, freien Kapitalverkehr und aktive Geldpolitik betreiben möchte. Es sind aber stets nur zwei der drei Ziele miteinander vereinbar. Da der freie Kapitalverkehr in westlichen Demokratien unverhandelbar ist, muss eine Entscheidung zwischen den beiden anderen Zielen getroffen werden. Oder mit anderen Worten: Bis in die 1970er-Jahre waren die Zentralbanken an die Aufgabe gebunden fixe Wechselkurse aufrecht zu erhalten. Erst danach konnten andere wirtschaftspolitische Ziele ins Auge gefasst werden. Die 1970er-Jahre waren zudem die hohe Zeit der Monetaristen. \textcite{Friedman1968, Friedman1976b} wollte die Zentralbanken zwar zugunsten einer fixen Wachstumsrate des Geldes "`abschaffen"', aber er stellte auch die Bedeutung der Geldpolitik eindrucksvoll in den Vordergrund. Ähnliches gilt für die "`Neue Klassische Makroökonomie"', die grundsätzlich jegliche wirtschaftspolitischen Eingriffe als nutzlos erachtete. Dennoch lieferten die Arbeiten von zum Beispiel \textcite{Kydland1977, Barro1976} wichtige Beiträge, vor allem zur Organisation von Zentralbanken als \textit{unabhängige} Institutionen. Die Theorien der modernen Politischen Ökonomie, die ab den späten 1980er-Jahren an Bedeutung erlangten (vgl. Kapitel \ref{Pol_Econ}), untermauerten dies. Die "`Neue Neoklassische Synthese"' schließlich machte die Zentralbanken - wie erwähnt - zum zentralen Player der Wirtschaftspolitik. Zunächst in der Theorie: Wenn es die neu-keynesianischen Elemente "`Nominalen Rigiditäten"' und "`Monopolistische Konkurrenz"' in der Praxis gibt, dann ist Geldpolitik in der kurzen Frist eben doch nicht wirkungslos. Fiskalpolitik hingegen spielt in den DSGE-Modellen der "`Neuen Neoklassische Synthese"' hingegen keine Rolle.

Mit der Erkenntnis, dass Geldpolitik \textit{der} zentrale Hebel einer modernen makroökonomischen Steuerung ist, traten Geldpolitik und Zentralbanken Ende der 1980er-Jahre in das Zentrum der Forschung der "`Neuen Politischen Ökonomie"'. Die berühmten Arbeiten von \textcite{Friedman1968} vorgeschlagen und später \textcite{Kydland1977} schlugen vor, dass die Geldpolitik einer fixen Regel folgen sollte und gänzlich unabhängig von diskretionären Entscheidungen sein sollte. Mit dem allmählichen Niedergang der "`Neuen Klassik"' und dem Aufstieg des "`Neukeynesianismus"' wurde klar, dass Geldpolitik als wirtschaftspolitische Maßnahme in der kurzen Frist doch wirksam wäre und dementsprechend nicht vollständig einer geldpolitischen Regel überlassen werden konnte. Natürlich waren sich die Ökonomen aber auch bewusst, dass Geldpolitik in den Händen von Politikern auch zu Seigniorage führen konnte. Seigniorage, also die Verlockung durch das bloße Drucken von Geld den Staatshaushalt aufzubessern, führt regelmäßig und unweigerlich zu Hyperinflation. Ausgehend von \textcite{Alesina1988c} wurde das Thema der "`Unabhängigkeit von Zentralbanken"' der wohl bedeutendste Forschungsbereich innerhalb der modernen "`Neuen Politischen Ökonomie"'. Alberto Alesina prägte diese Forschung wie kein zweiter.

HIER WEITER \textcite[S. 549]{Snowdon2005}













In der Praxis war die Entwicklung etwas zeitverzögert, aber heute gibt es auch hier soetwas wie einen globalen Konsens der Geldpolitik. In den USA begann der Umbruch in der Geldpolitik 1979 mit der Bestellung von Paul Volcker zum Leiter der Fed. Er führte einen schmerzhaften aber erfolgreichen Disinflations-Kurs und schaffte es, durch harte restriktive Geldpolitik, die hohe US-Inflation in den Griff zu bekommen. In der Greenspan-Ära von 1987-2006 wurde die enorme Macht eines Notenbankchefs erstmals einer breiten Bevölkerung bewusst. Seine Reden und Aussagen wurden vor allem an den Finanzmärkten peinlichst genau analysiert. In der Ära Bernanke 2006-2014 und auch Yellen (2014-2018) trat eine deutliche Verwissenschaftlichung der Zentralbank hervor. In Kontinentaleuropa prägte stets die äußerst auf Stabilität pochende Deutsche Bundesbank die Geldpolitik. Vor allem im Europäischem Währungssystem (EWS) und später in der Phase vor der Euro-Einführung 1999, in der die Teilnahme-Länder die Stabilitätskriterien nach Maastricht einhalten mussten. Die europäischen Zentralbanken hatten vor allem in den frühen 1990er Jahren damit zu kämpfen, die festgelegten Wechselkurse aufrechtzuerhalten. Spekulanten identifizierten immer wieder Überbewertungen verschiedener Währungen gegenüber der harten Deutschen Mark. Berühmt wurde der Investor George Soros, der 1992 auf die Überbewertung des Britischen Pfunds setzte und bei dessen Abwertung schließlich eine hohe Summe gewann. In dieser Zeit waren die europäischen Zentralbanken also so etwas wie "`Spieler"' auf den internationalen Devisenmärkten. Mit der Einführung des Euros und der damit verbundenen Schaffung der Europäischen Zentralbank (EZB), entstand der heute zweitwichtigste geldpolitische Akteur nach der US-Notenbank Fed. Die EZB ist dahingehend interessant, weil sie als unabhängiges Institut gegründet wurde und zudem als Primärziel einzig die Geldwertstabilität - also explizit festgelegtes Inflation-Targeting - festgeschrieben hat (siehe weiter unten in diesem Kapitel \ref{Inflation}). Die Zentralbanken haben sich also auch in der zweiten Hälfte des 20. Jahrhunderts kräftig gewandelt: Bis in die 1970er-Jahre mussten sie im Rahmen des "`Bretton-Woods-Systems"' die Währungskurse stabil halten. Eine Aufgabe, die im wesentlichen von der Politik vorgegeben wurde. Bald danach begann der Versuch der Geldmengensteuerung im Sinne von Friedman. Bis schließlich das Anstreben einer Zielinflation in den Vordergrund rückte.







Ideologie, oder Opportunismus? Folgen: Rules rather than discretion? Unabhängige Zentralbanken und Schuldenbremse.

George Stigler: Regulierung.


Gesamtübersicht Finanzwissenschaft - Public Economy und Neue Politische Ökonomie (insbesondere Buchanan)
Einbauen den Kameralismus \parencite{Backhaus2005}

Moderne Themen, die aus diesem Bereich kommen: Ungleichheit, institutionelles Wachstum, Innovation, Fiskalpolitik und Austerität, Dani Rodrik, Aghion in einem anderen Kapitel (aktuelle Themen)









		% Neue Politische Ökonomie


%\include{Teil_V}		% Heterodoxe Schulen
%%%%%%%%%%%%%%%%%%%%%% chapter.tex %%%%%%%%%%%%%%%%%%%%%%%%%%%%%%%%%
%
% sample chapter
%
% Use this file as a template for your own input.
%
%%%%%%%%%%%%%%%%%%%%%%%% Springer-Verlag %%%%%%%%%%%%%%%%%%%%%%%%%%

\chapter{Österreichische Schule}
\label{Austria}

Als Begründer der "`Österreichischen Schule der Nationalökonomie"' gilt gemeinhin Carl Menger. Als Mitbegründer der "`Neoklassik"' wurde er bereits in Kapitel \ref{Wiener Schule} behandelt. Aus dem Werk \textcite{Menger1871}: "`Die Grundsätze der Volkswirtschaftslehre"' wird häufig die Preis- und Werttheorie hervorgehoben. Tatsächlich spielte diese lange in der Österreichischen Schule eine große Rolle. So auch noch bei Hayek ("`Preise und Produktion"'). Früh, nämlich bereits durch den direkten Schüler Menger's, Eugen von Böhm-Bawerk wurde die Kapitalmarkttheorie Forschungsgegenstand der "`Österreicher"'. Diese wurde später - durch die sogenannten "`Zweite Generation"', vor allem durch deren Hauptvertreter Ludwig von Mises und dessen fundamentale Geldtheorie - weiterentwickelt. Bis hierher war die "`Österreichische Schule"' nicht weit weg von der damaligen "`Mainstream"'-Ökonomie. Von Anfang an waren die "`Österreicher"' vehemente Kritiker des um 1880 aufkommenden Sozialismus. Die Ideen von Marx waren zur damaligen Zeit die wichtigsten heterodoxen Ansichten - alle anderen Schulen könnte man heute als "`Mainstream"' bezeichnen. Die fast schon radikale Marktgläubigkeit ließ die Österreichische Schule nach 1945 aus dem Mainstream abgleiten. War diese bei Mises noch als Antwort auf den Sozialismus zu verstehen, ist sie bei Hayek die vehemente Ablehnung des Keynesianismus. Dieser wurde jedoch, wie wir wissen, zur führenden Schule nach 1945, daneben blieb nur wenig Platz für die Österreichische Schule, die sich in weiterer Folge auch von der "`Freiburger Schule"' (vgl. Kapitel \ref{Neoliberalismus}) abgrenzte, die in der deutschen Wirtschaftspolitik kurz großen Einfluss hatte. Auch von Milton Friedman und dessen Monetarismus grenzten sich die "`Österreicher"' stark ab. Diese Schule stellte ebenso den Markt in den Mittelpunkt. Hayek, der Hauptvertreter der dritten Generation, wirkte zwar im Zentrum der "`Monetaristischen Gegenrevolution ab 1970"', nämlich an der Universität in Chicago, aber er schaffte es nicht, im Kreis der Ökonomen um Friedman eine Rolle zu spielen. Wissenschaftlich verlor die Österreichische Schule damit nach dem Zweiten Weltkrieg rasch an Bedeutung. Neben der radikalen Marktgläubigkeit ist dies vor allem auf die fast gänzliche Ablehnung formal-mathematischer Methoden - die die "`Österreichische Schule"' durchgehend seit deren Begründung durch Menger vertrat - zurückzuführen. Spätestens seit der Etablierung der Theorie der rationalen Erwartungen in der "`Neuen Klassik"', gibt es kaum noch Mainstream-Publikationen, die auf die formal-mathematische Methode verzichten. Nach der "`Great Depression"' 2008 und auch mit dem Etablierung populistischer Politiker in  führenden Positionen, feierte die Österreichische Schule wieder ein Comeback. Allerdings eher aus wirtschafts-\textit{politischer} Sicht als aus wirtschafts-\textit{theoretischer}.

\section{Wieser: Der Schüler Mengers}
Wieser: Begriff des Grenznutzens. 

\section{Mises: Der Hauptvertreter}

\section{Hayek: Weg vom Mainstream, aber rein in die Politik}

\section{Rothbard, Kirzner \& die Tea Party}
Buchanan wird manchmal als Österreicher bezeichnet.


\section{Die Stockholmer Schule}
\label{cha:Stockholm}
Wicksell: Konzept des natürlichen Zinssatzes. Praktisch Wiederentdeckung von Wicksell durch Taylor-Rule, bzw. in DSGE-Modellen (daher vereinzelt auch Neo-Wicksellianische Schule genannt)       % Österreichische Schule
%%%%%%%%%%%%%%%%%%%%%% chapter.tex %%%%%%%%%%%%%%%%%%%%%%%%%%%%%%%%%
%
% sample chapter
%
% Use this file as a template for your own input.
%
%%%%%%%%%%%%%%%%%%%%%%%% Springer-Verlag %%%%%%%%%%%%%%%%%%%%%%%%%%

\chapter{Schumpeter: Der ganz eigene Weg}
\label{Schumpeter}

\section{Vorläufer Zyklentheorie: Kondratieff}

\section{Die schöpferische Zerstörung}
Eigentlich hätte sich \textit{Joseph Alois Schumpeter} ein eigenes Kapitel verdient. Die Zurechnung zur Österreichischen Schule hinkt ebenfalls gewaltig und ist am ehesten durch seine Herkunft gerechtfertigt, wobei auch das nur bedingt richtig ist. Aber der Reihe nach.

Schumpeter wurde 1883 in Trest geboren, das heute im südlichen Tschechien liegt, geboren. Trest (deutsch Triesch) gehörte damals zum österreichischen Teil der Doppelmonarchie Österreich-Ungarn. Er machte in Österreich früh eine beeindruckende Karriere: Jüngster Professor der Monarchie, später Finanzminister und Bankvorstand

Und trotzdem wirkt seine Biographie eher traurig als erfolgreich: Mit vier Jahren verstarb sein Vater. Die Mutter wurde zur wichtigen Bezugsperson. Seine Frau -- vielleicht die einzige, die er wirklich liebte, denn er galt sonst als Frauenheld und lebte dies für die damalige Zeit in skandalöser Weise aus -- starb bei der Geburt seines Kindes. Das Kind wenige Stunden später. 
Als Finanzminister trat er nach wenigen Monaten zurück. Die Bank, die er als Vorstand leitete, ging Pleite und führte ihn selbst an den Rand des Bankrotts. Wissenschaftlich war er geprägt von der Konkurrenz zu Keynes. Und wenn man diese Konkurrenz als Wettbewerb sehen will, zog er ständig den Kürzeren: Ein begonnenes geldpolitisches Werk vollendete Schumpeter nicht mehr, nachdem Keynes seine "`Treatise on Money"' veröffentlichte. Sein monumentales Wert zur Zyklentheorie "`Business Cycles"' stellte er 1939 fertig. Eine Unzeit wenn man bedenkt, dass in diesem Jahr der Zweite Weltkrieg begann, Konjunkturzyklen also wenig Interesse hervorriefen. Nach 1945 wiederum interessierte sich ebenso fast niemand für das Werk. Durch die "`General Theory"' von Keynes schien es so als gehörten Konjunkturzyklen für alle Zeit der Vergangenheit an: Keynes's Deficit Spendings sollten doch gerade diese Konjunkturzyklen glätten.
Und doch gilt Schumpeter heute als einer der bedeutensten Ökonomen des 20. Jahrhundert. Ich habe zu Anfang des Kapitels erwähnt, dass man Schumpeter nur schwierig einer ökonomischen Schule zuordnen kann. Vielleicht ein Grund warum er sich in seiner Gegenwart nur schwer durchsetzte. Vielleicht aber auch ein Grund dafür warum er heute unverändert aktuell ist: Seine Theorie ist praktisch zeitlos. Insbesondere die auf ihn zurückzuführende Idee der "`schöpferischen Zerstörung"' wird immer wieder zitiert. Und, wesentliche bedeutender: Kaum jemand widerspricht dieser Idee der schöpferischen Zerstörung! Keynes hatte seine Blütezeit, aber er wurde sehr bald auch kritisiert und später von \textit{Robert Lucas} wissenschaftlich für tot erklärt. Auf Schumpeter berufen sich noch heute -- fast 100 Jahre nach dem Erscheinen seines Werkes "`The Theory of Development"' -- Ökonomen unterschiedlichster Richtungen (Acemoglu und Paul Romer, früher Baumol und Minsky).

Interessant, dass sein großer Zeitgenosse John Maynard Keynes, das Thema des Marktversagens gänzlich unbehandelt ließ. Im Gegenteil, Keynes folgte diesbezüglich seinem Lehrer Marshall und übernahm das Konzept des perfekten Wettbewerbs auf den Einzelmärkten. In Bezug auf die "`Vertrustung"' der Ökonomie hatte Keynes auch kaum Bedenken, er glaubte an die "`Drei-Generationen-Theorie"', wonach Unternehmen nur mittelfristig erfolgreich sind \parencite[S. 93]{Snowdon2005}. Schumpeter vertrat diesbezüglich einen ganz anderen Ansatz und prophezeite, dass der Kapitalismus langfristig daran scheitern wird, weil die Monopolisierung der Märkte immer weiter voranschreitet, bis der Wettbewerb schließlich gänzlich ausgeschaltet sei.

Schumpeter hatte auch Einfluss auf Douglass North \textcite[S. 11]{Menard2014} 	    % Schumpeter
%%%%%%%%%%%%%%%%%%%%%% chapter.tex %%%%%%%%%%%%%%%%%%%%%%%%%%%%%%%%%
%
% sample chapter
%
% Use this file as a template for your own input.
%
%%%%%%%%%%%%%%%%%%%%%%%% Springer-Verlag %%%%%%%%%%%%%%%%%%%%%%%%%%

\chapter{Supply-Side-Economics}
\label{Supply-Side-Economics}

Der Begriff "`Supply-Side-Economics"', in deutsch meist mit "`Angebotsorientierte Wirtschaftspolitik"' übersetzt, wird häufig sehr allgemein für konservative Wirtschaftspolitik herangezogen. Im hier herangezogenen engeren Sinn, umfassen die "`Supply-Side-Economics"' ausschließlich die Arbeiten von Arthur Laffer und vor allem Robert Mundell in den 1960er- und 1970er-Jahren, die politisch großen Einfluss erlangten.
Die "`Supply-Side-Economics"' waren ein Teil der Anti-Keynesianischen Gegenrevolution. Sie war zwar ebenso konservativ wie der "`Monetarismus"' und die "`Neue Klassische Makroökonomie"', allerdings kann sie nicht als Teil einer der beiden Schulen gesehen werden, da sie inhaltlich - vor allem im Vergleich zum Monetarismus - und methodisch - vor allem im Vergleich zur "`Neuen Klassik"' -  anders ausgerichtet ist. Aber sie etablierte sich nicht als eigenständige wirtschafts\textit{wissenschaftliche} Schule. Berühmt geworden ist sie vielmehr durch ihre politische Umsetzung. Während sowohl der Keynesianismus, der Monetarismus, als auch die Neue Klassische Makroökonomie (und andere ökonomischen Schulen) auf grundlegenden theoretischen Werken aufgebaut wurden, fehlt die schriftliche wissenschaftliche Abhandlung der Supply-Side-Economics weitgehend. 

Als Hauptvertreter der "`Supply Side Economics"' kann zweifelsohne Robert Mundell angesehen werden. Eine in vielerlei Hinsicht "`interessante"' Persönlichkeit. Der in Kanada geborene Mundell war zeitlebens umtriebig, unorthodox und umstritten. Mit 24 Jahren promovierte er im Jahr 1956 am MIT um gleich danach an die University of Chicago als Post-Doktorand zu gehen. Schon 1957 zog es ihn weiter, nach Stanford und später nach Italien - dort kaufte er später ein Schlösschen - um 1961 bei Internationalen Währungsfonds (IWF) anzuheuern \parencite{Mundell1999}. In der Zeit schrieb er seine - aus wirtschafts\textit{wissenschaftlicher} Sicht bedeutendsten Werke. Heute zählen diese Werke längst zur Standard-Ökonomie und mit dem Feld der internationalen Makroökonomie begründete Mundell ein ganzes Forschungsgebiet. Anfang der 1960er Jahre war aber selbst diese Arbeit noch umstritten, weil das Bretton-Woods-System \footnote{Das Bretton-Woods-System wurde 1944 für die Zeit nach dem Zweiten Weltkrieg geschaffene und war ein internationales Währungssystem, dass stabile Wechselkurse sicherte. In dem System wurde der US-Doller an den Goldpreis gebunden. Eine Unze Gold kostete 35 US-Dollar. Staaten konnten sich dem Währungssystem anschließen und ihre Währung an den Dollar binden. Die US-Dollar konnten bei der Federal Reserve jederzeit gegen Gold eingetauscht werden} fixe Währungskurse sicherte und ein System mit international flexiblen Währungskursen zwischen den Industrieländern zu dieser Zeit noch undenkbar waren. In seinem Werk \textcite{Mundell1963} erweiterte er das keynesianische IS-LM-Modell um Gesichtspunkte des internationalen Handels und entwickelte damit das \textit{Mundell-Fleming-Modell}\footnote{Der Brite Marcus Fleming hatte unabhängig davon ein ähnliches Modell entwickelt, daher der Name.}. Das Modell führt in das IS-LM-Modell eine dritte Funktion, die ZZ-Kurve, ein. Diese bildet den internationalen Handel und damit das Außenhandelsdefizit (bzw. -überschuss) ab. Im Falle uneingeschränkter Kapitalmobilität und flexiblen Wechselkursen kann man mittels geldpolitischen Maßnahmen Wirtschaftspolitik betreiben, während Fiskalpolitik relativ unbedeutend wird, da zusätzliche Staatsausgaben zu einem großen Teil im Ausland verpuffen. Bei fixierten Wechselkursen hingegen muss man Geldpolitik betreiben um das Kursverhältnis im Gleichgewicht zu halten. Aktive Wirtschaftspolitik kann dann nur mittels Fiskalpolitik betrieben werden. Daraus leitete er außerdem das "`Impossible-Trinity-Problem"' ab: Bei freiem Kapitalverkehr kann ein Staat entweder unabhängige Geldpolitik betreiben, oder den Wechselkurs zu anderen Währungen fix halten. Beide Ziele zu erreichen ist unmöglich.
Bereits 1961 veröffentlichte er ein Werk zu "`optimalen Währungsräumen"' \parencite{Mundell1961}. Damit legte er die theoretische Grundlage für grenzübergreifende Währungsunionen. Er wird in diesem Zusammenhang immer wieder "`Vater des Euro"' genannt \parencite{Mundell2006}.
Sein drittes bemerkenswertes Paper in dieser Zeit wurde 1962 veröffentlicht und behandelte den "`Policy Mix"' aus Fiskal- und Geldpolitik \parencite{Mundell1962}. Darin kam schon seine unorthodoxe Sichtweise zum Ausdruck. In diesem Artikel plädierte er nämlich dafür, dass die Politik zur Stimulierung der Wirtschaft aktive Fiskalpolitik betreiben sollte, aber gleichzeitig zur Bekämpfung der Inflation die Zinsen erhöhen sollte \parencite[S. 121]{Appelbaum2019}. Er nahm darin zum Teil schon seine höchst umstrittenen angebotsorientierten Politikempfehlungen der 1970er Jahre voraus, wobei nicht klar ist, ob er mit de geforderten aktiven Fiskalpolitik Staatsausgaben, oder, wie später ausschließlich, Steuersenkungen meinte. Mundell wurde im Jahr 1962 übrigens 30 Jahre alt. 

Diese beeindruckende frühe Biographie Robert Mundell's ist im Grunde genommen ein Exkurs in diesem Kapitel. Der Forschungsbereich internationale Makroökonomie war zwar neu, aber doch Teil des keynesianischen Mainstreams. Im Jahr 1966 erhielt er eine Professur an der University of Chicago. Und passte seine Sichtweise scheinbar an das neue Umfeld an. "`Es war als ob er die klassische Ökonomie und die Sicht auf die lange Frist entdeckt hatte"', heißt es dazu in \textcite[S. 194]{Warsh}.
Während seiner Zeit in Chicago (er blieb nur bis 1971) entwickelte er sein höchst umstrittenes Konzept der "`Supply-Side-Economics"', ohne dabei aber auch nur eine wissenschaftliche Abhandlung darüber zu veröffentlichen \parencite[S. 192]{Warsh}. In den USA stieg die Inflationsrate in der zweiten Hälfte der 1960er Jahre auf die 5\%-Marke und in wissenschaftlichen Kreisen wurde erstmals darüber diskutiert, wie Inflation am besten entgegenzutreten war. Die damalige Standardantwort war jene der Keynesianer: Steuererhöhungen gegen die Inflation, dafür niedrige Zinsen und zusätzliche Staatsausgaben um Wirtschaftswachstum und Beschäftigung hoch zu halten. Die Monetaristen um Milton Friedman sahen in der Inflation ein rein monetäres Problem und wollten gegen Inflation schlicht das Geldmengenwachstum einschränken. Damit waren sie zwar noch nicht im Mainstream angekommen, aber die Idee war durch die Quantitätsgleichung theoretisch hinterlegt und fand immer mehr Zulauf. Und dann gab es eben Robert Mundell, der meinte gegen Inflation müsste man die Steuern senken und die Zinsen erhöhen. Die Idee Mundell's war Inflation damit zu Bekämpfen die Produktion durch niedrigere Steuern zu fördern. Mehr Produktion bringt mehr Waren und damit ein steigendes Angebot, das bei gleichbleibender Nachfrage zu sinkenden Preisen (oder zumindest zu geringerer Inflation) führt. Beide Lager, Keynesianer wie auch Monetaristen fanden dies geradezu lächerlich. Ende der 1960er-Jahre geriet das Bretton-Woods-System unter Druck. In den USA gab es zwar gute Wachstumszahlen, aber auch steigende Inflation und starke Außenhandelsdefizite. In dieser Zeit hatte auch Robert Mundell einen schweren Stand. Als Professor in Chicago wurden seine Ideen von der Kollegenschaft nicht ernst genommen \parencite[S. 195]{Warsh}. Er versagte als Präsident der "`Hemispherical Economic Society"' und als Herausgeber des renommierten "`Journal of Political Economy"'. Die Wirtschaftspolitik unter den Präsidenten Johnson und Nixon, Ende der 1960er Jahre, hielt nichts von seinen Vorschlägen. Schließlich floh er 1971 in seine kanadische Heimat an die University of Waterloo in Kitchener, Ontario. "`At last, Waterloo meets its Napoleon"', waren angeblich die hämischen Kommentare \parencite[S. 195]{Warsh}.

Aber der Siegeszug der "`Supply-Side-Economics"' stand erst bevor. Und eine Reihe glücklicher Zufälle verhalf ihr auf die Sprünge. Im Frühjahr 1974 kehrte Mundell zurück in die USA, an die Columbia University. "`Im Mai oder Juni"', wie \textcite{Mundell1998} in einem Interview sagte, gab es eine Konferenz des "`American Enterprise Institute"' organisiert von Arthur Laffer, dem zweiten berühmt gewordenen Vertreter der "`Supply-Side-Economics"'. Dort stellte Laffer außerdem den Journalisten Jude Wanniski vor. Die drei sollten gemeinsam ein höchst einflussreiches Gespann werden. Zunächst aber prognostizierte Mundell noch in beeindruckender Weise die kommende Rezession. 1973 schon war das Bretton-Woods-Systems endgültig zusammengebrochen. Auf der eben genannten Konferenz argumentierte Mundell, dass nur Steuersenkungen die vor der Tür stehende Kombination aus Inflation und Rezession (Stagflation) vermeiden könne. Im Herbst 1974 trat tatsächlich der Fall ein, dass Präsident Ford der steigenden Inflation mit Steuererhöhungen entgegentreten wollte. Wenig später fanden sich die USA in einer tiefen Rezession wieder \parencite[S. 195]{Warsh}. Damit war der Weg für die "`Supply-Side-Economics"' geebnet.

Serviette Laffer-Kurve s 126
reagan s. 142
Friedman S. 133 Friedman war eigentlich Gegenspieler in Sachen Inflation. Aber weil gleiche Intention


Nobelpreis: Eigentlich für Internationale Makroökonomie. Selbstvertrauen blieb: Zusammenhang seine Theorie Zusammenhang zum Ende der Sowjetunion.





Der erste bekannte Vertreter ist Arthur Laffer. Bekannt durch seine "`Laffer-Kurve"'. Damit konnte er den Politiker (Ronald Reagan???) davon überzeugen, dass Steuersenkungen Wirtschaftswachstum anregen können und in weiterer Folge die Gesamtsteuerleistung trotz niedrigerer Steuerbelastung höher ist, weil der Effekt des Wirtschaftswachstums den Effekt der Steuersenkung übertrifft. Die Schule auf diese Idee zu reduzieren, wäre aber eine unzulässige Vereinfachung.
Zweiter Robert Mundell: Er forderte während der Stagflation, dass (Dritter Weg neben Monetaristischer Geldmengensteuerung und keynesianischer Preisregulierung) die Steuern und die Staatsausgaben gesenkt werden sollten. (Kapitel 3??? in Stunde der Ökonomen)










      % Supply Side Economics     			!!! KORREKTURLESEN
%%%%%%%%%%%%%%%%%%%%%% chapter.tex %%%%%%%%%%%%%%%%%%%%%%%%%%%%%%%%%
%
% sample chapter
%
% Use this file as a template for your own input.
%
%%%%%%%%%%%%%%%%%%%%%%%% Springer-Verlag %%%%%%%%%%%%%%%%%%%%%%%%%%

\chapter{Neoliberalismus}
\label{Neoliberalismus}

\section{Den Neoliberalismus gibt es nicht!}


Eine der wohl bekanntesten ökonomischen Schulen ist der Neoliberalismus. Aber was ist eigentlich Neoliberalismus? Die Frage ist durchaus berechtigt und eines vorweggenommen: Sie ist nicht befriedigend zu beantworten!
Olivier Blanchard, einer der berühmtesten Ökonomen der Gegenwart, antwortete auf diese Frage, gestellt von einer österreichischen Zeitung: "`Ich habe keine Ahnung. Aber Sie können mich gern zitieren."' \parencite{Widmann2020} 


Es ist tatsächlich ganz und gar nicht unumstritten was mit Neoliberalismus gemeint ist. Sicher ist aber: Thinktanks und Medien, die eher dem linken Spektrum hinzuzurechnen sind, verwenden den Begriff gerne eher abwertend und behaupten dabei meist der Neoliberalismus sei die neue, führende ökonomische Schule, die wesentlichen Einfluss auf die wirtschaftspolitischen Entscheidungsträger hat. Nun, in den letzten Jahren haben sich in den westlichen Demokratien tatsächlich konservative Kräfte gegenüber den sozialdemokratischen Kräften weitgehend durchgesetzt. Und die konservativen Kräfte stehen wirtschaftspolitisch tatsächlich eher dem wirtschaftsliberalen Flügel nahe. Betrachtet man allerdings empirische Daten, dann muss man objektiv zu dem Ergebnis kommen, dass der "`Neoliberalismus"' nicht die führende ökonomische Schule sein kann: Wie \textcite{Aiginger2016} richtig darstellt sind die Staatsausgaben in \% des BIP zwar seit den 1980er Jahren in der EU nicht mehr wesentlich weiter gestiegen, aber auch nicht gesunken. Und am wichtigsten: Die Staatsausgaben in \% des BIPs belaufen sich in den meisten europäischen Staaten immerhin bei über 40\%. Es ist daher schwer zu argumentieren, der Neoliberalismus sei die führende Denkrichtung der Ökonomie.

Wie \textcite[S. 179]{Venugopal2015} recherchierte findet sich der Begriff "`Neoliberalismus"' in keinem der führenden Lehrbücher. Weder in \textit{Blanchard} noch in \textit{Mankiw} noch bei \textit{Obstfeld}, auch nicht bei \textit{Stiglitz} und \textit{Krugman} findet man ihn. Und auch in den führenden wirtschaftswissenschaftlichen Journals kommt der Begriff praktisch nicht vor \parencite[S. 179]{Venugopal2015}. Jedem Studierenden sei damit abgeraten seine Abschlussarbeit über "`Neoliberalismus"' zu schreiben. Klar ist damit auch, dass man sich auf unsicheres Terrain begibt, wenn man in einer ökonomischen Diskussion auf den Neoliberalismus stürzt. Es findet diesen Begriff schlicht und einfach in den Wirtschaftswissenschaften scheinbar nicht. Aber wie kann das sein, wenn doch außerhalb des rein wissenschaftlichen Diskurses der Begriff ständig verwendet wird?

Da der Begriff -- nennen wir es umgangssprachlich -- so häufig vorkommt, versuchen wir uns dennoch der Annäherung. Und hier findet man schnell einen -- zumindest meiner Ansicht nach -- überzeugenden Ansatz: Der Begriff "`Neoliberalismus"' wird etwas unscharf als Überbegriff für verschiedene ökonomische Schulen, die nach 1945 entstanden, verwendet. Diese sind, erstens, die vierte Generation der \textit{Österreichischen Schule} um \textsc{Friedrich August Hayek}, sowie deren amerikanische Nachfolger um \textsc{Murray Rothbard} und \textsc{Israel Kirzner}. Die zweite wichtige Gruppe, die mitumfasst ist, ist die \textit{Chicago School}, gegründet von \textsc{Frank Knight} und \textsc{Jacob Viner}. Wobei hier zwei Untergruppen genannt werden müssen. Erstens, der "`Monetarismus"' von \textsc{Milton Friedman} und seine Schüler, die "`Chicago Boys"', sowie die "`moderne Finance"', deren geistige Väter zu einem Großteil an der Universität Chicago tätig waren, aber ideologisch nicht unbedingt zum Neoliberalismus zu zählen sind. Drittens ist die "`Lucas-Kritik"' von \textsc{Robert Lucas} und die daraus abgeleitete "`Neue klassische Makroökonommie"' mit \textsc{Robert Joseph Barro} und \textsc{Thomas Sargent}  mitumfasst, wenn man von "`Neoliberalismus"' spricht. Viertens -- wobei hier die Abgrenzung nicht sehr scharf ist -- die Vertreter der "`Neuen Institutionenökonomik"' um \textsc{Oliver Williamson} und \textsc{Gary Stanley Becker}. Und, fünftens, die Interpretation von \textsc{James McGill Buchanan} der  "`Neuen Politische Ökonomie"'.

Die allermeisten der genannten Schulen wurden in diesem Buch bereits erwähnt, aber an sehr verschiedenen Stellen! Das gemeinsame aller dieser Schulen ist gar nicht so leicht zu finden. Klarerweise würde man alle als "`wirtschaftsliberal"' und  "`marktgläubig"' einschätzen. Allerdings ist dies eine eher oberflächliche Einteilung.


Dies drückt auch das Problem des "`Neoliberalismus"' aus: Er ist in dem Sinn keine geschlossene wissenschaftliche Schule. Er ist eher ein ideologischer Begriff, und zwar ein ideologischer Begriff für seine Gegner.
Ein Beispiel zeigt wie uneinheitlich der Begriff ist: \textsc{Milton Friedman} ist eine Ikone der Liberalen. Die einzige Aufgabe eines Unternehmens sei es Profite zu machen, keine Spur von "`Corporate Social Responsibility"'. Der Staat soll sich aus dem Leben der Menschen - auch wirtschaftlich - weitgehend zurückziehen. Aber: In seinem wirtschaftswissenschaftlichen Hauptwerk \textit{A Monetary History of the United States} \parencite{Friedman1962} kritisiert er, dass die US-Notenbank Fed die Geldmenge während der "`Great Depression"' in den 1930er Jahren nicht ausgeweitet hat und somit die Wirtschaftskrise in ihrer vollen Härte erst ermöglicht hat.
\textsc{Friedrich Hayek} hingegen kritisierte jeglichen Eingriff in den freien Markt und war der Ansicht, dass die Zentralbanken privatisiert werden sollten, die Geldversorgung also von privaten Unternehmen durchgeführt werden sollte. Friedman kritisierte Hayek noch 1999 vehement für diese Ansicht: "`I think the Austrian business-cycle theory has done the world a great deal of harm [in the 1930s]."' \parencite{Epstein1999} Manchmal wurde der Monetarismus als dem Keynesianismus näher gesehen als der Österreichischen Schule. Das Verhältnis zwischen den "`Österreichern"' und "`Chicago"'\parencite{Skousen2005} ist nicht so eindeutig positiv, als dass man beide unter den gemeinsamen Hut "`Neoliberalismus"' vereinigen könnte. Man sieht: Innerhalb des Begriffs des "`Neoliberalismus"' lässt sich viel vereinen was bei näherer Betrachtung recht unterschiedlich ist. Für Gegenspieler ist dies natürlich interessant: Mit einem einzigen Begriff umfasst man alle ökonomischen Schulen, deren Inhalte man ablehnt.


Aus heutiger Sicht ist schon alleine der Begriff "`Neoliberalismus"' missglückt. Wie sogar die Mont-Pelerin-Gesellschaft -- auf diese kommen wir später zurück -- in ihrer Gründungserklärung hinweisen muss, kommt es beim Wort "`liberal"' zu Verständnis-Schwierigkeiten. Gemeint ist die europäische Bedeutung von liberal: Maximale persönliche Freiheit und möglichst wenig Staatseingriffe. In der amerikanischen Bedeutung wird das Wort paradoxerweise für tendenziell genau das Gegenteil verwendet (siehe: https://www.montpelerin.org/statement-of-aims/). Das Wortpaar "`links"' und "`rechts"' im europäischen Sinn würde man in den USA am ehesten mit "`liberal"' und "`conservative"' übersetzen. Ich kann mich noch erinnern wie verwirrt ich als junger Student in Wien war als ich 2007 vom Buch "`Conscience of a Liberal"' von Paul Krugman hörte. Ich hatte Krugman immer als eher linken Ökonomen eingeschätzt. Nachdem ich im Buch schmökerte hatte ich den Titel noch eine Zeit lang als ironisch gemeint interpretiert - aus heutiger Sicht peinlich. Erst etwas später wurde ich über das europäisch - amerikanische Missverhältnis des Begriffs "`liberal"' aufgeklärt. "`Liberal"' steht in den USA also für eher linke politische und wirtschaftliche Ansichten, als genau das, was die sogenannten Neoliberalen ablehnen.

Die Entstehungs-Geschichte des Neoliberalismus ist interessant. Der US-amerikanische Journalist Walter Lippmann beschrieb in seinem 1937 erschienenen Buch  "`The Good Society"' \parencite{Lippmann1937} die Zukunft und die Erneuerung des Liberalismus. Und zwar des Liberalismus im politischen Sinn: An vorderster Stelle steht die individuelle Freiheit, diese kann es nur in Demokratien geben.  Das historische Umfeld ist bekannt: Die Weltwirtschaftskrise "`Great Depression"' war zwar weitgehend überwunden, allerdings waren die wirtschaftlichen Spätfolgen noch spürbar und politisch haben in vielen kontinentaleuropäischen Staaten Alleinherrscher die Macht übernommen. Der Liberalismus war zu dieser Zeit also weitgehend gescheitert: Von "`rechts"' fanden nationalistisch-faschistische Regime regen Zulauf, von "`links"' marxistisch-sozialistische Regime. In diesem Umfeld trafen sich Ende August des Jahres 1938 in Paris 26 -- heute würde man sagen Intellektuelle --  zum \textsc{Colloque Lippmann}. Die Teilnehmer waren allesamt überzeugte Liberale. 
Dazu zählten neben \textsc{Friedrich Hayek} und \textsc{Ludwig Mises} auch \textsc{Michael Polanyi, Wilhelm Röpke} sowie unter anderen \textsc{Alexander Rüstow}. Im Umfeld sich ausbreitender rechtsextremer und linksextremer Systeme wurde bei diesem Kolloquium primär hinterfragt was der Liberalismus -- in Anbetracht der düsteren wirtschaftlichen und politischen Situation in Europa -- falsch gemacht habe und wie der Liberalismus wieder federführend werden könnte. Einer der Wortführer war \textsc{Alexander Rüstow}. Unter anderem setzte sich sein Namensvorschlag -- eben "`Neoliberalismus"' -- gegenüber anderen Vorschlägen durch. Wahrscheinlich deshalb wird des \textsc{Colloque Lippmann} als Geburtsstunde des "`Neoliberalismus"' angesehen \parencite{Horn2018}.
Das die individuelle Freiheit zentrales Merkmal auch im Neoliberalismus bleiben musste war klar. Wie die Ökonomie dieser Schule ausgestaltet werden sollte, war aber schon bei seiner Geburtsstunde höchst umstritten:  Zwar vertraten \textsc{Friedrich Hayek} und \textsc{Ludwig Mises} natürlich schon beim Colloque Lippmann ihre wirtschaftsliberalen Thesen, aber es gab auch deutlichen Widerspruch. \textsc{Alexander Rüstow} und auch \textsc{Wilhelm Röpke} forderten etwa -- aus heutiger Sicht widersprüchlich wenn man von "`Neoliberalismus"' spricht -- einen "`starken Staat"'. Wirtschaftsliberal sollte in ihrem Sinn heißen, dass zwar tatsächlich die erbrachte Leistung in einem Umfeld individueller Freiheit über den persönlichen Erfolg entscheiden sollte, die \textit{Gestaltung des Umfelds} sollte aber durch einen starken Staat gesichert werden. So sollte es demnach ausgeprägte staatliche Institutionen geben. Insbesondere Marktmacht durch Monopolbildung \parencite[S. 69ff]{Hegner2000}, aber auch Vermögensbildung durch Vererbung \parencite[S. 58ff]{Hegner2000} großer Besitztümer sollten laut Rüstow verhindert werden. 

Die Unstimmigkeiten bezüglich der wirtschaftlichen Ausrichtung des Neoliberalismus gab es also schon von Beginn an. Offensichtlicher wurde dies beim zweiten wichtigen Event in der Geschichte des Neoliberalismus: Der Gründung der Mont Pèlerin Gesellschaft im Jahr 1947. Meines Erachtens nach wurde der "`Neoliberalismus"' -- zumindest in der Form wie er heute oft verstanden wird, nämlich eben als ökonomische Schule -- bei diesem Treffen in den schweizerischen Alpen begründet. Betrachtet man die Liste der Teilnehmer, sieht man, dass jetzt eben schon der \textsc{wirtschaftliche} Liberalismus im Vordergrund stand: Von den 39 Teilnehmern waren diesmal die meisten Ökonomen, unter anderem Friedrich Hayek, Milton Friedman, Ludwig von Mises, Walter Eucken, Wilhelm Röpke, George Stigler und Maurice Allais \parencite[S. 12--19]{Mirowski2009}. Letztgenannter war übrigens der einzige, der die bereits erwähnte Gründungserklärung "`(Statement of Aims"'), die im Rahmen des Mont-Pelerin-Treffens verfasst wurde, nicht unterzeichnete\parencite[S. 57]{Mirowski2009}. Auch abgesehen davon kam es auch während des Treffens rasch zu Unstimmigkeiten bezüglich der Ausrichtung. 

Bis hierher kann man durchaus vom Neoliberalismus als eine einheitliche, ökonomische Schule sprechen. Die erste Spaltung vollzog sich aber schon in den 1960er Jahren. Innerhalb der Mont-Pelerin-Gesellschaft tat sich schon bald ein Spalt auf zwischen den deutschen Vertretern um Wilhelm Röpke und Alexander Rüstow auf der einen Seite und den Vertretern der österreichischen Schule -- vor allem Friedrich Hayek und Ludwig von Mises -- sowie der aufstrebenden Chicago School -- damals vor allem noch um Frank Knight auf der anderen Seite. Die deutschen \textsc{Ordoliberalen} hatten zwar beim Colloque Lippmann und auch bei der Gründung der Mont-Pelerin Gesellschaft einen wesentlichen Einfluss, dieser schwand aber zunehmend. Die Gegensätze zur Österreichischen Schule waren die gleichen wie schon beim Colloque Lippmann. Der Ordoliberalismus wollte einen Staat, in dem die Spielregeln für den Markt klar definiert und durch Institutionen kontrolliert wurden. Die deutschen Vertreter hatten innerhalb der Mont-Pelerin Gesellschaft den Vorteil tatsächlich politischen Einfluss zu haben. Walter Eucken, Gründungsmitglied der Mont Pelerin Gesellschaft, galt als Vater der "`Sozialen Marktwirtschaft"', der erfolgreichen Wirtschaftsordnung des Nachkriegsdeutschland, er stand zeitlebens Friedrich Hayek sehr nah, wenn auch die inhaltlichen Übereinstimmungen zwischen den beiden nicht besonders hoch waren. Eucken starb bereits 1950. Innerhalb des Ordoliberalismus übernahm Alfred Müller-Armack die zentrale Stelle. Auch Röpke und Rüstow waren dieser Strömung zuzurechnen. Der starke Staat im Ordoliberalismus, aber auch aufkommende nationalistische Ideen (siehe unten) im Denken Wilhelm Röpkes, waren mit den Überzeugungen der Österreicher und Amerikaner innerhalb der Mont-Pelerin Gesellschaft nicht vereinbar. Wilhelm Röpke wurde zwar 1961 noch der erste Präsident der Mont Pelerin Gesellschaft nach Gründungspräsidenten Friedrich Hayek, aber kurz darauf wurden die Differenzen zwischen ihm und Hayek unüberwindbar und im Rahmen der sogenannten Hunold-Affäre trat Röpke aus der Mont Pelerin Gesellschaft aus. Die Ordoliberalen Ideen hatten in weiterer Folge keine Zukunft mehr innerhalb der Mont Pelerin Gesellschaft. Die erste Abspaltung im Neoliberalismus war vollzogen. Bis heute wird der Ordoliberalismus -- praktisch äquivalent mit der \textsc{Freiburger Schule} -- häufig auch als Neoliberalismus bezeichnet.

In der Mont-Pelerin Gesellschaft gab es ab Mitte der 1960er Jahre noch zwei zentrale Strömungen: Die Österreichische Schule um Hayek und die aufstrebende Chicago School. Was waren die zentralen Themen zu dieser Zeit? Erstens, politisch die "`Zweiteilung"' der Welt in den kapitalistischen Westen und den sozialistischen Osten. Der Kommunismus mit seinem Kollektivismus und zentraler Planwirtschaft war der gemeinsame Gegner der Mont-Pelerin-Gesellschaft. Zweitens, wirtschaftlich der Aufstieg des Keynesianismus. Die neoklassische Synthese als Kombination der Ideen der Neoklassik mit jenen von Keynes -- häufig als Keynesianismus bezeichnet -- übernahm rasch die Rolle der Mainstream-Ökonomie. Auch das war ein gemeinsamer Gegner. Die Mont-Pelerin-Gesellschaft sah sich geradezu als Gegenpol zu den Keynesianern. Drittens, aus ökonomischer Sicht hatte die Schaffung eines zentralen Währungssystems nach dem Zweiten Weltkrieg zentrale Bedeutung. Bis zur "`Great Depression"' galt der klassische Goldstandard als der Goldstandard der Währungssysteme ;-). Während der Krise konnte das System aber nicht mehr aufrecht gehalten werden und wurde nach und nach von den verschiedenen Staaten aufgekündigt. Aus heutiger Sicht ist das überraschend, aber eine Welt ohne fixe Wechselkurse zwischen den Währungen galt damals als unvorstellbar. Und so wurde noch während des Zweiten Weltkriegs im Jahr 1944 in der US-amerikanischen Stadt Bretton Woods das gleichnamige Währungssystem geschaffen. Ebenfalls ein auf Gold basierendes Wechselkurssystem, allerdings mit dem US-Dollar als Referenzwährung. In Bezug auf das internationale Währungssystem waren sich die Neoliberalen von Anfang an uneinig. Zwar lehnten alle das \textsc{Bretton-Woods-System} aufgrund der darin notwendigen, aber illiberalen Kapitalverkehrskontrollen ab, allerdings war man sich über Alternativen uneinig.
Die Vertreter der österreichischen Schule, insbesondere Mises, waren Verfechter des klassischen Goldstandards. Auch Hayek plädierte zunächst für den Goldstandard, als "`zweitbeste Lösung"'. Er wendet sich später der "`Privatisierung der Geldversorgung"' zu. Beide -- Mises und Hayek -- aber bleiben Zeit ihres Lebens Verfechter fixer Wechselkurse \parencite[S. 256]{Kolev2017} Die Chicago School hingegen hat kein Problem mit den flexiblen Wechselkursen. Im Gegenteil, Friedman bestärkte die US-Regierung in der Aufkündigung des "`Bretto-Woods-Vertrags"' und prophezeite eine stabilisierende Wirkung flexibler Wechselkurse. Den "`Goldstandard"' lehnte die Chicago-School strikt ab. Die Monetaristen um Milton Friedman waren überzeugt von der Überlegenheit von \textsc{Fiat-Geld} und unabhängigen Zentralbanken, die idealerweise für eine stabil wachsende Geldversorgung sorgen sollten.
Mit dem Ende des "`Bretton-Woods-Systems"' Anfang der 1970er Jahre nahm die Bedeutung der Geldpolitik zu. Die Unterschiede zwischen "`Chicago-School"' und "`Austrians"' wurden damit unübersehbar. Beide zusammenzufassen als "`Neoliberalismus"' ist unscharf. Die österreichische Schule war in Belang auf die Geldpolitik zu idealistisch. Jede Geldmengenerhöhung, so die "`Austrians"' bedeutet Inflation. Immer wenn der tatsächliche Zinssatz unter den natürlichen Zinssatz (im Sinne Wicksells) liegt, kommt es zu einem Boom der schlussendlich in einer Krise enden muss. Das klingt in der Theorie überzeugend, ist aber wirtschaftspolitisch nicht umsetzbar. Denn wie hoch ist der natürliche Zinssatz heute? Wie argumentiert man gegenüber dem unter Arbeitslosigkeit leidenden Wahlvolk, wenn man in einer Wirtschaftskrise nicht lenkend eingreift, sondern stattdessen auf "`Gesundschrumpfen"' setzt? Auch in Bezug auf die Methodik spielte die Zeit in den 1970er Jahren gegen die österreichische Schule: Quantitative Methoden, die mittels historischer Daten empirisch überprüft werden, werden von den "`Austrians"' abgelehnt. Genau diese Ansätze aber kamen in den 1970er Jahren in Mode, nicht zuletzt wegen der steigenden Verfügbarkeit empirischer Daten. 
Zwar schrieb \textcite[S. 102]{Kirzner1967}, dass man die Unterschiede zwischen der "`Österreichischen Schule"' und der "`Chicago-School"' nicht überschätzen soll, schließlich stimmen beide in den wesentlichen Fragen wirtschaftlichen Fragen überein, aber mit fortlaufender Zeit wurden die Unterschiede deutlicher. Wenn man so will kann man Anfang der 1970er Jahre von einer zweiten Spaltung im Neoliberalismus sprechen. Wobei die Österreichische Schule in Wirklichkeit in den 1970er Jahren in der Bedeutungslosigkeit versank. Mises starb 1973. \textsc{Hayek, Gottfried Haberler, Fritz Machlup} und \textit{Oskar Morgenstern} wurden alle um die Jahrhundertwende geboren und hatten ihren schöpferischen Zenit überschritten. Nur Hayek erlebte just in dieser Zeit seinen zweiten Frühling. Überraschend wurde er 1974 mit dem Nobelpreis für Wirtschaftswissenschaften ausgezeichnet. Aus wissenschaftlicher Sicht aber blieb im Neoliberalismus ab Anfang der 1970er Jahre ausschließlich die Chicago-School übrig. In dieser kurzen Phase -- zwischen Ölpreisschock 1973 und Lucas-Kritik 1976 -- war die Chicago-School die einzige wesentliche neoliberale Schule und gleichzeitig aus wissenschaftlicher Sicht auf ihrem Zenit.
Aus wirtschaftspolitischer Sicht folgte die Hochzeit des Neoliberalismus ab Anfang der 1980er Jahre. Hayek -- wie soeben erwähnt -- gewann, als seine Karriere als immerhin 75-jähriger schon am abklingen war, noch einmal wirtschaftspolitischen Einfluss. \textit{Margaret Thatcher} zog ihn als wirtschaftspolitischen Berater in Großbritannien heran und auch -- natürlich wesentlich umstrittener -- \textit{Augusto Pinochet} in Chile. Milton Friedman hatte schon bei der Abschaffung fixer Wechselkurse entscheidenden Einfluss. Er spielte auch als \textit{Ronald Reagan's} Berater in den USA eine entscheidende beratende Rolle. Die Zeit von "`Thatcherism"' und "`Reaganomics"'  gelten heute aus politischer Sicht als Hochzeit des Neoliberlismus. Nachhaltiger -- und wissenschaftlich fundierter -- war aber Friedman's Einfluss auf die Zentralbanken:  Die Deutsche Bundesbank etwa bezog sich direkt auf seinen Monetarismus und richtete ihre Politik ab 1975 an der Geldmengensteuerung aus (von 1975 bis 1987 an der Steuerung der Zentralbankengeldmenge) \parencite[S. 36]{BBK2016}. 




Der Keynesianismus galt durch die Stagflation in Folge des Ölpreisschocks aus der Mode. Die Lucas-Kritik wurde erst 1976 veröffentlicht.



Man sieht also: Ursprünglich hat der "`Neoliberalismus"' durchaus eine gemeinsame Wurzel, die auch eindeutig zurückgeführt werden kann (nämlich auf das Mont Pelerin Treffen, oder das Colloque Lippmann). Warum er heute dennoch nicht mehr im wissenschaftlichen Diskurs als Begriff verwendet wird sieht man aber auch am eben ausgeführten: Von Anfang an herrschte innerhalb der Teilnehmer Uneinigkeit über die wirtschaftliche Ausprägung des Neoliberalismus. Als erstes spaltete sich die deutsche Gruppe als Ordoliberalismus ab. Später kam es zu deutlichen Unterschieden zwischen der österreichischen Schule und der Chicago School. Bis in die Gegenwart sind die Unterschiede in den Ideen der Vertreter marktliberaler Ideen zu unterschiedlich als dass man tatsächlich von einer geschlossenen Schule des Neoliberalismus sprechen kann.



Der Neoliberalismus und die Mont-Pelerin-Gesellschaft im Speziellen wird in vielen Publikationen meines Erachtens als zu einflussreich beschrieben. Die hohe Anzahl der Nobelpreisträger, die auch Mitglied der Mont-Pelerin-Gesellschaft sind, sei darauf zurückzuführen, dass diese Gruppe indirekt auf die Vergabe Einfluss nehme, liest man hier. Oder, dass die führende Rolle des Keynesianismus durch die Mont-Pelerin-Gesellschaft untergraben wurde und die politischen Akteure auf die Seite des Neoliberalismus gezogen wurden, der noch dazu immer marktradikaler wurde. Vieles, dass man hier liest und im TV sieht mutet eher übertrieben -- ja fast wie Verschwörungstheorien -- an. Eines stimmt aber zweifellos: Es ist den Personen um Friedrich Hayek -- schlussendlich die führende Person in der Mont Pelerin Gesellschaft, wenn nicht im gesamten Neoliberalismus -- gelungen, ihn zu einem führenden Ökonomen hochzustilisieren. Fast schon legendär sind die "`Rap-Battles"' zwischen Keynes und Hayek. In Wirklichkeit ist der wirtschafts-wissenschaftliche Beitrag Hayeks nicht zu vergleichen mit jenem von beispielsweise John Maynard Keynes. Während seiner wissenschaftlichen Karriere war er kein führender Kopf an einer der führenden Universitäten. Zwar lehrte er an der damals extrem einflussreichen Universität Chicago, dort waren aber andere, allen voran Milton Friedman, die Taktgeber. Auch innerhalb der österreichischen Schule gilt nicht Hayek, sondern vielmehr Mises als der Hauptvertreter. Dessen wissenschaftliches Werk hatte auch nachhaltigeren Eindruck in der ökonomischen Gesellschaft. Und nicht zuletzt musste Hayek selbst eingestehen, dass seine Krisenbekämpfungstheorie zur "`Great Depression"', nämlich am besten nicht zu intervenieren, einfach falsch war. Die Ökonomie des 20. Jahrhunderts als einen Zweikampf zwischen Keynes und Hayek darzustellen, stellt dessen Einfluss sehr übertrieben dar.







\section{Ordoliberalismus}

Eucken und Röpke
Der Begriff "`Neoliberalismus"' bezog sich ursprünglich auf die deutsche ...
Die ursprüngliche Bedeutung -- und wie wir gesehen haben -- auch die einzige 
Freiburger Schule (dort oft auch Hayek dazugezählt)


Sehr pragmatischer Zugang: GmbH kritisch gesehen, weil das Risiko einseitig beschränkt. Für Erbschaftssteuern, weil eben \textit{wirklich} an Chancengleichheit interessiert. Im Zentrum stand auch eine Anti-Kartell-Ordnung.
Also: Marktorientiert und gegen den Keynesianismus. Schlanker Staat, aber auch starke Regulierungen!
Nicht formalisiert. 

Heutiger Vertreter zum Teil: Hans-Werner Sinn wenig formalisiert, sehr pragmatischer realistischer Zugang


%%%%%%%%%%%%%%%%%%%%%% chapter.tex %%%%%%%%%%%%%%%%%%%%%%%%%%%%%%%%%
%
% sample chapter
%
% Use this file as a template for your own input.
%
%%%%%%%%%%%%%%%%%%%%%%%% Springer-Verlag %%%%%%%%%%%%%%%%%%%%%%%%%%

\chapter{Post-Keynesianismus}
\label{Post-Keynes}


\section{Kalecki: Post-Keynesianer vor Keynes}

\section{Harrod \& Domar: Das erste Wachstumsmodell}
Man kann auf jeden Fall darüber streiten, ob das Harrod-Domar-Modell hier richtig angesiedelt ist oder nicht doch besser im "`Mainstream-Teil"' dieses Buches als keynesianisches Wachstumsmodell anzusiedeln wäre. Faktum ist, dass es nach dem Zweiten Weltkrieg das erste anerkannte, modell-theoretische Wachstumsmodell war. Entsprechend dem damaligen Zeitgeist ein typisch keynesianisches Modell, also eigentlich dem damaligen Mainstream entsprechend. Faktum ist aber auch, dass das Harrod-Domar-Modell mit dem Aufkommen der exogenen Wachstumsmodelle (vgl. Kapitel \ref{sec: Solow-Modell}) Mitte der 1950er Jahre, rasch an Bedeutung verlor. In der Neoklassischen Synthese - also der Verschmelzung von Teilbereichen von Keynes' General Theorie mit der Neoklassik - war kein Platz für das Harrod-Domar-Modell und die neuen Mainstream-Ökonomen erklärten es rasch als widerlegt, beziehungsweise als sehr unwahrscheinlichen Spezialfall \parencite{Solow1987}.

HIER WEITER: Snowdon/Vane S. 598

Harrod-Domar Modell (später Kontroverse Solow - Robinson!)



\section{Kaldor: Der Begründer des Post-Keynesianismus}
Durch die Ablehnung des IS-LM-Modells.

\section{Myrdal \& Sraffa}

\section{Linkskeynesianismus: Lernen \& Robinson}

Vergleiche Kapitel \ref{Neoklassik}:

Robinson und Lerner erkannten als erste die Notwendigkeit die Nutzentheorie von Pareto nochmal aufzugreifen
"`War of the two Cambridges"' \parencite{Tobin1985}

Kapitalkontroverse oder die Kontroverse der zwei Cambridges

Robinson, Economics of Imperfect Competition, p. 256; Lerner, "Elasticity of Substitution" (Review of Economic Studies, Oct. 1933, pp. 68-70

\section{Minsky: Die Absurdität des Finanzmarktgleichgewichts}


%%%%%%%%%%%%%%%%%%%%%% chapter.tex %%%%%%%%%%%%%%%%%%%%%%%%%%%%%%%%%
%
% sample chapter
%
% Use this file as a template for your own input.
%
%%%%%%%%%%%%%%%%%%%%%%%% Springer-Verlag %%%%%%%%%%%%%%%%%%%%%%%%%%

\chapter{Exkurs: Behavioral Economics}
\label{Behavioral}


Die Kritik am rational denkenden Menschen, den Homo Oeconomicus ist im wahrsten Sinne so alt wie das Konzept selbst. Der moderne Homo Oeconomicus ist einer, der seinen Nutzen im Sinne einer "`Von Neumann-Morgenstern-Nutzenfunktion"' \parencite{VonNeumann1944} maximiert. Diese hat den Vorteil den Nutzen mit quasi-kardinalen Werten angeben zu können. Nutzen kann seither quantifiziert werden, also mit konkreten Werten, unterlegt werden. Erst dadurch kann man den Nutzen mit Wahrscheinlichkeitswerten hinterlegen und zu Nutzenerwartungswerten weiterentwickeln. Laut \textcite[S. 3]{Selten2001} waren sich Oskar Morgenstern und John von Neumann dieser Schwäche ihres Konzepts nicht nur bewusst, sie nahmen dieses auch durchaus ernst.




\section{Allais: Es begann mit einem Paradoxon}

\section{Simon: Bounded Rationality}
Verweis auf Kapitel Institutionalismus: Simon, H.A., (1957), Models of Man: Social and Rational, New York- Wiley.
Simon (1955): Annahme vollständiger Rationalität wenig Zweckmäßig --> Bounded Rationality

\section{Kahneman und Tversky}
Prospect Theory

\section{Thaler}


%%%%%%%%%%%%%%%%%%%%%%%% part.tex %%%%%%%%%%%%%%%%%%%%%%%%%%%%%%%%%%
%
% sample part title
%
% Use this file as a template for your own input.
%
%%%%%%%%%%%%%%%%%%%%%%%% Springer-Verlag %%%%%%%%%%%%%%%%%%%%%%%%%%


\part{Was bringt die Zukunft?}
		% Zukunft
%%%%%%%%%%%%%%%%%%%%%% chapter.tex %%%%%%%%%%%%%%%%%%%%%%%%%%%%%%%%%
%
% sample chapter
%
% Use this file as a template for your own input.
%
%%%%%%%%%%%%%%%%%%%%%%%% Springer-Verlag %%%%%%%%%%%%%%%%%%%%%%%%%%

\chapter{Kritik am Finanzmarkt}
\label{Finanzmarkt}


\section{Shiller}

               % eher streichen!
%%%%%%%%%%%%%%%%%%%%% chapter.tex %%%%%%%%%%%%%%%%%%%%%%%%%%%%%%%%%
%
% sample chapter
%
% Use this file as a template for your own input.
%
%%%%%%%%%%%%%%%%%%%%%%%% Springer-Verlag %%%%%%%%%%%%%%%%%%%%%%%%%%

\chapter{Was lernte der Mainstream aus den Krisen}
\label{Krise}

Great Recession
und Covid19
Buch: What have we learned?
\section{Rajan}



%%%%%%%%%%%%%%%%%%%%% chapter.tex %%%%%%%%%%%%%%%%%%%%%%%%%%%%%%%%%
%
% sample chapter
%
% Use this file as a template for your own input.
%
%%%%%%%%%%%%%%%%%%%%%%%% Springer-Verlag %%%%%%%%%%%%%%%%%%%%%%%%%%

\chapter{Ungleichheit als wachsendes Problem}
\label{Ungleichheit}

Ricardo.
Produktionsfaktoren

funktionale vs. personelle Einkommensverteilung
Vermögensverteilung vs. Einkommensverteilung
Woher kommt das:
\textcite[S. 1620]{Aghion1999}
\textcite[S. 359]{Alesina1994a}
Die Anfänge: Bowley's Law
Kaldor und Lewis \textcite[S. 557]{Snowdon2005}


DAs in das Kapitel Ungleichheit:
Ungleichheit und Wachstum: Kaldor und Kuznets \parencite{Alesina1994a} und \parencite[S. 556]{Snowdon2005}: Grundsätzlich positiver Zusammenhang: Mehr Ungleichheit (in der Form der funktionalen Einkommensverteilung) führt zu mehr Kapitalakkumulation und daher zu mehr Wachstum. Ansatz der politischen Ökonomie: Zunächst \textcite{Hirschman1973}: Toleranz gegen Einkommensungleichheit. Ab gewissem Level fällt diese Toleranz. Schließlich: Politische Instabiliät.
Ansatz von Alesina et al.: Hohe Ungleichheit: Median-Wähler hat hohes Bedürfnis nach Umverteilung. Hohe Steuerlast reduziert Investitionen und drückt das Wachstum.

\textcite{Deininger1996}


%\section{Atkinson}

%\section{Piketty}
Great Recession



%%%%%%%%%%%%%%%%%%%%% chapter.tex %%%%%%%%%%%%%%%%%%%%%%%%%%%%%%%%%
%
% sample chapter
%
% Use this file as a template for your own input.
%
%%%%%%%%%%%%%%%%%%%%%%%% Springer-Verlag %%%%%%%%%%%%%%%%%%%%%%%%%%

\chapter{Armut als Herausforderung}
\label{Armut}

\section{Sen: Die Ökonomie der Wohlfahrt}

% \section{Deaton}

\section{Experimente gegen die Armut: Banerjee, Duflo \& Kremer}



%%%%%%%%%%%%%%%%%%%%%% chapter.tex %%%%%%%%%%%%%%%%%%%%%%%%%%%%%%%%%
%
% sample chapter
%
% Use this file as a template for your own input.
%
%%%%%%%%%%%%%%%%%%%%%%%% Springer-Verlag %%%%%%%%%%%%%%%%%%%%%%%%%%

\chapter{Zukunft ohne Zins?}
\label{Zins}

\section{Die weltweite Liquiditätsfalle}





%%%%%%%%%%%%%%%%%%%%% chapter.tex %%%%%%%%%%%%%%%%%%%%%%%%%%%%%%%%%
%
% sample chapter
%
% Use this file as a template for your own input.
%
%%%%%%%%%%%%%%%%%%%%%%%% Springer-Verlag %%%%%%%%%%%%%%%%%%%%%%%%%%

\chapter{Ökonomie außerhalb der Demokratie?}
\label{Demokratie}

%\section{Weltmacht China: ökonomische Aufstieg ohne demokratische Grundregeln}

Acemoglu vs. Milanovic




%%%%%%%%%%%%%%%%%%%%% chapter.tex %%%%%%%%%%%%%%%%%%%%%%%%%%%%%%%%%
%
% sample chapter
%
% Use this file as a template for your own input.
%
%%%%%%%%%%%%%%%%%%%%%%%% Springer-Verlag %%%%%%%%%%%%%%%%%%%%%%%%%%

\chapter{Umweltfragen als existenzielle Fragen}
\label{Umwelt}

Preis-Standard Ansatz

Ansatz von Nordhaus

%\section{Ostrom}






\backmatter%%%%%%%%%%%%%%%%%%%%%%%%%%%%%%%%%%%%%%%%%%%%%%%%%%%%%%%

\printbibliography  
\printindex


%%%%%%%%%%%%%%%%%%%%%%%%%%%%%%%%%%%%%%%%%%%%%%%%%%%%%%%%%%%%%%%%%%%%%%

\end{document}





