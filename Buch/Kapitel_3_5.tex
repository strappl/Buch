%%%%%%%%%%%%%%%%%%%%% chapter.tex %%%%%%%%%%%%%%%%%%%%%%%%%%%%%%%%%
%
% sample chapter
%
% Use this file as a template for your own input.
%
%%%%%%%%%%%%%%%%%%%%%%%% Springer-Verlag %%%%%%%%%%%%%%%%%%%%%%%%%%

\chapter{Neue Klassische Makroökonomie}
\label{Neue Makro}

In den 1970er Jahren steckte die ökonomische Forschung in der Sackgasse. Der Keynesianismus (in der Form der neoklassischen Synthese) galt, zumindest im angelsächsischen Raum, spätestens seit der Ölkrise von 1973 als überholt. Die Zeit spielte schon lange gegen die österreichische Schule: Ihre liberalen Theorien waren zu radikal um sie wirtschaftspolitisch umzusetzen. Und ihre konkreten Befürchtungen, etwa in Hinsicht auf internationale Geldsysteme erwiesen sich als zu pessimistisch. Zum Beispiel waren die Währungssysteme auch ohne Goldstandard relativ stabil. Zudem galt ihre Methodik als widerlegt: Mathematik und empirische Forschung setzten sich in der Wissenschaft immer mehr durch. 

Aber auch Milton Friedman's Monetarismus galt in seiner Grundidee als gescheitert: Der Monetarismus wurde durchaus wirtschaftspolitisch praktiziert. Aber die Konzentration auf Geldpolitik hatte nicht die erwartete stabilisierende Wirkung auf die Preise.

Wer aber dachte das Pendel würde nach dem wirtschaftsliberalen Monetarismus wieder nach links, in Richtung Keynesianismus ausschlagen, täuschte sich. Die Lucas-Kritik kam wie ein Paukenschlag über die Ökonomie und damit nichts anderes als Rückbesinnung auf die "`(Neo-)Klassik"'.


Interessant und wichtig ist der Unterschied im vorherrschenden ökonomischen Denken zwischen der wirtschafts-wissenschaftlichen Community und den Entscheidungsträgern in der Wirtschaftspolitik. Gerade in den 1970er und 1980er Jahren gab es hier einen beträchtlichen Time-Lag. Direkt nach dem Zweiten Weltkrieg wandten sich die meisten Wirtschaftswissenschafter dem Keynesianismus zu. Die Österreichische Schule lehnte dessen Ideen hingegen von Anfang an strikt ab und war nach 1945 noch ein wissenschaftlich bedeutender Gegenpol.\footnote{Mit dem Tod von Schumpeter und Mises und dem Aufkommen der Chicago School in der Makroökonomie, vor allem mit Milton Friedman's Monetarismus, verlor die Österreichische Schule in der akademischen Welt rasch an Bedeutung.} Der Keynesianismus (in der Form der neoklassischen Synthese) war hingegen unumstritten bis Mitte der 1960er Jahre die dominierende Schule. Spätestens mit dem Schock der Ölkrise 1973 verlor er aber diesen Status als "`State-of-the-Art"' in der Ökonomie. 
Schon Mitte der 1960er Jahre begann der wissenschaftliche Aufstieg des Monetarismus. Milton Friedman's und Anna Jacobson Schwartz' Werk \textit{A Monetary History of the United States, 1867–1960} im Jahr 1963 kann als wissenschaftlicher Grundstein des Monetarismus gesehen werden, der Höhepunkt erfolgte gegen Ende der 1960er Jahre. Und schon Mitte der 1970er Jahre - hier ist die Veröffentlichung der Lucas-Kritik im Jahr 1976 ein typischer zeitlicher Startpunkt - wurden die beiden vorher genannten Schulen schön langsam abgelöst, bzw. eigentlich im Endeffekt erweitert, von der Neuen Klassischen Makroökonomie.

In Bezug auf realpolitische Umsetzung kam die Zeit der liberalen Schulen erst später. Bis in die späten 1960er Jahre war der die Neoklassische Synthese sowohl unter demokratischen als auch republikanischen Präsidenten als wissenschaftliche Grundlage ihres wirtschaftspolitischen Handelns anerkannt \parencite[S. 12]{Woodford1999}. Das änderte sich in den 1970er Jahren: Die Wirtschaft der USA war in den 1970er Jahren von einer Berg-und-Talfahrt geprägt. Häufig spricht man in dieser Phase von "`Staglation"', also niedriger BIP-Wachstumsraten, bei permanent hoher Inflation. Genau diese wurde zunehmend als Problem wahrgenommen. Interessanterweise läutete ein demokratischer Präsident, nämlich Jimmy Carter, die Hinwendung zu liberaler Wirtschaftspolitik ein. Er bestellte Alfred Edward Kahn zu einem wichtigen wirtschaftspolitischen Berater, der in den Folgejahren vor allem die Deregulierung und Privatisierung weiter Teile der staatlichen Industrie (Luftfahrt) vorantrieb. 1979 wurde Paul Volcker zum Vorsitzenden der US-Notenbank Federal Reserve (Fed) bestellt. In weiterer Folge wurde Inflationsbekämpfung als primäres Ziel ausgegeben. Schon mit diesen beiden Schritten hatten sich die USA wirtschaftspolitisch deutlich den Monetarismus zugewendet. Eindeutiger wurde dies in den 1980er Jahren. Die \textit{Reaganomics} von Präsident Ronald Reagen in den USA, sowie der \textit{Thatcherism} in Großbritannien der "`Eisernen-Lady"'-Premierministerin Margaret Thatcher bezogen sich offen auf die liberale Chicago-School, aber auch die Österreichischen Schule. Friedrich Hayek und Milton Friedman fungierten dabei sogar direkt als wirtschaftspolitische Berater. Der Zeit-Lag ist deutlich sichtbar: Der politische Einfluss von Hayek und Friedman erfolgte Jahre nachdem ihre ökonomischen Schulen sich in der Wissenschaft ihren Zenit erreicht hatten.

Schon ab 1976 waren die wissenschaftlichen Erkenntnisse der "`Neuen Klassischen Makroökonomie"' bekannt. Zwar setzten sich deren Erkenntnisse nach und nach durch, den enorme und vor allem direkte wirtschaftspolitische Einfluss, den Hayek und Friedman erlangte, blieb Lucas, Sargent und Co aber verwehrt.



\section{Lucas' Kritik und Sargent's Beitrag}
In Bezug auf die dogmengeschichtliche Einordnung könnte man argumentieren, die Neue Klassische Makroökonomie wäre eine Weiterentwicklung des Monetarismus. Dafür sprechen aber eigentlich nur ideologische und geografische Gründe. Die Vertreter beider Schulen, also des Monetarismus und der Neuen Klassischen Makroökonomie, sind zumindest gesellschaftspolitisch dem Liberalismus zuzuordnen. Außerdem war Robert E. Lucas Student und später Professor an der University of Chicago. Gehörte also auch dem großen und erfolgreichen Zirkel an Ökonomen an, die nach Frank Knight und Friedman in Chicago lehrten. Allerdings kritisierte Robert Lucas den Monetarismus zu vehement \parencite[S. 121]{Lucas1972}, als dass man bei der Neuen Klassischen Makroökonomie als eine Erweiterung desselben sprechen könnte.  

Entscheidend ist aber ohnehin die inhaltliche Sichtweise und hier unterscheidet sich die Neue Klassische Makroökonomie vom Monetarismus ganz entscheidend. Sie brach gleich an mehreren Stellen mit den bisherigen Usancen in der Ökonomie. Am bekanntesten ist das Beispiel der Phillips-Kurve: Der negative Zusammenhang zwischen Inflation und Arbeitslosigkeit. Keynesianische Wirtschaftspolitik akzeptierte eine hohe Inflation, weil damit eine niedrige Arbeitslosigkeit verbunden sei. Tatsächlich ließ sich dieser Zusammenhang überraschend eindeutig bis Ende der 1960er Jahre feststellen. Danach aber folgten der Ölpreisschock und Jahre der "`Stagflation"'. (HIER EVENTUELL GRAFIK EINFÜGEN) Also Jahre in denen es zwar kaum Wirtschaftswachstum, aber sowohl hohe Inflation als auch hohe Arbeitslosigkeit gab. Die Monetaristen um Milton Friedman (aber auch Edmund Phelps) griffen die Keynesianer in Bezug auf die Phillips-Kurve an und argumentierten es mache in der Theorie keinen Sinn einen langfristigen Zusammenhang zwischen der nominalen Größe Inflation und der realen Größe Arbeitslosigkeit anzunehmen. Inflation kann daher nicht kausal für niedrige Arbeitslosigkeit sein kann. Die Stagflation der 1970er-Jahre deckte die Schwächen der damaligen Makroökonomie auf. Aber auch die Monetaristen konnten die empirischen Vorgänge nicht befriedigend erklären.

Stattdessen war die Geburtsstunde der "`Neuen Klassischen Makroökonomie"' gekommen. In seiner wahrlich bahnbrechenden "`Lucas' Kritik"' \parencite[S. 19ff]{Lucas1976} zeigte er, dass es geradezu naiv sei zu glauben Arbeitnehmer würden Verträge abschließen in denen Löhne festgeschrieben werden, die aus heutiger Sicht zwar "`fair"' sind, aber durch Inflation in einem Jahr deutlich weniger Kaufkraft hätten. Ebenso naiv sei es zu glauben, dass Arbeitgeber Verträge abschließen nur weil sie wissen, dass die darin festgelegten Löhne in einem Jahr ohnehin real viel geringer seien. Nein! Beide Parteien, sowohl Arbeitnehmer als auch Arbeitgeber wissen um den Einfluss der Inflation auf die Kaufkraft Bescheid und lassen ihre entsprechenden \textit{Erwartungen} -- natürlich -- auch einfließen in die Lohnverhandlungen. Aus diesem Beispiel lassen sich auch die für die Neue Klassische Makroökonomie so charakteristischen Grundannahmen ableiten: Erstens, die "`Rationalen Erwartungen"', zweitens, die Betonung des "`natürlichen Gleichgewichts"' und der daraus folgenden Wirkungslosigkeit von Fiskal- und Geldpolitik und drittens der Mikrofundierung der Makroökonomie.

\begin{enumerate}
	\item Rationale Erwartungen: Der Begriff der "`Rationalen Erwartungen"' wurde eigentlich schon durch \textcite{Muth1961} begründet und von \textcite{Lucas1972} bereits erstmals als für die gesamte Makroökonomie gültiges Theorem vorgeschlagen. Ab Mitte der 1970er Jahre, vor alle durch die Arbeiten von \textcite{Lucas1976} und \textcite{Sargent1975}, wurde das Konzept der Rationalen Erwartungen etabliert. Seit damals gelten sie als \textit{die} wesentliche Neuerung durch die Neue Klassische Makroökonomie und zählen - obwohl nicht unumstrittenen - zum Kern der modernen Mainstream-Ökonomie. So leicht sind die "`Rationalen Erwartungen"' gar nicht abzugrenzen. Den (mikroökonomischen) \textit{Homo Oeconomicus} -- also den rational \textit{entscheidenden} Mensch -- gab es in der Ökonomie schließlich schon lange. Auch die Bedeutung von Erwartungen war nicht neu. Bei Keynes zum Beispiel wurden Änderungen der Zukunftserwartungen als "`Animal Spirits"' bezeichnet. Diese waren bei Keynes ein wichtiges Konzept, dass aber als nicht modellierbar akzeptiert wurde. Die \textit{rationalen Erwartungen} umfassen eben mehr. Während der Homo Oeconomicus nutzen-maximierend vergangenheitsorientierte Information auswertet, umfassen die Rationalen Erwartungen auch die Tatsache, dass sich Menschen auch rational verhalten was die Informationsgewinnung betrifft und entsprechend rationale "`Vorhersagen"' zur zukünftigen Entwicklung wirtschaftlicher Aspekte treffen. Das heißt, eine bestimmte Wirtschaftspolitik bestimmt auch die Erwartungen der Menschen. Änderungen der Wirtschaftspolitik führen dementsprechend auch zu Änderungen der Erwartungshaltungen. Zusammengefasst: Erstens, Menschen machen vorhersagen, ohne dabei einen systematischen Fehler zu begehen (\textcite{Lucas2013}: Interview mit Lucas: "`Die Leute sind nicht verrückt"'). Das heißt, der Staat kann seine Bewohner nicht systematisch "`austricksen"'. Zweitens, Menschen bauen alle verfügbaren Informationen und Wirtschaftstheorien in ihre Entscheidungen ein. Diese Annahme ist umstritten. Auf Beispiele herunter gebrochen bedeuten Rationale Erwartungen folgendes:
	Hohe Budgetdefizite durch expansive Fiskalpolitik führen dazu, dass die Menschen steigende Steuerbelastung erwarten und ihre Sparquoten erhöhen. Anstatt der von der Politik erhofften Konjunktur-belebenden Wirkung führt die Fiskalpolitik zu einer Verdrängung privater Ausgaben durch staatliche Ausgaben. Politiker, die eine hohe Inflation bewusst nutzen wollen um die Arbeitslosenraten zu senken, werden enttäuscht: Erwarten die Menschen höhere Inflationsraten, fordern sie in den Lohnverhandlungen eine entsprechende Abgeltung dafür, womit das bewusste Ausnutzen der hohen Inflation zugunsten niedriger Arbeitslosenzahlen scheitert. Die neue klassische Makroökonomie konnte damit elegant das konkrete Phänomen der "`Stagflation"' der 1970er Jahre erklären. 	 
	Diese beiden Beispiele zeigen den enormen Nebeneffekt der Rationalen Erwartungen: Nämlich, dass weder Fiskalpolitik noch Geldpolitik\footnote{Nur wenn die politischen Handlungen absolut unvorhergesehen erfolgen, können damit kurzfristig der erwünschte Effekt eintreten.}, einen stabilisierenden Einfluss auf die gesamtwirtschaftliche Entwicklung haben \parencite{Sargent1975, Barro1976}. Wirtschaftspolitik \textit{steuert} nicht länger die ökonomische Entwicklung, sondern ist nur ein \textit{Player} in einem Spiel zwischen Politik und den Markteilnehmern \parencite{Kydland1977}. Dies brachte die Spieltheorie in die Makroökonomie, was ebenfalls dem Zeitgeist entsprach und die Neue Klassische Makroökonomie noch "`sexier"' machte.	
	Abgesehen davon ist dies aber natürlich auch der absolute Bruch mit den Lehren des Keynesianismus.
	Die Wirkungslosigkeit jeglicher damals bekannter wirtschaftspolitischer Elemente, führt uns direkt zum zweiten wesentlichen Punkt der Neuen Klassischen Makroökonomie.
	
	\item Dynamisches natürliches Gleichgewicht: Sowohl keynesianische als auch monetaristische Modelle akzeptierten, dass Lohn- und Preisanpassungen auf den Märkten mit einer gewissen \textit{zeitlichen} Verzögerung eintraten. Eine Erhöhung der Geldmenge zum Beispiel führe demnach zunächst zu einer Erhöhung der Produktion, gleichzeitig zu einer Verringerung der Arbeitslosigkeit und erst in weiterer Folge zu höheren Nominallöhnen und höheren Preisen. Erst nach mehreren Runden dieser Anpassungsprozesse sind die Löhne und Preise wieder im Gleichgewicht.
	In der "`Neuen Klassischen Makroökonomie"' gibt es aufgrund der rationalen Erwartungen diese langfristigen Anpassungsprozesse nicht. Dementsprechend sind alle Märkte auch in der kurzen Frist im Gleichgewicht und vollständige Konkurrenzmärkte\footnote{Auf diese Annahmen ist auch der Name "`Neue \textit{Klassische} Makroökonomie"' zurückzuführen.}. 
	Dies wiederum impliziert, dass Konjunkturschwankungen ausschließlich durch exogene Schocks verursacht werden. Diese Implikation ist notwendig, da Konjunkturschwankungen ständig empirisch zu beobachten sind, bei vollständiger Konkurrenz mit flexiblen Löhnen und Preisen müssten schließlich die Lohn- und Preisanpassungen laufend zu Markträumung und Glättung der Konjunkturschwankungen führen. Die "`Real Business Cycle"'-Theorie von Edward Prescott formalisierte diese Annahmen (siehe Kapitel \ref{RBC}).
	
	Dieser zweite Kernpunkt der "`Neuen Klassischen Makroökonomie"' wurde allerdings zum "`Sargnagel"' dieser ökonomischen Schule. Anfang der 1980er Jahre sah es so aus, als würde die Neue Klassische Makroökonomie zur alleinigen Mainstream-Ökonomie aufsteigen. Aber dafür stellten sich die vorausgesetzten Annahmen als zu realitätsfern heraus. Dass alle Märkte vollkommene Konkurrenzmärkte sind und sich ständig im Gleichgewicht befinden, ist einfach zu weit weg von täglichen Beobachtungen: Erstens war diese Annahme nicht vereinbar mit der empirischen Beobachtung von Arbeitslosenraten von fast 10\% in den USA der frühen 1980er Jahre. Zweitens, zeigten der Neukeynesianer \textcite{Fischer1977} mittels formaler Modelle, dass sich Löhne und Preise auch bei Berücksichtigung der Theorie der Rationalen Erwartungen nur langsam an veränderte Arbeitslosenraten anpassen.
	Und drittens, erwiesen sich die "`Real Business Cycle"' Modelle bald als wenig treffsicher. Sie implementierten die oben beschriebenen Annahmen, dass es ständig zu Markträumung kommt und es keine natürliche Arbeitslosigkeit gibt. Die Konjunkturzyklen würden dann primär durch Schwankungen im technischen Fortschritt verursacht. Die Modelle konnten zwar vor allem methodisch überzeugen, aber die empirisch beobachtete Konjunkturschwankungen nicht erklären.
	Selbst innerhalb der Neuen Klassischen Makroökonomie akzeptierte man bald, dass die Annahmen zu starr sind und verwarf einige davon\footnote{Dies "`nutzten"' die Neu-Keynesianer, die pragmatisch die bahnbrechenden Erkenntnisse der Neuen Klassik übernahmen, aber realistischere Marktannahmen zu Märkten, Rigiditäten, Arbeitslosigkeit und Wirtschaftspolitik verwendeten}. Was exogene Wachstumsmodelle betrifft, überließen die Neuen Klassiker bald den Neu-Keynesianern das Feld. Innerhalb der Neuen Klassik wendete man sich den "`Endogenen Wachstumstheorien"' zu.
	
	\item Mikrofundierung der Makroökonomie. Die keynesianischen und monetaristischen Modelle verwendeten typische makroökonomische Kennzahlen, wie zum Beispiel Bruttoinlandsprodukt, Konsum, Investitionen und Sparen. Diese Kennzahlen wurden in Modellen in einen Zusammenhang gebracht. Zum Teil wurden einzelne dieser Kennzahlen im Laufe der Zeit abgeändert dargestellt. So erweiterten die "`permanente Einkommenshypothese"' von Milton Friedman und die "`Lebenszyklushypothese"' von Franco Modigliani die keynesianische Konsumtheorie. Aber es blieb immer bei der Heranziehung statischer makroökonomischer Faktoren. Natürlich wusste man auch vor Lucas' Kritik, dass jede Entscheidung nicht ausschließlich auf statischen, gegenwärtigen Fakten basierten und Vermutungen über die Zukunft blieben nicht völlig ausgeklammert. Aber doch war Lucas' Kritik ein entscheidender Anstoß für ein Umdenken innerhalb der Ökonomie von statischen Überlegungen zur Implementierung dynamischer Erwartungen. Dementsprechend wurden auch ökonometrische Modelle völlig neu gedacht. Lange Zeit dominierten zuvor in der Makroökonomie statische keynesiansche Totalmodelle, beziehungsweise die neoklassischen, mikroökonomischen Walrasianischen Gleichgewichtsmodelle. 
	
	Die Neue Klassische Makroökonomie brachte auch hier eine Revolution: Die Mikrofundierung der Makroökonomie. Heute sind praktisch alle Makroökonomischen Modelle mikroökonomisch fundiert. Konkret bedeutet dies, dass man die Konzepte aus der Mikroökonomie, also die Nutzenmaximierung aus der Haushaltstheorie und die Gewinnmaximierung aus der Unternehmenstheorie heranzieht. Man kann aber nicht alle Individuen auf allen Märkten beobachten und deren Verhalten zu einem "`Gesamtverhalten"' aggregieren, also aufsummieren. Stattdessen behilft man sich eines \textit{repräsentativen Agenten}, also ein Mensch, der typisches Verhalten zeigen würde.	Mit den in der Mikroökonomie üblichen Techniken wird dann \textit{optimales} Verhalten des Agenten modelliert. Das heißt, man nimmt an, dass alle Modellgleichungen auf \textit{konsistenten Annahmen} beruhen. In diesem Umfeld optimiert der Agent sein Verhalten, wobei er \textit{rationale Erwartungen} über zukünftige Entwicklungen hat. Der Term rational ist hier im Sinne von "`stochastisch berechenbar"' zu verstehen. Die Modelle sind dynamisch, der repräsentative Agent passt also sein Optimierungsverhalten schlagartig auf Veränderungen in seinen Erwartungen oder den konsistenten Annahmen an. Damit ist das Verhalten des repräsentativen Agenten konsistent mit den Vorhersagen des Modells. 
	
	Der Leser mag sich aufgrund der komplizierten Formulierungen und Vielzahl an Annahmen denken, dass die Ökonomie mit diesen Modellen den Bezug zur Realität vollkommen verloren hat. Höheren Mathematik war endgültig in ökonomischen Modellen angekommen. Dies entsprach und entspricht dem Zeitgeist. Die Ökonomie wurde in der Folge zunehmend als Naturwissenschaft oder gar Formalwissenschaft betrieben, immer weniger als Sozialwissenschaft. Ein Umstand der seither mit wechselnder Vehemenz kritisiert wird, ob zu Recht oder zu Unrecht muss jeder Leser für sich entscheiden. Im Mainstream haben sich diese Modelle auf jeden Fall fest etabliert. Neben der Neuen Klassischen Makroökonmie, setzten auch die Neu-Keynesianer auf diese Art von ökonomischen Modellen. Die heute so häufig herangezogenen neukeynesianischen "`Dynamischen stochastischen allgemeinen Gleichgewichtsmodelle"' unterscheiden sich zwar deutlich von den dynamischen Gleichgewichtsmodellen der "`Neuen Klassischen Makroökonomie"', basieren aber im wesentlichen auf deren Ideen.  
	Man könnte sogar so weit gehen, dass diese Art der ökonomischen Modelle den Mainstream von den heterodoxen Schulen trennt. Sowohl die österreichische Schule als auch die Post-Keynesianer und natürlich die Verhaltensökonomen, lehnen diesen stark formalisierten Zugang jedenfalls strikt ab.


\end{enumerate}	


Die Neuen Klassiker erlebten rasch einen enormen Aufschwung und enorme Beachtung. Rasch war klar, dass ihre formell tatsächlich sehr schon dargestellten und abgehandelten Modellen, den keynesianischen und monetaristischen Modellen formal überlegen waren.
Vor allem die Mikrofundierung der Makroökonomischen Modelle stellte einen deutlichen und nachhaltigen Fortschritt dar. Schließlich akzeptierten auch die Keynesianer die modelltheoretische Überlegenheit\footnote{Die Implementierung keynesianischer Ideen in die Modellannahmen der Neu Klassiker machte die Keynesianer schließlich zu Neu-Keynesianern}
Damit konnte sich die Annahme der Rationalen Erwartungen rasch etablieren. Schließlich war dieses Konzept nicht neu, aber eben technisch schwer umsetzbar, da bei rationalen Erwartungen eine Wechselwirkung zwischen erwarteten zukünftigen Entwicklungen und heutigem Verhalten besteht. Genau das Problem wurde mit den Modellen der "`Neuen Klassiker"' gelöst. Und so sah es Anfang der 1980er Jahre danach aus, als würde sich die "`Neue Klassische Makroökonomie"' als neue, alleinige Mainstream-Ökonomie etablieren. Aber auch das erwies sich rasch als Trugschluss. Die Annahmen der Modelle waren einfach zu stark und zu starr als dass damit die Realität beschrieben werden konnte. 
		
Im Gegensatz zu den Neukeynesianern, die sich zu dieser Zeit ebenfalls formierten und deren Hauptthemen die verschiedenen Formen von Marktversagen waren, meinten die neuen Klassiker, dass alle Märkte selbstständig ein Gleichgewicht finden. Und das auch in der kurzen Frist! Rückbesinnung auf die Klassik eben. Dieser Punkt erwies sich rasch als nicht haltbar. Dies impliziert, dass es keine unfreiwillige Arbeitslosigkeit gäbe, was wohl unrealistisch ist. Genau das war auch der zentrale Angriffspunkt auf die Neue Klassik. "`Wenn der Arbeitsmarkt immer ins Gleichgewicht findet, dann heißt das, dass die Neuen Klassiker davon ausgehen, dass sich mitten in der "`Great Depression"' Millionen Amerikaner dafür entschieden haben jetzt mehr Freizeit zu konsumieren"' (cf \cite{Stiglitz1987}, p. 119), lautete ein hämischer Kommentar der Neu-Keynesianer. 
Warum aber war die Neue Klassik zu deren Beginn so erfolgreich? Meines Erachtens liegt der Grund hierfür in ihrer methodischen Überlegenheit gegenüber Keynesianern und Monetaristen. Wenn alle Modellannahmen eingehalten sind, dann führen die Modelle unwiderlegbar und sehr elegant zu eindeutigen Ergebnissen. Allerdings stellte man rasch fest, dass viele der Modellannahmen empirisch nicht zu halten sind. Robert Solow in einem Interview, dass in \textcite[S. 146]{Klamer1984} veröffentlicht wurde, brachte dies wohl am besten auf den Punkt:

\textit{"'Angenommen, jemand [...] sagt zu ihnen, er sei Napoleon Bonaparte. Das Letzte, was ich möchte, ist, mich mit ihm auf eine technische Diskussion über die Kavallerietaktik in der Schlacht von Austerlitz einzulassen. Wenn ich das tue, werde ich stillschweigend anerkennen, dass er Napoleon ist. Nun, Bob Lucas und Tom Sargent mögen nichts lieber, als technische Diskussionen vorzunehmen, denn dann haben Sie sich stillschweigend auf ihre Grundannahmen eingelassen. Ihre Aufmerksamkeit wird von der grundlegenden Schwäche der ganzen Geschichte abgelenkt."'}

Gerade als sich Anfang der 1980er-Jahre die Neue Klassik als neue Mainstream Ökonomie durchzusetzen schien, stiegen die Arbeitslosenquoten in den USA auf 10\%. Das war nicht vereinbar mit der angeblich ausschließlich freiwilligen Arbeitslosigkeit. Mittlerweile rücken auch die meisten Vertreter der neuen Klassik davon ab, dass sich alle Märkte auch in der kurzen First im Gleichgewicht befinden (Zitat).
Man darf daraus jetzt aber nicht schließen, die Neue Klassische Makroökonomie sei widerlegt und verschwunden. Ganz im Gegenteil! Sie hat viele wichtige und richtige Erweiterungen der Mainstream-Ökonomie gebracht. Erstens, die Methodik wurde revolutioniert. Mikrofundierte, dynamische Gleichgewichtsmodelle wurden von den Neu-Keynesianern rasch aufgenommen, erweitert und für sich beansprucht. Die Anerkennung für die Überwindung der veralteten Makromodelle der Keynesianer und Monetaristen steht aber den Neuen Klassikern zu. Zweitens, die Theorie der Rationalen Erwartungen war - trotz aller Kritik - ein Meilenstein in der Ökonomie, der bis heute State-of-the-Art ist.

Die Veröffentlichung der Lukas-Kritik gilt als eine Revolution in der Ökonomie. Warum aber blieben deren Vertreter, allen voran Robert Lucas, in der öffentlichen Wahrnehmung eher blass? Ein Aspekt ist sicherlich, dass sowohl Keynes als auch Friedman und Hayek auch außerhalb der wissenschaftlichen Ökonomie auftraten, vor allem als Politikberater. Ein weiterer Aspekt ist aber auch die Art der Kommunikation der Vertreter der "`Neuen Klassischen Makroökonomie"'. Diese war ungewöhnlich scharf: \textit{That [the Keynesian] predictions were wildly incorrect and that the doctrine on which they were based is fundamentally flawed are now simple matters of fact}, schrieben Lucas und Sargent in ihrem Artikel \textit{After Keynesian Macroeconomics} \parencite[S. 1]{Lucas1979}. Die Ideen der Neuen Klassiker wurden schon nach wenigen Jahren in die Modelle des bisherigen Mainstreams integriert, nicht jedoch die Leute, die Stimmung innerhalb der wirtschaftswissenschaftlichen Community war in den 1970er und 1980er Jahren vergiftet, beschreibt \cite{Blanchard2003} in seinem Standardlehrbuch. Ähnlich, wenn auch etwas diplomatischer drückte sich \cite{Samuelson1998} aus. Lucas und Co kümmerten die etablierten Ökonomen wenig, es scheint als hielten sie so wirklich gar nichts von ihnen. Umgekehrt erkannten vor allem die Vertreter der neoklassischen Synthese die inhaltliche Sinnhaftigkeit der Ideen der Neuen Klassiker und deren Schüler implementierten diese Ideen in ihre eigenen Modelle, deckten die vorhandenen Schwachpunkte der "`Neu Klassiker"' auf und wurden zu "`Neu-Keynesianern"'. Die Modelle näherten sich also an - vor allem weil die Mainstream-Ökonomen die Ideen der Neuen Klassiker aufnahmen - die dahinterstehenden Personen allerdings in keinster Weise. 

Um sich als Mainstream durchzusetzen waren die Ideen der Neuen Klassiker zu radikal. Die gänzliche Ablehnung der Synthese aus Neoklassik und Keynesianismus erwies sich als vorschnell. Überhaupt zeigte sich der Hauptvertreter der Neuen Klassischen Makroökonomie als wenig pragmatisch was seine ökonomische Sichtweise angeht. Im Jahre 2003 zum Beispiel veröffentlichte er einen seiner Artikel im \textit{American Economic Review} mit der These, dass \textit{macroeconomics in [the] original sense has succeeded: Its central problem of depression-prevention has been solved, for all practical purposes, and has in fact been solved for many decades} \parencite[S. 1]{Lucas2003}. 
Dass nur vier Jahre später mit der "`Great Recession"' die größte Krise seit den 1930er Jahren ausbrechen sollte, zeigte das Gegenteil. Bereits 1987 meinte er: \textit{The most poisonous [tendencies in economics], is to focus on questions of distribution} \parencite{Lucas1987}. Fragen der Einkommensverteilung sind aber tatsächlich gesellschaftlich wie ökonomisch immer bedeutender geworden. Auf die Frage, ob Ökonomie-Studierende heute noch Keynes lesen sollten, antwortete er 1998 mit einem schlichten "`No"' \parencite{Lucas2013}. Die "`Great Recession"' war somit so etwas wie die "`Widerlegung"' der reinen neuen klassischen Makroökonomie. Lucas sah die Krise, so wie zugegebenermaßen die meisten andern Ökonomen nicht nur nicht kommen, sondern, glaubte auch nicht, dass eine derart schwere Krise kommen könnte. Und während der Krise wendeten die Politiker schließlich gnadenlos keynesianisches "`Deficit Spending"' sowie eine extreme Geldpolitik an. 

Insgesamt darf man aus heutiger Sicht darf man aber nicht vergessen, dass diese ökonomische Schule die Wirtschaftswissenschaften tatsächlich revolutioniert hat. Viele der von ihr erstmals vorgebrachten Elemente wurden rasch vom Großteil der Ökonomen aller Richtungen übernommen und sind heute aus der Mainstream-Ökonomie nicht mehr wegzudenken. Robert Lucas zählt daher meines Erachtens zu den größten Ökonomen des 20. Jahrhunderts. Seine Arbeiten sind nicht so einprägend wie die "`General Theory"' von Keynes. Sein Auftritt ist nicht so überzeugend wie jener von Milton Friedman, der durch seine politischen Tätigkeiten auch weit außerhalb der wirtschaftswissenschaftlichen Community bekannt wurde. Aber Robert Lucas stand den beiden in nichts nach. Seine Ideen revolutionierten die Ökonomie des 20. Jahrhunderts und machten daraus eine andere Wissenschaft. Keynes wird oft als genialer "`Lebemann"' dargestellt. Er starb schon zehn Jahre nach der Veröffentlichung seine bahnbrechenden Werkes und musste es selbst nicht mehr gegen Angriffe verteidigen. Vielleicht ist er auch deshalb so populär. Friedman wird sehr kontrovers gesehen. Von den Liberalen noch im hohen Alter als Ikone gefeiert, durch seine Beratertätigkeit oft jedoch auch verhasst. Vor allem aber war er ein brillanter Redner mit charismatischen Auftritt. All das ist nicht die Stärke von Robert Lucas. Seine oben zitierten Aussagen scheinen eher unglücklich formuliert. In seinen Auftritten erscheint er sympathisch, aber nicht als der große Vortragende. Robert Lucas war dafür ein brillanter Wissenschaftler. Seine messerscharfen formalen Abwandlungen prägten Generationen von Studierenden und waren bei seinen Gegnern gefürchtet. Dafür gebührt ihm bleibende Anerkennung und Wertschätzung in der Ökonomie.


\section{Edward Prescott (Real Business Cycle Theorie)}
\label{RBC}
Rasch als falsch. Technischer Fortschritt im Vordergrund. Nobelpreis 2004.
Formal elegant und fortschrittlich: Zeitreihenmodelle
Zeitgeist: Random Walk (so wie in der neoklassischen Modern Finance durch Eugene Fama)

Aber die Theorie ist nicht tot:
Methodisch: Christopher Sims (Zeitreihenmodelle) und Weiterentwicklung 


\section{Barro: Ricardianische Äquivalenz}
Angelehnt an Sargent's \cite{Sargent1975}
Crowding-Out-Effekte

\section{Chicago School \& Neue Institutionenökonomik}
Coase, Becker, Williamson 

\section{Becker: Rational Choice Theory}




\section{Paul Romer: Endogenes Wachstumsmodell} Neue Wachstumstheorie









Gesamt am Ende



Insofern bin ich auch der Meinung, dass die neue Mainstream-Ökonomie nicht im Widerspruch zur Neuen Klassik steht und die Neue Klassik auch nicht als "`überwunden"' gelten sollte. Vielmehr ist unsere heutige Mainstream-Ökonomie eine "`Neue Synthese"' zwischen den Nachfolgern der Widersacher der 1970er und 1980er Jahre: Den Neu-Keynesianern und den Neuen Klassikern (siehe Kapitel \ref{Neue Neoklassische Synthese})


Was ist geblieben? Die Annahme der "`rationalen Erwartungen"' ist zwar umstritten, aber hat es dennoch in die ökonomischen Mainstream-Modelle geschafft. Unumstritten spielen rationale Erwartungen eine Rolle zum Beispiel in der Geldpolitik. Die Zentralbanken verfolgen mittlerweile nicht mehr nur ein Geldmengenziel und auch nicht nur ein Inflationsziel, sondern sie arbeiten auch konkret mit den Inflations\textit{erwartungen}.
Geblieben ist auch die Mikrofundierung der Makroökonomie. Die höhere Mathematik wurde von den Neuen Klassikern erstmals in die Ökonomie gebracht und hat sich dort etabliert. Heute Publikationen enthalten zum größten Teil hochformalisierte Modelle.

Was hat sich als falsch erwiesen? Der größte Fehler war das Festhalten am perfekten Funktionieren der Märkte auch in der kurzen Frist. Dass es keine unfreiwillige Arbeitslosigkeit gäbe war rasch empirisch nicht zu halten und modelltheoretisch nicht notwendig wie Diamond, Pissaridis und Mortensen zeigten. Außerdem entstanden rasch Neu-Keynesianische Modelle , die zeigten, dass die Theorie der Rationalen Erwartung auch dann aufrecht erhalten werden kann, wenn sich Löhne und Preise nicht sofort an das Gleichgewicht anpassen, sondern langsam über mehrere "`Entscheidungsrunden"'. Damit fiel auch die Annahme, dass Geldpolitik und Fiskalpolitik komplett unwirksam seien. Die aktuelle Forschung geht davon aus, dass zumindest in der kurzen Frist beide wirksam sind.





